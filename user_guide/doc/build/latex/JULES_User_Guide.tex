%% Generated by Sphinx.
\def\sphinxdocclass{report}
\documentclass[a4paper,10pt,openany,oneside,english]{sphinxmanual}
\ifdefined\pdfpxdimen
   \let\sphinxpxdimen\pdfpxdimen\else\newdimen\sphinxpxdimen
\fi \sphinxpxdimen=.75bp\relax
\ifdefined\pdfimageresolution
    \pdfimageresolution= \numexpr \dimexpr1in\relax/\sphinxpxdimen\relax
\fi
%% let collapsible pdf bookmarks panel have high depth per default
\PassOptionsToPackage{bookmarksdepth=5}{hyperref}

\PassOptionsToPackage{warn}{textcomp}
\usepackage[utf8]{inputenc}
\ifdefined\DeclareUnicodeCharacter
% support both utf8 and utf8x syntaxes
  \ifdefined\DeclareUnicodeCharacterAsOptional
    \def\sphinxDUC#1{\DeclareUnicodeCharacter{"#1}}
  \else
    \let\sphinxDUC\DeclareUnicodeCharacter
  \fi
  \sphinxDUC{00A0}{\nobreakspace}
  \sphinxDUC{2500}{\sphinxunichar{2500}}
  \sphinxDUC{2502}{\sphinxunichar{2502}}
  \sphinxDUC{2514}{\sphinxunichar{2514}}
  \sphinxDUC{251C}{\sphinxunichar{251C}}
  \sphinxDUC{2572}{\textbackslash}
\fi
\usepackage{cmap}
\usepackage[LGR,T1]{fontenc}
\usepackage{amsmath,amssymb,amstext}
\usepackage[english]{babel}
\usepackage{substitutefont}


\usepackage{tgtermes}
\usepackage{tgheros}
\renewcommand{\ttdefault}{txtt}


\expandafter\ifx\csname T@LGR\endcsname\relax
\else
% LGR was declared as font encoding
  \substitutefont{LGR}{\rmdefault}{cmr}
  \substitutefont{LGR}{\sfdefault}{cmss}
  \substitutefont{LGR}{\ttdefault}{cmtt}
\fi
\expandafter\ifx\csname T@X2\endcsname\relax
  \expandafter\ifx\csname T@T2A\endcsname\relax
  \else
  % T2A was declared as font encoding
    \substitutefont{T2A}{\rmdefault}{cmr}
    \substitutefont{T2A}{\sfdefault}{cmss}
    \substitutefont{T2A}{\ttdefault}{cmtt}
  \fi
\else
% X2 was declared as font encoding
  \substitutefont{X2}{\rmdefault}{cmr}
  \substitutefont{X2}{\sfdefault}{cmss}
  \substitutefont{X2}{\ttdefault}{cmtt}
\fi

\usepackage{textalpha}
\usepackage[Bjarne]{fncychap}
\usepackage{sphinx}

\fvset{fontsize=auto}
\usepackage{geometry}


% Include hyperref last.
\usepackage{hyperref}
% Fix anchor placement for figures with captions.
\usepackage{hypcap}% it must be loaded after hyperref.
% Set up styles of URL: it should be placed after hyperref.
\urlstyle{same}


\usepackage{sphinxmessages}
\setcounter{tocdepth}{0}



\title{Joint UK Land Environment Simulator (JULES) User Guide}
\date{Sep 12, 2023}
\release{7.3}
\author{Met Office}
\newcommand{\sphinxlogo}{\sphinxincludegraphics{jules_logo_pdf.png}\par}
\renewcommand{\releasename}{Release}
\makeindex
\begin{document}

\ifdefined\shorthandoff
  \ifnum\catcode`\=\string=\active\shorthandoff{=}\fi
  \ifnum\catcode`\"=\active\shorthandoff{"}\fi
\fi

\pagestyle{empty}
\sphinxmaketitle
\pagestyle{plain}
\sphinxtableofcontents
\pagestyle{normal}
\phantomsection\label{\detokenize{index::doc}}


\sphinxAtStartPar
The Joint UK Land Environment Simulator (JULES) is a computer model that simulates many soil and vegetation processes. This guide primarily describes the format of the input and output files, and does not include detailed descriptions of the science and representation of the processes in the model.

\sphinxAtStartPar
The first version of JULES was based on the Met Office Surface Exchange System (MOSES), the land surface model used in the \sphinxhref{http://www.metoffice.gov.uk/research/modelling-systems/unified-model}{Unified Model} (UM) of the \sphinxhref{http://www.metoffice.gov.uk}{UK Met Office}. After that initial split, the MOSES and JULES code bases evolved separately, but with JULES v2.1 these differences were reconciled with the UM. As of JULES v3.1, a single code repository is used for both standalone JULES and JULES in the UM.

\sphinxAtStartPar
Further information can be found on the JULES website: \sphinxurl{http://jules.jchmr.org/}.

\sphinxstepscope


\chapter{Release notes}
\label{\detokenize{release_notes/contents:release-notes}}\label{\detokenize{release_notes/contents::doc}}
\sphinxstepscope


\section{JULES version 7.3 Release Notes}
\label{\detokenize{release_notes/JULES7-3:jules-version-7-3-release-notes}}\label{\detokenize{release_notes/JULES7-3::doc}}
\sphinxAtStartPar
The JULES vn7.3 release consists of approximately 19 tickets from 9 authors, including work by many other people.

\sphinxAtStartPar
Full details of the tickets committed for JULES vn7.3 can be found on the \sphinxhref{https://code.metoffice.gov.uk/trac/jules/query?resolution=fixed\&milestone=JULES+v7.3+(Jun-23)}{JULES shared repository Trac system}.

\sphinxAtStartPar
Ticket numbers are indicated below, e.g. \#1180.


\subsection{General/Technical changes}
\label{\detokenize{release_notes/JULES7-3:general-technical-changes}}\begin{itemize}
\item {} 
\sphinxAtStartPar
Removed default numerical values from variables in the {\hyperref[\detokenize{namelists/jules_hydrology.nml:namelist-JULES_HYDROLOGY}]{\sphinxcrossref{\sphinxcode{\sphinxupquote{JULES\_HYDROLOGY}}}}} and {\hyperref[\detokenize{namelists/jules_rivers.nml:namelist-JULES_RIVERS}]{\sphinxcrossref{\sphinxcode{\sphinxupquote{JULES\_RIVERS}}}}} namelists. (\#1180)

\item {} 
\sphinxAtStartPar
Allow a model domain that straddles the edge of the input grid (for grids that are cyclic in longitude). (\#1301)

\item {} 
\sphinxAtStartPar
Simplification of the namelists required for OASIS\sphinxhyphen{}Rivers to aid maintenance of the LFRic coupled miniapp. (\#1385)

\item {} 
\sphinxAtStartPar
Water resource variables bundled in a new TYPE. (\#1401)

\item {} 
\sphinxAtStartPar
Clarification of timestep\sphinxhyphen{}related variables. (\#1403)

\item {} 
\sphinxAtStartPar
Change kind type names to avoid clash with Fortran intrinsics. (\#1408)

\item {} 
\sphinxAtStartPar
Removed some include files by moving code to module files. (\#1411, 1412. 1418)

\end{itemize}


\subsection{Bugs fixed}
\label{\detokenize{release_notes/JULES7-3:bugs-fixed}}\begin{itemize}
\item {} 
\sphinxAtStartPar
Fixed standalone diagnostics \sphinxtitleref{sat\_excess\_roff} and \sphinxtitleref{drain}. (\#1039)

\item {} 
\sphinxAtStartPar
Fix for possible floating point exceptions in veg2 code. (\#1155)

\item {} 
\sphinxAtStartPar
Bug fixes for irrigation code. (\#1386)

\item {} 
\sphinxAtStartPar
Bug fixes for OASIS\sphinxhyphen{}Rivers coupling field, \sphinxtitleref{rflow\_outflow}. (\#1435)

\end{itemize}


\subsection{Changes to testing}
\label{\detokenize{release_notes/JULES7-3:changes-to-testing}}\begin{itemize}
\item {} 
\sphinxAtStartPar
Updated \sphinxtitleref{suite\_report.py}. (\#1442)

\end{itemize}


\subsection{Documentation updates}
\label{\detokenize{release_notes/JULES7-3:documentation-updates}}\begin{itemize}
\item {} 
\sphinxAtStartPar
Simplified the description of platform files. (\#1353)

\item {} 
\sphinxAtStartPar
Minor syntax changes in the JULES Coding Standards. (\#1402)

\item {} 
\sphinxAtStartPar
Minor corrections to code and documentation. (\#1421, 1443)

\item {} 
\sphinxAtStartPar
Updates associated with many of the above changes, and release notes. (\#1384)

\end{itemize}

\sphinxAtStartPar
Documentation can be viewed on the github page \sphinxurl{http://jules-lsm.github.io/}.

\sphinxstepscope


\section{JULES version 7.2 Release Notes}
\label{\detokenize{release_notes/JULES7-2:jules-version-7-2-release-notes}}\label{\detokenize{release_notes/JULES7-2::doc}}
\sphinxAtStartPar
The JULES vn7.2 release consists of approximately 19 tickets from 17 authors, including work by many other people.

\sphinxAtStartPar
Full details of the tickets committed for JULES vn7.2 can be found on the \sphinxhref{https://code.metoffice.gov.uk/trac/jules/query?resolution=fixed\&milestone=JULES+v7.2+(Feb-23)}{JULES shared repository Trac system}.

\sphinxAtStartPar
Ticket numbers are indicated below, e.g. \#1317.


\subsection{General/Technical changes}
\label{\detokenize{release_notes/JULES7-2:general-technical-changes}}\begin{itemize}
\item {} 
\sphinxAtStartPar
Switch {\hyperref[\detokenize{namelists/model_grid.nml:JULES_LATLON::l_coord_latlon}]{\sphinxcrossref{\sphinxcode{\sphinxupquote{l\_coord\_latlon}}}}} specifies if the coordinate system is latitude\sphinxhyphen{}longitude or something else (e.g. a rotated grid). This makes it easier to model and postprocess grids that are not defined by latitude and longitude. (\#1317)

\item {} 
\sphinxAtStartPar
Addition of OASIS coupling capabilities to the river routing executable. (\#1191)

\item {} 
\sphinxAtStartPar
The RothC soil C model is now referred to as the 4\sphinxhyphen{}pool model. (\#1348)

\item {} 
\sphinxAtStartPar
Removed the remaining 2D field being passed into JULES via a module, replacing it with an argument list variable. (\#1376)

\item {} 
\sphinxAtStartPar
Removed a redundant print statement from the urban code. (\#1367)

\item {} 
\sphinxAtStartPar
Updated CABLE (as part of consolidating CABLE across its main applications, ACCESS\sphinxhyphen{}CM2, ESM1.5 and CABLE standalone), also corrected an inconsistency in the order of arguments for a few subroutines. (\#1373)

\item {} 
\sphinxAtStartPar
Increase compile\sphinxhyphen{}time checking with Cray’s CCE. (\#1061)

\item {} 
\sphinxAtStartPar
Updated configuration for NIWA. (\#1387)

\end{itemize}


\subsection{Bugs fixed}
\label{\detokenize{release_notes/JULES7-2:bugs-fixed}}\begin{itemize}
\item {} 
\sphinxAtStartPar
Improved numerical implementation of an EXP calculation in the spectral albedo scheme. (\#1189)

\item {} 
\sphinxAtStartPar
Full initialisation of soil carbon arrays used with the INFERNO fire model. (\#1356)

\item {} 
\sphinxAtStartPar
Fixed the upgrde macro chain. (\#1368)

\item {} 
\sphinxAtStartPar
Fixes to umdp3\_checker. (\#1397)

\end{itemize}


\subsection{Changes to testing}
\label{\detokenize{release_notes/JULES7-2:changes-to-testing}}\begin{itemize}
\item {} 
\sphinxAtStartPar
Rose stem testing extended to include overbank inundation. (\#999)

\item {} 
\sphinxAtStartPar
loobos\_gl4\_cable now generates output files and is included in rose stem testing at several sites. (\#1375)

\item {} 
\sphinxAtStartPar
Updated suite\_report.py and changed rosestem\_branch\_checker.py to use generic variable names to match the UKCA version. (\#1369, 1379)

\item {} 
\sphinxAtStartPar
Refactored rose stem tests for building JULES with FAB. (\#1354)

\end{itemize}


\subsection{Documentation updates}
\label{\detokenize{release_notes/JULES7-2:documentation-updates}}\begin{itemize}
\item {} 
\sphinxAtStartPar
Updates associated with many of the above changes, and release notes. (\#1381)

\item {} 
\sphinxAtStartPar
Updated JULES docs to ensure that they build with a more recent version of Sphinx \sphinxhyphen{} they can now be built using Sphinx 2.4.0 and Python 3.6.10. (\#1382)

\end{itemize}

\sphinxAtStartPar
Documentation can be viewed on the github page \sphinxurl{http://jules-lsm.github.io/}.

\sphinxstepscope


\section{JULES version 7.1 Release Notes}
\label{\detokenize{release_notes/JULES7-1:jules-version-7-1-release-notes}}\label{\detokenize{release_notes/JULES7-1::doc}}
\sphinxAtStartPar
The JULES vn7.1 release consists of approximately 20 tickets from 18 authors, including work by many other people.

\sphinxAtStartPar
Full details of the tickets committed for JULES vn7.1 can be found on the \sphinxhref{https://code.metoffice.gov.uk/trac/jules/query?resolution=fixed\&milestone=JULES+Oct-2022}{JULES shared repository Trac system}.

\sphinxAtStartPar
Ticket numbers are indicated below, e.g. \#931.


\subsection{Science changes}
\label{\detokenize{release_notes/JULES7-1:science-changes}}\begin{itemize}
\item {} 
\sphinxAtStartPar
Interactive gas\sphinxhyphen{}phase deposition routines from UKCA have been added to JULES, together with a variant version that removes the restriction on the surface tile configuration. See {\hyperref[\detokenize{namelists/jules_deposition.nml:JULES_DEPOSITION::dry_dep_model}]{\sphinxcrossref{\sphinxcode{\sphinxupquote{dry\_dep\_model}}}}}. (\#931)

\end{itemize}


\subsection{General/Technical changes}
\label{\detokenize{release_notes/JULES7-1:general-technical-changes}}\begin{itemize}
\item {} 
\sphinxAtStartPar
Added a switch to allow interpretation of times in the model and the driving data as local solar time \sphinxhyphen{} see {\hyperref[\detokenize{namelists/timesteps.nml:JULES_TIME::l_local_solar_time}]{\sphinxcrossref{\sphinxcode{\sphinxupquote{l\_local\_solar\_time}}}}}. (\#1327)

\item {} 
\sphinxAtStartPar
Added a switch {\hyperref[\detokenize{namelists/imogen.nml:IMOGEN_RUN_LIST::l_drive_with_global_temps}]{\sphinxcrossref{\sphinxcode{\sphinxupquote{l\_drive\_with\_global\_temps}}}}} so that JULES can be driven with global temperatures and climate patterns. (\#1322)

\item {} 
\sphinxAtStartPar
Made Medlyn stomatal conductance, Farquhar C$_{\text{3}}$ photosynthesis, and thermal acclimation available in the UM. (\#1246)

\item {} 
\sphinxAtStartPar
Further work towards allowing layered soil carbon ({\hyperref[\detokenize{namelists/jules_soil_biogeochem.nml:JULES_SOIL_BIOGEOCHEM::l_layeredc}]{\sphinxcrossref{\sphinxcode{\sphinxupquote{l\_layeredc}}}}} = TRUE ) in the UM. (\#1237)

\item {} 
\sphinxAtStartPar
A varying grey tile emissivity has been passed in to JULES \sphinxhyphen{} currently only available if selected in LFRic. (\#1247)

\item {} 
\sphinxAtStartPar
OMP improvements in various routines. (\#1310)

\item {} 
\sphinxAtStartPar
Enabled a call to CABLE albedo (part of ongoing work to provide access to CABLE). (\#1314)

\item {} 
\sphinxAtStartPar
MORUSES and anthropogenic heat related metadata migrated to jules\sphinxhyphen{}shared to facilitate their implementation in LFRic. The simple two\sphinxhyphen{}tile urban scheme now no longer requires W/R as an input unless the total urban fraction only is specified in the ancillary data. (\#1255)

\item {} 
\sphinxAtStartPar
Included the fab\_jules app which builds JULES using Fab software. (\#1331, 1364)

\item {} 
\sphinxAtStartPar
Updated compiler in JASMIN gfortran platform file. (\#1347)

\item {} 
\sphinxAtStartPar
Upgraded rose stem testing on Met Office EXZ to use latest availiable CPE, and fixes for EXZ. (\#1362, 1358)

\end{itemize}


\subsection{Bugs fixed}
\label{\detokenize{release_notes/JULES7-1:bugs-fixed}}\begin{itemize}
\item {} 
\sphinxAtStartPar
Corrected a bug in the \sphinxtitleref{surf\_ht\_flux} diagnostic when using the Flake lake model. (\#1340)

\item {} 
\sphinxAtStartPar
Fixing problems in the WARNING output messages. (\#1303)

\item {} 
\sphinxAtStartPar
Fixed a metadata/rose config error introduced by a bug in Rose. (\#1363)

\end{itemize}


\subsection{Changes to testing}
\label{\detokenize{release_notes/JULES7-1:changes-to-testing}}\begin{itemize}
\item {} 
\sphinxAtStartPar
Updated NCI rose stem tests. (\#1266)

\end{itemize}


\subsection{Documentation updates}
\label{\detokenize{release_notes/JULES7-1:documentation-updates}}\begin{itemize}
\item {} 
\sphinxAtStartPar
Updates associated with many of the above changes and to module leaders file, and release notes. (\#1275, 1343)

\end{itemize}

\sphinxAtStartPar
Documentation can be viewed on the github page \sphinxurl{http://jules-lsm.github.io/}.

\sphinxstepscope


\section{JULES version 7.0 Release Notes}
\label{\detokenize{release_notes/JULES7-0:jules-version-7-0-release-notes}}\label{\detokenize{release_notes/JULES7-0::doc}}
\sphinxAtStartPar
The JULES vn7.0 release consists of approximately 31 tickets from 20 authors, including work by many other people.

\sphinxAtStartPar
Full details of the tickets committed for JULES vn7.0 can be found on the \sphinxhref{https://code.metoffice.gov.uk/trac/jules/query?resolution=fixed\&milestone=JULES+Jun-2022}{JULES shared repository Trac system}.

\sphinxAtStartPar
Ticket numbers are indicated below, e.g. \#911.


\subsection{Science changes}
\label{\detokenize{release_notes/JULES7-0:science-changes}}\begin{itemize}
\item {} 
\sphinxAtStartPar
New functionality for modelling bioenergy crops \sphinxhyphen{} see {\hyperref[\detokenize{namelists/jules_vegetation.nml:JULES_VEGETATION::l_trif_biocrop}]{\sphinxcrossref{\sphinxcode{\sphinxupquote{l\_trif\_biocrop}}}}} and {\hyperref[\detokenize{namelists/jules_vegetation.nml:JULES_VEGETATION::l_ag_expand}]{\sphinxcrossref{\sphinxcode{\sphinxupquote{l\_ag\_expand}}}}}. (\#911)

\item {} 
\sphinxAtStartPar
Implemented a socio\sphinxhyphen{}economic factor in the fire ignition and suppression parameterisation in INFERNO, based on a Human Development Index (HDI). (\#1284)

\item {} 
\sphinxAtStartPar
A new logical switch {\hyperref[\detokenize{namelists/jules_surface.nml:JULES_SURFACE::l_mo_buoyancy_calc}]{\sphinxcrossref{\sphinxcode{\sphinxupquote{l\_mo\_buoyancy\_calc}}}}} enables an interactive buoyancy in the calculation of the surface transfer coefficients. (\#1242)

\end{itemize}


\subsection{General/Technical changes}
\label{\detokenize{release_notes/JULES7-0:general-technical-changes}}\begin{itemize}
\item {} 
\sphinxAtStartPar
Surface type IDs fully extended to JULES to allow extra surface configuration checks to take place. A routine to check compatible science options is now accessible to all parent models, allowing cross\sphinxhyphen{}namelist checking to take place once all the namelists have been read, removing the dependency on order. (\#1249)

\item {} 
\sphinxAtStartPar
Upgraded FLake driver to version 1.10 to include a bug fix from the FLake community and keep our copy of FLake aligned with the community code base. (\#1227)

\item {} 
\sphinxAtStartPar
JULES now passes sea ice surface heat flux (surf\_ht\_flux\_sice), sea ice top melt (sice\_melt) and sea ice sublimation (ei\_sice) from JULES to LFRic as part of the fluxes structure. These variables, and a few others, are no longer weighted by sea ice fraction before being passed out of JULES. The weighting by sea ice fraction should be done in the parent models. (\#1259)

\item {} 
\sphinxAtStartPar
Made stencil used in buddy\_sea option work for unstructured meshes, rather than assuming i+1, j+1 indexing will work. (\#1286)

\item {} 
\sphinxAtStartPar
Tidied soil code so that arguments follow UMDP3 order, and corrected some argument INTENTs. (\#843)

\item {} 
\sphinxAtStartPar
Streamlined standalone code dealing with input of ancillary fields. (\#1256)

\item {} 
\sphinxAtStartPar
In preparation for including layered soil carbon in the UM, the soil respiration returned to the UM now has an extra dimension which can potentially be used to represent layers. (\#1236)

\item {} 
\sphinxAtStartPar
Migration of {\hyperref[\detokenize{namelists/nveg_params.nml:namelist-JULES_NVEGPARM}]{\sphinxcrossref{\sphinxcode{\sphinxupquote{JULES\_NVEGPARM}}}}} and {\hyperref[\detokenize{namelists/jules_surface_types.nml:namelist-JULES_SURFACE_TYPES}]{\sphinxcrossref{\sphinxcode{\sphinxupquote{JULES\_SURFACE\_TYPES}}}}} used in LFRic to shared metadata held in jules\sphinxhyphen{}shared. Checking of {\hyperref[\detokenize{namelists/nveg_params.nml:namelist-JULES_NVEGPARM}]{\sphinxcrossref{\sphinxcode{\sphinxupquote{JULES\_NVEGPARM}}}}} moved to a new shared routine and added to UM. (\#1272)

\item {} 
\sphinxAtStartPar
Passing CABLE vars (TYPEs) from top level through to and into surf\_couple layer. (\#1226)

\item {} 
\sphinxAtStartPar
Updated the module load of the make\_jules\_release script. (\#1263)

\end{itemize}


\subsection{Bugs fixed}
\label{\detokenize{release_notes/JULES7-0:bugs-fixed}}\begin{itemize}
\item {} 
\sphinxAtStartPar
Fix for precision issue whereby infiltration of rainfall into snowpack could become larger than the incident rainfall, resulting in negative large\sphinxhyphen{}scale rain. (\#1092)

\item {} 
\sphinxAtStartPar
Bug fix for persistent small snow amounts \sphinxhyphen{} see {\hyperref[\detokenize{namelists/science_fixes.nml:JULES_TEMP_FIXES::l_fix_snow_frac}]{\sphinxcrossref{\sphinxcode{\sphinxupquote{l\_fix\_snow\_frac}}}}}. (\#1279)

\item {} 
\sphinxAtStartPar
Correction to the chlorophyll dependence of the oceanic albedo in the scheme of Jin et al. (2011). (\#1260)

\item {} 
\sphinxAtStartPar
Bug fix to allow calculation of the lw\_net diagnostic. (\#1270)

\item {} 
\sphinxAtStartPar
Fixes to UM routines identified by the NAG compiler. (\#1276)

\item {} 
\sphinxAtStartPar
Corrected namelist reading to ensure that FLake can be run with urban2t and MORUSES schemes in standalone JULES. (\#1277)

\item {} 
\sphinxAtStartPar
Fixed bug that prevented finalisation of initial output files. (\#1281)

\item {} 
\sphinxAtStartPar
Introduced namelist variable {\hyperref[\detokenize{namelists/ancillaries.nml:JULES_RIVERS_PROPS::coordinate_file}]{\sphinxcrossref{\sphinxcode{\sphinxupquote{coordinate\_file}}}}} to fix bug preventing use of file\sphinxhyphen{}name templating with river ancillaries. (\#1287)

\item {} 
\sphinxAtStartPar
Prevent faults caused by attempting to read an absent dummy argument. (\#1292)

\item {} 
\sphinxAtStartPar
Fixed benign OMP bug in snowpack\_mod. (\#1304)

\end{itemize}


\subsection{Changes to testing}
\label{\detokenize{release_notes/JULES7-0:changes-to-testing}}\begin{itemize}
\item {} 
\sphinxAtStartPar
Added rose stem test for the FLake lake model. (\#1277)

\item {} 
\sphinxAtStartPar
Altered the eraint rose stem apps to better represent river routing. (\#1299)

\item {} 
\sphinxAtStartPar
Increased resources requested for build in Met Office XC40 rose stem. (\#1291)

\item {} 
\sphinxAtStartPar
Changes to module load, mpiexec usage, and memory requested for rose stem for the Met Office EXZ platform. (\#1296, 1302, 1305, 1309)

\end{itemize}


\subsection{Documentation updates}
\label{\detokenize{release_notes/JULES7-0:documentation-updates}}\begin{itemize}
\item {} 
\sphinxAtStartPar
Updates associated with many of the above changes and to module leaders file, and release notes. (\#1275, 1300)

\end{itemize}

\sphinxAtStartPar
Documentation can be viewed on the github page \sphinxurl{http://jules-lsm.github.io/}.

\sphinxstepscope


\section{JULES version 6.3 Release Notes}
\label{\detokenize{release_notes/JULES6-3:jules-version-6-3-release-notes}}\label{\detokenize{release_notes/JULES6-3::doc}}
\sphinxAtStartPar
The JULES vn6.3 release consists of approximately 28 tickets from 14 authors, including work by many other people.

\sphinxAtStartPar
Full details of the tickets committed for JULES vn6.3 can be found on the \sphinxhref{https://code.metoffice.gov.uk/trac/jules/query?resolution=fixed\&milestone=JULES+v6.3+(Feb-22)}{JULES shared repository Trac system}.

\sphinxAtStartPar
Ticket numbers are indicated below, e.g. \#1142.


\subsection{Science changes}
\label{\detokenize{release_notes/JULES6-3:science-changes}}\begin{itemize}
\item {} 
\sphinxAtStartPar
Restructured the parameterisation of thermal acclimation of photosynthesis to permit a wider range of configurations \sphinxhyphen{} see {\hyperref[\detokenize{namelists/jules_vegetation.nml:JULES_VEGETATION::photo_acclim_model}]{\sphinxcrossref{\sphinxcode{\sphinxupquote{photo\_acclim\_model}}}}}. (\#1142)

\item {} 
\sphinxAtStartPar
Updated soil N limitation in layered CN model (only applies when {\hyperref[\detokenize{namelists/jules_soil_biogeochem.nml:JULES_SOIL_BIOGEOCHEM::l_layeredc}]{\sphinxcrossref{\sphinxcode{\sphinxupquote{l\_layeredc}}}}} = TRUE). (\#1213)

\end{itemize}


\subsection{General/Technical changes}
\label{\detokenize{release_notes/JULES6-3:general-technical-changes}}\begin{itemize}
\item {} 
\sphinxAtStartPar
Improved nitrogen leaching for vertically\sphinxhyphen{}resolved soil biogeochemistry. (\#1219)

\item {} 
\sphinxAtStartPar
Allowed the use of daily files for input and output (via time templating for input files). (\#1215)

\item {} 
\sphinxAtStartPar
Further developments towards the RED vegetation model. (\#1182)

\item {} 
\sphinxAtStartPar
Refactored IMOGEN in preparation for adding netcdf reading and variable resolution. (\#1214)

\item {} 
\sphinxAtStartPar
Added functionality for the first phase of a standalone Rivers executable. This uses JULES I/O and is still under development; it currently is not scientifically correct as a result of a timestamp issue and does not include dumping capability. (\#1084)

\item {} 
\sphinxAtStartPar
Namelists \sphinxtitleref{jules\_urban2t\_param} and \sphinxtitleref{jules\_urban\_switches} amalgamated into {\hyperref[\detokenize{namelists/urban.nml:namelist-JULES_URBAN}]{\sphinxcrossref{\sphinxcode{\sphinxupquote{JULES\_URBAN}}}}}, which is also now consistent with the module name. (\#1077)

\item {} 
\sphinxAtStartPar
Updated metadata for bedrock and layered soil carbon. (\#1234)

\item {} 
\sphinxAtStartPar
Framework for a shared metadata solution introduced initially with namelist items from \sphinxtitleref{jules\_surface} and \sphinxtitleref{jules\_vegetation} currently shared with LFRic. (\#1195)
More information on sharing JULES metadata can be found here: \sphinxhref{https://code.metoffice.gov.uk/trac/jules/wiki/SharingJULESmetadata}{SharingJULESmetadata}

\item {} 
\sphinxAtStartPar
Minor adjustment to how FLake variables are dealt with during initialisation. (\#1169)

\item {} 
\sphinxAtStartPar
Modularised and refactored subroutine vgrav. (\#1199)

\item {} 
\sphinxAtStartPar
Rivers variables bundled in a new TYPE. (\#1176)

\item {} 
\sphinxAtStartPar
Modularised further UM\sphinxhyphen{}only files. (\#1221)

\item {} 
\sphinxAtStartPar
Removed the soil carbon variable cs from the UM, in prepration for the introduction of layered soil carbon. (\#1210)

\item {} 
\sphinxAtStartPar
Removed snowmelt\_ij from fluxes\_mod. (\#1167)

\item {} 
\sphinxAtStartPar
Tidied code to remove compiler warnings. (\#1245)

\item {} 
\sphinxAtStartPar
Revised the implementation of defining the veg/soil parameters to be passed to CABLE as a single TYPE, following them being read in through namelists. We also implement a working variables TYPE to hold variables at the top\sphinxhyphen{}level so that they may be passed between pathways (i.e explicit, implicit); as well as between time steps (at least until they are elevated to prognostics level). (\#1223)

\item {} 
\sphinxAtStartPar
Added a further restriction on the type of processor used for rose stem testing on JASMIN. (\#1222)

\item {} 
\sphinxAtStartPar
Updated list of module leaders. (\#1240)

\end{itemize}


\subsection{Bugs fixed}
\label{\detokenize{release_notes/JULES6-3:bugs-fixed}}\begin{itemize}
\item {} 
\sphinxAtStartPar
Fixes to FLake surface exchange and decoupling FLake from the soil column. (\#1094)

\item {} 
\sphinxAtStartPar
Added missing metadata for PFT parameter \sphinxtitleref{dust\_veg} (only used in the UM). (\#1206)

\item {} 
\sphinxAtStartPar
Corrected the definitions of the surface longwave radiation diagnostics. (\#1220)

\item {} 
\sphinxAtStartPar
Fixed a bug that prevented files from being closed when running with MPI. (\#1241)

\item {} 
\sphinxAtStartPar
Ensured that soileccose types are set up correctly. (\#1229)

\item {} 
\sphinxAtStartPar
Ensured Python3 is in path on both MONSooN and JASMIN systems. (\#1216, 1231)

\end{itemize}


\subsection{Documentation updates}
\label{\detokenize{release_notes/JULES6-3:documentation-updates}}\begin{itemize}
\item {} 
\sphinxAtStartPar
Updates associated with many of the above changes, and release notes. (\#1243)

\end{itemize}

\sphinxAtStartPar
Documentation can be viewed on the github page \sphinxurl{http://jules-lsm.github.io/}.

\sphinxstepscope


\section{JULES version 6.2 Release Notes}
\label{\detokenize{release_notes/JULES6-2:jules-version-6-2-release-notes}}\label{\detokenize{release_notes/JULES6-2::doc}}
\sphinxAtStartPar
The JULES vn6.2 release consists of approximately 23 tickets from 15 authors, including work by many other people.

\sphinxAtStartPar
Full details of the tickets committed for JULES vn6.2 can be found on the \sphinxhref{https://code.metoffice.gov.uk/trac/jules/query?resolution=fixed\&milestone=JULES+v6.2+(Oct-21)}{JULES shared repository Trac system}.

\sphinxAtStartPar
Ticket numbers are indicated below, e.g. \#470.


\subsection{Science changes}
\label{\detokenize{release_notes/JULES6-2:science-changes}}\begin{itemize}
\item {} 
\sphinxAtStartPar
Added the ability to label and trace a subset of the soil carbon in the RothC layered model \sphinxhyphen{} see {\hyperref[\detokenize{namelists/jules_soil_biogeochem.nml:JULES_SOIL_BIOGEOCHEM::l_label_frac_cs}]{\sphinxcrossref{\sphinxcode{\sphinxupquote{l\_label\_frac\_cs}}}}}. (\#470)

\item {} 
\sphinxAtStartPar
Replaced original logical for corrections to coastal tiling with a 3 way switch ({\hyperref[\detokenize{namelists/science_fixes.nml:JULES_TEMP_FIXES::ctile_orog_fix}]{\sphinxcrossref{\sphinxcode{\sphinxupquote{ctile\_orog\_fix}}}}}). The latest improved fix combines the previous fix over sea with the original behaviour over land. (This is mainly relevant for runs with the Unified Model.) (\#1184)

\end{itemize}


\subsection{General/Technical changes}
\label{\detokenize{release_notes/JULES6-2:general-technical-changes}}\begin{itemize}
\item {} 
\sphinxAtStartPar
Improvements and bug fixes for river grid and flow directions. (\#1170)

\item {} 
\sphinxAtStartPar
Selected output variables are now available on the river grid. (\#1163)

\item {} 
\sphinxAtStartPar
Technical work coupling the RED vegetation model to JULES. This is early in a staged process and RED is not yet available for general use. (\#1034)

\item {} 
\sphinxAtStartPar
Only check for duplicate lat\sphinxhyphen{}lon coordinates if those are set using constant values. (\#1164)

\item {} 
\sphinxAtStartPar
Variables related to fluxes and atmospheric chemistry bundled into TYPEs. (\#1104, 1159)

\item {} 
\sphinxAtStartPar
Removed momentum calculations for LFRic as those are now done in bespoke LFRic code. (\#1144)

\item {} 
\sphinxAtStartPar
Further work towards enabling interoperability between the JULES and CABLE land surface models \sphinxhyphen{} this step dealing with CABLE prognostic variables (see {\hyperref[\detokenize{namelists/cable_prognostics.nml:namelist-CABLE_PROGS}]{\sphinxcrossref{\sphinxcode{\sphinxupquote{CABLE\_PROGS}}}}}). (\#1131)

\item {} 
\sphinxAtStartPar
Removed most compiler warnings flagged by UM builds. (\#1187)

\item {} 
\sphinxAtStartPar
Ported the JULES codebase to the Met Office’s EX1A system. (\#1193)

\item {} 
\sphinxAtStartPar
Updated Python scripts to be Python 3 compatible. (\#1091)

\item {} 
\sphinxAtStartPar
Preparing the JULES vn6.2 release. (\#1204)

\end{itemize}


\subsection{Bugs fixed}
\label{\detokenize{release_notes/JULES6-2:bugs-fixed}}\begin{itemize}
\item {} 
\sphinxAtStartPar
Bug fix to correct an array addressing issue in snow albedo calculations with {\hyperref[\detokenize{namelists/jules_radiation.nml:JULES_RADIATION::l_embedded_snow}]{\sphinxcrossref{\sphinxcode{\sphinxupquote{l\_embedded\_snow}}}}} = TRUE. (\#1064)

\item {} 
\sphinxAtStartPar
Deallocating a few more arrays in ancil\_info. (\#1190)

\item {} 
\sphinxAtStartPar
Fix to ensure that in coupled NWP models lake ice temperatures (where lakes are defined as sea points) evolve correctly \sphinxhyphen{} this is fixed by setting {\hyperref[\detokenize{namelists/science_fixes.nml:JULES_TEMP_FIXES::l_fix_lake_ice_temperatures}]{\sphinxcrossref{\sphinxcode{\sphinxupquote{l\_fix\_lake\_ice\_temperatures}}}}} = TRUE. (\#1161)

\item {} 
\sphinxAtStartPar
Prevent error in internal write of ctile\_orog\_fix (potentially many characters long) into a 3 character buffer. (\#1205)

\item {} 
\sphinxAtStartPar
Fix arguments to subroutine next\_time. (\#1209)

\end{itemize}


\subsection{Documentation updates}
\label{\detokenize{release_notes/JULES6-2:documentation-updates}}\begin{itemize}
\item {} 
\sphinxAtStartPar
Updates associated with many of the above changes, and release notes. (\#1156, 1186, 1192, 1198)

\end{itemize}

\sphinxAtStartPar
Documentation can be viewed on the github page \sphinxurl{http://jules-lsm.github.io/}.

\sphinxstepscope


\section{JULES version 6.1 Release Notes}
\label{\detokenize{release_notes/JULES6-1:jules-version-6-1-release-notes}}\label{\detokenize{release_notes/JULES6-1::doc}}
\sphinxAtStartPar
The JULES vn6.1 release consists of approximately 21 tickets from 13 authors, including work by many other people.

\sphinxAtStartPar
Full details of the tickets committed for JULES vn6.1 can be found on the \sphinxhref{https://code.metoffice.gov.uk/trac/jules/query?resolution=fixed\&milestone=JULES+v6.1+(Jun-21)}{JULES shared repository Trac system}.

\sphinxAtStartPar
Ticket numbers are indicated below, e.g. \#949.


\subsection{General/Technical changes}
\label{\detokenize{release_notes/JULES6-1:general-technical-changes}}\begin{itemize}
\item {} 
\sphinxAtStartPar
Added the technical infrastructure to activate irrigation in the UM (atmosphere model). (\#949)

\item {} 
\sphinxAtStartPar
Irrigation timestep is controlled by {\hyperref[\detokenize{namelists/jules_irrig.nml:JULES_IRRIG::nstep_irrig}]{\sphinxcrossref{\sphinxcode{\sphinxupquote{nstep\_irrig}}}}}. (\#1146)

\item {} 
\sphinxAtStartPar
Minor restructuring of water resources code. (\#1122)

\item {} 
\sphinxAtStartPar
The handling of the MORUSES roughness length in the surface exchange has been made more robust and the functionality for calculating urban morphology from the empirical relationships of Bohnenstengel et al. 2011 has also been fixed. (\#1106)

\item {} 
\sphinxAtStartPar
Forcing and lake variables bundled into TYPEs. (\#1136, 1135)

\item {} 
\sphinxAtStartPar
Conversion of initialisation and io/dump include files to modules, including some reorganisation required for River.Exe (standalone rivers\sphinxhyphen{}only executable). (\#1158)

\item {} 
\sphinxAtStartPar
CABLE lai\_pft and canht\_pft initialised from new namelist {\hyperref[\detokenize{namelists/cable_pftparm.nml:namelist-CABLE_PFTPARM}]{\sphinxcrossref{\sphinxcode{\sphinxupquote{CABLE\_PFTPARM}}}}}. (\#1119)

\item {} 
\sphinxAtStartPar
Miscellaneous small corrections to code and documentation. (\#1143)

\item {} 
\sphinxAtStartPar
Added new 1.5m visibility diagnostics to sfdiags (only for UM). (\#1134)

\item {} 
\sphinxAtStartPar
Improved performance in RA3 configurations. (\#1113)

\item {} 
\sphinxAtStartPar
Added further platform files for JASMIN. (\#1145)

\end{itemize}


\subsection{Bugs fixed}
\label{\detokenize{release_notes/JULES6-1:bugs-fixed}}\begin{itemize}
\item {} 
\sphinxAtStartPar
Corrected previously uninitialised PFT parameters when coupled to the UM. (\#1137)

\item {} 
\sphinxAtStartPar
Minor bug fixes and metadata updates for IMOGEN. (\#1126)

\item {} 
\sphinxAtStartPar
Bug fixes for seed\_rain. (\#1129, 1166)

\item {} 
\sphinxAtStartPar
Prevent memory issues when using the INFERNO fire scheme in UM configurations. (\#1149)

\item {} 
\sphinxAtStartPar
Bug fix to add Rose metadata and upgrade macro for {\hyperref[\detokenize{namelists/ancillaries.nml:namelist-JULES_VEGETATION_PROPS}]{\sphinxcrossref{\sphinxcode{\sphinxupquote{JULES\_VEGETATION\_PROPS}}}}} namelist. (\#1141)

\end{itemize}


\subsection{Changes to testing}
\label{\detokenize{release_notes/JULES6-1:changes-to-testing}}\begin{itemize}
\item {} 
\sphinxAtStartPar
Reflected changes in the JASMIN environment (parallel netcdf and SLURM scheduler) in the settings for rose stem. (\#1120)

\end{itemize}


\subsection{Documentation updates}
\label{\detokenize{release_notes/JULES6-1:documentation-updates}}\begin{itemize}
\item {} 
\sphinxAtStartPar
Revised presentation of the available diagnostics. (\#1139)

\item {} 
\sphinxAtStartPar
Updates associated with many of the above changes, and release notes. (\#1157)

\end{itemize}

\sphinxAtStartPar
Documentation can be viewed on the github page \sphinxurl{http://jules-lsm.github.io/}.

\sphinxstepscope


\section{JULES version 6.0 Release Notes}
\label{\detokenize{release_notes/JULES6-0:jules-version-6-0-release-notes}}\label{\detokenize{release_notes/JULES6-0::doc}}
\sphinxAtStartPar
The JULES vn6.0 release consists of approximately 18 tickets from 10 authors, including work by many other people.

\sphinxAtStartPar
Full details of the tickets committed for JULES vn6.0 can be found on the \sphinxhref{https://code.metoffice.gov.uk/trac/jules/query?resolution=fixed\&milestone=JULES+v6.0+(Feb-21)}{JULES shared repository Trac system}.

\sphinxAtStartPar
Ticket numbers are indicated below, e.g. \#847.


\subsection{Science changes}
\label{\detokenize{release_notes/JULES6-0:science-changes}}\begin{itemize}
\item {} 
\sphinxAtStartPar
Added a microbial scheme for methane production (see {\hyperref[\detokenize{namelists/jules_soil_biogeochem.nml:JULES_SOIL_BIOGEOCHEM::l_ch4_microbe}]{\sphinxcrossref{\sphinxcode{\sphinxupquote{l\_ch4\_microbe}}}}}). (\#847)

\item {} 
\sphinxAtStartPar
Changes to soil evaporation: added a PFT\sphinxhyphen{}specific factor to multiply soil conductance under the canopy ({\hyperref[\detokenize{namelists/pft_params.nml:JULES_PFTPARM::gsoil_f_io}]{\sphinxcrossref{\sphinxcode{\sphinxupquote{gsoil\_f\_io}}}}}) and a switch to limit soil conductance above the critical point ({\hyperref[\detokenize{namelists/jules_hydrology.nml:JULES_HYDROLOGY::l_limit_gsoil}]{\sphinxcrossref{\sphinxcode{\sphinxupquote{l\_limit\_gsoil}}}}}). (\#1093)

\item {} 
\sphinxAtStartPar
Added option (l\_icerough\_prognostic) to use the prognostic sea ice roughness length calculated in CICE, rather than a constant value (available only when coupled). (\#583)

\item {} 
\sphinxAtStartPar
Calculate sea ice penetrating solar radiation to be sent to sea ice model. (\#1100)

\end{itemize}


\subsection{General/Technical changes}
\label{\detokenize{release_notes/JULES6-0:general-technical-changes}}\begin{itemize}
\item {} 
\sphinxAtStartPar
Irrigation code adapted for use with {\hyperref[\detokenize{namelists/jules_water_resources.nml:JULES_WATER_RESOURCES::l_water_resources}]{\sphinxcrossref{\sphinxcode{\sphinxupquote{l\_water\_resources}}}}} = TRUE. (\#1086)

\item {} 
\sphinxAtStartPar
Surface type IDs have been made available to JULES on a switch \sphinxcode{\sphinxupquote{l\_surface\_type\_ids}}, however they currently have no functionality and are not passed to the JULES I/O. As part of the metadata consolidation project the namelists {\hyperref[\detokenize{namelists/jules_surface_types.nml:namelist-JULES_SURFACE_TYPES}]{\sphinxcrossref{\sphinxcode{\sphinxupquote{JULES\_SURFACE\_TYPES}}}}} and \sphinxtitleref{JULES\_LSM\_SWITCH} are now consolidated, the latter by amalgamating it with {\hyperref[\detokenize{namelists/model_environment.nml:namelist-JULES_MODEL_ENVIRONMENT}]{\sphinxcrossref{\sphinxcode{\sphinxupquote{JULES\_MODEL\_ENVIRONMENT}}}}}. The GUI panels are also now consistently named in UM and standalone. In parallel runs, informative output can now also be limited to Task 0 (see {\hyperref[\detokenize{namelists/jules_prnt_control.nml:namelist-JULES_PRNT_CONTROL}]{\sphinxcrossref{\sphinxcode{\sphinxupquote{JULES\_PRNT\_CONTROL}}}}}). (\#1095)

\item {} 
\sphinxAtStartPar
Prevented duplicate (lat,lon) pairs being prescribed. (\#1107)

\item {} 
\sphinxAtStartPar
Extended the TYPE design to sweep up several small modules. (\#1102)

\item {} 
\sphinxAtStartPar
Optimisation in snow control routine. (\#1110)

\item {} 
\sphinxAtStartPar
Removed some default values for values read via the {\hyperref[\detokenize{namelists/drive.nml:namelist-JULES_DRIVE}]{\sphinxcrossref{\sphinxcode{\sphinxupquote{JULES\_DRIVE}}}}} namelist, and improved triggering. (\#1033)

\item {} 
\sphinxAtStartPar
Initialised further namelist variables before use. (\#598)

\item {} 
\sphinxAtStartPar
Fixed bad characters in metadata. (\#1103)

\item {} 
\sphinxAtStartPar
Further work to keep JULES and UM metadata consistent. (\#1118)

\end{itemize}


\subsection{Bugs fixed}
\label{\detokenize{release_notes/JULES6-0:bugs-fixed}}\begin{itemize}
\item {} 
\sphinxAtStartPar
Added option ({\hyperref[\detokenize{namelists/science_fixes.nml:JULES_TEMP_FIXES::l_accurate_rho}]{\sphinxcrossref{\sphinxcode{\sphinxupquote{l\_accurate\_rho}}}}}) to use accurate calculation of surface air density in surface exchange calculations. (\#194)

\item {} 
\sphinxAtStartPar
Fix to remove erroneously large river flows at river mouths with RFM ({\hyperref[\detokenize{namelists/jules_rivers.nml:JULES_RIVERS::i_river_vn}]{\sphinxcrossref{\sphinxcode{\sphinxupquote{i\_river\_vn}}}}} = 2). (\#1081)

\item {} 
\sphinxAtStartPar
Fixed unallocated TRIFFID variable in the new progs structure. (\#1109)

\end{itemize}


\subsection{Documentation updates}
\label{\detokenize{release_notes/JULES6-0:documentation-updates}}\begin{itemize}
\item {} 
\sphinxAtStartPar
Updates associated with many of the above changes, and release notes. (\#1097, 1112)

\end{itemize}

\sphinxAtStartPar
Documentation can be viewed on the github page \sphinxurl{http://jules-lsm.github.io/}.

\sphinxstepscope


\section{JULES version 5.9 Release Notes}
\label{\detokenize{release_notes/JULES5-9:jules-version-5-9-release-notes}}\label{\detokenize{release_notes/JULES5-9::doc}}
\sphinxAtStartPar
The JULES vn5.9 release consists of approximately 21 tickets from 11 authors, including work by many other people.

\sphinxAtStartPar
Full details of the tickets committed for JULES vn5.9 can be found on the \sphinxhref{https://code.metoffice.gov.uk/trac/jules/query?resolution=fixed\&milestone=JULES+v5.9+(Oct-20)}{JULES shared repository Trac system}.

\sphinxAtStartPar
Ticket numbers are indicated below, e.g. \#499.


\subsection{Science changes}
\label{\detokenize{release_notes/JULES5-9:science-changes}}\begin{itemize}
\item {} 
\sphinxAtStartPar
The FLake lake model is now available in standalone JULES (as well as in the UM), modelling sub\sphinxhyphen{}surface conditions on the lake tile \sphinxhyphen{} see {\hyperref[\detokenize{namelists/jules_surface.nml:JULES_SURFACE::l_flake_model}]{\sphinxcrossref{\sphinxcode{\sphinxupquote{l\_flake\_model}}}}}. (\#499)

\item {} 
\sphinxAtStartPar
Fix to stop snow from accumulating on unfrozen lakes when running the FLake lake model. (\#1057)

\item {} 
\sphinxAtStartPar
Options to disaggregate the albedo of bare soil between the VIS and NIR regions and to include the zenith\sphinxhyphen{}angle dependence of the bare soil albedo have been included. See {\hyperref[\detokenize{namelists/jules_radiation.nml:JULES_RADIATION::l_hapke_soil}]{\sphinxcrossref{\sphinxcode{\sphinxupquote{l\_hapke\_soil}}}}} and {\hyperref[\detokenize{namelists/jules_radiation.nml:JULES_RADIATION::l_partition_albsoil}]{\sphinxcrossref{\sphinxcode{\sphinxupquote{l\_partition\_albsoil}}}}}. (\#1020)

\end{itemize}


\subsection{General/Technical changes}
\label{\detokenize{release_notes/JULES5-9:general-technical-changes}}\begin{itemize}
\item {} 
\sphinxAtStartPar
Modifications to ensure the irrigation namelist for standalone JULES is consistent with the UM namelist structure. This involved moving irrigation switches to namelist {\hyperref[\detokenize{namelists/jules_irrig.nml:namelist-JULES_IRRIG}]{\sphinxcrossref{\sphinxcode{\sphinxupquote{JULES\_IRRIG}}}}} and adding the {\hyperref[\detokenize{namelists/ancillaries.nml:namelist-JULES_IRRIG_PROPS}]{\sphinxcrossref{\sphinxcode{\sphinxupquote{JULES\_IRRIG\_PROPS}}}}} namelist for ancillary fields. (\#838)

\item {} 
\sphinxAtStartPar
Added initial control\sphinxhyphen{}level code for water resources. (\#1018)

\item {} 
\sphinxAtStartPar
The interface for the \sphinxtitleref{river\_control} subroutine now follows the model used by the \sphinxtitleref{surf\_couple\_*} routines. This is part of activities to create a standalone river routing capability. (\#1066)

\item {} 
\sphinxAtStartPar
Added latent heat flux diagnostics separately for land and sea/sea\sphinxhyphen{}ice (Unified Model runs only). (\#992)

\item {} 
\sphinxAtStartPar
Surface diagnostics for 1.5m temperature, specific and relative humidity, dewpoint temperature, and fog fraction separately over ocean (sea + sea ice) and land made available in the UM (not available in standalone JULES). (\#1074)

\item {} 
\sphinxAtStartPar
Removed unused arguments, to avoid UM compiler warnings. (\#1069)

\item {} 
\sphinxAtStartPar
Optimisation of code used in GA9. (\#1071)

\item {} 
\sphinxAtStartPar
Added \sphinxtitleref{ONLY}s to all \sphinxtitleref{USE} statements that did not have them, to avoid undesirable effects. (\#1072)

\item {} 
\sphinxAtStartPar
All transparent, non\sphinxhyphen{}functional metadata consolidated with the UM, including \sphinxtitleref{sort\sphinxhyphen{}key}, \sphinxtitleref{description}, \sphinxtitleref{url} and \sphinxtitleref{help}, which has been removed where duplicated by the JULES user guide or merged in. (\#1055)

\item {} 
\sphinxAtStartPar
Keep standalone and UM JULES meta data consistent. (\#1083)

\end{itemize}


\subsection{Bugs fixed}
\label{\detokenize{release_notes/JULES5-9:bugs-fixed}}\begin{itemize}
\item {} 
\sphinxAtStartPar
Fixed bug that affected the reading of soil ancillary fields when {\hyperref[\detokenize{namelists/ancillaries.nml:JULES_SOIL_PROPS::const_z}]{\sphinxcrossref{\sphinxcode{\sphinxupquote{const\_z}}}}} = TRUE in the JULES\_SOIL\_PROPS namelist. The bug had the potential to result in some soil ancillary fields being zero \sphinxhyphen{} which would likely have resulted in an obviously wrong result and/or a model crash. (\#1080)

\item {} 
\sphinxAtStartPar
Fixed bug in the reading of dumps containing rain\_seed fields. (\#1059)

\item {} 
\sphinxAtStartPar
Fixes for various issues related to argument intents. (\#1058, 1062)

\item {} 
\sphinxAtStartPar
Fix for UM configs with iseasurfalg=2 (m10 now initialised). (\#975)

\end{itemize}


\subsection{Changes to testing}
\label{\detokenize{release_notes/JULES5-9:changes-to-testing}}\begin{itemize}
\item {} 
\sphinxAtStartPar
The gswp2\_irrig\_limit rose stem tests are included in the set run at UKCEH. (\#1053)

\item {} 
\sphinxAtStartPar
Added a further rose stem test with irrigation. (\#838)

\end{itemize}


\subsection{Documentation updates}
\label{\detokenize{release_notes/JULES5-9:documentation-updates}}\begin{itemize}
\item {} 
\sphinxAtStartPar
Updates associated with many of the above changes, and release notes. (\#1078)

\end{itemize}

\sphinxAtStartPar
Documentation can be viewed on the github page \sphinxurl{http://jules-lsm.github.io/}.

\sphinxstepscope


\section{JULES version 5.8 Release Notes}
\label{\detokenize{release_notes/JULES5-8:jules-version-5-8-release-notes}}\label{\detokenize{release_notes/JULES5-8::doc}}
\sphinxAtStartPar
The JULES vn5.8 release consists of approximately 19 tickets from 9 authors, including work by many other people. This was an unusual release cycle as it was primarily aimed at technical developments in support of changes to the Unified Model system.

\sphinxAtStartPar
Full details of the tickets committed for JULES vn5.8 can be found on the \sphinxhref{https://code.metoffice.gov.uk/trac/jules/query?resolution=fixed\&milestone=JULES+v5.8+(Jun-20)}{JULES shared repository Trac system}.

\sphinxAtStartPar
Ticket numbers are indicated below, e.g. \#919.


\subsection{General/Technical changes}
\label{\detokenize{release_notes/JULES5-8:general-technical-changes}}\begin{itemize}
\item {} 
\sphinxAtStartPar
The surf\_couple* routines have been restructured to separate land and sea/sea\sphinxhyphen{}ice parts \sphinxhyphen{} making the code clearer and facilitating the implementation of CABLE into the JULES framework. (\#919)

\item {} 
\sphinxAtStartPar
Fields passed by argument rather than via USE statements. (\#1022, \#1024, \#1025, \#1028, \#1029, \#1030, \#1037, \#1038, \#1043)

\item {} 
\sphinxAtStartPar
Added {\hyperref[\detokenize{namelists/output.nml:JULES_OUTPUT::dump_period_unit}]{\sphinxcrossref{\sphinxcode{\sphinxupquote{dump\_period\_unit}}}}} to allow the units of {\hyperref[\detokenize{namelists/output.nml:JULES_OUTPUT::dump_period}]{\sphinxcrossref{\sphinxcode{\sphinxupquote{dump\_period}}}}} to be years or seconds. (\#1021)

\item {} 
\sphinxAtStartPar
Removal of the old qsat routines. (\#1015)

\item {} 
\sphinxAtStartPar
Access the gather\_field and scatter\_field subroutines via modules. (\#1051)

\item {} 
\sphinxAtStartPar
The url fields in HEAD metadata now have the “latest” tag, allowing both the UM and JULES to have the same url field. (\#1017)

\end{itemize}


\subsection{Bugs fixed}
\label{\detokenize{release_notes/JULES5-8:bugs-fixed}}\begin{itemize}
\item {} 
\sphinxAtStartPar
Bug fix to allow runs with soil tiling that do not rely on broadcastng values. (\#1048)

\item {} 
\sphinxAtStartPar
Bug fix to prevent array bounds error with river routing, and other minor changes to river routing. (\#1049)

\item {} 
\sphinxAtStartPar
Fix to allow compilation with pgfortran as part of UM\sphinxhyphen{}JULES. (\#1010)

\end{itemize}


\subsection{Changes to testing}
\label{\detokenize{release_notes/JULES5-8:changes-to-testing}}\begin{itemize}
\item {} 
\sphinxAtStartPar
Tidied some rose stem apps. (\#1050)

\end{itemize}


\subsection{Documentation updates}
\label{\detokenize{release_notes/JULES5-8:documentation-updates}}\begin{itemize}
\item {} 
\sphinxAtStartPar
Updates associated with many of the above changes, and release notes. (\#1040)

\end{itemize}

\sphinxAtStartPar
Documentation can be viewed on the github page \sphinxurl{http://jules-lsm.github.io/}.

\sphinxstepscope


\section{JULES version 5.7 Release Notes}
\label{\detokenize{release_notes/JULES5-7:jules-version-5-7-release-notes}}\label{\detokenize{release_notes/JULES5-7::doc}}
\sphinxAtStartPar
The JULES vn5.7 release consists of approximately 34 tickets from 16 authors, including work by many other people.

\sphinxAtStartPar
Full details of the tickets committed for JULES vn5.7 can be found on the \sphinxhref{https://code.metoffice.gov.uk/trac/jules/query?resolution=fixed\&milestone=JULES+v5.7+(Feb-20)}{JULES shared repository Trac system}.

\sphinxAtStartPar
Ticket numbers are indicated below, e.g. \#548.


\subsection{Science changes}
\label{\detokenize{release_notes/JULES5-7:science-changes}}\begin{itemize}
\item {} 
\sphinxAtStartPar
Soil tiling made available \sphinxhyphen{} see {\hyperref[\detokenize{namelists/jules_soil.nml:JULES_SOIL::l_tile_soil}]{\sphinxcrossref{\sphinxcode{\sphinxupquote{l\_tile\_soil}}}}}. Each land gridbox can be modelled using a single soil column or with a separate soil column for each surface tile. (\#548)

\item {} 
\sphinxAtStartPar
Enabled the simulation of multiple crops in a growing season, and therefore crop rotations \sphinxhyphen{} see {\hyperref[\detokenize{namelists/jules_vegetation.nml:JULES_VEGETATION::l_croprotate}]{\sphinxcrossref{\sphinxcode{\sphinxupquote{l\_croprotate}}}}}. (\#821)

\item {} 
\sphinxAtStartPar
Added thermal adaptation and acclimation of leaf\sphinxhyphen{}level photosynthesis \sphinxhyphen{} see {\hyperref[\detokenize{namelists/jules_vegetation.nml:JULES_VEGETATION::photo_acclim_model}]{\sphinxcrossref{\sphinxcode{\sphinxupquote{photo\_acclim\_model}}}}}. (\#863)

\item {} 
\sphinxAtStartPar
Remaining source/sink terms for inorganic nitrogen included with the ECOSSE soil model ({\hyperref[\detokenize{namelists/jules_soil_biogeochem.nml:JULES_SOIL_BIOGEOCHEM::soil_bgc_model}]{\sphinxcrossref{\sphinxcode{\sphinxupquote{soil\_bgc\_model}}}}} = 3). (\#788)

\end{itemize}


\subsection{General/Technical changes}
\label{\detokenize{release_notes/JULES5-7:general-technical-changes}}\begin{itemize}
\item {} 
\sphinxAtStartPar
New functionality added to allow for a new vegetation biogeochemical model. Introduced more extensive use of modules to control the read and write of variable states, and FORTRAN type objects to control data flows. This code is not yet suitable for general use. (\#888)

\item {} 
\sphinxAtStartPar
Technical work for the implementation of the Robust Ecosystem Demography (RED) dynamic vegetation model. (\#902)

\item {} 
\sphinxAtStartPar
Moved allocation statements into the modules that hold the variables (and out of the monolithic allocate\_jules\_arays.F90). (\#978)

\item {} 
\sphinxAtStartPar
Improved processing of the values read from namelist {\hyperref[\detokenize{namelists/jules_rivers.nml:namelist-JULES_OVERBANK}]{\sphinxcrossref{\sphinxcode{\sphinxupquote{JULES\_OVERBANK}}}}}. (\#987)

\item {} 
\sphinxAtStartPar
Added a KIND type to the declarations of REAL variables. (\#958, 996, 997)

\item {} 
\sphinxAtStartPar
Removed some redundant operations from the science code. (\#1001)

\item {} 
\sphinxAtStartPar
Removed redundant arguments for copydiag\_3d (affects UM runs only). (\#954)

\item {} 
\sphinxAtStartPar
Added STASH code controls for a new scale\sphinxhyphen{}dependent gust diagnostic in the UM. (\#984)

\item {} 
\sphinxAtStartPar
Trivial change to USE statements relating to a UM change to atm\_fields\_mod. (\#1003)

\item {} 
\sphinxAtStartPar
Further work on CABLE front end, adding namelists {\hyperref[\detokenize{namelists/cable_prognostics.nml:namelist-CABLE_PROGS}]{\sphinxcrossref{\sphinxcode{\sphinxupquote{CABLE\_PROGS}}}}} and {\hyperref[\detokenize{namelists/cable_surface_types.nml:namelist-CABLE_SURFACE_TYPES}]{\sphinxcrossref{\sphinxcode{\sphinxupquote{CABLE\_SURFACE\_TYPES}}}}}. (\#940)

\item {} 
\sphinxAtStartPar
Tidied to remove compiler warnings related to um\_parvars (decomposition) and unused variables (NAG compiler), and to make JULES compliant with changes to UM debug compiler flags. (\#963, 966, 1002)

\item {} 
\sphinxAtStartPar
Added some missing platforms to the metadata for fcm\sphinxhyphen{}make (so those are available in rose edit). (\#976)

\item {} 
\sphinxAtStartPar
Fixed the make\_jules\_release script to push latest documentation onto the master on GitHub. (\#957)

\item {} 
\sphinxAtStartPar
The example namelists (difficult to maintain) and benchmarking suite (obselete) have been removed. (\#928, 969)

\end{itemize}


\subsection{Bugs fixed}
\label{\detokenize{release_notes/JULES5-7:bugs-fixed}}\begin{itemize}
\item {} 
\sphinxAtStartPar
Bug fixes for the RothC\sphinxhyphen{}based soil biogeochemistry model ({\hyperref[\detokenize{namelists/jules_soil_biogeochem.nml:JULES_SOIL_BIOGEOCHEM::soil_bgc_model}]{\sphinxcrossref{\sphinxcode{\sphinxupquote{soil\_bgc\_model}}}}} = 2) with {\hyperref[\detokenize{namelists/jules_soil_biogeochem.nml:JULES_SOIL_BIOGEOCHEM::l_layeredc}]{\sphinxcrossref{\sphinxcode{\sphinxupquote{l\_layeredc}}}}} = TRUE and {\hyperref[\detokenize{namelists/jules_vegetation.nml:JULES_VEGETATION::l_nitrogen}]{\sphinxcrossref{\sphinxcode{\sphinxupquote{l\_nitrogen}}}}} = TRUE. (\#681)

\item {} 
\sphinxAtStartPar
Prevented race condition in leaf\_mod. (\#1009)

\end{itemize}


\subsection{Changes to testing}
\label{\detokenize{release_notes/JULES5-7:changes-to-testing}}\begin{itemize}
\item {} 
\sphinxAtStartPar
Added NAG compiler to Met Office linux test suite. (\#955)

\item {} 
\sphinxAtStartPar
Fixed JULES rose stem build jobs on the VM. (\#956)

\item {} 
\sphinxAtStartPar
Updated NIWA rose\sphinxhyphen{}stem configuration for Maui HPC. (\#965)

\item {} 
\sphinxAtStartPar
Changed the path to data and libraries for rose stem testing on JASMIN. The default is the new jules group\sphinxhyphen{}workspace, and an environment variable JASMIN\_JULES\_BASE\_DIR can be exported and picked up by rose stem. (\#967)

\end{itemize}


\subsection{Documentation updates}
\label{\detokenize{release_notes/JULES5-7:documentation-updates}}\begin{itemize}
\item {} 
\sphinxAtStartPar
Improved description of {\hyperref[\detokenize{namelists/jules_radiation.nml:JULES_RADIATION::l_embedded_snow}]{\sphinxcrossref{\sphinxcode{\sphinxupquote{l\_embedded\_snow}}}}}. (\#916)

\item {} 
\sphinxAtStartPar
Assorted clarifications in documentation and comments. (\#980)

\item {} 
\sphinxAtStartPar
Basic infrastructure provided to support future JULES Documentation Papers. (\#968)

\item {} 
\sphinxAtStartPar
Updates associated with many of the above changes, and release notes. (\#995)

\end{itemize}

\sphinxAtStartPar
Documentation can be viewed on the github page \sphinxurl{http://jules-lsm.github.io/}.

\sphinxstepscope


\section{JULES version 5.6 Release Notes}
\label{\detokenize{release_notes/JULES5-6:jules-version-5-6-release-notes}}\label{\detokenize{release_notes/JULES5-6::doc}}
\sphinxAtStartPar
The JULES vn5.6 release consists of approximately 14 tickets from 11 authors, including work by many other people.

\sphinxAtStartPar
Full details of the tickets committed for JULES vn5.6 can be found on the \sphinxhref{https://code.metoffice.gov.uk/trac/jules/query?resolution=fixed\&milestone=JULES+v5.6+release}{JULES shared repository Trac system}.

\sphinxAtStartPar
Ticket numbers are indicated below, e.g. \#864.


\subsection{Science changes}
\label{\detokenize{release_notes/JULES5-6:science-changes}}\begin{itemize}
\item {} 
\sphinxAtStartPar
Added the Farquhar model of photosynthesis for C$_{\text{3}}$ plants \sphinxhyphen{} see {\hyperref[\detokenize{namelists/jules_vegetation.nml:JULES_VEGETATION::photo_model}]{\sphinxcrossref{\sphinxcode{\sphinxupquote{photo\_model}}}}}. (\#864)

\item {} 
\sphinxAtStartPar
Added the COARE algorithm for drag over sea, including functionality to reduce the drag at very high wind speeds \sphinxhyphen{} only affects runs with the UM. (\#848)

\end{itemize}


\subsection{General/Technical changes}
\label{\detokenize{release_notes/JULES5-6:general-technical-changes}}\begin{itemize}
\item {} 
\sphinxAtStartPar
UM and JULES metadata consolidated for namelists {\hyperref[\detokenize{namelists/jules_radiation.nml:namelist-JULES_RADIATION}]{\sphinxcrossref{\sphinxcode{\sphinxupquote{JULES\_RADIATION}}}}} and {\hyperref[\detokenize{namelists/jules_vegetation.nml:namelist-JULES_VEGETATION}]{\sphinxcrossref{\sphinxcode{\sphinxupquote{JULES\_VEGETATION}}}}}. (\#822)

\item {} 
\sphinxAtStartPar
Preparatory work towards including irrigation demand in the UM. (\#811)

\item {} 
\sphinxAtStartPar
Allowed the dimension names {\hyperref[\detokenize{namelists/model_grid.nml:JULES_INPUT_GRID::sclayer_dim_name}]{\sphinxcrossref{\sphinxcode{\sphinxupquote{sclayer\_dim\_name}}}}}, {\hyperref[\detokenize{namelists/model_grid.nml:JULES_INPUT_GRID::tracer_dim_name}]{\sphinxcrossref{\sphinxcode{\sphinxupquote{tracer\_dim\_name}}}}}, {\hyperref[\detokenize{namelists/model_grid.nml:JULES_INPUT_GRID::bl_level_dim_name}]{\sphinxcrossref{\sphinxcode{\sphinxupquote{bl\_level\_dim\_name}}}}} to be read from namelist {\hyperref[\detokenize{namelists/model_grid.nml:namelist-JULES_INPUT_GRID}]{\sphinxcrossref{\sphinxcode{\sphinxupquote{JULES\_INPUT\_GRID}}}}}. (\#937)

\item {} 
\sphinxAtStartPar
Improved error handling in subroutines \sphinxcode{\sphinxupquote{set\_levels\_list}} and \sphinxcode{\sphinxupquote{set\_pseudo\_list}}. (\#935)

\item {} 
\sphinxAtStartPar
Updated the UM STASH diagnostics routines to use a modularised version of copydiag. (\#938)

\item {} 
\sphinxAtStartPar
Added further OpenMP optimisations for GA8 model routines. (\#941)

\item {} 
\sphinxAtStartPar
Additions of \sphinxcode{\sphinxupquote{\#if defined(LFRIC)}} to allow coupling of \sphinxcode{\sphinxupquote{surf\_couple\_extra}} to LFRic. (\#943)

\item {} 
\sphinxAtStartPar
Altered OMP directives to remove data race conditions. (\#952)

\item {} 
\sphinxAtStartPar
Removed the requirement to have some environment variables that are used in build configs defined/initiliased at the app/suite level. (\#939)

\end{itemize}


\subsection{Changes to testing}
\label{\detokenize{release_notes/JULES5-6:changes-to-testing}}\begin{itemize}
\item {} 
\sphinxAtStartPar
Updated rose stem testing to include JULES\sphinxhyphen{}ES\sphinxhyphen{}1.0 configuration. (\#915)

\item {} 
\sphinxAtStartPar
JULES can be built at the Bureau of Meteorology (Australia). (\#930)

\end{itemize}


\subsection{Documentation updates}
\label{\detokenize{release_notes/JULES5-6:documentation-updates}}\begin{itemize}
\item {} 
\sphinxAtStartPar
Updates associated with many of the above changes, and release notes. (\#950)

\end{itemize}

\sphinxAtStartPar
Note that compilation of the User Guide now requires Sphinx 1.4 or higher.

\sphinxAtStartPar
Documentation can be viewed on the github page \sphinxurl{http://jules-lsm.github.io/}.

\sphinxstepscope


\section{JULES version 5.5 Release Notes}
\label{\detokenize{release_notes/JULES5-5:jules-version-5-5-release-notes}}\label{\detokenize{release_notes/JULES5-5::doc}}
\sphinxAtStartPar
The JULES vn5.5 release consists of approximately 18 tickets from 14 authors, including work by many other people.

\sphinxAtStartPar
Full details of the tickets committed for JULES vn5.5 can be found on the \sphinxhref{https://code.metoffice.gov.uk/trac/jules/query?resolution=fixed\&milestone=JULES+v5.5+release}{JULES shared repository Trac system}.

\sphinxAtStartPar
Ticket numbers are indicated below, e.g. \#434.


\subsection{Science changes}
\label{\detokenize{release_notes/JULES5-5:science-changes}}\begin{itemize}
\item {} 
\sphinxAtStartPar
Methane feedbacks from natural wetlands added to IMOGEN \sphinxhyphen{} see {\hyperref[\detokenize{namelists/imogen.nml:IMOGEN_RUN_LIST::land_feed_ch4}]{\sphinxcrossref{\sphinxcode{\sphinxupquote{land\_feed\_ch4}}}}}. This is done by adding an anomaly relative to the default emissions. Users should confirm that the wetland emissions are correct \sphinxhyphen{} see the notes under {\hyperref[\detokenize{namelists/imogen.nml:IMOGEN_RUN_LIST::land_feed_ch4}]{\sphinxcrossref{\sphinxcode{\sphinxupquote{land\_feed\_ch4}}}}}.

\item {} 
\sphinxAtStartPar
Code changes for the GL9 configuration, including options to specify specific values for the roughness length of each Plant Functional Type ({\hyperref[\detokenize{namelists/jules_vegetation.nml:JULES_VEGETATION::l_spec_veg_z0}]{\sphinxcrossref{\sphinxcode{\sphinxupquote{l\_spec\_veg\_z0}}}}}) and to impose a maximum\sphinxhyphen{}allowed value of the canopy heat capacity for vegetation ({\hyperref[\detokenize{namelists/jules_vegetation.nml:JULES_VEGETATION::l_limit_canhc}]{\sphinxcrossref{\sphinxcode{\sphinxupquote{l\_limit\_canhc}}}}}). (\#903)

\item {} 
\sphinxAtStartPar
Additional options for distributed form drag (\sphinxtitleref{fd\_stab\_dep}) \sphinxhyphen{} not available in off\sphinxhyphen{}line JULES.  (\#870)

\end{itemize}


\subsection{General/Technical changes}
\label{\detokenize{release_notes/JULES5-5:general-technical-changes}}\begin{itemize}
\item {} 
\sphinxAtStartPar
Separate the calculation of plant\sphinxhyphen{}soil N fluxes from the updating of the soil stores \sphinxhyphen{} to allow the fluxes to be used with alternative soil models. (\#651)

\item {} 
\sphinxAtStartPar
Initial steps towards representing dry deposition in JULES: new namelists {\hyperref[\detokenize{namelists/jules_deposition.nml:namelist-JULES_DEPOSITION}]{\sphinxcrossref{\sphinxcode{\sphinxupquote{JULES\_DEPOSITION}}}}} and {\hyperref[\detokenize{namelists/jules_deposition.nml:namelist-JULES_DEPOSITION_SPECIES}]{\sphinxcrossref{\sphinxcode{\sphinxupquote{JULES\_DEPOSITION\_SPECIES}}}}}. Note that deposition cannot yet be modelled in JULES. (\#662)

\item {} 
\sphinxAtStartPar
River routing code tidied. (\#877)

\item {} 
\sphinxAtStartPar
The interface (API) between the JULES program and the CONTROL subroutine now includes the atmospheric forcing variables as input and the gridbox mean surface flxues as output. This creates a clean “managed API” that can be used by parent models to call the JULES code at the CONTROL subroutine level. (\#914)

\item {} 
\sphinxAtStartPar
Added diagnostics for absorbed photosynthetically active radiation (\sphinxtitleref{apar} and \sphinxtitleref{apar\_gb}) and gridbox mean leaf area index (\sphinxtitleref{lai\_gb}) \sphinxhyphen{} see {\hyperref[\detokenize{output-variables:output-variables-section}]{\sphinxcrossref{\DUrole{std,std-ref}{JULES Output variables}}}}. (\#614, 890)

\item {} 
\sphinxAtStartPar
Corrected the units given for the ozone flux diagnostic (\sphinxtitleref{flux\_o3\_stom}). (\#905)

\item {} 
\sphinxAtStartPar
Improved computational performance of ice\sphinxhyphen{}related and other routines. (\#866, 894)

\item {} 
\sphinxAtStartPar
Removed code only used for UM (atmospheric model) diagnostics; now using the sf\_diag structure. (\#899)

\item {} 
\sphinxAtStartPar
Changes for UM diagnostics \sphinxhyphen{} to accommodate the fact that copydiag is now in a module. (\#908)

\item {} 
\sphinxAtStartPar
Moved the CABLE soil parameters to a separate namelist {\hyperref[\detokenize{namelists/cable_soilparm.nml:namelist-CABLE_SOILPARM}]{\sphinxcrossref{\sphinxcode{\sphinxupquote{CABLE\_SOILPARM}}}}}. (\#900)

\item {} 
\sphinxAtStartPar
Improvements to code involved in creating the JULES release, including the documentation. (\#893)

\end{itemize}


\subsection{Bugs fixed}
\label{\detokenize{release_notes/JULES5-5:bugs-fixed}}\begin{itemize}
\item {} 
\sphinxAtStartPar
Removed use of uninitialised memory in RFM river routing scheme. (\#896)

\end{itemize}


\subsection{Documentation updates}
\label{\detokenize{release_notes/JULES5-5:documentation-updates}}\begin{itemize}
\item {} 
\sphinxAtStartPar
Updates associated with many of the above changes, and release notes. (\#924)

\end{itemize}

\sphinxAtStartPar
Documentation can be viewed on the github page \sphinxurl{http://jules-lsm.github.io/}.

\sphinxstepscope


\section{JULES version 5.4 Release Notes}
\label{\detokenize{release_notes/JULES5-4:jules-version-5-4-release-notes}}\label{\detokenize{release_notes/JULES5-4::doc}}
\sphinxAtStartPar
The JULES vn5.4 release consists of approximately 29 tickets from 13 authors, including work by many other people.

\sphinxAtStartPar
Full details of the tickets committed for JULES vn5.4 can be found on the \sphinxhref{https://code.metoffice.gov.uk/trac/jules/query?resolution=fixed\&milestone=JULES+v5.4+release}{JULES shared repository Trac system}.

\sphinxAtStartPar
Ticket numbers are indicated below, e.g. \#872.


\subsection{Science changes}
\label{\detokenize{release_notes/JULES5-4:science-changes}}\begin{itemize}
\item {} 
\sphinxAtStartPar
Improvements to fire\sphinxhyphen{}related vegetation mortality, including the addition of a PFT\sphinxhyphen{}specific fire mortality parameter {\hyperref[\detokenize{namelists/pft_params.nml:JULES_PFTPARM::fire_mort_io}]{\sphinxcrossref{\sphinxcode{\sphinxupquote{fire\_mort\_io}}}}} (previously mortality was taken directly from the burnt area as diagnosed by INFERNO and did not vary by PFT). (\#872)

\item {} 
\sphinxAtStartPar
Stomatal conductance can be modelled following the approach of \sphinxhref{https://doi.org/10.1111/j.1365-2486.2010.02375.x}{Medlyn et al. (2011)}, via the switch {\hyperref[\detokenize{namelists/jules_vegetation.nml:JULES_VEGETATION::stomata_model}]{\sphinxcrossref{\sphinxcode{\sphinxupquote{stomata\_model}}}}}. A single\sphinxhyphen{}parameter version of the model is coded, requiring the PFT\sphinxhyphen{}specific parameter {\hyperref[\detokenize{namelists/pft_params.nml:JULES_PFTPARM::g1_stomata_io}]{\sphinxcrossref{\sphinxcode{\sphinxupquote{g1\_stomata\_io}}}}}. (\#766)

\end{itemize}


\subsection{General/Technical changes}
\label{\detokenize{release_notes/JULES5-4:general-technical-changes}}\begin{itemize}
\item {} 
\sphinxAtStartPar
The RFM river routing scheme is now available to the UM (atmospheric model), and both standalone and UM runs use the same code. See {\hyperref[\detokenize{namelists/jules_rivers.nml:namelist-JULES_RIVERS}]{\sphinxcrossref{\sphinxcode{\sphinxupquote{JULES\_RIVERS}}}}}. (\#876)

\item {} 
\sphinxAtStartPar
The {\hyperref[\detokenize{namelists/jules_rivers.nml:namelist-JULES_RIVERS}]{\sphinxcrossref{\sphinxcode{\sphinxupquote{JULES\_RIVERS}}}}} namelist now controls river hydrology in both standalone and UM\sphinxhyphen{}coupled modes.  (The UM namelist ‘run\_rivers’ has been removed.) (\#867)

\item {} 
\sphinxAtStartPar
The surface conductance (\sphinxtitleref{gs}) is now part of the specification of the initial state (and included in dumps) only when it is required (i.e. only if {\hyperref[\detokenize{namelists/jules_vegetation.nml:JULES_VEGETATION::can_rad_mod}]{\sphinxcrossref{\sphinxcode{\sphinxupquote{can\_rad\_mod}}}}} = 1; see {\hyperref[\detokenize{namelists/initial_conditions.nml:list-of-initial-condition-variables}]{\sphinxcrossref{\DUrole{std,std-ref}{List of initial condition variables}}}}). (\#859)

\item {} 
\sphinxAtStartPar
Use \sphinxcode{\sphinxupquote{swap\_bounds}} routine(s) from \sphinxcode{\sphinxupquote{halo\_exchange\_mod}} module (not the old 2C subroutine) in UM runs. (\#367)

\item {} 
\sphinxAtStartPar
Extensive refactoring of \sphinxcode{\sphinxupquote{surf\_couple\_extra}}, including removal of \sphinxcode{\sphinxupquote{ifdef}} in argument list. (\#806, 833)

\item {} 
\sphinxAtStartPar
Tidied/refactorised the photosynthesis code. (\#817)

\item {} 
\sphinxAtStartPar
Improved checking and reporting of the IMOGEN setup. (\#850)

\item {} 
\sphinxAtStartPar
Tidied the header of \sphinxcode{\sphinxupquote{control.F90}}, removing duplicate and unused variables. (\#873)

\item {} 
\sphinxAtStartPar
Access subroutines \sphinxcode{\sphinxupquote{set\_levels\_list}} and \sphinxcode{\sphinxupquote{set\_pseudo\_list}} using modules, removing the need for \sphinxcode{\sphinxupquote{DEPENDS ON}}. (\#880)

\item {} 
\sphinxAtStartPar
Improved performance of land surface routines in RA and GA configurations. (\#861)

\item {} 
\sphinxAtStartPar
Set up CABLE directory structure and initialise essential variables for \sphinxtitleref{surf\_couple\_radiation}. (\#799)

\item {} 
\sphinxAtStartPar
Allowed variables used in the build process to have platform\sphinxhyphen{}specific defaults which can be overridden by the user. (\#853)

\item {} 
\sphinxAtStartPar
Met Office Cray users: Direct extract of code to the Cray is the default for meto\sphinxhyphen{}xc40\sphinxhyphen{}cce builds. Users are encouraged to remove fcm\_make2 tasks and set \sphinxcode{\sphinxupquote{JULES\_REMOTE = local}} to take advantage of faster end\sphinxhyphen{}to\sphinxhyphen{}end compilations and reduce load on the HPC. Set \sphinxcode{\sphinxupquote{JULES\_REMOTE = remote}} to retain builds which require an fcm\_make2 task. See {\hyperref[\detokenize{building-and-running/fcm:fcm-make-environment-variables}]{\sphinxcrossref{\DUrole{std,std-ref}{Environment variables used when building JULES using FCM make}}}}. (\#854)

\item {} 
\sphinxAtStartPar
Reviewed and simplified fcm\sphinxhyphen{}make metadata compulsory variables and made current apps validate. (\#855)

\item {} 
\sphinxAtStartPar
Improved selected metadata in the {\hyperref[\detokenize{namelists/jules_soil_biogeochem.nml:namelist-JULES_SOIL_BIOGEOCHEM}]{\sphinxcrossref{\sphinxcode{\sphinxupquote{JULES\_SOIL\_BIOGEOCHEM}}}}} and {\hyperref[\detokenize{namelists/jules_soil_ecosse.nml:namelist-JULES_SOIL_ECOSSE}]{\sphinxcrossref{\sphinxcode{\sphinxupquote{JULES\_SOIL\_ECOSSE}}}}} namelists to prevent errors when using the Rose GUI. (\#862)

\end{itemize}


\subsection{Bugs fixed}
\label{\detokenize{release_notes/JULES5-4:bugs-fixed}}\begin{itemize}
\item {} 
\sphinxAtStartPar
Fixed the radiatively\sphinxhyphen{}coupled roof in MORUSES, using the temporary logical {\hyperref[\detokenize{namelists/science_fixes.nml:JULES_TEMP_FIXES::l_fix_moruses_roof_rad_coupling}]{\sphinxcrossref{\sphinxcode{\sphinxupquote{l\_fix\_moruses\_roof\_rad\_coupling}}}}}, in the new namelist {\hyperref[\detokenize{namelists/science_fixes.nml:namelist-JULES_TEMP_FIXES}]{\sphinxcrossref{\sphinxcode{\sphinxupquote{JULES\_TEMP\_FIXES}}}}}. The supposedly radiatively\sphinxhyphen{}coupled roof is in fact \sphinxstylestrong{uncoupled} without this bug fix.   (\#610)

\item {} 
\sphinxAtStartPar
Corrected initialisation of frozen/unfrozen soil \sphinxhyphen{} no longer assumes constant soil properties with depth. (\#749)

\item {} 
\sphinxAtStartPar
Removed a bug in the snow scheme when {\hyperref[\detokenize{namelists/jules_surface.nml:JULES_SURFACE::l_point_data}]{\sphinxcrossref{\sphinxcode{\sphinxupquote{l\_point\_data}}}}} = TRUE and {\hyperref[\detokenize{namelists/jules_vegetation.nml:JULES_VEGETATION::can_model}]{\sphinxcrossref{\sphinxcode{\sphinxupquote{can\_model}}}}} = 4: the snow covered fraction formulation is now only used for tiles that do not use the snow canopy option (see {\hyperref[\detokenize{namelists/jules_snow.nml:JULES_SNOW::cansnowpft}]{\sphinxcrossref{\sphinxcode{\sphinxupquote{cansnowpft}}}}}), rather than for all tiles. (\#879)

\item {} 
\sphinxAtStartPar
Prevent out\sphinxhyphen{}of\sphinxhyphen{}bounds operations in sf\_exch. (\#846)

\item {} 
\sphinxAtStartPar
Ensure that \sphinxtitleref{ntype} is set before use in UM model runs. (\#878)

\item {} 
\sphinxAtStartPar
Corrected the units of the ocean near\sphinxhyphen{}surface chlorophyll content (used in the calculation of the ocean surface albedo), using the temporary logical \sphinxtitleref{l\_fix\_osa\_chloro}. Only affects runs with the UM. (\#874)

\end{itemize}


\subsection{Changes to testing}
\label{\detokenize{release_notes/JULES5-4:changes-to-testing}}\begin{itemize}
\item {} 
\sphinxAtStartPar
Rose\sphinxhyphen{}stem fcm\sphinxhyphen{}make tasks will ignore lock files that would otherwise prevent retriggering. (\#860)

\item {} 
\sphinxAtStartPar
Expanded coverage of the rose\sphinxhyphen{}stem metadata validation test to include more apps. (\#886)

\item {} 
\sphinxAtStartPar
Upgraded \sphinxcode{\sphinxupquote{suite\_report.py}}. (\#889)

\end{itemize}


\subsection{Documentation updates}
\label{\detokenize{release_notes/JULES5-4:documentation-updates}}\begin{itemize}
\item {} 
\sphinxAtStartPar
Updates associated with many of the above changes, and release notes. (\#881)

\item {} 
\sphinxAtStartPar
Example namelists point\_mead\_2\_crops have been updated to be consistent with the published JULES\sphinxhyphen{}crop runs in \sphinxhref{https://doi.org/10.5194/gmd-10-1291-2017}{Williams et al. (2017)}.

\end{itemize}

\sphinxAtStartPar
Documentation can be viewed on the github page \sphinxurl{http://jules-lsm.github.io/}.

\sphinxstepscope


\section{JULES version 5.3 Release Notes}
\label{\detokenize{release_notes/JULES5-3:jules-version-5-3-release-notes}}\label{\detokenize{release_notes/JULES5-3::doc}}
\sphinxAtStartPar
The JULES vn5.3 release consists of approximately 27 tickets from 17 authors, including work by many other people.

\sphinxAtStartPar
Full details of the tickets committed for JULES vn5.3 can be found on the \sphinxhref{https://code.metoffice.gov.uk/trac/jules/query?resolution=fixed\&milestone=JULES+v5.3+release}{JULES shared repository Trac system}.

\sphinxAtStartPar
Ticket numbers are indicated below, e.g. \#742.


\subsection{Science changes}
\label{\detokenize{release_notes/JULES5-3:science-changes}}\begin{itemize}
\item {} 
\sphinxAtStartPar
Improved initialisation of the surface exchange iteration ({\hyperref[\detokenize{namelists/jules_surface.nml:JULES_SURFACE::cor_mo_iter}]{\sphinxcrossref{\sphinxcode{\sphinxupquote{cor\_mo\_iter}}}}} = 4). (\#742)

\item {} 
\sphinxAtStartPar
Removed canopy radiation options ({\hyperref[\detokenize{namelists/jules_vegetation.nml:JULES_VEGETATION::can_rad_mod}]{\sphinxcrossref{\sphinxcode{\sphinxupquote{can\_rad\_mod}}}}}) 2 and 3. (\#791)

\item {} 
\sphinxAtStartPar
Added nitrification, denitrification and leaching to the ECOSSE soil model (which is not yet ready for use). (\#781)

\item {} 
\sphinxAtStartPar
Allowed the use of a variable Charnock parameter, provided by a wave model via coupling, instead of a constant value, in runs of the UM (atmosphere\sphinxhyphen{}ocean model). See iseasurfalg = 4 or 5 in the jules\_sea\_seaice namelist. (\#797)

\end{itemize}


\subsection{General/Technical changes}
\label{\detokenize{release_notes/JULES5-3:general-technical-changes}}\begin{itemize}
\item {} 
\sphinxAtStartPar
Coupling of river outflow from river grid to NEMO (ocean) grid via a 1D array. (\#624)

\item {} 
\sphinxAtStartPar
Added PBL gustiness parameter to namelist ({\hyperref[\detokenize{namelists/jules_surface.nml:JULES_SURFACE::beta_cnv_bl}]{\sphinxcrossref{\sphinxcode{\sphinxupquote{beta\_cnv\_bl}}}}}). (\#742)

\item {} 
\sphinxAtStartPar
Changes required to run the INFERNO fire model interactively in the UM. (\#800)

\item {} 
\sphinxAtStartPar
Initial modifications to allow irrigation code to be used in the UM. (\#809)

\item {} 
\sphinxAtStartPar
Optimised aspects of the canopy drag scheme. (\#795)

\item {} 
\sphinxAtStartPar
Resolved issues identified by OpenMP\sphinxhyphen{}related compiler warnings. (\#712)

\item {} 
\sphinxAtStartPar
Improved OpenMP coverage. (\#723, 815)

\item {} 
\sphinxAtStartPar
Improved performance in UM global ensemble configuration. (\#820)

\item {} 
\sphinxAtStartPar
Removed unused code from the surface scheme in standalone JULES. (\#753)

\item {} 
\sphinxAtStartPar
Removed idfefs from some of the surf\_couple routine argument lists. (\#830)

\item {} 
\sphinxAtStartPar
Introduced a framework to move to shared metadata between UM and standalone for ease of converting to and from UM and standalone apps, to assist with configuration management. This introduces the new namelist {\hyperref[\detokenize{namelists/model_environment.nml:namelist-JULES_MODEL_ENVIRONMENT}]{\sphinxcrossref{\sphinxcode{\sphinxupquote{JULES\_MODEL\_ENVIRONMENT}}}}}. (\#633)

\item {} 
\sphinxAtStartPar
Ported to NIWA’s XC50 platform. (\#814)

\item {} 
\sphinxAtStartPar
The meto\sphinxhyphen{}linux fcm\sphinxhyphen{}make configs have been converted to RHEL7. All Linux rose\sphinxhyphen{}stem tasks will run on RHEL7 SPICE nodes. Those using a meto\sphinxhyphen{}linux\sphinxhyphen{}intel\sphinxhyphen{}mpi build outside of rose\sphinxhyphen{}stem will need to set \sphinxtitleref{ROSE\_LAUNCHER\_LIST = mpiexec.hydra} or make an equivalent change to their personal \sphinxtitleref{rose.conf} file, and include the full path to the required MPI\sphinxhyphen{}enabled compiler. Refer to the \sphinxtitleref{runtime\sphinxhyphen{}linux\sphinxhyphen{}intel.rc} file in rose\sphinxhyphen{}stem for details. (\#835)

\item {} 
\sphinxAtStartPar
Corrected bug in the make\_jules\_release script. (\#796)

\item {} 
\sphinxAtStartPar
Miscellaneous administrative changes. (\#807, 839)

\end{itemize}


\subsection{Bugs fixed}
\label{\detokenize{release_notes/JULES5-3:bugs-fixed}}\begin{itemize}
\item {} 
\sphinxAtStartPar
Corrected a bug in the scalar roughness length diagnostic over sea when using anything other than a fixed roughness length, i.e. anything other than iseasurfalg=0. (\#794)

\item {} 
\sphinxAtStartPar
Fixed calls to mask compression routines in UM\sphinxhyphen{}only code. (\#826)

\end{itemize}


\subsection{Changes to testing}
\label{\detokenize{release_notes/JULES5-3:changes-to-testing}}\begin{itemize}
\item {} 
\sphinxAtStartPar
A test of the canopy drag scheme, loobos\_vegdrag, based on loobos\_gl8 has been added to rose\sphinxhyphen{}stem. (\#795)

\item {} 
\sphinxAtStartPar
Updated \sphinxtitleref{umdp3\_fixer.py} to run on *.inc files, and included these in the rose stem test. (\#776)

\item {} 
\sphinxAtStartPar
Implemented code to align continuation ampersands in column 79 as part of \sphinxtitleref{umdp3\_fixer.py}. (\#823)

\item {} 
\sphinxAtStartPar
The Met Office rose stem suite now runs all Linux tasks on SPICE, and the metadata checker task is now included in the JASMIN rose stem suite. (\#819)

\item {} 
\sphinxAtStartPar
Correction for rose stem on MONSooN. (\#852)

\end{itemize}


\subsection{Documentation updates}
\label{\detokenize{release_notes/JULES5-3:documentation-updates}}\begin{itemize}
\item {} 
\sphinxAtStartPar
Updates associated with many of the above changes, and release notes. (\#836)

\end{itemize}

\sphinxAtStartPar
Documentation can be viewed on the github page \sphinxurl{http://jules-lsm.github.io/}.

\sphinxstepscope


\section{JULES version 5.2 Release Notes}
\label{\detokenize{release_notes/JULES5-2:jules-version-5-2-release-notes}}\label{\detokenize{release_notes/JULES5-2::doc}}
\sphinxAtStartPar
The JULES vn5.2 release consists of approximately 52 tickets from 25 authors, including work by many other people.

\sphinxAtStartPar
Full details of the tickets committed for JULES vn5.2 can be found on the \sphinxhref{https://code.metoffice.gov.uk/trac/jules/query?resolution=fixed\&milestone=JULES+v5.2+release}{JULES shared repository Trac system}.

\sphinxAtStartPar
Ticket numbers are indicated below, e.g. \#754.


\subsection{Science changes}
\label{\detokenize{release_notes/JULES5-2:science-changes}}\begin{itemize}
\item {} 
\sphinxAtStartPar
Introduced option for vegetation canopy drag with optional correction for the roughness sublayer \sphinxhyphen{} see {\hyperref[\detokenize{namelists/jules_vegetation.nml:JULES_VEGETATION::l_vegdrag_pft}]{\sphinxcrossref{\sphinxcode{\sphinxupquote{l\_vegdrag\_pft}}}}} and {\hyperref[\detokenize{namelists/jules_vegetation.nml:JULES_VEGETATION::l_rsl_scalar}]{\sphinxcrossref{\sphinxcode{\sphinxupquote{l\_rsl\_scalar}}}}}. (\#754)

\item {} 
\sphinxAtStartPar
Soil decomposition added to the code for the ECOSSE soil model (which is not yet ready for use). (\#570)

\item {} 
\sphinxAtStartPar
Extension of the screen temperature decoupling diagnostics to screen humidity \sphinxhyphen{} only recommended for runs with the UM (atmospheric model). (\#508)

\item {} 
\sphinxAtStartPar
Added a new option to the sea albedo calculation to simulate the effect of freezing (sea\sphinxhyphen{}ice) below 271 K. (\#770)

\end{itemize}


\subsection{General/Technical changes}
\label{\detokenize{release_notes/JULES5-2:general-technical-changes}}\begin{itemize}
\item {} 
\sphinxAtStartPar
Enabled support for new routines to calculate qsat (the saturated water mixing ratio) which are now the default for standalone JULES. (\#685)

\item {} 
\sphinxAtStartPar
Improved interface checking in the surface, fire, FLake and river routing routines. (\#678, 728, 729)

\item {} 
\sphinxAtStartPar
The clay ancillary variable can now have multiple layers (note these are the soil biogeochemistry layers, not soil moisture layers). Users should note that an existing run with {\hyperref[\detokenize{namelists/jules_soil_biogeochem.nml:JULES_SOIL_BIOGEOCHEM::l_layeredc}]{\sphinxcrossref{\sphinxcode{\sphinxupquote{l\_layeredc}}}}} = T that tries to read a single\sphinxhyphen{}layered clay variable from a file with {\hyperref[\detokenize{namelists/ancillaries.nml:JULES_SOIL_PROPS::const_z}]{\sphinxcrossref{\sphinxcode{\sphinxupquote{const\_z}}}}} = F will no longer work; a multi\sphinxhyphen{}layered clay field must be provided in this case. All other configurations can be updated. (\#687)

\item {} 
\sphinxAtStartPar
Added further IMPLICIT NONE statements and fixed subroutine interface issues. (\#737)

\item {} 
\sphinxAtStartPar
Minor modifications to IMOGEN (\#430).

\item {} 
\sphinxAtStartPar
Removed the l\_flux\_bc switch from the UM\sphinxhyphen{}coupling argument list and standalone code. (\#775)

\item {} 
\sphinxAtStartPar
Improved use of coupled model diagnostic code in standalone runs. (\#740)

\item {} 
\sphinxAtStartPar
Diagnostics from INFERNO (interactive fire model) made available to the UM. (\#552)

\item {} 
\sphinxAtStartPar
JULES parameters included in the Random Parameter (RP) scheme (for UM runs). (\#675)

\item {} 
\sphinxAtStartPar
Rationalised some of the code for the reading and writing of dumps. (\#763)

\item {} 
\sphinxAtStartPar
Alignment of JULES and UM urban control and initialisation code. (\#319)

\item {} 
\sphinxAtStartPar
Reduced model overhead when running DrHook profiling. (\#782)

\item {} 
\sphinxAtStartPar
Fixed oddities found while investigating the use of CamFort. (\#769)

\item {} 
\sphinxAtStartPar
JULES source code fully compliant with the Fortran 2003 standard. (\#711)

\item {} 
\sphinxAtStartPar
Corrections to code comments and other minor changes. (\#725, 690)

\item {} 
\sphinxAtStartPar
Clarified units of variables in the vegetation code. (\#741)

\item {} 
\sphinxAtStartPar
Enable the reading of PFT and soil parameters for CABLE runs via the JULES\_PFTPARM\_CABLE and JULES\_NVEGPARM\_CABLE namelists. (\#694, 748)

\item {} 
\sphinxAtStartPar
Minor edits to ensure future compatibility of .inc files with the umdp3\_fixer used in Rose stem tests. (\#762)

\end{itemize}


\subsection{Changes to testing}
\label{\detokenize{release_notes/JULES5-2:changes-to-testing}}\begin{itemize}
\item {} 
\sphinxAtStartPar
Rose stem testing working on JASMIN. (\#744)

\item {} 
\sphinxAtStartPar
Improved the output of the umdp3 checker task in rose stem. (\#764)

\item {} 
\sphinxAtStartPar
Rose stem testing added for IMOGEN and GL7 and GL8 configurations. (\#706, 648, 773)

\item {} 
\sphinxAtStartPar
Added an OMP vs no\sphinxhyphen{}OMP Rose stem test for the ukv config to the MO XC40 and virtual machine platforms. (\#732)

\item {} 
\sphinxAtStartPar
Allowed the option of setting JULES\_REMOTE and JULES\_REMOTE\_HOST when running on the Met Office Cray (meto\sphinxhyphen{}xc40\sphinxhyphen{}cce). (\#755)

\item {} 
\sphinxAtStartPar
Resolved oversubscription problems and rationalised the meto\sphinxhyphen{}linux rose stem. (\#783)

\end{itemize}


\subsection{Bugs fixed}
\label{\detokenize{release_notes/JULES5-2:bugs-fixed}}\begin{itemize}
\item {} 
\sphinxAtStartPar
Fix to ensure TRIFFID competition does not try to access non\sphinxhyphen{}existent surface types. (\#647)

\item {} 
\sphinxAtStartPar
Fixed array/scalar mismatch in arguments to vegcarb. (\#682)

\item {} 
\sphinxAtStartPar
Corrected the dimensions given to the frac\_prev array in \sphinxtitleref{lotka\_eq\_jls.F90} (for runs with {\hyperref[\detokenize{namelists/jules_vegetation.nml:JULES_VEGETATION::l_trif_eq}]{\sphinxcrossref{\sphinxcode{\sphinxupquote{l\_trif\_eq}}}}} = T and {\hyperref[\detokenize{namelists/jules_vegetation.nml:JULES_VEGETATION::l_ht_compete}]{\sphinxcrossref{\sphinxcode{\sphinxupquote{l\_ht\_compete}}}}} = T). (\#765)

\item {} 
\sphinxAtStartPar
Fix to prevent floating point errors with CABLE. (\#694)

\item {} 
\sphinxAtStartPar
Use NINT to guard against imprecision in REAL/INTEGER conversion in routing code. (\#726)

\item {} 
\sphinxAtStartPar
Fixed bugs relating to windspeed\sphinxhyphen{}dependent unloading of snow from vegetation (UM only) and allowing soil rate modifier diagnostics in standalone runs. (\#740)

\item {} 
\sphinxAtStartPar
Correction related to indexing of snow fields in reconfiguration (UM only). (\#676)

\item {} 
\sphinxAtStartPar
Fixed certain snow diagnostics (UM stash fields 8,578 to 8,583). (\#720)

\item {} 
\sphinxAtStartPar
Correction to logic for canopy parameter updating (UM only). (\#746)

\item {} 
\sphinxAtStartPar
Example namelists updated for vn5.1. (\#722)

\item {} 
\sphinxAtStartPar
Fix of host specification in \sphinxtitleref{runtime.rc} for site cehwl1. (\#731)

\item {} 
\sphinxAtStartPar
Updated the configuration for University of Exeter (\sphinxtitleref{uoe\sphinxhyphen{}linux\sphinxhyphen{}gfortran.cfg}). (\#735)

\item {} 
\sphinxAtStartPar
Fix so rose stem IMOGEN tests work at NCI. (\#792)

\end{itemize}


\subsection{Documentation updates}
\label{\detokenize{release_notes/JULES5-2:documentation-updates}}\begin{itemize}
\item {} 
\sphinxAtStartPar
Removed the science configurations section from the documentation. (\#736).

\item {} 
\sphinxAtStartPar
Updated the documentation (mainly release notes and hydrological terminology). (\#738, 745)

\item {} 
\sphinxAtStartPar
Updated documentation of fcm\sphinxhyphen{}make JULES\_PLATFORM environment variable. (\#739)

\item {} 
\sphinxAtStartPar
Updated hyperlinks to Rose and FCM in the documentation. (\#786)

\end{itemize}

\sphinxAtStartPar
Documentation can be viewed on the github page \sphinxurl{http://jules-lsm.github.io/}.

\sphinxstepscope


\section{JULES version 5.1 Release Notes}
\label{\detokenize{release_notes/JULES5-1:jules-version-5-1-release-notes}}\label{\detokenize{release_notes/JULES5-1::doc}}
\sphinxAtStartPar
The JULES vn5.1 release consists of approximately 40 tickets from 17 authors, including work by many other people.

\sphinxAtStartPar
Full details of the tickets committed for JULES vn5.1 can be found on the \sphinxhref{https://code.metoffice.gov.uk/trac/jules/query?resolution=fixed\&milestone=JULES+v5.1+release}{JULES shared repository Trac system}.

\sphinxAtStartPar
Ticket numbers are indicated below, e.g. \#533.


\subsection{Science changes}
\label{\detokenize{release_notes/JULES5-1:science-changes}}\begin{itemize}
\item {} 
\sphinxAtStartPar
Addition of river overbank inundation module (for diagnostic calculation of overbank inundation fraction) \sphinxhyphen{} see l\_riv\_overbank in namelist {\hyperref[\detokenize{namelists/jules_rivers.nml:namelist-JULES_RIVERS}]{\sphinxcrossref{\sphinxcode{\sphinxupquote{JULES\_RIVERS}}}}} and namelist {\hyperref[\detokenize{namelists/jules_rivers.nml:namelist-JULES_OVERBANK}]{\sphinxcrossref{\sphinxcode{\sphinxupquote{JULES\_OVERBANK}}}}}. (\#679)

\item {} 
\sphinxAtStartPar
Changes to the layered soil biogeochemistry model: (i) the mean soil temperature over the TRIFFID time step is now used to determine whether a layer is unfrozen (ii) mixing is considered when calculating the final respiration. (\#663)

\item {} 
\sphinxAtStartPar
Options for improved treatment of thin snow comprising the introduction of basal melting of thin snow layers on warm ground. See {\hyperref[\detokenize{namelists/jules_snow.nml:JULES_SNOW::i_basal_melting_opt}]{\sphinxcrossref{\sphinxcode{\sphinxupquote{i\_basal\_melting\_opt}}}}}. (\#533)

\item {} 
\sphinxAtStartPar
Added capability to include latest harvest date to crop ancillary. See {\hyperref[\detokenize{namelists/ancillaries.nml:list-of-spatially-varying-crop-properties}]{\sphinxcrossref{\DUrole{std,std-ref}{List of spatially\sphinxhyphen{}varying crop properties}}}}. (\#653)

\item {} 
\sphinxAtStartPar
Account for crop harvest and fire in the carbon conservation diagnostics. (\#476)

\item {} 
\sphinxAtStartPar
Carbon conservation diagnostic now only calculated on TRIFFID timesteps. (\#643)

\item {} 
\sphinxAtStartPar
Added the JULES\sphinxhyphen{}standalone version of improved saturation vapour pressure (qsat) calculations. (\#635)

\item {} 
\sphinxAtStartPar
Introduced basic slab ocean by allowing the model to update the sea surface temperature based on the energy balance, in the same way as it does for land and sea\sphinxhyphen{}ice. Also allows for a fixed sea surface albedo (see \sphinxtitleref{fixed\_sea\_albedo}), rather than the parameterised options. (\#642)

\end{itemize}


\subsection{General/Technical changes}
\label{\detokenize{release_notes/JULES5-1:general-technical-changes}}\begin{itemize}
\item {} 
\sphinxAtStartPar
Wetland CH$_{\text{4}}$ emission parameters have been added to {\hyperref[\detokenize{namelists/jules_soil_biogeochem.nml:namelist-JULES_SOIL_BIOGEOCHEM}]{\sphinxcrossref{\sphinxcode{\sphinxupquote{JULES\_SOIL\_BIOGEOCHEM}}}}}. Parameters are {\hyperref[\detokenize{namelists/jules_soil_biogeochem.nml:JULES_SOIL_BIOGEOCHEM::t0_ch4}]{\sphinxcrossref{\sphinxcode{\sphinxupquote{t0\_ch4}}}}}, {\hyperref[\detokenize{namelists/jules_soil_biogeochem.nml:JULES_SOIL_BIOGEOCHEM::const_ch4_cs}]{\sphinxcrossref{\sphinxcode{\sphinxupquote{const\_ch4\_cs}}}}}, {\hyperref[\detokenize{namelists/jules_soil_biogeochem.nml:JULES_SOIL_BIOGEOCHEM::const_ch4_npp}]{\sphinxcrossref{\sphinxcode{\sphinxupquote{const\_ch4\_npp}}}}}, {\hyperref[\detokenize{namelists/jules_soil_biogeochem.nml:JULES_SOIL_BIOGEOCHEM::const_ch4_resps}]{\sphinxcrossref{\sphinxcode{\sphinxupquote{const\_ch4\_resps}}}}}, {\hyperref[\detokenize{namelists/jules_soil_biogeochem.nml:JULES_SOIL_BIOGEOCHEM::q10_ch4_cs}]{\sphinxcrossref{\sphinxcode{\sphinxupquote{q10\_ch4\_cs}}}}}, {\hyperref[\detokenize{namelists/jules_soil_biogeochem.nml:JULES_SOIL_BIOGEOCHEM::q10_ch4_npp}]{\sphinxcrossref{\sphinxcode{\sphinxupquote{q10\_ch4\_npp}}}}}, and {\hyperref[\detokenize{namelists/jules_soil_biogeochem.nml:JULES_SOIL_BIOGEOCHEM::q10_ch4_resps}]{\sphinxcrossref{\sphinxcode{\sphinxupquote{q10\_ch4\_resps}}}}}. (\#483)

\item {} 
\sphinxAtStartPar
i/o work to allow soil tiling to function correctly. (\#684)

\item {} 
\sphinxAtStartPar
Allow output of rate modifier diagnostics needed for offline spin up of soil carbon pools. (\#589)

\item {} 
\sphinxAtStartPar
Improvements to intermittent sampling for output. (\#605)

\item {} 
\sphinxAtStartPar
Removed repeated calculation of vegetation stocks of C and N from TRIFFID. (\#618)

\item {} 
\sphinxAtStartPar
Modularised JULES vegetation subroutines to enforce stricter interface checking. (\#668)

\item {} 
\sphinxAtStartPar
Replaced integer constants used with {\hyperref[\detokenize{namelists/jules_soil_biogeochem.nml:JULES_SOIL_BIOGEOCHEM::ch4_substrate}]{\sphinxcrossref{\sphinxcode{\sphinxupquote{ch4\_substrate}}}}} (the choice of substrate for wetland methane) with parameters. (\#628)

\item {} 
\sphinxAtStartPar
More time constants replaced by parameters. (\#625)

\item {} 
\sphinxAtStartPar
Namelist {\hyperref[\detokenize{namelists/model_environment.nml:JULES_MODEL_ENVIRONMENT::lsm_id}]{\sphinxcrossref{\sphinxcode{\sphinxupquote{lsm\_id}}}}} added to allow the selection of the land surface model: JULES or CABLE. Note that the CABLE science routines have not been implemented yet. (\#656)

\item {} 
\sphinxAtStartPar
Redundant “include” files removed and rationalisation of other modules. (\#689)

\item {} 
\sphinxAtStartPar
Made the JULES source code (as close as possible to) compliant with the Fortran 2003 standard. (\#699)

\item {} 
\sphinxAtStartPar
Added OpenMP to the snow scheme to improve parallel performance. (\#638)

\item {} 
\sphinxAtStartPar
IMOGEN identified as a new science module, and other changes to list of module leaders. (\#686, 666)

\item {} 
\sphinxAtStartPar
Added basic MORUSES rose stem test (upgraded UKV science to PS39) and enhanced metadata. (\#289)

\item {} 
\sphinxAtStartPar
Added a new script to rose stem to ensure apps are consistent with rose metadata. (\#645)

\item {} 
\sphinxAtStartPar
Expanded the fcm\sphinxhyphen{}make metadata to make it easier to apply additional compiler flags to builds. (\#632)

\item {} 
\sphinxAtStartPar
Implemented a timeout for the JULES rose\sphinxhyphen{}stem in instances where an app/task “submit\sphinxhyphen{}fail” or stalls. (\#672)

\item {} 
\sphinxAtStartPar
Updated settings for JULES on Met Office XC40 (standalone suites should be updated where necessary to pick up the new settings) and Met Office linux testing to use ifort 16.0. (\#630, 697).

\end{itemize}


\subsection{Bugs fixed}
\label{\detokenize{release_notes/JULES5-1:bugs-fixed}}\begin{itemize}
\item {} 
\sphinxAtStartPar
Fixed bug in soil hydrology to prevent water erroneously coming out of the bottom of the soil. Use {\hyperref[\detokenize{namelists/jules_soil.nml:JULES_SOIL::l_holdwater}]{\sphinxcrossref{\sphinxcode{\sphinxupquote{l\_holdwater}}}}} = T to allow water to be held on an impermeable layer. (\#171)

\item {} 
\sphinxAtStartPar
Bug fix affecting litterfall N flux when {\hyperref[\detokenize{namelists/jules_vegetation.nml:JULES_VEGETATION::l_landuse}]{\sphinxcrossref{\sphinxcode{\sphinxupquote{l\_landuse}}}}} = T. (\#617)

\item {} 
\sphinxAtStartPar
Bug\sphinxhyphen{}fix in the calculation of the albedo when {\hyperref[\detokenize{namelists/jules_radiation.nml:JULES_RADIATION::l_embedded_snow}]{\sphinxcrossref{\sphinxcode{\sphinxupquote{l\_embedded\_snow}}}}} = T. (\#533)

\item {} 
\sphinxAtStartPar
Corrected a scalar/array mismatch in the arguments to the \sphinxtitleref{plant\_growth\_n} subroutine. (\#670)

\item {} 
\sphinxAtStartPar
Fix to allow initialisation of CO$_{\text{2}}$ from a dump file. (\#680)

\item {} 
\sphinxAtStartPar
Minor fix for vegetation competition in runs with {\hyperref[\detokenize{namelists/jules_vegetation.nml:JULES_VEGETATION::l_ht_compete}]{\sphinxcrossref{\sphinxcode{\sphinxupquote{l\_ht\_compete}}}}} = F and {\hyperref[\detokenize{namelists/jules_vegetation.nml:JULES_VEGETATION::l_trif_crop}]{\sphinxcrossref{\sphinxcode{\sphinxupquote{l\_trif\_crop}}}}} = F. (\#627)

\item {} 
\sphinxAtStartPar
Corrected units for parameters \sphinxtitleref{dfp\_dcuo} and \sphinxtitleref{eta\_sl} in comments and documentation. (\#646, 659)

\item {} 
\sphinxAtStartPar
Minor correction to metadata for {\hyperref[\detokenize{namelists/ancillaries.nml:namelist-JULES_SOIL_PROPS}]{\sphinxcrossref{\sphinxcode{\sphinxupquote{JULES\_SOIL\_PROPS}}}}} namelist. (\#669)

\item {} 
\sphinxAtStartPar
Fixed an allocation bug in the UM (\#665).

\item {} 
\sphinxAtStartPar
Bug in tiling that could affect lake quantities (in UM runs using FLake). (\#641)

\item {} 
\sphinxAtStartPar
Fix of a minor OpenMP race condition. (\#674)

\item {} 
\sphinxAtStartPar
Moved an ALLOCATE statement to fix a memory bug in the UM affecting ESM runs. (\#639)

\end{itemize}


\subsection{Documentation updates}
\label{\detokenize{release_notes/JULES5-1:documentation-updates}}
\sphinxAtStartPar
Documentation can be viewed on the github page \sphinxurl{http://jules-lsm.github.io/}.

\sphinxstepscope


\section{JULES version 5.0 Release Notes}
\label{\detokenize{release_notes/JULES5-0:jules-version-5-0-release-notes}}\label{\detokenize{release_notes/JULES5-0::doc}}
\sphinxAtStartPar
The JULES vn5.0 release consists of approximately 34 tickets from 12 authors, including work by many other people.

\sphinxAtStartPar
Full details of the tickets committed for JULES vn5.0 can be found on the \sphinxhref{https://code.metoffice.gov.uk/trac/jules/query?resolution=fixed\&milestone=JULES+v5.0+release}{JULES shared repository Trac system}.

\sphinxAtStartPar
This release was unusual in that it was closed to science tickets so as to allow work of a more technical nature, principally ensuring compliance with the coding standards and improved optimisation.

\sphinxAtStartPar
The extensive styling changes that were introduced at vn5.0 have made minor changes to a large number of lines of code, meaning that updating your work from an older branch may not be as straightforward as with most release cycles. For advice on how best to upgrade your branch to vn5.0 see \sphinxhref{https://code.metoffice.gov.uk/trac/jules/wiki/MoveBranchesToVn5.0}{Notes on moving to vn5.0}.

\sphinxAtStartPar
Ticket numbers are indicated below, e.g. \#230.


\subsection{General/Technical changes}
\label{\detokenize{release_notes/JULES5-0:general-technical-changes}}\begin{itemize}
\item {} 
\sphinxAtStartPar
Applied a code style\sphinxhyphen{}checking script, also included as part of rose stem tests. (\#230, 387, 593)

\item {} 
\sphinxAtStartPar
Modularise the root\_frac() subroutine to improve interface checking. (\#571)

\item {} 
\sphinxAtStartPar
More use of OpenMP threading to surface and soil code to improve performance. (\#575, 600, 607)

\item {} 
\sphinxAtStartPar
Various code changes to prevent or warn if an inconsistent combination of vegetation flags and parameters is used. When TRIFFID is on and the phenology model is off (not recommended), LAI is now set to the balanced LAI rather than taken from the ({\hyperref[\detokenize{namelists/pft_params.nml:namelist-JULES_PFTPARM}]{\sphinxcrossref{\sphinxcode{\sphinxupquote{JULES\_PFTPARM}}}}}) namelist. (\#592)

\item {} 
\sphinxAtStartPar
Remove references to outdated or deprecated functions (\#577, 585), and other technical improvements. (\#596, 616)

\item {} 
\sphinxAtStartPar
More variables given initial values. (\#579)

\item {} 
\sphinxAtStartPar
Reduced size of outputs from rose stem tests. (\#580)

\item {} 
\sphinxAtStartPar
Superfluous messages from vegetation code removed. (\#581)

\item {} 
\sphinxAtStartPar
General code improvements to meet coding standards and use a consistent style. (\#574, 576, 578, 584, 594, 620, 621)

\item {} 
\sphinxAtStartPar
Namelists are printed before checking for errors, for easier debugging. (UM only, \#582)

\item {} 
\sphinxAtStartPar
Snowdepth diagnostic made available in the UM. (\#569)

\item {} 
\sphinxAtStartPar
Module leadership clarified across the code base. (\#611)

\item {} 
\sphinxAtStartPar
Unnecessary subversion properties removed. (\#597)

\end{itemize}


\subsection{Bugs fixed}
\label{\detokenize{release_notes/JULES5-0:bugs-fixed}}\begin{itemize}
\item {} 
\sphinxAtStartPar
Fixed minor bug in variable names when soil moisture prescribed. (\#573)

\item {} 
\sphinxAtStartPar
Prevent duplicate names for output streams. (\#590)

\item {} 
\sphinxAtStartPar
Fixed bug in vegetation competition routine lotka\_eq. (\#603)

\item {} 
\sphinxAtStartPar
Fixed bug in output variable smc\_avail\_top. (\#604)

\item {} 
\sphinxAtStartPar
Fixed minor bugs in use of l\_trait\_phys. (\#613)

\item {} 
\sphinxAtStartPar
Corrected wood product pool diagnostics in the UM. (\#587)

\item {} 
\sphinxAtStartPar
Bugs fixed in soil respiration, and in carbon conservation diagnostics, for runs with N limitation. (\#595)

\item {} 
\sphinxAtStartPar
Enables ability to perform bit comparable NRUN in the UM. (\#612)

\item {} 
\sphinxAtStartPar
Fix JULES rose stem so that it works on the MO XCS\sphinxhyphen{}C system (aka MONSooN). (\#615)

\item {} 
\sphinxAtStartPar
Fixed broken hyperlinks in the user guide. (\#636)

\end{itemize}


\subsection{Documentation updates}
\label{\detokenize{release_notes/JULES5-0:documentation-updates}}
\sphinxAtStartPar
Documentation can be viewed on the github page \sphinxurl{http://jules-lsm.github.io/}.

\sphinxstepscope


\section{JULES version 4.9 Release Notes}
\label{\detokenize{release_notes/JULES4-9:jules-version-4-9-release-notes}}\label{\detokenize{release_notes/JULES4-9::doc}}
\sphinxAtStartPar
The JULES vn4.9 release consists of approximately 60 tickets from 19 authors, including work by many other people.

\sphinxAtStartPar
Full details of the tickets committed for JULES vn4.9 can be found on the \sphinxhref{https://code.metoffice.gov.uk/trac/jules/query?resolution=fixed\&milestone=JULES+v4.9+release}{JULES shared repository Trac system}.

\sphinxAtStartPar
Ticket numbers are indicated below, e.g. \#262.


\subsection{Science changes}
\label{\detokenize{release_notes/JULES4-9:science-changes}}\begin{itemize}
\item {} 
\sphinxAtStartPar
New parameter {\hyperref[\detokenize{namelists/jules_hydrology.nml:JULES_HYDROLOGY::s_pdm}]{\sphinxcrossref{\sphinxcode{\sphinxupquote{s\_pdm}}}}} represents the minimum soil wetness below which there is no saturation excess surface runoff from the PDM model. s\_pdm can be made slope\sphinxhyphen{}dependent using {\hyperref[\detokenize{namelists/jules_hydrology.nml:JULES_HYDROLOGY::l_spdmvar}]{\sphinxcrossref{\sphinxcode{\sphinxupquote{l\_spdmvar}}}}}, with additional parameter {\hyperref[\detokenize{namelists/jules_hydrology.nml:JULES_HYDROLOGY::slope_pdm_max}]{\sphinxcrossref{\sphinxcode{\sphinxupquote{slope\_pdm\_max}}}}}. (\#262)

\item {} 
\sphinxAtStartPar
Extensions to the parameterisation of moisture stress on vegetation. There is now (a) the option of a stress function that is piece\sphinxhyphen{}wise linear in soil potential ({\hyperref[\detokenize{namelists/jules_vegetation.nml:JULES_VEGETATION::fsmc_shape}]{\sphinxcrossref{\sphinxcode{\sphinxupquote{fsmc\_shape}}}}} = 1) rather than soil volumetric moisture content ({\hyperref[\detokenize{namelists/jules_vegetation.nml:JULES_VEGETATION::fsmc_shape}]{\sphinxcrossref{\sphinxcode{\sphinxupquote{fsmc\_shape}}}}} = 0) and (b) the option of specifying pft\sphinxhyphen{}dependent soil potentials ({\hyperref[\detokenize{namelists/jules_vegetation.nml:JULES_VEGETATION::l_use_pft_psi}]{\sphinxcrossref{\sphinxcode{\sphinxupquote{l\_use\_pft\_psi}}}}}) at which the plant starts to experience water stress ({\hyperref[\detokenize{namelists/pft_params.nml:JULES_PFTPARM::psi_open_io}]{\sphinxcrossref{\sphinxcode{\sphinxupquote{psi\_open\_io}}}}}) and where the plant is fully stressed ({\hyperref[\detokenize{namelists/pft_params.nml:JULES_PFTPARM::psi_close_io}]{\sphinxcrossref{\sphinxcode{\sphinxupquote{psi\_close\_io}}}}}). (\#541)

\item {} 
\sphinxAtStartPar
Methane emissions can now be calculated from layered soil temperature (see {\hyperref[\detokenize{namelists/jules_soil_biogeochem.nml:JULES_SOIL_BIOGEOCHEM::l_ch4_tlayered}]{\sphinxcrossref{\sphinxcode{\sphinxupquote{l\_ch4\_tlayered}}}}}). Methane flux can be removed from soil carbon stocks to close the carbon budget (see {\hyperref[\detokenize{namelists/jules_soil_biogeochem.nml:JULES_SOIL_BIOGEOCHEM::l_ch4_interactive}]{\sphinxcrossref{\sphinxcode{\sphinxupquote{l\_ch4\_interactive}}}}}). The choice of substrate used for methane production is now controlled using {\hyperref[\detokenize{namelists/jules_soil_biogeochem.nml:JULES_SOIL_BIOGEOCHEM::ch4_substrate}]{\sphinxcrossref{\sphinxcode{\sphinxupquote{ch4\_substrate}}}}} instead of l\_wetland\_ch4\_npp. (\#468)

\item {} 
\sphinxAtStartPar
Updates to the INFERNO fire model (see {\hyperref[\detokenize{namelists/jules_vegetation.nml:JULES_VEGETATION::l_inferno}]{\sphinxcrossref{\sphinxcode{\sphinxupquote{l\_inferno}}}}}) including burning of the carbon in litter/soil pools. New diagnostics added including flammability (see {\hyperref[\detokenize{output-variables:output-variables-section}]{\sphinxcrossref{\DUrole{std,std-ref}{JULES Output variables}}}}). (\#502)

\item {} 
\sphinxAtStartPar
Extension to the scheme for calculating subgrid\sphinxhyphen{}scale snowpack properties and surface mass balance fields for icesheet coupling (see {\hyperref[\detokenize{namelists/jules_surface.nml:JULES_SURFACE::l_elev_land_ice}]{\sphinxcrossref{\sphinxcode{\sphinxupquote{l\_elev\_land\_ice}}}}}). Non\sphinxhyphen{}glaciated tiles ({\hyperref[\detokenize{namelists/jules_surface_types.nml:JULES_SURFACE_TYPES::elev_rock}]{\sphinxcrossref{\sphinxcode{\sphinxupquote{elev\_rock}}}}}) can now co\sphinxhyphen{}exist with the ice tiles ({\hyperref[\detokenize{namelists/jules_surface_types.nml:JULES_SURFACE_TYPES::elev_ice}]{\sphinxcrossref{\sphinxcode{\sphinxupquote{elev\_ice}}}}}) in the elevation classes, allowing for fractional ice extent in a gridbox. (\#294)

\item {} 
\sphinxAtStartPar
Allow soil moisture to be prescribed on a subset of soil levels (see {\hyperref[\detokenize{namelists/prescribed_data.nml:JULES_PRESCRIBED_DATASET::prescribed_levels}]{\sphinxcrossref{\sphinxcode{\sphinxupquote{prescribed\_levels}}}}}). (\#487)

\item {} 
\sphinxAtStartPar
Improvements to the layered soil C and N model (see {\hyperref[\detokenize{namelists/jules_soil_biogeochem.nml:JULES_SOIL_BIOGEOCHEM::l_layeredc}]{\sphinxcrossref{\sphinxcode{\sphinxupquote{l\_layeredc}}}}}) which is now fully ready for use. (\#454, \#526)

\item {} 
\sphinxAtStartPar
Layered C allowed with single\sphinxhyphen{}pool soil model. (\#526)

\item {} 
\sphinxAtStartPar
Preparatory work for the ECOSSE soil biogeochemical model. (\#444, \#518)

\item {} 
\sphinxAtStartPar
New functionality to limit the drag over the sea at high wind speeds (only for coupled models). (\#543)

\end{itemize}


\subsection{General/Technical changes}
\label{\detokenize{release_notes/JULES4-9:general-technical-changes}}\begin{itemize}
\item {} 
\sphinxAtStartPar
Further preparatory work for soil tiling, including i/o.

\item {} 
\sphinxAtStartPar
Allocated arrays are given initial values and improved error messages from memory allocation.

\item {} 
\sphinxAtStartPar
Further C fluxes and diagnostics made available to the UM.

\item {} 
\sphinxAtStartPar
{\hyperref[\detokenize{namelists/pft_params.nml:JULES_PFTPARM::fsmc_p0_io}]{\sphinxcrossref{\sphinxcode{\sphinxupquote{fsmc\_p0\_io}}}}} added to UM namelists.

\item {} 
\sphinxAtStartPar
Various UM\sphinxhyphen{}related improvements, including: single switch to allow runs with mixing ratio; avoiding hard\sphinxhyphen{}wired logical argument to swap\_bounds; removed old logical related to “New dynamics”; removed the stash super array; removed unused array bounds.

\item {} 
\sphinxAtStartPar
Wrapped non\sphinxhyphen{}LFRIC\sphinxhyphen{}compliant code using a preprocessor directive.

\item {} 
\sphinxAtStartPar
General code developments and improvements to meet the coding standards.

\item {} 
\sphinxAtStartPar
Added fcm make and rose stem configurations for building and testing JULES at NCI.

\item {} 
\sphinxAtStartPar
Added Rose stem tests for TRIP rivers running with 2D data and RFM rivers. (\#539, \#567, \#568)

\item {} 
\sphinxAtStartPar
Add more information to trac.log by improving suite\_report.py.

\item {} 
\sphinxAtStartPar
rose\sphinxhyphen{}stem supported for MONSooN xcs\sphinxhyphen{}c.

\item {} 
\sphinxAtStartPar
Various changes for Met Office, including nightly testing on xcs.

\item {} 
\sphinxAtStartPar
New tutorial group for rose\sphinxhyphen{}stem.

\item {} 
\sphinxAtStartPar
Change to suite\sphinxhyphen{}report.py for use with cylc 7.

\item {} 
\sphinxAtStartPar
Updates to the app update script.

\end{itemize}


\subsection{Bugs fixed}
\label{\detokenize{release_notes/JULES4-9:bugs-fixed}}\begin{itemize}
\item {} 
\sphinxAtStartPar
Corrected mismatches between certain subroutine interfaces and subroutine calls. (\#471, \#544)

\item {} 
\sphinxAtStartPar
Fixed bug in reading of certain soil tile variables. (\#498)

\item {} 
\sphinxAtStartPar
Fixed bugs in TRIFFID litter fluxes. (\#504)

\item {} 
\sphinxAtStartPar
TRIFFID altered to do fewer calculations on points without veg or soil. (\#509)

\item {} 
\sphinxAtStartPar
Tidied up code issues between UM and JULES. (\#522)

\item {} 
\sphinxAtStartPar
Added interactive nitrogen to the loobos\_jules\_es\_1p6 test and initialised more TRIFFID variables. (\#512, \#520)

\item {} 
\sphinxAtStartPar
Bug fix for regridding between land and river routing grids for regular lat/lon river grids where the land model grid is defined as 1d land point only grids. (\#524)

\item {} 
\sphinxAtStartPar
Fixed uninitialised variables in surf\_couple\_extra\_mod.F90. (\#546)

\item {} 
\sphinxAtStartPar
Bug fix in init\_rivers\_props.inc. (\#539)

\item {} 
\sphinxAtStartPar
Bug fix for crop ancillary variables in read\_dump and write\_dump. (\#551)

\item {} 
\sphinxAtStartPar
Corrected a bug in the sea ice albedo scheme, which affects the calculation of bare ice albedo when multilayer thermodynamics are used in coupled UM\sphinxhyphen{}JULES\sphinxhyphen{}NEMO\sphinxhyphen{}CICE. (\#547)

\item {} 
\sphinxAtStartPar
Bug fix to variable intent in surf\_couple\_implicit. (\#542)

\item {} 
\sphinxAtStartPar
Bug fix for UM runs relating to clash between STASH items (\#557)

\end{itemize}


\subsection{Documentation updates}
\label{\detokenize{release_notes/JULES4-9:documentation-updates}}
\sphinxAtStartPar
Coding standards, and documentation can be viewed on the github page \sphinxurl{http://jules-lsm.github.io/}.

\sphinxstepscope


\section{JULES version 4.8 Release Notes}
\label{\detokenize{release_notes/JULES4-8:jules-version-4-8-release-notes}}\label{\detokenize{release_notes/JULES4-8::doc}}
\sphinxAtStartPar
The JULES vn4.8 release consists of approximately 77 tickets from 26 authors, including work by many other people.

\sphinxAtStartPar
Full details of the tickets committed for JULES vn4.8 can be found on the \sphinxhref{https://code.metoffice.gov.uk/trac/jules/query?resolution=fixed\&milestone=JULES+v4.8+release}{JULES shared repository Trac system}.

\sphinxAtStartPar
Ticket numbers are indicated below, e.g. \#400.


\subsection{Science changes}
\label{\detokenize{release_notes/JULES4-8:science-changes}}\begin{itemize}
\item {} 
\sphinxAtStartPar
Changes to calculate transpiration on tiles and as a gridbox mean. A resistance factor based on stomatal resistance excluding soil is calculated and then multiplied by evapotranspiration. This affects the diagnostics ‘et\_stom’ and ‘et\_stom\_gb’. (\#400)

\item {} 
\sphinxAtStartPar
Addition of interactive fire/vegetation, selected using {\hyperref[\detokenize{namelists/jules_vegetation.nml:JULES_VEGETATION::l_trif_fire}]{\sphinxcrossref{\sphinxcode{\sphinxupquote{l\_trif\_fire}}}}}. Fire disturbance modifies vegetation dynamics and can be modelled by the INFERNO fire model or prescribed via an ancillary file. This is a preliminary version that will be developed further.  (\#456)

\item {} 
\sphinxAtStartPar
Added option to allow downward longwave and net shortwave radiation to be used to force the model (see {\hyperref[\detokenize{namelists/drive.nml:list-of-jules-forcing-variables}]{\sphinxcrossref{\DUrole{std,std-ref}{List of JULES forcing variables}}}}). JULES now stops if too many or incorrect radiation variablse are provided (rather than carrying on if a valid combination was found). (\#409)

\item {} 
\sphinxAtStartPar
Added an option for the treatment of graupel in input to snow scheme (primarily for the UM; see {\hyperref[\detokenize{namelists/jules_snow.nml:JULES_SNOW::graupel_options}]{\sphinxcrossref{\sphinxcode{\sphinxupquote{graupel\_options}}}}}). (\#414)

\item {} 
\sphinxAtStartPar
Looping required for soil tiling functionality added in a disabled state (nsoilt still hard\sphinxhyphen{}coded to 1). (\#379)

\item {} 
\sphinxAtStartPar
Added the ability for hydrol\_jls.F90 to work with tiled runoff once soil tiling is enabled in a later change. (\#341)

\end{itemize}


\subsection{General/Technical changes}
\label{\detokenize{release_notes/JULES4-8:general-technical-changes}}\begin{itemize}
\item {} 
\sphinxAtStartPar
Refactoring of albpft\_jls.F90 to be more efficient.

\item {} 
\sphinxAtStartPar
Improved code to calculate litter, landuse change and harvest fluxes.

\item {} 
\sphinxAtStartPar
New namelist {\hyperref[\detokenize{namelists/jules_soil_biogeochem.nml:namelist-JULES_SOIL_BIOGEOCHEM}]{\sphinxcrossref{\sphinxcode{\sphinxupquote{JULES\_SOIL\_BIOGEOCHEM}}}}}. The soil and vegetation components of TRIFFID have been separated.

\item {} 
\sphinxAtStartPar
{\hyperref[\detokenize{namelists/initial_conditions.nml:JULES_INITIAL::total_snow}]{\sphinxcrossref{\sphinxcode{\sphinxupquote{total\_snow}}}}} defaults to .FALSE.

\item {} 
\sphinxAtStartPar
{\hyperref[\detokenize{namelists/output.nml:JULES_OUTPUT::dump_period}]{\sphinxcrossref{\sphinxcode{\sphinxupquote{dump\_period}}}}} controls the frequency with which dumps are written.

\item {} 
\sphinxAtStartPar
River routining changes: user initialised surface and sub\sphinxhyphen{}surface storage and flows for RFM river routing. See {\hyperref[\detokenize{namelists/initial_conditions.nml:list-of-initial-condition-variables}]{\sphinxcrossref{\DUrole{std,std-ref}{List of initial condition variables}}}}. Two options for standalone river routing: ‘rfm’ or ‘trip’ in namelist {\hyperref[\detokenize{namelists/jules_rivers.nml:namelist-JULES_RIVERS}]{\sphinxcrossref{\sphinxcode{\sphinxupquote{JULES\_RIVERS}}}}}. New remapping utilities provide efficient translation between land points and river points vectors.

\item {} 
\sphinxAtStartPar
New and added diagnostics for UKESM, CMIP6 ‘NDVI\_land’. See {\hyperref[\detokenize{output-variables:output-variables-section}]{\sphinxcrossref{\DUrole{std,std-ref}{JULES Output variables}}}}.

\item {} 
\sphinxAtStartPar
Improved error checking and error messages.

\item {} 
\sphinxAtStartPar
Print statement improvements.

\item {} 
\sphinxAtStartPar
Urban namelists prefixed with ‘jules’.

\item {} 
\sphinxAtStartPar
Tidy up unused variables and an unused dummy argument in the FLake code.

\item {} 
\sphinxAtStartPar
General code developments and improvements to meet the coding standards.

\item {} 
\sphinxAtStartPar
A minor change to communication routines’ API.

\item {} 
\sphinxAtStartPar
OpenMP developments.

\item {} 
\sphinxAtStartPar
Rationalisation of UM ancil routines, part 1 (\#346)

\item {} 
\sphinxAtStartPar
Removed redundant code from the soil ancillary reading code. (\#450)

\item {} 
\sphinxAtStartPar
Rose stem configurations for JASMIN, NIWA, CEH, Exeter Uni, remote JASMIN and MetO XCS computer. Ensure that the same tests are run at all sites. cylc7 compatable

\item {} 
\sphinxAtStartPar
New or improved rose stem tests.

\item {} 
\sphinxAtStartPar
Improvements to create\sphinxhyphen{}rose\sphinxhyphen{}app and suite\_report.py

\end{itemize}


\subsection{Bugs fixed}
\label{\detokenize{release_notes/JULES4-8:bugs-fixed}}\begin{itemize}
\item {} 
\sphinxAtStartPar
Bug in the argument list to irrig\_dmd, where the surface\sphinxhyphen{}tiled irrigated fraction was passed instead of the gridbox mean. (\#389)

\item {} 
\sphinxAtStartPar
Various fixes related to argument INTENTs (snow \#396; soil respiration \#392, h\_blend\_orog \#452, surface flux code \#412).

\item {} 
\sphinxAtStartPar
Bug fixes for INFERNO fire model (soil carbon \#415, incorrect rainfall \#390, minor fixes \#371).

\item {} 
\sphinxAtStartPar
Bug fix in TRIFFID so soil nitrogen with l\_trif\_crop does not depend on processor configuration (\#372)

\item {} 
\sphinxAtStartPar
Bug fix for soil inorganic N in UM runs (\#395)

\item {} 
\sphinxAtStartPar
Bug fix in hydrology for when {\hyperref[\detokenize{namelists/jules_hydrology.nml:JULES_HYDROLOGY::l_wetland_unfrozen}]{\sphinxcrossref{\sphinxcode{\sphinxupquote{l\_wetland\_unfrozen}}}}} = .TRUE. (\#473)

\item {} 
\sphinxAtStartPar
Initialisation of TRIFFID diagnostics (\#313, \#474, \#479)

\item {} 
\sphinxAtStartPar
Minor update to correct standalone river routing grid definition for non\sphinxhyphen{}regular grids (e.g. UKV variable resolution). (\#410)

\item {} 
\sphinxAtStartPar
Bug fix in the calculation of NPP in TRIFFID with N limitation. (\#308)

\item {} 
\sphinxAtStartPar
Fixed error in frequency of calls to phenology in long standalone runs. (\#421)

\item {} 
\sphinxAtStartPar
Bugs fixed so IMOGEN will restart more accurately from a dump. (\#420)

\item {} 
\sphinxAtStartPar
Bug fixes and improvements for layered soil CN model (\#394, \#407)

\item {} 
\sphinxAtStartPar
Bug fix for albpft\_jls.F90 (\#458)

\item {} 
\sphinxAtStartPar
Bug fix to allow compilation without netCDF (\#464)

\item {} 
\sphinxAtStartPar
Bug fixes for metadata and upgrade macro (\#404, \#459, \#490)

\end{itemize}


\subsection{Documentation updates}
\label{\detokenize{release_notes/JULES4-8:documentation-updates}}
\sphinxAtStartPar
Coding standards, and documentation can be viewed on the github page \sphinxurl{http://jules-lsm.github.io/}.

\sphinxstepscope


\section{JULES version 4.7 Release Notes}
\label{\detokenize{release_notes/JULES4-7:jules-version-4-7-release-notes}}\label{\detokenize{release_notes/JULES4-7::doc}}
\sphinxAtStartPar
The JULES vn4.7 release consists of 47 tickets from 19 authors.

\sphinxAtStartPar
Full details of the tickets committed for JULES vn4.7 can be found on the \sphinxhref{https://code.metoffice.gov.uk/trac/jules/query?resolution=fixed\&milestone=JULES+v4.7+release}{JULES shared repository Trac system}.

\sphinxAtStartPar
Ticket numbers are indicated below, e.g. \#265.


\subsection{Science changes}
\label{\detokenize{release_notes/JULES4-7:science-changes}}\begin{itemize}
\item {} 
\sphinxAtStartPar
Enable soil tiling by the extraction of key calculations. These include: Infiltration rate and Soil moisture availability factor (beta). (\#265)

\item {} 
\sphinxAtStartPar
Modifications to the rate of growth of snow grains \sphinxhyphen{} it uses the ET scheme of Taillandier (2007), JGR, 112, F03003 for the rate of growth and to relayer the grain size using its inverse as this is more consistent with conservation of SSA reported by Gallet et al. (2011). (\#298)

\item {} 
\sphinxAtStartPar
JULES\sphinxhyphen{}CN: Soil\_CN ratio changed from hard\sphinxhyphen{}wired to the namelist in prep for a PPE of JULES and JULES\sphinxhyphen{}CN the bio and hum N pools change from being prognostic to diagnostic via the soil bgc at the start of this routine and during initialisation. (\#309, 288)

\item {} 
\sphinxAtStartPar
Add new irr\_crop option: irr\_crop = 0: continuous irrigation (i.e. the effectively the crop season is defined to last all year). It does not depend on crop characteristics (unlike irr\_crop=2, which uses the crop model, or irr\_crop=1, which estimates a typical crop season for that gridbox). (\#312)

\item {} 
\sphinxAtStartPar
Diagnostics for individual components of snowpack mass balance, as specified by ISMIP6. (\#314)

\item {} 
\sphinxAtStartPar
Diagnostics for components of surface radiation on land tiles. Requested by ISMIP6 for driving standalone icesheet models. (\#315)

\item {} 
\sphinxAtStartPar
User initialised river storage. When dump\_file=T, and use\_file=T for rivers\_sto\_rp, then rivers\_sto\_rp needs to be in the dumpfile. When use\_file=F for rivers\_sto\_rp, then rivers\_sto\_rp can be set to a constant value in this namelist. When dump\_file=F, the rivers\_sto\_rp is initialised to zero. (extra log message to say that rivers\_sto\_rp is initialised to zero in this case). Therefore a dump file from a run without l\_rivers=T and rivers\_type=’trip’ can now be used to initialise a run with l\_rivers=T and rivers\_type=’trip’. (\#316 and \#329 doc)

\item {} 
\sphinxAtStartPar
For fsmc\_mod = 1, change the water extraction pattern so that it is proportional to plant available water rather than total water in the soil layers. Note: fsmc\_mod=1 is not the recommended value in the JULES manual and it is not currently use in any of the documented configurations. This part of the code is also going through a detailed review as part of the soil moisture stress on vegetation group, which includes documenting the current status. (\#320)

\item {} 
\sphinxAtStartPar
Add a new cpft\sphinxhyphen{}dependent input variable initial\_c\_dvi\_io to specify when the crop should be initialised. (\#324, 356 doc)

\item {} 
\sphinxAtStartPar
Allow perturbations to driving data, by specify an amount to be added to driving temperatures and/or an amount to multiply the driving precip variables by. This is achieved by adding a switch (l\_perturb\_driving) and two input variables (temperature\_abs\_perturbation and precip\_rel\_perturbation) to the JULES\_DRIVE namelist. When l\_perturb\_driving is set to true, an amount (positive or negative) can be added to the driving temperature (temperature\_abs\_perturbation) and the precipitation variables can be multiplied by a factor (precip\_rel\_perturbation). (\#326)

\item {} 
\sphinxAtStartPar
Fluxes JULES rose stem tests added: Some new GSWP2 tests for forecast configurations, New diagnostics to many existing tests, Switches on profiles that were written but disabled in carbon cycle tests, Tweaks to how we run on SPICE. (\#330)

\item {} 
\sphinxAtStartPar
Change the defaults for the npft=9, ncpft=4 case in the rose upgrade macro vn44\_t136 so that first 5 pfts are the same as the npft=5 case and the C3/C4 crop tiles have the same values as the C3/C4 grasses. (\#336)

\item {} 
\sphinxAtStartPar
Allow sthuf (soil moisture) to be prescribed. (\#348)

\item {} 
\sphinxAtStartPar
Soil refactor, following on from adding module statements etc to radiation and snow, this works on src/science/soil to add MOUDLE statements and INTENTS. (\#351)

\item {} 
\sphinxAtStartPar
Update to metadata for jules\_rivers\_props so when grid\_is\_1d is false the following values in jules\_rivers\_props; nx\_grid, ny\_grid, reg\_lat1, reg\_lon1, reg\_dlat, reg\_dlon are required when running in parallel. Added a log message warning that the values are not used if grid\_is\_1d is false and the run is in parallel. (\#293)

\item {} 
\sphinxAtStartPar
Add the soil tile dimension to JULES as a hard\sphinxhyphen{}coded singleton. (\#305)

\item {} 
\sphinxAtStartPar
Fixed a bug in TRIFFID which causes loss of bit comparison that occurred when L\_TRIF\_CROP and L\_NITROGEN were both TRUE: the soil nitrogen prognostics (stash 442, 443, 446) became dependent on the PE configuration. (\#372)

\item {} 
\sphinxAtStartPar
Modularise and header refactor science/radiation Adds module and intents to the radiation code Removed implicit RESHAPES of some variables through subroutine calls and therefore removes nearly all the complicated dimensionalities in this area (ij, pfield, land\_pts), making the code simpler. (\#253)

\item {} 
\sphinxAtStartPar
Add vertically discretised soil C and N to TRIFFID. Adds a dimension to existing prognostics and discretizing existing code. The scheme is extended to link rooting profiles to availability and uptake of N requiring additional prognostics. (\#288)

\end{itemize}


\subsection{General/Technical changes}
\label{\detokenize{release_notes/JULES4-7:general-technical-changes}}\begin{itemize}
\item {} 
\sphinxAtStartPar
Improve coding standard docs.

\item {} 
\sphinxAtStartPar
Move from includes to use modules for ccarbon.h Retire l\_endgame as we only use endgame in the UM.

\item {} 
\sphinxAtStartPar
Fix warnings in the fcm make log.

\item {} 
\sphinxAtStartPar
OpenMP improvements.

\item {} 
\sphinxAtStartPar
Add valgrind profiling as execution option in the Rose GUI.

\item {} 
\sphinxAtStartPar
Extra documentation on; namelist order, Update to release notes.

\item {} 
\sphinxAtStartPar
Nightly rose stem test added to the MO system to test the head of the trunk nightly.

\item {} 
\sphinxAtStartPar
Replace all ENDIF, ENDDO, IF(, MAX(, MIN(, EXP( and GAMMA/gamma to r\_gamma with the working practices syntax.

\item {} 
\sphinxAtStartPar
Use “rose config dump” on the whole repository to tidy up the rose (Python) files.

\item {} 
\sphinxAtStartPar
Update cylc5 syntax to cylc6 (Python).

\item {} 
\sphinxAtStartPar
Improve the create\_rose app script. It now takes 5 arguments, vn\_from, vn\_to, namelist\_path, suite\_name and jules\_dir.

\end{itemize}


\subsection{Bugs fixed}
\label{\detokenize{release_notes/JULES4-7:bugs-fixed}}\begin{itemize}
\item {} 
\sphinxAtStartPar
Fixed the variable in the metstats\_timesteps subroutine that was being incorrectly set for first and last second of day, which lead to gaps in the fire indices along some lines of longitude. (\#323)

\item {} 
\sphinxAtStartPar
Fixed errors in UM GA7 AMIP code to run rose stem app with “rigorous” optimisation settings on SPICE. In particular: Formatted internal writes with incorrect format statements; ALLOCATEd variables not being DEALLOCATEd. (\#328)

\item {} 
\sphinxAtStartPar
Fixed example put namelists. Updated parameters can\_struct\_a\_io, fsmc\_mod\_io, and fsmc\_p0\_io in all example directories, and added missing parameters in the loobos\_point\_9pfts directory. (\#354)

\item {} 
\sphinxAtStartPar
Clay frac bug, IF test for l\_triffid=.false. added around clay\_gb so that the test does not fail as clay\_gb has not been populated as triffid is not run and therefore not clay\_gb is not initialised and populated. (\#307)

\item {} 
\sphinxAtStartPar
taux (wind stress) output inconsistency fixed by the initialisation of two variables in surf\_couple\_extra: cq\_cm\_u\_1(:,:) = 0.0 and cq\_cm\_v\_1(:,:) = 0.0 (\#339)

\item {} 
\sphinxAtStartPar
Set z0\_surft and lit\_c\_pft to zero in allocate\_jules\_arrays as they were not allocated and caused KGO failures. (\#365)

\item {} 
\sphinxAtStartPar
Fixed calc\_fsat so that values are not divded by small numbers. (\#342)

\item {} 
\sphinxAtStartPar
Fixed the bug in multi\sphinxhyphen{}layer snow use of tile\_map. It now checks to see if it is valid for the input dump configuration and converts from mapping tile IDs to pseudo levels. (\#302)

\end{itemize}


\subsection{Documentation updates}
\label{\detokenize{release_notes/JULES4-7:documentation-updates}}
\sphinxAtStartPar
Coding standards, and documentation can be viewed on the ‘github page \textless{}\sphinxurl{http://jules-lsm.github.io/}\textgreater{}\_’.

\sphinxstepscope


\section{JULES version 4.6 Release Notes}
\label{\detokenize{release_notes/JULES4-6:jules-version-4-6-release-notes}}\label{\detokenize{release_notes/JULES4-6::doc}}
\sphinxAtStartPar
The JULES vn4.6 release consists of 43 tickets from 22 authors across 52 commits.

\sphinxAtStartPar
Full details of the tickets committed for JULES vn4.6 can be found on the \sphinxhref{https://code.metoffice.gov.uk/trac/jules/query?resolution=fixed\&milestone=JULES+v4.6+release}{JULES shared repository Trac system}.


\subsection{Science changes}
\label{\detokenize{release_notes/JULES4-6:science-changes}}\begin{itemize}
\item {} 
\sphinxAtStartPar
Multiple ice tiles in a gridbox to simulate snowpacks at different elevations \sphinxhyphen{} see {\hyperref[\detokenize{namelists/jules_surface.nml:JULES_SURFACE::l_elev_land_ice}]{\sphinxcrossref{\sphinxcode{\sphinxupquote{l\_elev\_land\_ice}}}}}

\item {} 
\sphinxAtStartPar
Modifications to snowpack physics to better represent deep, compact firn/snow on ice sheets \sphinxhyphen{} description at {\hyperref[\detokenize{namelists/jules_surface.nml:JULES_SURFACE::l_elev_land_ice}]{\sphinxcrossref{\sphinxcode{\sphinxupquote{l\_elev\_land\_ice}}}}}

\item {} 
\sphinxAtStartPar
Option to link whole\sphinxhyphen{}plant maintenance respiration to soil moisture stress \sphinxhyphen{} see {\hyperref[\detokenize{namelists/jules_vegetation.nml:JULES_VEGETATION::l_scale_resp_pm}]{\sphinxcrossref{\sphinxcode{\sphinxupquote{l\_scale\_resp\_pm}}}}}

\item {} 
\sphinxAtStartPar
Read clay content of soil for soil carbon decomposition model from ancillary file \sphinxhyphen{} see {\hyperref[\detokenize{namelists/ancillaries.nml:namelist-JULES_SOIL_PROPS}]{\sphinxcrossref{\sphinxcode{\sphinxupquote{JULES\_SOIL\_PROPS}}}}}

\item {} 
\sphinxAtStartPar
Improved climate downscaling physics for ice elevation tiles

\item {} 
\sphinxAtStartPar
Calculate FAO Penman\sphinxhyphen{}Monteith evapotranspiration for reference crop \sphinxhyphen{} see diagnostic fao\_et0

\item {} 
\sphinxAtStartPar
Diagnostic form drag for sea ice (UM only)

\item {} 
\sphinxAtStartPar
Calculation of new sea ice variables required for CMIP6

\item {} 
\sphinxAtStartPar
Implement a canopy clumping factor \sphinxhyphen{} see {\hyperref[\detokenize{namelists/pft_params.nml:JULES_PFTPARM::can_struct_a_io}]{\sphinxcrossref{\sphinxcode{\sphinxupquote{can\_struct\_a\_io}}}}}

\item {} 
\sphinxAtStartPar
Allow for non\sphinxhyphen{}istropic scattering in plant canopies \sphinxhyphen{} see {\hyperref[\detokenize{namelists/jules_radiation.nml:JULES_RADIATION::l_niso_direct}]{\sphinxcrossref{\sphinxcode{\sphinxupquote{l\_niso\_direct}}}}}

\item {} 
\sphinxAtStartPar
Increased flexibility to represent soil moisture stress on vegetation \sphinxhyphen{} see {\hyperref[\detokenize{namelists/pft_params.nml:JULES_PFTPARM::fsmc_mod_io}]{\sphinxcrossref{\sphinxcode{\sphinxupquote{fsmc\_mod\_io}}}}} and {\hyperref[\detokenize{namelists/pft_params.nml:JULES_PFTPARM::fsmc_p0_io}]{\sphinxcrossref{\sphinxcode{\sphinxupquote{fsmc\_p0\_io}}}}}

\item {} 
\sphinxAtStartPar
Improved parameterisation of crop leaf senescence

\item {} 
\sphinxAtStartPar
New crop harvest diagnostics

\item {} 
\sphinxAtStartPar
Lake model FLake beneath multi\sphinxhyphen{}layer snow (UM only)

\end{itemize}


\subsection{Technical changes}
\label{\detokenize{release_notes/JULES4-6:technical-changes}}\begin{itemize}
\item {} 
\sphinxAtStartPar
JULES\sphinxhyphen{}C\sphinxhyphen{}1p1 Regression Tests

\item {} 
\sphinxAtStartPar
Remove the UM\_FLAKE CPP macro

\item {} 
\sphinxAtStartPar
Move sorp and n\_inorg\_turnover to namelist to enable user input.

\item {} 
\sphinxAtStartPar
VM rose stem bug fixed.

\item {} 
\sphinxAtStartPar
Add support for rose\sphinxhyphen{}stem on MONSooN

\item {} 
\sphinxAtStartPar
Move some of the hard\sphinxhyphen{}wired crop parameters to {\hyperref[\detokenize{namelists/crop_params.nml:namelist-JULES_CROPPARM}]{\sphinxcrossref{\sphinxcode{\sphinxupquote{JULES\_CROPPARM}}}}}

\item {} 
\sphinxAtStartPar
Remove UM descent.h include file and put values into JULES module descent.F90

\item {} 
\sphinxAtStartPar
Fix race conditions and improve OpenMP DEFAULTs

\item {} 
\sphinxAtStartPar
Modularise and header refactor science/snow

\end{itemize}


\subsection{Bugs fixed}
\label{\detokenize{release_notes/JULES4-6:bugs-fixed}}\begin{itemize}
\item {} 
\sphinxAtStartPar
Fixed GC3 tstar\_sice bug

\item {} 
\sphinxAtStartPar
Corrected canopy nitrogen profiles \sphinxhyphen{} see {\hyperref[\detokenize{namelists/jules_vegetation.nml:JULES_VEGETATION::l_leaf_n_resp_fix}]{\sphinxcrossref{\sphinxcode{\sphinxupquote{l\_leaf\_n\_resp\_fix}}}}}. This increases plant maintenance respiration.

\item {} 
\sphinxAtStartPar
Reduced stem respiration with trait\sphinxhyphen{}based physiology

\item {} 
\sphinxAtStartPar
Reinstated missing veg parms, for trait initialisation.

\item {} 
\sphinxAtStartPar
Soil respiration bug resolved

\item {} 
\sphinxAtStartPar
Corrected mistake in merging of crop PFT changes

\item {} 
\sphinxAtStartPar
Fixed N diagnostic to avoid runtime crashes

\item {} 
\sphinxAtStartPar
Fixed calculation of dust deposition exchange coefficient

\item {} 
\sphinxAtStartPar
Fixed wetland emission of methane with TRIFFID on

\item {} 
\sphinxAtStartPar
Fixed aerodynamic resistance diagnostic (UM diagnostic)

\end{itemize}


\subsection{Documentation updates}
\label{\detokenize{release_notes/JULES4-6:documentation-updates}}
\sphinxAtStartPar
Coding standards, and documentation can be viewed on the ‘github page \textless{}\sphinxurl{http://jules-lsm.github.io/}\textgreater{}\_’.

\sphinxstepscope


\section{JULES version 4.5 Release Notes}
\label{\detokenize{release_notes/JULES4-5:jules-version-4-5-release-notes}}\label{\detokenize{release_notes/JULES4-5::doc}}
\sphinxAtStartPar
The JULES vn4.5 release consists of 31 tickets from 19 authors across 35 commits.

\sphinxAtStartPar
Full details of the tickets committed for JULES vn4.5 can be found on the \sphinxhref{https://code.metoffice.gov.uk/trac/jules/query?resolution=fixed\&milestone=JULES+v4.5+release}{JULES shared repository Trac system}.


\subsection{Science changes}
\label{\detokenize{release_notes/JULES4-5:science-changes}}\begin{itemize}
\item {} 
\sphinxAtStartPar
UKCA dry deposition working with 13 surface tiles

\item {} 
\sphinxAtStartPar
Check litter flux carbon balance

\item {} 
\sphinxAtStartPar
Allow litter carbon fluxes from variable numbers of PFTs

\item {} 
\sphinxAtStartPar
Improved seasonal cycle of soil respiration: switch l\_soil\_resp\_lev2 alters how soil temperature and moisture are used for respiration calculation.

\item {} 
\sphinxAtStartPar
Added parameters to trait physiology for nitrogen in wood and roots \sphinxhyphen{} see {\hyperref[\detokenize{namelists/pft_params.nml:JULES_PFTPARM::nr_io}]{\sphinxcrossref{\sphinxcode{\sphinxupquote{nr\_io}}}}}, {\hyperref[\detokenize{namelists/pft_params.nml:JULES_PFTPARM::nsw_io}]{\sphinxcrossref{\sphinxcode{\sphinxupquote{nsw\_io}}}}} and {\hyperref[\detokenize{namelists/pft_params.nml:JULES_PFTPARM::hw_sw_io}]{\sphinxcrossref{\sphinxcode{\sphinxupquote{hw\_sw\_io}}}}}.

\item {} 
\sphinxAtStartPar
INFERNO model of fire emissions and burnt area \sphinxhyphen{} see {\hyperref[\detokenize{namelists/jules_vegetation.nml:JULES_VEGETATION::l_inferno}]{\sphinxcrossref{\sphinxcode{\sphinxupquote{l\_inferno}}}}}

\item {} 
\sphinxAtStartPar
Option to represent crops using triffid \sphinxhyphen{} see {\hyperref[\detokenize{namelists/jules_vegetation.nml:JULES_VEGETATION::l_trif_crop}]{\sphinxcrossref{\sphinxcode{\sphinxupquote{l\_trif\_crop}}}}}

\item {} 
\sphinxAtStartPar
Remove MORUSES hard\sphinxhyphen{}wired roof coupling

\item {} 
\sphinxAtStartPar
Add diagnostic for canopy FAPAR (Fraction of Absorbed Photosynthetically Active Radiation).

\item {} 
\sphinxAtStartPar
JULES\sphinxhyphen{}CN: enabled nitrogen limitation of NPP \sphinxhyphen{} see {\hyperref[\detokenize{namelists/jules_vegetation.nml:JULES_VEGETATION::l_nitrogen}]{\sphinxcrossref{\sphinxcode{\sphinxupquote{l\_nitrogen}}}}}

\item {} 
\sphinxAtStartPar
Added the FLake lake model into JULES (for UM use)

\end{itemize}


\subsection{Technical changes}
\label{\detokenize{release_notes/JULES4-5:technical-changes}}\begin{itemize}
\item {} 
\sphinxAtStartPar
Variable renaming to support soil tiling

\item {} 
\sphinxAtStartPar
Add JASMIN as a supported system for rose\sphinxhyphen{}stem jobs

\item {} 
\sphinxAtStartPar
Fix Cray compiler warnings

\item {} 
\sphinxAtStartPar
Protect print statements from the TRIP river routing code with PrintStatus

\end{itemize}


\subsection{Bugs fixed}
\label{\detokenize{release_notes/JULES4-5:bugs-fixed}}\begin{itemize}
\item {} 
\sphinxAtStartPar
Fix ozone diagnostics in JULES

\end{itemize}


\subsection{Documentation updates}
\label{\detokenize{release_notes/JULES4-5:documentation-updates}}
\sphinxAtStartPar
Coding standards, and documentation can be viewed on the ‘github page \textless{}\sphinxurl{http://jules-lsm.github.io/}\textgreater{}\_’.
\begin{itemize}
\item {} 
\sphinxAtStartPar
Update to coding standards to reflect and protect variable name changes

\item {} 
\sphinxAtStartPar
Represent crops using TRIFFID

\item {} 
\sphinxAtStartPar
JULES\sphinxhyphen{}CN

\item {} 
\sphinxAtStartPar
Nitrogen trait physiology

\item {} 
\sphinxAtStartPar
Check litter C flux carbon balance

\item {} 
\sphinxAtStartPar
kgC and kgN in netCDF units metadata

\item {} 
\sphinxAtStartPar
adding nfita to hydrology namelist

\item {} 
\sphinxAtStartPar
add FAPAR diagnostic

\end{itemize}

\sphinxstepscope


\section{JULES version 4.4 Release Notes}
\label{\detokenize{release_notes/JULES4-4:jules-version-4-4-release-notes}}\label{\detokenize{release_notes/JULES4-4::doc}}
\sphinxAtStartPar
The JULES vn4.4 release consists of 31 tickets from 17 authors across 35 commits. Two further tickets were removed from the release due to issues.

\sphinxAtStartPar
Full details of the tickets committed for JULES vn4.4 can be found on the \sphinxhref{https://code.metoffice.gov.uk/trac/jules/query?resolution=fixed\&milestone=JULES+v4.4+release}{JULES shared repository Trac system}.


\subsection{Science changes}
\label{\detokenize{release_notes/JULES4-4:science-changes}}\begin{itemize}
\item {} 
\sphinxAtStartPar
Add an option to set tile elevations to have absolute values above sea\sphinxhyphen{}level \sphinxhyphen{} see {\hyperref[\detokenize{namelists/model_grid.nml:JULES_SURF_HGT::l_elev_absolute_height}]{\sphinxcrossref{\sphinxcode{\sphinxupquote{l\_elev\_absolute\_height}}}}}

\item {} 
\sphinxAtStartPar
Adjustment to downward longwave radiation for elevated tiles \sphinxhyphen{} see {\hyperref[\detokenize{namelists/jules_surface.nml:JULES_SURFACE::l_elev_lw_down}]{\sphinxcrossref{\sphinxcode{\sphinxupquote{l\_elev\_lw\_down}}}}}

\item {} 
\sphinxAtStartPar
Nitrogen cycling \sphinxhyphen{} improved process representation for UKESM

\item {} 
\sphinxAtStartPar
Use bare soil momentum roughness length, if supplied as an ancillary field

\item {} 
\sphinxAtStartPar
Irrigation water taken first from deep soil layer then the river (code from the JULES Impact Model)

\item {} 
\sphinxAtStartPar
Improvements to rivers\_type=’trip’,’rfm’: storage in dump and partial parallelisation

\item {} 
\sphinxAtStartPar
Alternative forms for methane emissions from wetlands, using different substrates

\item {} 
\sphinxAtStartPar
Allow landuse with variable number of PFTs

\item {} 
\sphinxAtStartPar
BVOC emissions allowed with trait physiology

\item {} 
\sphinxAtStartPar
Change multi\sphinxhyphen{}layer snow indices of first layer

\end{itemize}


\subsection{Technical changes}
\label{\detokenize{release_notes/JULES4-4:technical-changes}}\begin{itemize}
\item {} 
\sphinxAtStartPar
JULES fcm\sphinxhyphen{}make configs updated to be incompatible with FCM 2015.07.0 and later

\item {} 
\sphinxAtStartPar
Rose\sphinxhyphen{}stem support for alternative Rose, Cylc and FCM versions

\item {} 
\sphinxAtStartPar
Addition of OpenMP directives to some JULES code.

\item {} 
\sphinxAtStartPar
Code coverage metrics for rose\sphinxhyphen{}stem suite

\item {} 
\sphinxAtStartPar
Preparation for JULES memory duplication

\item {} 
\sphinxAtStartPar
Removed snow\_grnd\_gb from closures benchmark test

\end{itemize}


\subsection{Bugs fixed}
\label{\detokenize{release_notes/JULES4-4:bugs-fixed}}\begin{itemize}
\item {} 
\sphinxAtStartPar
Remove factor of 0.5 from snow albedo temperature dependence

\item {} 
\sphinxAtStartPar
Fix a bug in the Nitrogen scheme for the implicit update of soil respiration when the soil C pool is exhausted.

\item {} 
\sphinxAtStartPar
Fix drift in vegetation fractions

\item {} 
\sphinxAtStartPar
Scaling bug in Taux

\item {} 
\sphinxAtStartPar
Bug fix for irrig\_water diagnostic \sphinxhyphen{} between crop seasons this is now zero

\item {} 
\sphinxAtStartPar
Met Office VM: fix for update\sphinxhyphen{}jules\sphinxhyphen{}scripts file for JULES 4.3

\item {} 
\sphinxAtStartPar
JULES namelist broadcasts are out of sync

\end{itemize}

\sphinxstepscope


\section{JULES version 4.3 Release Notes}
\label{\detokenize{release_notes/JULES4-3:jules-version-4-3-release-notes}}\label{\detokenize{release_notes/JULES4-3::doc}}
\sphinxAtStartPar
The JULES vn4.3 release consists of 36 tickets from 18 authors across 39 commits.

\sphinxAtStartPar
Full details of the tickets committed for JULES vn4.3 can be found on the \sphinxhref{https://code.metoffice.gov.uk/trac/jules/query?resolution=fixed\&milestone=JULES+v4.3+release}{JULES shared repository Trac system}.


\subsection{Science changes}
\label{\detokenize{release_notes/JULES4-3:science-changes}}\begin{itemize}
\item {} 
\sphinxAtStartPar
Enhancements to the multi\sphinxhyphen{}layer snow scheme for GL7.0 (Global Land configuration, version 7.0)
\begin{itemize}
\item {} 
\sphinxAtStartPar
Addition of ET metamorphism

\item {} 
\sphinxAtStartPar
Infiltration of rain water into the snow pack

\item {} 
\sphinxAtStartPar
Albedo of snow and relationship to plant canopies

\end{itemize}

\item {} 
\sphinxAtStartPar
Generalisation of the crop scheme to work with trait\sphinxhyphen{}based plant physiology and BVOC emissions

\item {} 
\sphinxAtStartPar
Update to wetland scheme (see {\hyperref[\detokenize{namelists/jules_hydrology.nml:JULES_HYDROLOGY::l_wetland_unfrozen}]{\sphinxcrossref{\sphinxcode{\sphinxupquote{l\_wetland\_unfrozen}}}}})

\item {} 
\sphinxAtStartPar
River routing updates to allow RFM with standalone JULES to be run with non\sphinxhyphen{}regular lat\sphinxhyphen{}lon grids

\item {} 
\sphinxAtStartPar
New JULES\sphinxhyphen{}C configuration, the prototype configuration for UKESM1

\item {} 
\sphinxAtStartPar
Sea\sphinxhyphen{}ice changes for GC3.0 (Global Coupled configuration, version 3.0)

\end{itemize}


\subsection{Technical changes}
\label{\detokenize{release_notes/JULES4-3:technical-changes}}\begin{itemize}
\item {} 
\sphinxAtStartPar
Revamp of compilation procedure (see {\hyperref[\detokenize{building-and-running/intro::doc}]{\sphinxcrossref{\DUrole{doc}{Building and running JULES}}}})
\begin{itemize}
\item {} 
\sphinxAtStartPar
Changes to the environment variables used to specify a build

\item {} 
\sphinxAtStartPar
Option to extract and mirror on local machine, preprocess and build on a remote machine (e.g. Met Office Cray XC40)

\item {} 
\sphinxAtStartPar
Addition of “platform configurations”, to reduce the number of environment variable definitions required to build on a known platform

\end{itemize}

\item {} 
\sphinxAtStartPar
Ancillary data (e.g. fractional coverage, soil data) is now saved to the dump file
\begin{itemize}
\item {} 
\sphinxAtStartPar
Each namelist in {\hyperref[\detokenize{namelists/ancillaries.nml::doc}]{\sphinxcrossref{\DUrole{doc}{ancillaries.nml}}}} gets a new flag, \sphinxcode{\sphinxupquote{read\_from\_dump}}, e.g. {\hyperref[\detokenize{namelists/ancillaries.nml:JULES_SOIL_PROPS::read_from_dump}]{\sphinxcrossref{\sphinxcode{\sphinxupquote{JULES\_SOIL\_PROPS::read\_from\_dump}}}}}

\item {} 
\sphinxAtStartPar
A dump file can now be used initialise an entire run, including ancillaries (except for river routing, for technical reasons)

\end{itemize}

\item {} 
\sphinxAtStartPar
Many additional \sphinxcode{\sphinxupquote{rose\sphinxhyphen{}stem}} tests

\item {} 
\sphinxAtStartPar
Replace testing for ice using \sphinxcode{\sphinxupquote{sm\_sat}} with logical arrays for soil and ice points

\item {} 
\sphinxAtStartPar
Restructuring of \sphinxcode{\sphinxupquote{rose\sphinxhyphen{}stem}} tests to allow for site configurations with more divergence between sites
\begin{itemize}
\item {} 
\sphinxAtStartPar
As a result, JULES is now routinely tested on 3 platforms \sphinxhyphen{} Intel and gfortran compilers on Linux and CCE on the Cray

\end{itemize}

\item {} 
\sphinxAtStartPar
Remove the hijacking of the ice tile as a second urban tile when using two\sphinxhyphen{}tile urban schemes in the UM

\item {} 
\sphinxAtStartPar
Replacement of old include files with modules

\end{itemize}


\subsection{Bugs fixed}
\label{\detokenize{release_notes/JULES4-3:bugs-fixed}}\begin{itemize}
\item {} 
\sphinxAtStartPar
Fixed a long\sphinxhyphen{}standing off\sphinxhyphen{}by\sphinxhyphen{}one error in the instantaneous interpolation code (mode \sphinxcode{\sphinxupquote{i}} \sphinxhyphen{} see {\hyperref[\detokenize{input/temporal-interpolation::doc}]{\sphinxcrossref{\DUrole{doc}{Temporal interpolation}}}})

\item {} 
\sphinxAtStartPar
Several small fixes to soil carbon and vegetation code

\item {} 
\sphinxAtStartPar
Fix in river routing for bit\sphinxhyphen{}comparison with different processor decompositions in the UM

\item {} 
\sphinxAtStartPar
Fix for IMOGEN in parallel mode

\item {} 
\sphinxAtStartPar
Fix initialisation of some diagnostics

\end{itemize}

\sphinxstepscope


\section{JULES version 4.2 Release Notes}
\label{\detokenize{release_notes/JULES4-2:jules-version-4-2-release-notes}}\label{\detokenize{release_notes/JULES4-2::doc}}
\sphinxAtStartPar
JULES version 4.2 is the first release where all development has taken place in the Met Office collaboration repository. On the whole, this has been a success, with module leaders beginning to use the Trac system to track and approve developments in their areas.

\sphinxAtStartPar
The JULES vn4.2 release consists of 35 tickets from 13 authors across 39 commits.


\subsection{Science changes}
\label{\detokenize{release_notes/JULES4-2:science-changes}}\begin{itemize}
\item {} 
\sphinxAtStartPar
TRIP and RFM river routing (see {\hyperref[\detokenize{namelists/jules_rivers.nml::doc}]{\sphinxcrossref{\DUrole{doc}{jules\_rivers.nml}}}})

\item {} 
\sphinxAtStartPar
Incorporation of widely used fire risk indices (see {\hyperref[\detokenize{namelists/fire.nml::doc}]{\sphinxcrossref{\DUrole{doc}{fire.nml}}}})

\item {} 
\sphinxAtStartPar
New soil thermal conductivity model that is more appropriate for organic soils (see {\hyperref[\detokenize{namelists/jules_soil.nml:JULES_SOIL::soilhc_method}]{\sphinxcrossref{\sphinxcode{\sphinxupquote{soilhc\_method}}}}})

\item {} 
\sphinxAtStartPar
Addition of “bedrock” column beneath the soil column, in which only thermal diffusion occurs (see {\hyperref[\detokenize{namelists/jules_soil.nml:JULES_SOIL::l_bedrock}]{\sphinxcrossref{\sphinxcode{\sphinxupquote{l\_bedrock}}}}})

\item {} 
\sphinxAtStartPar
New canopy radiation scheme in which the nitrogen follows an exponential decay (see {\hyperref[\detokenize{namelists/jules_vegetation.nml:JULES_VEGETATION::can_rad_mod}]{\sphinxcrossref{\sphinxcode{\sphinxupquote{can\_rad\_mod}}}}})

\item {} 
\sphinxAtStartPar
Updates to trait PFTs (see {\hyperref[\detokenize{namelists/jules_vegetation.nml:JULES_VEGETATION::l_trait_phys}]{\sphinxcrossref{\sphinxcode{\sphinxupquote{l\_trait\_phys}}}}}) to ensure that the dynamic and equilibrium solutions for vegetation fraction are equivalent

\item {} 
\sphinxAtStartPar
Crop model now conserves carbon

\end{itemize}


\subsection{Technical changes}
\label{\detokenize{release_notes/JULES4-2:technical-changes}}\begin{itemize}
\item {} 
\sphinxAtStartPar
Added flag to force the model grid to be 1D (see {\hyperref[\detokenize{namelists/model_grid.nml:JULES_MODEL_GRID::force_1d_grid}]{\sphinxcrossref{\sphinxcode{\sphinxupquote{force\_1d\_grid}}}}})

\item {} 
\sphinxAtStartPar
Several new diagnostics added (see {\hyperref[\detokenize{output-variables::doc}]{\sphinxcrossref{\DUrole{doc}{JULES Output variables}}}})

\item {} 
\sphinxAtStartPar
New rose\sphinxhyphen{}stem regression tests added

\item {} 
\sphinxAtStartPar
All rose\sphinxhyphen{}stem tests migrated to use NetCDF and \sphinxcode{\sphinxupquote{nccmp}}

\item {} 
\sphinxAtStartPar
Retirement of several include files

\item {} 
\sphinxAtStartPar
Removal of logging namelist \sphinxhyphen{} output location should now be controlled using pipes (i.e. \sphinxcode{\sphinxupquote{1\textgreater{}/my/file 2\textgreater{}\&1}}) or features of your \sphinxcode{\sphinxupquote{mpiexec}} or \sphinxcode{\sphinxupquote{mpirun}} program

\item {} 
\sphinxAtStartPar
Additional consolidation of shared (i.e. standalone and UM) control and initialisation routines

\item {} 
\sphinxAtStartPar
Optimisation of \sphinxcode{\sphinxupquote{sf\_stom}} and \sphinxcode{\sphinxupquote{leaf\_limits}}, resulting in \textasciitilde{}20\% speedup for {\hyperref[\detokenize{namelists/jules_vegetation.nml:JULES_VEGETATION::can_rad_mod}]{\sphinxcrossref{\sphinxcode{\sphinxupquote{can\_rad\_mod}}}}} = 4

\end{itemize}


\subsection{Bugs fixed}
\label{\detokenize{release_notes/JULES4-2:bugs-fixed}}\begin{itemize}
\item {} 
\sphinxAtStartPar
Bug in calculation of \sphinxcode{\sphinxupquote{n\_leaf}} and \sphinxcode{\sphinxupquote{n\_stem}} in \sphinxcode{\sphinxupquote{sf\_stom}}

\item {} 
\sphinxAtStartPar
Memory overwriting bug in TRIFFID

\item {} 
\sphinxAtStartPar
Differing error message lengths in UM

\item {} 
\sphinxAtStartPar
\sphinxcode{\sphinxupquote{hcons}} now passed to the snow scheme instead of \sphinxcode{\sphinxupquote{hcon(:,0)}}

\item {} 
\sphinxAtStartPar
Several bug fixes for IMOGEN

\item {} 
\sphinxAtStartPar
Patch to enable collective access for parallel NetCDF in NetCDF 4.2 onwards

\item {} 
\sphinxAtStartPar
\sphinxcode{\sphinxupquote{soil\_hyd}} now declares \sphinxcode{\sphinxupquote{ksz}} correctly (i.e. \sphinxcode{\sphinxupquote{ksz(npnts,0:nshyd)}} instead of \sphinxcode{\sphinxupquote{ksz(npnts,nshyd)}}) \sphinxhyphen{} this only affects runs where the soil properties vary for each layer

\item {} 
\sphinxAtStartPar
Correction to the accumulation of \sphinxcode{\sphinxupquote{g\_leaf\_phen}} in\sphinxhyphen{}between calls to TRIFFID

\end{itemize}

\sphinxAtStartPar
Full details of the tickets committed for JULES vn4.2 can be found on the \sphinxhref{https://code.metoffice.gov.uk/trac/jules/query?resolution=fixed\&milestone=JULES+v4.2+release}{collaboration repository Trac system}.

\sphinxstepscope


\section{JULES version 4.1 Release Notes}
\label{\detokenize{release_notes/JULES4-1:jules-version-4-1-release-notes}}\label{\detokenize{release_notes/JULES4-1::doc}}

\subsection{Irrigation demand}
\label{\detokenize{release_notes/JULES4-1:irrigation-demand}}
\sphinxAtStartPar
When enabled, irrigation demand adds water to the soil moisture up to the critical point, meaning that vegetation does not experience water stress. At the moment the amount of water added is not limited, although it will be limited in a future version.

\sphinxAtStartPar
There are two schemes that can be used to determine when to irrigate:
\begin{enumerate}
\sphinxsetlistlabels{\arabic}{enumi}{enumii}{}{.}%
\item {} 
\sphinxAtStartPar
This method calculates optimum planting dates for non\sphinxhyphen{}rice crops using averages from driving data and so requires at least one year of driving data. It was written by Nic Gedney and is based on the crop calendar from Doell \& Siebert (2002).

\item {} 
\sphinxAtStartPar
Uses development index (dvi) across all tiles from JULES\sphinxhyphen{}crop (written by Rutger Dankers). Maximum dvi must exceed \sphinxhyphen{}1 (indicates sowing) for irrigation to occur.

\end{enumerate}

\sphinxAtStartPar
See the documentation for {\hyperref[\detokenize{namelists/jules_irrig.nml:JULES_IRRIG::l_irrig_dmd}]{\sphinxcrossref{\sphinxcode{\sphinxupquote{l\_irrig\_dmd}}}}} for more details.


\subsection{Carbon cycle developments}
\label{\detokenize{release_notes/JULES4-1:carbon-cycle-developments}}
\sphinxAtStartPar
A number of carbon cycle developments are included in this release:
\begin{enumerate}
\sphinxsetlistlabels{\arabic}{enumi}{enumii}{}{.}%
\item {} 
\sphinxAtStartPar
Changes to the competition code to allow for flexible, height\sphinxhyphen{}based competition in TRIFFID. See {\hyperref[\detokenize{namelists/jules_vegetation.nml:JULES_VEGETATION::l_ht_compete}]{\sphinxcrossref{\sphinxcode{\sphinxupquote{l\_ht\_compete}}}}}.

\item {} 
\sphinxAtStartPar
Trait\sphinxhyphen{}based plant physiology that allows the plant physiology to be defined by parameters that are more readily definable from observations. See {\hyperref[\detokenize{namelists/jules_vegetation.nml:JULES_VEGETATION::l_trait_phys}]{\sphinxcrossref{\sphinxcode{\sphinxupquote{l\_trait\_phys}}}}}.

\item {} 
\sphinxAtStartPar
Code for simulating land\sphinxhyphen{}use change (e.g. forest clearance) including product pools (previously implemented in HadGEM2\sphinxhyphen{}ES). See {\hyperref[\detokenize{namelists/jules_vegetation.nml:JULES_VEGETATION::l_landuse}]{\sphinxcrossref{\sphinxcode{\sphinxupquote{l\_landuse}}}}}.

\item {} 
\sphinxAtStartPar
Time\sphinxhyphen{}varying agricultural fraction now officially supported. Time\sphinxhyphen{}varying CO2 concentration is also supported with some caveats. See {\hyperref[\detokenize{namelists/prescribed_data.nml:supported-prescribed-variables}]{\sphinxcrossref{\DUrole{std,std-ref}{the list of supported prescribed data variables}}}}.

\item {} 
\sphinxAtStartPar
Switch to enable/disable the adjustment of fractions during initialisation. See {\hyperref[\detokenize{namelists/jules_vegetation.nml:JULES_VEGETATION::l_recon}]{\sphinxcrossref{\sphinxcode{\sphinxupquote{l\_recon}}}}}.

\end{enumerate}


\subsection{Maximum and minimum output types}
\label{\detokenize{release_notes/JULES4-1:maximum-and-minimum-output-types}}
\sphinxAtStartPar
It is now possible to output the maximum and minimum value over the output period of any variable, e.g. monthly maximum.

\sphinxAtStartPar
See {\hyperref[\detokenize{namelists/output.nml:JULES_OUTPUT_PROFILE::output_type}]{\sphinxcrossref{\sphinxcode{\sphinxupquote{output\_type}}}}} for more information.


\subsection{Changes to the coupling routines}
\label{\detokenize{release_notes/JULES4-1:changes-to-the-coupling-routines}}
\sphinxAtStartPar
The routines that couple the JULES science to the UM and to the standalone wrapper have changed \sphinxhyphen{} see \sphinxcode{\sphinxupquote{src/control/standalone/control.F90}}.

\sphinxAtStartPar
This is mainly in order to simplify the coupling with the UM and to facilitate upcoming developments.


\subsection{Bugs and other changes}
\label{\detokenize{release_notes/JULES4-1:bugs-and-other-changes}}\begin{itemize}
\item {} 
\sphinxAtStartPar
It is now possible to run with a fixed 365\sphinxhyphen{}day calender (i.e. no leap years) using the {\hyperref[\detokenize{namelists/timesteps.nml:JULES_TIME::l_leap}]{\sphinxcrossref{\sphinxcode{\sphinxupquote{l\_leap}}}}} flag.

\item {} 
\sphinxAtStartPar
Corrected temperature limitation on soil respiration to be consistent with HadGEM2\sphinxhyphen{}ES.

\item {} 
\sphinxAtStartPar
Minor tweaks to the crop model.

\item {} 
\sphinxAtStartPar
Improvements to the BVOC emissions model, including linking with UKCA via the UM for coupled studies.

\item {} 
\sphinxAtStartPar
Changes to introduce the COARE algorithm for surface exchange over the ocean.

\item {} 
\sphinxAtStartPar
Improved error messages for I/O errors (e.g. namelist reading)

\item {} 
\sphinxAtStartPar
Fixed a bug where the start and end time of data are not initialised correctly when using the daily disaggregator.

\item {} 
\sphinxAtStartPar
Fixed a bug in tilepts caused by the use of short\sphinxhyphen{}circuiting logic that is not supported by some compilers.

\end{itemize}

\sphinxstepscope


\section{JULES version 4.0 Release Notes}
\label{\detokenize{release_notes/JULES4-0:jules-version-4-0-release-notes}}\label{\detokenize{release_notes/JULES4-0::doc}}

\subsection{JULES\sphinxhyphen{}Crop crop model}
\label{\detokenize{release_notes/JULES4-0:jules-crop-crop-model}}
\sphinxAtStartPar
JULES vn4.0 sees the introduction of the JULES\sphinxhyphen{}Crop crop model. This has been the result of many years of hard work from Tom Osbourne et. al. at the \sphinxhref{http://www.reading.ac.uk/}{University of Reading}.

\sphinxAtStartPar
A lot of the work done in getting it ready for the trunk and testing was done in the Met Office by Karina Williams and Jemma Gornall.


\subsection{Daily disaggregator for forcing data}
\label{\detokenize{release_notes/JULES4-0:daily-disaggregator-for-forcing-data}}
\sphinxAtStartPar
JULES can now be driven with daily forcing data, and the daily disaggregator will disaggregate the daily forcing down onto the model timestep.

\sphinxAtStartPar
For more information, see {\hyperref[\detokenize{namelists/drive.nml:JULES_DRIVE::l_daily_disagg}]{\sphinxcrossref{\sphinxcode{\sphinxupquote{l\_daily\_disagg}}}}}.


\subsection{Major namelist changes}
\label{\detokenize{release_notes/JULES4-0:major-namelist-changes}}
\sphinxAtStartPar
JULES vn4.0 also sees a major revamp of the science\sphinxhyphen{}related namelists. The monolithic JULES\_SWITCHES namelist, and various others, are gone, and have been replaced with science section namelists. For more details, see {\hyperref[\detokenize{namelists/contents::doc}]{\sphinxcrossref{\DUrole{doc}{The JULES namelist files}}}}.

\sphinxAtStartPar
This has been with the aim of providing a GUI for editing the JULES namelists using \sphinxhref{http://metomi.github.io/rose/doc/html/index.html}{Rose}, which is now available \sphinxhyphen{} see {\hyperref[\detokenize{building-and-running/rose::doc}]{\sphinxcrossref{\DUrole{doc}{Automatic upgrading and GUI using Rose}}}}.

\sphinxAtStartPar
It also has the advantage that the new namelists are cut\sphinxhyphen{}and\sphinxhyphen{}paste\sphinxhyphen{}able between the UM and JULES, which should make it easier to ensure that the same science is being used in online and offline runs.


\subsection{Removal of GNU make build files}
\label{\detokenize{release_notes/JULES4-0:removal-of-gnu-make-build-files}}
\sphinxAtStartPar
After a period of supporting two build systems (FCM make and GNU make), it has been decided that support for GNU make should be removed. The overhead of maintaining two build systems was getting too large, and FCM make is preferred for several reasons:
\begin{description}
\sphinxlineitem{Directory structure}\begin{itemize}
\item {} 
\sphinxAtStartPar
The directory level dependencies used by the JULES Makefile to ensure files are compiled in the correct order forced the directory structure to adapt to it.

\item {} 
\sphinxAtStartPar
FCM make does automatic dependency analysis for each file to ensure they are compiled in the correct order, meaning the directory structure doesn’t have to be compromised to keep the build system happy.

\end{itemize}

\sphinxlineitem{Dependencies}\begin{itemize}
\item {} 
\sphinxAtStartPar
The JULES GNU Makefiles required that dependencies be manually maintained, both in terms of the order of sub\sphinxhyphen{}makes and actual file dependencies within the sub\sphinxhyphen{}makes.

\item {} 
\sphinxAtStartPar
FCM make automatically detects all dependencies and does things in the correct order.

\end{itemize}

\sphinxlineitem{Parallel builds}\begin{itemize}
\item {} 
\sphinxAtStartPar
JULES builds with GNU make could not be parallelised, because of the use of directory level sub\sphinxhyphen{}makes.

\item {} 
\sphinxAtStartPar
FCM make considers each individual file, so builds can be parallelised.

\end{itemize}

\sphinxlineitem{Integration with Rose}\begin{itemize}
\item {} 
\sphinxAtStartPar
FCM make has good integration with Rose, allowing the Rose GUI for JULES to configure and run builds as well as the namelists.

\end{itemize}

\end{description}


\subsection{Bugs and other changes}
\label{\detokenize{release_notes/JULES4-0:bugs-and-other-changes}}\begin{itemize}
\item {} 
\sphinxAtStartPar
Output for land points not comparing between land\_only = T and F runs with 2D grid

\item {} 
\sphinxAtStartPar
Incorrect behaviour when \sphinxcode{\sphinxupquote{spinup\_end}} == \sphinxcode{\sphinxupquote{data\_end}}

\item {} 
\sphinxAtStartPar
Fixed overflow problem with \sphinxcode{\sphinxupquote{datetime\_diff}} when \sphinxcode{\sphinxupquote{datetime}}s are too far apart

\item {} 
\sphinxAtStartPar
Removed old implicit solver and ltimer code

\item {} 
\sphinxAtStartPar
Unified management of printing and error reporting for UM and standalone

\end{itemize}

\sphinxstepscope


\section{JULES version 3.4 Release Notes}
\label{\detokenize{release_notes/JULES3-4:jules-version-3-4-release-notes}}\label{\detokenize{release_notes/JULES3-4::doc}}
\sphinxAtStartPar
n.b. A critical memory leak was found in JULES v3.4 that necessitated a new release, designated v3.4.1.


\subsection{Changes to semantics of output}
\label{\detokenize{release_notes/JULES3-4:changes-to-semantics-of-output}}
\sphinxAtStartPar
The output semantics used since JULES vn3.2 (i.e. state variables captured at the start of a timestep, flux variables captured at the end) were confusing some users. The semi\sphinxhyphen{}implicit scheme in JULES is designed so that the state and fluxes at the end of a timestep are consistent with each other, but under the previous semantics these were staggered by one timestep in output files.

\sphinxAtStartPar
All variables are now captured at the end of a timestep, so state and flux variables at a particular timestep in output files will be consistent with each other. A new option has been added to request the output of initial state, however very few users will have a use for this. It is still the case that the value in the \sphinxcode{\sphinxupquote{time}} variable can be used to place snapshot data in time, and the values in \sphinxcode{\sphinxupquote{time\_bounds}} represent the interval over which a mean or accumulation applies.

\sphinxAtStartPar
More details can be found at {\hyperref[\detokenize{output::doc}]{\sphinxcrossref{\DUrole{doc}{JULES output}}}}.


\subsection{Input and/or output of variables with multiple ‘levels’ dimensions has been improved}
\label{\detokenize{release_notes/JULES3-4:input-and-or-output-of-variables-with-multiple-levels-dimensions-has-been-improved}}
\sphinxAtStartPar
In previous versions of JULES since vn3.1, variables could only be input or output with a single ‘levels’ dimension. In particular, this caused problems with variables in the new snow scheme, which have two ‘levels’ dimensions on top of the grid dimensions (tiles and snow levels). This led to compromises being made with the snow layer variables:
\begin{itemize}
\item {} 
\sphinxAtStartPar
It was only possible to initialise the snow layer variables using a constant value, from a previous dump or using {\hyperref[\detokenize{namelists/initial_conditions.nml:JULES_INITIAL::total_snow}]{\sphinxcrossref{\sphinxcode{\sphinxupquote{total\_snow}}}}}

\item {} 
\sphinxAtStartPar
In output files, the snow layer variables were represented using a separate variable for each tile

\end{itemize}

\sphinxAtStartPar
This problem is solved in JULES vn3.4 \sphinxhyphen{} it is now possible to input and output variables with multiple ‘levels’ dimensions (there is not even a restriction to two ‘levels’ dimensions). This means that both compromises for snow layer variables detailed above have been removed.


\subsection{Streamlined process for adding new variables for input and/or output}
\label{\detokenize{release_notes/JULES3-4:streamlined-process-for-adding-new-variables-for-input-and-or-output}}
\sphinxAtStartPar
Although fairly simple, the process for adding a new variable for input and/or output in JULES vn3.1 \sphinxhyphen{} vn3.3 required several edits to be made, and hence provided many opportunities to make mistakes. This process is simplified in JULES vn3.4 to require fewer edits. More details can be found at {\hyperref[\detokenize{code/io:implementing-new-variables-for-input-and-output}]{\sphinxcrossref{\DUrole{std,std-ref}{Implementing new variables for input and output}}}}.


\subsection{Other changes}
\label{\detokenize{release_notes/JULES3-4:other-changes}}\begin{description}
\sphinxlineitem{Tidying of boundary layer code}
\sphinxAtStartPar
Some small changes have been made to tidy up some of the boundary layer code (i.e. routines in \sphinxcode{\sphinxupquote{src/science/surface}}) \sphinxhyphen{} this is mostly removing unused variables and tidying up subroutine argument lists.

\sphinxlineitem{OpenMP related changes}
\sphinxAtStartPar
Some \sphinxhref{http://en.wikipedia.org/wiki/OpenMP}{OpenMP} directives have been added to certain loops. OpenMP is a form of shared\sphinxhyphen{}memory parallelism in which the user inserts directives (specially formatted code comments) providing information that allows the compiler to parallelise sections of code (in particular loops) without worrying about corrupting data. It is used in the UM, but is currently not enabled when compiling JULES standalone.

\end{description}


\subsection{Bugs fixed}
\label{\detokenize{release_notes/JULES3-4:bugs-fixed}}\begin{itemize}
\item {} 
\sphinxAtStartPar
Output (including dump files) not correctly generated for the last spin\sphinxhyphen{}up cycle when spin\sphinxhyphen{}up fails and {\hyperref[\detokenize{namelists/timesteps.nml:JULES_SPINUP::terminate_on_spinup_fail}]{\sphinxcrossref{\sphinxcode{\sphinxupquote{terminate\_on\_spinup\_fail}}}}} = TRUE

\item {} 
\sphinxAtStartPar
\sphinxcode{\sphinxupquote{lw\_net}} diagnostic does not include the contribution from the reflected incoming longwave if the emissivity is less than zero

\end{itemize}

\sphinxstepscope


\section{JULES version 3.3 Release Notes}
\label{\detokenize{release_notes/JULES3-3:jules-version-3-3-release-notes}}\label{\detokenize{release_notes/JULES3-3::doc}}

\subsection{Ability to run JULES in parallel}
\label{\detokenize{release_notes/JULES3-3:ability-to-run-jules-in-parallel}}
\sphinxAtStartPar
JULES can now run multiple points in parallel, using multiple cores on the same machine or a cluster of machines. This is accomplished using \sphinxhref{http://en.wikipedia.org/wiki/Message\_Passing\_Interface}{MPI (Message Passing Interface)}, a standardised message passing interface. Several implementations of MPI are available, the most commonly used being \sphinxhref{http://www.mpich.org/}{MPICH2} and \sphinxhref{http://www.open-mpi.org/}{OpenMPI}.

\sphinxAtStartPar
JULES takes advantage of the parallel I/O features in \sphinxhref{http://www.hdfgroup.org/HDF5/}{HDF5} / \sphinxhref{http://www.unidata.ucar.edu/software/netcdf/}{NetCDF4}. These are not enabled by default, and so must be explicitly enabled when HDF5 / NetCDF4 are compiled. More information on how to do this can be found on the \sphinxhref{http://www.unidata.ucar.edu/software/netcdf/docs/getting\_and\_building\_netcdf.html\#build\_parallel}{NetCDF website}.

\sphinxAtStartPar
Information on how to build and run JULES in parallel can be found in the JULES User Guide. \sphinxstyleemphasis{Note that although this development has proven stable during testing, it is still experimental and is considered to be for advanced users only.}


\subsection{Changes to documentation}
\label{\detokenize{release_notes/JULES3-3:changes-to-documentation}}
\sphinxAtStartPar
From a users point of view, the most important change is that the JULES documentation and coding standards are now provided in two forms \sphinxhyphen{} HTML (this is the preferred format) and PDF. The HTML documentation is also available on the web at \sphinxurl{http://jules-lsm.github.io/}.

\sphinxAtStartPar
This has been made possible by migrating the documentation from a single massive Word document to the \sphinxhref{http://sphinx-doc.org/}{Sphinx} documentation generator (with some custom extensions to better support Fortran namelists). Although originally intended to document Python projects, Sphinx’s extensibility has seen it adopted for a wide range of projects. Using Sphinx has several advantages over the previous monolithic Word document:
\begin{itemize}
\item {} 
\sphinxAtStartPar
Both forms of documentation (HTML and PDF) can be built from the same sources.

\item {} 
\sphinxAtStartPar
The documentation is now split into several smaller files that are combined by Sphinx at build\sphinxhyphen{}time, leading to increased readability.

\item {} 
\sphinxAtStartPar
\sphinxhref{http://docutils.sourceforge.net/rst.html}{reStructuredText}, the markup language used by Sphinx, is a plain text format, meaning that it can be version controlled much more effectively than a Word document (which is treated by Subversion as a single binary entity).

\item {} 
\sphinxAtStartPar
The only software required to update the documentation is your favourite text editor (rather than Word).

\end{itemize}

\sphinxAtStartPar
The \sphinxhref{https://puma.nerc.ac.uk/trac/JULES}{JULES repository on PUMA} has also been refactored so that configurations, documentation and examples sit in a separate project to the core Fortran code.


\subsection{Other changes}
\label{\detokenize{release_notes/JULES3-3:other-changes}}\begin{description}
\sphinxlineitem{Disambiguation of sea ice roughness lengths for heat and momentum}
\sphinxAtStartPar
Prior to vn3.3, these were implicitly assumed to be equal by the code. They can now be set separately in the namelist \sphinxcode{\sphinxupquote{JULES\_SURF\_PARAM}}.

\sphinxlineitem{Improvements to the numerics in the soil hydrology}
\sphinxAtStartPar
Previously, the soil hydrology scheme coped poorly with significant gradients in soil moisture because of the sensitive dependence of the hydraulic conductivity and soil water suction on the soil moisture. See the new switch \sphinxcode{\sphinxupquote{l\_dpsids\_dsdz}}.

\sphinxlineitem{Implicit numerics for land ice}
\sphinxAtStartPar
Previously, the updating of land ice temperatures was always explicit, limiting the thickness of soil levels that can be used with standard time steps. There is now an option for implicit numerics for land ice \sphinxhyphen{} see the new switch \sphinxcode{\sphinxupquote{l\_land\_ice\_imp}}.

\sphinxlineitem{Scaling of land surface albedo to agree with a given input}
\sphinxAtStartPar
An option has been added to prescribe the grid\sphinxhyphen{}box mean snow\sphinxhyphen{}free albedo to a given input (e.g. observations, climatology). See the new switch \sphinxcode{\sphinxupquote{l\_albedo\_obs}}. For SW albedos, the albedos of the individual tiles are scaled linearly so that the grid\sphinxhyphen{}box mean albedo matches the observations, within limits for each tile. When VIS and NIR albedos are required then the input parameters are scaled and corrected in a similar manner. The change was included in the Global Land configuration at vn5.0: \sphinxurl{https://code.metoffice.gov.uk/trac/GL/ticket/8}.

\sphinxlineitem{BVOC emissions now on a switch}
\sphinxAtStartPar
Previously, BVOC emissions diagnostics were calculated all the time, regardless of whether they were output. A new switch \sphinxhyphen{} \sphinxcode{\sphinxupquote{l\_bvoc\_emis}} \sphinxhyphen{} has been added to enable the calculation of these diagnostics only when required.

\sphinxlineitem{Improvements to logging}
\sphinxAtStartPar
A new namelist file \sphinxhyphen{} \sphinxcode{\sphinxupquote{logging.nml}} \sphinxhyphen{} has been added to give more control over log output from JULES. Previously all output was directed to \sphinxcode{\sphinxupquote{stdout}}.

\sphinxlineitem{Specify namelist directory as an argument}
\sphinxAtStartPar
It is now possible to specify the directory containing the namelist files as a command line argument to JULES. If no argument is given, JULES looks for the namelist files in the current working directory. Previously, JULES had to be executed in the directory containing the namelists \sphinxhyphen{} this change should make it easier to run JULES in batch mode.

\end{description}


\subsection{Bugs fixed}
\label{\detokenize{release_notes/JULES3-3:bugs-fixed}}\begin{itemize}
\item {} 
\sphinxAtStartPar
Initialisation of \sphinxcode{\sphinxupquote{chr1p5m}} and \sphinxcode{\sphinxupquote{resfs}} in \sphinxcode{\sphinxupquote{sf\_exch}}.

\item {} 
\sphinxAtStartPar
Fix for potential divide\sphinxhyphen{}by\sphinxhyphen{}zero in \sphinxcode{\sphinxupquote{sf\_stom}} when running with \sphinxcode{\sphinxupquote{can\_rad\_mod = 1}}.

\item {} 
\sphinxAtStartPar
Various UM\sphinxhyphen{}related fixes not relevant to standalone JULES (ENDGAME, aerosol deposition scheme, etc.).

\end{itemize}

\sphinxstepscope


\section{JULES version 3.2 Release Notes}
\label{\detokenize{release_notes/JULES3-2:jules-version-3-2-release-notes}}\label{\detokenize{release_notes/JULES3-2::doc}}
\sphinxAtStartPar
JULES version 3.2 sees several enhancements and bug fixes in both the science and control code.


\subsection{Standard Configurations}
\label{\detokenize{release_notes/JULES3-2:standard-configurations}}
\sphinxAtStartPar
A set of standard science configurations have been defined. These are based on well tested operational Met Office models, and are intended to cover a wide range of use cases.


\subsection{Improvements to output}
\label{\detokenize{release_notes/JULES3-2:improvements-to-output}}
\sphinxAtStartPar
In JULES version 3.1, under some circumstances, it was not entirely clear how the timestamps in output files applied to the values. This has been thoroughly addressed in version 3.2.

\sphinxAtStartPar
Changes have also been made to the attributes of output variables:
\begin{itemize}
\item {} 
\sphinxAtStartPar
The units attribute for output variables has been updated to be compliant with \sphinxhref{http://www.unidata.ucar.edu/software/udunits/}{UDUNITS2}.

\item {} 
\sphinxAtStartPar
A \sphinxhref{http://cfconventions.org}{CF conventions} coordinates attribute has been added to all output variables that explicitly links the latitude and longitude to the data.

\end{itemize}


\subsection{Biogenic Volatile Organic Compound (BVOC) emissions}
\label{\detokenize{release_notes/JULES3-2:biogenic-volatile-organic-compound-bvoc-emissions}}
\sphinxAtStartPar
Code written by Federica Pacifico for isoprene emissions has been implemented and extended to include monoterpene, acetone and methanol emissions. This addition is purely diagnostic in the standalone model (i.e. provides new output variables, but has no feedbacks), but will allow the UM to implement interactive BVOC emissions (i.e. with feedbacks) in the future.

\sphinxAtStartPar
A paper has been written describing and evaluating the isoprene emission scheme \sphinxhyphen{} Pacifico et. al., 2011. Atmos. Chem. and Phys., 11, 4371\sphinxhyphen{}4389 (\sphinxhref{http://www.atmos-chem-phys.net/11/4371/2011/acp-11-4371-2011.pdf}{PDF}).


\subsection{Alternative build system}
\label{\detokenize{release_notes/JULES3-2:alternative-build-system}}
\sphinxAtStartPar
It is now possible to build JULES using FCM make. FCM is a set of tools developed by the Met Office for managing and building source code, with a particular focus on making it easy to build large Fortran programs (such as JULES). FCM is open source software, and can be downloaded for free from \sphinxhref{https://github.com/metomi/fcm}{Github}.


\subsection{Bugs fixed}
\label{\detokenize{release_notes/JULES3-2:bugs-fixed}}\begin{itemize}
\item {} 
\sphinxAtStartPar
Array bounds error with \sphinxcode{\sphinxupquote{SICE\_INDEX\_NCAT}}.

\item {} 
\sphinxAtStartPar
Incorrect usage of \sphinxcode{\sphinxupquote{COR\_MO\_ITER}}.

\item {} 
\sphinxAtStartPar
Monthly/yearly output files not rolling over properly on certain configurations of GFortran.

\item {} 
\sphinxAtStartPar
A collection of small memory leaks.

\item {} 
\sphinxAtStartPar
Not able to read or write ASCII dumps with the new snow scheme on.

\item {} 
\sphinxAtStartPar
Use fixed dimension names for output files (rather than using those given for input files).

\item {} 
\sphinxAtStartPar
Using \sphinxcode{\sphinxupquote{can\_rad\_mod = 5}} causes night\sphinxhyphen{}time dark respiration to be 0 under certain circumstances.

\end{itemize}

\sphinxstepscope


\section{JULES version 3.1 Release Notes}
\label{\detokenize{release_notes/JULES3-1:jules-version-3-1-release-notes}}\label{\detokenize{release_notes/JULES3-1::doc}}
\sphinxAtStartPar
JULES version 3.1 sees little change to the science of JULES, but contains several major developments intended to make development easier going forward.


\subsection{Restructuring of the code}
\label{\detokenize{release_notes/JULES3-1:restructuring-of-the-code}}
\sphinxAtStartPar
The directory structure of the JULES code has been changed to be more logical and allow for a cleaner separation between control, initialisation, I/O and science code. This includes the introduction of directories containing UM\sphinxhyphen{}specific code for initialisation in the UM. This was done as part of the work to completely remove (MOSES and) JULES code from the UM code repository \sphinxhyphen{} it now sits in its own repository.


\subsection{New I/O framework}
\label{\detokenize{release_notes/JULES3-1:new-i-o-framework}}
\sphinxAtStartPar
The input and output code has been completely revamped in order to modularise and simplify the code. It allows for data to be input on any timestep and interpolated down to the model timestep. Support for outputting of means and accumulations remains. NetCDF is now the only supported binary format (although it should be relatively simple to write drivers for other output formats if desired), and ASCII files are allowed for data at a single location only. Support for the GrADS flat binary format has been dropped, although the NetCDF output should be usable with GrADS with very little work.


\subsection{User Interface changes}
\label{\detokenize{release_notes/JULES3-1:user-interface-changes}}
\sphinxAtStartPar
The user interface also sees significant changes. The monolithic .jin run control file has been replaced by several smaller files containing Fortran namelists for input of options and parameters. This is more consistent with the UM, and offers the opportunity to adapt UM tools to provide a GUI for running JULES in the future.


\subsection{Other changes}
\label{\detokenize{release_notes/JULES3-1:other-changes}}
\sphinxAtStartPar
There are several not\sphinxhyphen{}insignificant changes to the science code:
\begin{itemize}
\item {} 
\sphinxAtStartPar
Structures are now used for dimensioning variables \sphinxhyphen{} this allows for more flexibility of grids than the old system of row\_length/rows and halos.

\item {} 
\sphinxAtStartPar
Move to a new implicit solver \sphinxhyphen{} \sphinxcode{\sphinxupquote{sf\_impl2}} is now used rather than \sphinxcode{\sphinxupquote{sf\_impl}} for consistency with the UM. However, the way the implicit coupling is set up means it operates in a similar way to the old scheme.

\item {} 
\sphinxAtStartPar
A change in the way fresh snow is handled in the multi\sphinxhyphen{}layer snow scheme \sphinxhyphen{} the density of fresh snow is now prescribed by a new variable (\sphinxcode{\sphinxupquote{rho\_snow\_fresh}}). Suggested by Cécile Ménard and implemented by Doug Clark.

\item {} 
\sphinxAtStartPar
Bug fix from Doug Clark for the multi\sphinxhyphen{}layer snow scheme that fixes problems with the model oscillating between 0 and 1 snow layers every timestep, preventing snow melt.

\item {} 
\sphinxAtStartPar
Changes to the sea\sphinxhyphen{}ice surface exchange when operating as part of the UM. This will not affect the majority of users.

\item {} 
\sphinxAtStartPar
Slight changes to the coupling between the explicit and implicit schemes. The vast majority of users will not need to worry about this.

\end{itemize}

\sphinxstepscope


\section{JULES version 3.0 Release Notes}
\label{\detokenize{release_notes/JULES3-0:jules-version-3-0-release-notes}}\label{\detokenize{release_notes/JULES3-0::doc}}
\sphinxAtStartPar
The major change in version 3.0 is the introduction of the IMOGEN impacts tool. IMOGEN is a system where JULES is gridded on to surface land points, and is forced with an emulation of climate change using “pattern\sphinxhyphen{}scaling” calibrated against the Hadley Centre GCM. This climate change impacts system has the advantage that:
\begin{itemize}
\item {} 
\sphinxAtStartPar
The pattern\sphinxhyphen{}scaling allows estimates of climate change for a broad range of emissions scenarios.

\item {} 
\sphinxAtStartPar
New process understanding can be tested for its global implications.

\item {} 
\sphinxAtStartPar
New process understanding can also be checked for stability before full inclusion in a GCM.

\item {} 
\sphinxAtStartPar
By adding climate change anomalies to datasets such as the CRU dataset, then GCM biases can be removed.

\end{itemize}

\sphinxAtStartPar
It must be recognised that the system is “off\sphinxhyphen{}line”, and so if major changes to the land surface occur there might be local and regional feedbacks that can only be predicted using a fully coupled GCM. Hence IMOGEN doesn’t replace GCMs, but it does give a very powerful first\sphinxhyphen{}look as to potential land surface changes in an anthropogenically forced varying climate. This was accomplished with help from Mark Lomas at the University of Sheffield and Chris Huntingford at CEH.

\sphinxAtStartPar
There are also several small bug fixes:
\begin{itemize}
\item {} 
\sphinxAtStartPar
A fix effecting fluxes in \sphinxtitleref{sf\_stom} from Lina Mercado at CEH. This bug fix was announced on the mailing list.

\item {} 
\sphinxAtStartPar
Small fixes for potential evaporation and canopy snow depth from the UM.

\item {} 
\sphinxAtStartPar
A small issue with some memory not being deallocated at the end of a run.

\end{itemize}

\sphinxstepscope


\section{JULES version 2.2 Release Notes}
\label{\detokenize{release_notes/JULES2-2:jules-version-2-2-release-notes}}\label{\detokenize{release_notes/JULES2-2::doc}}
\sphinxAtStartPar
Along with fixes for known bugs, the changes made for version 2.2 mostly consist of several small additions to the science code. Changes to the control code have mostly been limited to bug\sphinxhyphen{}fixes.
\begin{itemize}
\item {} 
\sphinxAtStartPar
New options for treatment of urban tiles \sphinxhyphen{} inclusion of the Met Office Reading Urban Surface Exchange Scheme (MORUSES) and a simple two tile urban scheme.

\item {} 
\sphinxAtStartPar
Effects of ozone damage on stomata from Stephen Sitch at the University of Leeds.

\item {} 
\sphinxAtStartPar
New treatment of direct/diffuse radiation in the canopy from Lina Mercado at CEH.

\item {} 
\sphinxAtStartPar
A new switch allows the competing vegetation portion of TRIFFID to be switched on and off independently of the rest of TRIFFID (i.e. it is now possible to use the RothC soil carbon without having changing vegetation fractions).

\end{itemize}

\sphinxAtStartPar
There have also been changes made to the way JULES is compiled, due to the re\sphinxhyphen{}integration with the Met Office Unified Model (UM). The UM uses preprocessor directives to compile different versions of routines depending on the selected science options. For compatibility with this system, JULES will now require a compiler with a preprocessor. This should not be noticed by the majority of users \sphinxhyphen{} most modern compilers include a preprocessor and the Makefile deals with setting up the appropriate preprocessor options.

\sphinxAtStartPar
Finally, JULES was added to the UM code repository as a mirror of the JULES repository at (UM version vn7.5, JULES vn2.2).

\sphinxstepscope


\section{JULES version 2.1 Release Notes}
\label{\detokenize{release_notes/JULES2-1:jules-version-2-1-release-notes}}\label{\detokenize{release_notes/JULES2-1::doc}}
\sphinxAtStartPar
Versions 2.1.1 and 2.1.2 were released to fix major bugs found in v2.1 \sphinxhyphen{} they contain no new features.

\sphinxAtStartPar
Version 2.1 of JULES includes extensive modifications to the descriptions of the processes and to the control\sphinxhyphen{}level code (such as input and output). These are covered briefly below. Several bug fixes and minor changes to make the code more robust have also been applied. All files are now technically FORTRAN90 (.f90) although many are simply reformatted FORTRAN77 files in which continuation lines are now indicated by the use of the ‘\&’ character.


\subsection{Process descriptions}
\label{\detokenize{release_notes/JULES2-1:process-descriptions}}
\sphinxAtStartPar
The main change is that a new multi\sphinxhyphen{}layer snow scheme is available. This scheme was developed by Richard Essery at the University of Edinburgh and co\sphinxhyphen{}workers. At the time of writing there is little scientific documentation of this development, but this will be made available as soon as possible. In brief, the older, simple scheme represents the snowpack as a single layer with prescribed properties such as density, whereas the new scheme has a variable number of layers according to the depth of snow present, and each layer has prognostic temperature, density, grain size, and solid and liquid water content. The new scheme reverts to the previous, simpler scheme if \sphinxcode{\sphinxupquote{nsmax = 0}} or when the snowpack becomes very thin.

\sphinxAtStartPar
A four\sphinxhyphen{}pool soil carbon model based on the RothC model now replaces the single pool model when dynamic vegetation (TRIFFID) is selected.

\sphinxAtStartPar
There have been several major changes that most users will not notice or need be concerned about. These include a change in the linearization procedure that is used in the calculation of surface energy fluxes (described in the technical documentation). A standard interface is now used to calculate fluxes over land, sea and sea ice. Each surface tile now has an elevation relative to the gridbox mean.

\sphinxAtStartPar
These changes mean that, even with the new snow scheme switched off (\sphinxcode{\sphinxupquote{nsmax=0}}), results from v2.1 will generally not be identical to those from v2.0.


\subsection{Control\sphinxhyphen{}level code}
\label{\detokenize{release_notes/JULES2-1:control-level-code}}
\sphinxAtStartPar
The major change at v2.1 to the control\sphinxhyphen{}level code is that NetCDF output is now supported. Both diagnostic and restart files (dumps) can be in NetCDF format. There have been several changes to the run control file, partly to reflect new science but also in an attempt to organise the file better. These changes mean that run control and restart files from JULES v2.0 are not compatible with v2.1 (although they could be reformatted without too much difficulty).

\sphinxAtStartPar
Finally, since JULESvn1.0 the MOSES and JULES code bases have been evolving separately, but with JULES v2.1 these differences have been reconciled with the UM.

\sphinxstepscope


\section{JULES version 2.0 Release Notes}
\label{\detokenize{release_notes/JULES2-0:jules-version-2-0-release-notes}}\label{\detokenize{release_notes/JULES2-0::doc}}
\sphinxAtStartPar
The physical processes and their representation in version 2.0 have not changed from version 1. However, version 2.0 is much more flexible in terms of input and output, and allows JULES to be run on a grid of points. New features include:
\begin{itemize}
\item {} 
\sphinxAtStartPar
Ability to run on a grid.

\item {} 
\sphinxAtStartPar
Choice of ASCII or binary formats for input and output files (also limited support of NetCDF input).

\item {} 
\sphinxAtStartPar
More flexible surface types \sphinxhyphen{} number and types can vary.

\item {} 
\sphinxAtStartPar
Optional time\sphinxhyphen{}varying, prescribed vegetation properties.

\item {} 
\sphinxAtStartPar
More choice of meteorological input variables.

\item {} 
\sphinxAtStartPar
Optional automatic spin\sphinxhyphen{}up.

\item {} 
\sphinxAtStartPar
Enhanced diagnostics \sphinxhyphen{} large choice of variables, frequency of output, sampling frequency, etc.

\end{itemize}

\sphinxstepscope


\section{JULES version 1.0}
\label{\detokenize{release_notes/JULES1-0:jules-version-1-0}}\label{\detokenize{release_notes/JULES1-0::doc}}
\sphinxAtStartPar
Initial public release of the JULES code. JULES v1.0 will only run for a single point and only supports ASCII data.

\sphinxAtStartPar
For further information about these releases, please see \sphinxhref{http://jules.jchmr.org/content/code\#archive-versions}{here}.

\sphinxstepscope


\chapter{Overview of JULES}
\label{\detokenize{overview/intro:overview-of-jules}}\label{\detokenize{overview/intro::doc}}
\sphinxAtStartPar
This section provides a brief overview of JULES and an introduction to some of the key switches that determine what the model will simulate in a given run.

\sphinxstepscope


\section{Overview}
\label{\detokenize{overview/overview:overview}}\label{\detokenize{overview/overview::doc}}
\sphinxAtStartPar
This section provides a brief overview of JULES, largely so as to provide background information and introduce terms used in the rest of the manual. Further details can be found at the \sphinxhref{http://jules.jchmr.org/}{JULES website} and in the two \sphinxhref{http://jules.jchmr.org/content/about}{JULES description papers} (Best et al., 2011, GMD and Clark et al., 2011, GMD).

\sphinxAtStartPar
For both gridded and single point runs, JULES views each gridbox as consisting of a number of surface types. The fractional area of each surface type is either prescribed by the user or modelled by the TRIFFID sub\sphinxhyphen{}model. Each surface type is represented by a surface tile, and a separate energy balance is calculated for each surface tile. The gridbox average energy balance is found by weighting the values from each surface tile. In its standard form, JULES recognises nine surface types: broadleaf trees, needleleaf trees, C3 (temperate) grass, C4 (tropical) grass, shrubs, urban, inland water, bare soil and ice. These 9 types are modelled as 9 surface tiles. A land gridbox is either any mixture of the first 8 surface types, or is land ice. Note that, from version 2.0, one is not limited to these 9 standard surface types (unless running TRIFFID).

\sphinxAtStartPar
Soil processes are modelled in several layers. Each surface tile can be associated with its own soil tile, or all surface tiles can interact with a shared soil column. Each gridbox requires meteorological driving variables (such as air temperature) and variables that describe the soil properties at that location. It is also possible to prescribe certain characteristics of the vegetation, such as Leaf Area Index, to vary between gridboxes.

\sphinxAtStartPar
JULES can be run for any number of gridboxes from one upwards. The number of gridboxes is limited by the availability of computing power and suitable input data. When run on a grid, JULES models the average state of the land surface within the area of the gridbox and most quantities are taken to be homogeneous within the gridbox (with options to include subgrid\sphinxhyphen{}scale variability of a few, such as rainfall). In that case, the input data are also area averages. JULES can also be run “at a point”, with inputs that are taken to represent conditions at that point \sphinxhyphen{} this configuration might be used when field measurements of meteorological conditions are available.

\sphinxstepscope


\section{Some key switches}
\label{\detokenize{overview/key-switches:some-key-switches}}\label{\detokenize{overview/key-switches::doc}}
\sphinxAtStartPar
There are many variables that act together to determine how a run of JULES is set up and these are covered in detail in {\hyperref[\detokenize{namelists/contents::doc}]{\sphinxcrossref{\DUrole{doc}{The JULES namelist files}}}}. Additionally, \sphinxhref{http://jules.jchmr.org/content/configurations}{configurations} illustrate suitable combinations of options. Here we highlight a few key switches that select broad areas of science, particularly for the benefit of new users.

\sphinxAtStartPar
The phenology model for natural vegetation can be enabled using {\hyperref[\detokenize{namelists/jules_vegetation.nml:JULES_VEGETATION::l_phenol}]{\sphinxcrossref{\sphinxcode{\sphinxupquote{l\_phenol}}}}} which uses the leaf turnover rate to calculate a time\sphinxhyphen{}varying Leaf Area Index (LAI).

\sphinxAtStartPar
To simulate carbon stocks in natural vegetation, the TRIFFID dynamic vegetation model can be enabled via the switch {\hyperref[\detokenize{namelists/jules_vegetation.nml:JULES_VEGETATION::l_triffid}]{\sphinxcrossref{\sphinxcode{\sphinxupquote{l\_triffid}}}}}. When TRIFFID is on, competition between tiles is switched on with {\hyperref[\detokenize{namelists/jules_vegetation.nml:JULES_VEGETATION::l_veg_compete}]{\sphinxcrossref{\sphinxcode{\sphinxupquote{l\_veg\_compete}}}}} and the effect of nitrogen on vegetation growth is enabled via {\hyperref[\detokenize{namelists/jules_vegetation.nml:JULES_VEGETATION::l_nitrogen}]{\sphinxcrossref{\sphinxcode{\sphinxupquote{l\_nitrogen}}}}}.

\sphinxAtStartPar
The crop model, which is simulates phenology and carbon stocks in crops, can be switched on by setting the number of crop tiles {\hyperref[\detokenize{namelists/jules_surface_types.nml:JULES_SURFACE_TYPES::ncpft}]{\sphinxcrossref{\sphinxcode{\sphinxupquote{ncpft}}}}} to a non\sphinxhyphen{}zero value.

\sphinxAtStartPar
The crop model and TRIFFID cannot currently be used together. To simulate agricultural areas within TRIFFID, a fraction of the gridbox can be reserved for agricultural Plant Functional Types (PFTs) (as defined by {\hyperref[\detokenize{namelists/triffid_params.nml:JULES_TRIFFID::crop_io}]{\sphinxcrossref{\sphinxcode{\sphinxupquote{crop\_io}}}}} \textgreater{} 0). Agricultural PFT competition and a representation of harvest carbon can be switched on with {\hyperref[\detokenize{namelists/jules_vegetation.nml:JULES_VEGETATION::l_trif_crop}]{\sphinxcrossref{\sphinxcode{\sphinxupquote{l\_trif\_crop}}}}}.

\sphinxAtStartPar
If neither the phenology model nor the crop model are used, LAI for each vegetation tile can be set to a constant ({\hyperref[\detokenize{namelists/pft_params.nml:JULES_PFTPARM::lai_io}]{\sphinxcrossref{\sphinxcode{\sphinxupquote{lai\_io}}}}}) or a time series or seasonal cycle can be prescribed ({\hyperref[\detokenize{namelists/prescribed_data.nml:namelist-JULES_PRESCRIBED}]{\sphinxcrossref{\sphinxcode{\sphinxupquote{JULES\_PRESCRIBED}}}}}).

\sphinxAtStartPar
To simulate carbon and nitrogen stocks in the soil, the 4\sphinxhyphen{}pool model should be selected by setting {\hyperref[\detokenize{namelists/jules_soil_biogeochem.nml:JULES_SOIL_BIOGEOCHEM::soil_bgc_model}]{\sphinxcrossref{\sphinxcode{\sphinxupquote{soil\_bgc\_model}}}}} = 2. This option adds prognostic soil pools and must be used with the TRIFFID vegetation model. If TRIFFID is not used, prescribed soil pools must be invoked via {\hyperref[\detokenize{namelists/jules_soil_biogeochem.nml:JULES_SOIL_BIOGEOCHEM::soil_bgc_model}]{\sphinxcrossref{\sphinxcode{\sphinxupquote{soil\_bgc\_model}}}}} = 1. Layered soil pools are used if {\hyperref[\detokenize{namelists/jules_soil_biogeochem.nml:JULES_SOIL_BIOGEOCHEM::l_layeredc}]{\sphinxcrossref{\sphinxcode{\sphinxupquote{l\_layeredc}}}}} = .TRUE..

\sphinxAtStartPar
A multi\sphinxhyphen{}layer snow model can be selected using {\hyperref[\detokenize{namelists/jules_snow.nml:JULES_SNOW::nsmax}]{\sphinxcrossref{\sphinxcode{\sphinxupquote{nsmax}}}}}. Parameterisations of surface and subsurface runoff generation are controlled using {\hyperref[\detokenize{namelists/jules_hydrology.nml:JULES_HYDROLOGY::l_top}]{\sphinxcrossref{\sphinxcode{\sphinxupquote{l\_top}}}}} and {\hyperref[\detokenize{namelists/jules_hydrology.nml:JULES_HYDROLOGY::l_pdm}]{\sphinxcrossref{\sphinxcode{\sphinxupquote{l\_pdm}}}}}, while the routing of water in rivers uses {\hyperref[\detokenize{namelists/jules_rivers.nml:JULES_RIVERS::l_rivers}]{\sphinxcrossref{\sphinxcode{\sphinxupquote{l\_rivers}}}}}.

\sphinxstepscope


\chapter{Building and running JULES}
\label{\detokenize{building-and-running/intro:building-and-running-jules}}\label{\detokenize{building-and-running/intro::doc}}
\sphinxAtStartPar
This section details the options available for compiling and running JULES.

\sphinxstepscope


\section{Considerations}
\label{\detokenize{building-and-running/considerations:considerations}}\label{\detokenize{building-and-running/considerations::doc}}
\sphinxAtStartPar
Depending on your use case, there are two main things that you need to consider:


\subsection{Do I need NetCDF?}
\label{\detokenize{building-and-running/considerations:do-i-need-netcdf}}
\sphinxAtStartPar
\sphinxhref{http://www.unidata.ucar.edu/software/netcdf/}{NetCDF} is a data format (and associated software libraries) specifically designed for large\sphinxhyphen{}scale scientific data. It has two major benefits over raw binary data:
\begin{enumerate}
\sphinxsetlistlabels{\arabic}{enumi}{enumii}{}{.}%
\item {} 
\sphinxAtStartPar
It is machine\sphinxhyphen{}independent, so \sphinxhref{http://en.wikipedia.org/wiki/Endianness}{endianness} is not an issue when moving datasets between machines

\item {} 
\sphinxAtStartPar
It is self\sphinxhyphen{}describing, so as well as containing raw data, NetCDF files also contain metadata describing the data (e.g. variable names, units, origin). Many tools are capable of exploiting this metadata to simplify processing.

\end{enumerate}

\sphinxAtStartPar
JULES can be built with or without NetCDF, however \sphinxstyleemphasis{building JULES without NetCDF limits the functionality of JULES}. Without NetCDF, JULES will use a dummy NetCDF library which allows the program to build but provides no functionality. Any attempt to use NetCDF files as input with this option will result in a runtime error. All input files must be columnar ASCII, meaning that the user is restricted to running at a single point only. Output files will automatically use a columnar ASCII format with headers. File formats are discussed in more detail in {\hyperref[\detokenize{input/overview::doc}]{\sphinxcrossref{\DUrole{doc}{Input files for JULES}}}}.


\subsection{Do I need parallel processing?}
\label{\detokenize{building-and-running/considerations:do-i-need-parallel-processing}}
\begin{sphinxadmonition}{note}{Note:}
\sphinxAtStartPar
For running JULES at a single point, parallel processing provides no advantage. However, if JULES is already compiled with OpenMP or MPI enabled, it is still possible to run a single point by simply specifying the number of OpenMP threads and/or MPI tasks to be 1.
\end{sphinxadmonition}

\sphinxAtStartPar
JULES is capable of exploiting parallel processing techniques to reduce processing time for distributed/gridded simulations. There are two different methods JULES can use:
\begin{description}
\sphinxlineitem{OpenMP}
\sphinxAtStartPar
\sphinxhref{http://en.wikipedia.org/wiki/OpenMP}{OpenMP} is a form of compiler\sphinxhyphen{}assisted parallelisation that uses directives for shared\sphinxhyphen{}memory, loop\sphinxhyphen{}level parallelism across multiple cores on a machine (OpenMP is \sphinxstyleemphasis{not} capable of utilising a cluster of machines).

\sphinxAtStartPar
This form of parallelism is not as effective as MPI, but may provide some speedup and does not require a specially compiled NetCDF library.

\sphinxlineitem{MPI}
\sphinxAtStartPar
\sphinxhref{http://en.wikipedia.org/wiki/Message\_Passing\_Interface}{MPI (Message Passing Interface)} is a standardised message passing interface. MPI coordinates the running of multiple ‘tasks’ in parallel, potentially on several machines (or nodes), and provides mechanisms for these tasks to communicate with each other.

\sphinxAtStartPar
JULES takes advantage of the parallel I/O features available in \sphinxhref{http://www.hdfgroup.org/HDF5/}{HDF5} and \sphinxhref{http://www.unidata.ucar.edu/software/netcdf/}{NetCDF4}, which enable multiple MPI tasks to read from and write to the same NetCDF file(s) at the same time. These features must be explicitly enabled when NetCDF is compiled (see {\hyperref[\detokenize{building-and-running/required-software::doc}]{\sphinxcrossref{\DUrole{doc}{Required software}}}}).

\end{description}

\sphinxAtStartPar
It is also possible to use MPI and OpenMP together, where each MPI task has a number of OpenMP threads, however this is very advanced and beyond the scope of this document.

\sphinxstepscope


\section{Required software}
\label{\detokenize{building-and-running/required-software:required-software}}\label{\detokenize{building-and-running/required-software::doc}}
\sphinxAtStartPar
Building a JULES executable requires FCM and one of the supported Fortran compilers (see {\hyperref[\detokenize{building-and-running/fcm::doc}]{\sphinxcrossref{\DUrole{doc}{Building JULES using FCM}}}}). The Fortran 90 NetCDF interface library is required to use gridded data (i.e. data for more than a single location).

\sphinxAtStartPar
To be able to automatically upgrade namelists between JULES versions or use a GUI to configure JULES runs, Rose is required.

\sphinxAtStartPar
All of this software is freely available:
\begin{itemize}
\item {} 
\sphinxAtStartPar
GFortran, the GNU GCC Fortran compiler \sphinxhyphen{} \sphinxurl{http://www.gnu.org/software/gcc/fortran}

\item {} 
\sphinxAtStartPar
FCM \sphinxhyphen{} \sphinxurl{http://metomi.github.io/fcm/doc}

\item {} 
\sphinxAtStartPar
Rose \sphinxhyphen{} \sphinxurl{http://metomi.github.io/rose/doc/html/index.html}

\item {} 
\sphinxAtStartPar
NetCDF libraries \sphinxhyphen{} \sphinxurl{http://www.unidata.ucar.edu/software/netcdf}

\end{itemize}

\sphinxAtStartPar
JULES has only been tested on Linux but, given a suitable Fortran compiler, should run on any Unix\sphinxhyphen{}like system with minimal changes. The recommended way to attempt to run JULES on Windows is via the Linux compatability layer \sphinxhref{http://www.cygwin.com/}{Cygwin}, although this is untested.


\subsection{Building JULES with NetCDF}
\label{\detokenize{building-and-running/required-software:building-jules-with-netcdf}}
\sphinxAtStartPar
To build JULES with NetCDF, it must be told where to find the NetCDF library files. JULES needs two pieces of information \sphinxhyphen{} the directory containing the NetCDF archive files, \sphinxcode{\sphinxupquote{netcdf.a}} and \sphinxcode{\sphinxupquote{netcdff.a}} (the \sphinxstyleemphasis{‘NetCDF library path’}), and the directory containing the NetCDF Fortran 90 module file, \sphinxcode{\sphinxupquote{netcdf.mod}} (the \sphinxstyleemphasis{‘NetCDF include path’}). In a standard NetCDF install, these are often \sphinxcode{\sphinxupquote{/usr/lib}} and \sphinxcode{\sphinxupquote{/usr/include}} or \sphinxcode{\sphinxupquote{/usr/local/lib}} and \sphinxcode{\sphinxupquote{/usr/local/include}} respectively.

\sphinxAtStartPar
If the \sphinxcode{\sphinxupquote{nc\sphinxhyphen{}config}} program is installed on your system (run \sphinxcode{\sphinxupquote{which nc\sphinxhyphen{}config}} to find out), this can be used to determine values for the NetCDF library path (\sphinxcode{\sphinxupquote{nc\sphinxhyphen{}config \sphinxhyphen{}\sphinxhyphen{}flibs}}) and NetCDF include path (\sphinxcode{\sphinxupquote{nc\sphinxhyphen{}config \sphinxhyphen{}\sphinxhyphen{}includedir}}). When JULES is built with NetCDF, users can supply either ASCII or NetCDF input files, and all output will be NetCDF.


\subsection{Building and running JULES with MPI}
\label{\detokenize{building-and-running/required-software:building-and-running-jules-with-mpi}}
\begin{sphinxadmonition}{warning}{Warning:}
\sphinxAtStartPar
For advanced users only
\end{sphinxadmonition}

\sphinxAtStartPar
In order to build and run JULES with MPI, additional software is required:
\begin{enumerate}
\sphinxsetlistlabels{\arabic}{enumi}{enumii}{}{.}%
\item {} 
\sphinxAtStartPar
An implementation of MPI compiled using the same compiler you will be using to compile JULES. Several implementations of MPI are available, the most commonly used being \sphinxhref{http://www.mpich.org/}{MPICH2} and \sphinxhref{http://www.open-mpi.org/}{OpenMPI}.

\begin{sphinxadmonition}{note}{Note:}
\sphinxAtStartPar
The \sphinxcode{\sphinxupquote{bin}} directory of your MPI installation must be in your \sphinxcode{\sphinxupquote{\$PATH}}
\end{sphinxadmonition}

\item {} 
\sphinxAtStartPar
A version of HDF5/NetCDF4 compiled \sphinxstyleemphasis{with parallel I/O enabled}, using the MPI implementation installed above. This is \sphinxstyleemphasis{not} the default way to compile NetCDF, and must be explicitly enabled. More information on how to do this can be found on the \sphinxhref{http://www.unidata.ucar.edu/software/netcdf/docs/getting\_and\_building\_netcdf.html\#build\_parallel}{NetCDF website}.

\end{enumerate}

\sphinxstepscope


\section{Building JULES using FCM}
\label{\detokenize{building-and-running/fcm:building-jules-using-fcm}}\label{\detokenize{building-and-running/fcm::doc}}
\sphinxAtStartPar
FCM is a code management and build system developed by the Met Office with a particular focus on simplifying the process of building large Fortran programs. In this section, we will be using the build tool \sphinxhyphen{} FCM make.

\sphinxAtStartPar
As part of the build process, FCM make will analyse the dependencies of every Fortran file and automatically compile them in the correct order.

\sphinxAtStartPar
FCM make must be given a configuration file that it uses to determine how to build the source code. Extensive documentation on FCM make configuration files is \sphinxhref{http://metomi.github.io/fcm/doc/user\_guide/make.html}{available online}.

\sphinxAtStartPar
Help pages for the FCM make command itself (rather than the configuration file) can be accessed using the command:

\begin{sphinxVerbatim}[commandchars=\\\{\}]
fcm\PYG{+w}{ }\PYG{n+nb}{help}\PYG{+w}{ }make
\end{sphinxVerbatim}

\sphinxAtStartPar
The FCM configuration file for building JULES is \sphinxcode{\sphinxupquote{etc/fcm\sphinxhyphen{}make/make.cfg}}. This file uses the environment variables below to determine the settings to use when compiling JULES.

\sphinxAtStartPar
Running FCM make with this configuration file will create some files and directories in the specified build directory (see the \sphinxcode{\sphinxupquote{\sphinxhyphen{}C}} option of \sphinxcode{\sphinxupquote{fcm make}}; defaults to the current working directory). The JULES executable will be produced in the specified build directory at \sphinxcode{\sphinxupquote{build/bin/jules.exe}}.


\subsection{Environment variables used when building JULES using FCM make}
\label{\detokenize{building-and-running/fcm:environment-variables-used-when-building-jules-using-fcm-make}}\label{\detokenize{building-and-running/fcm:fcm-make-environment-variables}}\begin{description}
\sphinxlineitem{\sphinxcode{\sphinxupquote{JULES\_PLATFORM}}}
\sphinxAtStartPar
Used to select settings for a pre\sphinxhyphen{}defined platform. The default values of other variables may depend on the choice of this setting; differences from the generic defaults are included in the descriptions below.

\begin{sphinxadmonition}{note}{Note:}
\sphinxAtStartPar
If you have many users using the same platform to run JULES, you may want to contribute a suitable platform configuration.
\end{sphinxadmonition}


\begin{savenotes}\sphinxattablestart
\centering
\begin{tabulary}{\linewidth}[t]{|T|T|}
\hline
\sphinxstyletheadfamily 
\sphinxAtStartPar
Permitted value
&\sphinxstyletheadfamily 
\sphinxAtStartPar
Purpose
\\
\hline
\sphinxAtStartPar
\sphinxcode{\sphinxupquote{custom}}
&
\sphinxAtStartPar
\sphinxstylestrong{Default.} Use a custom configuration entirely determined by the other environment
variables. The default values of those variables are set in this platform’s configuration file.
\\
\hline
\sphinxAtStartPar
\sphinxcode{\sphinxupquote{vm}}
&
\sphinxAtStartPar
Use settings for the \sphinxtitleref{JULES development virtual machine}.
\\
\hline
\sphinxAtStartPar
\sphinxcode{\sphinxupquote{ceh}}
&
\sphinxAtStartPar
Use settings for the GFortran compiler on the CEH Linux systems.
\\
\hline
\sphinxAtStartPar
\sphinxcode{\sphinxupquote{jasmin\sphinxhyphen{}lotus\sphinxhyphen{}intel}}
&
\sphinxAtStartPar
Use settings for the Intel compiler on the Lotus system at JASMIN.
\\
\hline
\sphinxAtStartPar
\sphinxcode{\sphinxupquote{jasmin\sphinxhyphen{}gcc\sphinxhyphen{}nompi}}
&
\sphinxAtStartPar
Use settings for the gfortran compiler on the JASMIN Cylc server.
\\
\hline
\sphinxAtStartPar
\sphinxcode{\sphinxupquote{jasmin\sphinxhyphen{}intel\sphinxhyphen{}nompi}}
&
\sphinxAtStartPar
Use settings for the intel compiler on the JASMIN Cylc server.
\\
\hline
\sphinxAtStartPar
\sphinxcode{\sphinxupquote{meto\sphinxhyphen{}linux\sphinxhyphen{}gfortran}}
&
\sphinxAtStartPar
Use settings for the GFortran compiler on Met Office Linux systems.
\\
\hline
\sphinxAtStartPar
\sphinxcode{\sphinxupquote{meto\sphinxhyphen{}linux\sphinxhyphen{}nagfor}}
&
\sphinxAtStartPar
Use settings for the NAG compiler on Met Office Linux systems.

\sphinxAtStartPar
\sphinxstylestrong{Warning:} This build configuration is intended for correctness checking only, not production runs.
\\
\hline
\sphinxAtStartPar
\sphinxcode{\sphinxupquote{meto\sphinxhyphen{}linux\sphinxhyphen{}intel\sphinxhyphen{}nompi}}
&
\sphinxAtStartPar
Use settings for the Intel compiler \sphinxstyleemphasis{without} MPI on Met Office Linux systems.
\\
\hline
\sphinxAtStartPar
\sphinxcode{\sphinxupquote{meto\sphinxhyphen{}linux\sphinxhyphen{}intel\sphinxhyphen{}mpi}}
&
\sphinxAtStartPar
Use settings for the Intel compiler \sphinxstyleemphasis{with} MPI on Met Office Linux systems.
\\
\hline
\sphinxAtStartPar
\sphinxcode{\sphinxupquote{meto\sphinxhyphen{}xc40\sphinxhyphen{}cce}}
&
\sphinxAtStartPar
Use settings for the Cray Compiler Environment on the Met Office Cray XC40 system.
\\
\hline
\sphinxAtStartPar
\sphinxcode{\sphinxupquote{uoe\sphinxhyphen{}linux\sphinxhyphen{}gfortran}}
&
\sphinxAtStartPar
Use settings for the GFortran compiler on University of Exeter Linux system (SL7).
\\
\hline
\end{tabulary}
\par
\sphinxattableend\end{savenotes}

\sphinxlineitem{\sphinxcode{\sphinxupquote{JULES\_REMOTE}}, \sphinxcode{\sphinxupquote{JULES\_REMOTE\_HOST}}, \sphinxcode{\sphinxupquote{JULES\_REMOTE\_PATH}}}
\begin{sphinxadmonition}{warning}{Warning:}
\sphinxAtStartPar
Advanced users only
\end{sphinxadmonition}

\sphinxAtStartPar
Used to determine whether the build will happen on a local or remote machine.


\begin{savenotes}\sphinxattablestart
\centering
\begin{tabulary}{\linewidth}[t]{|T|T|}
\hline
\sphinxstyletheadfamily 
\sphinxAtStartPar
Permitted value
&\sphinxstyletheadfamily 
\sphinxAtStartPar
Purpose
\\
\hline
\sphinxAtStartPar
\sphinxcode{\sphinxupquote{local}}
&
\sphinxAtStartPar
\sphinxstylestrong{Default.} All compilation occurs on the local machine.
\\
\hline
\sphinxAtStartPar
\sphinxcode{\sphinxupquote{remote}}
&
\sphinxAtStartPar
Code is extracted on the local machine and mirrored to
\sphinxcode{\sphinxupquote{\$\{JULES\_REMOTE\_HOST\}@\$\{JULES\_REMOTE\_PATH\}}}, where \sphinxcode{\sphinxupquote{JULES\_REMOTE\_HOST}} is
the name of the remote machine and \sphinxcode{\sphinxupquote{JULES\_REMOTE\_PATH}} is the path on the
remote machine.

\sphinxAtStartPar
The compilation can then be completed on the remote machine. See below for an
example.
\\
\hline
\end{tabulary}
\par
\sphinxattableend\end{savenotes}

\sphinxlineitem{\sphinxcode{\sphinxupquote{JULES\_COMPILER}}}
\sphinxAtStartPar
Used to select compiler specific settings.


\begin{savenotes}\sphinxattablestart
\centering
\begin{tabulary}{\linewidth}[t]{|T|T|}
\hline
\sphinxstyletheadfamily 
\sphinxAtStartPar
Permitted value
&\sphinxstyletheadfamily 
\sphinxAtStartPar
Purpose
\\
\hline
\sphinxAtStartPar
\sphinxcode{\sphinxupquote{gfortran}}
&
\sphinxAtStartPar
\sphinxstylestrong{Default.} Use settings for the \sphinxhref{http://www.gnu.org/software/gcc/fortran/}{GNU Fortran compiler}.
\\
\hline
\sphinxAtStartPar
\sphinxcode{\sphinxupquote{intel}}
&
\sphinxAtStartPar
Use settings for the \sphinxhref{http://software.intel.com/en-us/articles/fortran-compilers/}{Intel Fortran compiler}.
\\
\hline
\sphinxAtStartPar
\sphinxcode{\sphinxupquote{nagfor}}
&
\sphinxAtStartPar
Use settings for the \sphinxhref{https://www.nag.co.uk/nag-compiler}{NAG Fortran compiler}.
\\
\hline
\sphinxAtStartPar
\sphinxcode{\sphinxupquote{cray}}
&
\sphinxAtStartPar
Use settings for the \sphinxhref{http://docs.cray.com/cgi-bin/craydoc.cgi?mode=SiteMap;f=xc\_sitemap}{Cray Compiler Environment}.
\\
\hline
\end{tabulary}
\par
\sphinxattableend\end{savenotes}

\sphinxlineitem{\sphinxcode{\sphinxupquote{JULES\_BUILD}}}
\sphinxAtStartPar
Used to select the type of build.


\begin{savenotes}\sphinxattablestart
\centering
\begin{tabulary}{\linewidth}[t]{|T|T|}
\hline
\sphinxstyletheadfamily 
\sphinxAtStartPar
Permitted value
&\sphinxstyletheadfamily 
\sphinxAtStartPar
Purpose
\\
\hline
\sphinxAtStartPar
\sphinxcode{\sphinxupquote{normal}}
&
\sphinxAtStartPar
\sphinxstylestrong{Default.} Compile JULES normally.
\\
\hline
\sphinxAtStartPar
\sphinxcode{\sphinxupquote{debug}}
&
\sphinxAtStartPar
Compile JULES with additional settings for debugging.
\\
\hline
\sphinxAtStartPar
\sphinxcode{\sphinxupquote{fast}}
&
\sphinxAtStartPar
Compile JULES with additional settings for faster execution.
\\
\hline
\end{tabulary}
\par
\sphinxattableend\end{savenotes}

\sphinxlineitem{\sphinxcode{\sphinxupquote{JULES\_OMP}}}
\sphinxAtStartPar
Used to determine whether to build with OpenMP or not.


\begin{savenotes}\sphinxattablestart
\centering
\begin{tabulary}{\linewidth}[t]{|T|T|}
\hline
\sphinxstyletheadfamily 
\sphinxAtStartPar
Permitted value
&\sphinxstyletheadfamily 
\sphinxAtStartPar
Purpose
\\
\hline
\sphinxAtStartPar
\sphinxcode{\sphinxupquote{noomp}}
&
\sphinxAtStartPar
\sphinxstylestrong{Default.} Compile JULES with OpenMP off.
\\
\hline
\sphinxAtStartPar
\sphinxcode{\sphinxupquote{omp}}
&
\sphinxAtStartPar
Compile JULES with OpenMP on.
\\
\hline
\end{tabulary}
\par
\sphinxattableend\end{savenotes}

\sphinxlineitem{\sphinxcode{\sphinxupquote{JULES\_MPI}}}
\sphinxAtStartPar
Used to determine whether to build with MPI enabled or not.


\begin{savenotes}\sphinxattablestart
\centering
\begin{tabulary}{\linewidth}[t]{|T|T|}
\hline
\sphinxstyletheadfamily 
\sphinxAtStartPar
Permitted value
&\sphinxstyletheadfamily 
\sphinxAtStartPar
Purpose
\\
\hline
\sphinxAtStartPar
\sphinxcode{\sphinxupquote{nompi}}
&
\sphinxAtStartPar
\sphinxstylestrong{Default.} Compile JULES without MPI support.
\\
\hline
\sphinxAtStartPar
\sphinxcode{\sphinxupquote{mpi}}
&
\sphinxAtStartPar
Compile JULES with MPI support.
\\
\hline
\end{tabulary}
\par
\sphinxattableend\end{savenotes}

\sphinxlineitem{\sphinxcode{\sphinxupquote{JULES\_NETCDF}}}
\sphinxAtStartPar
Indicates whether to use a dummy NetCDF library or a ‘real’ NetCDF library.


\begin{savenotes}\sphinxattablestart
\centering
\begin{tabular}[t]{|*{2}{\X{1}{2}|}}
\hline
\sphinxstyletheadfamily 
\sphinxAtStartPar
Permitted value
&\sphinxstyletheadfamily 
\sphinxAtStartPar
Purpose
\\
\hline
\sphinxAtStartPar
\sphinxcode{\sphinxupquote{nonetcdf}}
&
\sphinxAtStartPar
\sphinxstylestrong{Default.} Use a dummy NetCDF library.
\\
\hline
\sphinxAtStartPar
\sphinxcode{\sphinxupquote{netcdf}}
&
\sphinxAtStartPar
Use a ‘real’ NetCDF library.

\sphinxAtStartPar
The NetCDF installation to use is specified using one of:
\begin{itemize}
\item {} 
\sphinxAtStartPar
\sphinxcode{\sphinxupquote{JULES\_NETCDF\_PATH}}

\item {} 
\sphinxAtStartPar
\sphinxcode{\sphinxupquote{JULES\_NETCDF\_INC\_PATH}} and \sphinxcode{\sphinxupquote{JULES\_NETCDF\_LIB\_PATH}}

\end{itemize}
\\
\hline
\end{tabular}
\par
\sphinxattableend\end{savenotes}

\sphinxlineitem{\sphinxcode{\sphinxupquote{JULES\_NETCDF\_PATH}}}
\sphinxAtStartPar
Path to NetCDF installation.

\sphinxAtStartPar
This sets \sphinxcode{\sphinxupquote{JULES\_NETCDF\_INC\_PATH = \$JULES\_NETCDF\_PATH/include}} and \sphinxcode{\sphinxupquote{JULES\_NETCDF\_LIB\_PATH = \$JULES\_NETCDF\_PATH/lib}}. These can be overridden by setting the variables directly.

\sphinxlineitem{\sphinxcode{\sphinxupquote{JULES\_NETCDF\_INC\_PATH}}}
\sphinxAtStartPar
Path to NetCDF include directory (i.e. directory containing \sphinxcode{\sphinxupquote{netcdf.mod}}).

\sphinxlineitem{\sphinxcode{\sphinxupquote{JULES\_NETCDF\_LIB\_PATH}}}
\sphinxAtStartPar
Path to NetCDF library directory (i.e. directory containing \sphinxcode{\sphinxupquote{libnetcdff.a}} and \sphinxcode{\sphinxupquote{libnetcdf.a}}).

\end{description}

\begin{sphinxadmonition}{note}{Note:}
\sphinxAtStartPar
When compiled in parallel mode, NetCDF must be statically linked. This means the compiler must be able to find all required library and include files (i.e. for NetCDF, HDF5, curl and zlib) in \sphinxcode{\sphinxupquote{JULES\_NETCDF\_INC\_PATH}}, \sphinxcode{\sphinxupquote{JULES\_NETCDF\_LIB\_PATH}} or the default search path.
\end{sphinxadmonition}
\begin{description}
\sphinxlineitem{\sphinxcode{\sphinxupquote{JULES\_FFLAGS\_EXTRA}}}
\sphinxAtStartPar
Any additional compiler flags you wish to add to the build. For example, to activate additional compiler checks.

\sphinxlineitem{\sphinxcode{\sphinxupquote{JULES\_LDFLAGS\_EXTRA}}}
\sphinxAtStartPar
Any additional library flags you wish to add to the build. This may need to include both the linker flags themselves and, if you are linking in a new library, the flags specifying the path to the new library object.

\end{description}

\begin{sphinxadmonition}{note}{Note:}
\sphinxAtStartPar
When adding a completely new external dependency it is likely you will need to edit or override the FCM make build configuration files. The FCM make tool performs a dependency analysis on the JULES source tree to ensure all of the required files are present. Any new external sources must be added to the list of exclusions from this analysis or the build will fail when the external files cannot be found in the JULES working copy.
\end{sphinxadmonition}
\begin{description}
\sphinxlineitem{\sphinxcode{\sphinxupquote{JULES\_SOURCE}}}
\sphinxAtStartPar
The full path to the copy of JULES being compiled. This could be a directory path or an FCM/Subversion/file URL to a repository location. This variable is used by the configuration file contained in many Rose fcm\_make apps, but is not read by JULES itself.

\end{description}


\subsection{Example FCM make commands}
\label{\detokenize{building-and-running/fcm:example-fcm-make-commands}}
\sphinxAtStartPar
To create a normal JULES executable without NetCDF using the GFortran compiler (taking advantage of the default values for the environment variables):

\begin{sphinxVerbatim}[commandchars=\\\{\}]
\PYGZdl{}\PYG{+w}{ }fcm\PYG{+w}{ }make\PYG{+w}{ }\PYGZhy{}j\PYG{+w}{ }\PYG{l+m}{2}\PYG{+w}{ }\PYGZhy{}f\PYG{+w}{ }etc/fcm\PYGZhy{}make/make.cfg\PYG{+w}{ }\PYGZhy{}\PYGZhy{}new
\end{sphinxVerbatim}

\sphinxAtStartPar
To create a fast JULES executable with NetCDF using the Intel compiler:

\begin{sphinxVerbatim}[commandchars=\\\{\}]
\PYGZdl{}\PYG{+w}{ }\PYG{n+nb}{export}\PYG{+w}{ }\PYG{n+nv}{JULES\PYGZus{}COMPILER}\PYG{o}{=}intel
\PYGZdl{}\PYG{+w}{ }\PYG{n+nb}{export}\PYG{+w}{ }\PYG{n+nv}{JULES\PYGZus{}BUILD}\PYG{o}{=}fast
\PYGZdl{}\PYG{+w}{ }\PYG{n+nb}{export}\PYG{+w}{ }\PYG{n+nv}{JULES\PYGZus{}NETCDF}\PYG{o}{=}netcdf
\PYGZdl{}\PYG{+w}{ }\PYG{n+nb}{export}\PYG{+w}{ }\PYG{n+nv}{JULES\PYGZus{}NETCDF\PYGZus{}PATH}\PYG{o}{=}/path/to/netcdf\PYG{+w}{  }\PYG{c+c1}{\PYGZsh{} Replace this with the correct path}
\PYGZdl{}\PYG{+w}{ }fcm\PYG{+w}{ }make\PYG{+w}{ }\PYGZhy{}j\PYG{+w}{ }\PYG{l+m}{2}\PYG{+w}{ }\PYGZhy{}f\PYG{+w}{ }etc/fcm\PYGZhy{}make/make.cfg\PYG{+w}{ }\PYGZhy{}\PYGZhy{}new
\end{sphinxVerbatim}

\sphinxAtStartPar
To create a fast JULES executable with NetCDF using the GFortran compiler on a Met Office Linux system (making use of the platform setting):

\begin{sphinxVerbatim}[commandchars=\\\{\}]
\PYGZdl{}\PYG{+w}{ }\PYG{n+nb}{export}\PYG{+w}{ }\PYG{n+nv}{JULES\PYGZus{}PLATFORM}\PYG{o}{=}meto\PYGZhy{}linux\PYGZhy{}gfortran
\PYGZdl{}\PYG{+w}{ }\PYG{n+nb}{export}\PYG{+w}{ }\PYG{n+nv}{JULES\PYGZus{}BUILD}\PYG{o}{=}fast
\PYGZdl{}\PYG{+w}{ }\PYG{n+nb}{export}\PYG{+w}{ }\PYG{n+nv}{JULES\PYGZus{}NETCDF}\PYG{o}{=}netcdf\PYG{+w}{  }\PYG{c+c1}{\PYGZsh{} Note that we don\PYGZsq{}t need to specify paths}
\PYGZdl{}\PYG{+w}{ }fcm\PYG{+w}{ }make\PYG{+w}{ }\PYGZhy{}j\PYG{+w}{ }\PYG{l+m}{2}\PYG{+w}{ }\PYGZhy{}f\PYG{+w}{ }etc/fcm\PYGZhy{}make/make.cfg\PYG{+w}{ }\PYGZhy{}\PYGZhy{}new
\end{sphinxVerbatim}

\sphinxAtStartPar
To create a normal JULES executable with NetCDF and OpenMP using the Intel compiler on a remote machine:

\begin{sphinxVerbatim}[commandchars=\\\{\}]
localhost\PYG{+w}{ }\PYGZdl{}\PYG{+w}{ }\PYG{n+nb}{export}\PYG{+w}{ }\PYG{n+nv}{JULES\PYGZus{}REMOTE}\PYG{o}{=}remote
localhost\PYG{+w}{ }\PYGZdl{}\PYG{+w}{ }\PYG{n+nb}{export}\PYG{+w}{ }\PYG{n+nv}{JULES\PYGZus{}REMOTE\PYGZus{}HOST}\PYG{o}{=}my\PYGZhy{}host
localhost\PYG{+w}{ }\PYGZdl{}\PYG{+w}{ }\PYG{n+nb}{export}\PYG{+w}{ }\PYG{n+nv}{JULES\PYGZus{}REMOTE\PYGZus{}PATH}\PYG{o}{=}/path/on/remote/host
localhost\PYG{+w}{ }\PYGZdl{}\PYG{+w}{ }\PYG{n+nb}{export}\PYG{+w}{ }\PYG{n+nv}{JULES\PYGZus{}COMPILER}\PYG{o}{=}intel
localhost\PYG{+w}{ }\PYGZdl{}\PYG{+w}{ }\PYG{n+nb}{export}\PYG{+w}{ }\PYG{n+nv}{JULES\PYGZus{}OMP}\PYG{o}{=}omp
localhost\PYG{+w}{ }\PYGZdl{}\PYG{+w}{ }\PYG{n+nb}{export}\PYG{+w}{ }\PYG{n+nv}{JULES\PYGZus{}NETCDF}\PYG{o}{=}netcdf
localhost\PYG{+w}{ }\PYGZdl{}\PYG{+w}{ }\PYG{n+nb}{export}\PYG{+w}{ }\PYG{n+nv}{JULES\PYGZus{}NETCDF\PYGZus{}PATH}\PYG{o}{=}/path/to/netcdf\PYG{+w}{  }\PYG{c+c1}{\PYGZsh{} Replace this with the path ON THE REMOTE MACHINE}
localhost\PYG{+w}{ }\PYGZdl{}\PYG{+w}{ }fcm\PYG{+w}{ }make\PYG{+w}{ }\PYGZhy{}f\PYG{+w}{ }etc/fcm\PYGZhy{}make/make.cfg\PYG{+w}{ }\PYGZhy{}\PYGZhy{}new\PYG{+w}{  }\PYG{c+c1}{\PYGZsh{} This does the extract and mirror steps}
localhost\PYG{+w}{ }\PYGZdl{}\PYG{+w}{ }ssh\PYG{+w}{ }\PYGZhy{}Y\PYG{+w}{ }my\PYGZhy{}host
my\PYGZhy{}host\PYG{+w}{ }\PYGZdl{}\PYG{+w}{ }\PYG{n+nb}{cd}\PYG{+w}{ }/path/on/remote/host
my\PYGZhy{}host\PYG{+w}{ }\PYGZdl{}\PYG{+w}{ }fcm\PYG{+w}{ }make\PYG{+w}{ }\PYGZhy{}j\PYG{+w}{ }\PYG{l+m}{4}\PYG{+w}{ }\PYGZhy{}\PYGZhy{}new\PYG{+w}{  }\PYG{c+c1}{\PYGZsh{} This does the preprocess and build steps}
\end{sphinxVerbatim}

\sphinxAtStartPar
To create a normal JULES executable with MPI enabled, using the Intel compiler with array bounds checking turned on:

\begin{sphinxVerbatim}[commandchars=\\\{\}]
\PYGZdl{}\PYG{+w}{ }\PYG{n+nb}{export}\PYG{+w}{ }\PYG{n+nv}{JULES\PYGZus{}COMPILER}\PYG{o}{=}intel
\PYGZdl{}\PYG{+w}{ }\PYG{n+nb}{export}\PYG{+w}{ }\PYG{n+nv}{JULES\PYGZus{}MPI}\PYG{o}{=}mpi
\PYGZdl{}\PYG{+w}{ }\PYG{n+nb}{export}\PYG{+w}{ }\PYG{n+nv}{JULES\PYGZus{}NETCDF}\PYG{o}{=}netcdf\PYG{+w}{  }\PYG{c+c1}{\PYGZsh{} We have to use NetCDF for distributed simulations}
\PYGZdl{}\PYG{+w}{ }\PYG{n+nb}{export}\PYG{+w}{ }\PYG{n+nv}{JULES\PYGZus{}NETCDF\PYGZus{}PATH}\PYG{o}{=}/path/to/parallel/netcdf\PYG{+w}{  }\PYG{c+c1}{\PYGZsh{} NetCDF must be compiled with parallel I/O enabled}
\PYGZdl{}\PYG{+w}{ }\PYG{n+nb}{export}\PYG{+w}{ }\PYG{n+nv}{JULES\PYGZus{}FFLAGS\PYGZus{}EXTRA}\PYG{o}{=}\PYG{l+s+s2}{\PYGZdq{}\PYGZhy{}check bounds\PYGZdq{}}\PYG{+w}{  }\PYG{c+c1}{\PYGZsh{} Must be quoted because of the space}
\PYGZdl{}\PYG{+w}{ }fcm\PYG{+w}{ }make\PYG{+w}{ }\PYGZhy{}j\PYG{+w}{ }\PYG{l+m}{2}\PYG{+w}{ }\PYGZhy{}f\PYG{+w}{ }etc/fcm\PYGZhy{}make/make.cfg\PYG{+w}{ }\PYGZhy{}\PYGZhy{}new
\end{sphinxVerbatim}


\subsection{Tips for effective use of FCM make}
\label{\detokenize{building-and-running/fcm:tips-for-effective-use-of-fcm-make}}\begin{itemize}
\item {} 
\sphinxAtStartPar
To check the current values of the environment variables JULES will use to build, use the command \sphinxcode{\sphinxupquote{env | grep JULES}}

\item {} 
\sphinxAtStartPar
If you always use the same compilation options for JULES, consider adding the export lines to the \sphinxcode{\sphinxupquote{.profile}} file in your \sphinxcode{\sphinxupquote{\$HOME}} directory. Commands in the \sphinxcode{\sphinxupquote{.profile}} file are automatically executed in any shell that you open, so defining environment variables there ensures your build environment remains consistent across shells and restarts of your computer. The definitions can still be overridden on the command line if required.

\end{itemize}

\sphinxstepscope


\section{Running JULES}
\label{\detokenize{building-and-running/running-jules:running-jules}}\label{\detokenize{building-and-running/running-jules::doc}}
\sphinxAtStartPar
The user interface of JULES consists of several files with the extension \sphinxcode{\sphinxupquote{.nml}} containing Fortran namelists. These files and the namelist members are documented in more detail in {\hyperref[\detokenize{namelists/contents::doc}]{\sphinxcrossref{\DUrole{doc}{The JULES namelist files}}}}. These namelists are grouped together in a single directory. That directory is referred to as the \sphinxstyleemphasis{namelist directory} for a JULES run. In most use cases, this is practically abstracted away by the use of the rose/cylc workflow. This provides a GUI and rich ecosystem for integration of JULES into a larger workflow (eg compile\sphinxhyphen{}run\sphinxhyphen{}analyse).

\sphinxAtStartPar
Once a {\hyperref[\detokenize{building-and-running/fcm::doc}]{\sphinxcrossref{\DUrole{doc}{JULES executable is compiled}}}} and the {\hyperref[\detokenize{namelists/contents::doc}]{\sphinxcrossref{\DUrole{doc}{namelists}}}} are set up, JULES can be run in one of two ways:
\begin{enumerate}
\sphinxsetlistlabels{\arabic}{enumi}{enumii}{}{.}%
\item {} 
\sphinxAtStartPar
Run the JULES executable in the namelist directory with no arguments:

\begin{sphinxVerbatim}[commandchars=\\\{\}]
\PYG{n+nb}{cd}\PYG{+w}{ }/path/to/namelist/dir
/path/to/jules.exe
\end{sphinxVerbatim}

\item {} 
\sphinxAtStartPar
Run the JULES executable with the namelist directory as an argument:

\begin{sphinxVerbatim}[commandchars=\\\{\}]
/path/to/jules.exe\PYG{+w}{  }/path/to/namelist/dir
\end{sphinxVerbatim}

\end{enumerate}

\begin{sphinxadmonition}{warning}{Warning:}
\sphinxAtStartPar
Any relative paths given to JULES via the namelists (e.g. {\hyperref[\detokenize{namelists/ancillaries.nml:JULES_FRAC::file}]{\sphinxcrossref{\sphinxcode{\sphinxupquote{file}}}}} in {\hyperref[\detokenize{namelists/ancillaries.nml:namelist-JULES_FRAC}]{\sphinxcrossref{\sphinxcode{\sphinxupquote{JULES\_FRAC}}}}}) will be interpreted \sphinxstyleemphasis{relative to the current working directory}.

\sphinxAtStartPar
This means that if the user plans to use the second method to run JULES (e.g. in a batch environment), it is advisable to use fully\sphinxhyphen{}qualified path names for all files specified in the namelists.

\sphinxAtStartPar
To allow runs to be portable across different machines, it is common to specify data files relative to the namelist directory. In this case, JULES must be run using the first method to allow the relative paths to be resolved correctly.
\end{sphinxadmonition}


\subsection{General example of running JULES from the command line}
\label{\detokenize{building-and-running/running-jules:general-example-of-running-jules-from-the-command-line}}\begin{enumerate}
\sphinxsetlistlabels{\arabic}{enumi}{enumii}{}{.}%
\item {} 
\sphinxAtStartPar
Move into the JULES root directory (the directory containing \sphinxcode{\sphinxupquote{includes}}, \sphinxcode{\sphinxupquote{src}} etc.):

\begin{sphinxVerbatim}[commandchars=\\\{\}]
\PYGZdl{}\PYG{+w}{ }\PYG{n+nb}{cd}\PYG{+w}{ }/jules/root/dir
\end{sphinxVerbatim}

\item {} 
\sphinxAtStartPar
Build JULES:

\begin{sphinxVerbatim}[commandchars=\\\{\}]
\PYGZdl{}\PYG{+w}{ }fcm\PYG{+w}{ }make\PYG{+w}{ }\PYGZhy{}f\PYG{+w}{ }etc/fcm\PYGZhy{}make/make.cfg
\end{sphinxVerbatim}

\item {} 
\sphinxAtStartPar
Move into the namelist directory:

\begin{sphinxVerbatim}[commandchars=\\\{\}]
\PYGZdl{}\PYG{+w}{ }\PYG{n+nb}{cd}\PYG{+w}{ }/path/to/namelist/dir
\end{sphinxVerbatim}

\item {} 
\sphinxAtStartPar
Run the JULES executable:

\begin{sphinxVerbatim}[commandchars=\\\{\}]
\PYGZdl{}\PYG{+w}{ }/path/to/jules.exe
\end{sphinxVerbatim}

\end{enumerate}


\subsection{Running JULES with OpenMP}
\label{\detokenize{building-and-running/running-jules:running-jules-with-openmp}}
\sphinxAtStartPar
If JULES is compiled with OpenMP, then it must be told how many OpenMP threads to use. This is done using the environment variable \sphinxcode{\sphinxupquote{OMP\_NUM\_THREADS}}:

\begin{sphinxVerbatim}[commandchars=\\\{\}]
\PYGZdl{}\PYG{+w}{ }\PYG{n+nb}{export}\PYG{+w}{ }\PYG{n+nv}{OMP\PYGZus{}NUM\PYGZus{}THREADS}\PYG{o}{=}\PYG{l+m}{4}\PYG{+w}{  }\PYG{c+c1}{\PYGZsh{} Use 4 threads for OpenMP parallel regions}
\PYGZdl{}\PYG{+w}{ }/path/to/jules.exe
\end{sphinxVerbatim}


\subsection{Running JULES with MPI}
\label{\detokenize{building-and-running/running-jules:running-jules-with-mpi}}
\sphinxAtStartPar
When running JULES using MPI, JULES attempts to find a suitable decomposition of the grid depending on how many MPI tasks are made available to it. Each MPI task can then be thought of as its own independent version of JULES, with each task being responsible for a portion of the grid. Each task reads its portion of the input file(s), performs calculations on those points and outputs its portion of the output file(s). Tasks only communicate in order to read and write dump files \sphinxhyphen{} this ensures that dump files are consistent regardless of decomposition, i.e. a dump from any run (MPI or not; different numbers of MPI tasks), can be used to (re\sphinxhyphen{})start any other run and produce identical results, providing the overall model grids are the same.

\sphinxAtStartPar
None of the namelists or namelist members are parallel\sphinxhyphen{}specific \sphinxhyphen{} the same {\hyperref[\detokenize{namelists/contents::doc}]{\sphinxcrossref{\DUrole{doc}{JULES namelists}}}} can be used to run JULES with or without MPI, and the final results will be identical.

\sphinxAtStartPar
If JULES is compiled with MPI, then it must be run using commands from your MPI distribution (usually called \sphinxcode{\sphinxupquote{mpiexec}} and/or \sphinxcode{\sphinxupquote{mpirun}}):

\begin{sphinxVerbatim}[commandchars=\\\{\}]
\PYGZdl{}\PYG{+w}{ }mpirun\PYG{+w}{ }\PYGZhy{}n\PYG{+w}{ }\PYG{l+m}{4}\PYG{+w}{ }/path/to/jules.exe\PYG{+w}{  }\PYG{c+c1}{\PYGZsh{} Run JULES using 4 MPI tasks}
\end{sphinxVerbatim}

\sphinxAtStartPar
Detailed discussion of \sphinxcode{\sphinxupquote{mpiexec}}/\sphinxcode{\sphinxupquote{mpirun}} is beyond the scope of this document \sphinxhyphen{} please refer to the documentation for your chosen MPI distribution for the available options and features.

\sphinxstepscope


\section{Automatic upgrading and GUI using Rose}
\label{\detokenize{building-and-running/rose:automatic-upgrading-and-gui-using-rose}}\label{\detokenize{building-and-running/rose::doc}}
\sphinxAtStartPar
\sphinxhref{http://metomi.github.io/rose/doc/html/index.html}{Rose} is a collection of tools for managing the building and running of scientific applications.


\sphinxstrong{See also:}
\nopagebreak


\sphinxAtStartPar
Please familiarise yourself with the \sphinxhref{http://metomi.github.io/rose/doc/html/index.html}{Rose documentation} before continuing with this section.



\begin{sphinxadmonition}{note}{Note:}
\sphinxAtStartPar
This section assumes \sphinxhref{http://metomi.github.io/rose/doc/html/installation.html}{Rose is installed}.

\sphinxAtStartPar
We will not be using Rose Bush or Rosie, so those components need not be installed.

\sphinxAtStartPar
It is not necessary to install Cylc, but some functionality will not be available. This will be noted as we go.
\end{sphinxadmonition}

\sphinxAtStartPar
JULES uses Rose primarily to provide a graphical interface for configuring and running JULES, but also to allow automatic upgrading of JULES runs from one version to the next.

\sphinxAtStartPar
A Rose suite for JULES will normally contain two applications \sphinxhyphen{} an \sphinxcode{\sphinxupquote{fcm\_make}} application for building JULES and a \sphinxcode{\sphinxupquote{jules}} application for configuring the namelists and running JULES.


\subsection{Creating a Rose suite from existing namelists}
\label{\detokenize{building-and-running/rose:creating-a-rose-suite-from-existing-namelists}}
\sphinxAtStartPar
To enable users to quickly transition to Rose and the extra functionality it provides, a tool is distributed with JULES that can convert existing namelists to a Rose suite.

\sphinxAtStartPar
To convert vn3.4 namelists to a vn4.7 Rose suite, run the following command in the directory containing the namelists:

\begin{sphinxVerbatim}[commandchars=\\\{\}]
create\PYGZus{}rose\PYGZus{}app\PYG{+w}{ }vn3.4\PYG{+w}{ }vn4.7\PYG{+w}{ }namelist\PYGZus{}path\PYG{+w}{ }suite\PYGZus{}name\PYG{+w}{ }jules\PYGZus{}dir
\end{sphinxVerbatim}

\sphinxAtStartPar
Where \sphinxcode{\sphinxupquote{jules\_dir}} is the path to the root directory of the most recent JULES code release on your machine.

\sphinxAtStartPar
The \sphinxcode{\sphinxupquote{namelist\_path}} can be the full or a relative path.

\sphinxAtStartPar
This will create a directory called \sphinxcode{\sphinxupquote{suite\_name}} in \textasciitilde{}/roses/ directory which contains a fully functional Rose suite.

\sphinxAtStartPar
To convert namelists to a Rose suite without upgrading the version, just give the same version for both.


\subsection{Using Rose to upgrade existing namelists}
\label{\detokenize{building-and-running/rose:using-rose-to-upgrade-existing-namelists}}
\sphinxAtStartPar
It is not necessary to use Rose to configure and run JULES \sphinxhyphen{} Rose can be used just to upgrade existing namelists (at vn3.4 or later).

\sphinxAtStartPar
In order to use Rose to upgrade existing namelists from vn3.4 to vn4.0, just execute the following commands in the directory containing your namelists:

\begin{sphinxVerbatim}[commandchars=\\\{\}]
\PYG{c+c1}{\PYGZsh{} Creates a Rose suite at rose\PYGZhy{}suite}
\PYG{n+nv}{\PYGZdl{}JULES\PYGZus{}ROOT}/bin/create\PYGZus{}rose\PYGZus{}app\PYG{+w}{ }vn3.4\PYG{+w}{ }vn4.0

\PYG{c+c1}{\PYGZsh{} Remove the current namelists}
rm\PYG{+w}{ }\PYGZhy{}rf\PYG{+w}{ }*.nml

\PYG{c+c1}{\PYGZsh{} Use Rose to generate the new namelists}
rose\PYG{+w}{ }app\PYGZhy{}run\PYG{+w}{ }\PYGZhy{}i\PYG{+w}{ }\PYGZhy{}C\PYG{+w}{ }rose\PYGZhy{}suite/app/jules

\PYG{c+c1}{\PYGZsh{} Remove the Rose suite and other generated files}
rm\PYG{+w}{ }\PYGZhy{}rf\PYG{+w}{ }rose\PYGZhy{}suite\PYG{+w}{ }.rose\PYGZhy{}config\PYGZus{}processors\PYGZhy{}file.db\PYG{+w}{ }rose\PYGZhy{}app\PYGZhy{}run.conf
\end{sphinxVerbatim}


\subsection{Upgrading an existing JULES Rose suite}
\label{\detokenize{building-and-running/rose:upgrading-an-existing-jules-rose-suite}}
\sphinxAtStartPar
Upgrading an existing JULES Rose suite is even more simple than upgrading the namelist files directly. To see the versions it is possible to upgrade to, run the command:

\begin{sphinxVerbatim}[commandchars=\\\{\}]
\PYG{+w}{  }rose\PYG{+w}{ }app\PYGZhy{}upgrade\PYG{+w}{ }\PYGZhy{}M\PYG{+w}{ }\PYG{n+nv}{\PYGZdl{}JULES\PYGZus{}ROOT}/rose\PYGZhy{}meta\PYG{+w}{ }\PYGZhy{}C\PYG{+w}{ }/path/to/rose/suite/app/jules\PYG{+w}{ }\PYGZhy{}\PYGZhy{}all\PYGZhy{}versions
\PYG{o}{\PYGZam{}\PYGZam{}}\PYG{+w}{ }rose\PYG{+w}{ }macro\PYG{+w}{ }\PYGZhy{}\PYGZhy{}fix\PYG{+w}{ }\PYGZhy{}C\PYG{+w}{ }app/jules

\PYG{+w}{  }rose\PYG{+w}{ }app\PYGZhy{}upgrade\PYG{+w}{ }\PYGZhy{}M\PYG{+w}{ }\PYG{n+nv}{\PYGZdl{}JULES\PYGZus{}ROOT}/rose\PYGZhy{}meta\PYG{+w}{ }\PYGZhy{}C\PYG{+w}{ }/path/to/rose/suite/app/fcm\PYGZus{}make\PYG{+w}{ }\PYGZhy{}\PYGZhy{}all\PYGZhy{}versions
\PYG{o}{\PYGZam{}\PYGZam{}}\PYG{+w}{ }rose\PYG{+w}{ }macro\PYG{+w}{ }\PYGZhy{}\PYGZhy{}fix\PYG{+w}{ }\PYGZhy{}C\PYG{+w}{ }app/fcm\PYGZus{}make
\end{sphinxVerbatim}

\sphinxAtStartPar
To then upgrade to one of those versions, the command is:

\begin{sphinxVerbatim}[commandchars=\\\{\}]
\PYG{+w}{  }rose\PYG{+w}{ }app\PYGZhy{}upgrade\PYG{+w}{ }\PYGZhy{}M\PYG{+w}{ }\PYG{n+nv}{\PYGZdl{}JULES\PYGZus{}ROOT}/rose\PYGZhy{}meta\PYG{+w}{ }\PYGZhy{}C\PYG{+w}{ }/path/to/rose/suite/app/jules\PYG{+w}{ }\PYGZlt{}version\PYGZgt{}
\PYG{o}{\PYGZam{}\PYGZam{}}\PYG{+w}{ }rose\PYG{+w}{ }macro\PYG{+w}{ }\PYGZhy{}\PYGZhy{}fix\PYG{+w}{ }\PYGZhy{}C\PYG{+w}{ }app/jules

\PYG{+w}{  }rose\PYG{+w}{ }app\PYGZhy{}upgrade\PYG{+w}{ }\PYGZhy{}M\PYG{+w}{ }\PYG{n+nv}{\PYGZdl{}JULES\PYGZus{}ROOT}/rose\PYGZhy{}meta\PYG{+w}{ }\PYGZhy{}C\PYG{+w}{ }/path/to/rose/suite/app/fcm\PYGZus{}make\PYG{+w}{ }\PYGZlt{}version\PYGZgt{}
\PYG{o}{\PYGZam{}\PYGZam{}}\PYG{+w}{ }rose\PYG{+w}{ }macro\PYG{+w}{ }\PYGZhy{}\PYGZhy{}fix\PYG{+w}{ }\PYGZhy{}C\PYG{+w}{ }app/fcm\PYGZus{}make
\end{sphinxVerbatim}


\subsection{Configuring JULES with a graphical interface}
\label{\detokenize{building-and-running/rose:configuring-jules-with-a-graphical-interface}}
\sphinxAtStartPar
Using a Rose suite to run JULES has the advantage that it can be configured graphically using \sphinxhref{http://metomi.github.io/rose/doc/html/api/command-reference.html\#rose-config-edit}{Rose Config Edit}.

\sphinxAtStartPar
To launch the graphical editor, the following command is used:

\begin{sphinxVerbatim}[commandchars=\\\{\}]
\PYG{c+c1}{\PYGZsh{} To edit the whole suite, including build configuration}
rose\PYG{+w}{ }config\PYGZhy{}edit\PYG{+w}{ }\PYGZhy{}M\PYG{+w}{ }\PYG{n+nv}{\PYGZdl{}JULES\PYGZus{}ROOT}/rose\PYGZhy{}meta\PYG{+w}{ }\PYGZhy{}C\PYG{+w}{ }/path/to/rose/suite\PYG{+w}{ }\PYG{p}{\PYGZam{}}

\PYG{c+c1}{\PYGZsh{} To edit just the namelists}
rose\PYG{+w}{ }config\PYGZhy{}edit\PYG{+w}{ }\PYGZhy{}M\PYG{+w}{ }\PYG{n+nv}{\PYGZdl{}JULES\PYGZus{}ROOT}/rose\PYGZhy{}meta\PYG{+w}{ }\PYGZhy{}C\PYG{+w}{ }/path/to/rose/suite/app/jules\PYG{+w}{ }\PYG{p}{\PYGZam{}}
\end{sphinxVerbatim}

\sphinxAtStartPar
where \sphinxcode{\sphinxupquote{\$JULES\_ROOT}} is the root directory of your JULES installation. For more information on using the config editor, see \sphinxhref{http://metomi.github.io/rose/doc/html/api/command-reference.html\#rose-config-edit}{the Rose documentation}

\sphinxAtStartPar
Clicking on a variable name in the editor opens the corresponding page in this documentation.


\subsection{Running a JULES Rose suite}
\label{\detokenize{building-and-running/rose:running-a-jules-rose-suite}}

\subsubsection{Without Cylc}
\label{\detokenize{building-and-running/rose:without-cylc}}
\sphinxAtStartPar
To run JULES from a Rose suite without Cylc, we just use Rose to generate the namelists. JULES is then built and run as normal \sphinxhyphen{} see {\hyperref[\detokenize{building-and-running/intro::doc}]{\sphinxcrossref{\DUrole{doc}{Building and running JULES}}}}.

\sphinxAtStartPar
To generate namelists in the current directory from a Rose suite at \sphinxcode{\sphinxupquote{/path/to/rose/suite}}, use the following command:

\begin{sphinxVerbatim}[commandchars=\\\{\}]
rose\PYG{+w}{ }app\PYGZhy{}run\PYG{+w}{ }\PYGZhy{}i\PYG{+w}{ }\PYGZhy{}C\PYG{+w}{ }/path/to/rose/suite/app/jules
\end{sphinxVerbatim}


\subsubsection{With Cylc}
\label{\detokenize{building-and-running/rose:with-cylc}}
\begin{sphinxadmonition}{warning}{Warning:}
\sphinxAtStartPar
This requires Cylc to be installed and configured.
\end{sphinxadmonition}

\sphinxAtStartPar
Once a JULES Rose suite has been suitably configured using the graphical editor, it can be run using the following command:

\begin{sphinxVerbatim}[commandchars=\\\{\}]
rose\PYG{+w}{ }suite\PYGZhy{}run\PYG{+w}{ }\PYGZhy{}C\PYG{+w}{ }/path/to/rose/suite
\end{sphinxVerbatim}

\sphinxAtStartPar
This will set the suite running, and will launch the \sphinxhref{http://cylc.github.io/cylc/}{Cylc} GUI to allow you to see the status of your suite as it runs. The GUI also allows you to view log files etc. \sphinxhyphen{} these can be useful when a job fails!

\sphinxstepscope


\chapter{Input files for JULES}
\label{\detokenize{input/overview:input-files-for-jules}}\label{\detokenize{input/overview::doc}}
\sphinxAtStartPar
The recommended file format for use with JULES is NetCDF, although an ASCII format is also supported for data at a single location only. NetCDF is recommended since in this format, the metadata are provided in a standardised manner that many other tools and applications can interpret. The file handling code of JULES is written in a modular way that aims to make it easy for the user to add support for other file formats if they desire. Any user that does this is strongly encouraged to contribute their code back to the community.

\sphinxstepscope


\section{General principles}
\label{\detokenize{input/principles:general-principles}}\label{\detokenize{input/principles::doc}}
\sphinxAtStartPar
JULES supports the input and output of gridded data on both 1D (e.g. vector of land points) and 2D (e.g. latitude/longitude) grids, with zero or more additional ‘levels’ dimensions (e.g. for soil layers). A 2D grid is the usual way to think about gridded data, i.e. with x and y dimensions; however a 1D grid can be more flexible and space\sphinxhyphen{}efficient. An example of a 1D grid is a land\sphinxhyphen{}points\sphinxhyphen{}only grid (as used in the GSWP2 and WATCH datasets). In this case, these data are supplied as a vector of land points, which avoids storing information about sea and sea\sphinxhyphen{}ice points that are not being processed.

\sphinxAtStartPar
In JULES, the input grid is comprised of the following information:
\begin{enumerate}
\sphinxsetlistlabels{\arabic}{enumi}{enumii}{}{.}%
\item {} 
\sphinxAtStartPar
Whether the grid is 1D or 2D (ASCII or NetCDF).

\item {} 
\sphinxAtStartPar
The size of each grid dimension (ASCII or NetCDF).

\item {} 
\sphinxAtStartPar
The name of each dimension in the file(s) (NetCDF only).

\end{enumerate}

\sphinxAtStartPar
The input grid is specified by the user in the namelist {\hyperref[\detokenize{namelists/model_grid.nml:namelist-JULES_INPUT_GRID}]{\sphinxcrossref{\sphinxcode{\sphinxupquote{JULES\_INPUT\_GRID}}}}}. The model grid is then constructed by selecting the desired points from this input grid, the default being that only the land points in the input grid will be processed. All output is on the model grid.

\begin{sphinxadmonition}{note}{Note:}
\sphinxAtStartPar
All input data must use the same grid, including any ancillaries and initial conditions.
\end{sphinxadmonition}

\sphinxAtStartPar
JULES infers the format of input files from the file extension. The recognised file extensions are:
\begin{description}
\sphinxlineitem{ASCII files}
\sphinxAtStartPar
\sphinxcode{\sphinxupquote{.asc}}, \sphinxcode{\sphinxupquote{.txt}} and \sphinxcode{\sphinxupquote{.dat}}

\sphinxlineitem{NetCDF files}
\sphinxAtStartPar
\sphinxcode{\sphinxupquote{.nc}} and \sphinxcode{\sphinxupquote{.cdf}}

\end{description}

\sphinxstepscope


\section{ASCII files}
\label{\detokenize{input/ascii:ascii-files}}\label{\detokenize{input/ascii::doc}}
\sphinxAtStartPar
JULES only supports the use of ASCII files for data at a single location. In this case, the input grid can be specified either as a 1D grid with length 1 or as a 2D grid of size 1 x 1. The data should be laid out in columns with one timestep of data per row (with time increasing with the number of rows). For variables with additional ‘levels’ dimensions (e.g. soil layers), the values for each level should be in consecutive columns.

\begin{sphinxadmonition}{note}{Note:}
\sphinxAtStartPar
Variables should be given to JULES in the order they appear in the file, and there should be no unused variables in between. This may mean that some datasets may require pre\sphinxhyphen{}processing for use with JULES, even if they are already columnar.
\end{sphinxadmonition}

\sphinxAtStartPar
If the first character of a line is either \sphinxcode{\sphinxupquote{\#}} or \sphinxcode{\sphinxupquote{!}}, the line is taken to be a comment. JULES reads no information from comments \sphinxhyphen{} they are purely for annotating the dataset for users.


\subsection{Example ASCII input}
\label{\detokenize{input/ascii:example-ascii-input}}

\subsubsection{ASCII meteorological forcing data}
\label{\detokenize{input/ascii:ascii-meteorological-forcing-data}}
\begin{sphinxVerbatim}[commandchars=\\\{\}]
\PYG{c+c1}{\PYGZsh{} Meteorological data for Loobos, 1997.}
\PYG{c+c1}{\PYGZsh{} One year of 30 minute data.}
\PYG{c+c1}{\PYGZsh{}   Down  Down      Rainfall      Snowfall     Air       Wind                 Specific}
\PYG{c+c1}{\PYGZsh{}   SWR   LWR        rate           rate      temp.      speed     Pressure   humidity}
\PYG{c+c1}{\PYGZsh{}(W m\PYGZhy{}2) (W m\PYGZhy{}2)  (kg m\PYGZhy{}2 s\PYGZhy{}1) (kg m\PYGZhy{}2 s\PYGZhy{}1)   (K)       (m s\PYGZhy{}1) )   (Pa)      (kg kg\PYGZhy{}1)}
    \PYG{l+m+mf}{0.0}  \PYG{l+m+mf}{187.8}     \PYG{l+m+mf}{0.000E+00}     \PYG{l+m+mf}{0.000E+00}   \PYG{l+m+mf}{259.800}     \PYG{l+m+mf}{2.017}     \PYG{l+m+mf}{102400.0}   \PYG{l+m+mf}{1.384E\PYGZhy{}03}
    \PYG{l+m+mf}{0.0}  \PYG{l+m+mf}{186.9}     \PYG{l+m+mf}{0.000E+00}     \PYG{l+m+mf}{0.000E+00}   \PYG{l+m+mf}{259.700}     \PYG{l+m+mf}{3.770}     \PYG{l+m+mf}{102400.0}   \PYG{l+m+mf}{1.384E\PYGZhy{}03}
    \PYG{l+m+mf}{0.0}  \PYG{l+m+mf}{186.7}     \PYG{l+m+mf}{0.000E+00}     \PYG{l+m+mf}{0.000E+00}   \PYG{l+m+mf}{259.600}     \PYG{l+m+mf}{4.290}     \PYG{l+m+mf}{102400.0}   \PYG{l+m+mf}{1.373E\PYGZhy{}03}
\PYG{c+c1}{\PYGZsh{} ...}
\end{sphinxVerbatim}

\sphinxAtStartPar
Each row represents a timestep of data. Each column represents a variable. Driving variables have no additional dimension.


\subsubsection{Initial conditions}
\label{\detokenize{input/ascii:initial-conditions}}
\begin{sphinxVerbatim}[commandchars=\\\{\}]
\PYG{c+c1}{\PYGZsh{} sthuf(1:sm\PYGZus{}levels)            t\PYGZus{}soil(1:sm\PYGZus{}levels)}
  \PYG{l+m+mf}{0.749}  \PYG{l+m+mf}{0.743}  \PYG{l+m+mf}{0.754}  \PYG{l+m+mf}{0.759}    \PYG{l+m+mf}{276.78}  \PYG{l+m+mf}{277.46}  \PYG{l+m+mf}{278.99}  \PYG{l+m+mf}{282.48}
\end{sphinxVerbatim}

\sphinxAtStartPar
Although only one ‘timestep’ of data is supplied, the data must still be laid out in columns. These variables have a value for each soil layer, which are given in consecutive columns. This quickly becomes cumbersome for large numbers of variables, which is why NetCDF is recommended even
for data at a single point.


\subsubsection{Time varying data with an additional dimension}
\label{\detokenize{input/ascii:time-varying-data-with-an-additional-dimension}}
\begin{sphinxVerbatim}[commandchars=\\\{\}]
\PYG{c+c1}{\PYGZsh{} lai(1:npft)                canht(1:npft)}
  \PYG{l+m+mf}{0.0}  \PYG{l+m+mf}{0.0}  \PYG{l+m+mf}{0.2}  \PYG{l+m+mf}{0.0}  \PYG{l+m+mf}{0.0}    \PYG{l+m+mf}{0.0}  \PYG{l+m+mf}{0.0}  \PYG{l+m+mf}{0.6}  \PYG{l+m+mf}{0.0}  \PYG{l+m+mf}{0.0}
  \PYG{l+m+mf}{0.0}  \PYG{l+m+mf}{0.0}  \PYG{l+m+mf}{0.2}  \PYG{l+m+mf}{0.0}  \PYG{l+m+mf}{0.0}    \PYG{l+m+mf}{0.0}  \PYG{l+m+mf}{0.0}  \PYG{l+m+mf}{0.6}  \PYG{l+m+mf}{0.0}  \PYG{l+m+mf}{0.0}
  \PYG{l+m+mf}{0.0}  \PYG{l+m+mf}{0.0}  \PYG{l+m+mf}{0.2}  \PYG{l+m+mf}{0.0}  \PYG{l+m+mf}{0.0}    \PYG{l+m+mf}{0.0}  \PYG{l+m+mf}{0.0}  \PYG{l+m+mf}{0.7}  \PYG{l+m+mf}{0.0}  \PYG{l+m+mf}{0.0}
\PYG{c+c1}{\PYGZsh{} ...}
\end{sphinxVerbatim}

\sphinxAtStartPar
These variables have one value for each plant functional type (see {\hyperref[\detokenize{overview/intro::doc}]{\sphinxcrossref{\DUrole{doc}{Overview of JULES}}}}). For each variable, the values for each pft are in consecutive columns. Each row is one timestep of data.

\sphinxstepscope


\section{NetCDF files}
\label{\detokenize{input/netcdf:netcdf-files}}\label{\detokenize{input/netcdf::doc}}
\sphinxAtStartPar
For gridded data, NetCDF is the only supported format. Although ASCII files can be used for data at a single location, NetCDF is also the preferred format for such data (due to the reasons discussed in {\hyperref[\detokenize{input/overview::doc}]{\sphinxcrossref{\DUrole{doc}{Input files for JULES}}}}). Files are not expected to use specific dimension or variable names \sphinxhyphen{} these are specified via the {\hyperref[\detokenize{namelists/contents::doc}]{\sphinxcrossref{\DUrole{doc}{JULES namelists}}}}. The only expectations placed on NetCDF input are:
\begin{itemize}
\item {} 
\sphinxAtStartPar
All input files use the same grid.

\item {} 
\sphinxAtStartPar
All input files use the same dimension names (for grid dimensions, any additional dimensions and the time dimension).

\item {} 
\sphinxAtStartPar
The dimensions for each variable appear in the correct order \sphinxhyphen{} \sphinxcode{\sphinxupquote{(points, z1, z2, ..., t)}} for a 1D grid and \sphinxcode{\sphinxupquote{(x, y, z1, z2, ..., t)}} for a 2D grid, where the \sphinxcode{\sphinxupquote{z1, z2, ...}} (levels) and \sphinxcode{\sphinxupquote{t}} (time) dimensions are only present when the variable and context in which the variable is being used require them.

\item {} 
\sphinxAtStartPar
If using NetCDF for data at a single location, the grid dimensions are still expected to exist with size 1.

\end{itemize}

\sphinxstepscope


\section{File name templating}
\label{\detokenize{input/file-name-templating:file-name-templating}}\label{\detokenize{input/file-name-templating::doc}}
\sphinxAtStartPar
If the names of input files follow particular patterns, JULES can use a substitution template rather than requiring a potentially long list of file names. Templating comes in two forms, time templating and variable name templating, which can be used separately or together.

\sphinxAtStartPar
Substitution strings are 3\sphinxhyphen{}character strings, starting with \sphinxcode{\sphinxupquote{\%}}. JULES will automatically detect the use of either form of templating by checking for the presence of the substitution strings in file names.


\subsection{Time templating}
\label{\detokenize{input/file-name-templating:time-templating}}
\sphinxAtStartPar
If any of the time templating substitution strings are present in a file name, then JULES assumes time\sphinxhyphen{}templating is to be used. The valid substitution strings for time templating are:


\begin{savenotes}\sphinxattablestart
\centering
\begin{tabulary}{\linewidth}[t]{|T|T|}
\hline
\sphinxstyletheadfamily 
\sphinxAtStartPar
Substitution string
&\sphinxstyletheadfamily 
\sphinxAtStartPar
Replaced with
\\
\hline
\sphinxAtStartPar
\sphinxcode{\sphinxupquote{\%y4}}
&
\sphinxAtStartPar
4\sphinxhyphen{}digit year
\\
\hline
\sphinxAtStartPar
\sphinxcode{\sphinxupquote{\%y2}}
&
\sphinxAtStartPar
2\sphinxhyphen{}digit year
\\
\hline
\sphinxAtStartPar
\sphinxcode{\sphinxupquote{\%m2}}
&
\sphinxAtStartPar
2\sphinxhyphen{}digit month
\\
\hline
\sphinxAtStartPar
\sphinxcode{\sphinxupquote{\%m1}}
&
\sphinxAtStartPar
1\sphinxhyphen{} or 2\sphinxhyphen{}digit month
\\
\hline
\sphinxAtStartPar
\sphinxcode{\sphinxupquote{\%mc}}
&
\sphinxAtStartPar
3\sphinxhyphen{}character month abbreviation
\\
\hline
\sphinxAtStartPar
\sphinxcode{\sphinxupquote{\%d2}}
&
\sphinxAtStartPar
2\sphinxhyphen{}digit day of month
\\
\hline
\end{tabulary}
\par
\sphinxattableend\end{savenotes}

\sphinxAtStartPar
JULES will automatically detect the period (or frequency) of files based on the specific substitution strings in the following manner:

\begin{figure}[htbp]
\centering
\capstart

\noindent\sphinxincludegraphics{{time_templating_flow_diagram}.png}
\caption{Flow diagram showing detection of file period from time templated string}\label{\detokenize{input/file-name-templating:id1}}\end{figure}

\sphinxAtStartPar
This means that monthly files must also have a year substitution string present, and daily files must have both month and year substitution strings present. Only yearly, monthly and daily files are allowed with time templating, with each file containing a single period (year, month or day respectively) of data. For yearly files, the first data in each file must apply from 00:00:00 on 1st January for each year. For monthly files, the first data in the file must apply from 00:00:00 on the 1st of the month. For daily files, the first data in the file must apply from 00:00:00 on the given day. Other configurations can be specified using a list of files with their respective start times.


\subsection{Variable name templating}
\label{\detokenize{input/file-name-templating:variable-name-templating}}
\sphinxAtStartPar
Variable name templating can be used when related variables are stored in separate files with file names that are identical apart from a section that indicates what variable is in each file. Examples of the use of this are given in the next section. JULES will automatically detect if the variable name substitution string \sphinxhyphen{} \sphinxcode{\sphinxupquote{\%vv}} \sphinxhyphen{} is present in a file name, and apply variable name templating if appropriate.


\subsection{Examples of file name templating}
\label{\detokenize{input/file-name-templating:examples-of-file-name-templating}}

\subsubsection{Time templating only}
\label{\detokenize{input/file-name-templating:time-templating-only}}
\sphinxAtStartPar
Data is in monthly files with all related variables in the same file.

\sphinxAtStartPar
Template:

\begin{sphinxVerbatim}[commandchars=\\\{\}]
/data/met\PYGZus{}data\PYGZus{}\PYGZpc{}y4\PYGZpc{}m2.nc
\end{sphinxVerbatim}

\sphinxAtStartPar
Example filenames:

\begin{sphinxVerbatim}[commandchars=\\\{\}]
/data/met\PYGZus{}data\PYGZus{}199001.nc
/data/met\PYGZus{}data\PYGZus{}199002.nc
...
/data/met\PYGZus{}data\PYGZus{}200410.nc
\end{sphinxVerbatim}


\subsubsection{Variable name templating only}
\label{\detokenize{input/file-name-templating:variable-name-templating-only}}
\sphinxAtStartPar
Ancillary (non\sphinxhyphen{}time\sphinxhyphen{}varying) data with each variable in similarly named but separate files.

\sphinxAtStartPar
Template:

\begin{sphinxVerbatim}[commandchars=\\\{\}]
/ancil/soil\PYGZus{}\PYGZpc{}vv.nc
\end{sphinxVerbatim}

\sphinxAtStartPar
Example filenames:

\begin{sphinxVerbatim}[commandchars=\\\{\}]
/data/soil\PYGZus{}satcon.nc
/data/soil\PYGZus{}sathh.nc
\end{sphinxVerbatim}


\subsubsection{Time and variable name templating together}
\label{\detokenize{input/file-name-templating:time-and-variable-name-templating-together}}
\sphinxAtStartPar
Data is in monthly files with each variable in similarly named but separate files.

\sphinxAtStartPar
Template:

\begin{sphinxVerbatim}[commandchars=\\\{\}]
/data/\PYGZpc{}vv\PYGZus{}\PYGZpc{}y4\PYGZpc{}mc.nc
\end{sphinxVerbatim}

\sphinxAtStartPar
Example filenames:

\begin{sphinxVerbatim}[commandchars=\\\{\}]
/data/Rain\PYGZus{}1990jan.nc
/data/Wind\PYGZus{}1990jan.nc
...
/data/Rain\PYGZus{}2000oct.nc
/data/Wind\PYGZus{}2000oct.nc
\end{sphinxVerbatim}


\subsubsection{Variable name templating with a list of files}
\label{\detokenize{input/file-name-templating:variable-name-templating-with-a-list-of-files}}
\sphinxAtStartPar
Data in 6\sphinxhyphen{}monthly files with each variable in similarly named but separate files.

\sphinxAtStartPar
Since the time templating cannot handle 6\sphinxhyphen{}monthly files, the files and their start times must be specified as a list. However, variable name templating can still be used.

\sphinxAtStartPar
Also note that it is possible to use a substitution string more than once in a template.

\sphinxAtStartPar
Template list:

\begin{sphinxVerbatim}[commandchars=\\\{\}]
./\PYGZpc{}vv/met\PYGZus{}\PYGZpc{}vv\PYGZus{}199001.nc
./\PYGZpc{}vv/met\PYGZus{}\PYGZpc{}vv\PYGZus{}199007.nc
...
./\PYGZpc{}vv/met\PYGZus{}\PYGZpc{}vv\PYGZus{}199801.nc
\end{sphinxVerbatim}

\sphinxAtStartPar
Example filenames:

\begin{sphinxVerbatim}[commandchars=\\\{\}]
./Rain/met\PYGZus{}Rain\PYGZus{}199001.nc
./Wind/met\PYGZus{}Wind\PYGZus{}199001.nc
./Rain/met\PYGZus{}Rain\PYGZus{}199007.nc
./Wind/met\PYGZus{}Wind\PYGZus{}199007.nc
...
./Rain/met\PYGZus{}Rain\PYGZus{}199801.nc
./Wind/met\PYGZus{}Wind\PYGZus{}199801.nc
\end{sphinxVerbatim}

\sphinxstepscope


\section{Temporal interpolation}
\label{\detokenize{input/temporal-interpolation:temporal-interpolation}}\label{\detokenize{input/temporal-interpolation::doc}}
\sphinxAtStartPar
Time\sphinxhyphen{}varying data as inputs into JULES are provided in two types \sphinxhyphen{} instantaneous states (e.g. air temperature, surface pressure, lai) or fluxes (e.g. radiation, precipitation). Because the data are on discrete timesteps, the value of an instantaneous variable applies at the timestamp (e.g. air temperature at 0800). However, values of the fluxes represent averages over the data timestep (e.g. 3\sphinxhyphen{}hour average rates). Different datasets supply the data as averages over the previous data timestep (backwards average) or the next data timestep (forwards average).

\sphinxAtStartPar
In order for the numerics to remain stable, it is recommended to run JULES with a model timestep of 1 hour or shorter. If the data timestep is longer than the model timestep, interpolation is required. How interpolation is performed for a particular variable depends on whether the variable is an instantaneous state or a flux.


\subsection{Interpolation flags}
\label{\detokenize{input/temporal-interpolation:interpolation-flags}}
\sphinxAtStartPar
When JULES needs to know what type of interpolation to use for a variable, the following flags are used.
\begin{description}
\sphinxlineitem{\sphinxcode{\sphinxupquote{i}}}
\sphinxAtStartPar
Linear interpolation from the data timestep to the model timestep.

\sphinxAtStartPar
For instantaneous data (e.g. air temperature, surface pressure), this is almost always the flag that should be used.

\sphinxlineitem{\sphinxcode{\sphinxupquote{nb}}, \sphinxcode{\sphinxupquote{nc}} and \sphinxcode{\sphinxupquote{nf}}}
\sphinxAtStartPar
Values will be held constant with time for all model timesteps associated with a particular data timestep.

\sphinxAtStartPar
One of these flags should be used for flux variables that are \sphinxstyleemphasis{discontinuous} by nature, e.g. precipitation.

\sphinxAtStartPar
\sphinxcode{\sphinxupquote{nb}} should be used if the dataset uses backwards average values, \sphinxcode{\sphinxupquote{nf}} should be used if the data set uses forwards average values and \sphinxcode{\sphinxupquote{nc}} should be used if the dataset uses centred average values (this is quite rare).

\sphinxlineitem{\sphinxcode{\sphinxupquote{b}}, \sphinxcode{\sphinxupquote{c}} and \sphinxcode{\sphinxupquote{f}}}
\sphinxAtStartPar
Data is interpolated using a simplified version of the Sheng and Zwiers (1998) %
\begin{footnote}[1]\sphinxAtStartFootnote
Sheng and Zwiers (1998) An improved scheme for time\sphinxhyphen{}dependent boundary conditions in atmospheric general circulation models, Climate Dynamics, 14, 609\sphinxhyphen{}613.
%
\end{footnote} method that conserves the period means of the data.

\sphinxAtStartPar
One of these flags should be used for flux variables that are \sphinxstyleemphasis{continuous} in nature, e.g. radiation.

\sphinxAtStartPar
In order to ensure conservation of the average, these flags should be used only if the data period is an even multiple of the model timestep (i.e., if \sphinxcode{\sphinxupquote{data\_period = 2 * n * timestep\_len; n = 1, 2, 3, ...}}). The curve\sphinxhyphen{}fitting process tends to produce occasional values near turning points that fall outside the range of the input values.

\sphinxAtStartPar
Similar to above, \sphinxcode{\sphinxupquote{b}} should be used if the dataset uses backwards average values, \sphinxcode{\sphinxupquote{f}} should be used if the data set uses forwards average values and \sphinxcode{\sphinxupquote{c}} should be used if the dataset uses centred average values.

\end{description}

\sphinxAtStartPar
In order to perform interpolation, JULES may require input data for one or two data timesteps that fall before or after the times for the integration:


\begin{savenotes}\sphinxattablestart
\centering
\begin{tabulary}{\linewidth}[t]{|T|T|}
\hline
\sphinxstyletheadfamily 
\sphinxAtStartPar
Flag
&\sphinxstyletheadfamily 
\sphinxAtStartPar
Extra data timesteps required
\\
\hline
\sphinxAtStartPar
\sphinxcode{\sphinxupquote{nf}}
&
\sphinxAtStartPar
Only requires data that falls within the integration times
\\
\hline
\sphinxAtStartPar
\sphinxcode{\sphinxupquote{i}}, \sphinxcode{\sphinxupquote{nb}}, \sphinxcode{\sphinxupquote{nc}}
&
\sphinxAtStartPar
Requires one data timestep beyond the end of the integration
\\
\hline
\sphinxAtStartPar
\sphinxcode{\sphinxupquote{nb}}
&
\sphinxAtStartPar
Requires two data timesteps beyond the end of the integration
\\
\hline
\sphinxAtStartPar
\sphinxcode{\sphinxupquote{nf}}
&
\sphinxAtStartPar
Requires one data timestep before the start and one data timestep beyond the end of the integration
\\
\hline
\sphinxAtStartPar
\sphinxcode{\sphinxupquote{nc}}
&
\sphinxAtStartPar
Requires one data timestep before the start and two data timesteps beyond the end of the integration
\\
\hline
\end{tabulary}
\par
\sphinxattableend\end{savenotes}

\sphinxAtStartPar
Also, note that for centred data (flags \sphinxcode{\sphinxupquote{c}} and \sphinxcode{\sphinxupquote{nc}}) the time of the data should be given as that at the start of the averaging period, rather than the centre, e.g. the 3\sphinxhyphen{}hour average over 06H to 09H, centred at 07:30H, should be treated as having timestamp 06H.

\begin{figure}[htbp]
\centering
\capstart

\noindent\sphinxincludegraphics{{interp_examples}.png}
\caption{Examples of data interpolated with \sphinxcode{\sphinxupquote{i}}, \sphinxcode{\sphinxupquote{nb}}, \sphinxcode{\sphinxupquote{nf}}, \sphinxcode{\sphinxupquote{b}} and \sphinxcode{\sphinxupquote{f}}, plotted against the data they are derived from}\label{\detokenize{input/temporal-interpolation:id3}}\end{figure}

\sphinxstepscope


\chapter{JULES output}
\label{\detokenize{output:jules-output}}\label{\detokenize{output::doc}}
\sphinxAtStartPar
JULES separates output into one or more output ‘profiles’. Within each profile, all variables selected for output are written to the same file with the same frequency (also referred to as the ‘output period’). The output period can be any multiple of the model timestep, including calendar months or years.

\sphinxAtStartPar
Most output is provided on the model grid only. Some variables are provided on the river routing model grid instead. Each output profile can contain \sphinxstylestrong{either} model grid \sphinxstylestrong{or} river routing model grid variables, but not both.

\sphinxAtStartPar
Each output file contains the latitude and longitude of each point to allow the points to be located in a grid if desired (e.g. for visualisation). Output files also contain two time related variables to locate the values in time (this is described in more detail {\hyperref[\detokenize{output:associating-output-values-with-the-correct-time}]{\emph{below}}}).

\sphinxAtStartPar
JULES is capable of performing five different types of time\sphinxhyphen{}processing \sphinxhyphen{} snapshot (instantaneous) values, time averages, time minima, time maxima and time accumulations. Snapshots are instantaneous values produced during the first model timestep of each output period. Time averages, minima, maxima and accumulations are calculated over the output period. Each output variable is annotated with a \sphinxhref{http://cfconventions.org/Data/cf-conventions/cf-conventions-1.6/build/cf-conventions.html\#cell-methods}{CF convention \sphinxcode{\sphinxupquote{cell\_methods}} attribute} to indicate whether it is a snapshot value (\sphinxcode{\sphinxupquote{time : point}}), time average (\sphinxcode{\sphinxupquote{time : mean}}), time minimum (\sphinxcode{\sphinxupquote{time : minimum}}), time maximum (\sphinxcode{\sphinxupquote{time : maximum}}) or time accumulation (\sphinxcode{\sphinxupquote{time : sum}}).

\sphinxAtStartPar
Each profile can be considered as a separate data stream. By using more than one profile the user can, for example:
\begin{itemize}
\item {} 
\sphinxAtStartPar
Output one set of variables to one file, and other variables to another file.

\item {} 
\sphinxAtStartPar
Write instantaneous values to one file, and time\sphinxhyphen{}averaged values to another.

\item {} 
\sphinxAtStartPar
Write low\sphinxhyphen{}frequency output throughout the run to one file, and high\sphinxhyphen{}frequency output from a smaller part of the run (e.g. a ‘special observation period’) to another file.

\end{itemize}

\sphinxAtStartPar
All output files will be NetCDF if JULES is compiled with ‘proper’ NetCDF libraries (see {\hyperref[\detokenize{building-and-running/intro::doc}]{\sphinxcrossref{\DUrole{doc}{Building and running JULES}}}}). Otherwise all output will be in columnar ASCII files.


\section{Associating output values with the correct time}
\label{\detokenize{output:associating-output-values-with-the-correct-time}}
\sphinxAtStartPar
JULES output files contain two time related variables to allow model output to be associated with the correct model time:
\begin{description}
\sphinxlineitem{\sphinxcode{\sphinxupquote{time}}}
\sphinxAtStartPar
For each output period, this variable contains the time of the end of the output period. This is the time that any snapshot values apply at.

\sphinxlineitem{\sphinxcode{\sphinxupquote{time\_bounds}}}
\sphinxAtStartPar
For each output period, this variable contains two values \sphinxhyphen{} the start and end of the output period. The output period is then the half\sphinxhyphen{}open interval given by:

\begin{sphinxVerbatim}[commandchars=\\\{\}]
\PYG{n}{time\PYGZus{}bounds}\PYG{p}{(}\PYG{l+m+mi}{1}\PYG{p}{)}\PYG{+w}{ }\PYG{o}{\PYGZlt{}}\PYG{+w}{ }\PYG{n+nb}{time}\PYG{+w}{ }\PYG{o}{\PYGZlt{}}\PYG{o}{=}\PYG{+w}{ }\PYG{n}{time\PYGZus{}bounds}\PYG{p}{(}\PYG{l+m+mi}{2}\PYG{p}{)}
\end{sphinxVerbatim}

\sphinxAtStartPar
This is the interval that means, minima, maxima and accumulations are calculated over.

\end{description}

\sphinxAtStartPar
During each model timestep, JULES captures values for output at the end of the timestep (i.e. after all the science code). This means that in output files, snapshot data at a particular timestep is:
\begin{itemize}
\item {} 
\sphinxAtStartPar
The state of the model at the end of the model timestep.

\item {} 
\sphinxAtStartPar
The fluxes that produced that state over the model timestep.

\end{itemize}

\sphinxAtStartPar
Due to the way the model equations work, this ensures that all output at a given timestep in the output files is consistent.


\section{Initial data}
\label{\detokenize{output:initial-data}}\label{\detokenize{output:id1}}
\sphinxAtStartPar
With the formulation given above, the initial state of the model (i.e. the state at the beginning of the first timestep of a section) is never output (except to dump files). For the majority of users, this will not be an issue. If the initial state is required, it is possible for an output profile to output the initial state for each section of a run (i.e. initial state of each spinup cycle and the main run) to a separate file \sphinxhyphen{} see {\hyperref[\detokenize{namelists/output.nml:JULES_OUTPUT_PROFILE::output_initial}]{\sphinxcrossref{\sphinxcode{\sphinxupquote{output\_initial}}}}}.

\begin{sphinxadmonition}{warning}{Warning:}
\sphinxAtStartPar
In initial data files, \sphinxstylestrong{only snapshot values for state variables will be valid}. All other variables specified in the output profile will exist in the file, but their values will be garbage \sphinxhyphen{} \sphinxstyleemphasis{not necessarily NAN} \sphinxhyphen{} so use these files with caution.
\end{sphinxadmonition}


\section{Dump files}
\label{\detokenize{output:dump-files}}
\sphinxAtStartPar
JULES writes dump files (a snapshot of the current model state) at several points during a run. These can be used to restart the model from that point if desired. The times that dump files are written are:
\begin{itemize}
\item {} 
\sphinxAtStartPar
After initialisation is complete, immediately before the start of the run (initial state).

\item {} 
\sphinxAtStartPar
Before starting each cycle of spin\sphinxhyphen{}up.

\item {} 
\sphinxAtStartPar
Before starting the main run.

\item {} 
\sphinxAtStartPar
At the end of the run (final state).

\item {} 
\sphinxAtStartPar
At the start of each calendar year.

\end{itemize}

\sphinxAtStartPar
Each dump is marked with the model date and time that it was produced.

\sphinxAtStartPar
Prior to vn4.3, the dump file contained sufficient prognostic variables such that the pre\sphinxhyphen{}dump model state could be recreated. From vn 4.3 onwards, the dump file now includes ancillary data. The model can optionally restart from these data rather than the values given in the ancillaries namelists. Latitude and longitude information are also now written to (but not read from) the dump file to aid users wishing to interrogate dump files for debugging or other purposes.

\sphinxstepscope


\chapter{The JULES namelist files}
\label{\detokenize{namelists/contents:the-jules-namelist-files}}\label{\detokenize{namelists/contents::doc}}
\sphinxAtStartPar
Each run of JULES is controlled by a number of files containing Fortran namelists. These files specify details including:
\begin{itemize}
\item {} 
\sphinxAtStartPar
Switches to allow different model configurations to be selected at run\sphinxhyphen{}time.

\item {} 
\sphinxAtStartPar
Start and end times for the run.

\item {} 
\sphinxAtStartPar
What input data to use and how to read it.

\item {} 
\sphinxAtStartPar
How to construct the model grid.

\item {} 
\sphinxAtStartPar
Values for various parameters.

\item {} 
\sphinxAtStartPar
The required output.

\end{itemize}

\sphinxAtStartPar
These files have specific names, and JULES expects all these files to exist for every run (even when their contents are not required). JULES also expects that the namelists within each file appear in the order given below.

\sphinxstepscope


\section{Introduction to Fortran namelists}
\label{\detokenize{namelists/intro:introduction-to-fortran-namelists}}\label{\detokenize{namelists/intro::doc}}
\sphinxAtStartPar
Each namelist file read by JULES contains one or more Fortran namelists. Any content that does not form part of a namelist group is not read or interpreted in any way by Fortran, and so can be used as comments.

\sphinxAtStartPar
A Fortran namelist combines several related variables (referred to as ‘members’ of the namelist) together, which are then read with a single statement. The members can appear in any order. A Fortran namelist takes the following format:

\begin{sphinxVerbatim}[commandchars=\\\{\}]
\PYG{n+nn}{\PYGZam{}GROUP\PYGZus{}NAME}
  \PYG{n+nv}{char\PYGZus{}variable} \PYG{o}{=} \PYG{l+s+s2}{\PYGZdq{}a char variable\PYGZdq{}}\PYG{p}{,}
  \PYG{n+nv}{logical\PYGZus{}variable} \PYG{o}{=} \PYG{l+s+ss}{T}\PYG{p}{,}
  \PYG{n+nv}{nitems} \PYG{o}{=} \PYG{l+m+mi}{5}\PYG{p}{,}
  \PYG{n+nv}{list\PYGZus{}variable} \PYG{o}{=} \PYG{l+m+mf}{0.1}  \PYG{l+m+mf}{0.2}  \PYG{l+m+mf}{0.3}  \PYG{l+m+mf}{0.4}  \PYG{l+m+mf}{0.5}
\PYG{n+nn}{/}
\end{sphinxVerbatim}

\sphinxAtStartPar
The namelist definition is anything that appears between \sphinxcode{\sphinxupquote{\&GROUP\_NAME}} and \sphinxcode{\sphinxupquote{/}}. Values are then declared for the namelist members using the form \sphinxcode{\sphinxupquote{member\_name = member\_value}}. The member names are determined by the definition of the namelist in the Fortran source code. The member names for the JULES namelists are documented in the following sections.

\sphinxAtStartPar
Values for character variables must be enclosed in either single(’ ‘) or double (” “) quotes. Logical values can be specified using \sphinxcode{\sphinxupquote{.TRUE.}}/\sphinxcode{\sphinxupquote{.FALSE.}} or with the shorthand \sphinxcode{\sphinxupquote{T}}/\sphinxcode{\sphinxupquote{F}}. Integer and real values are specified simply by giving the value. The vast majority of compilers (all tested compilers) allow lists to be specified either horizontally or vertically, depending on preference. The following definitions are
identical:

\begin{sphinxVerbatim}[commandchars=\\\{\}]
\PYG{n+nv}{list\PYGZus{}variable} \PYG{o}{=} \PYG{l+m+mf}{0.1}  \PYG{l+m+mf}{0.2}  \PYG{l+m+mf}{0.3}  \PYG{l+m+mf}{0.4}  \PYG{l+m+mf}{0.5}

\PYG{n+nv}{list\PYGZus{}variable}\PYG{p}{(}\PYG{l+m+mi}{1}\PYG{p}{)} \PYG{o}{=} \PYG{l+m+mf}{0.1}
\PYG{n+nv}{list\PYGZus{}variable}\PYG{p}{(}\PYG{l+m+mi}{2}\PYG{p}{)} \PYG{o}{=} \PYG{l+m+mf}{0.2}
\PYG{n+nv}{list\PYGZus{}variable}\PYG{p}{(}\PYG{l+m+mi}{3}\PYG{p}{)} \PYG{o}{=} \PYG{l+m+mf}{0.3}
\PYG{n+nv}{list\PYGZus{}variable}\PYG{p}{(}\PYG{l+m+mi}{4}\PYG{p}{)} \PYG{o}{=} \PYG{l+m+mf}{0.4}
\PYG{n+nv}{list\PYGZus{}variable}\PYG{p}{(}\PYG{l+m+mi}{5}\PYG{p}{)} \PYG{o}{=} \PYG{l+m+mf}{0.5}
\end{sphinxVerbatim}

\sphinxAtStartPar
Namelists are an ideal input mechanism for programs like JULES that have a large number of inputs, most of which users never change from the default. Since each variable can have a sensible default value specified in the code, the user need only specify variables they wish to change from the default. This can substantially reduce the size and complexity of the namelist files. For example, suppose that in the above
example the namelist member \sphinxcode{\sphinxupquote{logical\_variable}} has a default value of \sphinxcode{\sphinxupquote{.TRUE.}}. Then the following namelist specification is equivalent to that above:

\begin{sphinxVerbatim}[commandchars=\\\{\}]
\PYG{n+nn}{\PYGZam{}GROUP\PYGZus{}NAME}
  \PYG{n+nv}{char\PYGZus{}variable} \PYG{o}{=} \PYG{l+s+s2}{\PYGZdq{}a char variable\PYGZdq{}}\PYG{p}{,}
  \PYG{n+nv}{nitems} \PYG{o}{=} \PYG{l+m+mi}{5}\PYG{p}{,}
  \PYG{n+nv}{list\PYGZus{}variable} \PYG{o}{=} \PYG{l+m+mf}{0.1}  \PYG{l+m+mf}{0.2}  \PYG{l+m+mf}{0.3}  \PYG{l+m+mf}{0.4}  \PYG{l+m+mf}{0.5}
\PYG{n+nn}{/}
\end{sphinxVerbatim}

\sphinxstepscope


\section{\sphinxstyleliteralintitle{\sphinxupquote{jules\_prnt\_control.nml}}}
\label{\detokenize{namelists/jules_prnt_control.nml:jules-prnt-control-nml}}\label{\detokenize{namelists/jules_prnt_control.nml::doc}}
\sphinxAtStartPar
This file contains one namelist called {\hyperref[\detokenize{namelists/jules_prnt_control.nml:namelist-JULES_PRNT_CONTROL}]{\sphinxcrossref{\sphinxcode{\sphinxupquote{JULES\_PRNT\_CONTROL}}}}}.

\sphinxAtStartPar
This namelist sets options for output of diagnostic and informative messages.


\subsection{\sphinxstyleliteralintitle{\sphinxupquote{JULES\_PRNT\_CONTROL}} namelist members}
\label{\detokenize{namelists/jules_prnt_control.nml:namelist-JULES_PRNT_CONTROL}}\label{\detokenize{namelists/jules_prnt_control.nml:jules-prnt-control-namelist-members}}\index{JULES\_PRNT\_CONTROL (namelist)@\spxentry{JULES\_PRNT\_CONTROL}\spxextra{namelist}|spxpagem}\index{prnt\_writers (in namelist JULES\_PRNT\_CONTROL)@\spxentry{prnt\_writers}\spxextra{in namelist JULES\_PRNT\_CONTROL}|spxpagem}

\begin{fulllineitems}
\phantomsection\label{\detokenize{namelists/jules_prnt_control.nml:JULES_PRNT_CONTROL::prnt_writers}}
\pysigstartsignatures
\pysigline{\sphinxcode{\sphinxupquote{JULES\_PRNT\_CONTROL::}}\sphinxbfcode{\sphinxupquote{prnt\_writers}}}
\pysigstopsignatures\begin{quote}\begin{description}
\sphinxlineitem{Type}
\sphinxAtStartPar
integer

\sphinxlineitem{Permitted}
\sphinxAtStartPar
1,2

\sphinxlineitem{Default}
\sphinxAtStartPar
1

\end{description}\end{quote}

\sphinxAtStartPar
Selects which tasks in a parallel job will write informative
output.


\begin{savenotes}\sphinxattablestart
\centering
\begin{tabulary}{\linewidth}[t]{|T|T|}
\hline

\sphinxAtStartPar
1
&
\sphinxAtStartPar
All tasks write output
\\
\hline
\sphinxAtStartPar
2
&
\sphinxAtStartPar
Only the first task (Task 0) writes output
\\
\hline
\end{tabulary}
\par
\sphinxattableend\end{savenotes}

\end{fulllineitems}


\sphinxstepscope


\section{\sphinxstyleliteralintitle{\sphinxupquote{jules\_surface\_types.nml}}}
\label{\detokenize{namelists/jules_surface_types.nml:jules-surface-types-nml}}\label{\detokenize{namelists/jules_surface_types.nml::doc}}\phantomsection\label{\detokenize{namelists/jules_surface_types.nml:top-jules-surface-types}}
\sphinxAtStartPar
This file configures the surface types used by JULES. It contains one
namelist called {\hyperref[\detokenize{namelists/jules_surface_types.nml:namelist-JULES_SURFACE_TYPES}]{\sphinxcrossref{\sphinxcode{\sphinxupquote{JULES\_SURFACE\_TYPES}}}}}.

\sphinxAtStartPar
The surface type IDs, which were introduced in the UM in order to
identify the surface types present in input and output, have been made
available to standalone. The defined surface type IDs are given in the
description here in brackets (\#). In order to keep the GUI from
appearing cluttered, the surface types have been added with
\sphinxcode{\sphinxupquote{compulsory=false}}, unless they are attached to specific science
options allowing them to be triggered off, which would be the
preferred method. A \sphinxcode{\sphinxupquote{compulsory=false}} surface type, can be added
and removed in the GUI window as described in the table below. To:
\begin{quote}


\begin{savenotes}\sphinxattablestart
\centering
\begin{tabulary}{\linewidth}[t]{|T|T|}
\hline

\sphinxAtStartPar
Add
&
\sphinxAtStartPar
“Add latent variable” using the right click menu, which opens a list of defined surface types.
\\
\hline
\sphinxAtStartPar
Remove
&
\sphinxAtStartPar
“Remove” the variable using the cog menu.
\\
\hline
\end{tabulary}
\par
\sphinxattableend\end{savenotes}
\end{quote}

\begin{sphinxadmonition}{note}{Note:}
\sphinxAtStartPar
Please be aware that while the surface type IDs have been
made available and are used to check the surface type
configuration at runtime, they are not yet used by the JULES
I/O.
\end{sphinxadmonition}


\subsection{\sphinxstyleliteralintitle{\sphinxupquote{JULES\_SURFACE\_TYPES}} namelist members}
\label{\detokenize{namelists/jules_surface_types.nml:namelist-JULES_SURFACE_TYPES}}\label{\detokenize{namelists/jules_surface_types.nml:jules-surface-types-namelist-members}}\index{JULES\_SURFACE\_TYPES (namelist)@\spxentry{JULES\_SURFACE\_TYPES}\spxextra{namelist}|spxpagem}
\begin{sphinxadmonition}{note}{Note:}
\sphinxAtStartPar
The total number of surface types to be modelled is called \sphinxcode{\sphinxupquote{ntype}}, and is given by \sphinxcode{\sphinxupquote{ntype = npft + nnvg}}.

\sphinxAtStartPar
In the original setup, JULES models 5 vegetation types and 4 non\sphinxhyphen{}vegetation types (\sphinxcode{\sphinxupquote{npft = 5}}, \sphinxcode{\sphinxupquote{nnvg = 4}}). However, the model domain need not contain all 9 types, e.g. the domain could consist of a single point with 100\% grass. The amount of each type in the domain is normally set in {\hyperref[\detokenize{namelists/ancillaries.nml:namelist-JULES_FRAC}]{\sphinxcrossref{\sphinxcode{\sphinxupquote{JULES\_FRAC}}}}}.

\sphinxAtStartPar
If the crop model is active (i.e. \sphinxcode{\sphinxupquote{ncpft}} \textgreater{} 0), then \sphinxcode{\sphinxupquote{nnpft = npft \sphinxhyphen{} ncpft}} where \sphinxcode{\sphinxupquote{nnpft}} is the number of natural PFTs.

\sphinxAtStartPar
Vegetation surfaces must always be present first in any list of surfaces.
\end{sphinxadmonition}
\index{npft (in namelist JULES\_SURFACE\_TYPES)@\spxentry{npft}\spxextra{in namelist JULES\_SURFACE\_TYPES}|spxpagem}

\begin{fulllineitems}
\phantomsection\label{\detokenize{namelists/jules_surface_types.nml:JULES_SURFACE_TYPES::npft}}
\pysigstartsignatures
\pysigline{\sphinxcode{\sphinxupquote{JULES\_SURFACE\_TYPES::}}\sphinxbfcode{\sphinxupquote{npft}}}
\pysigstopsignatures\begin{quote}\begin{description}
\sphinxlineitem{Type}
\sphinxAtStartPar
integer

\sphinxlineitem{Permitted}
\sphinxAtStartPar
\textgreater{}= 1

\sphinxlineitem{Default}
\sphinxAtStartPar
\sphinxhyphen{}32768

\end{description}\end{quote}

\sphinxAtStartPar
The number of plant functional types (PFTs) to be modelled.

\end{fulllineitems}

\index{ncpft (in namelist JULES\_SURFACE\_TYPES)@\spxentry{ncpft}\spxextra{in namelist JULES\_SURFACE\_TYPES}|spxpagem}

\begin{fulllineitems}
\phantomsection\label{\detokenize{namelists/jules_surface_types.nml:JULES_SURFACE_TYPES::ncpft}}
\pysigstartsignatures
\pysigline{\sphinxcode{\sphinxupquote{JULES\_SURFACE\_TYPES::}}\sphinxbfcode{\sphinxupquote{ncpft}}}
\pysigstopsignatures\begin{quote}\begin{description}
\sphinxlineitem{Type}
\sphinxAtStartPar
integer

\sphinxlineitem{Permitted}
\sphinxAtStartPar
\textless{} npft

\sphinxlineitem{Default}
\sphinxAtStartPar
0

\end{description}\end{quote}

\sphinxAtStartPar
The number of crop plant functional types to be modelled.

\end{fulllineitems}

\index{nnvg (in namelist JULES\_SURFACE\_TYPES)@\spxentry{nnvg}\spxextra{in namelist JULES\_SURFACE\_TYPES}|spxpagem}

\begin{fulllineitems}
\phantomsection\label{\detokenize{namelists/jules_surface_types.nml:JULES_SURFACE_TYPES::nnvg}}
\pysigstartsignatures
\pysigline{\sphinxcode{\sphinxupquote{JULES\_SURFACE\_TYPES::}}\sphinxbfcode{\sphinxupquote{nnvg}}}
\pysigstopsignatures\begin{quote}\begin{description}
\sphinxlineitem{Type}
\sphinxAtStartPar
integer

\sphinxlineitem{Permitted}
\sphinxAtStartPar
\textgreater{}= 1

\sphinxlineitem{Default}
\sphinxAtStartPar
\sphinxhyphen{}32768

\end{description}\end{quote}

\sphinxAtStartPar
The number of non\sphinxhyphen{}plant surface types to be modelled.

\end{fulllineitems}


\begin{sphinxadmonition}{note}{Non\sphinxhyphen{}vegetated surface types}

\sphinxAtStartPar
A negative value, when permitted, indicates that the surface type
is not in use.
\index{urban (in namelist JULES\_SURFACE\_TYPES)@\spxentry{urban}\spxextra{in namelist JULES\_SURFACE\_TYPES}|spxpagem}

\begin{fulllineitems}
\phantomsection\label{\detokenize{namelists/jules_surface_types.nml:JULES_SURFACE_TYPES::urban}}
\pysigstartsignatures
\pysigline{\sphinxcode{\sphinxupquote{JULES\_SURFACE\_TYPES::}}\sphinxbfcode{\sphinxupquote{urban}}}
\pysigstopsignatures\begin{quote}\begin{description}
\sphinxlineitem{Type}
\sphinxAtStartPar
integer

\sphinxlineitem{Permitted}
\sphinxAtStartPar
\sphinxhyphen{}1, npft+1:ntype

\sphinxlineitem{Default}
\sphinxAtStartPar
\sphinxhyphen{}32768

\end{description}\end{quote}

\sphinxAtStartPar
Index of the urban surface type (\#6).

\sphinxAtStartPar
Can only be used if {\hyperref[\detokenize{namelists/jules_surface.nml:JULES_SURFACE::l_urban2t}]{\sphinxcrossref{\sphinxcode{\sphinxupquote{l\_urban2t}}}}} = FALSE.

\end{fulllineitems}

\index{lake (in namelist JULES\_SURFACE\_TYPES)@\spxentry{lake}\spxextra{in namelist JULES\_SURFACE\_TYPES}|spxpagem}

\begin{fulllineitems}
\phantomsection\label{\detokenize{namelists/jules_surface_types.nml:JULES_SURFACE_TYPES::lake}}
\pysigstartsignatures
\pysigline{\sphinxcode{\sphinxupquote{JULES\_SURFACE\_TYPES::}}\sphinxbfcode{\sphinxupquote{lake}}}
\pysigstopsignatures\begin{quote}\begin{description}
\sphinxlineitem{Type}
\sphinxAtStartPar
integer

\sphinxlineitem{Permitted}
\sphinxAtStartPar
npft+1:ntype

\sphinxlineitem{Default}
\sphinxAtStartPar
\sphinxhyphen{}32768

\end{description}\end{quote}

\sphinxAtStartPar
Index of the lake surface type (\#7).

\end{fulllineitems}

\index{soil (in namelist JULES\_SURFACE\_TYPES)@\spxentry{soil}\spxextra{in namelist JULES\_SURFACE\_TYPES}|spxpagem}

\begin{fulllineitems}
\phantomsection\label{\detokenize{namelists/jules_surface_types.nml:JULES_SURFACE_TYPES::soil}}
\pysigstartsignatures
\pysigline{\sphinxcode{\sphinxupquote{JULES\_SURFACE\_TYPES::}}\sphinxbfcode{\sphinxupquote{soil}}}
\pysigstopsignatures\begin{quote}\begin{description}
\sphinxlineitem{Type}
\sphinxAtStartPar
integer

\sphinxlineitem{Permitted}
\sphinxAtStartPar
npft+1:ntype

\sphinxlineitem{Default}
\sphinxAtStartPar
\sphinxhyphen{}32768

\end{description}\end{quote}

\sphinxAtStartPar
Index of the soil surface type (\#8).

\begin{sphinxadmonition}{note}{Note:}
\sphinxAtStartPar
A soil surface type must be given (although the fraction may be set to zero).
\end{sphinxadmonition}

\end{fulllineitems}

\index{ice (in namelist JULES\_SURFACE\_TYPES)@\spxentry{ice}\spxextra{in namelist JULES\_SURFACE\_TYPES}|spxpagem}

\begin{fulllineitems}
\phantomsection\label{\detokenize{namelists/jules_surface_types.nml:JULES_SURFACE_TYPES::ice}}
\pysigstartsignatures
\pysigline{\sphinxcode{\sphinxupquote{JULES\_SURFACE\_TYPES::}}\sphinxbfcode{\sphinxupquote{ice}}}
\pysigstopsignatures\begin{quote}\begin{description}
\sphinxlineitem{Type}
\sphinxAtStartPar
integer

\sphinxlineitem{Permitted}
\sphinxAtStartPar
npft+1:ntype

\sphinxlineitem{Default}
\sphinxAtStartPar
\sphinxhyphen{}32768

\end{description}\end{quote}

\sphinxAtStartPar
Index of the ice surface type (\#9).

\begin{sphinxadmonition}{note}{Note:}
\sphinxAtStartPar
In the UM the ice surface type must be specified (although the fraction may be set to zero).
\end{sphinxadmonition}

\end{fulllineitems}


\begin{sphinxadmonition}{note}{Multiple ice tiles allowed to exist in an ice gridbox}

\sphinxAtStartPar
These surface types can only be used when multiple ice tiles are
allowed in a gridbox i.e. when
{\hyperref[\detokenize{namelists/jules_surface.nml:JULES_SURFACE::l_elev_land_ice}]{\sphinxcrossref{\sphinxcode{\sphinxupquote{l\_elev\_land\_ice}}}}} = TRUE.
\index{elev\_ice (in namelist JULES\_SURFACE\_TYPES)@\spxentry{elev\_ice}\spxextra{in namelist JULES\_SURFACE\_TYPES}|spxpagem}

\begin{fulllineitems}
\phantomsection\label{\detokenize{namelists/jules_surface_types.nml:JULES_SURFACE_TYPES::elev_ice}}
\pysigstartsignatures
\pysigline{\sphinxcode{\sphinxupquote{JULES\_SURFACE\_TYPES::}}\sphinxbfcode{\sphinxupquote{elev\_ice}}}
\pysigstopsignatures\begin{quote}\begin{description}
\sphinxlineitem{Type}
\sphinxAtStartPar
integer

\sphinxlineitem{Permitted}
\sphinxAtStartPar
\sphinxhyphen{}1,npft+1:ntype

\sphinxlineitem{Default}
\sphinxAtStartPar
\sphinxhyphen{}32768

\end{description}\end{quote}

\sphinxAtStartPar
Indices of the elevated ice types (\#901\sphinxhyphen{}925).

\sphinxAtStartPar
Must be grouped together with values \sphinxcode{\sphinxupquote{npft \textless{} elev\_ice \textless{}=
ntype}} OR \sphinxcode{\sphinxupquote{elev\_ice = \sphinxhyphen{}1}} to indicate they are not used
(i.e. all elevated rock instead).

\end{fulllineitems}

\index{elev\_rock (in namelist JULES\_SURFACE\_TYPES)@\spxentry{elev\_rock}\spxextra{in namelist JULES\_SURFACE\_TYPES}|spxpagem}

\begin{fulllineitems}
\phantomsection\label{\detokenize{namelists/jules_surface_types.nml:JULES_SURFACE_TYPES::elev_rock}}
\pysigstartsignatures
\pysigline{\sphinxcode{\sphinxupquote{JULES\_SURFACE\_TYPES::}}\sphinxbfcode{\sphinxupquote{elev\_rock}}}
\pysigstopsignatures\begin{quote}\begin{description}
\sphinxlineitem{Type}
\sphinxAtStartPar
integer

\sphinxlineitem{Permitted}
\sphinxAtStartPar
\sphinxhyphen{}1,npft+1:ntype

\sphinxlineitem{Default}
\sphinxAtStartPar
\sphinxhyphen{}32768

\end{description}\end{quote}

\sphinxAtStartPar
Indices of the elevated non\sphinxhyphen{}glaciated bedrock types
(\#926\sphinxhyphen{}950).

\sphinxAtStartPar
Must be grouped together, with values \sphinxcode{\sphinxupquote{npft \textless{} elev\_rock \textless{}=
ntype}} OR \sphinxcode{\sphinxupquote{elev\_rock = \sphinxhyphen{}1}} to indicate they are not used
(i.e. all elevated ice instead).

\end{fulllineitems}

\end{sphinxadmonition}

\begin{sphinxadmonition}{note}{Two\sphinxhyphen{}tile urban schemes including MORUSES}

\sphinxAtStartPar
These surface types can only be used when {\hyperref[\detokenize{namelists/jules_surface.nml:JULES_SURFACE::l_urban2t}]{\sphinxcrossref{\sphinxcode{\sphinxupquote{l\_urban2t}}}}} = TRUE.
\index{urban\_canyon (in namelist JULES\_SURFACE\_TYPES)@\spxentry{urban\_canyon}\spxextra{in namelist JULES\_SURFACE\_TYPES}|spxpagem}

\begin{fulllineitems}
\phantomsection\label{\detokenize{namelists/jules_surface_types.nml:JULES_SURFACE_TYPES::urban_canyon}}
\pysigstartsignatures
\pysigline{\sphinxcode{\sphinxupquote{JULES\_SURFACE\_TYPES::}}\sphinxbfcode{\sphinxupquote{urban\_canyon}}}
\pysigstopsignatures\begin{quote}\begin{description}
\sphinxlineitem{Type}
\sphinxAtStartPar
integer

\sphinxlineitem{Permitted}
\sphinxAtStartPar
npft+1:ntype

\sphinxlineitem{Default}
\sphinxAtStartPar
\sphinxhyphen{}32768

\end{description}\end{quote}

\sphinxAtStartPar
Index of the urban canyon surface type (\#601).

\end{fulllineitems}

\index{urban\_roof (in namelist JULES\_SURFACE\_TYPES)@\spxentry{urban\_roof}\spxextra{in namelist JULES\_SURFACE\_TYPES}|spxpagem}

\begin{fulllineitems}
\phantomsection\label{\detokenize{namelists/jules_surface_types.nml:JULES_SURFACE_TYPES::urban_roof}}
\pysigstartsignatures
\pysigline{\sphinxcode{\sphinxupquote{JULES\_SURFACE\_TYPES::}}\sphinxbfcode{\sphinxupquote{urban\_roof}}}
\pysigstopsignatures\begin{quote}\begin{description}
\sphinxlineitem{Type}
\sphinxAtStartPar
integer

\sphinxlineitem{Permitted}
\sphinxAtStartPar
npft+1:ntype

\sphinxlineitem{Default}
\sphinxAtStartPar
\sphinxhyphen{}32768

\end{description}\end{quote}

\sphinxAtStartPar
Index of the urban roof surface type (\#602).

\end{fulllineitems}


\begin{sphinxadmonition}{note}{Note:}
\sphinxAtStartPar
When giving urban fraction data (see {\hyperref[\detokenize{namelists/ancillaries.nml:namelist-JULES_FRAC}]{\sphinxcrossref{\sphinxcode{\sphinxupquote{JULES\_FRAC}}}}}), total \sphinxstyleemphasis{urban} fraction may be given instead of the separate canyon and roof fractions by entering it under the canyon fraction. When initialising if the roof fraction is zero, the canyon fraction will be interpreted as the total \sphinxstyleemphasis{urban} fraction and be partitioned according to the canyon fraction (W/R, see {\hyperref[\detokenize{namelists/ancillaries.nml:namelist-URBAN_PROPERTIES}]{\sphinxcrossref{\sphinxcode{\sphinxupquote{URBAN\_PROPERTIES}}}}}).
\end{sphinxadmonition}
\end{sphinxadmonition}
\end{sphinxadmonition}

\begin{sphinxadmonition}{note}{Surface types with \sphinxstyleliteralintitle{\sphinxupquote{compulsory=false}}}

\sphinxAtStartPar
These are required to allow the surface type configuration to be
checked at runtime and for surface types to be identified in the
output headers. These are added as a latent variable. Remove the
surface type if it is not required (see explanation at the
{\hyperref[\detokenize{namelists/jules_surface_types.nml:top-jules-surface-types}]{\sphinxcrossref{\DUrole{std,std-ref}{top}}}} of this page).
\index{usr\_type (in namelist JULES\_SURFACE\_TYPES)@\spxentry{usr\_type}\spxextra{in namelist JULES\_SURFACE\_TYPES}|spxpagem}

\begin{fulllineitems}
\phantomsection\label{\detokenize{namelists/jules_surface_types.nml:JULES_SURFACE_TYPES::usr_type}}
\pysigstartsignatures
\pysigline{\sphinxcode{\sphinxupquote{JULES\_SURFACE\_TYPES::}}\sphinxbfcode{\sphinxupquote{usr\_type}}}
\pysigstopsignatures\begin{quote}\begin{description}
\sphinxlineitem{Type}
\sphinxAtStartPar
integer

\sphinxlineitem{Permitted}
\sphinxAtStartPar
1:ntype

\sphinxlineitem{Default}
\sphinxAtStartPar
\sphinxhyphen{}32768

\end{description}\end{quote}

\sphinxAtStartPar
Index of user specified surface type (\#10\sphinxhyphen{}99).

\sphinxAtStartPar
A user surface type can be used when experimenting with new
surface configurations without a code change. These can be
either vegetated or non\sphinxhyphen{}vegetated and are used solely to assign an
ID number.

\end{fulllineitems}


\begin{sphinxadmonition}{note}{Vegetated surface types}

\sphinxAtStartPar
A negative value, when permitted, indicates that the surface
type is not in use.
\index{brd\_leaf (in namelist JULES\_SURFACE\_TYPES)@\spxentry{brd\_leaf}\spxextra{in namelist JULES\_SURFACE\_TYPES}|spxpagem}

\begin{fulllineitems}
\phantomsection\label{\detokenize{namelists/jules_surface_types.nml:JULES_SURFACE_TYPES::brd_leaf}}
\pysigstartsignatures
\pysigline{\sphinxcode{\sphinxupquote{JULES\_SURFACE\_TYPES::}}\sphinxbfcode{\sphinxupquote{brd\_leaf}}}
\pysigstopsignatures\begin{quote}\begin{description}
\sphinxlineitem{Type}
\sphinxAtStartPar
integer

\sphinxlineitem{Permitted}
\sphinxAtStartPar
1:npft

\sphinxlineitem{Default}
\sphinxAtStartPar
\sphinxhyphen{}32768

\end{description}\end{quote}

\sphinxAtStartPar
Index of the original broadleaf PFT surface type (\#1).

\end{fulllineitems}

\index{brd\_leaf\_dec (in namelist JULES\_SURFACE\_TYPES)@\spxentry{brd\_leaf\_dec}\spxextra{in namelist JULES\_SURFACE\_TYPES}|spxpagem}

\begin{fulllineitems}
\phantomsection\label{\detokenize{namelists/jules_surface_types.nml:JULES_SURFACE_TYPES::brd_leaf_dec}}
\pysigstartsignatures
\pysigline{\sphinxcode{\sphinxupquote{JULES\_SURFACE\_TYPES::}}\sphinxbfcode{\sphinxupquote{brd\_leaf\_dec}}}
\pysigstopsignatures\begin{quote}\begin{description}
\sphinxlineitem{Type}
\sphinxAtStartPar
integer

\sphinxlineitem{Permitted}
\sphinxAtStartPar
1:npft

\sphinxlineitem{Default}
\sphinxAtStartPar
\sphinxhyphen{}32768

\end{description}\end{quote}

\sphinxAtStartPar
Index of broadleaf (decidous) PFT surface type (\#101)

\end{fulllineitems}

\index{brd\_leaf\_eg\_trop (in namelist JULES\_SURFACE\_TYPES)@\spxentry{brd\_leaf\_eg\_trop}\spxextra{in namelist JULES\_SURFACE\_TYPES}|spxpagem}

\begin{fulllineitems}
\phantomsection\label{\detokenize{namelists/jules_surface_types.nml:JULES_SURFACE_TYPES::brd_leaf_eg_trop}}
\pysigstartsignatures
\pysigline{\sphinxcode{\sphinxupquote{JULES\_SURFACE\_TYPES::}}\sphinxbfcode{\sphinxupquote{brd\_leaf\_eg\_trop}}}
\pysigstopsignatures\begin{quote}\begin{description}
\sphinxlineitem{Type}
\sphinxAtStartPar
integer

\sphinxlineitem{Permitted}
\sphinxAtStartPar
1:npft

\sphinxlineitem{Default}
\sphinxAtStartPar
\sphinxhyphen{}32768

\end{description}\end{quote}

\sphinxAtStartPar
Index of broadleaf (evergreen tropical) PFT surface type (\#102).

\end{fulllineitems}

\index{brd\_leaf\_eg\_temp (in namelist JULES\_SURFACE\_TYPES)@\spxentry{brd\_leaf\_eg\_temp}\spxextra{in namelist JULES\_SURFACE\_TYPES}|spxpagem}

\begin{fulllineitems}
\phantomsection\label{\detokenize{namelists/jules_surface_types.nml:JULES_SURFACE_TYPES::brd_leaf_eg_temp}}
\pysigstartsignatures
\pysigline{\sphinxcode{\sphinxupquote{JULES\_SURFACE\_TYPES::}}\sphinxbfcode{\sphinxupquote{brd\_leaf\_eg\_temp}}}
\pysigstopsignatures\begin{quote}\begin{description}
\sphinxlineitem{Type}
\sphinxAtStartPar
integer

\sphinxlineitem{Permitted}
\sphinxAtStartPar
1:npft

\sphinxlineitem{Default}
\sphinxAtStartPar
\sphinxhyphen{}32768

\end{description}\end{quote}

\sphinxAtStartPar
Index of broadleaf (evergreen temperate) PFT surface type (\#103).

\end{fulllineitems}

\index{ndl\_leaf (in namelist JULES\_SURFACE\_TYPES)@\spxentry{ndl\_leaf}\spxextra{in namelist JULES\_SURFACE\_TYPES}|spxpagem}

\begin{fulllineitems}
\phantomsection\label{\detokenize{namelists/jules_surface_types.nml:JULES_SURFACE_TYPES::ndl_leaf}}
\pysigstartsignatures
\pysigline{\sphinxcode{\sphinxupquote{JULES\_SURFACE\_TYPES::}}\sphinxbfcode{\sphinxupquote{ndl\_leaf}}}
\pysigstopsignatures\begin{quote}\begin{description}
\sphinxlineitem{Type}
\sphinxAtStartPar
integer

\sphinxlineitem{Permitted}
\sphinxAtStartPar
1:npft

\sphinxlineitem{Default}
\sphinxAtStartPar
\sphinxhyphen{}32768

\end{description}\end{quote}

\sphinxAtStartPar
Index of original needleleaf PFT surface type (\#2).

\end{fulllineitems}

\index{ndl\_leaf\_dec (in namelist JULES\_SURFACE\_TYPES)@\spxentry{ndl\_leaf\_dec}\spxextra{in namelist JULES\_SURFACE\_TYPES}|spxpagem}

\begin{fulllineitems}
\phantomsection\label{\detokenize{namelists/jules_surface_types.nml:JULES_SURFACE_TYPES::ndl_leaf_dec}}
\pysigstartsignatures
\pysigline{\sphinxcode{\sphinxupquote{JULES\_SURFACE\_TYPES::}}\sphinxbfcode{\sphinxupquote{ndl\_leaf\_dec}}}
\pysigstopsignatures\begin{quote}\begin{description}
\sphinxlineitem{Type}
\sphinxAtStartPar
integer

\sphinxlineitem{Permitted}
\sphinxAtStartPar
1:npft

\sphinxlineitem{Default}
\sphinxAtStartPar
\sphinxhyphen{}32768

\end{description}\end{quote}

\sphinxAtStartPar
Index of needleleaf (deciduous) PFT surface type (\#201).

\end{fulllineitems}

\index{ndl\_leaf\_eg (in namelist JULES\_SURFACE\_TYPES)@\spxentry{ndl\_leaf\_eg}\spxextra{in namelist JULES\_SURFACE\_TYPES}|spxpagem}

\begin{fulllineitems}
\phantomsection\label{\detokenize{namelists/jules_surface_types.nml:JULES_SURFACE_TYPES::ndl_leaf_eg}}
\pysigstartsignatures
\pysigline{\sphinxcode{\sphinxupquote{JULES\_SURFACE\_TYPES::}}\sphinxbfcode{\sphinxupquote{ndl\_leaf\_eg}}}
\pysigstopsignatures\begin{quote}\begin{description}
\sphinxlineitem{Type}
\sphinxAtStartPar
integer

\sphinxlineitem{Permitted}
\sphinxAtStartPar
1:npft

\sphinxlineitem{Default}
\sphinxAtStartPar
\sphinxhyphen{}32768

\end{description}\end{quote}

\sphinxAtStartPar
Index of needleleaf (evergreen) PFT surface type (\#202).

\end{fulllineitems}

\index{c3\_grass (in namelist JULES\_SURFACE\_TYPES)@\spxentry{c3\_grass}\spxextra{in namelist JULES\_SURFACE\_TYPES}|spxpagem}

\begin{fulllineitems}
\phantomsection\label{\detokenize{namelists/jules_surface_types.nml:JULES_SURFACE_TYPES::c3_grass}}
\pysigstartsignatures
\pysigline{\sphinxcode{\sphinxupquote{JULES\_SURFACE\_TYPES::}}\sphinxbfcode{\sphinxupquote{c3\_grass}}}
\pysigstopsignatures\begin{quote}\begin{description}
\sphinxlineitem{Type}
\sphinxAtStartPar
integer

\sphinxlineitem{Permitted}
\sphinxAtStartPar
1:npft

\sphinxlineitem{Default}
\sphinxAtStartPar
\sphinxhyphen{}32768

\end{description}\end{quote}

\sphinxAtStartPar
Index of original C3 grass PFT surface type (\#3).

\end{fulllineitems}

\index{c3\_crop (in namelist JULES\_SURFACE\_TYPES)@\spxentry{c3\_crop}\spxextra{in namelist JULES\_SURFACE\_TYPES}|spxpagem}

\begin{fulllineitems}
\phantomsection\label{\detokenize{namelists/jules_surface_types.nml:JULES_SURFACE_TYPES::c3_crop}}
\pysigstartsignatures
\pysigline{\sphinxcode{\sphinxupquote{JULES\_SURFACE\_TYPES::}}\sphinxbfcode{\sphinxupquote{c3\_crop}}}
\pysigstopsignatures\begin{quote}\begin{description}
\sphinxlineitem{Type}
\sphinxAtStartPar
integer

\sphinxlineitem{Permitted}
\sphinxAtStartPar
1:npft

\sphinxlineitem{Default}
\sphinxAtStartPar
\sphinxhyphen{}32768

\end{description}\end{quote}

\sphinxAtStartPar
Index of C3 crop PFT surface type (\#301).

\end{fulllineitems}

\index{c3\_pasture (in namelist JULES\_SURFACE\_TYPES)@\spxentry{c3\_pasture}\spxextra{in namelist JULES\_SURFACE\_TYPES}|spxpagem}

\begin{fulllineitems}
\phantomsection\label{\detokenize{namelists/jules_surface_types.nml:JULES_SURFACE_TYPES::c3_pasture}}
\pysigstartsignatures
\pysigline{\sphinxcode{\sphinxupquote{JULES\_SURFACE\_TYPES::}}\sphinxbfcode{\sphinxupquote{c3\_pasture}}}
\pysigstopsignatures\begin{quote}\begin{description}
\sphinxlineitem{Type}
\sphinxAtStartPar
integer

\sphinxlineitem{Permitted}
\sphinxAtStartPar
1:npft

\sphinxlineitem{Default}
\sphinxAtStartPar
\sphinxhyphen{}32768

\end{description}\end{quote}

\sphinxAtStartPar
Index of C3 pasture PFT surface type (\#302).

\end{fulllineitems}

\index{c4\_grass (in namelist JULES\_SURFACE\_TYPES)@\spxentry{c4\_grass}\spxextra{in namelist JULES\_SURFACE\_TYPES}|spxpagem}

\begin{fulllineitems}
\phantomsection\label{\detokenize{namelists/jules_surface_types.nml:JULES_SURFACE_TYPES::c4_grass}}
\pysigstartsignatures
\pysigline{\sphinxcode{\sphinxupquote{JULES\_SURFACE\_TYPES::}}\sphinxbfcode{\sphinxupquote{c4\_grass}}}
\pysigstopsignatures\begin{quote}\begin{description}
\sphinxlineitem{Type}
\sphinxAtStartPar
integer

\sphinxlineitem{Permitted}
\sphinxAtStartPar
1:npft

\sphinxlineitem{Default}
\sphinxAtStartPar
\sphinxhyphen{}32768

\end{description}\end{quote}

\sphinxAtStartPar
Index of original C4 grass PFT surface type (\#4).

\end{fulllineitems}

\index{c4\_crop (in namelist JULES\_SURFACE\_TYPES)@\spxentry{c4\_crop}\spxextra{in namelist JULES\_SURFACE\_TYPES}|spxpagem}

\begin{fulllineitems}
\phantomsection\label{\detokenize{namelists/jules_surface_types.nml:JULES_SURFACE_TYPES::c4_crop}}
\pysigstartsignatures
\pysigline{\sphinxcode{\sphinxupquote{JULES\_SURFACE\_TYPES::}}\sphinxbfcode{\sphinxupquote{c4\_crop}}}
\pysigstopsignatures\begin{quote}\begin{description}
\sphinxlineitem{Type}
\sphinxAtStartPar
integer

\sphinxlineitem{Permitted}
\sphinxAtStartPar
1:npft

\sphinxlineitem{Default}
\sphinxAtStartPar
\sphinxhyphen{}32768

\end{description}\end{quote}

\sphinxAtStartPar
Index of C4 crop PFT surface type (\#401).

\end{fulllineitems}

\index{c4\_pasture (in namelist JULES\_SURFACE\_TYPES)@\spxentry{c4\_pasture}\spxextra{in namelist JULES\_SURFACE\_TYPES}|spxpagem}

\begin{fulllineitems}
\phantomsection\label{\detokenize{namelists/jules_surface_types.nml:JULES_SURFACE_TYPES::c4_pasture}}
\pysigstartsignatures
\pysigline{\sphinxcode{\sphinxupquote{JULES\_SURFACE\_TYPES::}}\sphinxbfcode{\sphinxupquote{c4\_pasture}}}
\pysigstopsignatures\begin{quote}\begin{description}
\sphinxlineitem{Type}
\sphinxAtStartPar
integer

\sphinxlineitem{Permitted}
\sphinxAtStartPar
1:npft

\sphinxlineitem{Default}
\sphinxAtStartPar
\sphinxhyphen{}32768

\end{description}\end{quote}

\sphinxAtStartPar
Index of C4 pasture PFT surface type (\#402).

\end{fulllineitems}

\index{shrub (in namelist JULES\_SURFACE\_TYPES)@\spxentry{shrub}\spxextra{in namelist JULES\_SURFACE\_TYPES}|spxpagem}

\begin{fulllineitems}
\phantomsection\label{\detokenize{namelists/jules_surface_types.nml:JULES_SURFACE_TYPES::shrub}}
\pysigstartsignatures
\pysigline{\sphinxcode{\sphinxupquote{JULES\_SURFACE\_TYPES::}}\sphinxbfcode{\sphinxupquote{shrub}}}
\pysigstopsignatures\begin{quote}\begin{description}
\sphinxlineitem{Type}
\sphinxAtStartPar
integer

\sphinxlineitem{Permitted}
\sphinxAtStartPar
1:npft

\sphinxlineitem{Default}
\sphinxAtStartPar
\sphinxhyphen{}32768

\end{description}\end{quote}

\sphinxAtStartPar
Index of original shrub PFT surface type (\#5).

\end{fulllineitems}

\index{shrub\_dec (in namelist JULES\_SURFACE\_TYPES)@\spxentry{shrub\_dec}\spxextra{in namelist JULES\_SURFACE\_TYPES}|spxpagem}

\begin{fulllineitems}
\phantomsection\label{\detokenize{namelists/jules_surface_types.nml:JULES_SURFACE_TYPES::shrub_dec}}
\pysigstartsignatures
\pysigline{\sphinxcode{\sphinxupquote{JULES\_SURFACE\_TYPES::}}\sphinxbfcode{\sphinxupquote{shrub\_dec}}}
\pysigstopsignatures\begin{quote}\begin{description}
\sphinxlineitem{Type}
\sphinxAtStartPar
integer

\sphinxlineitem{Permitted}
\sphinxAtStartPar
1:npft

\sphinxlineitem{Default}
\sphinxAtStartPar
\sphinxhyphen{}32768

\end{description}\end{quote}

\sphinxAtStartPar
Index of shrub (deciduous) PFT surface type (\#501).

\end{fulllineitems}

\index{shrub\_eg (in namelist JULES\_SURFACE\_TYPES)@\spxentry{shrub\_eg}\spxextra{in namelist JULES\_SURFACE\_TYPES}|spxpagem}

\begin{fulllineitems}
\phantomsection\label{\detokenize{namelists/jules_surface_types.nml:JULES_SURFACE_TYPES::shrub_eg}}
\pysigstartsignatures
\pysigline{\sphinxcode{\sphinxupquote{JULES\_SURFACE\_TYPES::}}\sphinxbfcode{\sphinxupquote{shrub\_eg}}}
\pysigstopsignatures\begin{quote}\begin{description}
\sphinxlineitem{Type}
\sphinxAtStartPar
integer

\sphinxlineitem{Permitted}
\sphinxAtStartPar
1:npft

\sphinxlineitem{Default}
\sphinxAtStartPar
\sphinxhyphen{}32768

\end{description}\end{quote}

\sphinxAtStartPar
Index of shrub (evergreen) PFT surface type (\#502).

\end{fulllineitems}

\end{sphinxadmonition}
\end{sphinxadmonition}

\sphinxstepscope


\section{\sphinxstyleliteralintitle{\sphinxupquote{cable\_surface\_types.nml}}}
\label{\detokenize{namelists/cable_surface_types.nml:cable-surface-types-nml}}\label{\detokenize{namelists/cable_surface_types.nml::doc}}
\sphinxAtStartPar
This file configures the surface types used by CABLE. It contains one namelist called {\hyperref[\detokenize{namelists/cable_surface_types.nml:namelist-CABLE_SURFACE_TYPES}]{\sphinxcrossref{\sphinxcode{\sphinxupquote{CABLE\_SURFACE\_TYPES}}}}}.


\subsection{\sphinxstyleliteralintitle{\sphinxupquote{CABLE\_SURFACE\_TYPES}} namelist members}
\label{\detokenize{namelists/cable_surface_types.nml:namelist-CABLE_SURFACE_TYPES}}\label{\detokenize{namelists/cable_surface_types.nml:cable-surface-types-namelist-members}}\index{CABLE\_SURFACE\_TYPES (namelist)@\spxentry{CABLE\_SURFACE\_TYPES}\spxextra{namelist}|spxpagem}\index{npft\_cable (in namelist CABLE\_SURFACE\_TYPES)@\spxentry{npft\_cable}\spxextra{in namelist CABLE\_SURFACE\_TYPES}|spxpagem}

\begin{fulllineitems}
\phantomsection\label{\detokenize{namelists/cable_surface_types.nml:CABLE_SURFACE_TYPES::npft_cable}}
\pysigstartsignatures
\pysigline{\sphinxcode{\sphinxupquote{CABLE\_SURFACE\_TYPES::}}\sphinxbfcode{\sphinxupquote{npft\_cable}}}
\pysigstopsignatures\begin{quote}\begin{description}
\sphinxlineitem{Type}
\sphinxAtStartPar
integer

\sphinxlineitem{Permitted}
\sphinxAtStartPar
\textgreater{}= 1

\sphinxlineitem{Default}
\sphinxAtStartPar
\sphinxhyphen{}32768

\end{description}\end{quote}

\sphinxAtStartPar
The number of plant functional types to be modelled.

\end{fulllineitems}

\index{nnvg\_cable (in namelist CABLE\_SURFACE\_TYPES)@\spxentry{nnvg\_cable}\spxextra{in namelist CABLE\_SURFACE\_TYPES}|spxpagem}

\begin{fulllineitems}
\phantomsection\label{\detokenize{namelists/cable_surface_types.nml:CABLE_SURFACE_TYPES::nnvg_cable}}
\pysigstartsignatures
\pysigline{\sphinxcode{\sphinxupquote{CABLE\_SURFACE\_TYPES::}}\sphinxbfcode{\sphinxupquote{nnvg\_cable}}}
\pysigstopsignatures\begin{quote}\begin{description}
\sphinxlineitem{Type}
\sphinxAtStartPar
integer

\sphinxlineitem{Permitted}
\sphinxAtStartPar
\textgreater{}= 1

\sphinxlineitem{Default}
\sphinxAtStartPar
\sphinxhyphen{}32768

\end{description}\end{quote}

\sphinxAtStartPar
The number of non\sphinxhyphen{}plant surface types to be modelled.

\end{fulllineitems}


\begin{sphinxadmonition}{note}{Note:}
\sphinxAtStartPar
The total number of surface types to be modelled is called \sphinxcode{\sphinxupquote{ntype\_cable}}, and is given by \sphinxcode{\sphinxupquote{ntype\_cable = npft\_cable + nnvg\_cable}}.

\sphinxAtStartPar
In the standard setup, CABLE models 13 vegetation types and 4 non\sphinxhyphen{}vegetation types (\sphinxcode{\sphinxupquote{npft\_cable = 13}}, \sphinxcode{\sphinxupquote{nnvg\_cable = 4}}). However, the model domain need not contain all 13 types, e.g. the domain could consist of a single point with 100\% grass. The amount of each type in the domain is normally set in {\hyperref[\detokenize{namelists/ancillaries.nml:namelist-JULES_FRAC}]{\sphinxcrossref{\sphinxcode{\sphinxupquote{JULES\_FRAC}}}}}.
\end{sphinxadmonition}
\index{urban\_drive (in namelist CABLE\_SURFACE\_TYPES)@\spxentry{urban\_drive}\spxextra{in namelist CABLE\_SURFACE\_TYPES}|spxpagem}

\begin{fulllineitems}
\phantomsection\label{\detokenize{namelists/cable_surface_types.nml:CABLE_SURFACE_TYPES::urban_drive}}
\pysigstartsignatures
\pysigline{\sphinxcode{\sphinxupquote{CABLE\_SURFACE\_TYPES::}}\sphinxbfcode{\sphinxupquote{urban\_drive}}}
\pysigstopsignatures\begin{quote}\begin{description}
\sphinxlineitem{Type}
\sphinxAtStartPar
integer

\sphinxlineitem{Default}
\sphinxAtStartPar
\sphinxhyphen{}32768

\end{description}\end{quote}

\end{fulllineitems}

\index{lakes\_cable (in namelist CABLE\_SURFACE\_TYPES)@\spxentry{lakes\_cable}\spxextra{in namelist CABLE\_SURFACE\_TYPES}|spxpagem}

\begin{fulllineitems}
\phantomsection\label{\detokenize{namelists/cable_surface_types.nml:CABLE_SURFACE_TYPES::lakes_cable}}
\pysigstartsignatures
\pysigline{\sphinxcode{\sphinxupquote{CABLE\_SURFACE\_TYPES::}}\sphinxbfcode{\sphinxupquote{lakes\_cable}}}
\pysigstopsignatures\begin{quote}\begin{description}
\sphinxlineitem{Type}
\sphinxAtStartPar
integer

\sphinxlineitem{Default}
\sphinxAtStartPar
\sphinxhyphen{}32768

\end{description}\end{quote}

\sphinxAtStartPar
Index of the lakes surface type.

\sphinxAtStartPar
A negative value indicates no lakes surface type.

\end{fulllineitems}

\index{barren\_cable (in namelist CABLE\_SURFACE\_TYPES)@\spxentry{barren\_cable}\spxextra{in namelist CABLE\_SURFACE\_TYPES}|spxpagem}

\begin{fulllineitems}
\phantomsection\label{\detokenize{namelists/cable_surface_types.nml:CABLE_SURFACE_TYPES::barren_cable}}
\pysigstartsignatures
\pysigline{\sphinxcode{\sphinxupquote{CABLE\_SURFACE\_TYPES::}}\sphinxbfcode{\sphinxupquote{barren\_cable}}}
\pysigstopsignatures\begin{quote}\begin{description}
\sphinxlineitem{Type}
\sphinxAtStartPar
integer

\sphinxlineitem{Permitted}
\sphinxAtStartPar
\textgreater{}= 1

\sphinxlineitem{Default}
\sphinxAtStartPar
\sphinxhyphen{}32768

\end{description}\end{quote}

\sphinxAtStartPar
Index of the barren soil surface type.

\begin{sphinxadmonition}{note}{Note:}
\sphinxAtStartPar
A barren soil surface type must be given.
\end{sphinxadmonition}

\end{fulllineitems}

\index{ice\_cable (in namelist CABLE\_SURFACE\_TYPES)@\spxentry{ice\_cable}\spxextra{in namelist CABLE\_SURFACE\_TYPES}|spxpagem}

\begin{fulllineitems}
\phantomsection\label{\detokenize{namelists/cable_surface_types.nml:CABLE_SURFACE_TYPES::ice_cable}}
\pysigstartsignatures
\pysigline{\sphinxcode{\sphinxupquote{CABLE\_SURFACE\_TYPES::}}\sphinxbfcode{\sphinxupquote{ice\_cable}}}
\pysigstopsignatures\begin{quote}\begin{description}
\sphinxlineitem{Type}
\sphinxAtStartPar
integer

\sphinxlineitem{Default}
\sphinxAtStartPar
\sphinxhyphen{}32768

\end{description}\end{quote}

\sphinxAtStartPar
Index of the ice surface type.

\sphinxAtStartPar
A negative value indicates no ice surface type.

\end{fulllineitems}


\sphinxstepscope


\section{\sphinxstyleliteralintitle{\sphinxupquote{model\_environment.nml}}}
\label{\detokenize{namelists/model_environment.nml:model-environment-nml}}\label{\detokenize{namelists/model_environment.nml::doc}}
\sphinxAtStartPar
This file sets the model environment options e.g. whether JULES is
coupled to the UM or run in a standalone environment. It contains one
namelist called {\hyperref[\detokenize{namelists/model_environment.nml:namelist-JULES_MODEL_ENVIRONMENT}]{\sphinxcrossref{\sphinxcode{\sphinxupquote{JULES\_MODEL\_ENVIRONMENT}}}}}.

\sphinxAtStartPar
There are many JULES science options that are in shared namelists, so
they can be read both by standalone and by a model driving JULES
e.g. the UM. However some options either make no scientific sense or
the necessary input data are not available to the environment in which
JULES is being driven as the plumbing has not yet been done. This
causes problems for example when creating standalone apps from UM
configurations. This namelist allows the environment in which JULES is
being run to be specified so that options that are unavailable can be
made inaccessible via the metadata and thus will not appear in the
gui. Warnings can also be issued if options are inappropriately set.

\sphinxAtStartPar
This namelist also describes the flavour of the land surface model
being used. CABLE is in the process of being incorporated into JULES
and other flavours of JULES is in development e.g. a standalone rivers
app.


\subsection{\sphinxstyleliteralintitle{\sphinxupquote{JULES\_MODEL\_ENVIRONMENT}} namelist members}
\label{\detokenize{namelists/model_environment.nml:namelist-JULES_MODEL_ENVIRONMENT}}\label{\detokenize{namelists/model_environment.nml:jules-model-environment-namelist-members}}\index{JULES\_MODEL\_ENVIRONMENT (namelist)@\spxentry{JULES\_MODEL\_ENVIRONMENT}\spxextra{namelist}|spxpagem}\index{l\_jules\_parent (in namelist JULES\_MODEL\_ENVIRONMENT)@\spxentry{l\_jules\_parent}\spxextra{in namelist JULES\_MODEL\_ENVIRONMENT}|spxpagem}

\begin{fulllineitems}
\phantomsection\label{\detokenize{namelists/model_environment.nml:JULES_MODEL_ENVIRONMENT::l_jules_parent}}
\pysigstartsignatures
\pysigline{\sphinxcode{\sphinxupquote{JULES\_MODEL\_ENVIRONMENT::}}\sphinxbfcode{\sphinxupquote{l\_jules\_parent}}}
\pysigstopsignatures\begin{quote}\begin{description}
\sphinxlineitem{Type}
\sphinxAtStartPar
integer

\sphinxlineitem{Default}
\sphinxAtStartPar
imdi

\end{description}\end{quote}

\sphinxAtStartPar
Switch to identify the environment in which JULES is being run. The
switch should only be used to allow science options, which are not
available in the specified model environment, to be trigger ignored
and checked that they are set appropriately at run\sphinxhyphen{}time.
\begin{quote}


\begin{savenotes}\sphinxattablestart
\centering
\begin{tabulary}{\linewidth}[t]{|T|T|}
\hline

\sphinxAtStartPar
0
&
\sphinxAtStartPar
JULES is being run standalone. 
Any options that are only available to the parent model (e.g. the UM) will be trigger ignored.
\\
\hline
\sphinxAtStartPar
1
&
\sphinxAtStartPar
JULES is being run coupled to the UM. 
\\
\hline
\sphinxAtStartPar
2
&
\sphinxAtStartPar
JULES is being run coupled via OASIS (available to Rivers\sphinxhyphen{}only executable only). 
Options not available to the UM are trigger\sphinxhyphen{}ignored.
\\
\hline
\end{tabulary}
\par
\sphinxattableend\end{savenotes}
\end{quote}

\begin{sphinxadmonition}{warning}{Warning:}
\sphinxAtStartPar
No science code should be associated with this switch, only what
science options are available.
\end{sphinxadmonition}

\begin{sphinxadmonition}{note}{Note:}
\sphinxAtStartPar
The metadata of the parent model only actually allows the
appropriate option to be specified i.e. in standalone only 0 is
permitted and in the UM only 1 is permitted. Any other parent
models are listed here for information only. It is not
appropriate to include a list of the unavailable options
here. However, information for namelists that have been
consolidated will appear in the following checking routines as
they are completed.
\begin{itemize}
\item {} 
\sphinxAtStartPar
\sphinxhref{https://code.metoffice.gov.uk/trac/jules/browser/main/trunk/src/control/standalone/check\_unavailable\_options\_mod.F90}{src/control/standalone/check\_unavailable\_options\_mod.F90}

\item {} 
\sphinxAtStartPar
\sphinxhref{https://code.metoffice.gov.uk/trac/jules/browser/main/trunk/src/control/um/check\_jules\_unavailable\_options\_mod.F90}{src/control/um/check\_jules\_unavailable\_options\_mod.F90}

\end{itemize}
\end{sphinxadmonition}

\end{fulllineitems}

\index{lsm\_id (in namelist JULES\_MODEL\_ENVIRONMENT)@\spxentry{lsm\_id}\spxextra{in namelist JULES\_MODEL\_ENVIRONMENT}|spxpagem}

\begin{fulllineitems}
\phantomsection\label{\detokenize{namelists/model_environment.nml:JULES_MODEL_ENVIRONMENT::lsm_id}}
\pysigstartsignatures
\pysigline{\sphinxcode{\sphinxupquote{JULES\_MODEL\_ENVIRONMENT::}}\sphinxbfcode{\sphinxupquote{lsm\_id}}}
\pysigstopsignatures\begin{quote}\begin{description}
\sphinxlineitem{Type}
\sphinxAtStartPar
integer

\sphinxlineitem{Default}
\sphinxAtStartPar
MDI

\end{description}\end{quote}

\sphinxAtStartPar
Switch for land surface model flavour.
\begin{quote}


\begin{savenotes}\sphinxattablestart
\centering
\begin{tabulary}{\linewidth}[t]{|T|T|}
\hline

\sphinxAtStartPar
1
&
\sphinxAtStartPar
JULES land surface model
\\
\hline
\sphinxAtStartPar
2
&
\sphinxAtStartPar
CABLE land surface model
\\
\hline
\end{tabulary}
\par
\sphinxattableend\end{savenotes}
\end{quote}

\begin{sphinxadmonition}{note}{Note:}
\sphinxAtStartPar
The CABLE model has not yet been implemented within the JULES repository.
\end{sphinxadmonition}

\end{fulllineitems}


\sphinxstepscope


\section{\sphinxstyleliteralintitle{\sphinxupquote{jules\_surface.nml}}}
\label{\detokenize{namelists/jules_surface.nml:jules-surface-nml}}\label{\detokenize{namelists/jules_surface.nml::doc}}
\sphinxAtStartPar
This file sets the surface options. It contains one namelist called {\hyperref[\detokenize{namelists/jules_surface.nml:namelist-JULES_SURFACE}]{\sphinxcrossref{\sphinxcode{\sphinxupquote{JULES\_SURFACE}}}}}.


\subsection{\sphinxstyleliteralintitle{\sphinxupquote{JULES\_SURFACE}} namelist members}
\label{\detokenize{namelists/jules_surface.nml:namelist-JULES_SURFACE}}\label{\detokenize{namelists/jules_surface.nml:jules-surface-namelist-members}}\index{JULES\_SURFACE (namelist)@\spxentry{JULES\_SURFACE}\spxextra{namelist}|spxpagem}\index{all\_tiles (in namelist JULES\_SURFACE)@\spxentry{all\_tiles}\spxextra{in namelist JULES\_SURFACE}|spxpagem}

\begin{fulllineitems}
\phantomsection\label{\detokenize{namelists/jules_surface.nml:JULES_SURFACE::all_tiles}}
\pysigstartsignatures
\pysigline{\sphinxcode{\sphinxupquote{JULES\_SURFACE::}}\sphinxbfcode{\sphinxupquote{all\_tiles}}}
\pysigstopsignatures\begin{quote}\begin{description}
\sphinxlineitem{Type}
\sphinxAtStartPar
integer

\sphinxlineitem{Permitted}
\sphinxAtStartPar
0,1

\sphinxlineitem{Default}
\sphinxAtStartPar
0

\end{description}\end{quote}

\sphinxAtStartPar
Perform calculations of tile properties on all tiles (except land ice) for all gridpoints even when the tile fraction is zero.
\begin{enumerate}
\sphinxsetlistlabels{\arabic}{enumi}{enumii}{}{.}%
\setcounter{enumi}{-1}
\item {} 
\sphinxAtStartPar
Off

\item {} 
\sphinxAtStartPar
On

\end{enumerate}

\end{fulllineitems}

\index{cor\_mo\_iter (in namelist JULES\_SURFACE)@\spxentry{cor\_mo\_iter}\spxextra{in namelist JULES\_SURFACE}|spxpagem}

\begin{fulllineitems}
\phantomsection\label{\detokenize{namelists/jules_surface.nml:JULES_SURFACE::cor_mo_iter}}
\pysigstartsignatures
\pysigline{\sphinxcode{\sphinxupquote{JULES\_SURFACE::}}\sphinxbfcode{\sphinxupquote{cor\_mo\_iter}}}
\pysigstopsignatures\begin{quote}\begin{description}
\sphinxlineitem{Type}
\sphinxAtStartPar
integer

\sphinxlineitem{Permitted}
\sphinxAtStartPar
1\sphinxhyphen{}4

\sphinxlineitem{Default}
\sphinxAtStartPar
1

\end{description}\end{quote}

\sphinxAtStartPar
Corrections to Monin\sphinxhyphen{}Obukhov surface exchange calculation. Please see also \sphinxhref{https://code.metoffice.gov.uk/doc/um/latest/papers/umdp\_024.pdf}{UMDP24 “The Parametrization of Boundary Layer Processes” (section 8.4.1)}.
\begin{enumerate}
\sphinxsetlistlabels{\arabic}{enumi}{enumii}{}{.}%
\item {} 
\sphinxAtStartPar
Correct convective gustiness in low winds

\item {} 
\sphinxAtStartPar
Correct U* in dust scheme,

\item {} 
\sphinxAtStartPar
Limit Obukhov length in low winds

\item {} 
\sphinxAtStartPar
Improve the initialisation of the iteration

\end{enumerate}

\begin{sphinxadmonition}{note}{Note:}
\sphinxAtStartPar
Option 4 should be the preferred option.
\end{sphinxadmonition}

\end{fulllineitems}

\index{beta\_cnv\_bl (in namelist JULES\_SURFACE)@\spxentry{beta\_cnv\_bl}\spxextra{in namelist JULES\_SURFACE}|spxpagem}

\begin{fulllineitems}
\phantomsection\label{\detokenize{namelists/jules_surface.nml:JULES_SURFACE::beta_cnv_bl}}
\pysigstartsignatures
\pysigline{\sphinxcode{\sphinxupquote{JULES\_SURFACE::}}\sphinxbfcode{\sphinxupquote{beta\_cnv\_bl}}}
\pysigstopsignatures\begin{quote}\begin{description}
\sphinxlineitem{Type}
\sphinxAtStartPar
real

\sphinxlineitem{Permitted}
\sphinxAtStartPar
\textgreater{}=0.0

\end{description}\end{quote}

\sphinxAtStartPar
Dimensionless coefficient scaling the boundary layer convective
gustiness contribution to surface exchange.  Historically this was
set to 0.08 but is recommended to be reduced to 0.04 when gustiness
from convective downdraughts is included, either from the
convection parametrization or when convection is resolved (so
resolutions \textasciitilde{}1km or finer). Please see also \sphinxhref{https://code.metoffice.gov.uk/doc/um/latest/papers/umdp\_024.pdf}{UMDP24 “The
Parametrization of Boundary Layer Processes” (section 8.1)}.

\end{fulllineitems}

\index{l\_aggregate (in namelist JULES\_SURFACE)@\spxentry{l\_aggregate}\spxextra{in namelist JULES\_SURFACE}|spxpagem}

\begin{fulllineitems}
\phantomsection\label{\detokenize{namelists/jules_surface.nml:JULES_SURFACE::l_aggregate}}
\pysigstartsignatures
\pysigline{\sphinxcode{\sphinxupquote{JULES\_SURFACE::}}\sphinxbfcode{\sphinxupquote{l\_aggregate}}}
\pysigstopsignatures\begin{quote}\begin{description}
\sphinxlineitem{Type}
\sphinxAtStartPar
logical

\sphinxlineitem{Default}
\sphinxAtStartPar
F

\end{description}\end{quote}

\sphinxAtStartPar
Switch controlling number of surface tiles for each gridbox.

\sphinxAtStartPar
This is used to set the number of surface energy balances that are solved for each gridbox (\sphinxcode{\sphinxupquote{nsurft}}).
\begin{description}
\sphinxlineitem{TRUE}
\sphinxAtStartPar
Aggregate parameter values are used to solve a single energy balance per gridbox. This option sets \sphinxcode{\sphinxupquote{nsurft = 1}}.

\sphinxlineitem{FALSE}
\sphinxAtStartPar
A separate energy balance is calculated for each surface type. This option sets \sphinxcode{\sphinxupquote{nsurft = ntype}}.

\end{description}

\end{fulllineitems}

\index{i\_aggregate\_opt (in namelist JULES\_SURFACE)@\spxentry{i\_aggregate\_opt}\spxextra{in namelist JULES\_SURFACE}|spxpagem}

\begin{fulllineitems}
\phantomsection\label{\detokenize{namelists/jules_surface.nml:JULES_SURFACE::i_aggregate_opt}}
\pysigstartsignatures
\pysigline{\sphinxcode{\sphinxupquote{JULES\_SURFACE::}}\sphinxbfcode{\sphinxupquote{i\_aggregate\_opt}}}
\pysigstopsignatures\begin{quote}\begin{description}
\sphinxlineitem{Type}
\sphinxAtStartPar
integer

\sphinxlineitem{Permitted}
\sphinxAtStartPar
0\sphinxhyphen{}1

\sphinxlineitem{Default}
\sphinxAtStartPar
0

\end{description}\end{quote}

\sphinxAtStartPar
Option for aggregating surface properties to surface tiles:
\begin{enumerate}
\sphinxsetlistlabels{\arabic}{enumi}{enumii}{}{.}%
\setcounter{enumi}{-1}
\item {} 
\sphinxAtStartPar
Aggregate momentum roughness lengths and set the thermal roughness length as a given fraction of this (in practice the ratio of roughness lengths for the first surface type).

\item {} 
\sphinxAtStartPar
Aggregate the thermal roughness lengths separately from the momentum roughness lengths using an analogous algorithm.

\end{enumerate}

\begin{sphinxadmonition}{note}{Note:}
\sphinxAtStartPar
This option is ignored unless {\hyperref[\detokenize{namelists/jules_surface.nml:JULES_SURFACE::l_aggregate}]{\sphinxcrossref{\sphinxcode{\sphinxupquote{l\_aggregate}}}}} is true.
\end{sphinxadmonition}

\end{fulllineitems}

\index{l\_epot\_corr (in namelist JULES\_SURFACE)@\spxentry{l\_epot\_corr}\spxextra{in namelist JULES\_SURFACE}|spxpagem}

\begin{fulllineitems}
\phantomsection\label{\detokenize{namelists/jules_surface.nml:JULES_SURFACE::l_epot_corr}}
\pysigstartsignatures
\pysigline{\sphinxcode{\sphinxupquote{JULES\_SURFACE::}}\sphinxbfcode{\sphinxupquote{l\_epot\_corr}}}
\pysigstopsignatures\begin{quote}\begin{description}
\sphinxlineitem{Type}
\sphinxAtStartPar
logical

\sphinxlineitem{Default}
\sphinxAtStartPar
F

\end{description}\end{quote}
\begin{description}
\sphinxlineitem{TRUE}
\sphinxAtStartPar
Use correction to the calculation of potential evaporation.

\sphinxlineitem{FALSE}
\sphinxAtStartPar
No effect.

\end{description}

\end{fulllineitems}

\index{l\_point\_data (in namelist JULES\_SURFACE)@\spxentry{l\_point\_data}\spxextra{in namelist JULES\_SURFACE}|spxpagem}

\begin{fulllineitems}
\phantomsection\label{\detokenize{namelists/jules_surface.nml:JULES_SURFACE::l_point_data}}
\pysigstartsignatures
\pysigline{\sphinxcode{\sphinxupquote{JULES\_SURFACE::}}\sphinxbfcode{\sphinxupquote{l\_point\_data}}}
\pysigstopsignatures\begin{quote}\begin{description}
\sphinxlineitem{Type}
\sphinxAtStartPar
logical

\sphinxlineitem{Default}
\sphinxAtStartPar
F

\end{description}\end{quote}

\sphinxAtStartPar
Flag indicating if driving data are point or area\sphinxhyphen{}average values. This affects the treatment of precipitation input and how snow affects the albedo.
\begin{description}
\sphinxlineitem{TRUE}
\sphinxAtStartPar
Driving data are point data. Precipitation is not distributed in space (see FALSE below) and is all assumed to be large\sphinxhyphen{}scale in origin. The albedo formulation is suitable for a point.

\sphinxlineitem{FALSE}
\sphinxAtStartPar
Driving data are area averages. The precipitation inputs are assumed to be exponentially distributed in space, as in UMDP25, and can include convective and large\sphinxhyphen{}scale components. The albedo formulation is suitable for a gridbox.

\end{description}

\end{fulllineitems}

\index{l\_land\_ice\_imp (in namelist JULES\_SURFACE)@\spxentry{l\_land\_ice\_imp}\spxextra{in namelist JULES\_SURFACE}|spxpagem}

\begin{fulllineitems}
\phantomsection\label{\detokenize{namelists/jules_surface.nml:JULES_SURFACE::l_land_ice_imp}}
\pysigstartsignatures
\pysigline{\sphinxcode{\sphinxupquote{JULES\_SURFACE::}}\sphinxbfcode{\sphinxupquote{l\_land\_ice\_imp}}}
\pysigstopsignatures\begin{quote}\begin{description}
\sphinxlineitem{Type}
\sphinxAtStartPar
logical

\sphinxlineitem{Default}
\sphinxAtStartPar
F

\end{description}\end{quote}

\sphinxAtStartPar
Switch to control the use of implicit numerics to update land ice temperatures.
\begin{description}
\sphinxlineitem{TRUE}
\sphinxAtStartPar
Use implicit numerics to update land ice temperatures.

\sphinxlineitem{FALSE}
\sphinxAtStartPar
Use explicit numerics to update land ice temperatures.

\end{description}

\end{fulllineitems}

\index{l\_anthrop\_heat\_src (in namelist JULES\_SURFACE)@\spxentry{l\_anthrop\_heat\_src}\spxextra{in namelist JULES\_SURFACE}|spxpagem}

\begin{fulllineitems}
\phantomsection\label{\detokenize{namelists/jules_surface.nml:JULES_SURFACE::l_anthrop_heat_src}}
\pysigstartsignatures
\pysigline{\sphinxcode{\sphinxupquote{JULES\_SURFACE::}}\sphinxbfcode{\sphinxupquote{l\_anthrop\_heat\_src}}}
\pysigstopsignatures\begin{quote}\begin{description}
\sphinxlineitem{Type}
\sphinxAtStartPar
logical

\sphinxlineitem{Default}
\sphinxAtStartPar
F

\end{description}\end{quote}

\sphinxAtStartPar
Switch for inclusion of anthropogenic contribution to the surface heat flux from \sphinxstyleemphasis{urban} surface types. If {\hyperref[\detokenize{namelists/jules_surface.nml:JULES_SURFACE::l_urban2t}]{\sphinxcrossref{\sphinxcode{\sphinxupquote{l\_urban2t}}}}} then the anthropogenic heat will be distributed between the {\hyperref[\detokenize{namelists/jules_surface_types.nml:JULES_SURFACE_TYPES::urban_canyon}]{\sphinxcrossref{\sphinxcode{\sphinxupquote{urban\_canyon}}}}} and {\hyperref[\detokenize{namelists/jules_surface_types.nml:JULES_SURFACE_TYPES::urban_roof}]{\sphinxcrossref{\sphinxcode{\sphinxupquote{urban\_roof}}}}} according to {\hyperref[\detokenize{namelists/urban.nml:JULES_URBAN::anthrop_heat_scale}]{\sphinxcrossref{\sphinxcode{\sphinxupquote{anthrop\_heat\_scale}}}}}, otherwise it is added to {\hyperref[\detokenize{namelists/jules_surface_types.nml:JULES_SURFACE_TYPES::urban}]{\sphinxcrossref{\sphinxcode{\sphinxupquote{urban}}}}} only.
\begin{description}
\sphinxlineitem{TRUE}
\sphinxAtStartPar
Add anthropogenic effect.

\sphinxlineitem{FALSE}
\sphinxAtStartPar
No effect.

\end{description}

\end{fulllineitems}

\index{iscrntdiag (in namelist JULES\_SURFACE)@\spxentry{iscrntdiag}\spxextra{in namelist JULES\_SURFACE}|spxpagem}

\begin{fulllineitems}
\phantomsection\label{\detokenize{namelists/jules_surface.nml:JULES_SURFACE::iscrntdiag}}
\pysigstartsignatures
\pysigline{\sphinxcode{\sphinxupquote{JULES\_SURFACE::}}\sphinxbfcode{\sphinxupquote{iscrntdiag}}}
\pysigstopsignatures\begin{quote}\begin{description}
\sphinxlineitem{Type}
\sphinxAtStartPar
integer

\sphinxlineitem{Permitted}
\sphinxAtStartPar
0\sphinxhyphen{}3 (standalone: 0 or 1 only)

\sphinxlineitem{Default}
\sphinxAtStartPar
0

\end{description}\end{quote}

\sphinxAtStartPar
Switch controlling method for diagnosing screen temperature.
\begin{enumerate}
\sphinxsetlistlabels{\arabic}{enumi}{enumii}{}{.}%
\setcounter{enumi}{-1}
\item {} 
\sphinxAtStartPar
Use surface similarity theory (no decoupling).

\item {} 
\sphinxAtStartPar
Use surface similarity theory but allow decoupling in very
stable conditions based on the quasi\sphinxhyphen{}equilibrium radiative
solution.

\item {} 
\sphinxAtStartPar
Diagnose the screen temperature including transient effects and
radiative cooling.

\item {} 
\sphinxAtStartPar
Diagnose the screen temperature and humidity including transient
effects and radiative cooling. The diagnosis of the screen
temperature follows option 2. This is an experimental option and
is undergoing development and additional testing.

\end{enumerate}

\begin{sphinxadmonition}{note}{Note:}
\sphinxAtStartPar
Option 0 should be the preferred option in standalone i.e. no decoupling until the decoupled options are fully tested in standalone scenarios.
\end{sphinxadmonition}

\end{fulllineitems}

\index{l\_elev\_lw\_down (in namelist JULES\_SURFACE)@\spxentry{l\_elev\_lw\_down}\spxextra{in namelist JULES\_SURFACE}|spxpagem}

\begin{fulllineitems}
\phantomsection\label{\detokenize{namelists/jules_surface.nml:JULES_SURFACE::l_elev_lw_down}}
\pysigstartsignatures
\pysigline{\sphinxcode{\sphinxupquote{JULES\_SURFACE::}}\sphinxbfcode{\sphinxupquote{l\_elev\_lw\_down}}}
\pysigstopsignatures\begin{quote}\begin{description}
\sphinxlineitem{Type}
\sphinxAtStartPar
logical

\sphinxlineitem{Default}
\sphinxAtStartPar
false

\end{description}\end{quote}

\sphinxAtStartPar
If surface tiles are set to be at an elevation offset from the gridbox mean altitude (see {\hyperref[\detokenize{namelists/model_grid.nml:namelist-JULES_SURF_HGT}]{\sphinxcrossref{\sphinxcode{\sphinxupquote{JULES\_SURF\_HGT}}}}}) this switch controls
whether downwelling longwave radiation is adjusted along with surface air temperature and relative humidity.

\sphinxAtStartPar
If true, the downwelling longwave for each surface tile not at the gridbox mean height is adjusted by an amount
proportional to the fourth power of the adjustment that has been made to the surface air temperature. The adjustments are then
scaled such that the sum over all surface tiles conserves the gridbox mean energy in the original forcing.

\end{fulllineitems}

\index{l\_elev\_land\_ice (in namelist JULES\_SURFACE)@\spxentry{l\_elev\_land\_ice}\spxextra{in namelist JULES\_SURFACE}|spxpagem}

\begin{fulllineitems}
\phantomsection\label{\detokenize{namelists/jules_surface.nml:JULES_SURFACE::l_elev_land_ice}}
\pysigstartsignatures
\pysigline{\sphinxcode{\sphinxupquote{JULES\_SURFACE::}}\sphinxbfcode{\sphinxupquote{l\_elev\_land\_ice}}}
\pysigstopsignatures\begin{quote}\begin{description}
\sphinxlineitem{Type}
\sphinxAtStartPar
logical

\sphinxlineitem{Default}
\sphinxAtStartPar
false

\end{description}\end{quote}

\sphinxAtStartPar
Allows multiple ice surface tiles to exist in an ice gridbox, usually with each representing a different elevation ({\hyperref[\detokenize{namelists/model_grid.nml:namelist-JULES_SURF_HGT}]{\sphinxcrossref{\sphinxcode{\sphinxupquote{JULES\_SURF\_HGT}}}}})
band on in icesheet areas so that a sub\sphinxhyphen{}gridscale surface mass balance term (a strong function of altitude) can be derived for forcing
icesheet/glacier models.  When enabled, ice tiles in a gridbox do not use the usual (gridbox mean) JULES soil/ice subsurface model,
but each tile has an independent single layer bedrock\sphinxhyphen{}type solid ice boundary condition under the snowpack.

\sphinxAtStartPar
In addition, when selected, dense snowpacks on elevated ice gridboxes are parameterised to behave more like firn in two ways:
1) The meltwater\sphinxhyphen{}holding capacity of snow layers reduces as a linear function of their density, becoming zero
above the pore\sphinxhyphen{}closure density of 850 kg/m\textasciicircum{}2 so as to restrict retention of melt within the snowpack.
2) Where the top few centimetres of the pack has a density appropriate to firn/bare ice
and the grain\sphinxhyphen{}size physics otherwise used for snow albedo become less appropriate,
surface albedo becomes a function of density, tending towards that of bare ice as density increases
(see {\hyperref[\detokenize{namelists/jules_snow.nml:JULES_SNOW::rho_firn_albedo}]{\sphinxcrossref{\sphinxcode{\sphinxupquote{rho\_firn\_albedo}}}}}, {\hyperref[\detokenize{namelists/jules_snow.nml:JULES_SNOW::amax}]{\sphinxcrossref{\sphinxcode{\sphinxupquote{amax}}}}}, {\hyperref[\detokenize{namelists/jules_snow.nml:JULES_SNOW::aicemax}]{\sphinxcrossref{\sphinxcode{\sphinxupquote{aicemax}}}}}).

\sphinxAtStartPar
If this scheme is enabled, a depth for the bedrock layer must be provided ({\hyperref[\detokenize{namelists/jules_soil.nml:JULES_SOIL::dzsoil_elev}]{\sphinxcrossref{\sphinxcode{\sphinxupquote{dzsoil\_elev}}}}}) and the new tile
numbers must be specified ({\hyperref[\detokenize{namelists/jules_surface_types.nml:namelist-JULES_SURFACE_TYPES}]{\sphinxcrossref{\sphinxcode{\sphinxupquote{JULES\_SURFACE\_TYPES}}}}}) as either type {\hyperref[\detokenize{namelists/jules_surface_types.nml:JULES_SURFACE_TYPES::elev_ice}]{\sphinxcrossref{\sphinxcode{\sphinxupquote{elev\_ice}}}}} (for fully glaciated areas) or {\hyperref[\detokenize{namelists/jules_surface_types.nml:JULES_SURFACE_TYPES::elev_rock}]{\sphinxcrossref{\sphinxcode{\sphinxupquote{elev\_rock}}}}} (for
non\sphinxhyphen{}glaciated areas where the bedrock may become exposed under a thin snow layer). The total number of non\sphinxhyphen{}vegetated surface tiles, and
their surface properties ({\hyperref[\detokenize{namelists/nveg_params.nml:namelist-JULES_NVEGPARM}]{\sphinxcrossref{\sphinxcode{\sphinxupquote{JULES\_NVEGPARM}}}}}, usually set to be the same as the normal ice tile) must be set accordingly,
as with any surface tile.

\end{fulllineitems}

\index{l\_flake\_model (in namelist JULES\_SURFACE)@\spxentry{l\_flake\_model}\spxextra{in namelist JULES\_SURFACE}|spxpagem}

\begin{fulllineitems}
\phantomsection\label{\detokenize{namelists/jules_surface.nml:JULES_SURFACE::l_flake_model}}
\pysigstartsignatures
\pysigline{\sphinxcode{\sphinxupquote{JULES\_SURFACE::}}\sphinxbfcode{\sphinxupquote{l\_flake\_model}}}
\pysigstopsignatures\begin{quote}\begin{description}
\sphinxlineitem{Type}
\sphinxAtStartPar
logical

\sphinxlineitem{Default}
\sphinxAtStartPar
false

\end{description}\end{quote}

\sphinxAtStartPar
Switch for using the freshwater lake model ‘FLake’ on the lake/inland\sphinxhyphen{}water surface tile. More information on the FLake model can be found on \sphinxhref{http://www.flake.igb-berlin.de/}{the FLake website}. A description of how FLake is coupled to JULES can be found in \sphinxhref{http://www.borenv.net/BER/pdfs/ber15/ber15-501.pdf}{Rooney and Jones 2010}.

\sphinxAtStartPar
When using FLake, it is not necessary to use a canopy representation of lake properties so {\hyperref[\detokenize{namelists/nveg_params.nml:JULES_NVEGPARM::catch_nvg_io}]{\sphinxcrossref{\sphinxcode{\sphinxupquote{catch\_nvg\_io}}}}},
{\hyperref[\detokenize{namelists/nveg_params.nml:JULES_NVEGPARM::ch_nvg_io}]{\sphinxcrossref{\sphinxcode{\sphinxupquote{ch\_nvg\_io}}}}} and {\hyperref[\detokenize{namelists/nveg_params.nml:JULES_NVEGPARM::vf_nvg_io}]{\sphinxcrossref{\sphinxcode{\sphinxupquote{vf\_nvg\_io}}}}} should all be set to zero for the lake tile.

\end{fulllineitems}

\index{l\_urban2t (in namelist JULES\_SURFACE)@\spxentry{l\_urban2t}\spxextra{in namelist JULES\_SURFACE}|spxpagem}

\begin{fulllineitems}
\phantomsection\label{\detokenize{namelists/jules_surface.nml:JULES_SURFACE::l_urban2t}}
\pysigstartsignatures
\pysigline{\sphinxcode{\sphinxupquote{JULES\_SURFACE::}}\sphinxbfcode{\sphinxupquote{l\_urban2t}}}
\pysigstopsignatures\begin{quote}\begin{description}
\sphinxlineitem{Type}
\sphinxAtStartPar
logical

\sphinxlineitem{Default}
\sphinxAtStartPar
false

\end{description}\end{quote}

\sphinxAtStartPar
Switch for using the two\sphinxhyphen{}tile urban schemes (including MORUSES). This allows two urban surface tiles ({\hyperref[\detokenize{namelists/jules_surface_types.nml:JULES_SURFACE_TYPES::urban_canyon}]{\sphinxcrossref{\sphinxcode{\sphinxupquote{urban\_canyon}}}}} and {\hyperref[\detokenize{namelists/jules_surface_types.nml:JULES_SURFACE_TYPES::urban_roof}]{\sphinxcrossref{\sphinxcode{\sphinxupquote{urban\_roof}}}}}) to be used instead of one.
Additional parameters must be supplied via {\hyperref[\detokenize{namelists/nveg_params.nml:namelist-JULES_NVEGPARM}]{\sphinxcrossref{\sphinxcode{\sphinxupquote{JULES\_NVEGPARM}}}}}, with some able to be provided by MORUSES (see {\hyperref[\detokenize{namelists/urban.nml:namelist-JULES_URBAN}]{\sphinxcrossref{\sphinxcode{\sphinxupquote{JULES\_URBAN}}}}}).

\end{fulllineitems}

\index{l\_mo\_buoyancy\_calc (in namelist JULES\_SURFACE)@\spxentry{l\_mo\_buoyancy\_calc}\spxextra{in namelist JULES\_SURFACE}|spxpagem}

\begin{fulllineitems}
\phantomsection\label{\detokenize{namelists/jules_surface.nml:JULES_SURFACE::l_mo_buoyancy_calc}}
\pysigstartsignatures
\pysigline{\sphinxcode{\sphinxupquote{JULES\_SURFACE::}}\sphinxbfcode{\sphinxupquote{l\_mo\_buoyancy\_calc}}}
\pysigstopsignatures\begin{quote}\begin{description}
\sphinxlineitem{Type}
\sphinxAtStartPar
logical

\sphinxlineitem{Default}
\sphinxAtStartPar
false

\end{description}\end{quote}

\sphinxAtStartPar
Default JULES (l\_mo\_buoyancy\_flux = false) uses the buoyancy from the previous timestep to calculate the surface transfer coefficients. In coupled simulations this can lead to unrealistic surface temperatures if the stability suddenly
switches from stable to unstable, due to the low turbulence determined by the stable buoyancy flux.

\sphinxAtStartPar
With the interactive buoyancy flux option (l\_mo\_buoyancy\_flux = true) the surface energy balance and buoyancy flux are calculated within the iterative calculation for the Monin\sphinxhyphen{}Obukhov similarity theory for the surface exchange
coefficients. On occations when the stability is around neutral it is possible that the iterative calculation does not converge. In this case the larger of the last two calculated transfer coefficients is then used to prevent
any unrealistic surface temperatures.

\end{fulllineitems}


\begin{sphinxadmonition}{note}{Surface parameters}
\index{hleaf (in namelist JULES\_SURFACE)@\spxentry{hleaf}\spxextra{in namelist JULES\_SURFACE}|spxpagem}

\begin{fulllineitems}
\phantomsection\label{\detokenize{namelists/jules_surface.nml:JULES_SURFACE::hleaf}}
\pysigstartsignatures
\pysigline{\sphinxcode{\sphinxupquote{JULES\_SURFACE::}}\sphinxbfcode{\sphinxupquote{hleaf}}}
\pysigstopsignatures\begin{quote}\begin{description}
\sphinxlineitem{Type}
\sphinxAtStartPar
real

\sphinxlineitem{Default}
\sphinxAtStartPar
5.7e4

\end{description}\end{quote}

\sphinxAtStartPar
Specific heat capacity of leaves (J K$^{\text{\sphinxhyphen{}1}}$ per kg carbon).

\sphinxAtStartPar
See Hadley Centre Technical Note 30, p6, available from \sphinxhref{http://www.metoffice.gov.uk/learning/library/publications/science/climate-science-technical-notes}{the Met Office Library}.

\end{fulllineitems}

\index{hwood (in namelist JULES\_SURFACE)@\spxentry{hwood}\spxextra{in namelist JULES\_SURFACE}|spxpagem}

\begin{fulllineitems}
\phantomsection\label{\detokenize{namelists/jules_surface.nml:JULES_SURFACE::hwood}}
\pysigstartsignatures
\pysigline{\sphinxcode{\sphinxupquote{JULES\_SURFACE::}}\sphinxbfcode{\sphinxupquote{hwood}}}
\pysigstopsignatures\begin{quote}\begin{description}
\sphinxlineitem{Type}
\sphinxAtStartPar
real

\sphinxlineitem{Default}
\sphinxAtStartPar
1.1e4

\end{description}\end{quote}

\sphinxAtStartPar
Specific heat capacity of wood (J K$^{\text{\sphinxhyphen{}1}}$ per kg carbon).

\sphinxAtStartPar
See Hadley Centre Technical Note 30, p6, available from \sphinxhref{http://www.metoffice.gov.uk/learning/library/publications/science/climate-science-technical-notes}{the Met Office Library}.

\end{fulllineitems}

\index{beta1 (in namelist JULES\_SURFACE)@\spxentry{beta1}\spxextra{in namelist JULES\_SURFACE}|spxpagem}

\begin{fulllineitems}
\phantomsection\label{\detokenize{namelists/jules_surface.nml:JULES_SURFACE::beta1}}
\pysigstartsignatures
\pysigline{\sphinxcode{\sphinxupquote{JULES\_SURFACE::}}\sphinxbfcode{\sphinxupquote{beta1}}}
\pysigstopsignatures\begin{quote}\begin{description}
\sphinxlineitem{Type}
\sphinxAtStartPar
real

\sphinxlineitem{Default}
\sphinxAtStartPar
0.83

\end{description}\end{quote}

\sphinxAtStartPar
Coupling coefficient for co\sphinxhyphen{}limitation in photosynthesis model.

\sphinxAtStartPar
See Cox et al. (1999), Eq.61.

\end{fulllineitems}

\index{beta2 (in namelist JULES\_SURFACE)@\spxentry{beta2}\spxextra{in namelist JULES\_SURFACE}|spxpagem}

\begin{fulllineitems}
\phantomsection\label{\detokenize{namelists/jules_surface.nml:JULES_SURFACE::beta2}}
\pysigstartsignatures
\pysigline{\sphinxcode{\sphinxupquote{JULES\_SURFACE::}}\sphinxbfcode{\sphinxupquote{beta2}}}
\pysigstopsignatures\begin{quote}\begin{description}
\sphinxlineitem{Type}
\sphinxAtStartPar
real

\sphinxlineitem{Default}
\sphinxAtStartPar
0.93

\end{description}\end{quote}

\sphinxAtStartPar
Coupling coefficient for co\sphinxhyphen{}limitation in photosynthesis model.

\sphinxAtStartPar
See Cox et al. (1999), Eq.61.

\end{fulllineitems}

\index{fwe\_c3 (in namelist JULES\_SURFACE)@\spxentry{fwe\_c3}\spxextra{in namelist JULES\_SURFACE}|spxpagem}

\begin{fulllineitems}
\phantomsection\label{\detokenize{namelists/jules_surface.nml:JULES_SURFACE::fwe_c3}}
\pysigstartsignatures
\pysigline{\sphinxcode{\sphinxupquote{JULES\_SURFACE::}}\sphinxbfcode{\sphinxupquote{fwe\_c3}}}
\pysigstopsignatures\begin{quote}\begin{description}
\sphinxlineitem{Type}
\sphinxAtStartPar
real

\sphinxlineitem{Default}
\sphinxAtStartPar
0.5

\end{description}\end{quote}

\sphinxAtStartPar
Constant in expression for limitation of photosynthesis by transport of products, for C3 plants.

\sphinxAtStartPar
See Cox et al. (1999) Eq.60.

\end{fulllineitems}

\index{fwe\_c4 (in namelist JULES\_SURFACE)@\spxentry{fwe\_c4}\spxextra{in namelist JULES\_SURFACE}|spxpagem}

\begin{fulllineitems}
\phantomsection\label{\detokenize{namelists/jules_surface.nml:JULES_SURFACE::fwe_c4}}
\pysigstartsignatures
\pysigline{\sphinxcode{\sphinxupquote{JULES\_SURFACE::}}\sphinxbfcode{\sphinxupquote{fwe\_c4}}}
\pysigstopsignatures\begin{quote}\begin{description}
\sphinxlineitem{Type}
\sphinxAtStartPar
real

\sphinxlineitem{Default}
\sphinxAtStartPar
20000.0

\end{description}\end{quote}

\sphinxAtStartPar
Constant in expression for limitation of photosynthesis by transport of products, for C4 plants.

\sphinxAtStartPar
See Cox et al. (1999) Eq.60.

\end{fulllineitems}

\end{sphinxadmonition}

\sphinxstepscope


\section{\sphinxstyleliteralintitle{\sphinxupquote{jules\_radiation.nml}}}
\label{\detokenize{namelists/jules_radiation.nml:jules-radiation-nml}}\label{\detokenize{namelists/jules_radiation.nml::doc}}
\sphinxAtStartPar
This file sets the radiation options. It contains one namelist called {\hyperref[\detokenize{namelists/jules_radiation.nml:namelist-JULES_RADIATION}]{\sphinxcrossref{\sphinxcode{\sphinxupquote{JULES\_RADIATION}}}}}.


\subsection{\sphinxstyleliteralintitle{\sphinxupquote{JULES\_RADIATION}} namelist members}
\label{\detokenize{namelists/jules_radiation.nml:namelist-JULES_RADIATION}}\label{\detokenize{namelists/jules_radiation.nml:jules-radiation-namelist-members}}\index{JULES\_RADIATION (namelist)@\spxentry{JULES\_RADIATION}\spxextra{namelist}|spxpagem}\index{l\_cosz (in namelist JULES\_RADIATION)@\spxentry{l\_cosz}\spxextra{in namelist JULES\_RADIATION}|spxpagem}

\begin{fulllineitems}
\phantomsection\label{\detokenize{namelists/jules_radiation.nml:JULES_RADIATION::l_cosz}}
\pysigstartsignatures
\pysigline{\sphinxcode{\sphinxupquote{JULES\_RADIATION::}}\sphinxbfcode{\sphinxupquote{l\_cosz}}}
\pysigstopsignatures\begin{quote}\begin{description}
\sphinxlineitem{Type}
\sphinxAtStartPar
logical

\sphinxlineitem{Default}
\sphinxAtStartPar
T

\end{description}\end{quote}

\sphinxAtStartPar
Switch for calculation of solar zenith angle.
\begin{description}
\sphinxlineitem{TRUE}
\sphinxAtStartPar
Calculate zenith angle.

\sphinxlineitem{FALSE}
\sphinxAtStartPar
Assume constant zenith angle of zero, meaning sun is directly overhead.

\end{description}

\sphinxAtStartPar
n.b. assuming that the sun is directly overhead may overestimate primary productivity if {\hyperref[\detokenize{namelists/jules_vegetation.nml:JULES_VEGETATION::l_triffid}]{\sphinxcrossref{\sphinxcode{\sphinxupquote{l\_triffid}}}}} = TRUE (see GPP on {\hyperref[\detokenize{output-variables:output-variables-section}]{\sphinxcrossref{\DUrole{std,std-ref}{JULES Output variables}}}}).

\end{fulllineitems}

\index{l\_spec\_albedo (in namelist JULES\_RADIATION)@\spxentry{l\_spec\_albedo}\spxextra{in namelist JULES\_RADIATION}|spxpagem}

\begin{fulllineitems}
\phantomsection\label{\detokenize{namelists/jules_radiation.nml:JULES_RADIATION::l_spec_albedo}}
\pysigstartsignatures
\pysigline{\sphinxcode{\sphinxupquote{JULES\_RADIATION::}}\sphinxbfcode{\sphinxupquote{l\_spec\_albedo}}}
\pysigstopsignatures\begin{quote}\begin{description}
\sphinxlineitem{Type}
\sphinxAtStartPar
logical

\sphinxlineitem{Default}
\sphinxAtStartPar
F

\end{description}\end{quote}

\sphinxAtStartPar
Switch for the two\sphinxhyphen{}stream spectral land\sphinxhyphen{}surface albedo model.
\begin{description}
\sphinxlineitem{TRUE}
\sphinxAtStartPar
Use spectral albedo with VIS and NIR components.

\sphinxlineitem{FALSE}
\sphinxAtStartPar
Use a single (averaged) waveband albedo.

\end{description}

\end{fulllineitems}

\index{l\_spec\_alb\_bs (in namelist JULES\_RADIATION)@\spxentry{l\_spec\_alb\_bs}\spxextra{in namelist JULES\_RADIATION}|spxpagem}

\begin{fulllineitems}
\phantomsection\label{\detokenize{namelists/jules_radiation.nml:JULES_RADIATION::l_spec_alb_bs}}
\pysigstartsignatures
\pysigline{\sphinxcode{\sphinxupquote{JULES\_RADIATION::}}\sphinxbfcode{\sphinxupquote{l\_spec\_alb\_bs}}}
\pysigstopsignatures\begin{quote}\begin{description}
\sphinxlineitem{Type}
\sphinxAtStartPar
logical

\sphinxlineitem{Default}
\sphinxAtStartPar
F

\end{description}\end{quote}

\sphinxAtStartPar
Switch for albedo model, when spectral albedo is being used.

\sphinxAtStartPar
Requires {\hyperref[\detokenize{namelists/jules_radiation.nml:JULES_RADIATION::l_spec_albedo}]{\sphinxcrossref{\sphinxcode{\sphinxupquote{l\_spec\_albedo}}}}} = TRUE.
\begin{description}
\sphinxlineitem{TRUE}
\sphinxAtStartPar
Produces a single albedo for use by both the direct and diffuse beams (a ‘blue’ sky albedo). This currently copies the diffuse beam albedo for the direct beam.

\sphinxlineitem{FALSE}
\sphinxAtStartPar
Produces both a direct (‘black’ sky) and a diffuse (‘white’ sky) albedo.

\end{description}

\end{fulllineitems}

\index{l\_niso\_direct (in namelist JULES\_RADIATION)@\spxentry{l\_niso\_direct}\spxextra{in namelist JULES\_RADIATION}|spxpagem}

\begin{fulllineitems}
\phantomsection\label{\detokenize{namelists/jules_radiation.nml:JULES_RADIATION::l_niso_direct}}
\pysigstartsignatures
\pysigline{\sphinxcode{\sphinxupquote{JULES\_RADIATION::}}\sphinxbfcode{\sphinxupquote{l\_niso\_direct}}}
\pysigstopsignatures\begin{quote}\begin{description}
\sphinxlineitem{Type}
\sphinxAtStartPar
logical

\sphinxlineitem{Default}
\sphinxAtStartPar
F

\end{description}\end{quote}

\sphinxAtStartPar
Switch for using full non\sphinxhyphen{}isotropic expression for direct
scattering in plant canopies when using the two\sphinxhyphen{}stream canopy
radiation model.

\sphinxAtStartPar
Requires {\hyperref[\detokenize{namelists/jules_radiation.nml:JULES_RADIATION::l_spec_albedo}]{\sphinxcrossref{\sphinxcode{\sphinxupquote{l\_spec\_albedo}}}}} = TRUE.
\begin{description}
\sphinxlineitem{TRUE}
\sphinxAtStartPar
Use full non\sphinxhyphen{}isotropic expression for scattering in plant canopies.

\sphinxlineitem{FALSE}
\sphinxAtStartPar
Use the original isotropic expression.

\end{description}

\end{fulllineitems}

\index{l\_snow\_albedo (in namelist JULES\_RADIATION)@\spxentry{l\_snow\_albedo}\spxextra{in namelist JULES\_RADIATION}|spxpagem}

\begin{fulllineitems}
\phantomsection\label{\detokenize{namelists/jules_radiation.nml:JULES_RADIATION::l_snow_albedo}}
\pysigstartsignatures
\pysigline{\sphinxcode{\sphinxupquote{JULES\_RADIATION::}}\sphinxbfcode{\sphinxupquote{l\_snow\_albedo}}}
\pysigstopsignatures\begin{quote}\begin{description}
\sphinxlineitem{Type}
\sphinxAtStartPar
logical

\sphinxlineitem{Default}
\sphinxAtStartPar
F

\end{description}\end{quote}

\sphinxAtStartPar
Switch for using prognostic snow properties, which represents the
effect of snow aging and soot deposition, in model albedo.

\sphinxAtStartPar
Requires {\hyperref[\detokenize{namelists/jules_radiation.nml:JULES_RADIATION::l_spec_albedo}]{\sphinxcrossref{\sphinxcode{\sphinxupquote{l\_spec\_albedo}}}}} = TRUE.
\begin{description}
\sphinxlineitem{TRUE}
\sphinxAtStartPar
Use prognostic snow properties for albedo.

\sphinxlineitem{FALSE}
\sphinxAtStartPar
Calculate albedo of snow using only snow depth.

\end{description}

\end{fulllineitems}

\index{l\_embedded\_snow (in namelist JULES\_RADIATION)@\spxentry{l\_embedded\_snow}\spxextra{in namelist JULES\_RADIATION}|spxpagem}

\begin{fulllineitems}
\phantomsection\label{\detokenize{namelists/jules_radiation.nml:JULES_RADIATION::l_embedded_snow}}
\pysigstartsignatures
\pysigline{\sphinxcode{\sphinxupquote{JULES\_RADIATION::}}\sphinxbfcode{\sphinxupquote{l\_embedded\_snow}}}
\pysigstopsignatures\begin{quote}\begin{description}
\sphinxlineitem{Type}
\sphinxAtStartPar
logical

\sphinxlineitem{Default}
\sphinxAtStartPar
F

\end{description}\end{quote}

\sphinxAtStartPar
Switch to account for pft LAI and pft height in calculation of snow albedo.
\begin{description}
\sphinxlineitem{TRUE}
\sphinxAtStartPar
Use the embedded canopy snow albedo model. This is exclusive of {\hyperref[\detokenize{namelists/jules_radiation.nml:JULES_RADIATION::l_snow_albedo}]{\sphinxcrossref{\sphinxcode{\sphinxupquote{l\_snow\_albedo}}}}}.

\sphinxlineitem{FALSE}
\sphinxAtStartPar
No effect.

\end{description}

\end{fulllineitems}

\index{l\_mask\_snow\_orog (in namelist JULES\_RADIATION)@\spxentry{l\_mask\_snow\_orog}\spxextra{in namelist JULES\_RADIATION}|spxpagem}

\begin{fulllineitems}
\phantomsection\label{\detokenize{namelists/jules_radiation.nml:JULES_RADIATION::l_mask_snow_orog}}
\pysigstartsignatures
\pysigline{\sphinxcode{\sphinxupquote{JULES\_RADIATION::}}\sphinxbfcode{\sphinxupquote{l\_mask\_snow\_orog}}}
\pysigstopsignatures\begin{quote}\begin{description}
\sphinxlineitem{Type}
\sphinxAtStartPar
logical

\sphinxlineitem{Default}
\sphinxAtStartPar
F

\end{description}\end{quote}

\sphinxAtStartPar
Switch for orographic masking of snow, which decreases the albedo
of snow in mountainous regions.
\begin{description}
\sphinxlineitem{TRUE}
\sphinxAtStartPar
Include orographic masking of snow in calculating albedo.

\sphinxlineitem{FALSE}
\sphinxAtStartPar
No effect.

\end{description}

\end{fulllineitems}

\index{l\_albedo\_obs (in namelist JULES\_RADIATION)@\spxentry{l\_albedo\_obs}\spxextra{in namelist JULES\_RADIATION}|spxpagem}

\begin{fulllineitems}
\phantomsection\label{\detokenize{namelists/jules_radiation.nml:JULES_RADIATION::l_albedo_obs}}
\pysigstartsignatures
\pysigline{\sphinxcode{\sphinxupquote{JULES\_RADIATION::}}\sphinxbfcode{\sphinxupquote{l\_albedo\_obs}}}
\pysigstopsignatures\begin{quote}\begin{description}
\sphinxlineitem{Type}
\sphinxAtStartPar
logical

\sphinxlineitem{Default}
\sphinxAtStartPar
F

\end{description}\end{quote}

\sphinxAtStartPar
Switch for applying a scaling factor to the albedo values, on
surface tiles, so that the resultant aggregate albedo matches
observations. The supplied albedos should be from an observed
climatology or analysis system and be supplied via an ancillary
file.
\begin{description}
\sphinxlineitem{TRUE}
\sphinxAtStartPar
Scale the albedo values on tiles within the physical limits
supplied in {\hyperref[\detokenize{namelists/pft_params.nml:namelist-JULES_PFTPARM}]{\sphinxcrossref{\sphinxcode{\sphinxupquote{JULES\_PFTPARM}}}}} and
{\hyperref[\detokenize{namelists/nveg_params.nml:namelist-JULES_NVEGPARM}]{\sphinxcrossref{\sphinxcode{\sphinxupquote{JULES\_NVEGPARM}}}}}. When
{\hyperref[\detokenize{namelists/jules_radiation.nml:JULES_RADIATION::l_spec_albedo}]{\sphinxcrossref{\sphinxcode{\sphinxupquote{l\_spec\_albedo}}}}} = TRUE, VIS and NIR
components are required and when
{\hyperref[\detokenize{namelists/jules_radiation.nml:JULES_RADIATION::l_spec_albedo}]{\sphinxcrossref{\sphinxcode{\sphinxupquote{l\_spec\_albedo}}}}} = FALSE the single
(averaged) waveband albedo is required.

\begin{sphinxadmonition}{note}{Note:}
\sphinxAtStartPar
Observed albedo(s) must be prescribed in
{\hyperref[\detokenize{namelists/prescribed_data.nml::doc}]{\sphinxcrossref{\DUrole{doc}{prescribed\_data.nml}}}}.
\end{sphinxadmonition}

\sphinxlineitem{FALSE}
\sphinxAtStartPar
Do not scale the albedo values on tiles.

\end{description}

\end{fulllineitems}

\index{l\_spec\_sea\_alb (in namelist JULES\_RADIATION)@\spxentry{l\_spec\_sea\_alb}\spxextra{in namelist JULES\_RADIATION}|spxpagem}

\begin{fulllineitems}
\phantomsection\label{\detokenize{namelists/jules_radiation.nml:JULES_RADIATION::l_spec_sea_alb}}
\pysigstartsignatures
\pysigline{\sphinxcode{\sphinxupquote{JULES\_RADIATION::}}\sphinxbfcode{\sphinxupquote{l\_spec\_sea\_alb}}}
\pysigstopsignatures\begin{quote}\begin{description}
\sphinxlineitem{Type}
\sphinxAtStartPar
logical

\sphinxlineitem{Default}
\sphinxAtStartPar
F

\end{description}\end{quote}

\sphinxAtStartPar
Switch to use spectrally varying open sea albedos
\begin{description}
\sphinxlineitem{TRUE}
\sphinxAtStartPar
When {\hyperref[\detokenize{namelists/jules_radiation.nml:JULES_RADIATION::i_sea_alb_method}]{\sphinxcrossref{\sphinxcode{\sphinxupquote{i\_sea\_alb\_method}}}}} = 1 or 2,
spectrally varying sea albedos are produced only when the spectral
file contains 6 SW bands identical to those used in HadGEM1.

\sphinxAtStartPar
When {\hyperref[\detokenize{namelists/jules_radiation.nml:JULES_RADIATION::i_sea_alb_method}]{\sphinxcrossref{\sphinxcode{\sphinxupquote{i\_sea\_alb\_method}}}}} = 3, the spectral
variability is calculated as per the Jin et al. (2011)
parameterisation.

\sphinxlineitem{FALSE}
\sphinxAtStartPar
Uses the calculated broadband sea albedo instead.

\end{description}

\end{fulllineitems}

\index{i\_sea\_alb\_method (in namelist JULES\_RADIATION)@\spxentry{i\_sea\_alb\_method}\spxextra{in namelist JULES\_RADIATION}|spxpagem}

\begin{fulllineitems}
\phantomsection\label{\detokenize{namelists/jules_radiation.nml:JULES_RADIATION::i_sea_alb_method}}
\pysigstartsignatures
\pysigline{\sphinxcode{\sphinxupquote{JULES\_RADIATION::}}\sphinxbfcode{\sphinxupquote{i\_sea\_alb\_method}}}
\pysigstopsignatures\begin{quote}\begin{description}
\sphinxlineitem{Type}
\sphinxAtStartPar
integer

\sphinxlineitem{Default}
\sphinxAtStartPar
None

\end{description}\end{quote}

\sphinxAtStartPar
Choice of model for the Ocean Surface Albedo (open water, ice free)
\begin{enumerate}
\sphinxsetlistlabels{\arabic}{enumi}{enumii}{}{.}%
\item {} 
\sphinxAtStartPar
Diffuse albedo constant (0.06), direct albedo from Briegleb and
Ramanathan (1982).

\item {} 
\sphinxAtStartPar
Diffuse albedo constant (0.06), direct albedo from Barker and
Li (1995).

\item {} 
\sphinxAtStartPar
Direct and diffuse albedo from Jin et al. (2011).

\item {} 
\sphinxAtStartPar
Fixed global value, defined by {\hyperref[\detokenize{namelists/jules_radiation.nml:JULES_RADIATION::fixed_sea_albedo}]{\sphinxcrossref{\sphinxcode{\sphinxupquote{fixed\_sea\_albedo}}}}}.

\item {} 
\sphinxAtStartPar
Fixed global value, defined by
{\hyperref[\detokenize{namelists/jules_radiation.nml:JULES_RADIATION::fixed_sea_albedo}]{\sphinxcrossref{\sphinxcode{\sphinxupquote{fixed\_sea\_albedo}}}}}, above 271K and
variable below this to simulate sea\sphinxhyphen{}ice following Liu et
al. (2007), Joshi \& Haberle (2012) and Turbet et al. (2016).

\end{enumerate}

\end{fulllineitems}

\index{fixed\_sea\_albedo (in namelist JULES\_RADIATION)@\spxentry{fixed\_sea\_albedo}\spxextra{in namelist JULES\_RADIATION}|spxpagem}

\begin{fulllineitems}
\phantomsection\label{\detokenize{namelists/jules_radiation.nml:JULES_RADIATION::fixed_sea_albedo}}
\pysigstartsignatures
\pysigline{\sphinxcode{\sphinxupquote{JULES\_RADIATION::}}\sphinxbfcode{\sphinxupquote{fixed\_sea\_albedo}}}
\pysigstopsignatures\begin{quote}\begin{description}
\sphinxlineitem{Type}
\sphinxAtStartPar
real

\sphinxlineitem{Default}
\sphinxAtStartPar
None

\end{description}\end{quote}

\sphinxAtStartPar
The global value of sea albedo to use if {\hyperref[\detokenize{namelists/jules_radiation.nml:JULES_RADIATION::i_sea_alb_method}]{\sphinxcrossref{\sphinxcode{\sphinxupquote{i\_sea\_alb\_method}}}}} = 4, 5

\end{fulllineitems}

\index{wght\_alb (in namelist JULES\_RADIATION)@\spxentry{wght\_alb}\spxextra{in namelist JULES\_RADIATION}|spxpagem}

\begin{fulllineitems}
\phantomsection\label{\detokenize{namelists/jules_radiation.nml:JULES_RADIATION::wght_alb}}
\pysigstartsignatures
\pysigline{\sphinxcode{\sphinxupquote{JULES\_RADIATION::}}\sphinxbfcode{\sphinxupquote{wght\_alb}}}
\pysigstopsignatures\begin{quote}\begin{description}
\sphinxlineitem{Type}
\sphinxAtStartPar
real(4)

\sphinxlineitem{Default}
\sphinxAtStartPar
MDI

\end{description}\end{quote}

\sphinxAtStartPar
Weights to form the overall albedo from its components (VIS direct,
VIS diffuse, NIR direct, NIR diffuse)
(Ideally, if {\hyperref[\detokenize{namelists/jules_radiation.nml:JULES_RADIATION::l_partition_albsoil}]{\sphinxcrossref{\sphinxcode{\sphinxupquote{l\_partition\_albsoil}}}}} = T,
{\hyperref[\detokenize{namelists/jules_radiation.nml:JULES_RADIATION::wght_alb}]{\sphinxcrossref{\sphinxcode{\sphinxupquote{wght\_alb}}}}} and
{\hyperref[\detokenize{namelists/jules_radiation.nml:JULES_RADIATION::swdn_frac_albsoil}]{\sphinxcrossref{\sphinxcode{\sphinxupquote{swdn\_frac\_albsoil}}}}} should be consistent, with
{\hyperref[\detokenize{namelists/jules_radiation.nml:JULES_RADIATION::swdn_frac_albsoil}]{\sphinxcrossref{\sphinxcode{\sphinxupquote{swdn\_frac\_albsoil}}}}} equal to
\(\sum_{3,4}\)
{\hyperref[\detokenize{namelists/jules_radiation.nml:JULES_RADIATION::wght_alb}]{\sphinxcrossref{\sphinxcode{\sphinxupquote{wght\_alb}}}}} \(/ \sum_1^4\)
{\hyperref[\detokenize{namelists/jules_radiation.nml:JULES_RADIATION::wght_alb}]{\sphinxcrossref{\sphinxcode{\sphinxupquote{wght\_alb}}}}}.
However, {\hyperref[\detokenize{namelists/jules_radiation.nml:JULES_RADIATION::swdn_frac_albsoil}]{\sphinxcrossref{\sphinxcode{\sphinxupquote{swdn\_frac\_albsoil}}}}}
is applied only to bare soil and having a single parameter
is more transparent to the user, while
{\hyperref[\detokenize{namelists/jules_radiation.nml:JULES_RADIATION::wght_alb}]{\sphinxcrossref{\sphinxcode{\sphinxupquote{wght\_alb}}}}}
is used only in diagnostics in standalone JULES and may have
historical settings. Hence, the consistency of these two variables
is not enforced.)

\end{fulllineitems}

\index{l\_hapke\_soil (in namelist JULES\_RADIATION)@\spxentry{l\_hapke\_soil}\spxextra{in namelist JULES\_RADIATION}|spxpagem}

\begin{fulllineitems}
\phantomsection\label{\detokenize{namelists/jules_radiation.nml:JULES_RADIATION::l_hapke_soil}}
\pysigstartsignatures
\pysigline{\sphinxcode{\sphinxupquote{JULES\_RADIATION::}}\sphinxbfcode{\sphinxupquote{l\_hapke\_soil}}}
\pysigstopsignatures\begin{quote}\begin{description}
\sphinxlineitem{Type}
\sphinxAtStartPar
logical

\sphinxlineitem{Default}
\sphinxAtStartPar
F

\end{description}\end{quote}

\sphinxAtStartPar
Switch to enable Hapke’s model of soil albedo to include a zenith\sphinxhyphen{}angle
dependence
\begin{description}
\sphinxlineitem{TRUE}
\sphinxAtStartPar
Apply a zenith\sphinxhyphen{}angle dependence to the direct albedo.

\sphinxlineitem{FALSE}
\sphinxAtStartPar
Use the diffuse albedo for the direct beam as well.

\end{description}

\end{fulllineitems}

\index{l\_partition\_albsoil (in namelist JULES\_RADIATION)@\spxentry{l\_partition\_albsoil}\spxextra{in namelist JULES\_RADIATION}|spxpagem}

\begin{fulllineitems}
\phantomsection\label{\detokenize{namelists/jules_radiation.nml:JULES_RADIATION::l_partition_albsoil}}
\pysigstartsignatures
\pysigline{\sphinxcode{\sphinxupquote{JULES\_RADIATION::}}\sphinxbfcode{\sphinxupquote{l\_partition\_albsoil}}}
\pysigstopsignatures\begin{quote}\begin{description}
\sphinxlineitem{Type}
\sphinxAtStartPar
logical

\sphinxlineitem{Default}
\sphinxAtStartPar
F

\end{description}\end{quote}

\sphinxAtStartPar
Switch to apply a spectral partitioning to the soil albedo.
\begin{description}
\sphinxlineitem{TRUE}
\sphinxAtStartPar
Partition the soil albedo between the visible and near infrared parts
of the spectrum using {\hyperref[\detokenize{namelists/jules_radiation.nml:JULES_RADIATION::ratio_albsoil}]{\sphinxcrossref{\sphinxcode{\sphinxupquote{ratio\_albsoil}}}}} and
{\hyperref[\detokenize{namelists/jules_radiation.nml:JULES_RADIATION::swdn_frac_albsoil}]{\sphinxcrossref{\sphinxcode{\sphinxupquote{swdn\_frac\_albsoil}}}}}.

\sphinxlineitem{FALSE}
\sphinxAtStartPar
Apply the broadband albedo in both spectral regions.

\end{description}

\end{fulllineitems}

\index{ratio\_albsoil (in namelist JULES\_RADIATION)@\spxentry{ratio\_albsoil}\spxextra{in namelist JULES\_RADIATION}|spxpagem}

\begin{fulllineitems}
\phantomsection\label{\detokenize{namelists/jules_radiation.nml:JULES_RADIATION::ratio_albsoil}}
\pysigstartsignatures
\pysigline{\sphinxcode{\sphinxupquote{JULES\_RADIATION::}}\sphinxbfcode{\sphinxupquote{ratio\_albsoil}}}
\pysigstopsignatures\begin{quote}\begin{description}
\sphinxlineitem{Type}
\sphinxAtStartPar
real

\sphinxlineitem{Default}
\sphinxAtStartPar
MDI

\end{description}\end{quote}

\sphinxAtStartPar
Ratio of the NIR to the VIS albedo of bare soil.
Used if {\hyperref[\detokenize{namelists/jules_radiation.nml:JULES_RADIATION::l_partition_albsoil}]{\sphinxcrossref{\sphinxcode{\sphinxupquote{l\_partition\_albsoil}}}}} = T.

\end{fulllineitems}

\index{swdn\_frac\_albsoil (in namelist JULES\_RADIATION)@\spxentry{swdn\_frac\_albsoil}\spxextra{in namelist JULES\_RADIATION}|spxpagem}

\begin{fulllineitems}
\phantomsection\label{\detokenize{namelists/jules_radiation.nml:JULES_RADIATION::swdn_frac_albsoil}}
\pysigstartsignatures
\pysigline{\sphinxcode{\sphinxupquote{JULES\_RADIATION::}}\sphinxbfcode{\sphinxupquote{swdn\_frac\_albsoil}}}
\pysigstopsignatures\begin{quote}\begin{description}
\sphinxlineitem{Type}
\sphinxAtStartPar
real

\sphinxlineitem{Default}
\sphinxAtStartPar
MDI

\end{description}\end{quote}

\sphinxAtStartPar
The fraction of the total downward SW radiation assumed to be in the
NIR part of the spectrum for partitioning the soil albedo.
Used if {\hyperref[\detokenize{namelists/jules_radiation.nml:JULES_RADIATION::l_partition_albsoil}]{\sphinxcrossref{\sphinxcode{\sphinxupquote{l\_partition\_albsoil}}}}} = T.
(Ideally, {\hyperref[\detokenize{namelists/jules_radiation.nml:JULES_RADIATION::wght_alb}]{\sphinxcrossref{\sphinxcode{\sphinxupquote{wght\_alb}}}}} and
{\hyperref[\detokenize{namelists/jules_radiation.nml:JULES_RADIATION::swdn_frac_albsoil}]{\sphinxcrossref{\sphinxcode{\sphinxupquote{swdn\_frac\_albsoil}}}}} should be consistent, with
{\hyperref[\detokenize{namelists/jules_radiation.nml:JULES_RADIATION::swdn_frac_albsoil}]{\sphinxcrossref{\sphinxcode{\sphinxupquote{swdn\_frac\_albsoil}}}}} equal to
\(\sum_{3,4}\)
{\hyperref[\detokenize{namelists/jules_radiation.nml:JULES_RADIATION::wght_alb}]{\sphinxcrossref{\sphinxcode{\sphinxupquote{wght\_alb}}}}} \(/ \sum_1^4\)
{\hyperref[\detokenize{namelists/jules_radiation.nml:JULES_RADIATION::wght_alb}]{\sphinxcrossref{\sphinxcode{\sphinxupquote{wght\_alb}}}}}.
However, {\hyperref[\detokenize{namelists/jules_radiation.nml:JULES_RADIATION::swdn_frac_albsoil}]{\sphinxcrossref{\sphinxcode{\sphinxupquote{swdn\_frac\_albsoil}}}}} is applied
only to bare soil and having a single parameter is more transparent
to the user, while {\hyperref[\detokenize{namelists/jules_radiation.nml:JULES_RADIATION::wght_alb}]{\sphinxcrossref{\sphinxcode{\sphinxupquote{wght\_alb}}}}}
is used only in diagnostics in standalone JULES and may have
historical settings. Hence, the consistency of these two variables
is not enforced.)

\end{fulllineitems}



\sphinxstrong{See also:}
\nopagebreak


\sphinxAtStartPar
References:
\begin{itemize}
\item {} 
\sphinxAtStartPar
Barker, H.W. and Li, Z. (1995), Improved Simulation of Clear\sphinxhyphen{}Sky
Shortwave Radiative Transfer in the CCC\sphinxhyphen{}GCM. J. Climate, 8,
2213\textendash{}2223, \sphinxhref{https://doi.org/10.1175/1520-0442\%281995\%29008\%3C2213\%3AISOCSS\%3E2.0.CO\%3B2}{doi:10.1175/1520\sphinxhyphen{}0442(1995)008\textless{}2213:ISOCSS\textgreater{}2.0.CO;2}

\item {} 
\sphinxAtStartPar
Briegleb, B. and Ramanathan, V. (1982), Spectral and Diurnal
Variations in Clear Sky Planetary Albedo. J. Appl. Meteor., 21,
1160\textendash{}1171, \sphinxhref{https://doi.org/10.1175/1520-0450\%281982\%29021\%3C1160\%3ASADVIC\%3E2.0.CO\%3B2}{doi:10.1175/1520\sphinxhyphen{}0450(1982)021\textless{}1160:SADVIC\textgreater{}2.0.CO;2}

\item {} 
\sphinxAtStartPar
Liu, J. , Zhang, Z. , Inoue, J. and Horton, R. M. (2007),
Evaluation of snow/ice albedo parameterizations and their impacts
on sea ice simulations. Int. J. Climatol., 27:
81\sphinxhyphen{}91. \sphinxhref{https://doi.org/10.1002/joc.1373}{doi:10.1002/joc.1373}

\item {} 
\sphinxAtStartPar
Zhonghai Jin, Yanli Qiao, Yingjian Wang, Yonghua Fang, and
Weining Yi, “A new parameterization of spectral and broadband
ocean surface albedo”, Opt. Express 19, 26429\sphinxhyphen{}26443 (2011),
\sphinxhref{https://doi.org/10.1364/OE.19.026429}{doi:10.1364/OE.19.026429}

\item {} 
\sphinxAtStartPar
B. Hapke, “Bidirectional reflectance spectroscopy: 1. Theory”,
J. Geophys. Res. 86(B4), 3039\sphinxhyphen{}3054 (1981),
\sphinxhref{http://doi.org/10.1029/JB086iB04p03039}{doi:10.1029/JB086iB04p03039}

\item {} 
\sphinxAtStartPar
Manoj M. Joshi and
Robert M. Haberle. Astrobiology. Jan 2012. ahead of print
\sphinxhref{http://doi.org/10.1089/ast.2011.0668}{doi:10.1089/ast.2011.0668}

\item {} 
\sphinxAtStartPar
Martin Turbet, Jérémy Leconte, Franck Selsis, Emeline Bolmont,
François Forget, Ignasi Ribas, Sean N. Raymond and Guillem
Anglada\sphinxhyphen{}Escudé (2016), The habitability of Proxima Centauri
b \sphinxhyphen{} II. Possible climates and observability, A\&A, 596, A112,
\sphinxhref{https://doi.org/10.1051/0004-6361/201629577}{doi:10.1051/0004\sphinxhyphen{}6361/201629577}

\end{itemize}



\sphinxstepscope


\section{\sphinxstyleliteralintitle{\sphinxupquote{jules\_hydrology.nml}}}
\label{\detokenize{namelists/jules_hydrology.nml:jules-hydrology-nml}}\label{\detokenize{namelists/jules_hydrology.nml::doc}}
\sphinxAtStartPar
This file sets the hydrology options. It contains one namelist called {\hyperref[\detokenize{namelists/jules_hydrology.nml:namelist-JULES_HYDROLOGY}]{\sphinxcrossref{\sphinxcode{\sphinxupquote{JULES\_HYDROLOGY}}}}}.


\subsection{\sphinxstyleliteralintitle{\sphinxupquote{JULES\_HYDROLOGY}} namelist members}
\label{\detokenize{namelists/jules_hydrology.nml:namelist-JULES_HYDROLOGY}}\label{\detokenize{namelists/jules_hydrology.nml:jules-hydrology-namelist-members}}\index{JULES\_HYDROLOGY (namelist)@\spxentry{JULES\_HYDROLOGY}\spxextra{namelist}|spxpagem}\index{l\_top (in namelist JULES\_HYDROLOGY)@\spxentry{l\_top}\spxextra{in namelist JULES\_HYDROLOGY}|spxpagem}

\begin{fulllineitems}
\phantomsection\label{\detokenize{namelists/jules_hydrology.nml:JULES_HYDROLOGY::l_top}}
\pysigstartsignatures
\pysigline{\sphinxcode{\sphinxupquote{JULES\_HYDROLOGY::}}\sphinxbfcode{\sphinxupquote{l\_top}}}
\pysigstopsignatures\begin{quote}\begin{description}
\sphinxlineitem{Type}
\sphinxAtStartPar
logical

\sphinxlineitem{Default}
\sphinxAtStartPar
F

\end{description}\end{quote}

\sphinxAtStartPar
Switch for a TOPMODEL\sphinxhyphen{}type model of runoff production.
\begin{description}
\sphinxlineitem{TRUE}
\sphinxAtStartPar
Use a TOPMODEL\sphinxhyphen{}type scheme. This is based on Gedney and Cox (2003); see also Clark and Gedney (2008).

\sphinxlineitem{FALSE}
\sphinxAtStartPar
No TOPMODEL scheme.

\end{description}


\sphinxstrong{See also:}
\nopagebreak


\sphinxAtStartPar
References:
\begin{itemize}
\item {} 
\sphinxAtStartPar
Gedney, N. and P.M.Cox, 2003 , The sensitivity of global climate model simulations to the representation of soil moisture heterogeneity, J. Hydrometeorology, 4, 1265\sphinxhyphen{}1275.

\item {} 
\sphinxAtStartPar
Clark and Gedney, 2008, Representing the effects of subgrid variability of soil moisture on runoff generation in a land surface model, Journal of Geophysical Research \sphinxhyphen{} Atmospheres, 113, D10111, doi:10.1029/2007JD008940.

\end{itemize}



\end{fulllineitems}

\index{l\_pdm (in namelist JULES\_HYDROLOGY)@\spxentry{l\_pdm}\spxextra{in namelist JULES\_HYDROLOGY}|spxpagem}

\begin{fulllineitems}
\phantomsection\label{\detokenize{namelists/jules_hydrology.nml:JULES_HYDROLOGY::l_pdm}}
\pysigstartsignatures
\pysigline{\sphinxcode{\sphinxupquote{JULES\_HYDROLOGY::}}\sphinxbfcode{\sphinxupquote{l\_pdm}}}
\pysigstopsignatures\begin{quote}\begin{description}
\sphinxlineitem{Type}
\sphinxAtStartPar
logical

\sphinxlineitem{Default}
\sphinxAtStartPar
F

\end{description}\end{quote}

\sphinxAtStartPar
Switch for a PDM\sphinxhyphen{}type model of runoff production.

\sphinxAtStartPar
PDM is the Probability Distributed Model (Moore, 1985), implemented in JULES following Clark and Gedney (2008).
\begin{description}
\sphinxlineitem{TRUE}
\sphinxAtStartPar
Use a PDM scheme.

\sphinxlineitem{FALSE}
\sphinxAtStartPar
No PDM scheme.

\end{description}


\sphinxstrong{See also:}
\nopagebreak


\sphinxAtStartPar
References:
\begin{itemize}
\item {} 
\sphinxAtStartPar
Moore, R. J. (1985), The probability\sphinxhyphen{}distributed principle and runoff production at point and basin scales, Hydrol. Sci. J., 30, 273\sphinxhyphen{}297.

\item {} 
\sphinxAtStartPar
Clark and Gedney, 2008, Representing the effects of subgrid variability of soil moisture on runoff generation in a land surface model, Journal of Geophysical Research \sphinxhyphen{} Atmospheres, 113, D10111, doi:10.1029/2007JD008940.

\end{itemize}



\begin{sphinxadmonition}{note}{Note:}
\sphinxAtStartPar
Setting {\hyperref[\detokenize{namelists/jules_hydrology.nml:JULES_HYDROLOGY::l_top}]{\sphinxcrossref{\sphinxcode{\sphinxupquote{l\_top}}}}} = FALSE and {\hyperref[\detokenize{namelists/jules_hydrology.nml:JULES_HYDROLOGY::l_pdm}]{\sphinxcrossref{\sphinxcode{\sphinxupquote{l\_pdm}}}}} = FALSE selects a more basic runoff production scheme. In this scheme, surface runoff comes only from infiltration excess runoff (no saturation excess runoff), and subsurface runoff comes only from free drainage from the deepest soil layer (no lateral flow from mid\sphinxhyphen{}layers), as described in Essery et al. (2001, HCTN 30).
\end{sphinxadmonition}

\end{fulllineitems}

\index{l\_limit\_gsoil (in namelist JULES\_HYDROLOGY)@\spxentry{l\_limit\_gsoil}\spxextra{in namelist JULES\_HYDROLOGY}|spxpagem}

\begin{fulllineitems}
\phantomsection\label{\detokenize{namelists/jules_hydrology.nml:JULES_HYDROLOGY::l_limit_gsoil}}
\pysigstartsignatures
\pysigline{\sphinxcode{\sphinxupquote{JULES\_HYDROLOGY::}}\sphinxbfcode{\sphinxupquote{l\_limit\_gsoil}}}
\pysigstopsignatures\begin{quote}\begin{description}
\sphinxlineitem{Type}
\sphinxAtStartPar
logical

\sphinxlineitem{Default}
\sphinxAtStartPar
F

\end{description}\end{quote}
\begin{description}
\sphinxlineitem{TRUE}
\sphinxAtStartPar
Limit the soil conductance to the value when the top layer soil moisture is at the critical soil moisture. Below this threshold, the soil conductance follows Best et al. (2011) equation 7.

\sphinxlineitem{FALSE}
\sphinxAtStartPar
Allow the soil conductance to increase as the top layer soil moisture goes above the critical soil moisture, as in Best et al. (2011) equation 7.

\end{description}

\end{fulllineitems}


\begin{sphinxadmonition}{note}{Only used if \sphinxstyleliteralintitle{\sphinxupquote{l\_top}} = TRUE}
\index{zw\_max (in namelist JULES\_HYDROLOGY)@\spxentry{zw\_max}\spxextra{in namelist JULES\_HYDROLOGY}|spxpagem}

\begin{fulllineitems}
\phantomsection\label{\detokenize{namelists/jules_hydrology.nml:JULES_HYDROLOGY::zw_max}}
\pysigstartsignatures
\pysigline{\sphinxcode{\sphinxupquote{JULES\_HYDROLOGY::}}\sphinxbfcode{\sphinxupquote{zw\_max}}}
\pysigstopsignatures\begin{quote}\begin{description}
\sphinxlineitem{Type}
\sphinxAtStartPar
real

\sphinxlineitem{Default}
\sphinxAtStartPar
None

\end{description}\end{quote}

\sphinxAtStartPar
The maximum allowed depth to the water table (m).

\sphinxAtStartPar
This is the depth from the soil surface to the bottom of an additional layer that is used to track water tables below the standard soil model (which has layer thicknesses given by {\hyperref[\detokenize{namelists/jules_soil.nml:JULES_SOIL::dzsoil_io}]{\sphinxcrossref{\sphinxcode{\sphinxupquote{dzsoil\_io}}}}}). A value of \textasciitilde{}10m can often be used (though the previous default value was 6m) \sphinxhyphen{} the suitability of any value depends on values of the ancillary variable \sphinxcode{\sphinxupquote{fexp}} (see {\hyperref[\detokenize{namelists/ancillaries.nml:list-of-topmodel-params}]{\sphinxcrossref{\DUrole{std,std-ref}{List of TOPMODEL parameters}}}}) and the sum of the soil layer thicknesses (denoted \sphinxtitleref{sum\_dzsoil} here). The saturated hydraulic conductivity declines exponentially with depth in the additional deep TOPMODEL layer, with decay parameter \sphinxcode{\sphinxupquote{fexp}}, and should be sufficiently small at depth \sphinxcode{\sphinxupquote{zw\_max}} that the flow at this depth can be neglected, that is  \sphinxtitleref{EXP(\sphinxhyphen{}fexp(zw\_max\sphinxhyphen{}sum\_dzsoil))} should be sufficiently small at all locations.  (As a minimum guide, the code tests that the value of this expression is \textless{}= 0.05 and a warning is printed where this condition is not met; users should check model output logs for these messages.)

\end{fulllineitems}

\index{ti\_max (in namelist JULES\_HYDROLOGY)@\spxentry{ti\_max}\spxextra{in namelist JULES\_HYDROLOGY}|spxpagem}

\begin{fulllineitems}
\phantomsection\label{\detokenize{namelists/jules_hydrology.nml:JULES_HYDROLOGY::ti_max}}
\pysigstartsignatures
\pysigline{\sphinxcode{\sphinxupquote{JULES\_HYDROLOGY::}}\sphinxbfcode{\sphinxupquote{ti\_max}}}
\pysigstopsignatures\begin{quote}\begin{description}
\sphinxlineitem{Type}
\sphinxAtStartPar
real

\sphinxlineitem{Default}
\sphinxAtStartPar
None

\end{description}\end{quote}

\sphinxAtStartPar
The maximum possible value of the topographic index. A value of 10.0 can be used.

\end{fulllineitems}

\index{ti\_wetl (in namelist JULES\_HYDROLOGY)@\spxentry{ti\_wetl}\spxextra{in namelist JULES\_HYDROLOGY}|spxpagem}

\begin{fulllineitems}
\phantomsection\label{\detokenize{namelists/jules_hydrology.nml:JULES_HYDROLOGY::ti_wetl}}
\pysigstartsignatures
\pysigline{\sphinxcode{\sphinxupquote{JULES\_HYDROLOGY::}}\sphinxbfcode{\sphinxupquote{ti\_wetl}}}
\pysigstopsignatures\begin{quote}\begin{description}
\sphinxlineitem{Type}
\sphinxAtStartPar
real

\sphinxlineitem{Default}
\sphinxAtStartPar
None

\end{description}\end{quote}

\sphinxAtStartPar
A calibration parameter used in the calculation of the wetland fraction.

\sphinxAtStartPar
It is used to increment the “critical” value of the topographic index that is used to calculate the saturated fraction of the gridbox. It excludes locations with large values of the topographic index from the wetland fraction. A value of 1.5 can be used.

\end{fulllineitems}


\begin{sphinxadmonition}{note}{Note:}
\sphinxAtStartPar
When TOPMODEL is on (i.e. {\hyperref[\detokenize{namelists/jules_hydrology.nml:JULES_HYDROLOGY::l_top}]{\sphinxcrossref{\sphinxcode{\sphinxupquote{l\_top}}}}} = TRUE), JULES follows Gedney \& Cox (2003, J Hydromet, eqn 14) in assuming that wetlands occur where gridcell elevation is low enough (assumed to be where topographic index is large enough) that the water table is above the land surface (topidx \textgreater{} {\hyperref[\detokenize{namelists/jules_hydrology.nml:JULES_HYDROLOGY::ti_wetl}]{\sphinxcrossref{\sphinxcode{\sphinxupquote{ti\_wetl}}}}}) but not above the land surface by enough that streamflow may be assumed to occur (topidx \textless{} {\hyperref[\detokenize{namelists/jules_hydrology.nml:JULES_HYDROLOGY::ti_max}]{\sphinxcrossref{\sphinxcode{\sphinxupquote{ti\_max}}}}}). Both {\hyperref[\detokenize{namelists/jules_hydrology.nml:JULES_HYDROLOGY::ti_wetl}]{\sphinxcrossref{\sphinxcode{\sphinxupquote{ti\_wetl}}}}} and {\hyperref[\detokenize{namelists/jules_hydrology.nml:JULES_HYDROLOGY::ti_max}]{\sphinxcrossref{\sphinxcode{\sphinxupquote{ti\_max}}}}} are levels calibrated from observed wetland fractions. So, if the water table is above the surface then JULES can calculate an areal fraction of total inundation (fsat) and also the areal fraction that is inundated but shallow enough to be stagnant/non\sphinxhyphen{}flowing (fwetl, with fwetl\textless{}=fsat), which is the ‘wetland fraction’.
\end{sphinxadmonition}
\index{nfita (in namelist JULES\_HYDROLOGY)@\spxentry{nfita}\spxextra{in namelist JULES\_HYDROLOGY}|spxpagem}

\begin{fulllineitems}
\phantomsection\label{\detokenize{namelists/jules_hydrology.nml:JULES_HYDROLOGY::nfita}}
\pysigstartsignatures
\pysigline{\sphinxcode{\sphinxupquote{JULES\_HYDROLOGY::}}\sphinxbfcode{\sphinxupquote{nfita}}}
\pysigstopsignatures\begin{quote}\begin{description}
\sphinxlineitem{Type}
\sphinxAtStartPar
integer

\sphinxlineitem{Default}
\sphinxAtStartPar
None

\end{description}\end{quote}

\sphinxAtStartPar
The number of values tried when fitting wetland and saturation fractions to water table depth in the initialisation. A value of 20 can be used.

\sphinxAtStartPar
This controls the range of \sphinxcode{\sphinxupquote{cfit}} values tried in
\sphinxcode{\sphinxupquote{calc\_fit\_fsat.F90}} where \sphinxcode{\sphinxupquote{cfitmax = 0.15 * nfita}}

\end{fulllineitems}

\index{l\_wetland\_unfrozen (in namelist JULES\_HYDROLOGY)@\spxentry{l\_wetland\_unfrozen}\spxextra{in namelist JULES\_HYDROLOGY}|spxpagem}

\begin{fulllineitems}
\phantomsection\label{\detokenize{namelists/jules_hydrology.nml:JULES_HYDROLOGY::l_wetland_unfrozen}}
\pysigstartsignatures
\pysigline{\sphinxcode{\sphinxupquote{JULES\_HYDROLOGY::}}\sphinxbfcode{\sphinxupquote{l\_wetland\_unfrozen}}}
\pysigstopsignatures\begin{quote}\begin{description}
\sphinxlineitem{Type}
\sphinxAtStartPar
logical

\sphinxlineitem{Default}
\sphinxAtStartPar
F

\end{description}\end{quote}

\end{fulllineitems}

\begin{description}
\sphinxlineitem{TRUE}
\sphinxAtStartPar
Treat the calculations of wetland and surface saturation fractions more like those of an unfrozen soil.

\sphinxlineitem{FALSE}
\sphinxAtStartPar
Use standard wetland and surface saturation fraction calculations.

\end{description}
\end{sphinxadmonition}

\begin{sphinxadmonition}{note}{Only used if \sphinxstyleliteralintitle{\sphinxupquote{l\_pdm}} = TRUE}
\index{dz\_pdm (in namelist JULES\_HYDROLOGY)@\spxentry{dz\_pdm}\spxextra{in namelist JULES\_HYDROLOGY}|spxpagem}

\begin{fulllineitems}
\phantomsection\label{\detokenize{namelists/jules_hydrology.nml:JULES_HYDROLOGY::dz_pdm}}
\pysigstartsignatures
\pysigline{\sphinxcode{\sphinxupquote{JULES\_HYDROLOGY::}}\sphinxbfcode{\sphinxupquote{dz\_pdm}}}
\pysigstopsignatures\begin{quote}\begin{description}
\sphinxlineitem{Type}
\sphinxAtStartPar
real

\sphinxlineitem{Default}
\sphinxAtStartPar
None

\end{description}\end{quote}

\sphinxAtStartPar
The depth of soil considered by PDM (m).

\sphinxAtStartPar
A value of \textasciitilde{}1m can be used.

\end{fulllineitems}

\index{b\_pdm (in namelist JULES\_HYDROLOGY)@\spxentry{b\_pdm}\spxextra{in namelist JULES\_HYDROLOGY}|spxpagem}

\begin{fulllineitems}
\phantomsection\label{\detokenize{namelists/jules_hydrology.nml:JULES_HYDROLOGY::b_pdm}}
\pysigstartsignatures
\pysigline{\sphinxcode{\sphinxupquote{JULES\_HYDROLOGY::}}\sphinxbfcode{\sphinxupquote{b\_pdm}}}
\pysigstopsignatures\begin{quote}\begin{description}
\sphinxlineitem{Type}
\sphinxAtStartPar
real

\sphinxlineitem{Default}
\sphinxAtStartPar
None

\end{description}\end{quote}

\sphinxAtStartPar
PDM shape parameter (exponent) of the Pareto distribution controlling spatial variability of storage capacity. A value \textasciitilde{}1 can be used. b=0 implies a constant storage capacity at all points.

\end{fulllineitems}

\index{l\_spdmvar (in namelist JULES\_HYDROLOGY)@\spxentry{l\_spdmvar}\spxextra{in namelist JULES\_HYDROLOGY}|spxpagem}

\begin{fulllineitems}
\phantomsection\label{\detokenize{namelists/jules_hydrology.nml:JULES_HYDROLOGY::l_spdmvar}}
\pysigstartsignatures
\pysigline{\sphinxcode{\sphinxupquote{JULES\_HYDROLOGY::}}\sphinxbfcode{\sphinxupquote{l\_spdmvar}}}
\pysigstopsignatures\begin{quote}\begin{description}
\sphinxlineitem{Type}
\sphinxAtStartPar
logical

\sphinxlineitem{Default}
\sphinxAtStartPar
F

\end{description}\end{quote}
\begin{description}
\sphinxlineitem{TRUE}
\sphinxAtStartPar
Use a linear function of topographic slope to calculate S0/Smax (the minimum soil water storage below which there is no saturation excess runoff from PDM, expressed as a fraction of the maximum storage Smax): S0/Smax=MAX(0.0,1\sphinxhyphen{}(slope/{\hyperref[\detokenize{namelists/jules_hydrology.nml:JULES_HYDROLOGY::slope_pdm_max}]{\sphinxcrossref{\sphinxcode{\sphinxupquote{slope\_pdm\_max}}}}})). The slope is read as an ancillary field (see {\hyperref[\detokenize{namelists/ancillaries.nml:namelist-JULES_PDM}]{\sphinxcrossref{\sphinxcode{\sphinxupquote{JULES\_PDM}}}}}).

\sphinxAtStartPar
This function will result in high S0/Smax values for flatter regions and low values for steeper regions, and has been tested for catchments in Great Britain.

\sphinxlineitem{FALSE}
\sphinxAtStartPar
Use a fixed value for S0/Smax, specified in {\hyperref[\detokenize{namelists/jules_hydrology.nml:JULES_HYDROLOGY::s_pdm}]{\sphinxcrossref{\sphinxcode{\sphinxupquote{s\_pdm}}}}}.

\end{description}

\end{fulllineitems}


\begin{sphinxadmonition}{note}{Only used if \sphinxstyleliteralintitle{\sphinxupquote{l\_spdmvar}} = TRUE}
\index{slope\_pdm\_max (in namelist JULES\_HYDROLOGY)@\spxentry{slope\_pdm\_max}\spxextra{in namelist JULES\_HYDROLOGY}|spxpagem}

\begin{fulllineitems}
\phantomsection\label{\detokenize{namelists/jules_hydrology.nml:JULES_HYDROLOGY::slope_pdm_max}}
\pysigstartsignatures
\pysigline{\sphinxcode{\sphinxupquote{JULES\_HYDROLOGY::}}\sphinxbfcode{\sphinxupquote{slope\_pdm\_max}}}
\pysigstopsignatures\begin{quote}\begin{description}
\sphinxlineitem{Type}
\sphinxAtStartPar
real

\sphinxlineitem{Default}
\sphinxAtStartPar
None

\end{description}\end{quote}

\sphinxAtStartPar
The maximum topographic slope (deg) in the linear function of slope to calculate S0/Smax. Slopes above this value will result in a S0/Smax value of zero.

\sphinxAtStartPar
A value of 6.0 has been tested for slope fields calculated from a high resolution DEM dataset (50m IHDTM for Great Britain).

\sphinxAtStartPar
For slopes calculated from coarser DEM datasets, a lower value might be more appropriate as fine\sphinxhyphen{}resolution features of the terrain are not included.

\end{fulllineitems}

\end{sphinxadmonition}

\begin{sphinxadmonition}{note}{Only used if \sphinxstyleliteralintitle{\sphinxupquote{l\_spdmvar}} = FALSE}
\index{s\_pdm (in namelist JULES\_HYDROLOGY)@\spxentry{s\_pdm}\spxextra{in namelist JULES\_HYDROLOGY}|spxpagem}

\begin{fulllineitems}
\phantomsection\label{\detokenize{namelists/jules_hydrology.nml:JULES_HYDROLOGY::s_pdm}}
\pysigstartsignatures
\pysigline{\sphinxcode{\sphinxupquote{JULES\_HYDROLOGY::}}\sphinxbfcode{\sphinxupquote{s\_pdm}}}
\pysigstopsignatures
\end{fulllineitems}

\begin{quote}\begin{description}
\sphinxlineitem{type}
\sphinxAtStartPar
real

\sphinxlineitem{permitted}
\sphinxAtStartPar
0\sphinxhyphen{}1

\sphinxlineitem{default}
\sphinxAtStartPar
None

\end{description}\end{quote}

\sphinxAtStartPar
Minimum soil water storage below which there is no saturation excess runoff from PDM, expressed as a fraction of the maximum storage Smax)

\sphinxAtStartPar
e.g. A value of 0 indicates that surface saturation can occur for any
value of water storage. A value of 0.5 would indicate that
no surface runoff is produced until the soil is 50\% saturated.
\end{sphinxadmonition}
\end{sphinxadmonition}

\sphinxstepscope


\section{\sphinxstyleliteralintitle{\sphinxupquote{jules\_soil.nml}}}
\label{\detokenize{namelists/jules_soil.nml:jules-soil-nml}}\label{\detokenize{namelists/jules_soil.nml::doc}}
\sphinxAtStartPar
This file sets the soil options and parameters. It contains one namelist called {\hyperref[\detokenize{namelists/jules_soil.nml:namelist-JULES_SOIL}]{\sphinxcrossref{\sphinxcode{\sphinxupquote{JULES\_SOIL}}}}}.


\subsection{\sphinxstyleliteralintitle{\sphinxupquote{JULES\_SOIL}} namelist members}
\label{\detokenize{namelists/jules_soil.nml:namelist-JULES_SOIL}}\label{\detokenize{namelists/jules_soil.nml:jules-soil-namelist-members}}\index{JULES\_SOIL (namelist)@\spxentry{JULES\_SOIL}\spxextra{namelist}|spxpagem}\index{sm\_levels (in namelist JULES\_SOIL)@\spxentry{sm\_levels}\spxextra{in namelist JULES\_SOIL}|spxpagem}

\begin{fulllineitems}
\phantomsection\label{\detokenize{namelists/jules_soil.nml:JULES_SOIL::sm_levels}}
\pysigstartsignatures
\pysigline{\sphinxcode{\sphinxupquote{JULES\_SOIL::}}\sphinxbfcode{\sphinxupquote{sm\_levels}}}
\pysigstopsignatures\begin{quote}\begin{description}
\sphinxlineitem{Type}
\sphinxAtStartPar
integer

\sphinxlineitem{Permitted}
\sphinxAtStartPar
\textgreater{}= 1

\sphinxlineitem{Default}
\sphinxAtStartPar
4

\end{description}\end{quote}

\sphinxAtStartPar
Number of soil layers.

\sphinxAtStartPar
A value of 4 is often used, with soil layer depths that have been tuned using this.

\begin{sphinxadmonition}{warning}{Warning:}
\sphinxAtStartPar
If {\hyperref[\detokenize{namelists/jules_surface_types.nml:JULES_SURFACE_TYPES::ncpft}]{\sphinxcrossref{\sphinxcode{\sphinxupquote{ncpft}}}}} \textgreater{} 0, \sphinxcode{\sphinxupquote{sm\_levels \textgreater{}= 3}} is required.
\end{sphinxadmonition}

\end{fulllineitems}

\index{l\_vg\_soil (in namelist JULES\_SOIL)@\spxentry{l\_vg\_soil}\spxextra{in namelist JULES\_SOIL}|spxpagem}

\begin{fulllineitems}
\phantomsection\label{\detokenize{namelists/jules_soil.nml:JULES_SOIL::l_vg_soil}}
\pysigstartsignatures
\pysigline{\sphinxcode{\sphinxupquote{JULES\_SOIL::}}\sphinxbfcode{\sphinxupquote{l\_vg\_soil}}}
\pysigstopsignatures\begin{quote}\begin{description}
\sphinxlineitem{Type}
\sphinxAtStartPar
logical

\sphinxlineitem{Default}
\sphinxAtStartPar
F

\end{description}\end{quote}

\sphinxAtStartPar
Switch for van Genuchten soil hydraulic model.
\begin{description}
\sphinxlineitem{TRUE}
\sphinxAtStartPar
Use van Genuchten model.

\sphinxlineitem{FALSE}
\sphinxAtStartPar
Use Brooks and Corey model %
\begin{footnote}[1]\sphinxAtStartFootnote
In the JULES2.0 User Manual this was described as the ‘Clapp and Hornberger’ model and the JULES code still refers to ‘Clapp and Hornberger’ rather than ‘Brooks and Corey’. In fact there are important differences between these two hydraulic models (Marthews et al. 2014,GMD). There has been confusion in the literature and in past documentation of MOSES/JULES, but JULES uses the Brooks and Corey model when {\hyperref[\detokenize{namelists/jules_soil.nml:JULES_SOIL::l_vg_soil}]{\sphinxcrossref{\sphinxcode{\sphinxupquote{l\_vg\_soil}}}}} = FALSE .

\sphinxAtStartPar
References:
* Brooks RH \& Corey AT (1964). Hydraulic properties of porous media. Colorado State University Hydrology Papers 3.
* Clapp RB \& Hornberger GM (1978). Empirical Equations for Some Soil Hydraulic Properties. Water Resources Research 14:601\sphinxhyphen{}604.
%
\end{footnote}.

\end{description}


\sphinxstrong{See also:}
\nopagebreak


\sphinxAtStartPar
References:
\begin{itemize}
\item {} 
\sphinxAtStartPar
Brooks, R.H. and A.T. Corey, 1964, Hydraulic properties of porous media. Colorado State University Hydrology Papers 3.

\item {} 
\sphinxAtStartPar
van Genuchten, M.T., 1980, A Closed\sphinxhyphen{}form Equation for Predicting the Hydraulic Conductivity of Unsaturated Soils. Soil Science Society of America Journal, 44:892\sphinxhyphen{}898.

\end{itemize}



\end{fulllineitems}

\index{l\_dpsids\_dsdz (in namelist JULES\_SOIL)@\spxentry{l\_dpsids\_dsdz}\spxextra{in namelist JULES\_SOIL}|spxpagem}

\begin{fulllineitems}
\phantomsection\label{\detokenize{namelists/jules_soil.nml:JULES_SOIL::l_dpsids_dsdz}}
\pysigstartsignatures
\pysigline{\sphinxcode{\sphinxupquote{JULES\_SOIL::}}\sphinxbfcode{\sphinxupquote{l\_dpsids\_dsdz}}}
\pysigstopsignatures\begin{quote}\begin{description}
\sphinxlineitem{Type}
\sphinxAtStartPar
logical

\sphinxlineitem{Default}
\sphinxAtStartPar
F

\end{description}\end{quote}

\sphinxAtStartPar
Switch to calculate vertical gradient of soil suction with the assumption of linearity only for fractional saturation (consistent with the calculation of hydraulic conductivity).

\end{fulllineitems}

\index{l\_soil\_sat\_down (in namelist JULES\_SOIL)@\spxentry{l\_soil\_sat\_down}\spxextra{in namelist JULES\_SOIL}|spxpagem}

\begin{fulllineitems}
\phantomsection\label{\detokenize{namelists/jules_soil.nml:JULES_SOIL::l_soil_sat_down}}
\pysigstartsignatures
\pysigline{\sphinxcode{\sphinxupquote{JULES\_SOIL::}}\sphinxbfcode{\sphinxupquote{l\_soil\_sat\_down}}}
\pysigstopsignatures\begin{quote}\begin{description}
\sphinxlineitem{Type}
\sphinxAtStartPar
logical

\sphinxlineitem{Default}
\sphinxAtStartPar
F

\end{description}\end{quote}

\sphinxAtStartPar
Switch for dealing with supersaturated soil layers. If a soil layer becomes supersaturated, the water in excess of saturation will be put into the layer below or above according to this switch.
\begin{description}
\sphinxlineitem{TRUE (Down)}
\sphinxAtStartPar
Any excess is put into the layer below. Any excess from the bottom
layer becomes subsurface runoff.

\sphinxlineitem{FALSE (Up)}
\sphinxAtStartPar
Any excess is put into the layer above. Any excess from the top layer becomes surface runoff. This option was used in JULES2.0.

\end{description}

\end{fulllineitems}

\index{l\_holdwater (in namelist JULES\_SOIL)@\spxentry{l\_holdwater}\spxextra{in namelist JULES\_SOIL}|spxpagem}

\begin{fulllineitems}
\phantomsection\label{\detokenize{namelists/jules_soil.nml:JULES_SOIL::l_holdwater}}
\pysigstartsignatures
\pysigline{\sphinxcode{\sphinxupquote{JULES\_SOIL::}}\sphinxbfcode{\sphinxupquote{l\_holdwater}}}
\pysigstopsignatures\begin{quote}\begin{description}
\sphinxlineitem{Type}
\sphinxAtStartPar
logical

\sphinxlineitem{Default}
\sphinxAtStartPar
F

\end{description}\end{quote}

\sphinxAtStartPar
This switch fixes a problem in soil hydrology, whereby if a layer goes supersaturated during the implicit calulation, the excess water is pushed out of the soil column (\sphinxcode{\sphinxupquote{l\_holdwater = FALSE}}) instead of into an adjacent layer (\sphinxcode{\sphinxupquote{l\_holdwater = TRUE}}).
\begin{description}
\sphinxlineitem{TRUE}
\sphinxAtStartPar
Supersaturated soil moisture from implicit calculation goes into an adjacent layer
(above or below depending on {\hyperref[\detokenize{namelists/jules_soil.nml:JULES_SOIL::l_soil_sat_down}]{\sphinxcrossref{\sphinxcode{\sphinxupquote{l\_soil\_sat\_down}}}}}). This option was added in JULES 5.1.

\sphinxlineitem{FALSE}
\sphinxAtStartPar
Supersaturated soil moisture from implicit calculation goes out of the base of the soil column.

\end{description}

\end{fulllineitems}

\index{soilhc\_method (in namelist JULES\_SOIL)@\spxentry{soilhc\_method}\spxextra{in namelist JULES\_SOIL}|spxpagem}

\begin{fulllineitems}
\phantomsection\label{\detokenize{namelists/jules_soil.nml:JULES_SOIL::soilhc_method}}
\pysigstartsignatures
\pysigline{\sphinxcode{\sphinxupquote{JULES\_SOIL::}}\sphinxbfcode{\sphinxupquote{soilhc\_method}}}
\pysigstopsignatures\begin{quote}\begin{description}
\sphinxlineitem{Type}
\sphinxAtStartPar
integer

\sphinxlineitem{Permitted}
\sphinxAtStartPar
1, 2 or 3

\sphinxlineitem{Default}
\sphinxAtStartPar
1

\end{description}\end{quote}

\sphinxAtStartPar
Switch for soil thermal conductivity model.
\begin{enumerate}
\sphinxsetlistlabels{\arabic}{enumi}{enumii}{}{.}%
\item {} 
\begin{DUlineblock}{0em}
\item[] Use approach of Cox et al (1999), as in JULES2.0.
\item[] This is likely to predict values of soil thermal conductivity that are too low (Dharssi et al, 2009).
\end{DUlineblock}

\item {} 
\begin{DUlineblock}{0em}
\item[] Use approach of Dharssi et al (2009), which was adapted from Johansen (1975) and described by Peters\sphinxhyphen{}Lidard et al. (1998).
\item[] This is not recommended for organic soils.
\end{DUlineblock}

\item {} 
\begin{DUlineblock}{0em}
\item[] Use approach of Chadburn et al (2015).
\item[] This is recommended when using organic soils, which can have a much lower saturated thermal conductivity than mineral soils.
\end{DUlineblock}

\end{enumerate}


\sphinxstrong{See also:}
\nopagebreak


\sphinxAtStartPar
References:
\begin{itemize}
\item {} 
\sphinxAtStartPar
Chadburn et al (2015). An improved representation of physical permafrost dynamics in a global land\sphinxhyphen{}surface scheme. Geoscientific Model Development

\item {} 
\sphinxAtStartPar
Dharssi et al (2009). New soil physical properties implemented in the Unified Model at PS18. Met Office Technical note 528

\item {} 
\sphinxAtStartPar
Johansen (1975). Thermal conductivity of soils. PhD thesis. University of Trondheim, Norway

\item {} 
\sphinxAtStartPar
Peters\sphinxhyphen{}Lidard et al (1998). The effect of soil thermal conductivity parameterisation on surface energy fluxes and temperatures. J. Atmos. Sci. 55:1209\sphinxhyphen{}1224

\end{itemize}



\end{fulllineitems}

\index{l\_bedrock (in namelist JULES\_SOIL)@\spxentry{l\_bedrock}\spxextra{in namelist JULES\_SOIL}|spxpagem}

\begin{fulllineitems}
\phantomsection\label{\detokenize{namelists/jules_soil.nml:JULES_SOIL::l_bedrock}}
\pysigstartsignatures
\pysigline{\sphinxcode{\sphinxupquote{JULES\_SOIL::}}\sphinxbfcode{\sphinxupquote{l\_bedrock}}}
\pysigstopsignatures\begin{quote}\begin{description}
\sphinxlineitem{Type}
\sphinxAtStartPar
logical

\sphinxlineitem{Default}
\sphinxAtStartPar
F

\end{description}\end{quote}

\sphinxAtStartPar
Switch for using a thermal bedrock column beneath the soil column. The bedrock has no hydrological processes \sphinxhyphen{} diffusion of heat is the only process represented.

\sphinxAtStartPar
Properties of the bedrock can be set using {\hyperref[\detokenize{namelists/jules_soil.nml:JULES_SOIL::ns_deep}]{\sphinxcrossref{\sphinxcode{\sphinxupquote{ns\_deep}}}}}, {\hyperref[\detokenize{namelists/jules_soil.nml:JULES_SOIL::hcapdeep}]{\sphinxcrossref{\sphinxcode{\sphinxupquote{hcapdeep}}}}}, {\hyperref[\detokenize{namelists/jules_soil.nml:JULES_SOIL::hcondeep}]{\sphinxcrossref{\sphinxcode{\sphinxupquote{hcondeep}}}}} and {\hyperref[\detokenize{namelists/jules_soil.nml:JULES_SOIL::dzdeep}]{\sphinxcrossref{\sphinxcode{\sphinxupquote{dzdeep}}}}}.
\begin{description}
\sphinxlineitem{TRUE}
\sphinxAtStartPar
An additional bedrock column is used below the soil column.

\sphinxlineitem{FALSE}
\sphinxAtStartPar
No effect.

\end{description}


\sphinxstrong{See also:}
\nopagebreak


\sphinxAtStartPar
For full details see Chadburn et al. (2015)



\end{fulllineitems}


\begin{sphinxadmonition}{note}{Bedrock parameters (only used if \sphinxstyleliteralintitle{\sphinxupquote{l\_bedrock}} = TRUE)}
\index{ns\_deep (in namelist JULES\_SOIL)@\spxentry{ns\_deep}\spxextra{in namelist JULES\_SOIL}|spxpagem}

\begin{fulllineitems}
\phantomsection\label{\detokenize{namelists/jules_soil.nml:JULES_SOIL::ns_deep}}
\pysigstartsignatures
\pysigline{\sphinxcode{\sphinxupquote{JULES\_SOIL::}}\sphinxbfcode{\sphinxupquote{ns\_deep}}}
\pysigstopsignatures\begin{quote}\begin{description}
\sphinxlineitem{Type}
\sphinxAtStartPar
integer

\sphinxlineitem{Permitted}
\sphinxAtStartPar
\textgreater{}= 1

\sphinxlineitem{Default}
\sphinxAtStartPar
100

\end{description}\end{quote}

\sphinxAtStartPar
The number of levels in the thermal\sphinxhyphen{}only bedrock.

\end{fulllineitems}

\index{hcapdeep (in namelist JULES\_SOIL)@\spxentry{hcapdeep}\spxextra{in namelist JULES\_SOIL}|spxpagem}

\begin{fulllineitems}
\phantomsection\label{\detokenize{namelists/jules_soil.nml:JULES_SOIL::hcapdeep}}
\pysigstartsignatures
\pysigline{\sphinxcode{\sphinxupquote{JULES\_SOIL::}}\sphinxbfcode{\sphinxupquote{hcapdeep}}}
\pysigstopsignatures\begin{quote}\begin{description}
\sphinxlineitem{Type}
\sphinxAtStartPar
real

\sphinxlineitem{Default}
\sphinxAtStartPar
2100000.0

\end{description}\end{quote}

\sphinxAtStartPar
The heat capacity of the bedrock (J K$^{\text{\sphinxhyphen{}1}}$ m$^{\text{\sphinxhyphen{}3}}$ ).

\end{fulllineitems}

\index{hcondeep (in namelist JULES\_SOIL)@\spxentry{hcondeep}\spxextra{in namelist JULES\_SOIL}|spxpagem}

\begin{fulllineitems}
\phantomsection\label{\detokenize{namelists/jules_soil.nml:JULES_SOIL::hcondeep}}
\pysigstartsignatures
\pysigline{\sphinxcode{\sphinxupquote{JULES\_SOIL::}}\sphinxbfcode{\sphinxupquote{hcondeep}}}
\pysigstopsignatures\begin{quote}\begin{description}
\sphinxlineitem{Type}
\sphinxAtStartPar
real

\sphinxlineitem{Default}
\sphinxAtStartPar
8.6

\end{description}\end{quote}

\sphinxAtStartPar
The heat conductivity of the bedrock (W m$^{\text{\sphinxhyphen{}2}}$ K$^{\text{\sphinxhyphen{}1}}$ ).

\end{fulllineitems}

\index{dzdeep (in namelist JULES\_SOIL)@\spxentry{dzdeep}\spxextra{in namelist JULES\_SOIL}|spxpagem}

\begin{fulllineitems}
\phantomsection\label{\detokenize{namelists/jules_soil.nml:JULES_SOIL::dzdeep}}
\pysigstartsignatures
\pysigline{\sphinxcode{\sphinxupquote{JULES\_SOIL::}}\sphinxbfcode{\sphinxupquote{dzdeep}}}
\pysigstopsignatures\begin{quote}\begin{description}
\sphinxlineitem{Type}
\sphinxAtStartPar
real

\sphinxlineitem{Default}
\sphinxAtStartPar
0.5

\end{description}\end{quote}

\sphinxAtStartPar
The thickness of the bedrock layers (m).

\end{fulllineitems}

\end{sphinxadmonition}
\index{cs\_min (in namelist JULES\_SOIL)@\spxentry{cs\_min}\spxextra{in namelist JULES\_SOIL}|spxpagem}

\begin{fulllineitems}
\phantomsection\label{\detokenize{namelists/jules_soil.nml:JULES_SOIL::cs_min}}
\pysigstartsignatures
\pysigline{\sphinxcode{\sphinxupquote{JULES\_SOIL::}}\sphinxbfcode{\sphinxupquote{cs\_min}}}
\pysigstopsignatures\begin{quote}\begin{description}
\sphinxlineitem{Type}
\sphinxAtStartPar
real

\sphinxlineitem{Default}
\sphinxAtStartPar
1.0e\sphinxhyphen{}6

\end{description}\end{quote}

\sphinxAtStartPar
Minimum allowed soil carbon (kg m$^{\text{\sphinxhyphen{}2}}$).

\end{fulllineitems}

\index{zsmc (in namelist JULES\_SOIL)@\spxentry{zsmc}\spxextra{in namelist JULES\_SOIL}|spxpagem}

\begin{fulllineitems}
\phantomsection\label{\detokenize{namelists/jules_soil.nml:JULES_SOIL::zsmc}}
\pysigstartsignatures
\pysigline{\sphinxcode{\sphinxupquote{JULES\_SOIL::}}\sphinxbfcode{\sphinxupquote{zsmc}}}
\pysigstopsignatures\begin{quote}\begin{description}
\sphinxlineitem{Type}
\sphinxAtStartPar
real

\sphinxlineitem{Permitted}
\sphinxAtStartPar
\textgreater{} 0

\sphinxlineitem{Default}
\sphinxAtStartPar
1.0

\end{description}\end{quote}

\sphinxAtStartPar
If a depth\sphinxhyphen{}averaged soil moisture diagnostic is requested, the average is calculated from the surface to this depth (m).

\end{fulllineitems}

\index{zst (in namelist JULES\_SOIL)@\spxentry{zst}\spxextra{in namelist JULES\_SOIL}|spxpagem}

\begin{fulllineitems}
\phantomsection\label{\detokenize{namelists/jules_soil.nml:JULES_SOIL::zst}}
\pysigstartsignatures
\pysigline{\sphinxcode{\sphinxupquote{JULES\_SOIL::}}\sphinxbfcode{\sphinxupquote{zst}}}
\pysigstopsignatures\begin{quote}\begin{description}
\sphinxlineitem{Type}
\sphinxAtStartPar
real

\sphinxlineitem{Permitted}
\sphinxAtStartPar
\textgreater{} 0

\sphinxlineitem{Default}
\sphinxAtStartPar
1.0

\end{description}\end{quote}

\sphinxAtStartPar
The depth (0.0\sphinxhyphen{}\textgreater{}zst) to which the soil temperature is averaged for use in the calculation of wetland methane emissions (m).

\end{fulllineitems}

\index{confrac (in namelist JULES\_SOIL)@\spxentry{confrac}\spxextra{in namelist JULES\_SOIL}|spxpagem}

\begin{fulllineitems}
\phantomsection\label{\detokenize{namelists/jules_soil.nml:JULES_SOIL::confrac}}
\pysigstartsignatures
\pysigline{\sphinxcode{\sphinxupquote{JULES\_SOIL::}}\sphinxbfcode{\sphinxupquote{confrac}}}
\pysigstopsignatures\begin{quote}\begin{description}
\sphinxlineitem{Type}
\sphinxAtStartPar
real

\sphinxlineitem{Permitted}
\sphinxAtStartPar
0 \textless{}= confrac \textless{}= 1

\sphinxlineitem{Default}
\sphinxAtStartPar
0.3

\end{description}\end{quote}

\sphinxAtStartPar
The fraction of the gridbox assumed to be covered by convective precipitation.

\end{fulllineitems}

\index{dzsoil\_io (in namelist JULES\_SOIL)@\spxentry{dzsoil\_io}\spxextra{in namelist JULES\_SOIL}|spxpagem}

\begin{fulllineitems}
\phantomsection\label{\detokenize{namelists/jules_soil.nml:JULES_SOIL::dzsoil_io}}
\pysigstartsignatures
\pysigline{\sphinxcode{\sphinxupquote{JULES\_SOIL::}}\sphinxbfcode{\sphinxupquote{dzsoil\_io}}}
\pysigstopsignatures\begin{quote}\begin{description}
\sphinxlineitem{Type}
\sphinxAtStartPar
real(sm\_levels)

\sphinxlineitem{Default}
\sphinxAtStartPar
None

\end{description}\end{quote}

\sphinxAtStartPar
The soil layer depths (m), starting with the uppermost layer.

\sphinxAtStartPar
Note that the soil layer depths (and hence the total soil depth) are constant across the domain.

\sphinxAtStartPar
It is recommended that JULES uses layer depths of 0.1, 0.25, 0.65 and 2.0m, giving a total depth of 3.0m, unless there is good reason not to.

\end{fulllineitems}

\index{dzsoil\_elev (in namelist JULES\_SOIL)@\spxentry{dzsoil\_elev}\spxextra{in namelist JULES\_SOIL}|spxpagem}

\begin{fulllineitems}
\phantomsection\label{\detokenize{namelists/jules_soil.nml:JULES_SOIL::dzsoil_elev}}
\pysigstartsignatures
\pysigline{\sphinxcode{\sphinxupquote{JULES\_SOIL::}}\sphinxbfcode{\sphinxupquote{dzsoil\_elev}}}
\pysigstopsignatures\begin{quote}\begin{description}
\sphinxlineitem{Type}
\sphinxAtStartPar
real

\sphinxlineitem{Default}
\sphinxAtStartPar
None

\end{description}\end{quote}

\sphinxAtStartPar
Depth of the tiled solid\sphinxhyphen{}ice bedrock\sphinxhyphen{}type layer used underneath individual ice tiles if {\hyperref[\detokenize{namelists/jules_surface.nml:JULES_SURFACE::l_elev_land_ice}]{\sphinxcrossref{\sphinxcode{\sphinxupquote{l\_elev\_land\_ice}}}}} is TRUE.
Effectively this sets the amount of thermal buffering each surface tile has to heat fluxes penetrating through the snowpack.

\end{fulllineitems}

\index{l\_tile\_soil (in namelist JULES\_SOIL)@\spxentry{l\_tile\_soil}\spxextra{in namelist JULES\_SOIL}|spxpagem}

\begin{fulllineitems}
\phantomsection\label{\detokenize{namelists/jules_soil.nml:JULES_SOIL::l_tile_soil}}
\pysigstartsignatures
\pysigline{\sphinxcode{\sphinxupquote{JULES\_SOIL::}}\sphinxbfcode{\sphinxupquote{l\_tile\_soil}}}
\pysigstopsignatures\begin{quote}\begin{description}
\sphinxlineitem{Type}
\sphinxAtStartPar
logical

\sphinxlineitem{Default}
\sphinxAtStartPar
False

\end{description}\end{quote}

\sphinxAtStartPar
Switch to set the number of soil tiles to equal the number of surface tiles. Each soil tile has independent properties.

\sphinxAtStartPar
See also {\hyperref[\detokenize{namelists/jules_soil.nml:JULES_SOIL::l_broadcast_ancils}]{\sphinxcrossref{\sphinxcode{\sphinxupquote{l\_broadcast\_ancils}}}}} and {\hyperref[\detokenize{namelists/initial_conditions.nml:JULES_INITIAL::l_broadcast_soilt}]{\sphinxcrossref{\sphinxcode{\sphinxupquote{l\_broadcast\_soilt}}}}}.

\begin{sphinxadmonition}{note}{Note:}
\sphinxAtStartPar
Setting {\hyperref[\detokenize{namelists/jules_soil.nml:JULES_SOIL::l_tile_soil}]{\sphinxcrossref{\sphinxcode{\sphinxupquote{l\_tile\_soil}}}}} = TRUE means a separate soil tile exists for each surface tile (rather than all surface tiles using the same, single soil tile). This also alters the names of many of the soil prognostic and ancillary variables that are used (see elsewhere), with the suffix “\_soilt’’ being added to indicate the presence of soil tiling. The switches {\hyperref[\detokenize{namelists/jules_soil.nml:JULES_SOIL::l_broadcast_ancils}]{\sphinxcrossref{\sphinxcode{\sphinxupquote{l\_broadcast\_ancils}}}}} and {\hyperref[\detokenize{namelists/initial_conditions.nml:JULES_INITIAL::l_broadcast_soilt}]{\sphinxcrossref{\sphinxcode{\sphinxupquote{l\_broadcast\_soilt}}}}} allow soil tiling to be used with input files that do not contain soil tile information. Setting {\hyperref[\detokenize{namelists/jules_soil.nml:JULES_SOIL::l_broadcast_ancils}]{\sphinxcrossref{\sphinxcode{\sphinxupquote{l\_broadcast\_ancils}}}}} = TRUE means that a soil ancillary file that does not contain soil tiles can be used in a tiled run. Setting {\hyperref[\detokenize{namelists/initial_conditions.nml:JULES_INITIAL::l_broadcast_soilt}]{\sphinxcrossref{\sphinxcode{\sphinxupquote{l\_broadcast\_soilt}}}}} = TRUE means an initital state file that does not contain soil tiles can be used to initialise a run with soil tiles.
\end{sphinxadmonition}

\end{fulllineitems}

\index{l\_broadcast\_ancils (in namelist JULES\_SOIL)@\spxentry{l\_broadcast\_ancils}\spxextra{in namelist JULES\_SOIL}|spxpagem}

\begin{fulllineitems}
\phantomsection\label{\detokenize{namelists/jules_soil.nml:JULES_SOIL::l_broadcast_ancils}}
\pysigstartsignatures
\pysigline{\sphinxcode{\sphinxupquote{JULES\_SOIL::}}\sphinxbfcode{\sphinxupquote{l\_broadcast\_ancils}}}
\pysigstopsignatures\begin{quote}\begin{description}
\sphinxlineitem{Type}
\sphinxAtStartPar
logical

\sphinxlineitem{Default}
\sphinxAtStartPar
False

\end{description}\end{quote}

\sphinxAtStartPar
Switch to allow non\sphinxhyphen{}soil tiled ancillary files to be broadcast to all soil tiles. Only active when {\hyperref[\detokenize{namelists/jules_soil.nml:JULES_SOIL::l_tile_soil}]{\sphinxcrossref{\sphinxcode{\sphinxupquote{l\_tile\_soil}}}}} is True. When reading ancillaries from the dump file, use {\hyperref[\detokenize{namelists/initial_conditions.nml:JULES_INITIAL::l_broadcast_soilt}]{\sphinxcrossref{\sphinxcode{\sphinxupquote{l\_broadcast\_soilt}}}}} instead.

\end{fulllineitems}


\sphinxstepscope


\section{\sphinxstyleliteralintitle{\sphinxupquote{jules\_vegetation.nml}}}
\label{\detokenize{namelists/jules_vegetation.nml:jules-vegetation-nml}}\label{\detokenize{namelists/jules_vegetation.nml::doc}}
\sphinxAtStartPar
This file sets the vegetation options. It contains one namelist called {\hyperref[\detokenize{namelists/jules_vegetation.nml:namelist-JULES_VEGETATION}]{\sphinxcrossref{\sphinxcode{\sphinxupquote{JULES\_VEGETATION}}}}}.


\subsection{\sphinxstyleliteralintitle{\sphinxupquote{JULES\_VEGETATION}} namelist members}
\label{\detokenize{namelists/jules_vegetation.nml:namelist-JULES_VEGETATION}}\label{\detokenize{namelists/jules_vegetation.nml:jules-vegetation-namelist-members}}\index{JULES\_VEGETATION (namelist)@\spxentry{JULES\_VEGETATION}\spxextra{namelist}|spxpagem}\index{l\_trait\_phys (in namelist JULES\_VEGETATION)@\spxentry{l\_trait\_phys}\spxextra{in namelist JULES\_VEGETATION}|spxpagem}

\begin{fulllineitems}
\phantomsection\label{\detokenize{namelists/jules_vegetation.nml:JULES_VEGETATION::l_trait_phys}}
\pysigstartsignatures
\pysigline{\sphinxcode{\sphinxupquote{JULES\_VEGETATION::}}\sphinxbfcode{\sphinxupquote{l\_trait\_phys}}}
\pysigstopsignatures\begin{quote}\begin{description}
\sphinxlineitem{Type}
\sphinxAtStartPar
logical

\sphinxlineitem{Default}
\sphinxAtStartPar
F

\end{description}\end{quote}

\sphinxAtStartPar
Switch for using trait\sphinxhyphen{}based physiology.
\begin{description}
\sphinxlineitem{TRUE}
\sphinxAtStartPar
Vcmax is calculated based on observed leaf traits. Leaf
nitrogen (nmass: kgN kgLeaf$^{\text{\sphinxhyphen{}1}}$) and leaf mass (LMA:
kgLeaf m$^{\text{\sphinxhyphen{}2}}$) can be based on observations from the TRY
database. Vcmax (umol CO$_{\text{2}}$ m$^{\text{\sphinxhyphen{}2}}$ s$^{\text{\sphinxhyphen{}1}}$) is based
on linear regressions as in {\hyperref[\detokenize{namelists/jules_vegetation.nml:references-vegetation}]{\sphinxcrossref{\DUrole{std,std-ref}{Kattge et al. 2009}}}}. Two additional
parameters are needed: vint and vsl \sphinxhyphen{} the intercept and slope,
respectively, that relate the leaf nitrogen to vcmax. Sigl is
replaced with LMA (sigl=LMA*Cmass, where Cmass is the
kgC kgLeaf$^{\text{\sphinxhyphen{}1}}$ and is 0.4).

\sphinxlineitem{FALSE}
\sphinxAtStartPar
Vcmax is calculated based on parameters nl0 (kgN kgC$^{\text{\sphinxhyphen{}1}}$) and neff.

\end{description}

\end{fulllineitems}

\index{l\_phenol (in namelist JULES\_VEGETATION)@\spxentry{l\_phenol}\spxextra{in namelist JULES\_VEGETATION}|spxpagem}

\begin{fulllineitems}
\phantomsection\label{\detokenize{namelists/jules_vegetation.nml:JULES_VEGETATION::l_phenol}}
\pysigstartsignatures
\pysigline{\sphinxcode{\sphinxupquote{JULES\_VEGETATION::}}\sphinxbfcode{\sphinxupquote{l\_phenol}}}
\pysigstopsignatures\begin{quote}\begin{description}
\sphinxlineitem{Type}
\sphinxAtStartPar
logical

\sphinxlineitem{Default}
\sphinxAtStartPar
F

\end{description}\end{quote}

\sphinxAtStartPar
Switch for vegetation phenology model.
\begin{description}
\sphinxlineitem{TRUE}
\sphinxAtStartPar
Use phenology model.

\sphinxlineitem{FALSE}
\sphinxAtStartPar
Do not use phenology model.

\end{description}

\end{fulllineitems}

\index{l\_triffid (in namelist JULES\_VEGETATION)@\spxentry{l\_triffid}\spxextra{in namelist JULES\_VEGETATION}|spxpagem}

\begin{fulllineitems}
\phantomsection\label{\detokenize{namelists/jules_vegetation.nml:JULES_VEGETATION::l_triffid}}
\pysigstartsignatures
\pysigline{\sphinxcode{\sphinxupquote{JULES\_VEGETATION::}}\sphinxbfcode{\sphinxupquote{l\_triffid}}}
\pysigstopsignatures\begin{quote}\begin{description}
\sphinxlineitem{Type}
\sphinxAtStartPar
logical

\sphinxlineitem{Default}
\sphinxAtStartPar
F

\end{description}\end{quote}

\sphinxAtStartPar
Switch for dynamic vegetation model (TRIFFID) except for competition.
\begin{description}
\sphinxlineitem{TRUE}
\sphinxAtStartPar
Use TRIFFID. In this case soil carbon is modelled using four pools
(biomass, humus, decomposable plant material, resistant plant material).

\sphinxlineitem{FALSE}
\sphinxAtStartPar
Do not use TRIFFID. A single soil carbon pool is used.

\end{description}

\end{fulllineitems}

\index{l\_veg\_compete (in namelist JULES\_VEGETATION)@\spxentry{l\_veg\_compete}\spxextra{in namelist JULES\_VEGETATION}|spxpagem}

\begin{fulllineitems}
\phantomsection\label{\detokenize{namelists/jules_vegetation.nml:JULES_VEGETATION::l_veg_compete}}
\pysigstartsignatures
\pysigline{\sphinxcode{\sphinxupquote{JULES\_VEGETATION::}}\sphinxbfcode{\sphinxupquote{l\_veg\_compete}}}
\pysigstopsignatures\begin{quote}\begin{description}
\sphinxlineitem{Type}
\sphinxAtStartPar
logical

\sphinxlineitem{Default}
\sphinxAtStartPar
T

\end{description}\end{quote}

\sphinxAtStartPar
Switch for competing vegetation.

\sphinxAtStartPar
Only used if {\hyperref[\detokenize{namelists/jules_vegetation.nml:JULES_VEGETATION::l_triffid}]{\sphinxcrossref{\sphinxcode{\sphinxupquote{l\_triffid}}}}} = TRUE.
\begin{description}
\sphinxlineitem{TRUE}
\sphinxAtStartPar
TRIFFID will let the different PFTs compete against each other and modify the vegetation fractions.

\sphinxlineitem{FALSE}
\sphinxAtStartPar
Vegetation fractions do not change.

\end{description}

\end{fulllineitems}

\index{l\_ht\_compete (in namelist JULES\_VEGETATION)@\spxentry{l\_ht\_compete}\spxextra{in namelist JULES\_VEGETATION}|spxpagem}

\begin{fulllineitems}
\phantomsection\label{\detokenize{namelists/jules_vegetation.nml:JULES_VEGETATION::l_ht_compete}}
\pysigstartsignatures
\pysigline{\sphinxcode{\sphinxupquote{JULES\_VEGETATION::}}\sphinxbfcode{\sphinxupquote{l\_ht\_compete}}}
\pysigstopsignatures\begin{quote}\begin{description}
\sphinxlineitem{Type}
\sphinxAtStartPar
logical

\sphinxlineitem{Default}
\sphinxAtStartPar
F

\end{description}\end{quote}

\sphinxAtStartPar
Only used if {\hyperref[\detokenize{namelists/jules_vegetation.nml:JULES_VEGETATION::l_triffid}]{\sphinxcrossref{\sphinxcode{\sphinxupquote{l\_triffid}}}}} = TRUE.
\begin{description}
\sphinxlineitem{TRUE}
\sphinxAtStartPar
Use height\sphinxhyphen{}based vegetation competition (recommended).

\sphinxAtStartPar
This allows for a generic number of PFTs. When
{\hyperref[\detokenize{namelists/jules_vegetation.nml:JULES_VEGETATION::l_trif_eq}]{\sphinxcrossref{\sphinxcode{\sphinxupquote{l\_trif\_eq}}}}} = TRUE, this is implemented by
\sphinxcode{\sphinxupquote{lotka\_eq\_jls.F90}}. When {\hyperref[\detokenize{namelists/jules_vegetation.nml:JULES_VEGETATION::l_trif_eq}]{\sphinxcrossref{\sphinxcode{\sphinxupquote{l\_trif\_eq}}}}} = FALSE, it is
implemented in \sphinxcode{\sphinxupquote{lotka\_noeq\_jls.F90}} when
{\hyperref[\detokenize{namelists/jules_vegetation.nml:JULES_VEGETATION::l_trif_crop}]{\sphinxcrossref{\sphinxcode{\sphinxupquote{l\_trif\_crop}}}}} = FALSE and in
\sphinxcode{\sphinxupquote{lotka\_noeq\_subset\_jls.F90}} when {\hyperref[\detokenize{namelists/jules_vegetation.nml:JULES_VEGETATION::l_trif_crop}]{\sphinxcrossref{\sphinxcode{\sphinxupquote{l\_trif\_crop}}}}} =
TRUE.

\sphinxlineitem{FALSE}
\sphinxAtStartPar
Use the vegetation competition described in {\hyperref[\detokenize{namelists/jules_vegetation.nml:references-vegetation}]{\sphinxcrossref{\DUrole{std,std-ref}{HCTN24}}}}.

\sphinxAtStartPar
This is hard\sphinxhyphen{}wired for 5 PFTs (BT, NT, C3, C4, SH, in that
order) with co\sphinxhyphen{}competition for grasses and trees in
\sphinxcode{\sphinxupquote{lokta\_jls.F90}}.

\end{description}

\end{fulllineitems}

\index{l\_nitrogen (in namelist JULES\_VEGETATION)@\spxentry{l\_nitrogen}\spxextra{in namelist JULES\_VEGETATION}|spxpagem}

\begin{fulllineitems}
\phantomsection\label{\detokenize{namelists/jules_vegetation.nml:JULES_VEGETATION::l_nitrogen}}
\pysigstartsignatures
\pysigline{\sphinxcode{\sphinxupquote{JULES\_VEGETATION::}}\sphinxbfcode{\sphinxupquote{l\_nitrogen}}}
\pysigstopsignatures\begin{quote}\begin{description}
\sphinxlineitem{Type}
\sphinxAtStartPar
logical

\sphinxlineitem{Default}
\sphinxAtStartPar
F

\end{description}\end{quote}

\sphinxAtStartPar
Only used if {\hyperref[\detokenize{namelists/jules_vegetation.nml:JULES_VEGETATION::l_triffid}]{\sphinxcrossref{\sphinxcode{\sphinxupquote{l\_triffid}}}}} = TRUE.
\begin{description}
\sphinxlineitem{TRUE}
\sphinxAtStartPar
Enable Nitrogen limitation of carbon uptake. A nitrogen
deposition field should be provided otherwise no N deposition is
assumed.

\sphinxlineitem{FALSE}
\sphinxAtStartPar
No Nitrogen limitation. Nitrogen fluxes are calculated as diagnostics only.

\end{description}

\end{fulllineitems}

\index{l\_trif\_eq (in namelist JULES\_VEGETATION)@\spxentry{l\_trif\_eq}\spxextra{in namelist JULES\_VEGETATION}|spxpagem}

\begin{fulllineitems}
\phantomsection\label{\detokenize{namelists/jules_vegetation.nml:JULES_VEGETATION::l_trif_eq}}
\pysigstartsignatures
\pysigline{\sphinxcode{\sphinxupquote{JULES\_VEGETATION::}}\sphinxbfcode{\sphinxupquote{l\_trif\_eq}}}
\pysigstopsignatures\begin{quote}\begin{description}
\sphinxlineitem{Type}
\sphinxAtStartPar
logical

\sphinxlineitem{Default}
\sphinxAtStartPar
T

\end{description}\end{quote}

\sphinxAtStartPar
Switch for equilibrium vegetation model (i.e., an equilibrium solution of TRIFFID).

\sphinxAtStartPar
Only used if {\hyperref[\detokenize{namelists/jules_vegetation.nml:JULES_VEGETATION::l_triffid}]{\sphinxcrossref{\sphinxcode{\sphinxupquote{l\_triffid}}}}} = TRUE.
\begin{description}
\sphinxlineitem{TRUE}
\sphinxAtStartPar
Use equilibrium TRIFFID.

\sphinxlineitem{FALSE}
\sphinxAtStartPar
Do not use equilibrium TRIFFID.

\end{description}

\end{fulllineitems}

\index{phenol\_period (in namelist JULES\_VEGETATION)@\spxentry{phenol\_period}\spxextra{in namelist JULES\_VEGETATION}|spxpagem}

\begin{fulllineitems}
\phantomsection\label{\detokenize{namelists/jules_vegetation.nml:JULES_VEGETATION::phenol_period}}
\pysigstartsignatures
\pysigline{\sphinxcode{\sphinxupquote{JULES\_VEGETATION::}}\sphinxbfcode{\sphinxupquote{phenol\_period}}}
\pysigstopsignatures\begin{quote}\begin{description}
\sphinxlineitem{Type}
\sphinxAtStartPar
integer

\sphinxlineitem{Permitted}
\sphinxAtStartPar
\textgreater{}= 1

\sphinxlineitem{Default}
\sphinxAtStartPar
None

\end{description}\end{quote}

\sphinxAtStartPar
Period for calls to phenology model in \sphinxstyleemphasis{days}. Only relevant if {\hyperref[\detokenize{namelists/jules_vegetation.nml:JULES_VEGETATION::l_phenol}]{\sphinxcrossref{\sphinxcode{\sphinxupquote{l\_phenol}}}}} = TRUE.

\end{fulllineitems}

\index{triffid\_period (in namelist JULES\_VEGETATION)@\spxentry{triffid\_period}\spxextra{in namelist JULES\_VEGETATION}|spxpagem}

\begin{fulllineitems}
\phantomsection\label{\detokenize{namelists/jules_vegetation.nml:JULES_VEGETATION::triffid_period}}
\pysigstartsignatures
\pysigline{\sphinxcode{\sphinxupquote{JULES\_VEGETATION::}}\sphinxbfcode{\sphinxupquote{triffid\_period}}}
\pysigstopsignatures\begin{quote}\begin{description}
\sphinxlineitem{Type}
\sphinxAtStartPar
integer

\sphinxlineitem{Permitted}
\sphinxAtStartPar
\textgreater{}= 1

\sphinxlineitem{Default}
\sphinxAtStartPar
None

\end{description}\end{quote}

\sphinxAtStartPar
Period for calls to TRIFFID model in \sphinxstyleemphasis{days}. Only relevant if one of {\hyperref[\detokenize{namelists/jules_vegetation.nml:JULES_VEGETATION::l_triffid}]{\sphinxcrossref{\sphinxcode{\sphinxupquote{l\_triffid}}}}} or {\hyperref[\detokenize{namelists/jules_vegetation.nml:JULES_VEGETATION::l_trif_eq}]{\sphinxcrossref{\sphinxcode{\sphinxupquote{l\_trif\_eq}}}}} is TRUE.

\end{fulllineitems}

\index{l\_gleaf\_fix (in namelist JULES\_VEGETATION)@\spxentry{l\_gleaf\_fix}\spxextra{in namelist JULES\_VEGETATION}|spxpagem}

\begin{fulllineitems}
\phantomsection\label{\detokenize{namelists/jules_vegetation.nml:JULES_VEGETATION::l_gleaf_fix}}
\pysigstartsignatures
\pysigline{\sphinxcode{\sphinxupquote{JULES\_VEGETATION::}}\sphinxbfcode{\sphinxupquote{l\_gleaf\_fix}}}
\pysigstopsignatures\begin{quote}\begin{description}
\sphinxlineitem{Type}
\sphinxAtStartPar
logical

\sphinxlineitem{Default}
\sphinxAtStartPar
T

\end{description}\end{quote}

\sphinxAtStartPar
Switch for fixing a bug in the accumulation of \sphinxcode{\sphinxupquote{g\_leaf\_phen\_acc}}.

\sphinxAtStartPar
This bug occurs because \sphinxcode{\sphinxupquote{veg2}} is called on TRIFFID timesteps and
\sphinxcode{\sphinxupquote{veg1}} is called on phenol timesteps, but \sphinxcode{\sphinxupquote{veg1}} did not
previously accumulate \sphinxcode{\sphinxupquote{g\_leaf\_phen\_acc}} in the same way as
\sphinxcode{\sphinxupquote{veg2}}.
\begin{description}
\sphinxlineitem{TRUE}
\sphinxAtStartPar
\sphinxcode{\sphinxupquote{veg1}} accumulates \sphinxcode{\sphinxupquote{g\_leaf\_phen\_acc}} between calls to
TRIFFID. This is important if {\hyperref[\detokenize{namelists/jules_vegetation.nml:JULES_VEGETATION::triffid_period}]{\sphinxcrossref{\sphinxcode{\sphinxupquote{triffid\_period}}}}} \textgreater{}
{\hyperref[\detokenize{namelists/jules_vegetation.nml:JULES_VEGETATION::phenol_period}]{\sphinxcrossref{\sphinxcode{\sphinxupquote{phenol\_period}}}}}.

\sphinxlineitem{FALSE}
\sphinxAtStartPar
\sphinxcode{\sphinxupquote{veg1}} does not accumulate \sphinxcode{\sphinxupquote{g\_leaf\_phen\_acc}} between calls to TRIFFID.

\end{description}

\end{fulllineitems}

\index{l\_bvoc\_emis (in namelist JULES\_VEGETATION)@\spxentry{l\_bvoc\_emis}\spxextra{in namelist JULES\_VEGETATION}|spxpagem}

\begin{fulllineitems}
\phantomsection\label{\detokenize{namelists/jules_vegetation.nml:JULES_VEGETATION::l_bvoc_emis}}
\pysigstartsignatures
\pysigline{\sphinxcode{\sphinxupquote{JULES\_VEGETATION::}}\sphinxbfcode{\sphinxupquote{l\_bvoc\_emis}}}
\pysigstopsignatures\begin{quote}\begin{description}
\sphinxlineitem{Type}
\sphinxAtStartPar
logical

\sphinxlineitem{Default}
\sphinxAtStartPar
F

\end{description}\end{quote}

\sphinxAtStartPar
Switch to enable calculation of BVOC emissions.
\begin{description}
\sphinxlineitem{TRUE}
\sphinxAtStartPar
BVOC emissions diagnostics will be calculated.

\sphinxlineitem{FALSE}
\sphinxAtStartPar
BVOC emissions diagnostics will not be calculated.

\end{description}

\end{fulllineitems}

\index{l\_o3\_damage (in namelist JULES\_VEGETATION)@\spxentry{l\_o3\_damage}\spxextra{in namelist JULES\_VEGETATION}|spxpagem}

\begin{fulllineitems}
\phantomsection\label{\detokenize{namelists/jules_vegetation.nml:JULES_VEGETATION::l_o3_damage}}
\pysigstartsignatures
\pysigline{\sphinxcode{\sphinxupquote{JULES\_VEGETATION::}}\sphinxbfcode{\sphinxupquote{l\_o3\_damage}}}
\pysigstopsignatures\begin{quote}\begin{description}
\sphinxlineitem{Type}
\sphinxAtStartPar
logical

\sphinxlineitem{Default}
\sphinxAtStartPar
F

\end{description}\end{quote}

\sphinxAtStartPar
Switch for ozone damage.
\begin{description}
\sphinxlineitem{TRUE}
\sphinxAtStartPar
Ozone damage is on.

\begin{sphinxadmonition}{note}{Note:}
\sphinxAtStartPar
Ozone concentration in ppb must be prescribed in {\hyperref[\detokenize{namelists/prescribed_data.nml::doc}]{\sphinxcrossref{\DUrole{doc}{prescribed\_data.nml}}}}.
\end{sphinxadmonition}

\sphinxlineitem{FALSE}
\sphinxAtStartPar
No effect.

\end{description}

\end{fulllineitems}

\index{l\_stem\_resp\_fix (in namelist JULES\_VEGETATION)@\spxentry{l\_stem\_resp\_fix}\spxextra{in namelist JULES\_VEGETATION}|spxpagem}

\begin{fulllineitems}
\phantomsection\label{\detokenize{namelists/jules_vegetation.nml:JULES_VEGETATION::l_stem_resp_fix}}
\pysigstartsignatures
\pysigline{\sphinxcode{\sphinxupquote{JULES\_VEGETATION::}}\sphinxbfcode{\sphinxupquote{l\_stem\_resp\_fix}}}
\pysigstopsignatures\begin{quote}\begin{description}
\sphinxlineitem{Type}
\sphinxAtStartPar
logical

\sphinxlineitem{Default}
\sphinxAtStartPar
F

\end{description}\end{quote}

\sphinxAtStartPar
Switch for bug fix for stem respiration to use balanced LAI to
derive respiring stem mass. The switch is included for backwards
compatibility with existing configurations. Future updates should
include this change.
\begin{description}
\sphinxlineitem{TRUE}
\sphinxAtStartPar
Respiring stem mass is derived allometrically.

\sphinxlineitem{FALSE}
\sphinxAtStartPar
Respiring stem mass varies with seasonal LAI.

\sphinxAtStartPar
In the case of a Broadleaf tree in the winter (no leaves) this
would mean stem respiration is scaled to 0.

\end{description}

\end{fulllineitems}

\index{l\_scale\_resp\_pm (in namelist JULES\_VEGETATION)@\spxentry{l\_scale\_resp\_pm}\spxextra{in namelist JULES\_VEGETATION}|spxpagem}

\begin{fulllineitems}
\phantomsection\label{\detokenize{namelists/jules_vegetation.nml:JULES_VEGETATION::l_scale_resp_pm}}
\pysigstartsignatures
\pysigline{\sphinxcode{\sphinxupquote{JULES\_VEGETATION::}}\sphinxbfcode{\sphinxupquote{l\_scale\_resp\_pm}}}
\pysigstopsignatures\begin{quote}\begin{description}
\sphinxlineitem{Type}
\sphinxAtStartPar
logical

\sphinxlineitem{Default}
\sphinxAtStartPar
F

\end{description}\end{quote}

\sphinxAtStartPar
Scale whole plant maintenance respiration by the soil moisture
stress factor, instead of only scaling leaf respiration.
\begin{description}
\sphinxlineitem{TRUE}
\sphinxAtStartPar
Soil moisture stress reduces leaf, root, and stem maintenance respiration.

\sphinxlineitem{FALSE}
\sphinxAtStartPar
Soil moisture stress only reduces leaf maintenance respiration.

\end{description}

\end{fulllineitems}

\index{fsmc\_shape (in namelist JULES\_VEGETATION)@\spxentry{fsmc\_shape}\spxextra{in namelist JULES\_VEGETATION}|spxpagem}

\begin{fulllineitems}
\phantomsection\label{\detokenize{namelists/jules_vegetation.nml:JULES_VEGETATION::fsmc_shape}}
\pysigstartsignatures
\pysigline{\sphinxcode{\sphinxupquote{JULES\_VEGETATION::}}\sphinxbfcode{\sphinxupquote{fsmc\_shape}}}
\pysigstopsignatures\begin{quote}\begin{description}
\sphinxlineitem{Type}
\sphinxAtStartPar
integer

\sphinxlineitem{Permitted}
\sphinxAtStartPar
0,1

\sphinxlineitem{Default}
\sphinxAtStartPar
0

\end{description}\end{quote}

\sphinxAtStartPar
Shape of soil moisture stress function on vegetation (fsmc).
\begin{enumerate}
\sphinxsetlistlabels{\arabic}{enumi}{enumii}{}{.}%
\setcounter{enumi}{-1}
\item {} 
\sphinxAtStartPar
Piece\sphinxhyphen{}wise linear in vol. soil moisture.

\item {} 
\sphinxAtStartPar
Piece\sphinxhyphen{}wise linear in soil potential. Currently only allowed when
{\hyperref[\detokenize{namelists/ancillaries.nml:JULES_SOIL_PROPS::const_z}]{\sphinxcrossref{\sphinxcode{\sphinxupquote{const\_z}}}}} = T and
{\hyperref[\detokenize{namelists/jules_vegetation.nml:JULES_VEGETATION::l_use_pft_psi}]{\sphinxcrossref{\sphinxcode{\sphinxupquote{l\_use\_pft\_psi}}}}} = T.

\end{enumerate}

\begin{sphinxadmonition}{note}{Note:}
\sphinxAtStartPar
The option {\hyperref[\detokenize{namelists/jules_vegetation.nml:JULES_VEGETATION::fsmc_shape}]{\sphinxcrossref{\sphinxcode{\sphinxupquote{fsmc\_shape}}}}} = 1 is
still in development. Users should ensure that results
are as expected, and provide feedback where deficiencies
are identified.
\end{sphinxadmonition}

\end{fulllineitems}

\index{l\_use\_pft\_psi (in namelist JULES\_VEGETATION)@\spxentry{l\_use\_pft\_psi}\spxextra{in namelist JULES\_VEGETATION}|spxpagem}

\begin{fulllineitems}
\phantomsection\label{\detokenize{namelists/jules_vegetation.nml:JULES_VEGETATION::l_use_pft_psi}}
\pysigstartsignatures
\pysigline{\sphinxcode{\sphinxupquote{JULES\_VEGETATION::}}\sphinxbfcode{\sphinxupquote{l\_use\_pft\_psi}}}
\pysigstopsignatures\begin{quote}\begin{description}
\sphinxlineitem{Type}
\sphinxAtStartPar
logical

\sphinxlineitem{Default}
\sphinxAtStartPar
F

\end{description}\end{quote}

\sphinxAtStartPar
Switch for parameters in the soil moisture stress on vegetation function (fsmc).
\begin{description}
\sphinxlineitem{TRUE}
\sphinxAtStartPar
Fsmc is calculated from {\hyperref[\detokenize{namelists/pft_params.nml:JULES_PFTPARM::psi_close_io}]{\sphinxcrossref{\sphinxcode{\sphinxupquote{psi\_close\_io}}}}}
and {\hyperref[\detokenize{namelists/pft_params.nml:JULES_PFTPARM::psi_open_io}]{\sphinxcrossref{\sphinxcode{\sphinxupquote{psi\_open\_io}}}}}.

\sphinxlineitem{FALSE}
\sphinxAtStartPar
Fsmc is calculated from \sphinxcode{\sphinxupquote{sm\_wilt}} and \sphinxcode{\sphinxupquote{sm\_crit}} in
{\hyperref[\detokenize{namelists/ancillaries.nml:namelist-JULES_SOIL_PROPS}]{\sphinxcrossref{\sphinxcode{\sphinxupquote{JULES\_SOIL\_PROPS}}}}} and
{\hyperref[\detokenize{namelists/pft_params.nml:JULES_PFTPARM::fsmc_p0_io}]{\sphinxcrossref{\sphinxcode{\sphinxupquote{fsmc\_p0\_io}}}}}.

\end{description}

\begin{sphinxadmonition}{note}{Note:}
\sphinxAtStartPar
Soil respiration and surface conductance of bare soil
respectively will depend on \sphinxcode{\sphinxupquote{sm\_wilt}} and \sphinxcode{\sphinxupquote{sm\_crit}}
in {\hyperref[\detokenize{namelists/ancillaries.nml:namelist-JULES_SOIL_PROPS}]{\sphinxcrossref{\sphinxcode{\sphinxupquote{JULES\_SOIL\_PROPS}}}}}, regardless of the setting
of {\hyperref[\detokenize{namelists/jules_vegetation.nml:JULES_VEGETATION::fsmc_shape}]{\sphinxcrossref{\sphinxcode{\sphinxupquote{fsmc\_shape}}}}}.
\end{sphinxadmonition}

\begin{sphinxadmonition}{note}{Note:}
\sphinxAtStartPar
The option {\hyperref[\detokenize{namelists/jules_vegetation.nml:JULES_VEGETATION::l_use_pft_psi}]{\sphinxcrossref{\sphinxcode{\sphinxupquote{l\_use\_pft\_psi}}}}} = T
is still in development. Users should ensure that results
are as expected, and provide feedback where deficiencies
are identified.
\end{sphinxadmonition}

\end{fulllineitems}

\index{l\_vegcan\_soilfx (in namelist JULES\_VEGETATION)@\spxentry{l\_vegcan\_soilfx}\spxextra{in namelist JULES\_VEGETATION}|spxpagem}

\begin{fulllineitems}
\phantomsection\label{\detokenize{namelists/jules_vegetation.nml:JULES_VEGETATION::l_vegcan_soilfx}}
\pysigstartsignatures
\pysigline{\sphinxcode{\sphinxupquote{JULES\_VEGETATION::}}\sphinxbfcode{\sphinxupquote{l\_vegcan\_soilfx}}}
\pysigstopsignatures\begin{quote}\begin{description}
\sphinxlineitem{Type}
\sphinxAtStartPar
logical

\sphinxlineitem{Default}
\sphinxAtStartPar
F

\end{description}\end{quote}

\sphinxAtStartPar
Switch for enhancement to canopy model to allow for conduction in
the soil below the vegetative canopy, reducing coupling between the
soil and the canopy.
\begin{description}
\sphinxlineitem{TRUE}
\sphinxAtStartPar
Allow for conduction in the soil.

\sphinxlineitem{FALSE}
\sphinxAtStartPar
No effect.

\end{description}

\end{fulllineitems}

\index{l\_leaf\_n\_resp\_fix (in namelist JULES\_VEGETATION)@\spxentry{l\_leaf\_n\_resp\_fix}\spxextra{in namelist JULES\_VEGETATION}|spxpagem}

\begin{fulllineitems}
\phantomsection\label{\detokenize{namelists/jules_vegetation.nml:JULES_VEGETATION::l_leaf_n_resp_fix}}
\pysigstartsignatures
\pysigline{\sphinxcode{\sphinxupquote{JULES\_VEGETATION::}}\sphinxbfcode{\sphinxupquote{l\_leaf\_n\_resp\_fix}}}
\pysigstopsignatures\begin{quote}\begin{description}
\sphinxlineitem{Type}
\sphinxAtStartPar
logical

\sphinxlineitem{Default}
\sphinxAtStartPar
F

\end{description}\end{quote}

\sphinxAtStartPar
Switch for bug fix for leaf nitrogen content used in the
calculation of plant maintenance respiration. The switch is
included for backwards compatibility with existing
configurations. Runs with {\hyperref[\detokenize{namelists/jules_vegetation.nml:JULES_VEGETATION::can_rad_mod}]{\sphinxcrossref{\sphinxcode{\sphinxupquote{can\_rad\_mod}}}}} = 1, 4 or 5 are
affected.
\begin{description}
\sphinxlineitem{TRUE}
\sphinxAtStartPar
Use correct forms for canopy\sphinxhyphen{}average leaf N content.

\sphinxlineitem{FALSE}
\sphinxAtStartPar
No effect.

\end{description}

\end{fulllineitems}

\index{l\_landuse (in namelist JULES\_VEGETATION)@\spxentry{l\_landuse}\spxextra{in namelist JULES\_VEGETATION}|spxpagem}

\begin{fulllineitems}
\phantomsection\label{\detokenize{namelists/jules_vegetation.nml:JULES_VEGETATION::l_landuse}}
\pysigstartsignatures
\pysigline{\sphinxcode{\sphinxupquote{JULES\_VEGETATION::}}\sphinxbfcode{\sphinxupquote{l\_landuse}}}
\pysigstopsignatures\begin{quote}\begin{description}
\sphinxlineitem{Type}
\sphinxAtStartPar
logical

\sphinxlineitem{Default}
\sphinxAtStartPar
F

\end{description}\end{quote}

\sphinxAtStartPar
Switch for using landuse change in conjunction with TRIFFID

\sphinxAtStartPar
Only used if {\hyperref[\detokenize{namelists/jules_vegetation.nml:JULES_VEGETATION::l_triffid}]{\sphinxcrossref{\sphinxcode{\sphinxupquote{l\_triffid}}}}} = TRUE.
\begin{description}
\sphinxlineitem{TRUE}
\sphinxAtStartPar
Land use change is implemented within TRIFFID. Litter fluxes
are split between soil and wood product pools. Requires
additional prognostics covering the product pools and the
agricultural fraction from the previous TRIFFID call.

\sphinxlineitem{FALSE}
\sphinxAtStartPar
All litter fluxes enter the soil

\end{description}

\end{fulllineitems}

\index{l\_recon (in namelist JULES\_VEGETATION)@\spxentry{l\_recon}\spxextra{in namelist JULES\_VEGETATION}|spxpagem}

\begin{fulllineitems}
\phantomsection\label{\detokenize{namelists/jules_vegetation.nml:JULES_VEGETATION::l_recon}}
\pysigstartsignatures
\pysigline{\sphinxcode{\sphinxupquote{JULES\_VEGETATION::}}\sphinxbfcode{\sphinxupquote{l\_recon}}}
\pysigstopsignatures\begin{quote}\begin{description}
\sphinxlineitem{Type}
\sphinxAtStartPar
logical

\sphinxlineitem{Default}
\sphinxAtStartPar
T

\end{description}\end{quote}

\sphinxAtStartPar
Switch for reconfiguring vegetation fractions. Also initialises
vegetation and soil biogeochemistry at land ice points. With the
ECOSSE soil model this switch also ensures that the initial
condition for soil biogeochemistry is internally consistent.
\begin{description}
\sphinxlineitem{TRUE}
\sphinxAtStartPar
For soil points (land points with no ice) ensure vegetation
fractions are at least a minimum value and reduce other
fractions accordingly.

\sphinxlineitem{FALSE}
\sphinxAtStartPar
Do not apply the minimum vegetation fractions. This is useful
when some points are 100\% lake and urban, in which case
reconfiguration leads to a total surface tile fraction of greater
than 1.

\end{description}

\end{fulllineitems}

\index{l\_prescsow (in namelist JULES\_VEGETATION)@\spxentry{l\_prescsow}\spxextra{in namelist JULES\_VEGETATION}|spxpagem}

\begin{fulllineitems}
\phantomsection\label{\detokenize{namelists/jules_vegetation.nml:JULES_VEGETATION::l_prescsow}}
\pysigstartsignatures
\pysigline{\sphinxcode{\sphinxupquote{JULES\_VEGETATION::}}\sphinxbfcode{\sphinxupquote{l\_prescsow}}}
\pysigstopsignatures\begin{quote}\begin{description}
\sphinxlineitem{Type}
\sphinxAtStartPar
logical

\sphinxlineitem{Default}
\sphinxAtStartPar
F

\end{description}\end{quote}

\sphinxAtStartPar
Switch that determines how crop sowing dates are defined. Only used
if {\hyperref[\detokenize{namelists/jules_surface_types.nml:JULES_SURFACE_TYPES::ncpft}]{\sphinxcrossref{\sphinxcode{\sphinxupquote{ncpft}}}}} \textgreater{} 0.
\begin{description}
\sphinxlineitem{TRUE}
\sphinxAtStartPar
Sowing dates prescribed in {\hyperref[\detokenize{namelists/ancillaries.nml:namelist-JULES_CROP_PROPS}]{\sphinxcrossref{\sphinxcode{\sphinxupquote{JULES\_CROP\_PROPS}}}}} are used.

\sphinxlineitem{FALSE}
\sphinxAtStartPar
Sowing dates are determined by the model.

\end{description}

\end{fulllineitems}

\index{l\_trif\_crop (in namelist JULES\_VEGETATION)@\spxentry{l\_trif\_crop}\spxextra{in namelist JULES\_VEGETATION}|spxpagem}

\begin{fulllineitems}
\phantomsection\label{\detokenize{namelists/jules_vegetation.nml:JULES_VEGETATION::l_trif_crop}}
\pysigstartsignatures
\pysigline{\sphinxcode{\sphinxupquote{JULES\_VEGETATION::}}\sphinxbfcode{\sphinxupquote{l\_trif\_crop}}}
\pysigstopsignatures\begin{quote}\begin{description}
\sphinxlineitem{Type}
\sphinxAtStartPar
logical

\sphinxlineitem{Default}
\sphinxAtStartPar
F

\end{description}\end{quote}

\sphinxAtStartPar
Switch controlling the treatment of agricultural PFTs. Where
agricultural PFTs are defined by the
{\hyperref[\detokenize{namelists/triffid_params.nml:JULES_TRIFFID::crop_io}]{\sphinxcrossref{\sphinxcode{\sphinxupquote{crop\_io}}}}} parameter.
\begin{description}
\sphinxlineitem{TRUE}
\sphinxAtStartPar
In the non\sphinxhyphen{}agricultural area natural PFT competition is
calculated by a call to a new version of the lotka routine and
in each agricultural area agricultural\sphinxhyphen{}PFT competition is
calculated by an additional call to the new version of the
lotka routine. Crop and pasture areas are defined by the
{\hyperref[\detokenize{namelists/ancillaries.nml:JULES_AGRIC::frac_agr}]{\sphinxcrossref{\sphinxcode{\sphinxupquote{frac\_agr}}}}} and
{\hyperref[\detokenize{namelists/ancillaries.nml:JULES_AGRIC::frac_past}]{\sphinxcrossref{\sphinxcode{\sphinxupquote{frac\_past}}}}} variables
respectively. Additionally, to represent harvesting, a fraction
of crop litter is added to the fast wood products pool instead
of the soil carbon pools.

\sphinxlineitem{FALSE}
\sphinxAtStartPar
Vegetation competition is calculated for natural and crop PFTs
together, with natural PFTs excluded from the agricultural area
that is defined by the {\hyperref[\detokenize{namelists/ancillaries.nml:JULES_AGRIC::frac_agr}]{\sphinxcrossref{\sphinxcode{\sphinxupquote{frac\_agr}}}}}
variable. Agricultural PFTs can also grow in natural areas
where they are interpreted as natural grasses.

\end{description}

\end{fulllineitems}

\index{l\_trif\_biocrop (in namelist JULES\_VEGETATION)@\spxentry{l\_trif\_biocrop}\spxextra{in namelist JULES\_VEGETATION}|spxpagem}

\begin{fulllineitems}
\phantomsection\label{\detokenize{namelists/jules_vegetation.nml:JULES_VEGETATION::l_trif_biocrop}}
\pysigstartsignatures
\pysigline{\sphinxcode{\sphinxupquote{JULES\_VEGETATION::}}\sphinxbfcode{\sphinxupquote{l\_trif\_biocrop}}}
\pysigstopsignatures\begin{quote}
\begin{quote}\begin{description}
\sphinxlineitem{type}
\sphinxAtStartPar
logical

\sphinxlineitem{default}
\sphinxAtStartPar
F

\end{description}\end{quote}

\sphinxAtStartPar
Allows for representation of bioenergy crops with continuous or periodic harvesting of agricultural PFTs at prescribed intervals. Requires {\hyperref[\detokenize{namelists/jules_vegetation.nml:JULES_VEGETATION::l_trif_crop}]{\sphinxcrossref{\sphinxcode{\sphinxupquote{l\_trif\_crop}}}}} = TRUE.
\begin{description}
\sphinxlineitem{TRUE}
\sphinxAtStartPar
Crop, pasture, and bioenergy crop areas are defined by the {\hyperref[\detokenize{namelists/ancillaries.nml:JULES_AGRIC::frac_agr}]{\sphinxcrossref{\sphinxcode{\sphinxupquote{frac\_agr}}}}}, {\hyperref[\detokenize{namelists/ancillaries.nml:JULES_AGRIC::frac_past}]{\sphinxcrossref{\sphinxcode{\sphinxupquote{frac\_past}}}}}, {\hyperref[\detokenize{namelists/ancillaries.nml:JULES_AGRIC::frac_biocrop}]{\sphinxcrossref{\sphinxcode{\sphinxupquote{frac\_biocrop}}}}} variables respectively. Harvests are permitted from any land class and enabled for each PFT separately using the {\hyperref[\detokenize{namelists/triffid_params.nml:JULES_TRIFFID::harvest_type_io}]{\sphinxcrossref{\sphinxcode{\sphinxupquote{harvest\_type\_io}}}}} variable. Harvesting may be continuous (as per the existing scheme in  {\hyperref[\detokenize{namelists/jules_vegetation.nml:JULES_VEGETATION::l_trif_crop}]{\sphinxcrossref{\sphinxcode{\sphinxupquote{l\_trif\_crop}}}}}, when {\hyperref[\detokenize{namelists/triffid_params.nml:JULES_TRIFFID::harvest_type_io}]{\sphinxcrossref{\sphinxcode{\sphinxupquote{harvest\_type\_io}}}}} is 1), or performed at prescribed intervals defined using the {\hyperref[\detokenize{namelists/triffid_params.nml:JULES_TRIFFID::harvest_freq_io}]{\sphinxcrossref{\sphinxcode{\sphinxupquote{harvest\_freq\_io}}}}} and {\hyperref[\detokenize{namelists/triffid_params.nml:JULES_TRIFFID::harvest_ht_io}]{\sphinxcrossref{\sphinxcode{\sphinxupquote{harvest\_ht\_io}}}}} variables (when {\hyperref[\detokenize{namelists/triffid_params.nml:JULES_TRIFFID::harvest_type_io}]{\sphinxcrossref{\sphinxcode{\sphinxupquote{harvest\_type\_io}}}}} is 2).

\sphinxlineitem{FALSE}
\sphinxAtStartPar
Land use classes, PFT partitioning, and harvests are as defined by the {\hyperref[\detokenize{namelists/jules_vegetation.nml:JULES_VEGETATION::l_trif_crop}]{\sphinxcrossref{\sphinxcode{\sphinxupquote{l\_trif\_crop}}}}} switch.

\end{description}
\end{quote}


\sphinxstrong{See also:}
\nopagebreak


\sphinxAtStartPar
References:
\begin{itemize}
\item {} 
\sphinxAtStartPar
Littleton et al., 2020, JULES\sphinxhyphen{}BE: representation of bioenergy crops and harvesting in the Joint UK Land Environment Simulator vn5.1, Geosci. Model Dev., \sphinxurl{https://doi.org/10.5194/gmd-13-1123-2020}

\end{itemize}



\end{fulllineitems}

\index{l\_ag\_expand (in namelist JULES\_VEGETATION)@\spxentry{l\_ag\_expand}\spxextra{in namelist JULES\_VEGETATION}|spxpagem}

\begin{fulllineitems}
\phantomsection\label{\detokenize{namelists/jules_vegetation.nml:JULES_VEGETATION::l_ag_expand}}
\pysigstartsignatures
\pysigline{\sphinxcode{\sphinxupquote{JULES\_VEGETATION::}}\sphinxbfcode{\sphinxupquote{l\_ag\_expand}}}
\pysigstopsignatures\begin{quote}\begin{description}
\sphinxlineitem{Type}
\sphinxAtStartPar
logical

\sphinxlineitem{Default}
\sphinxAtStartPar
F

\end{description}\end{quote}

\sphinxAtStartPar
Allows for assisted expansion of agricultural crop areas. Requires {\hyperref[\detokenize{namelists/jules_vegetation.nml:JULES_VEGETATION::l_landuse}]{\sphinxcrossref{\sphinxcode{\sphinxupquote{l\_landuse}}}}} = TRUE.
\begin{description}
\sphinxlineitem{TRUE}
\sphinxAtStartPar
Automatically plant out new crop areas with target PFTs.

\sphinxlineitem{FALSE}
\sphinxAtStartPar
No automatic increase of PFT fraction when land class fraction increases.

\end{description}

\end{fulllineitems}

\index{can\_model (in namelist JULES\_VEGETATION)@\spxentry{can\_model}\spxextra{in namelist JULES\_VEGETATION}|spxpagem}

\begin{fulllineitems}
\phantomsection\label{\detokenize{namelists/jules_vegetation.nml:JULES_VEGETATION::can_model}}
\pysigstartsignatures
\pysigline{\sphinxcode{\sphinxupquote{JULES\_VEGETATION::}}\sphinxbfcode{\sphinxupquote{can\_model}}}
\pysigstopsignatures\begin{quote}\begin{description}
\sphinxlineitem{Type}
\sphinxAtStartPar
integer

\sphinxlineitem{Permitted}
\sphinxAtStartPar
1\sphinxhyphen{}4

\sphinxlineitem{Default}
\sphinxAtStartPar
4

\end{description}\end{quote}

\sphinxAtStartPar
Choice of canopy model for vegetation:
\begin{enumerate}
\sphinxsetlistlabels{\arabic}{enumi}{enumii}{}{.}%
\item {} 
\sphinxAtStartPar
No distinct canopy (i.e. surface is represented as a single entity for radiative processes).

\item {} 
\sphinxAtStartPar
Radiative canopy with no heat capacity.

\item {} 
\sphinxAtStartPar
Radiative canopy with heat capacity. This option is deprecated, with 4 preferred.

\item {} 
\sphinxAtStartPar
As 3 but with a representation of snow beneath the canopy. This option is preferred to 3.

\end{enumerate}

\begin{sphinxadmonition}{note}{Note:}\begin{description}
\sphinxlineitem{{\hyperref[\detokenize{namelists/jules_vegetation.nml:JULES_VEGETATION::can_model}]{\sphinxcrossref{\sphinxcode{\sphinxupquote{can\_model}}}}} = 1 does not mean that there is no}
\sphinxAtStartPar
vegetation canopy. It means that the surface is
represented as a single entity, rather than having
distinct surface and canopy levels for the purposes of
radiative processes.

\end{description}
\end{sphinxadmonition}

\end{fulllineitems}

\index{can\_rad\_mod (in namelist JULES\_VEGETATION)@\spxentry{can\_rad\_mod}\spxextra{in namelist JULES\_VEGETATION}|spxpagem}

\begin{fulllineitems}
\phantomsection\label{\detokenize{namelists/jules_vegetation.nml:JULES_VEGETATION::can_rad_mod}}
\pysigstartsignatures
\pysigline{\sphinxcode{\sphinxupquote{JULES\_VEGETATION::}}\sphinxbfcode{\sphinxupquote{can\_rad\_mod}}}
\pysigstopsignatures\begin{quote}\begin{description}
\sphinxlineitem{Type}
\sphinxAtStartPar
integer

\sphinxlineitem{Permitted}
\sphinxAtStartPar
1, 4, 5, 6

\sphinxlineitem{Default}
\sphinxAtStartPar
4

\end{description}\end{quote}

\sphinxAtStartPar
Options for treatment of canopy radiation.
\begin{enumerate}
\sphinxsetlistlabels{\arabic}{enumi}{enumii}{}{.}%
\item {} 
\sphinxAtStartPar
A single canopy layer for which radiation absorption is
calculated using Beer’s law. Leaf\sphinxhyphen{}level photosynthesis is scaled
to the canopy level using the ‘big leaf’ approach. Leaf
nitrogen, photosynthetic capacity, i.e the Vcmax parameter, and
leaf photosynthesis vary exponentially through the canopy with
radiation.

\end{enumerate}
\begin{enumerate}
\sphinxsetlistlabels{\arabic}{enumi}{enumii}{}{.}%
\setcounter{enumi}{3}
\item {} 
\sphinxAtStartPar
Multi\sphinxhyphen{}layer approach for radiation interception following the
two\sphinxhyphen{}stream approach of {\hyperref[\detokenize{namelists/jules_vegetation.nml:references-vegetation}]{\sphinxcrossref{\DUrole{std,std-ref}{Sellers et
al. (1992)}}}}. This approach takes into account leaf
angle distribution, zenith angle, and differentiates absorption
of direct and diffuse radiation. It has an exponential decline
of leaf N through the canopy and includes inhibition of leaf
respiration in the light. Canopy photosynthesis and conductance
are calculated as the sum over all layers.

\item {} 
\sphinxAtStartPar
This is an improvement of option 4, including:
\begin{itemize}
\item {} 
\sphinxAtStartPar
Sunfleck penetration though the canopy.

\item {} 
\sphinxAtStartPar
Division of sunlit and shaded leaves within each canopy level.

\item {} 
\sphinxAtStartPar
A modified version of inhibition of leaf respiration in the light.

\end{itemize}

\item {} 
\sphinxAtStartPar
This is an improvement of option 5, including an exponential
decline of leaf N with canopy height proportional to LAI,
following Beer’s law.

\end{enumerate}

\begin{sphinxadmonition}{note}{Note:}
\sphinxAtStartPar
{\hyperref[\detokenize{namelists/jules_vegetation.nml:JULES_VEGETATION::can_rad_mod}]{\sphinxcrossref{\sphinxcode{\sphinxupquote{can\_rad\_mod}}}}} = 1 and 6 are recommended.
\end{sphinxadmonition}

\begin{sphinxadmonition}{note}{Note:}
\sphinxAtStartPar
When using {\hyperref[\detokenize{namelists/jules_vegetation.nml:JULES_VEGETATION::can_rad_mod}]{\sphinxcrossref{\sphinxcode{\sphinxupquote{can\_rad\_mod}}}}} = 4, 5 or 6 it is
recommended to use driving data that contains direct and
diffuse radiation separately rather than a constant
diffuse fraction.
\end{sphinxadmonition}


\sphinxstrong{See also:}
\nopagebreak


\sphinxAtStartPar
Descriptions of option 1 can be found in
{\hyperref[\detokenize{namelists/jules_vegetation.nml:references-vegetation}]{\sphinxcrossref{\DUrole{std,std-ref}{Jogireddy et al. (2006)}}}}, and an
application of option 4 can be found in {\hyperref[\detokenize{namelists/jules_vegetation.nml:references-vegetation}]{\sphinxcrossref{\DUrole{std,std-ref}{Mercado
et al. (2007)}}}}. Options 1 to 5 are
described in {\hyperref[\detokenize{namelists/jules_vegetation.nml:references-vegetation}]{\sphinxcrossref{\DUrole{std,std-ref}{Clark et al (2011)}}}}.



\end{fulllineitems}

\index{ilayers (in namelist JULES\_VEGETATION)@\spxentry{ilayers}\spxextra{in namelist JULES\_VEGETATION}|spxpagem}

\begin{fulllineitems}
\phantomsection\label{\detokenize{namelists/jules_vegetation.nml:JULES_VEGETATION::ilayers}}
\pysigstartsignatures
\pysigline{\sphinxcode{\sphinxupquote{JULES\_VEGETATION::}}\sphinxbfcode{\sphinxupquote{ilayers}}}
\pysigstopsignatures\begin{quote}\begin{description}
\sphinxlineitem{Type}
\sphinxAtStartPar
integer

\sphinxlineitem{Permitted}
\sphinxAtStartPar
\textgreater{}= 0

\sphinxlineitem{Default}
\sphinxAtStartPar
10

\end{description}\end{quote}

\sphinxAtStartPar
Number of layers for canopy radiation model. Only used for {\hyperref[\detokenize{namelists/jules_vegetation.nml:JULES_VEGETATION::can_rad_mod}]{\sphinxcrossref{\sphinxcode{\sphinxupquote{can\_rad\_mod}}}}} = 4, 5 or 6.

\sphinxAtStartPar
These layers are used for the calculations of radiation interception and photosynthesis.

\end{fulllineitems}

\index{photo\_model (in namelist JULES\_VEGETATION)@\spxentry{photo\_model}\spxextra{in namelist JULES\_VEGETATION}|spxpagem}

\begin{fulllineitems}
\phantomsection\label{\detokenize{namelists/jules_vegetation.nml:JULES_VEGETATION::photo_model}}
\pysigstartsignatures
\pysigline{\sphinxcode{\sphinxupquote{JULES\_VEGETATION::}}\sphinxbfcode{\sphinxupquote{photo\_model}}}
\pysigstopsignatures\begin{quote}\begin{description}
\sphinxlineitem{Type}
\sphinxAtStartPar
integer

\sphinxlineitem{Permitted}
\sphinxAtStartPar
1 or 2

\sphinxlineitem{Default}
\sphinxAtStartPar
none

\end{description}\end{quote}

\sphinxAtStartPar
Choice for model of leaf photosynthesis.

\sphinxAtStartPar
Possible values are:
\begin{enumerate}
\sphinxsetlistlabels{\arabic}{enumi}{enumii}{}{.}%
\item {} 
\begin{DUlineblock}{0em}
\item[] C$_{\text{3}}$ and C$_{\text{4}}$ plants use the models of Collatz et al., 1991 and 1992, respectively. These were used in the original JULES model.
\end{DUlineblock}

\item {} 
\begin{DUlineblock}{0em}
\item[] C$_{\text{3}}$ plants use the model of Farquhar et al. (1980); C$_{\text{4}}$ plants use the model of Collatz et al. (1992).
\end{DUlineblock}

\end{enumerate}

\begin{sphinxadmonition}{warning}{Warning:}
\sphinxAtStartPar
The Farquhar model can only be used if {\hyperref[\detokenize{namelists/jules_vegetation.nml:JULES_VEGETATION::can_rad_mod}]{\sphinxcrossref{\sphinxcode{\sphinxupquote{can\_rad\_mod}}}}} = 1, 5 or 6. Code has not been written for other values of {\hyperref[\detokenize{namelists/jules_vegetation.nml:JULES_VEGETATION::can_rad_mod}]{\sphinxcrossref{\sphinxcode{\sphinxupquote{can\_rad\_mod}}}}}.
\end{sphinxadmonition}


\sphinxstrong{See also:}
\nopagebreak


\sphinxAtStartPar
References:
\begin{itemize}
\item {} 
\sphinxAtStartPar
Collatz et al., 1991, Physiological and environmental regulation of stomatal conductance, photosynthesis, and transpiration \textendash{} a model that includes a laminar boundary layer, Agricultural and Forest Meteorology, \sphinxurl{https://doi.org/10.1016/0168-1923(91)90002-8}.

\item {} 
\sphinxAtStartPar
Collatz et al., 1992, Coupled Photosynthesis\sphinxhyphen{}Stomatal Conductance Model for Leaves of C$_{\text{4}}$ Plants, Australian Journal of Plant Physiology, \sphinxurl{https://doi.org/10.1071/PP9920519}.

\item {} 
\sphinxAtStartPar
Farquhar et al., 1980, A biochemical model of photosynthetic CO$_{\text{2}}$ assimilation in leaves of C$_{\text{3}}$ species, Planta, \sphinxurl{https://doi.org/10.1007/BF0038623}.

\end{itemize}



\end{fulllineitems}

\index{stomata\_model (in namelist JULES\_VEGETATION)@\spxentry{stomata\_model}\spxextra{in namelist JULES\_VEGETATION}|spxpagem}

\begin{fulllineitems}
\phantomsection\label{\detokenize{namelists/jules_vegetation.nml:JULES_VEGETATION::stomata_model}}
\pysigstartsignatures
\pysigline{\sphinxcode{\sphinxupquote{JULES\_VEGETATION::}}\sphinxbfcode{\sphinxupquote{stomata\_model}}}
\pysigstopsignatures\begin{quote}\begin{description}
\sphinxlineitem{Type}
\sphinxAtStartPar
integer

\sphinxlineitem{Permitted}
\sphinxAtStartPar
1 or 2

\sphinxlineitem{Default}
\sphinxAtStartPar
1

\end{description}\end{quote}

\sphinxAtStartPar
Choice for model of stomatal conductance.

\sphinxAtStartPar
Possible values are:
\begin{enumerate}
\sphinxsetlistlabels{\arabic}{enumi}{enumii}{}{.}%
\item {} 
\sphinxAtStartPar
The original JULES model, including the Jacobs closure \sphinxhyphen{} see
Eqn.9 of {\hyperref[\detokenize{namelists/jules_vegetation.nml:references-vegetation}]{\sphinxcrossref{\DUrole{std,std-ref}{Best et al. (2011)}}}}.

\item {} 
\sphinxAtStartPar
The model of {\hyperref[\detokenize{namelists/jules_vegetation.nml:references-vegetation}]{\sphinxcrossref{\DUrole{std,std-ref}{Medlyn et al. (2011)}}}} \sphinxhyphen{} see
Eqn.11 of that paper, and {\hyperref[\detokenize{namelists/jules_vegetation.nml:references-vegetation}]{\sphinxcrossref{\DUrole{std,std-ref}{Medlyn et al
(2012)}}}}. Note that as implemented the model uses a
single parameter (g$_{\text{1}}$, assuming that g$_{\text{0}}$ = 0).

\end{enumerate}

\begin{sphinxadmonition}{warning}{Warning:}
\sphinxAtStartPar
Only the original (Jacobs) model can currently be used with the
UM (Option 1).
\end{sphinxadmonition}

\end{fulllineitems}

\index{frac\_min (in namelist JULES\_VEGETATION)@\spxentry{frac\_min}\spxextra{in namelist JULES\_VEGETATION}|spxpagem}

\begin{fulllineitems}
\phantomsection\label{\detokenize{namelists/jules_vegetation.nml:JULES_VEGETATION::frac_min}}
\pysigstartsignatures
\pysigline{\sphinxcode{\sphinxupquote{JULES\_VEGETATION::}}\sphinxbfcode{\sphinxupquote{frac\_min}}}
\pysigstopsignatures\begin{quote}\begin{description}
\sphinxlineitem{Type}
\sphinxAtStartPar
real

\sphinxlineitem{Default}
\sphinxAtStartPar
1.0e\sphinxhyphen{}6

\end{description}\end{quote}

\sphinxAtStartPar
Minimum fraction that a PFT is allowed to cover if TRIFFID is used.

\end{fulllineitems}

\index{frac\_seed (in namelist JULES\_VEGETATION)@\spxentry{frac\_seed}\spxextra{in namelist JULES\_VEGETATION}|spxpagem}

\begin{fulllineitems}
\phantomsection\label{\detokenize{namelists/jules_vegetation.nml:JULES_VEGETATION::frac_seed}}
\pysigstartsignatures
\pysigline{\sphinxcode{\sphinxupquote{JULES\_VEGETATION::}}\sphinxbfcode{\sphinxupquote{frac\_seed}}}
\pysigstopsignatures\begin{quote}\begin{description}
\sphinxlineitem{Type}
\sphinxAtStartPar
real

\sphinxlineitem{Default}
\sphinxAtStartPar
0.01

\end{description}\end{quote}

\sphinxAtStartPar
Seed fraction for TRIFFID.

\end{fulllineitems}

\index{pow (in namelist JULES\_VEGETATION)@\spxentry{pow}\spxextra{in namelist JULES\_VEGETATION}|spxpagem}

\begin{fulllineitems}
\phantomsection\label{\detokenize{namelists/jules_vegetation.nml:JULES_VEGETATION::pow}}
\pysigstartsignatures
\pysigline{\sphinxcode{\sphinxupquote{JULES\_VEGETATION::}}\sphinxbfcode{\sphinxupquote{pow}}}
\pysigstopsignatures\begin{quote}\begin{description}
\sphinxlineitem{Type}
\sphinxAtStartPar
real

\sphinxlineitem{Default}
\sphinxAtStartPar
5.241e\sphinxhyphen{}4

\end{description}\end{quote}

\sphinxAtStartPar
Power in sigmodial function used to get competition coefficients.

\sphinxAtStartPar
See Hadley Centre Technical Note 24, Eq.3.

\end{fulllineitems}

\index{l\_inferno (in namelist JULES\_VEGETATION)@\spxentry{l\_inferno}\spxextra{in namelist JULES\_VEGETATION}|spxpagem}

\begin{fulllineitems}
\phantomsection\label{\detokenize{namelists/jules_vegetation.nml:JULES_VEGETATION::l_inferno}}
\pysigstartsignatures
\pysigline{\sphinxcode{\sphinxupquote{JULES\_VEGETATION::}}\sphinxbfcode{\sphinxupquote{l\_inferno}}}
\pysigstopsignatures\begin{quote}\begin{description}
\sphinxlineitem{Type}
\sphinxAtStartPar
logical

\sphinxlineitem{Default}
\sphinxAtStartPar
F

\end{description}\end{quote}

\sphinxAtStartPar
Switch that determines whether interactive fires (INFERNO) is
used. This allows for the diagnostic of burnt area, burnt carbon
and a variety of fire emissions.
\begin{description}
\sphinxlineitem{TRUE}
\sphinxAtStartPar
INFERNO is used to provide diagnostic fire variables

\sphinxlineitem{FALSE}
\sphinxAtStartPar
INFERNO is not used.

\end{description}

\end{fulllineitems}

\index{ignition\_method (in namelist JULES\_VEGETATION)@\spxentry{ignition\_method}\spxextra{in namelist JULES\_VEGETATION}|spxpagem}

\begin{fulllineitems}
\phantomsection\label{\detokenize{namelists/jules_vegetation.nml:JULES_VEGETATION::ignition_method}}
\pysigstartsignatures
\pysigline{\sphinxcode{\sphinxupquote{JULES\_VEGETATION::}}\sphinxbfcode{\sphinxupquote{ignition\_method}}}
\pysigstopsignatures\begin{quote}\begin{description}
\sphinxlineitem{Type}
\sphinxAtStartPar
integer

\sphinxlineitem{Permitted}
\sphinxAtStartPar
1, 2, 3

\sphinxlineitem{Default}
\sphinxAtStartPar
1

\end{description}\end{quote}

\sphinxAtStartPar
Switch to determine the type of ignition used (ubiquitous or prescribed with population and lightning)
\begin{enumerate}
\sphinxsetlistlabels{\arabic}{enumi}{enumii}{}{.}%
\item {} 
\sphinxAtStartPar
INFERNO uses ubiquitous (constant) ignitions, of 1.67 fires km$^{\text{\sphinxhyphen{}2}}$ s$^{\text{\sphinxhyphen{}1}}$ (1.5 from humans, 0.17 from lightning).

\item {} 
\sphinxAtStartPar
INFERNO uses prescribed lightning ignitions, either from an ancillary or the UM.
Meanwhile humans are assumed to ignite 1.5 fires km$^{\text{\sphinxhyphen{}2}}$ s$^{\text{\sphinxhyphen{}1}}$.

\item {} 
\sphinxAtStartPar
INFERNO uses prescribed ignition using Population Density and Lightning Frequency (Cloud\sphinxhyphen{}to\sphinxhyphen{}Ground).
These must be provided as prescribed data to the JULES run.

\end{enumerate}

\end{fulllineitems}

\index{l\_trif\_fire (in namelist JULES\_VEGETATION)@\spxentry{l\_trif\_fire}\spxextra{in namelist JULES\_VEGETATION}|spxpagem}

\begin{fulllineitems}
\phantomsection\label{\detokenize{namelists/jules_vegetation.nml:JULES_VEGETATION::l_trif_fire}}
\pysigstartsignatures
\pysigline{\sphinxcode{\sphinxupquote{JULES\_VEGETATION::}}\sphinxbfcode{\sphinxupquote{l\_trif\_fire}}}
\pysigstopsignatures\begin{quote}\begin{description}
\sphinxlineitem{Type}
\sphinxAtStartPar
logical

\sphinxlineitem{Default}
\sphinxAtStartPar
F

\end{description}\end{quote}

\sphinxAtStartPar
Switch that determines whether interactive fire is used. This allows for burnt area to link with dynamic
vegetation.

\sphinxAtStartPar
Only used if {\hyperref[\detokenize{namelists/jules_vegetation.nml:JULES_VEGETATION::l_triffid}]{\sphinxcrossref{\sphinxcode{\sphinxupquote{l\_triffid}}}}} = TRUE.
\begin{description}
\sphinxlineitem{TRUE}
\sphinxAtStartPar
Burnt area is calculated in INFERNO and passed to TRIFFID to
calculate vegetation dynamics. Carbon is also removed from DPM
and RPM pools in SOILCARB.

\sphinxlineitem{FALSE}
\sphinxAtStartPar
Burnt area is zero unless prescribed via an ancillary file.

\end{description}

\end{fulllineitems}

\index{l\_vegdrag\_pft (in namelist JULES\_VEGETATION)@\spxentry{l\_vegdrag\_pft}\spxextra{in namelist JULES\_VEGETATION}|spxpagem}

\begin{fulllineitems}
\phantomsection\label{\detokenize{namelists/jules_vegetation.nml:JULES_VEGETATION::l_vegdrag_pft}}
\pysigstartsignatures
\pysigline{\sphinxcode{\sphinxupquote{JULES\_VEGETATION::}}\sphinxbfcode{\sphinxupquote{l\_vegdrag\_pft}}}
\pysigstopsignatures\begin{quote}\begin{description}
\sphinxlineitem{Type}
\sphinxAtStartPar
logical(npft)

\sphinxlineitem{Default}
\sphinxAtStartPar
F

\end{description}\end{quote}

\sphinxAtStartPar
Switch for using vegetation canopy drag scheme on each PFT.
\begin{description}
\sphinxlineitem{TRUE}
\sphinxAtStartPar
Use a vegetative drag scheme. This is based on {\hyperref[\detokenize{namelists/jules_vegetation.nml:references-vegetation}]{\sphinxcrossref{\DUrole{std,std-ref}{Harman and Finnigan (2007)}}}}.

\sphinxlineitem{FALSE}
\sphinxAtStartPar
Do not use vegetative drag scheme.

\end{description}

\end{fulllineitems}

\index{l\_rsl\_scalar (in namelist JULES\_VEGETATION)@\spxentry{l\_rsl\_scalar}\spxextra{in namelist JULES\_VEGETATION}|spxpagem}

\begin{fulllineitems}
\phantomsection\label{\detokenize{namelists/jules_vegetation.nml:JULES_VEGETATION::l_rsl_scalar}}
\pysigstartsignatures
\pysigline{\sphinxcode{\sphinxupquote{JULES\_VEGETATION::}}\sphinxbfcode{\sphinxupquote{l\_rsl\_scalar}}}
\pysigstopsignatures\begin{quote}\begin{description}
\sphinxlineitem{Type}
\sphinxAtStartPar
logical

\sphinxlineitem{Default}
\sphinxAtStartPar
F

\end{description}\end{quote}

\sphinxAtStartPar
Switch for using a roughness sublayer correction scheme in scalar
variables. This is based on {\hyperref[\detokenize{namelists/jules_vegetation.nml:references-vegetation}]{\sphinxcrossref{\DUrole{std,std-ref}{Harman and Finnigan
(2008)}}}}.

\sphinxAtStartPar
Only use if any {\hyperref[\detokenize{namelists/jules_vegetation.nml:JULES_VEGETATION::l_vegdrag_pft}]{\sphinxcrossref{\sphinxcode{\sphinxupquote{l\_vegdrag\_pft}}}}} = TRUE.
\begin{description}
\sphinxlineitem{TRUE}
\sphinxAtStartPar
Use a roughness sublayer correction scheme in scalar variables.

\sphinxlineitem{FALSE}
\sphinxAtStartPar
Do not use a roughness sublayer correction scheme in scalar variables.

\end{description}

\end{fulllineitems}

\index{c1\_usuh (in namelist JULES\_VEGETATION)@\spxentry{c1\_usuh}\spxextra{in namelist JULES\_VEGETATION}|spxpagem}

\begin{fulllineitems}
\phantomsection\label{\detokenize{namelists/jules_vegetation.nml:JULES_VEGETATION::c1_usuh}}
\pysigstartsignatures
\pysigline{\sphinxcode{\sphinxupquote{JULES\_VEGETATION::}}\sphinxbfcode{\sphinxupquote{c1\_usuh}}}
\pysigstopsignatures\begin{quote}\begin{description}
\sphinxlineitem{Type}
\sphinxAtStartPar
real

\sphinxlineitem{Permitted}
\sphinxAtStartPar
\textgreater{}= 0

\sphinxlineitem{Default}
\sphinxAtStartPar
None

\end{description}\end{quote}

\sphinxAtStartPar
u*/U(h) at the top of dense canopy. See {\hyperref[\detokenize{namelists/jules_vegetation.nml:references-vegetation}]{\sphinxcrossref{\DUrole{std,std-ref}{Massman (1997)}}}}.

\sphinxAtStartPar
Only use if any {\hyperref[\detokenize{namelists/jules_vegetation.nml:JULES_VEGETATION::l_vegdrag_pft}]{\sphinxcrossref{\sphinxcode{\sphinxupquote{l\_vegdrag\_pft}}}}} = TRUE.

\end{fulllineitems}

\index{c2\_usuh (in namelist JULES\_VEGETATION)@\spxentry{c2\_usuh}\spxextra{in namelist JULES\_VEGETATION}|spxpagem}

\begin{fulllineitems}
\phantomsection\label{\detokenize{namelists/jules_vegetation.nml:JULES_VEGETATION::c2_usuh}}
\pysigstartsignatures
\pysigline{\sphinxcode{\sphinxupquote{JULES\_VEGETATION::}}\sphinxbfcode{\sphinxupquote{c2\_usuh}}}
\pysigstopsignatures\begin{quote}\begin{description}
\sphinxlineitem{Type}
\sphinxAtStartPar
real

\sphinxlineitem{Permitted}
\sphinxAtStartPar
\textgreater{}= 0

\sphinxlineitem{Default}
\sphinxAtStartPar
None

\end{description}\end{quote}

\sphinxAtStartPar
u*/U(h) at substrate under canopy. See {\hyperref[\detokenize{namelists/jules_vegetation.nml:references-vegetation}]{\sphinxcrossref{\DUrole{std,std-ref}{Massman (1997)}}}}.

\sphinxAtStartPar
Only use if any {\hyperref[\detokenize{namelists/jules_vegetation.nml:JULES_VEGETATION::l_vegdrag_pft}]{\sphinxcrossref{\sphinxcode{\sphinxupquote{l\_vegdrag\_pft}}}}} = TRUE.

\end{fulllineitems}

\index{c3\_usuh (in namelist JULES\_VEGETATION)@\spxentry{c3\_usuh}\spxextra{in namelist JULES\_VEGETATION}|spxpagem}

\begin{fulllineitems}
\phantomsection\label{\detokenize{namelists/jules_vegetation.nml:JULES_VEGETATION::c3_usuh}}
\pysigstartsignatures
\pysigline{\sphinxcode{\sphinxupquote{JULES\_VEGETATION::}}\sphinxbfcode{\sphinxupquote{c3\_usuh}}}
\pysigstopsignatures\begin{quote}\begin{description}
\sphinxlineitem{Type}
\sphinxAtStartPar
real

\sphinxlineitem{Permitted}
\sphinxAtStartPar
\textgreater{}= 0

\sphinxlineitem{Default}
\sphinxAtStartPar
None

\end{description}\end{quote}

\sphinxAtStartPar
This is used in the exponent of equation weighting dense and sparse
vegetation to get u*/U(h) in neutral condition. See {\hyperref[\detokenize{namelists/jules_vegetation.nml:references-vegetation}]{\sphinxcrossref{\DUrole{std,std-ref}{Massman
(1997)}}}}. The default value is taken from {\hyperref[\detokenize{namelists/jules_vegetation.nml:references-vegetation}]{\sphinxcrossref{\DUrole{std,std-ref}{Wang
(2012)}}}}.

\sphinxAtStartPar
Only use if any {\hyperref[\detokenize{namelists/jules_vegetation.nml:JULES_VEGETATION::l_vegdrag_pft}]{\sphinxcrossref{\sphinxcode{\sphinxupquote{l\_vegdrag\_pft}}}}} = TRUE.

\end{fulllineitems}

\index{cd\_leaf (in namelist JULES\_VEGETATION)@\spxentry{cd\_leaf}\spxextra{in namelist JULES\_VEGETATION}|spxpagem}

\begin{fulllineitems}
\phantomsection\label{\detokenize{namelists/jules_vegetation.nml:JULES_VEGETATION::cd_leaf}}
\pysigstartsignatures
\pysigline{\sphinxcode{\sphinxupquote{JULES\_VEGETATION::}}\sphinxbfcode{\sphinxupquote{cd\_leaf}}}
\pysigstopsignatures\begin{quote}\begin{description}
\sphinxlineitem{Type}
\sphinxAtStartPar
real

\sphinxlineitem{Permitted}
\sphinxAtStartPar
0:1

\sphinxlineitem{Default}
\sphinxAtStartPar
None

\end{description}\end{quote}

\sphinxAtStartPar
Leaf level drag coefficient.

\sphinxAtStartPar
Only use if any {\hyperref[\detokenize{namelists/jules_vegetation.nml:JULES_VEGETATION::l_vegdrag_pft}]{\sphinxcrossref{\sphinxcode{\sphinxupquote{l\_vegdrag\_pft}}}}} = TRUE.

\end{fulllineitems}

\index{stanton\_leaf (in namelist JULES\_VEGETATION)@\spxentry{stanton\_leaf}\spxextra{in namelist JULES\_VEGETATION}|spxpagem}

\begin{fulllineitems}
\phantomsection\label{\detokenize{namelists/jules_vegetation.nml:JULES_VEGETATION::stanton_leaf}}
\pysigstartsignatures
\pysigline{\sphinxcode{\sphinxupquote{JULES\_VEGETATION::}}\sphinxbfcode{\sphinxupquote{stanton\_leaf}}}
\pysigstopsignatures\begin{quote}\begin{description}
\sphinxlineitem{Type}
\sphinxAtStartPar
real

\sphinxlineitem{Permitted}
\sphinxAtStartPar
0:1

\sphinxlineitem{Default}
\sphinxAtStartPar
None

\end{description}\end{quote}

\sphinxAtStartPar
Leaf\sphinxhyphen{}level Stanton number

\sphinxAtStartPar
Only use if {\hyperref[\detokenize{namelists/jules_vegetation.nml:JULES_VEGETATION::l_rsl_scalar}]{\sphinxcrossref{\sphinxcode{\sphinxupquote{l\_rsl\_scalar}}}}} = TRUE.

\end{fulllineitems}

\index{l\_spec\_veg\_z0 (in namelist JULES\_VEGETATION)@\spxentry{l\_spec\_veg\_z0}\spxextra{in namelist JULES\_VEGETATION}|spxpagem}

\begin{fulllineitems}
\phantomsection\label{\detokenize{namelists/jules_vegetation.nml:JULES_VEGETATION::l_spec_veg_z0}}
\pysigstartsignatures
\pysigline{\sphinxcode{\sphinxupquote{JULES\_VEGETATION::}}\sphinxbfcode{\sphinxupquote{l\_spec\_veg\_z0}}}
\pysigstopsignatures\begin{quote}\begin{description}
\sphinxlineitem{Type}
\sphinxAtStartPar
logical

\sphinxlineitem{Default}
\sphinxAtStartPar
F

\end{description}\end{quote}

\sphinxAtStartPar
Switch for using specified values of the vegetation roughness
length rather than being determined by the canopy height.
\begin{description}
\sphinxlineitem{TRUE}
\sphinxAtStartPar
Vegetation roughness lengths are specified for each PFT in
{\hyperref[\detokenize{namelists/pft_params.nml:JULES_PFTPARM::z0v_io}]{\sphinxcrossref{\sphinxcode{\sphinxupquote{z0v\_io}}}}}.

\sphinxlineitem{FALSE}
\sphinxAtStartPar
Vegetation roughness lengths are calculated using canopy
heights and parameter {\hyperref[\detokenize{namelists/pft_params.nml:JULES_PFTPARM::dz0v_dh_io}]{\sphinxcrossref{\sphinxcode{\sphinxupquote{dz0v\_dh\_io}}}}}.

\end{description}

\end{fulllineitems}

\index{l\_limit\_canhc (in namelist JULES\_VEGETATION)@\spxentry{l\_limit\_canhc}\spxextra{in namelist JULES\_VEGETATION}|spxpagem}

\begin{fulllineitems}
\phantomsection\label{\detokenize{namelists/jules_vegetation.nml:JULES_VEGETATION::l_limit_canhc}}
\pysigstartsignatures
\pysigline{\sphinxcode{\sphinxupquote{JULES\_VEGETATION::}}\sphinxbfcode{\sphinxupquote{l\_limit\_canhc}}}
\pysigstopsignatures\begin{quote}\begin{description}
\sphinxlineitem{Type}
\sphinxAtStartPar
logical

\sphinxlineitem{Default}
\sphinxAtStartPar
F

\end{description}\end{quote}

\sphinxAtStartPar
Switch for limiting the canopy heat capacity for vegetation, which
is calculated from the canopy height.

\sphinxAtStartPar
Using the SIMARD canopy height ancillary gives very large heat
capacities in the Amazon, so this switch limits the areal heat
capacity to 1.15e5 J kg$^{\text{\sphinxhyphen{}1}}$ m$^{\text{\sphinxhyphen{}2}}$, which is the value
calculated by the default broadleaf tree height of 19.01 m.
\begin{description}
\sphinxlineitem{TRUE}
\sphinxAtStartPar
Vegetation areal heat capacity limited.

\sphinxlineitem{FALSE}
\sphinxAtStartPar
Vegetation areal heat capacity unlimited.

\end{description}

\end{fulllineitems}


\begin{sphinxadmonition}{note}{Only used with the Farquhar model of leaf photosynthesis (\sphinxstyleliteralintitle{\sphinxupquote{photo\_model}} = 2).}
\index{photo\_acclim\_model (in namelist JULES\_VEGETATION)@\spxentry{photo\_acclim\_model}\spxextra{in namelist JULES\_VEGETATION}|spxpagem}

\begin{fulllineitems}
\phantomsection\label{\detokenize{namelists/jules_vegetation.nml:JULES_VEGETATION::photo_acclim_model}}
\pysigstartsignatures
\pysigline{\sphinxcode{\sphinxupquote{JULES\_VEGETATION::}}\sphinxbfcode{\sphinxupquote{photo\_acclim\_model}}}
\pysigstopsignatures\begin{quote}\begin{description}
\sphinxlineitem{Type}
\sphinxAtStartPar
integer

\sphinxlineitem{Permitted}
\sphinxAtStartPar
0, 1, 2, or 3

\sphinxlineitem{Default}
\sphinxAtStartPar
None

\end{description}\end{quote}

\sphinxAtStartPar
Choice for model of thermal response of photosynthetic capacity.
Possible values are:
\begin{enumerate}
\sphinxsetlistlabels{\arabic}{enumi}{enumii}{}{.}%
\setcounter{enumi}{-1}
\item {} 
\begin{DUlineblock}{0em}
\item[] No adaptation or acclimation.
\end{DUlineblock}

\item {} 
\begin{DUlineblock}{0em}
\item[] Thermal adaptation \sphinxhyphen{} plant response to temperature varies geographically in response to a static “home” temperature.
\end{DUlineblock}

\item {} 
\begin{DUlineblock}{0em}
\item[] Thermal acclimation \sphinxhyphen{} plant response to temperature varies geographically and temporally in response to a dynamic “growth” temperature.
\end{DUlineblock}

\item {} 
\begin{DUlineblock}{0em}
\item[] Thermal adaptation and acclimation \sphinxhyphen{} plant response to temperature varies geographically and temporally in response to a static “home” temperature and a dynamic “growth” temperature.
\end{DUlineblock}

\end{enumerate}

\begin{sphinxadmonition}{note}{Note:}
\sphinxAtStartPar
When {\hyperref[\detokenize{namelists/jules_vegetation.nml:JULES_VEGETATION::photo_acclim_model}]{\sphinxcrossref{\sphinxcode{\sphinxupquote{photo\_acclim\_model}}}}} = 1 or 3 is used, the user must supply the long\sphinxhyphen{}term home temperature as ancillary field \sphinxcode{\sphinxupquote{t\_home\_gb}} in {\hyperref[\detokenize{namelists/ancillaries.nml:namelist-JULES_VEGETATION_PROPS}]{\sphinxcrossref{\sphinxcode{\sphinxupquote{JULES\_VEGETATION\_PROPS}}}}}.  When {\hyperref[\detokenize{namelists/jules_vegetation.nml:JULES_VEGETATION::photo_acclim_model}]{\sphinxcrossref{\sphinxcode{\sphinxupquote{photo\_acclim\_model}}}}} = 2 or 3 is used, the user must supply the running mean growth temperature as initial condition \sphinxcode{\sphinxupquote{t\_growth\_gb}} in {\hyperref[\detokenize{namelists/initial_conditions.nml:namelist-JULES_INITIAL}]{\sphinxcrossref{\sphinxcode{\sphinxupquote{JULES\_INITIAL}}}}}.
\end{sphinxadmonition}

\end{fulllineitems}

\index{photo\_act\_model (in namelist JULES\_VEGETATION)@\spxentry{photo\_act\_model}\spxextra{in namelist JULES\_VEGETATION}|spxpagem}

\begin{fulllineitems}
\phantomsection\label{\detokenize{namelists/jules_vegetation.nml:JULES_VEGETATION::photo_act_model}}
\pysigstartsignatures
\pysigline{\sphinxcode{\sphinxupquote{JULES\_VEGETATION::}}\sphinxbfcode{\sphinxupquote{photo\_act\_model}}}
\pysigstopsignatures\begin{quote}\begin{description}
\sphinxlineitem{Type}
\sphinxAtStartPar
integer

\sphinxlineitem{Permitted}
\sphinxAtStartPar
1 or 2

\sphinxlineitem{Default}
\sphinxAtStartPar
None

\end{description}\end{quote}

\sphinxAtStartPar
Choice of model for the activation energies of J$_{\text{max}}$ and V$_{\text{cmax}}$.
\begin{enumerate}
\sphinxsetlistlabels{\arabic}{enumi}{enumii}{}{.}%
\item {} 
\begin{DUlineblock}{0em}
\item[] Activation energies vary by PFT but not by land point, and are NOT subject to acclimation.
\end{DUlineblock}

\item {} 
\begin{DUlineblock}{0em}
\item[] Activation energies vary by land point but not by PFT, and are subject to acclimation.
\end{DUlineblock}

\end{enumerate}

\begin{sphinxadmonition}{note}{Note:}
\sphinxAtStartPar
When {\hyperref[\detokenize{namelists/jules_vegetation.nml:JULES_VEGETATION::photo_act_model}]{\sphinxcrossref{\sphinxcode{\sphinxupquote{photo\_act\_model}}}}} = 1 is used, activation energies are calculated using {\hyperref[\detokenize{namelists/pft_params.nml:JULES_PFTPARM::act_jmax_io}]{\sphinxcrossref{\sphinxcode{\sphinxupquote{act\_jmax\_io}}}}} and {\hyperref[\detokenize{namelists/pft_params.nml:JULES_PFTPARM::act_vcmax_io}]{\sphinxcrossref{\sphinxcode{\sphinxupquote{act\_vcmax\_io}}}}}.  When {\hyperref[\detokenize{namelists/jules_vegetation.nml:JULES_VEGETATION::photo_act_model}]{\sphinxcrossref{\sphinxcode{\sphinxupquote{photo\_act\_model}}}}} = 2 is used, activation energies are calculated using {\hyperref[\detokenize{namelists/jules_vegetation.nml:JULES_VEGETATION::act_j_coef}]{\sphinxcrossref{\sphinxcode{\sphinxupquote{act\_j\_coef}}}}} and {\hyperref[\detokenize{namelists/jules_vegetation.nml:JULES_VEGETATION::act_v_coef}]{\sphinxcrossref{\sphinxcode{\sphinxupquote{act\_v\_coef}}}}}.
\end{sphinxadmonition}

\begin{sphinxadmonition}{warning}{Warning:}
\sphinxAtStartPar
A value of 1 (PFT\sphinxhyphen{}dependent) must be used if {\hyperref[\detokenize{namelists/jules_vegetation.nml:JULES_VEGETATION::photo_acclim_model}]{\sphinxcrossref{\sphinxcode{\sphinxupquote{photo\_acclim\_model}}}}} = 0 (no adaptation or acclimation).
\end{sphinxadmonition}

\end{fulllineitems}

\index{photo\_jv\_model (in namelist JULES\_VEGETATION)@\spxentry{photo\_jv\_model}\spxextra{in namelist JULES\_VEGETATION}|spxpagem}

\begin{fulllineitems}
\phantomsection\label{\detokenize{namelists/jules_vegetation.nml:JULES_VEGETATION::photo_jv_model}}
\pysigstartsignatures
\pysigline{\sphinxcode{\sphinxupquote{JULES\_VEGETATION::}}\sphinxbfcode{\sphinxupquote{photo\_jv\_model}}}
\pysigstopsignatures\begin{quote}\begin{description}
\sphinxlineitem{Type}
\sphinxAtStartPar
integer

\sphinxlineitem{Permitted}
\sphinxAtStartPar
1 or 2

\sphinxlineitem{Default}
\sphinxAtStartPar
None

\end{description}\end{quote}

\sphinxAtStartPar
Choice for model of for the variation of J$_{\text{25}}$/V$_{\text{25}}$.
\begin{enumerate}
\sphinxsetlistlabels{\arabic}{enumi}{enumii}{}{.}%
\item {} 
\begin{DUlineblock}{0em}
\item[] J$_{\text{25}}$ is found by scaling V$_{\text{25}}$ by the given ratio J$_{\text{25}}$/V$_{\text{25}}$, that is, all the variation in the ratio comes from varying J$_{\text{25}}$ (while V$_{\text{25}}$ remains fixed).
\end{DUlineblock}

\item {} 
\begin{DUlineblock}{0em}
\item[] J25 and V25 are calculated assuming that the total amount of nitrogen allocated to photosynthesis remains constant, thus any change in J25 requires a compensatory change in V25 \sphinxhyphen{} as used in {\hyperref[\detokenize{namelists/jules_vegetation.nml:references-vegetation}]{\sphinxcrossref{\DUrole{std,std-ref}{Mercado et al. (2018)}}}}.
\end{DUlineblock}

\end{enumerate}

\begin{sphinxadmonition}{warning}{Warning:}
\sphinxAtStartPar
A value of 1 (simple scaling) must be used if {\hyperref[\detokenize{namelists/jules_vegetation.nml:JULES_VEGETATION::photo_acclim_model}]{\sphinxcrossref{\sphinxcode{\sphinxupquote{photo\_acclim\_model}}}}} = 0 (no adaptation or acclimation).
\end{sphinxadmonition}

\end{fulllineitems}

\end{sphinxadmonition}

\begin{sphinxadmonition}{note}{Only used with \sphinxstyleliteralintitle{\sphinxupquote{photo\_jv\_model}} = 2.}
\index{n\_alloc\_jmax (in namelist JULES\_VEGETATION)@\spxentry{n\_alloc\_jmax}\spxextra{in namelist JULES\_VEGETATION}|spxpagem}

\begin{fulllineitems}
\phantomsection\label{\detokenize{namelists/jules_vegetation.nml:JULES_VEGETATION::n_alloc_jmax}}
\pysigstartsignatures
\pysigline{\sphinxcode{\sphinxupquote{JULES\_VEGETATION::}}\sphinxbfcode{\sphinxupquote{n\_alloc\_jmax}}}
\pysigstopsignatures\begin{quote}\begin{description}
\sphinxlineitem{Type}
\sphinxAtStartPar
real

\sphinxlineitem{Default}
\sphinxAtStartPar
None

\end{description}\end{quote}

\sphinxAtStartPar
Constant relating nitrogen allocation to J$_{\text{max}}$ (mol CO$^{\text{2}}$ m$^{\text{\sphinxhyphen{}2}}$ s$^{\text{\sphinxhyphen{}1}}$ {[}kg m$^{\text{\sphinxhyphen{}2}}${]}$^{\text{\sphinxhyphen{}1}}$). This is 5.3 in Eq.5 of {\hyperref[\detokenize{namelists/jules_vegetation.nml:references-vegetation}]{\sphinxcrossref{\DUrole{std,std-ref}{Mercado et al. (2018)}}}}.

\end{fulllineitems}

\index{n\_alloc\_vcmax (in namelist JULES\_VEGETATION)@\spxentry{n\_alloc\_vcmax}\spxextra{in namelist JULES\_VEGETATION}|spxpagem}

\begin{fulllineitems}
\phantomsection\label{\detokenize{namelists/jules_vegetation.nml:JULES_VEGETATION::n_alloc_vcmax}}
\pysigstartsignatures
\pysigline{\sphinxcode{\sphinxupquote{JULES\_VEGETATION::}}\sphinxbfcode{\sphinxupquote{n\_alloc\_vcmax}}}
\pysigstopsignatures\begin{quote}\begin{description}
\sphinxlineitem{Type}
\sphinxAtStartPar
real

\sphinxlineitem{Default}
\sphinxAtStartPar
None

\end{description}\end{quote}

\sphinxAtStartPar
Constant relating nitrogen allocation to V$_{\text{cmax}}$ (mol CO$^{\text{2}}$ m$^{\text{\sphinxhyphen{}2}}$ s$^{\text{\sphinxhyphen{}1}}$ {[}kg m$^{\text{\sphinxhyphen{}2}}${]}$^{\text{\sphinxhyphen{}1}}$). This is 3.8 in Eq.5 of {\hyperref[\detokenize{namelists/jules_vegetation.nml:references-vegetation}]{\sphinxcrossref{\DUrole{std,std-ref}{Mercado et al. (2018)}}}}.

\end{fulllineitems}

\end{sphinxadmonition}

\begin{sphinxadmonition}{note}{Only used with thermal adaptation or acclimation of photosynthesis (\sphinxstyleliteralintitle{\sphinxupquote{photo\_acclim\_model}} = 1, 2 or 3).}

\sphinxAtStartPar
The thermal adaptation/acclimation scheme in JULES is structured following Eq. 13 of {\hyperref[\detokenize{namelists/jules_vegetation.nml:references-vegetation}]{\sphinxcrossref{\DUrole{std,std-ref}{Kumarathunge et al. (2019)}}}}, in which C3 photosynthetic capacity is allowed to vary at each land point as a function of a static home temperature (T$_{\text{h}}$) and a dynamic growth temperature (T$_{\text{g}}$).  This is achieved by calculating five parameters used in the Farquhar photosynthesis scheme as functions of those temperature fields, rather than using fixed parameters from {\hyperref[\detokenize{namelists/pft_params.nml:namelist-JULES_PFTPARM}]{\sphinxcrossref{\sphinxcode{\sphinxupquote{JULES\_PFTPARM}}}}}.  Each parameter, Q, is calculated as a linear function of T$_{\text{h}}$ and T$_{\text{g}}$:

\sphinxAtStartPar
Q(T$_{\text{h}}$, T$_{\text{g}}$) = Q$_{\text{coef}}$(0) + Q$_{\text{coef}}$(1) T$_{\text{h}}$ + Q$_{\text{coef}}$(2) T$_{\text{g}}$.

\sphinxAtStartPar
The following namelist members specify the coefficients, Q$_{\text{coef}}$, used for each parameter.  Note that, in each case, the units for Q$_{\text{coef}}$(1) and Q$_{\text{coef}}$(2) have an extra factor K$^{\text{\sphinxhyphen{}1}}$ relative to the units for Q$_{\text{coef}}$(0).  This structure can be configured to represent the acclimation scheme of {\hyperref[\detokenize{namelists/jules_vegetation.nml:references-vegetation}]{\sphinxcrossref{\DUrole{std,std-ref}{Kattge and Knorr (2007)}}}}, as used by {\hyperref[\detokenize{namelists/jules_vegetation.nml:references-vegetation}]{\sphinxcrossref{\DUrole{std,std-ref}{Mercado et al. (2018)}}}}, and the scheme of {\hyperref[\detokenize{namelists/jules_vegetation.nml:references-vegetation}]{\sphinxcrossref{\DUrole{std,std-ref}{Kumarathunge et al. (2019)}}}}.

\begin{sphinxadmonition}{note}{Note:}
\sphinxAtStartPar
If {\hyperref[\detokenize{namelists/jules_vegetation.nml:JULES_VEGETATION::photo_acclim_model}]{\sphinxcrossref{\sphinxcode{\sphinxupquote{photo\_acclim\_model}}}}} = 1 is used all Q$_{\text{coef}}$(2) must equal 0.0, and if {\hyperref[\detokenize{namelists/jules_vegetation.nml:JULES_VEGETATION::photo_acclim_model}]{\sphinxcrossref{\sphinxcode{\sphinxupquote{photo\_acclim\_model}}}}} = 2 is used all Q$_{\text{coef}}$(1) must equal 0.0.
\end{sphinxadmonition}
\index{act\_j\_coef (in namelist JULES\_VEGETATION)@\spxentry{act\_j\_coef}\spxextra{in namelist JULES\_VEGETATION}|spxpagem}

\begin{fulllineitems}
\phantomsection\label{\detokenize{namelists/jules_vegetation.nml:JULES_VEGETATION::act_j_coef}}
\pysigstartsignatures
\pysigline{\sphinxcode{\sphinxupquote{JULES\_VEGETATION::}}\sphinxbfcode{\sphinxupquote{act\_j\_coef}}}
\pysigstopsignatures\begin{quote}\begin{description}
\sphinxlineitem{Type}
\sphinxAtStartPar
real(3)

\sphinxlineitem{Default}
\sphinxAtStartPar
None

\end{description}\end{quote}

\sphinxAtStartPar
Coefficients for the activation energy for J$_{\text{max}}$ (J mol$^{\text{\sphinxhyphen{}1}}$ and J mol$^{\text{\sphinxhyphen{}1}}$ K$^{\text{\sphinxhyphen{}1}}$).  Replaces the use of {\hyperref[\detokenize{namelists/pft_params.nml:JULES_PFTPARM::act_jmax_io}]{\sphinxcrossref{\sphinxcode{\sphinxupquote{act\_jmax\_io}}}}}.

\end{fulllineitems}

\index{act\_v\_coef (in namelist JULES\_VEGETATION)@\spxentry{act\_v\_coef}\spxextra{in namelist JULES\_VEGETATION}|spxpagem}

\begin{fulllineitems}
\phantomsection\label{\detokenize{namelists/jules_vegetation.nml:JULES_VEGETATION::act_v_coef}}
\pysigstartsignatures
\pysigline{\sphinxcode{\sphinxupquote{JULES\_VEGETATION::}}\sphinxbfcode{\sphinxupquote{act\_v\_coef}}}
\pysigstopsignatures\begin{quote}\begin{description}
\sphinxlineitem{Type}
\sphinxAtStartPar
real(3)

\sphinxlineitem{Default}
\sphinxAtStartPar
None

\end{description}\end{quote}

\sphinxAtStartPar
Coefficients for the activation energy for V$_{\text{cmax}}$ (J mol$^{\text{\sphinxhyphen{}1}}$ and J mol$^{\text{\sphinxhyphen{}1}}$ K$^{\text{\sphinxhyphen{}1}}$).  Replaces the use of {\hyperref[\detokenize{namelists/pft_params.nml:JULES_PFTPARM::act_vcmax_io}]{\sphinxcrossref{\sphinxcode{\sphinxupquote{act\_vcmax\_io}}}}}.

\end{fulllineitems}

\index{dsj\_coef (in namelist JULES\_VEGETATION)@\spxentry{dsj\_coef}\spxextra{in namelist JULES\_VEGETATION}|spxpagem}

\begin{fulllineitems}
\phantomsection\label{\detokenize{namelists/jules_vegetation.nml:JULES_VEGETATION::dsj_coef}}
\pysigstartsignatures
\pysigline{\sphinxcode{\sphinxupquote{JULES\_VEGETATION::}}\sphinxbfcode{\sphinxupquote{dsj\_coef}}}
\pysigstopsignatures\begin{quote}\begin{description}
\sphinxlineitem{Type}
\sphinxAtStartPar
real(3)

\sphinxlineitem{Default}
\sphinxAtStartPar
None

\end{description}\end{quote}

\sphinxAtStartPar
Coefficients for entropy factor for J$_{\text{max}}$ (J mol$^{\text{\sphinxhyphen{}1}}$ K$^{\text{\sphinxhyphen{}1}}$ and J mol$^{\text{\sphinxhyphen{}1}}$ K$^{\text{\sphinxhyphen{}2}}$).  Replaces the use of {\hyperref[\detokenize{namelists/pft_params.nml:JULES_PFTPARM::deact_jmax_io}]{\sphinxcrossref{\sphinxcode{\sphinxupquote{deact\_jmax\_io}}}}}.

\end{fulllineitems}

\index{dsv\_coef (in namelist JULES\_VEGETATION)@\spxentry{dsv\_coef}\spxextra{in namelist JULES\_VEGETATION}|spxpagem}

\begin{fulllineitems}
\phantomsection\label{\detokenize{namelists/jules_vegetation.nml:JULES_VEGETATION::dsv_coef}}
\pysigstartsignatures
\pysigline{\sphinxcode{\sphinxupquote{JULES\_VEGETATION::}}\sphinxbfcode{\sphinxupquote{dsv\_coef}}}
\pysigstopsignatures\begin{quote}\begin{description}
\sphinxlineitem{Type}
\sphinxAtStartPar
real(3)

\sphinxlineitem{Default}
\sphinxAtStartPar
None

\end{description}\end{quote}

\sphinxAtStartPar
Coefficients for the entropy factor for V$_{\text{cmax}}$ (J mol$^{\text{\sphinxhyphen{}1}}$ K$^{\text{\sphinxhyphen{}1}}$ and J mol$^{\text{\sphinxhyphen{}1}}$ K$^{\text{\sphinxhyphen{}2}}$).  Replaces the use of {\hyperref[\detokenize{namelists/pft_params.nml:JULES_PFTPARM::deact_vcmax_io}]{\sphinxcrossref{\sphinxcode{\sphinxupquote{deact\_vcmax\_io}}}}}.

\end{fulllineitems}

\index{jv25\_coef (in namelist JULES\_VEGETATION)@\spxentry{jv25\_coef}\spxextra{in namelist JULES\_VEGETATION}|spxpagem}

\begin{fulllineitems}
\phantomsection\label{\detokenize{namelists/jules_vegetation.nml:JULES_VEGETATION::jv25_coef}}
\pysigstartsignatures
\pysigline{\sphinxcode{\sphinxupquote{JULES\_VEGETATION::}}\sphinxbfcode{\sphinxupquote{jv25\_coef}}}
\pysigstopsignatures\begin{quote}\begin{description}
\sphinxlineitem{Type}
\sphinxAtStartPar
real(3)

\sphinxlineitem{Default}
\sphinxAtStartPar
None

\end{description}\end{quote}

\sphinxAtStartPar
Coefficients for the ratio J$_{\text{25}}$/V$_{\text{25}}$ (mol electrons {[}mol$^{\text{\sphinxhyphen{}1}}$ CO$_{\text{2}}${]} and (mol electrons {[}mol$^{\text{\sphinxhyphen{}1}}$ CO$_{\text{2}}${]} K$^{\text{\sphinxhyphen{}1}}$).  Replaces the use of {\hyperref[\detokenize{namelists/pft_params.nml:JULES_PFTPARM::jv25_ratio_io}]{\sphinxcrossref{\sphinxcode{\sphinxupquote{jv25\_ratio\_io}}}}}.

\end{fulllineitems}

\end{sphinxadmonition}

\begin{sphinxadmonition}{note}{Only used with thermal acclimation of photosynthesis (\sphinxstyleliteralintitle{\sphinxupquote{photo\_acclim\_model}} = 2 or 3).}
\index{n\_day\_photo\_acclim (in namelist JULES\_VEGETATION)@\spxentry{n\_day\_photo\_acclim}\spxextra{in namelist JULES\_VEGETATION}|spxpagem}

\begin{fulllineitems}
\phantomsection\label{\detokenize{namelists/jules_vegetation.nml:JULES_VEGETATION::n_day_photo_acclim}}
\pysigstartsignatures
\pysigline{\sphinxcode{\sphinxupquote{JULES\_VEGETATION::}}\sphinxbfcode{\sphinxupquote{n\_day\_photo\_acclim}}}
\pysigstopsignatures\begin{quote}\begin{description}
\sphinxlineitem{Type}
\sphinxAtStartPar
real

\sphinxlineitem{Default}
\sphinxAtStartPar
None

\end{description}\end{quote}

\sphinxAtStartPar
Time constant (days) for the exponential moving average of temperature that is used as the growth temperature. Given a step function as input, the smoothed output has fallen to 1/e (approx. 37\%) of the initial value after this number of days.

\end{fulllineitems}

\end{sphinxadmonition}
\index{l\_croprotate (in namelist JULES\_VEGETATION)@\spxentry{l\_croprotate}\spxextra{in namelist JULES\_VEGETATION}|spxpagem}

\begin{fulllineitems}
\phantomsection\label{\detokenize{namelists/jules_vegetation.nml:JULES_VEGETATION::l_croprotate}}
\pysigstartsignatures
\pysigline{\sphinxcode{\sphinxupquote{JULES\_VEGETATION::}}\sphinxbfcode{\sphinxupquote{l\_croprotate}}}
\pysigstopsignatures\begin{quote}\begin{description}
\sphinxlineitem{Type}
\sphinxAtStartPar
logical

\sphinxlineitem{Default}
\sphinxAtStartPar
F

\end{description}\end{quote}

\sphinxAtStartPar
Switch that enables sequential cropping (crop rotations).
Only used
if {\hyperref[\detokenize{namelists/jules_surface_types.nml:JULES_SURFACE_TYPES::ncpft}]{\sphinxcrossref{\sphinxcode{\sphinxupquote{ncpft}}}}} \textgreater{} 0 and
if {\hyperref[\detokenize{namelists/jules_vegetation.nml:JULES_VEGETATION::l_prescsow}]{\sphinxcrossref{\sphinxcode{\sphinxupquote{l\_prescsow}}}}} = T.
\begin{description}
\sphinxlineitem{TRUE}
\sphinxAtStartPar
Sowing dates and latest harvest dates prescribed in
{\hyperref[\detokenize{namelists/ancillaries.nml:namelist-JULES_CROP_PROPS}]{\sphinxcrossref{\sphinxcode{\sphinxupquote{JULES\_CROP\_PROPS}}}}} are used. The method is implemented in
{\hyperref[\detokenize{namelists/jules_vegetation.nml:references-vegetation}]{\sphinxcrossref{\DUrole{std,std-ref}{Mathison et al. (2019)}}}}.

\sphinxlineitem{FALSE}
\sphinxAtStartPar
The crop model is used in its standard form with a single crop per year

\end{description}

\end{fulllineitems}



\subsection{\sphinxstyleliteralintitle{\sphinxupquote{JULES\_VEGETATION}} references}
\label{\detokenize{namelists/jules_vegetation.nml:jules-vegetation-references}}\label{\detokenize{namelists/jules_vegetation.nml:references-vegetation}}\begin{itemize}
\item {} 
\sphinxAtStartPar
Best et al., 2011, The Joint UK Land Environment Simulator (JULES),
model description \textendash{} Part 1: Energy and water fluxes, Geosci. Model
Dev., \sphinxurl{https://doi.org/10.5194/gmd-4-677-2011}.

\item {} 
\sphinxAtStartPar
Clark et al., 2011, The Joint UK Land Environment Simulator (JULES)
model description \textendash{} Part 2: Carbon fluxes and vegetation dynamics,
Geosci. Model Dev., 4, 701\sphinxhyphen{}722,
\sphinxurl{https://doi.org/10.5194/gmd-4-701-2011}

\item {} 
\sphinxAtStartPar
Harman, I.N. \& Finnigan, J.J. (2007), A simple unified theory for
flow in the canopy and roughness sublayer. Boundary\sphinxhyphen{}Layer Meteorol.
123: 339. \sphinxurl{https://doi.org/10.1007/s10546-006-9145-6}

\item {} 
\sphinxAtStartPar
Harman, I.N. \& Finnigan, J.J. (2008), Scalar Concentration Profiles in the
Canopy and Roughness Sublayer. Boundary\sphinxhyphen{}Layer Meteorol.
129: 323. \sphinxurl{https://doi.org/10.1007/s10546-008-9328-4}

\item {} 
\sphinxAtStartPar
HCTN24, Hadley Centre Technical Note 24, available from \sphinxhref{http://www.metoffice.gov.uk/learning/library/publications/science/climate-science-technical-notes}{the Met Office Library}.
For ease the direct link to this document is:
\sphinxhref{https://digital.nmla.metoffice.gov.uk/IO\_cc8f146a-d524-4243-88fc-e3a3bcd782e7}{HCTN24 “Description of the “TRIFFID” Dynamic Global Vegetation Model”}.

\item {} 
\sphinxAtStartPar
Jogireddy, V., Cox, P. M., Huntingford, C., Harding, R. J., and
Mercado, L. M.:  An improved description of canopy light
interception for use in a GCM land\sphinxhyphen{}surface scheme: calibration and
testing against carbon fluxes at a coniferous forest, Hadley Centre
Technical Note 63, Hadley Centre, Met Office, Exeter,
UK, 2006. \sphinxurl{https://digital.nmla.metoffice.gov.uk/IO\_7873ea05-61ec-4615-b030-6bc33397d675}

\item {} 
\sphinxAtStartPar
Kattge, J. and Knorr, W., 2007,
Temperature acclimation in a biochemical model of photosynthesis:
a reanalysis of data from 36 species,
Plant, Cell and Environment, 30: 1176\textendash{}1190,
\sphinxurl{https://doi.org/10.1111/j.1365-3040.2007.01690.x}.

\item {} 
\sphinxAtStartPar
Kattge, J. , Knorr, W. , Raddatz, T. and Wirth, C. (2009),
Quantifying photosynthetic capacity and its relationship to leaf
nitrogen content for global\sphinxhyphen{}scale terrestrial biosphere
models. Global Change Biology, 15:
976\sphinxhyphen{}991. \sphinxurl{https://doi.org/doi:10.1111/j.1365-2486.2008.01744.x}

\item {} 
\sphinxAtStartPar
Kumarathunge, D. P. et al (2019), Acclimation and adaptation components of
the temperature dependence of plant photosynthesis at the global scale, New
Phytologist, 222: 768\sphinxhyphen{}784, \sphinxurl{https://doi.org/10.1111/nph.15668}

\item {} 
\sphinxAtStartPar
Massman, W. J. (1997), An Analytical One\sphinxhyphen{}Dimensional Model of
Momentum Transfer by Vegetation of Arbitrary Structure,
Boundary\sphinxhyphen{}Layer Meteorol. 83: 407\sphinxhyphen{}421.

\item {} 
\sphinxAtStartPar
Medlyn, B. E., Duursma, R. A., Eamus, D. , Ellsworth, D. S.,
Prentice, I. C., Barton, C. V., Crous, K. Y., De angelis, P.,
Freeman, M. and Wingate, L. (2011), Reconciling the optimal and
empirical approaches to modelling stomatal conductance. Global
Change Biology, 17:
2134\sphinxhyphen{}2144. \sphinxurl{https://doi.org/10.1111/j.1365-2486.2010.02375.x}

\item {} 
\sphinxAtStartPar
Medlyn, B. E., Duursma, R. A., Eamus, D. , Ellsworth, D. S.,
Prentice, I. C., Barton, C. V., Crous, K. Y., De angelis, P.,
Freeman, M. and Wingate, L. (2012), Reconciling the optimal and
empirical approaches to modelling stomatal conductance. Global
Change Biology, 18:
3476\sphinxhyphen{}3476. \sphinxurl{https://doi.org/10.1111/j.1365-2486.2012.02790.x}.

\item {} 
\sphinxAtStartPar
Mercado, L. M., Huntingford, C., Gash, J. H. C., Cox, P. M., and
Jogireddy, V.: Improving the representation of radiative
interception and photosynthesis for climate model applications,
Tellus B, 59,
553\textendash{}565, 2007. \sphinxurl{https://doi.org/10.1111/j.1600-0889.2007.00256.x}

\item {} 
\sphinxAtStartPar
Mercado et al., 2018, Large sensitivity in land carbon storage due to
geographical and temporal variation in the thermal response of
photosynthetic capacity, New Phytologist, 218: 1462\textendash{}1477,
\sphinxurl{https://doi.org/10.1111/nph.15100}.

\item {} 
\sphinxAtStartPar
Sellers et al., 1992, Canopy reflectance, photosynthesis, and
transpiration. III. A reanalysis using improved leaf models and a
new canopy integration scheme. Remote Sens. Environ., 42, 187\sphinxhyphen{}216,
\sphinxurl{https://doi.org/10.1016/0034-4257(92)90102-P}

\item {} 
\sphinxAtStartPar
Wang, W. (2012), An Analytical Model for Mean Wind Profiles in
Sparse Canopies. Boundary\sphinxhyphen{}Layer Meteorol
142: 383. \sphinxurl{https://doi.org/10.1007/s10546-011-9687-0}

\item {} 
\sphinxAtStartPar
Mathison, C , Challinor, A. J., Deva, C., Falloon, P., Garrigues, S.,
Moulin, S., Williams, K., and Wiltshire, A. (2019),
Developing a sequential cropping capability in the JULESvn5.2
land\textendash{}surface model, Geosci. Model Dev. Discuss.,
\sphinxurl{https://doi.org/10.5194/gmd-2019-85}, in review, 2019.

\end{itemize}

\sphinxstepscope


\section{\sphinxstyleliteralintitle{\sphinxupquote{jules\_soil\_biogeochem.nml}}}
\label{\detokenize{namelists/jules_soil_biogeochem.nml:jules-soil-biogeochem-nml}}\label{\detokenize{namelists/jules_soil_biogeochem.nml::doc}}
\sphinxAtStartPar
This file sets options and parameters for soil biogeochemistry.

\sphinxAtStartPar
If using the single\sphinxhyphen{}pool or 4\sphinxhyphen{}pool soil models, all soil parameters are read from this file.

\sphinxAtStartPar
If using the ECOSSE soil model, most soil parameters are read from a separate file ({\hyperref[\detokenize{namelists/jules_soil_ecosse.nml:jules-soil-ecosse-namelist}]{\sphinxcrossref{\DUrole{std,std-ref}{jules\_soil\_ecosse.nml}}}}).


\subsection{\sphinxstyleliteralintitle{\sphinxupquote{JULES\_SOIL\_BIOGEOCHEM}} namelist members}
\label{\detokenize{namelists/jules_soil_biogeochem.nml:namelist-JULES_SOIL_BIOGEOCHEM}}\label{\detokenize{namelists/jules_soil_biogeochem.nml:jules-soil-biogeochem-namelist-members}}\index{JULES\_SOIL\_BIOGEOCHEM (namelist)@\spxentry{JULES\_SOIL\_BIOGEOCHEM}\spxextra{namelist}|spxpagem}\index{soil\_bgc\_model (in namelist JULES\_SOIL\_BIOGEOCHEM)@\spxentry{soil\_bgc\_model}\spxextra{in namelist JULES\_SOIL\_BIOGEOCHEM}|spxpagem}

\begin{fulllineitems}
\phantomsection\label{\detokenize{namelists/jules_soil_biogeochem.nml:JULES_SOIL_BIOGEOCHEM::soil_bgc_model}}
\pysigstartsignatures
\pysigline{\sphinxcode{\sphinxupquote{JULES\_SOIL\_BIOGEOCHEM::}}\sphinxbfcode{\sphinxupquote{soil\_bgc\_model}}}
\pysigstopsignatures\begin{quote}\begin{description}
\sphinxlineitem{Type}
\sphinxAtStartPar
integer

\sphinxlineitem{Permitted}
\sphinxAtStartPar
1, 2 or 3

\sphinxlineitem{Default}
\sphinxAtStartPar
1

\end{description}\end{quote}

\sphinxAtStartPar
Choice for model of soil biogeochemistry.

\sphinxAtStartPar
Possible values are:
\begin{enumerate}
\sphinxsetlistlabels{\arabic}{enumi}{enumii}{}{.}%
\item {} 
\begin{DUlineblock}{0em}
\item[] A single\sphinxhyphen{}pool model of soil carbon turnover in which the pool is not prognostic (not updated).
\item[] This must be used when the TRIFFID vegetation model is not selected ({\hyperref[\detokenize{namelists/jules_vegetation.nml:JULES_VEGETATION::l_triffid}]{\sphinxcrossref{\sphinxcode{\sphinxupquote{l\_triffid}}}}} = FALSE).
\end{DUlineblock}

\item {} 
\begin{DUlineblock}{0em}
\item[] A 4\sphinxhyphen{}pool model of soil organic carbon and nitrogen, originally based on the Jenkinson (1990) model, with a single pool of inorganic N.
\item[] Historically this was bundled with the TRIFFID vegetation model.
\item[] This can only be used if the TRIFFID vegetation model is selected ({\hyperref[\detokenize{namelists/jules_vegetation.nml:JULES_VEGETATION::l_triffid}]{\sphinxcrossref{\sphinxcode{\sphinxupquote{l\_triffid}}}}} = TRUE).
\end{DUlineblock}

\item {} 
\begin{DUlineblock}{0em}
\item[] A 4\sphinxhyphen{}pool model of soil organic carbon and nitrogen, and 2 inorganic N pools (ammonium and nitrate), based on the ECOSSE model (Smith et al., 2010).
\item[] This can only be used if the TRIFFID vegetation model is selected ({\hyperref[\detokenize{namelists/jules_vegetation.nml:JULES_VEGETATION::l_triffid}]{\sphinxcrossref{\sphinxcode{\sphinxupquote{l\_triffid}}}}} = TRUE).
\item[] This can also be run without nitrogen ({\hyperref[\detokenize{namelists/jules_soil_ecosse.nml:JULES_SOIL_ECOSSE::l_soil_n}]{\sphinxcrossref{\sphinxcode{\sphinxupquote{l\_soil\_n}}}}} = FALSE).
\end{DUlineblock}

\end{enumerate}

\begin{sphinxadmonition}{warning}{Warning:}
\sphinxAtStartPar
The ECOSSE model in JULES is still in development and is not fully functional in this version. The code is included to allow further development. Users should not try to use ECOSSE.
\end{sphinxadmonition}


\sphinxstrong{See also:}
\nopagebreak


\sphinxAtStartPar
References:
\begin{itemize}
\item {} 
\sphinxAtStartPar
Jenkinson, D.S., 1990. The turnover of organic carbon and nitrogen in soil. Philosophical Transactions of the Royal Society of London. Series B: Biological Sciences, 329(1255), pp.361\sphinxhyphen{}368. (\sphinxurl{https://doi.org/10.1098/rstb.1990.0177})

\item {} 
\sphinxAtStartPar
Smith et al., 2010, Estimating changes in Scottish soil carbon stocks using ECOSSE. I. Model description and uncertainties, Climate Research, 45: 179\sphinxhyphen{}192. (\sphinxurl{https://doi.org/10.3354/cr00899}).

\end{itemize}



\end{fulllineitems}


\begin{sphinxadmonition}{note}{Parameters that can be used with all soil models}
\index{q10\_soil (in namelist JULES\_SOIL\_BIOGEOCHEM)@\spxentry{q10\_soil}\spxextra{in namelist JULES\_SOIL\_BIOGEOCHEM}|spxpagem}

\begin{fulllineitems}
\phantomsection\label{\detokenize{namelists/jules_soil_biogeochem.nml:JULES_SOIL_BIOGEOCHEM::q10_soil}}
\pysigstartsignatures
\pysigline{\sphinxcode{\sphinxupquote{JULES\_SOIL\_BIOGEOCHEM::}}\sphinxbfcode{\sphinxupquote{q10\_soil}}}
\pysigstopsignatures\begin{quote}\begin{description}
\sphinxlineitem{Type}
\sphinxAtStartPar
real

\sphinxlineitem{Default}
\sphinxAtStartPar
2.0

\end{description}\end{quote}

\sphinxAtStartPar
Q10 factor for soil respiration.

\sphinxAtStartPar
With the single\sphinxhyphen{}pool or 4\sphinxhyphen{}pool models this is only used if {\hyperref[\detokenize{namelists/jules_soil_biogeochem.nml:JULES_SOIL_BIOGEOCHEM::l_q10}]{\sphinxcrossref{\sphinxcode{\sphinxupquote{l\_q10}}}}} = TRUE.

\sphinxAtStartPar
With the ECOSSE model this is only used if {\hyperref[\detokenize{namelists/jules_soil_ecosse.nml:JULES_SOIL_ECOSSE::temp_modifier}]{\sphinxcrossref{\sphinxcode{\sphinxupquote{temp\_modifier}}}}} = 1.

\sphinxAtStartPar
See Hadley Centre Technical Note 24, Eq.17, available from \sphinxhref{http://www.metoffice.gov.uk/learning/library/publications/science/climate-science-technical-notes}{the Met Office Library}.

\end{fulllineitems}

\end{sphinxadmonition}

\begin{sphinxadmonition}{note}{Parameters for the single\sphinxhyphen{}pool model (only used if \sphinxstyleliteralintitle{\sphinxupquote{soil\_bgc\_model}} = 1)}
\index{kaps (in namelist JULES\_SOIL\_BIOGEOCHEM)@\spxentry{kaps}\spxextra{in namelist JULES\_SOIL\_BIOGEOCHEM}|spxpagem}

\begin{fulllineitems}
\phantomsection\label{\detokenize{namelists/jules_soil_biogeochem.nml:JULES_SOIL_BIOGEOCHEM::kaps}}
\pysigstartsignatures
\pysigline{\sphinxcode{\sphinxupquote{JULES\_SOIL\_BIOGEOCHEM::}}\sphinxbfcode{\sphinxupquote{kaps}}}
\pysigstopsignatures\begin{quote}\begin{description}
\sphinxlineitem{Type}
\sphinxAtStartPar
real

\sphinxlineitem{Default}
\sphinxAtStartPar
0.5e\sphinxhyphen{}8

\end{description}\end{quote}

\sphinxAtStartPar
Specific soil respiration rate at 25 degC and optimum soil moisture (s$^{\text{\sphinxhyphen{}1}}$).

\sphinxAtStartPar
See Hadley Centre Technical Note 24, Eq.16, available from \sphinxhref{http://www.metoffice.gov.uk/learning/library/publications/science/climate-science-technical-notes}{the Met Office Library}.

\end{fulllineitems}

\end{sphinxadmonition}

\begin{sphinxadmonition}{note}{Parameters for the single\sphinxhyphen{}pool and 4\sphinxhyphen{}pool models (only used if \sphinxstyleliteralintitle{\sphinxupquote{soil\_bgc\_model}} = 1 or 2)}
\index{l\_q10 (in namelist JULES\_SOIL\_BIOGEOCHEM)@\spxentry{l\_q10}\spxextra{in namelist JULES\_SOIL\_BIOGEOCHEM}|spxpagem}

\begin{fulllineitems}
\phantomsection\label{\detokenize{namelists/jules_soil_biogeochem.nml:JULES_SOIL_BIOGEOCHEM::l_q10}}
\pysigstartsignatures
\pysigline{\sphinxcode{\sphinxupquote{JULES\_SOIL\_BIOGEOCHEM::}}\sphinxbfcode{\sphinxupquote{l\_q10}}}
\pysigstopsignatures\begin{quote}\begin{description}
\sphinxlineitem{Type}
\sphinxAtStartPar
logical

\sphinxlineitem{Default}
\sphinxAtStartPar
T

\end{description}\end{quote}

\sphinxAtStartPar
Switch for use of Q10 approach when calculating soil respiration.
\begin{description}
\sphinxlineitem{TRUE}\begin{quote}

\sphinxAtStartPar
Use Q10 approach (Equation 65 in Clark et al., 2011).
\end{quote}

\begin{sphinxadmonition}{note}{Note:}
\sphinxAtStartPar
This is always enforced if the single\sphinxhyphen{}pool model is selected ({\hyperref[\detokenize{namelists/jules_soil_biogeochem.nml:JULES_SOIL_BIOGEOCHEM::soil_bgc_model}]{\sphinxcrossref{\sphinxcode{\sphinxupquote{soil\_bgc\_model}}}}} = 1) and was used in JULES2.0.
\end{sphinxadmonition}

\sphinxlineitem{FALSE}
\sphinxAtStartPar
Use the approach of Jenkinson (1990) (Equation 66 in Clark et al., 2011).

\end{description}


\sphinxstrong{See also:}
\nopagebreak


\sphinxAtStartPar
References:
\begin{itemize}
\item {} 
\sphinxAtStartPar
Jenkinson, D.S., 1990. The turnover of organic carbon and nitrogen in soil. Philosophical Transactions of the Royal Society of London. Series B: Biological Sciences, 329(1255), pp.361\sphinxhyphen{}368. (\sphinxurl{https://doi.org/10.1098/rstb.1990.0177})

\item {} 
\sphinxAtStartPar
Clark, D. B., Mercado, L. M., Sitch, S., Jones, C. D., Gedney, N., Best, M. J., Pryor, M., Rooney, G. G., Essery, R. L. H., Blyth, E., Boucher, O., Harding, R. J., Huntingford, C., and Cox, P. M.: The Joint UK Land Environment Simulator (JULES), model description \textendash{} Part 2: Carbon fluxes and vegetation dynamics, Geosci. Model Dev., 4, 701\textendash{}722, (\sphinxurl{https://doi.org/10.5194/gmd-4-701-2011}), 2011.

\end{itemize}



\end{fulllineitems}

\index{l\_soil\_resp\_lev2 (in namelist JULES\_SOIL\_BIOGEOCHEM)@\spxentry{l\_soil\_resp\_lev2}\spxextra{in namelist JULES\_SOIL\_BIOGEOCHEM}|spxpagem}

\begin{fulllineitems}
\phantomsection\label{\detokenize{namelists/jules_soil_biogeochem.nml:JULES_SOIL_BIOGEOCHEM::l_soil_resp_lev2}}
\pysigstartsignatures
\pysigline{\sphinxcode{\sphinxupquote{JULES\_SOIL\_BIOGEOCHEM::}}\sphinxbfcode{\sphinxupquote{l\_soil\_resp\_lev2}}}
\pysigstopsignatures\begin{quote}\begin{description}
\sphinxlineitem{Type}
\sphinxAtStartPar
logical

\sphinxlineitem{Default}
\sphinxAtStartPar
F

\end{description}\end{quote}

\sphinxAtStartPar
Switch affecting the temperature and moisture used for soil respiration calculation.
\begin{description}
\sphinxlineitem{TRUE}
\sphinxAtStartPar
Temperature and total (frozen+unfrozen) moisture content of the second soil layer are used.

\sphinxlineitem{FALSE}
\sphinxAtStartPar
Temperature and unfrozen moisture content of the first (topmost) soil layer are used.

\begin{sphinxadmonition}{note}{Note:}
\sphinxAtStartPar
If layered soil C is used ({\hyperref[\detokenize{namelists/jules_soil_biogeochem.nml:JULES_SOIL_BIOGEOCHEM::l_layeredc}]{\sphinxcrossref{\sphinxcode{\sphinxupquote{l\_layeredc}}}}} = TRUE) the temperature and moisture of each soil layer is used to calculation respiration from that layer.
\end{sphinxadmonition}

\end{description}

\end{fulllineitems}

\end{sphinxadmonition}

\begin{sphinxadmonition}{note}{Parameters for the 4\sphinxhyphen{}pool model (only used if \sphinxstyleliteralintitle{\sphinxupquote{soil\_bgc\_model}} = 2)}
\index{l\_layeredc (in namelist JULES\_SOIL\_BIOGEOCHEM)@\spxentry{l\_layeredc}\spxextra{in namelist JULES\_SOIL\_BIOGEOCHEM}|spxpagem}

\begin{fulllineitems}
\phantomsection\label{\detokenize{namelists/jules_soil_biogeochem.nml:JULES_SOIL_BIOGEOCHEM::l_layeredc}}
\pysigstartsignatures
\pysigline{\sphinxcode{\sphinxupquote{JULES\_SOIL\_BIOGEOCHEM::}}\sphinxbfcode{\sphinxupquote{l\_layeredc}}}
\pysigstopsignatures\begin{quote}\begin{description}
\sphinxlineitem{Type}
\sphinxAtStartPar
logical

\sphinxlineitem{Default}
\sphinxAtStartPar
F

\end{description}\end{quote}

\sphinxAtStartPar
Switch for using the layered soil carbon model.

\sphinxAtStartPar
If the 4\sphinxhyphen{}pool model is used ({\hyperref[\detokenize{namelists/jules_soil_biogeochem.nml:JULES_SOIL_BIOGEOCHEM::soil_bgc_model}]{\sphinxcrossref{\sphinxcode{\sphinxupquote{soil\_bgc\_model}}}}} = 2) this uses the approach of Burke et al. (2017) and two extra parameters are required: {\hyperref[\detokenize{namelists/jules_soil_biogeochem.nml:JULES_SOIL_BIOGEOCHEM::tau_resp}]{\sphinxcrossref{\sphinxcode{\sphinxupquote{tau\_resp}}}}}, {\hyperref[\detokenize{namelists/jules_soil_biogeochem.nml:JULES_SOIL_BIOGEOCHEM::tau_lit}]{\sphinxcrossref{\sphinxcode{\sphinxupquote{tau\_lit}}}}}.

\sphinxAtStartPar
Layered soil nitrogen is also available if the nitrogen cycle is switched on ({\hyperref[\detokenize{namelists/jules_vegetation.nml:JULES_VEGETATION::l_nitrogen}]{\sphinxcrossref{\sphinxcode{\sphinxupquote{l\_nitrogen}}}}} = TRUE), but this is a highly experimental version which needs further evaluation and so should be used with extreme caution. One additional parameter is required for layered soil nitrogen: {\hyperref[\detokenize{namelists/jules_soil_biogeochem.nml:JULES_SOIL_BIOGEOCHEM::diff_n_pft}]{\sphinxcrossref{\sphinxcode{\sphinxupquote{diff\_n\_pft}}}}}.
\begin{description}
\sphinxlineitem{TRUE}
\sphinxAtStartPar
The number and thickness of layers in the soil carbon model are set equal to those in the soil moisture model ({\hyperref[\detokenize{namelists/jules_soil.nml:namelist-JULES_SOIL}]{\sphinxcrossref{\sphinxcode{\sphinxupquote{JULES\_SOIL}}}}}).

\sphinxlineitem{FALSE}
\sphinxAtStartPar
There are no specific layers in the soil carbon model (a single, bulk pool).

\end{description}


\sphinxstrong{See also:}
\nopagebreak


\sphinxAtStartPar
References:
\begin{itemize}
\item {} 
\sphinxAtStartPar
Burke, E. J., Chadburn, S. E., and Ekici, A.: A vertical representation of soil carbon in the JULES land surface scheme (vn4.3\_permafrost) with a focus on permafrost regions, Geosci. Model Dev., 10, 959\sphinxhyphen{}975, doi:10.5194/gmd\sphinxhyphen{}10\sphinxhyphen{}959\sphinxhyphen{}2017, 2017.

\end{itemize}



\end{fulllineitems}

\index{l\_label\_frac\_cs (in namelist JULES\_SOIL\_BIOGEOCHEM)@\spxentry{l\_label\_frac\_cs}\spxextra{in namelist JULES\_SOIL\_BIOGEOCHEM}|spxpagem}

\begin{fulllineitems}
\phantomsection\label{\detokenize{namelists/jules_soil_biogeochem.nml:JULES_SOIL_BIOGEOCHEM::l_label_frac_cs}}
\pysigstartsignatures
\pysigline{\sphinxcode{\sphinxupquote{JULES\_SOIL\_BIOGEOCHEM::}}\sphinxbfcode{\sphinxupquote{l\_label\_frac\_cs}}}
\pysigstopsignatures\begin{quote}\begin{description}
\sphinxlineitem{Type}
\sphinxAtStartPar
logical

\sphinxlineitem{Default}
\sphinxAtStartPar
F

\end{description}\end{quote}

\sphinxAtStartPar
Switch for labelling and tracing a subset of the layered soil carbon ({\hyperref[\detokenize{namelists/jules_soil_biogeochem.nml:JULES_SOIL_BIOGEOCHEM::l_layeredc}]{\sphinxcrossref{\sphinxcode{\sphinxupquote{l\_layeredc}}}}} = TRUE). It uses the approach of Burke et al.(2017). This requires the 4\sphinxhyphen{}pool model to be used ({\hyperref[\detokenize{namelists/jules_soil_biogeochem.nml:JULES_SOIL_BIOGEOCHEM::soil_bgc_model}]{\sphinxcrossref{\sphinxcode{\sphinxupquote{soil\_bgc\_model}}}}} = 2). The fraction of labelled soil carbon needs to be specified as part of the model’s initial state.
\begin{description}
\sphinxlineitem{TRUE}
\sphinxAtStartPar
A user\sphinxhyphen{}defined fraction of soil carbon is labelled.

\sphinxlineitem{FALSE}
\sphinxAtStartPar
None of the soil carbon is labelled.

\end{description}


\sphinxstrong{See also:}
\nopagebreak


\sphinxAtStartPar
References:
\begin{itemize}
\item {} 
\sphinxAtStartPar
Burke, E. J., Chadburn, S. E., and Ekici, A.: A vertical representation of soil carbon in the JULES land surface scheme (vn4.3\_permafrost) with a focus on permafrost regions, Geosci. Model Dev., 10, 959\sphinxhyphen{}975, doi:10.5194/gmd\sphinxhyphen{}10\sphinxhyphen{}959\sphinxhyphen{}2017, 2017.

\end{itemize}



\end{fulllineitems}

\index{kaps\_4pool (in namelist JULES\_SOIL\_BIOGEOCHEM)@\spxentry{kaps\_4pool}\spxextra{in namelist JULES\_SOIL\_BIOGEOCHEM}|spxpagem}

\begin{fulllineitems}
\phantomsection\label{\detokenize{namelists/jules_soil_biogeochem.nml:JULES_SOIL_BIOGEOCHEM::kaps_4pool}}
\pysigstartsignatures
\pysigline{\sphinxcode{\sphinxupquote{JULES\_SOIL\_BIOGEOCHEM::}}\sphinxbfcode{\sphinxupquote{kaps\_4pool}}}
\pysigstopsignatures\begin{quote}\begin{description}
\sphinxlineitem{Type}
\sphinxAtStartPar
real(4)

\sphinxlineitem{Default}
\sphinxAtStartPar
3.22e\sphinxhyphen{}7, 9.65e\sphinxhyphen{}9, 2.12e\sphinxhyphen{}8, 6.43e\sphinxhyphen{}10

\end{description}\end{quote}

\sphinxAtStartPar
Specific soil respiration rate for the 4\sphinxhyphen{}pool submodel for each soil carbon pool (decomposable plant material, resistant plant material, biomass, humus).

\end{fulllineitems}

\index{bio\_hum\_cn (in namelist JULES\_SOIL\_BIOGEOCHEM)@\spxentry{bio\_hum\_cn}\spxextra{in namelist JULES\_SOIL\_BIOGEOCHEM}|spxpagem}

\begin{fulllineitems}
\phantomsection\label{\detokenize{namelists/jules_soil_biogeochem.nml:JULES_SOIL_BIOGEOCHEM::bio_hum_cn}}
\pysigstartsignatures
\pysigline{\sphinxcode{\sphinxupquote{JULES\_SOIL\_BIOGEOCHEM::}}\sphinxbfcode{\sphinxupquote{bio\_hum\_cn}}}
\pysigstopsignatures\begin{quote}\begin{description}
\sphinxlineitem{Type}
\sphinxAtStartPar
real

\sphinxlineitem{Default}
\sphinxAtStartPar
10.0

\end{description}\end{quote}

\sphinxAtStartPar
Parameter controlling ratio of C to N for BIO and HUM pools.

\end{fulllineitems}

\index{sorp (in namelist JULES\_SOIL\_BIOGEOCHEM)@\spxentry{sorp}\spxextra{in namelist JULES\_SOIL\_BIOGEOCHEM}|spxpagem}

\begin{fulllineitems}
\phantomsection\label{\detokenize{namelists/jules_soil_biogeochem.nml:JULES_SOIL_BIOGEOCHEM::sorp}}
\pysigstartsignatures
\pysigline{\sphinxcode{\sphinxupquote{JULES\_SOIL\_BIOGEOCHEM::}}\sphinxbfcode{\sphinxupquote{sorp}}}
\pysigstopsignatures\begin{quote}\begin{description}
\sphinxlineitem{Type}
\sphinxAtStartPar
real

\sphinxlineitem{Default}
\sphinxAtStartPar
10.0

\end{description}\end{quote}

\sphinxAtStartPar
Parameter controlling the leaching of inorganic N through the soil profile. A factor of 1 means that in a timestep all the inorganic N is available for leaching. The default value of 10 means that 10\% of inorganic N is available for leaching.

\end{fulllineitems}

\index{n\_inorg\_turnover (in namelist JULES\_SOIL\_BIOGEOCHEM)@\spxentry{n\_inorg\_turnover}\spxextra{in namelist JULES\_SOIL\_BIOGEOCHEM}|spxpagem}

\begin{fulllineitems}
\phantomsection\label{\detokenize{namelists/jules_soil_biogeochem.nml:JULES_SOIL_BIOGEOCHEM::n_inorg_turnover}}
\pysigstartsignatures
\pysigline{\sphinxcode{\sphinxupquote{JULES\_SOIL\_BIOGEOCHEM::}}\sphinxbfcode{\sphinxupquote{n\_inorg\_turnover}}}
\pysigstopsignatures\begin{quote}\begin{description}
\sphinxlineitem{Type}
\sphinxAtStartPar
real

\sphinxlineitem{Default}
\sphinxAtStartPar
1.0

\end{description}\end{quote}

\sphinxAtStartPar
Parameter controlling the lifetime of the inorganic N pool. A value of 1 implies the whole pool will turnover in 360 days.

\end{fulllineitems}

\index{tau\_resp (in namelist JULES\_SOIL\_BIOGEOCHEM)@\spxentry{tau\_resp}\spxextra{in namelist JULES\_SOIL\_BIOGEOCHEM}|spxpagem}

\begin{fulllineitems}
\phantomsection\label{\detokenize{namelists/jules_soil_biogeochem.nml:JULES_SOIL_BIOGEOCHEM::tau_resp}}
\pysigstartsignatures
\pysigline{\sphinxcode{\sphinxupquote{JULES\_SOIL\_BIOGEOCHEM::}}\sphinxbfcode{\sphinxupquote{tau\_resp}}}
\pysigstopsignatures\begin{quote}\begin{description}
\sphinxlineitem{Type}
\sphinxAtStartPar
real

\sphinxlineitem{Default}
\sphinxAtStartPar
2.0

\end{description}\end{quote}

\sphinxAtStartPar
Parameter controlling decay of respiration with depth (m\sphinxhyphen{}1). Only used with layered soil carbon ({\hyperref[\detokenize{namelists/jules_soil_biogeochem.nml:JULES_SOIL_BIOGEOCHEM::l_layeredc}]{\sphinxcrossref{\sphinxcode{\sphinxupquote{l\_layeredc}}}}} = TRUE).

\end{fulllineitems}

\index{diff\_n\_pft (in namelist JULES\_SOIL\_BIOGEOCHEM)@\spxentry{diff\_n\_pft}\spxextra{in namelist JULES\_SOIL\_BIOGEOCHEM}|spxpagem}

\begin{fulllineitems}
\phantomsection\label{\detokenize{namelists/jules_soil_biogeochem.nml:JULES_SOIL_BIOGEOCHEM::diff_n_pft}}
\pysigstartsignatures
\pysigline{\sphinxcode{\sphinxupquote{JULES\_SOIL\_BIOGEOCHEM::}}\sphinxbfcode{\sphinxupquote{diff\_n\_pft}}}
\pysigstopsignatures\begin{quote}\begin{description}
\sphinxlineitem{Type}
\sphinxAtStartPar
real

\sphinxlineitem{Default}
\sphinxAtStartPar
5.0

\end{description}\end{quote}

\sphinxAtStartPar
Parameter controlling the rate of re\sphinxhyphen{}filling of the available inorganic nitrogen pool (1/360 days). This parameter determines how quickly the inorganic nitrogen reaches the roots after the roots uptake from the soil around them. This should be quicker than the turnover rate of inorganic nitrogen. In addition, it has to be small compared with the triffid timestep (360/triffid\_period) otherwise the available inorganic nitrogen becomes unstable. Hence the choice of the default value 5. Only used with layered soil carbon and nitrogen scheme ({\hyperref[\detokenize{namelists/jules_soil_biogeochem.nml:JULES_SOIL_BIOGEOCHEM::l_layeredc}]{\sphinxcrossref{\sphinxcode{\sphinxupquote{l\_layeredc}}}}} = TRUE and {\hyperref[\detokenize{namelists/jules_vegetation.nml:JULES_VEGETATION::l_nitrogen}]{\sphinxcrossref{\sphinxcode{\sphinxupquote{l\_nitrogen}}}}} = TRUE). When {\hyperref[\detokenize{namelists/jules_vegetation.nml:JULES_VEGETATION::l_trif_eq}]{\sphinxcrossref{\sphinxcode{\sphinxupquote{l\_trif\_eq}}}}} = TRUE or {\hyperref[\detokenize{namelists/jules_soil_biogeochem.nml:JULES_SOIL_BIOGEOCHEM::diff_n_pft}]{\sphinxcrossref{\sphinxcode{\sphinxupquote{diff\_n\_pft}}}}} is greater than (0.5 * 360 / {\hyperref[\detokenize{namelists/jules_vegetation.nml:JULES_VEGETATION::triffid_period}]{\sphinxcrossref{\sphinxcode{\sphinxupquote{triffid\_period}}}}}) then all of the inorganic nitrogen pool is deemed to be available.

\end{fulllineitems}

\end{sphinxadmonition}

\begin{sphinxadmonition}{note}{Parameters for the 4\sphinxhyphen{}pool\sphinxhyphen{} or ECOSSE\sphinxhyphen{}based models (only used if \sphinxstyleliteralintitle{\sphinxupquote{soil\_bgc\_model}} = 2 or 3):}
\index{tau\_lit (in namelist JULES\_SOIL\_BIOGEOCHEM)@\spxentry{tau\_lit}\spxextra{in namelist JULES\_SOIL\_BIOGEOCHEM}|spxpagem}

\begin{fulllineitems}
\phantomsection\label{\detokenize{namelists/jules_soil_biogeochem.nml:JULES_SOIL_BIOGEOCHEM::tau_lit}}
\pysigstartsignatures
\pysigline{\sphinxcode{\sphinxupquote{JULES\_SOIL\_BIOGEOCHEM::}}\sphinxbfcode{\sphinxupquote{tau\_lit}}}
\pysigstopsignatures\begin{quote}\begin{description}
\sphinxlineitem{Type}
\sphinxAtStartPar
real

\sphinxlineitem{Default}
\sphinxAtStartPar
5.0

\end{description}\end{quote}

\sphinxAtStartPar
Parameter controlling the decay of litter with depth (m\sphinxhyphen{}1).
With 4\sphinxhyphen{}pool, this is only used with layered soil carbon ({\hyperref[\detokenize{namelists/jules_soil_biogeochem.nml:JULES_SOIL_BIOGEOCHEM::l_layeredc}]{\sphinxcrossref{\sphinxcode{\sphinxupquote{l\_layeredc}}}}} = TRUE).
With ECOSSE, this is only used with {\hyperref[\detokenize{namelists/jules_soil_ecosse.nml:JULES_SOIL_ECOSSE::plant_input_profile}]{\sphinxcrossref{\sphinxcode{\sphinxupquote{plant\_input\_profile}}}}} = 2.

\end{fulllineitems}

\end{sphinxadmonition}

\begin{sphinxadmonition}{note}{Methane parameters and switches. Can only be used with the single\sphinxhyphen{}pool and 4\sphinxhyphen{}pool models (\sphinxstyleliteralintitle{\sphinxupquote{soil\_bgc\_model}} = 1 or 2).}

\begin{sphinxadmonition}{warning}{Warning:}
\sphinxAtStartPar
Some parameters may need to be re\sphinxhyphen{}tuned for different soil biogeochemistry models.
\end{sphinxadmonition}
\index{l\_ch4\_tlayered (in namelist JULES\_SOIL\_BIOGEOCHEM)@\spxentry{l\_ch4\_tlayered}\spxextra{in namelist JULES\_SOIL\_BIOGEOCHEM}|spxpagem}

\begin{fulllineitems}
\phantomsection\label{\detokenize{namelists/jules_soil_biogeochem.nml:JULES_SOIL_BIOGEOCHEM::l_ch4_tlayered}}
\pysigstartsignatures
\pysigline{\sphinxcode{\sphinxupquote{JULES\_SOIL\_BIOGEOCHEM::}}\sphinxbfcode{\sphinxupquote{l\_ch4\_tlayered}}}
\pysigstopsignatures\begin{quote}\begin{description}
\sphinxlineitem{Type}
\sphinxAtStartPar
logical

\sphinxlineitem{Default}
\sphinxAtStartPar
F

\end{description}\end{quote}

\sphinxAtStartPar
Switch to calculate methane emissions based on layered soil temperature.
\begin{description}
\sphinxlineitem{TRUE}
\sphinxAtStartPar
Methane emission is calculated from layered soil temperatures.

\sphinxlineitem{FALSE}
\sphinxAtStartPar
Methane emission is calculated from top 1m average soil temperature (default).

\end{description}

\end{fulllineitems}

\index{l\_ch4\_interactive (in namelist JULES\_SOIL\_BIOGEOCHEM)@\spxentry{l\_ch4\_interactive}\spxextra{in namelist JULES\_SOIL\_BIOGEOCHEM}|spxpagem}

\begin{fulllineitems}
\phantomsection\label{\detokenize{namelists/jules_soil_biogeochem.nml:JULES_SOIL_BIOGEOCHEM::l_ch4_interactive}}
\pysigstartsignatures
\pysigline{\sphinxcode{\sphinxupquote{JULES\_SOIL\_BIOGEOCHEM::}}\sphinxbfcode{\sphinxupquote{l\_ch4\_interactive}}}
\pysigstopsignatures\begin{quote}\begin{description}
\sphinxlineitem{Type}
\sphinxAtStartPar
logical

\sphinxlineitem{Default}
\sphinxAtStartPar
F

\end{description}\end{quote}

\sphinxAtStartPar
Switch to couple the methane emission into the carbon cycle. In order to use this the methane must be calculated from layered soil temperature ({\hyperref[\detokenize{namelists/jules_soil_biogeochem.nml:JULES_SOIL_BIOGEOCHEM::l_ch4_tlayered}]{\sphinxcrossref{\sphinxcode{\sphinxupquote{l\_ch4\_tlayered}}}}} = TRUE).
\begin{description}
\sphinxlineitem{TRUE}
\sphinxAtStartPar
Methane flux is subtracted from soil carbon stocks.

\sphinxlineitem{FALSE}
\sphinxAtStartPar
Methane emission is only diagnostic (default).

\end{description}

\end{fulllineitems}

\index{l\_ch4\_microbe (in namelist JULES\_SOIL\_BIOGEOCHEM)@\spxentry{l\_ch4\_microbe}\spxextra{in namelist JULES\_SOIL\_BIOGEOCHEM}|spxpagem}

\begin{fulllineitems}
\phantomsection\label{\detokenize{namelists/jules_soil_biogeochem.nml:JULES_SOIL_BIOGEOCHEM::l_ch4_microbe}}
\pysigstartsignatures
\pysigline{\sphinxcode{\sphinxupquote{JULES\_SOIL\_BIOGEOCHEM::}}\sphinxbfcode{\sphinxupquote{l\_ch4\_microbe}}}
\pysigstopsignatures\begin{quote}\begin{description}
\sphinxlineitem{Type}
\sphinxAtStartPar
logical

\sphinxlineitem{Default}
\sphinxAtStartPar
F

\end{description}\end{quote}

\sphinxAtStartPar
Switch to enable the microbial methane production scheme (represents the dynamics of methanogens and a dissolved substrate pool). See Chadburn et al. (2020).

\begin{sphinxadmonition}{note}{Note:}
\sphinxAtStartPar
This will only be applied to the methane production from your chosen {\hyperref[\detokenize{namelists/jules_soil_biogeochem.nml:JULES_SOIL_BIOGEOCHEM::ch4_substrate}]{\sphinxcrossref{\sphinxcode{\sphinxupquote{ch4\_substrate}}}}}. The scheme has been calibrated with {\hyperref[\detokenize{namelists/jules_soil_biogeochem.nml:JULES_SOIL_BIOGEOCHEM::ch4_substrate}]{\sphinxcrossref{\sphinxcode{\sphinxupquote{ch4\_substrate}}}}} = 1.
\end{sphinxadmonition}
\begin{description}
\sphinxlineitem{TRUE}
\sphinxAtStartPar
Microbial dynamics simulated in methane scheme.

\sphinxlineitem{FALSE}
\sphinxAtStartPar
No microbial dynamics, decomposition of substrate translates immediately to methane emissions.

\end{description}

\end{fulllineitems}

\index{ch4\_substrate (in namelist JULES\_SOIL\_BIOGEOCHEM)@\spxentry{ch4\_substrate}\spxextra{in namelist JULES\_SOIL\_BIOGEOCHEM}|spxpagem}

\begin{fulllineitems}
\phantomsection\label{\detokenize{namelists/jules_soil_biogeochem.nml:JULES_SOIL_BIOGEOCHEM::ch4_substrate}}
\pysigstartsignatures
\pysigline{\sphinxcode{\sphinxupquote{JULES\_SOIL\_BIOGEOCHEM::}}\sphinxbfcode{\sphinxupquote{ch4\_substrate}}}
\pysigstopsignatures\begin{quote}\begin{description}
\sphinxlineitem{Type}
\sphinxAtStartPar
integer

\sphinxlineitem{Permitted}
\sphinxAtStartPar
1, 2 or 3

\sphinxlineitem{Default}
\sphinxAtStartPar
1

\end{description}\end{quote}

\sphinxAtStartPar
Choice of substrate for wetland methane. This controls the calculation method for the methane flux that is used to update soil carbon (only if {\hyperref[\detokenize{namelists/jules_soil_biogeochem.nml:JULES_SOIL_BIOGEOCHEM::l_ch4_interactive}]{\sphinxcrossref{\sphinxcode{\sphinxupquote{l\_ch4\_interactive}}}}} = TRUE) and to populate the variable fch4\_wetl (seen by the atmospheric model in coupled mode).

\sphinxAtStartPar
Possible values are:
\begin{enumerate}
\sphinxsetlistlabels{\arabic}{enumi}{enumii}{}{.}%
\item {} 
\begin{DUlineblock}{0em}
\item[] Using soil carbon as substrate (default).
\end{DUlineblock}

\item {} 
\begin{DUlineblock}{0em}
\item[] Using NPP as substrate.
\end{DUlineblock}

\item {} 
\begin{DUlineblock}{0em}
\item[] Using soil respiration as substrate.
\end{DUlineblock}

\end{enumerate}

\sphinxAtStartPar
This replaces the previous switch l\_wetland\_ch4\_npp.

\end{fulllineitems}

\index{t0\_ch4 (in namelist JULES\_SOIL\_BIOGEOCHEM)@\spxentry{t0\_ch4}\spxextra{in namelist JULES\_SOIL\_BIOGEOCHEM}|spxpagem}

\begin{fulllineitems}
\phantomsection\label{\detokenize{namelists/jules_soil_biogeochem.nml:JULES_SOIL_BIOGEOCHEM::t0_ch4}}
\pysigstartsignatures
\pysigline{\sphinxcode{\sphinxupquote{JULES\_SOIL\_BIOGEOCHEM::}}\sphinxbfcode{\sphinxupquote{t0\_ch4}}}
\pysigstopsignatures\begin{quote}\begin{description}
\sphinxlineitem{Type}
\sphinxAtStartPar
real

\sphinxlineitem{Default}
\sphinxAtStartPar
273.15

\end{description}\end{quote}

\sphinxAtStartPar
Reference temperature for the Q10 function CH4 emission calculation

\end{fulllineitems}

\index{const\_ch4\_cs (in namelist JULES\_SOIL\_BIOGEOCHEM)@\spxentry{const\_ch4\_cs}\spxextra{in namelist JULES\_SOIL\_BIOGEOCHEM}|spxpagem}

\begin{fulllineitems}
\phantomsection\label{\detokenize{namelists/jules_soil_biogeochem.nml:JULES_SOIL_BIOGEOCHEM::const_ch4_cs}}
\pysigstartsignatures
\pysigline{\sphinxcode{\sphinxupquote{JULES\_SOIL\_BIOGEOCHEM::}}\sphinxbfcode{\sphinxupquote{const\_ch4\_cs}}}
\pysigstopsignatures\begin{quote}\begin{description}
\sphinxlineitem{Type}
\sphinxAtStartPar
real

\sphinxlineitem{Default}
\sphinxAtStartPar
7.41e\sphinxhyphen{}12

\end{description}\end{quote}

\sphinxAtStartPar
Scale factor for wetland CH4 emissions when soil carbon is taken as the substrate for ch4 emissions ({\hyperref[\detokenize{namelists/jules_soil_biogeochem.nml:JULES_SOIL_BIOGEOCHEM::ch4_substrate}]{\sphinxcrossref{\sphinxcode{\sphinxupquote{ch4\_substrate}}}}} = 1)

\begin{sphinxadmonition}{note}{Note:}
\sphinxAtStartPar
In the UM the recommended value depends on {\hyperref[\detokenize{namelists/jules_vegetation.nml:JULES_VEGETATION::l_triffid}]{\sphinxcrossref{\sphinxcode{\sphinxupquote{l\_triffid}}}}} as follows:

\begin{DUlineblock}{0em}
\item[] {\hyperref[\detokenize{namelists/jules_vegetation.nml:JULES_VEGETATION::l_triffid}]{\sphinxcrossref{\sphinxcode{\sphinxupquote{l\_triffid}}}}} = FALSE, {\hyperref[\detokenize{namelists/jules_soil_biogeochem.nml:JULES_SOIL_BIOGEOCHEM::const_ch4_cs}]{\sphinxcrossref{\sphinxcode{\sphinxupquote{const\_ch4\_cs}}}}} = 5.41e\sphinxhyphen{}12
\item[] {\hyperref[\detokenize{namelists/jules_vegetation.nml:JULES_VEGETATION::l_triffid}]{\sphinxcrossref{\sphinxcode{\sphinxupquote{l\_triffid}}}}} = TRUE, {\hyperref[\detokenize{namelists/jules_soil_biogeochem.nml:JULES_SOIL_BIOGEOCHEM::const_ch4_cs}]{\sphinxcrossref{\sphinxcode{\sphinxupquote{const\_ch4\_cs}}}}} = 5.41e\sphinxhyphen{}10
\end{DUlineblock}
\end{sphinxadmonition}

\end{fulllineitems}

\index{q10\_ch4\_cs (in namelist JULES\_SOIL\_BIOGEOCHEM)@\spxentry{q10\_ch4\_cs}\spxextra{in namelist JULES\_SOIL\_BIOGEOCHEM}|spxpagem}

\begin{fulllineitems}
\phantomsection\label{\detokenize{namelists/jules_soil_biogeochem.nml:JULES_SOIL_BIOGEOCHEM::q10_ch4_cs}}
\pysigstartsignatures
\pysigline{\sphinxcode{\sphinxupquote{JULES\_SOIL\_BIOGEOCHEM::}}\sphinxbfcode{\sphinxupquote{q10\_ch4\_cs}}}
\pysigstopsignatures\begin{quote}\begin{description}
\sphinxlineitem{Type}
\sphinxAtStartPar
real

\sphinxlineitem{Default}
\sphinxAtStartPar
3.7

\end{description}\end{quote}

\sphinxAtStartPar
Q10 value for wetland CH4 emissions when soil carbon is taken as the substrate for ch4 emissions ({\hyperref[\detokenize{namelists/jules_soil_biogeochem.nml:JULES_SOIL_BIOGEOCHEM::ch4_substrate}]{\sphinxcrossref{\sphinxcode{\sphinxupquote{ch4\_substrate}}}}} = 1)

\end{fulllineitems}

\index{const\_ch4\_npp (in namelist JULES\_SOIL\_BIOGEOCHEM)@\spxentry{const\_ch4\_npp}\spxextra{in namelist JULES\_SOIL\_BIOGEOCHEM}|spxpagem}

\begin{fulllineitems}
\phantomsection\label{\detokenize{namelists/jules_soil_biogeochem.nml:JULES_SOIL_BIOGEOCHEM::const_ch4_npp}}
\pysigstartsignatures
\pysigline{\sphinxcode{\sphinxupquote{JULES\_SOIL\_BIOGEOCHEM::}}\sphinxbfcode{\sphinxupquote{const\_ch4\_npp}}}
\pysigstopsignatures\begin{quote}\begin{description}
\sphinxlineitem{Type}
\sphinxAtStartPar
real

\sphinxlineitem{Default}
\sphinxAtStartPar
9.99e\sphinxhyphen{}3

\end{description}\end{quote}

\sphinxAtStartPar
Scale factor for wetland CH4 emissions when NPP is taken as the substrate for ch4 emissions ({\hyperref[\detokenize{namelists/jules_soil_biogeochem.nml:JULES_SOIL_BIOGEOCHEM::ch4_substrate}]{\sphinxcrossref{\sphinxcode{\sphinxupquote{ch4\_substrate}}}}} = 2)

\end{fulllineitems}

\index{q10\_ch4\_npp (in namelist JULES\_SOIL\_BIOGEOCHEM)@\spxentry{q10\_ch4\_npp}\spxextra{in namelist JULES\_SOIL\_BIOGEOCHEM}|spxpagem}

\begin{fulllineitems}
\phantomsection\label{\detokenize{namelists/jules_soil_biogeochem.nml:JULES_SOIL_BIOGEOCHEM::q10_ch4_npp}}
\pysigstartsignatures
\pysigline{\sphinxcode{\sphinxupquote{JULES\_SOIL\_BIOGEOCHEM::}}\sphinxbfcode{\sphinxupquote{q10\_ch4\_npp}}}
\pysigstopsignatures\begin{quote}\begin{description}
\sphinxlineitem{Type}
\sphinxAtStartPar
real

\sphinxlineitem{Default}
\sphinxAtStartPar
1.5

\end{description}\end{quote}

\sphinxAtStartPar
Q10 value for wetland CH4 emissions when npp is taken as the substrate for ch4 emissions ({\hyperref[\detokenize{namelists/jules_soil_biogeochem.nml:JULES_SOIL_BIOGEOCHEM::ch4_substrate}]{\sphinxcrossref{\sphinxcode{\sphinxupquote{ch4\_substrate}}}}} = 2)

\end{fulllineitems}

\index{const\_ch4\_resps (in namelist JULES\_SOIL\_BIOGEOCHEM)@\spxentry{const\_ch4\_resps}\spxextra{in namelist JULES\_SOIL\_BIOGEOCHEM}|spxpagem}

\begin{fulllineitems}
\phantomsection\label{\detokenize{namelists/jules_soil_biogeochem.nml:JULES_SOIL_BIOGEOCHEM::const_ch4_resps}}
\pysigstartsignatures
\pysigline{\sphinxcode{\sphinxupquote{JULES\_SOIL\_BIOGEOCHEM::}}\sphinxbfcode{\sphinxupquote{const\_ch4\_resps}}}
\pysigstopsignatures\begin{quote}\begin{description}
\sphinxlineitem{Type}
\sphinxAtStartPar
real

\sphinxlineitem{Default}
\sphinxAtStartPar
4.36e\sphinxhyphen{}3

\end{description}\end{quote}

\sphinxAtStartPar
Scale factor for wetland CH4 emissions when soil respiration is taken as the substrate for ch4 emissions  ({\hyperref[\detokenize{namelists/jules_soil_biogeochem.nml:JULES_SOIL_BIOGEOCHEM::ch4_substrate}]{\sphinxcrossref{\sphinxcode{\sphinxupquote{ch4\_substrate}}}}} = 3)

\end{fulllineitems}

\index{q10\_ch4\_resps (in namelist JULES\_SOIL\_BIOGEOCHEM)@\spxentry{q10\_ch4\_resps}\spxextra{in namelist JULES\_SOIL\_BIOGEOCHEM}|spxpagem}

\begin{fulllineitems}
\phantomsection\label{\detokenize{namelists/jules_soil_biogeochem.nml:JULES_SOIL_BIOGEOCHEM::q10_ch4_resps}}
\pysigstartsignatures
\pysigline{\sphinxcode{\sphinxupquote{JULES\_SOIL\_BIOGEOCHEM::}}\sphinxbfcode{\sphinxupquote{q10\_ch4\_resps}}}
\pysigstopsignatures\begin{quote}\begin{description}
\sphinxlineitem{Type}
\sphinxAtStartPar
real

\sphinxlineitem{Default}
\sphinxAtStartPar
1.5

\end{description}\end{quote}

\sphinxAtStartPar
Q10 value for wetland CH4 emissions when soil respiration is taken as the substrate for ch4 emissions ({\hyperref[\detokenize{namelists/jules_soil_biogeochem.nml:JULES_SOIL_BIOGEOCHEM::ch4_substrate}]{\sphinxcrossref{\sphinxcode{\sphinxupquote{ch4\_substrate}}}}} = 3)

\end{fulllineitems}

\index{ch4\_cpow (in namelist JULES\_SOIL\_BIOGEOCHEM)@\spxentry{ch4\_cpow}\spxextra{in namelist JULES\_SOIL\_BIOGEOCHEM}|spxpagem}

\begin{fulllineitems}
\phantomsection\label{\detokenize{namelists/jules_soil_biogeochem.nml:JULES_SOIL_BIOGEOCHEM::ch4_cpow}}
\pysigstartsignatures
\pysigline{\sphinxcode{\sphinxupquote{JULES\_SOIL\_BIOGEOCHEM::}}\sphinxbfcode{\sphinxupquote{ch4\_cpow}}}
\pysigstopsignatures\begin{quote}\begin{description}
\sphinxlineitem{Type}
\sphinxAtStartPar
real

\sphinxlineitem{Default}
\sphinxAtStartPar
1.0

\end{description}\end{quote}

\sphinxAtStartPar
Power of soil carbon used to calculate methane emissions with soil carbon as substrate ({\hyperref[\detokenize{namelists/jules_soil_biogeochem.nml:JULES_SOIL_BIOGEOCHEM::ch4_substrate}]{\sphinxcrossref{\sphinxcode{\sphinxupquote{ch4\_substrate}}}}} = 1). Methane production is calculated as cs$^{\text{ch4\_cpow}}$. A value of 1.0 is default, but a value of 2/3 is consistent with an assumption that only the surfaces of the organic matter are accessible.

\begin{sphinxadmonition}{note}{Note:}
\sphinxAtStartPar
{\hyperref[\detokenize{namelists/jules_soil_biogeochem.nml:JULES_SOIL_BIOGEOCHEM::const_ch4_cs}]{\sphinxcrossref{\sphinxcode{\sphinxupquote{const\_ch4\_cs}}}}} will need retuning if this parameter is changed.
\end{sphinxadmonition}

\end{fulllineitems}

\end{sphinxadmonition}

\begin{sphinxadmonition}{note}{Methane parameters only used with layered soil temperatures (\sphinxstyleliteralintitle{\sphinxupquote{l\_ch4\_tlayered}} = TRUE).}
\index{tau\_ch4 (in namelist JULES\_SOIL\_BIOGEOCHEM)@\spxentry{tau\_ch4}\spxextra{in namelist JULES\_SOIL\_BIOGEOCHEM}|spxpagem}

\begin{fulllineitems}
\phantomsection\label{\detokenize{namelists/jules_soil_biogeochem.nml:JULES_SOIL_BIOGEOCHEM::tau_ch4}}
\pysigstartsignatures
\pysigline{\sphinxcode{\sphinxupquote{JULES\_SOIL\_BIOGEOCHEM::}}\sphinxbfcode{\sphinxupquote{tau\_ch4}}}
\pysigstopsignatures\begin{quote}\begin{description}
\sphinxlineitem{Type}
\sphinxAtStartPar
real

\sphinxlineitem{Default}
\sphinxAtStartPar
6.5

\end{description}\end{quote}

\sphinxAtStartPar
Exponent in the exponential decline of methane emissions with soil depth (m\sphinxhyphen{}1).
This empirically represents methane oxidation/emission processes, which only allow a fraction of the methane produced in the soil to reach the atmosphere.

\end{fulllineitems}

\end{sphinxadmonition}

\begin{sphinxadmonition}{note}{Methane parameters only used with microbial methane scheme (\sphinxstyleliteralintitle{\sphinxupquote{l\_ch4\_microbe}} = TRUE).}
\index{k2\_ch4 (in namelist JULES\_SOIL\_BIOGEOCHEM)@\spxentry{k2\_ch4}\spxextra{in namelist JULES\_SOIL\_BIOGEOCHEM}|spxpagem}

\begin{fulllineitems}
\phantomsection\label{\detokenize{namelists/jules_soil_biogeochem.nml:JULES_SOIL_BIOGEOCHEM::k2_ch4}}
\pysigstartsignatures
\pysigline{\sphinxcode{\sphinxupquote{JULES\_SOIL\_BIOGEOCHEM::}}\sphinxbfcode{\sphinxupquote{k2\_ch4}}}
\pysigstopsignatures\begin{quote}\begin{description}
\sphinxlineitem{Type}
\sphinxAtStartPar
real

\sphinxlineitem{Default}
\sphinxAtStartPar
0.01

\end{description}\end{quote}

\sphinxAtStartPar
Baseline methanogenic respiration rate (hr\sphinxhyphen{}1).

\end{fulllineitems}

\index{kd\_ch4 (in namelist JULES\_SOIL\_BIOGEOCHEM)@\spxentry{kd\_ch4}\spxextra{in namelist JULES\_SOIL\_BIOGEOCHEM}|spxpagem}

\begin{fulllineitems}
\phantomsection\label{\detokenize{namelists/jules_soil_biogeochem.nml:JULES_SOIL_BIOGEOCHEM::kd_ch4}}
\pysigstartsignatures
\pysigline{\sphinxcode{\sphinxupquote{JULES\_SOIL\_BIOGEOCHEM::}}\sphinxbfcode{\sphinxupquote{kd\_ch4}}}
\pysigstopsignatures\begin{quote}\begin{description}
\sphinxlineitem{Type}
\sphinxAtStartPar
real

\sphinxlineitem{Default}
\sphinxAtStartPar
0.0003

\end{description}\end{quote}

\sphinxAtStartPar
Baseline methanogenic mortality rate (hr\sphinxhyphen{}1).

\end{fulllineitems}

\index{rho\_ch4 (in namelist JULES\_SOIL\_BIOGEOCHEM)@\spxentry{rho\_ch4}\spxextra{in namelist JULES\_SOIL\_BIOGEOCHEM}|spxpagem}

\begin{fulllineitems}
\phantomsection\label{\detokenize{namelists/jules_soil_biogeochem.nml:JULES_SOIL_BIOGEOCHEM::rho_ch4}}
\pysigstartsignatures
\pysigline{\sphinxcode{\sphinxupquote{JULES\_SOIL\_BIOGEOCHEM::}}\sphinxbfcode{\sphinxupquote{rho\_ch4}}}
\pysigstopsignatures\begin{quote}\begin{description}
\sphinxlineitem{Type}
\sphinxAtStartPar
real

\sphinxlineitem{Default}
\sphinxAtStartPar
47.0

\end{description}\end{quote}

\sphinxAtStartPar
Factor in substrate limitation function (related to half saturation of substrate for methanogenic respiration) ( (mgC/m3)\sphinxhyphen{}1 ).

\end{fulllineitems}

\index{q10\_mic\_ch4 (in namelist JULES\_SOIL\_BIOGEOCHEM)@\spxentry{q10\_mic\_ch4}\spxextra{in namelist JULES\_SOIL\_BIOGEOCHEM}|spxpagem}

\begin{fulllineitems}
\phantomsection\label{\detokenize{namelists/jules_soil_biogeochem.nml:JULES_SOIL_BIOGEOCHEM::q10_mic_ch4}}
\pysigstartsignatures
\pysigline{\sphinxcode{\sphinxupquote{JULES\_SOIL\_BIOGEOCHEM::}}\sphinxbfcode{\sphinxupquote{q10\_mic\_ch4}}}
\pysigstopsignatures\begin{quote}\begin{description}
\sphinxlineitem{Type}
\sphinxAtStartPar
real

\sphinxlineitem{Default}
\sphinxAtStartPar
4.3

\end{description}\end{quote}

\sphinxAtStartPar
Q10 factor for methanogens.

\end{fulllineitems}

\index{cue\_ch4 (in namelist JULES\_SOIL\_BIOGEOCHEM)@\spxentry{cue\_ch4}\spxextra{in namelist JULES\_SOIL\_BIOGEOCHEM}|spxpagem}

\begin{fulllineitems}
\phantomsection\label{\detokenize{namelists/jules_soil_biogeochem.nml:JULES_SOIL_BIOGEOCHEM::cue_ch4}}
\pysigstartsignatures
\pysigline{\sphinxcode{\sphinxupquote{JULES\_SOIL\_BIOGEOCHEM::}}\sphinxbfcode{\sphinxupquote{cue\_ch4}}}
\pysigstopsignatures\begin{quote}\begin{description}
\sphinxlineitem{Type}
\sphinxAtStartPar
real

\sphinxlineitem{Default}
\sphinxAtStartPar
0.03

\end{description}\end{quote}

\sphinxAtStartPar
Carbon use efficiency of methanogenic growth.

\end{fulllineitems}

\index{mu\_ch4 (in namelist JULES\_SOIL\_BIOGEOCHEM)@\spxentry{mu\_ch4}\spxextra{in namelist JULES\_SOIL\_BIOGEOCHEM}|spxpagem}

\begin{fulllineitems}
\phantomsection\label{\detokenize{namelists/jules_soil_biogeochem.nml:JULES_SOIL_BIOGEOCHEM::mu_ch4}}
\pysigstartsignatures
\pysigline{\sphinxcode{\sphinxupquote{JULES\_SOIL\_BIOGEOCHEM::}}\sphinxbfcode{\sphinxupquote{mu\_ch4}}}
\pysigstopsignatures\begin{quote}\begin{description}
\sphinxlineitem{Type}
\sphinxAtStartPar
real

\sphinxlineitem{Default}
\sphinxAtStartPar
0.00042

\end{description}\end{quote}

\sphinxAtStartPar
Threshold growth rate below which methanogens die (hr\sphinxhyphen{}1).

\end{fulllineitems}

\index{frz\_ch4 (in namelist JULES\_SOIL\_BIOGEOCHEM)@\spxentry{frz\_ch4}\spxextra{in namelist JULES\_SOIL\_BIOGEOCHEM}|spxpagem}

\begin{fulllineitems}
\phantomsection\label{\detokenize{namelists/jules_soil_biogeochem.nml:JULES_SOIL_BIOGEOCHEM::frz_ch4}}
\pysigstartsignatures
\pysigline{\sphinxcode{\sphinxupquote{JULES\_SOIL\_BIOGEOCHEM::}}\sphinxbfcode{\sphinxupquote{frz\_ch4}}}
\pysigstopsignatures\begin{quote}\begin{description}
\sphinxlineitem{Type}
\sphinxAtStartPar
real

\sphinxlineitem{Default}
\sphinxAtStartPar
0.5

\end{description}\end{quote}

\sphinxAtStartPar
Factor to reduce CH4 substrate production when soil is sufficiently frozen (only in microbial scheme).

\end{fulllineitems}

\index{alpha\_ch4 (in namelist JULES\_SOIL\_BIOGEOCHEM)@\spxentry{alpha\_ch4}\spxextra{in namelist JULES\_SOIL\_BIOGEOCHEM}|spxpagem}

\begin{fulllineitems}
\phantomsection\label{\detokenize{namelists/jules_soil_biogeochem.nml:JULES_SOIL_BIOGEOCHEM::alpha_ch4}}
\pysigstartsignatures
\pysigline{\sphinxcode{\sphinxupquote{JULES\_SOIL\_BIOGEOCHEM::}}\sphinxbfcode{\sphinxupquote{alpha\_ch4}}}
\pysigstopsignatures\begin{quote}\begin{description}
\sphinxlineitem{Type}
\sphinxAtStartPar
real

\sphinxlineitem{Default}
\sphinxAtStartPar
0.001

\end{description}\end{quote}

\sphinxAtStartPar
Ratio between maintenance and growth respiration rates for methanogens.

\end{fulllineitems}

\index{ev\_ch4 (in namelist JULES\_SOIL\_BIOGEOCHEM)@\spxentry{ev\_ch4}\spxextra{in namelist JULES\_SOIL\_BIOGEOCHEM}|spxpagem}

\begin{fulllineitems}
\phantomsection\label{\detokenize{namelists/jules_soil_biogeochem.nml:JULES_SOIL_BIOGEOCHEM::ev_ch4}}
\pysigstartsignatures
\pysigline{\sphinxcode{\sphinxupquote{JULES\_SOIL\_BIOGEOCHEM::}}\sphinxbfcode{\sphinxupquote{ev\_ch4}}}
\pysigstopsignatures\begin{quote}\begin{description}
\sphinxlineitem{Type}
\sphinxAtStartPar
real

\sphinxlineitem{Default}
\sphinxAtStartPar
5.0

\end{description}\end{quote}

\sphinxAtStartPar
Timescale over which methanogenic traits adapt to temperature change (yr)

\end{fulllineitems}

\index{q10\_ev\_ch4 (in namelist JULES\_SOIL\_BIOGEOCHEM)@\spxentry{q10\_ev\_ch4}\spxextra{in namelist JULES\_SOIL\_BIOGEOCHEM}|spxpagem}

\begin{fulllineitems}
\phantomsection\label{\detokenize{namelists/jules_soil_biogeochem.nml:JULES_SOIL_BIOGEOCHEM::q10_ev_ch4}}
\pysigstartsignatures
\pysigline{\sphinxcode{\sphinxupquote{JULES\_SOIL\_BIOGEOCHEM::}}\sphinxbfcode{\sphinxupquote{q10\_ev\_ch4}}}
\pysigstopsignatures\begin{quote}\begin{description}
\sphinxlineitem{Type}
\sphinxAtStartPar
real

\sphinxlineitem{Default}
\sphinxAtStartPar
2.2

\end{description}\end{quote}

\sphinxAtStartPar
Q10 for temperature response of methanogenic traits under adaptation

\end{fulllineitems}

\end{sphinxadmonition}


\sphinxstrong{See also:}
\nopagebreak

\begin{quote}

\sphinxAtStartPar
Reference for microbial methane scheme:
\end{quote}
\begin{itemize}
\item {} 
\sphinxAtStartPar
Chadburn, S. E. et al (2020),
Modeled Microbial Dynamics Explain the Apparent Temperature Sensitivity
of Wetland Methane Emissions. Global Biogeochemical Cycles, 34:
e2020GB006678. \sphinxurl{https://doi.org/10.1029/2020GB006678}

\end{itemize}



\sphinxstepscope


\section{\sphinxstyleliteralintitle{\sphinxupquote{jules\_soil\_ecosse.nml}}}
\label{\detokenize{namelists/jules_soil_ecosse.nml:jules-soil-ecosse-nml}}\label{\detokenize{namelists/jules_soil_ecosse.nml:jules-soil-ecosse-namelist}}\label{\detokenize{namelists/jules_soil_ecosse.nml::doc}}
\sphinxAtStartPar
This file sets options and parameters for the ECOSSE model of soil biogeochemistry. It is used only if the ECOSSE model is chosen ({\hyperref[\detokenize{namelists/jules_soil_biogeochem.nml:JULES_SOIL_BIOGEOCHEM::soil_bgc_model}]{\sphinxcrossref{\sphinxcode{\sphinxupquote{soil\_bgc\_model}}}}} = 3).

\begin{sphinxadmonition}{warning}{Warning:}
\sphinxAtStartPar
The ECOSSE model in JULES is still in development and is not fully functional in this version. The code is included to allow further development. Users should not try to use ECOSSE.
\end{sphinxadmonition}


\sphinxstrong{See also:}
\nopagebreak


\sphinxAtStartPar
References:
\begin{itemize}
\item {} 
\sphinxAtStartPar
Smith et al., 2010, Estimating changes in Scottish soil carbon stocks using ECOSSE. I. Model description and uncertainties, Climate Research, 45: 179\sphinxhyphen{}192. (\sphinxurl{https://doi.org/10.3354/cr00899}).

\item {} 
\sphinxAtStartPar
Clark, D. B., Mercado, L. M., Sitch, S., Jones, C. D., Gedney, N., Best, M. J., Pryor, M., Rooney, G. G., Essery, R. L. H., Blyth, E., Boucher, O., Harding, R. J., Huntingford, C., and Cox, P. M.: The Joint UK Land Environment Simulator (JULES), model description \textendash{} Part 2: Carbon fluxes and vegetation dynamics, Geosci. Model Dev., 4, 701\textendash{}722, (\sphinxurl{https://doi.org/10.5194/gmd-4-701-2011}), 2011.

\end{itemize}




\subsection{\sphinxstyleliteralintitle{\sphinxupquote{JULES\_SOIL\_ECOSSE}} namelist members}
\label{\detokenize{namelists/jules_soil_ecosse.nml:namelist-JULES_SOIL_ECOSSE}}\label{\detokenize{namelists/jules_soil_ecosse.nml:jules-soil-ecosse-namelist-members}}\index{JULES\_SOIL\_ECOSSE (namelist)@\spxentry{JULES\_SOIL\_ECOSSE}\spxextra{namelist}|spxpagem}\index{l\_soil\_n (in namelist JULES\_SOIL\_ECOSSE)@\spxentry{l\_soil\_n}\spxextra{in namelist JULES\_SOIL\_ECOSSE}|spxpagem}

\begin{fulllineitems}
\phantomsection\label{\detokenize{namelists/jules_soil_ecosse.nml:JULES_SOIL_ECOSSE::l_soil_n}}
\pysigstartsignatures
\pysigline{\sphinxcode{\sphinxupquote{JULES\_SOIL\_ECOSSE::}}\sphinxbfcode{\sphinxupquote{l\_soil\_n}}}
\pysigstopsignatures\begin{quote}\begin{description}
\sphinxlineitem{Type}
\sphinxAtStartPar
logical

\sphinxlineitem{Default}
\sphinxAtStartPar
T

\end{description}\end{quote}

\sphinxAtStartPar
Switch to include soil nitrogen in ECOSSE.
\begin{description}
\sphinxlineitem{TRUE}
\sphinxAtStartPar
Model soil carbon and nitrogen.

\sphinxlineitem{FALSE}
\sphinxAtStartPar
Model only soil carbon.

\end{description}

\end{fulllineitems}

\index{l\_match\_layers (in namelist JULES\_SOIL\_ECOSSE)@\spxentry{l\_match\_layers}\spxextra{in namelist JULES\_SOIL\_ECOSSE}|spxpagem}

\begin{fulllineitems}
\phantomsection\label{\detokenize{namelists/jules_soil_ecosse.nml:JULES_SOIL_ECOSSE::l_match_layers}}
\pysigstartsignatures
\pysigline{\sphinxcode{\sphinxupquote{JULES\_SOIL\_ECOSSE::}}\sphinxbfcode{\sphinxupquote{l\_match\_layers}}}
\pysigstopsignatures\begin{quote}\begin{description}
\sphinxlineitem{Type}
\sphinxAtStartPar
logical

\sphinxlineitem{Default}
\sphinxAtStartPar
T

\end{description}\end{quote}

\sphinxAtStartPar
Switch to match ECOSSE soil C and N layers to those for soil moisture.
\begin{description}
\sphinxlineitem{TRUE}
\sphinxAtStartPar
Use the same layering as for soil moisture. The number of soil carbon layers will equal {\hyperref[\detokenize{namelists/jules_soil.nml:JULES_SOIL::sm_levels}]{\sphinxcrossref{\sphinxcode{\sphinxupquote{sm\_levels}}}}}, with layer thicknesses given by {\hyperref[\detokenize{namelists/jules_soil.nml:JULES_SOIL::dzsoil_io}]{\sphinxcrossref{\sphinxcode{\sphinxupquote{dzsoil\_io}}}}}.

\sphinxlineitem{FALSE}
\sphinxAtStartPar
The number of layers will be specified by {\hyperref[\detokenize{namelists/jules_soil_ecosse.nml:JULES_SOIL_ECOSSE::dim_cslayer}]{\sphinxcrossref{\sphinxcode{\sphinxupquote{dim\_cslayer}}}}} and the thicknesses by {\hyperref[\detokenize{namelists/jules_soil_ecosse.nml:JULES_SOIL_ECOSSE::dz_soilc_io}]{\sphinxcrossref{\sphinxcode{\sphinxupquote{dz\_soilc\_io}}}}}.

\end{description}

\end{fulllineitems}

\index{dim\_cslayer (in namelist JULES\_SOIL\_ECOSSE)@\spxentry{dim\_cslayer}\spxextra{in namelist JULES\_SOIL\_ECOSSE}|spxpagem}

\begin{fulllineitems}
\phantomsection\label{\detokenize{namelists/jules_soil_ecosse.nml:JULES_SOIL_ECOSSE::dim_cslayer}}
\pysigstartsignatures
\pysigline{\sphinxcode{\sphinxupquote{JULES\_SOIL\_ECOSSE::}}\sphinxbfcode{\sphinxupquote{dim\_cslayer}}}
\pysigstopsignatures\begin{quote}\begin{description}
\sphinxlineitem{Type}
\sphinxAtStartPar
integer

\sphinxlineitem{Permitted}
\sphinxAtStartPar
\textgreater{}= 1

\sphinxlineitem{Default}
\sphinxAtStartPar
None

\end{description}\end{quote}

\sphinxAtStartPar
The number of ECOSSE soil carbon layers. Not used if {\hyperref[\detokenize{namelists/jules_soil_ecosse.nml:JULES_SOIL_ECOSSE::l_match_layers}]{\sphinxcrossref{\sphinxcode{\sphinxupquote{l\_match\_layers}}}}} = TRUE. Despite the similar name, this parameter is unrelated to {\hyperref[\detokenize{namelists/model_grid.nml:JULES_INPUT_GRID::sclayer_dim_name}]{\sphinxcrossref{\sphinxcode{\sphinxupquote{sclayer\_dim\_name}}}}}.

\end{fulllineitems}

\index{dz\_soilc\_io (in namelist JULES\_SOIL\_ECOSSE)@\spxentry{dz\_soilc\_io}\spxextra{in namelist JULES\_SOIL\_ECOSSE}|spxpagem}

\begin{fulllineitems}
\phantomsection\label{\detokenize{namelists/jules_soil_ecosse.nml:JULES_SOIL_ECOSSE::dz_soilc_io}}
\pysigstartsignatures
\pysigline{\sphinxcode{\sphinxupquote{JULES\_SOIL\_ECOSSE::}}\sphinxbfcode{\sphinxupquote{dz\_soilc\_io}}}
\pysigstopsignatures\begin{quote}\begin{description}
\sphinxlineitem{Type}
\sphinxAtStartPar
real(dim\_cslayer)

\sphinxlineitem{Permitted}
\sphinxAtStartPar
\textgreater{} 0

\sphinxlineitem{Default}
\sphinxAtStartPar
None

\end{description}\end{quote}

\sphinxAtStartPar
Thicknesses of the ECOSSE soil carbon layers (m). Not used if {\hyperref[\detokenize{namelists/jules_soil_ecosse.nml:JULES_SOIL_ECOSSE::l_match_layers}]{\sphinxcrossref{\sphinxcode{\sphinxupquote{l\_match\_layers}}}}} = TRUE. In most cases the total depth must equal that of the soil moisture layers (see {\hyperref[\detokenize{namelists/jules_soil.nml:JULES_SOIL::dzsoil_io}]{\sphinxcrossref{\sphinxcode{\sphinxupquote{dzsoil\_io}}}}}). The exception is the case of a single layer, bulk model ({\hyperref[\detokenize{namelists/jules_soil_ecosse.nml:JULES_SOIL_ECOSSE::dim_cslayer}]{\sphinxcrossref{\sphinxcode{\sphinxupquote{dim\_cslayer}}}}} = 1), for which dz\_soilc(1) is interpreted as the representative or averaging depth for the bulk layer (e.g. the temperature of the bulk model is taken as the average over this depth).

\end{fulllineitems}

\index{dt\_soilc (in namelist JULES\_SOIL\_ECOSSE)@\spxentry{dt\_soilc}\spxextra{in namelist JULES\_SOIL\_ECOSSE}|spxpagem}

\begin{fulllineitems}
\phantomsection\label{\detokenize{namelists/jules_soil_ecosse.nml:JULES_SOIL_ECOSSE::dt_soilc}}
\pysigstartsignatures
\pysigline{\sphinxcode{\sphinxupquote{JULES\_SOIL\_ECOSSE::}}\sphinxbfcode{\sphinxupquote{dt\_soilc}}}
\pysigstopsignatures\begin{quote}\begin{description}
\sphinxlineitem{Type}
\sphinxAtStartPar
REAL

\sphinxlineitem{Permitted}
\sphinxAtStartPar
\textgreater{}= {\hyperref[\detokenize{namelists/timesteps.nml:JULES_TIME::timestep_len}]{\sphinxcrossref{\sphinxcode{\sphinxupquote{timestep\_len}}}}}

\sphinxlineitem{Default}
\sphinxAtStartPar
{\hyperref[\detokenize{namelists/timesteps.nml:JULES_TIME::timestep_len}]{\sphinxcrossref{\sphinxcode{\sphinxupquote{timestep\_len}}}}}

\end{description}\end{quote}

\sphinxAtStartPar
The timestep length for ECOSSE (seconds). The main JULES timestep length {\hyperref[\detokenize{namelists/timesteps.nml:JULES_TIME::timestep_len}]{\sphinxcrossref{\sphinxcode{\sphinxupquote{timestep\_len}}}}} must be a multiple of this timestep length, so that ECOSSE is called on JULES timesteps. The TRIFFID timestep length {\hyperref[\detokenize{namelists/jules_vegetation.nml:JULES_VEGETATION::triffid_period}]{\sphinxcrossref{\sphinxcode{\sphinxupquote{triffid\_period}}}}} (converted to seconds) must also be a multiple of the ECOSSE timestep length.

\end{fulllineitems}

\index{l\_driver\_ave (in namelist JULES\_SOIL\_ECOSSE)@\spxentry{l\_driver\_ave}\spxextra{in namelist JULES\_SOIL\_ECOSSE}|spxpagem}

\begin{fulllineitems}
\phantomsection\label{\detokenize{namelists/jules_soil_ecosse.nml:JULES_SOIL_ECOSSE::l_driver_ave}}
\pysigstartsignatures
\pysigline{\sphinxcode{\sphinxupquote{JULES\_SOIL\_ECOSSE::}}\sphinxbfcode{\sphinxupquote{l\_driver\_ave}}}
\pysigstopsignatures\begin{quote}\begin{description}
\sphinxlineitem{Type}
\sphinxAtStartPar
logical

\sphinxlineitem{Default}
\sphinxAtStartPar
T

\end{description}\end{quote}

\sphinxAtStartPar
Switch controlling the averaging of the physical driving variables that are input to ECOSSE (e.g. soil temperature).
\begin{description}
\sphinxlineitem{TRUE}
\sphinxAtStartPar
Average the driving variables over the ECOSSE timestep length.

\sphinxlineitem{FALSE}
\sphinxAtStartPar
Use instantaneous values of the driving variables at the time when ECOSSE is called.

\end{description}

\end{fulllineitems}

\index{plant\_input\_profile (in namelist JULES\_SOIL\_ECOSSE)@\spxentry{plant\_input\_profile}\spxextra{in namelist JULES\_SOIL\_ECOSSE}|spxpagem}

\begin{fulllineitems}
\phantomsection\label{\detokenize{namelists/jules_soil_ecosse.nml:JULES_SOIL_ECOSSE::plant_input_profile}}
\pysigstartsignatures
\pysigline{\sphinxcode{\sphinxupquote{JULES\_SOIL\_ECOSSE::}}\sphinxbfcode{\sphinxupquote{plant\_input\_profile}}}
\pysigstopsignatures\begin{quote}\begin{description}
\sphinxlineitem{Type}
\sphinxAtStartPar
INTEGER

\sphinxlineitem{Permitted}
\sphinxAtStartPar
1 or 2

\sphinxlineitem{Default}
\sphinxAtStartPar
1

\end{description}\end{quote}

\sphinxAtStartPar
Switch for the vertical distribution of litterfall inputs of C and N to the soil.

\sphinxAtStartPar
Possible values are:
\begin{enumerate}
\sphinxsetlistlabels{\arabic}{enumi}{enumii}{}{.}%
\item {} 
\begin{DUlineblock}{0em}
\item[] Fraction {\hyperref[\detokenize{namelists/jules_soil_ecosse.nml:JULES_SOIL_ECOSSE::pi_sfc_frac}]{\sphinxcrossref{\sphinxcode{\sphinxupquote{pi\_sfc\_frac}}}}} of inputs are distributed uniformly in a surface layer of depth {\hyperref[\detokenize{namelists/jules_soil_ecosse.nml:JULES_SOIL_ECOSSE::pi_sfc_depth}]{\sphinxcrossref{\sphinxcode{\sphinxupquote{pi\_sfc\_depth}}}}}, the remainder are distributed according to the distribution of roots.
\end{DUlineblock}

\item {} 
\begin{DUlineblock}{0em}
\item[] Inputs decrease exponentially with depth with decay constant {\hyperref[\detokenize{namelists/jules_soil_biogeochem.nml:JULES_SOIL_BIOGEOCHEM::tau_lit}]{\sphinxcrossref{\sphinxcode{\sphinxupquote{tau\_lit}}}}}.
\end{DUlineblock}

\end{enumerate}

\end{fulllineitems}

\index{pi\_sfc\_frac (in namelist JULES\_SOIL\_ECOSSE)@\spxentry{pi\_sfc\_frac}\spxextra{in namelist JULES\_SOIL\_ECOSSE}|spxpagem}

\begin{fulllineitems}
\phantomsection\label{\detokenize{namelists/jules_soil_ecosse.nml:JULES_SOIL_ECOSSE::pi_sfc_frac}}
\pysigstartsignatures
\pysigline{\sphinxcode{\sphinxupquote{JULES\_SOIL\_ECOSSE::}}\sphinxbfcode{\sphinxupquote{pi\_sfc\_frac}}}
\pysigstopsignatures\begin{quote}\begin{description}
\sphinxlineitem{Type}
\sphinxAtStartPar
REAL

\sphinxlineitem{Default}
\sphinxAtStartPar
0.3

\end{description}\end{quote}

\sphinxAtStartPar
Fraction of plant litterfall that is added to the surface soil layer of depth {\hyperref[\detokenize{namelists/jules_soil_ecosse.nml:JULES_SOIL_ECOSSE::pi_sfc_depth}]{\sphinxcrossref{\sphinxcode{\sphinxupquote{pi\_sfc\_depth}}}}}. Only used if {\hyperref[\detokenize{namelists/jules_soil_ecosse.nml:JULES_SOIL_ECOSSE::plant_input_profile}]{\sphinxcrossref{\sphinxcode{\sphinxupquote{plant\_input\_profile}}}}} = 1.

\end{fulllineitems}

\index{pi\_sfc\_depth (in namelist JULES\_SOIL\_ECOSSE)@\spxentry{pi\_sfc\_depth}\spxextra{in namelist JULES\_SOIL\_ECOSSE}|spxpagem}

\begin{fulllineitems}
\phantomsection\label{\detokenize{namelists/jules_soil_ecosse.nml:JULES_SOIL_ECOSSE::pi_sfc_depth}}
\pysigstartsignatures
\pysigline{\sphinxcode{\sphinxupquote{JULES\_SOIL\_ECOSSE::}}\sphinxbfcode{\sphinxupquote{pi\_sfc\_depth}}}
\pysigstopsignatures\begin{quote}\begin{description}
\sphinxlineitem{Type}
\sphinxAtStartPar
REAL

\sphinxlineitem{Default}
\sphinxAtStartPar
0.1

\end{description}\end{quote}

\sphinxAtStartPar
Depth of soil over which fraction {\hyperref[\detokenize{namelists/jules_soil_ecosse.nml:JULES_SOIL_ECOSSE::pi_sfc_frac}]{\sphinxcrossref{\sphinxcode{\sphinxupquote{pi\_sfc\_frac}}}}} of plant litterfall is added (m). Only used if {\hyperref[\detokenize{namelists/jules_soil_ecosse.nml:JULES_SOIL_ECOSSE::plant_input_profile}]{\sphinxcrossref{\sphinxcode{\sphinxupquote{plant\_input\_profile}}}}} = 1.

\end{fulllineitems}

\index{temp\_modifier (in namelist JULES\_SOIL\_ECOSSE)@\spxentry{temp\_modifier}\spxextra{in namelist JULES\_SOIL\_ECOSSE}|spxpagem}

\begin{fulllineitems}
\phantomsection\label{\detokenize{namelists/jules_soil_ecosse.nml:JULES_SOIL_ECOSSE::temp_modifier}}
\pysigstartsignatures
\pysigline{\sphinxcode{\sphinxupquote{JULES\_SOIL\_ECOSSE::}}\sphinxbfcode{\sphinxupquote{temp\_modifier}}}
\pysigstopsignatures\begin{quote}\begin{description}
\sphinxlineitem{Type}
\sphinxAtStartPar
INTEGER

\sphinxlineitem{Permitted}
\sphinxAtStartPar
1 or 2

\sphinxlineitem{Default}
\sphinxAtStartPar
2

\end{description}\end{quote}

\sphinxAtStartPar
Switch for the form of the temperature rate modifier for decomposition.
\begin{enumerate}
\sphinxsetlistlabels{\arabic}{enumi}{enumii}{}{.}%
\item {} 
\begin{DUlineblock}{0em}
\item[] Use a Q10 approach (Eqn. 65 of Clark et al. (2011))
\end{DUlineblock}

\item {} 
\begin{DUlineblock}{0em}
\item[] Use the Smith et al. (2010) form of the modifier (Eqn.1 of Smith et al. (2010))
\end{DUlineblock}

\end{enumerate}

\end{fulllineitems}

\index{water\_modifier (in namelist JULES\_SOIL\_ECOSSE)@\spxentry{water\_modifier}\spxextra{in namelist JULES\_SOIL\_ECOSSE}|spxpagem}

\begin{fulllineitems}
\phantomsection\label{\detokenize{namelists/jules_soil_ecosse.nml:JULES_SOIL_ECOSSE::water_modifier}}
\pysigstartsignatures
\pysigline{\sphinxcode{\sphinxupquote{JULES\_SOIL\_ECOSSE::}}\sphinxbfcode{\sphinxupquote{water\_modifier}}}
\pysigstopsignatures\begin{quote}
\begin{quote}\begin{description}
\sphinxlineitem{type}
\sphinxAtStartPar
INTEGER

\sphinxlineitem{permitted}
\sphinxAtStartPar
1 or 2

\sphinxlineitem{default}
\sphinxAtStartPar
2

\end{description}\end{quote}

\sphinxAtStartPar
Switch for the form of the water rate modifier for decomposition and nitrification.
\end{quote}
\begin{enumerate}
\sphinxsetlistlabels{\arabic}{enumi}{enumii}{}{.}%
\item {} 
\begin{DUlineblock}{0em}
\item[] Use the Clark et al. (2011) form of the modifier (Eqn. 67 of Clark et al. (2011)). Note however that only the unfrozen water is considered here.
\end{DUlineblock}

\item {} 
\begin{DUlineblock}{0em}
\item[] Use the Smith et al. (2010) form of the modifier.
\end{DUlineblock}

\end{enumerate}

\end{fulllineitems}

\index{decomp\_rate (in namelist JULES\_SOIL\_ECOSSE)@\spxentry{decomp\_rate}\spxextra{in namelist JULES\_SOIL\_ECOSSE}|spxpagem}

\begin{fulllineitems}
\phantomsection\label{\detokenize{namelists/jules_soil_ecosse.nml:JULES_SOIL_ECOSSE::decomp_rate}}
\pysigstartsignatures
\pysigline{\sphinxcode{\sphinxupquote{JULES\_SOIL\_ECOSSE::}}\sphinxbfcode{\sphinxupquote{decomp\_rate}}}
\pysigstopsignatures\begin{quote}\begin{description}
\sphinxlineitem{Type}
\sphinxAtStartPar
REAL(4)

\sphinxlineitem{Default}
\sphinxAtStartPar
3.22e\sphinxhyphen{}7, 9.65e\sphinxhyphen{}9, 2.12e\sphinxhyphen{}8, 6.43e\sphinxhyphen{}10

\end{description}\end{quote}

\sphinxAtStartPar
Rate constant for decomposition of each soil carbon pool (s$^{\text{\sphinxhyphen{}1}}$).

\sphinxAtStartPar
Note that these default values are also those for use with the 4\sphinxhyphen{}pool model ({\hyperref[\detokenize{namelists/jules_soil_biogeochem.nml:JULES_SOIL_BIOGEOCHEM::kaps_4pool}]{\sphinxcrossref{\sphinxcode{\sphinxupquote{kaps\_4pool}}}}}).

\end{fulllineitems}

\index{decomp\_ph\_rate\_min (in namelist JULES\_SOIL\_ECOSSE)@\spxentry{decomp\_ph\_rate\_min}\spxextra{in namelist JULES\_SOIL\_ECOSSE}|spxpagem}

\begin{fulllineitems}
\phantomsection\label{\detokenize{namelists/jules_soil_ecosse.nml:JULES_SOIL_ECOSSE::decomp_ph_rate_min}}
\pysigstartsignatures
\pysigline{\sphinxcode{\sphinxupquote{JULES\_SOIL\_ECOSSE::}}\sphinxbfcode{\sphinxupquote{decomp\_ph\_rate\_min}}}
\pysigstopsignatures\begin{quote}\begin{description}
\sphinxlineitem{Type}
\sphinxAtStartPar
REAL

\sphinxlineitem{Default}
\sphinxAtStartPar
0.2

\end{description}\end{quote}

\sphinxAtStartPar
Minimum allowed value of pH rate modifier for decomposition.

\end{fulllineitems}

\index{decomp\_ph\_min (in namelist JULES\_SOIL\_ECOSSE)@\spxentry{decomp\_ph\_min}\spxextra{in namelist JULES\_SOIL\_ECOSSE}|spxpagem}

\begin{fulllineitems}
\phantomsection\label{\detokenize{namelists/jules_soil_ecosse.nml:JULES_SOIL_ECOSSE::decomp_ph_min}}
\pysigstartsignatures
\pysigline{\sphinxcode{\sphinxupquote{JULES\_SOIL\_ECOSSE::}}\sphinxbfcode{\sphinxupquote{decomp\_ph\_min}}}
\pysigstopsignatures\begin{quote}\begin{description}
\sphinxlineitem{Type}
\sphinxAtStartPar
REAL

\sphinxlineitem{Permitted}
\sphinxAtStartPar
decomp\_ph\_min \textless{}= decomp\_ph\_max

\sphinxlineitem{Default}
\sphinxAtStartPar
1.0

\end{description}\end{quote}

\sphinxAtStartPar
Soil pH below which rate of decomposition is minimum.

\end{fulllineitems}

\index{decomp\_ph\_max (in namelist JULES\_SOIL\_ECOSSE)@\spxentry{decomp\_ph\_max}\spxextra{in namelist JULES\_SOIL\_ECOSSE}|spxpagem}

\begin{fulllineitems}
\phantomsection\label{\detokenize{namelists/jules_soil_ecosse.nml:JULES_SOIL_ECOSSE::decomp_ph_max}}
\pysigstartsignatures
\pysigline{\sphinxcode{\sphinxupquote{JULES\_SOIL\_ECOSSE::}}\sphinxbfcode{\sphinxupquote{decomp\_ph\_max}}}
\pysigstopsignatures\begin{quote}\begin{description}
\sphinxlineitem{Type}
\sphinxAtStartPar
REAL

\sphinxlineitem{Permitted}
\sphinxAtStartPar
decomp\_ph\_max \textgreater{}= decomp\_ph\_min

\sphinxlineitem{Default}
\sphinxAtStartPar
4.5

\end{description}\end{quote}

\sphinxAtStartPar
Soil pH above which rate of decomposition is maximum.

\end{fulllineitems}

\index{decomp\_temp\_coeff\_smith (in namelist JULES\_SOIL\_ECOSSE)@\spxentry{decomp\_temp\_coeff\_smith}\spxextra{in namelist JULES\_SOIL\_ECOSSE}|spxpagem}

\begin{fulllineitems}
\phantomsection\label{\detokenize{namelists/jules_soil_ecosse.nml:JULES_SOIL_ECOSSE::decomp_temp_coeff_smith}}
\pysigstartsignatures
\pysigline{\sphinxcode{\sphinxupquote{JULES\_SOIL\_ECOSSE::}}\sphinxbfcode{\sphinxupquote{decomp\_temp\_coeff\_smith}}}
\pysigstopsignatures\begin{quote}\begin{description}
\sphinxlineitem{Type}
\sphinxAtStartPar
REAL(3)

\sphinxlineitem{Default}
\sphinxAtStartPar
47.9, 106.0, 18.3

\end{description}\end{quote}

\sphinxAtStartPar
Constants in the 4\sphinxhyphen{}pool from of the decomposition temperature modifier ({\hyperref[\detokenize{namelists/jules_soil_ecosse.nml:JULES_SOIL_ECOSSE::temp_modifier}]{\sphinxcrossref{\sphinxcode{\sphinxupquote{temp\_modifier}}}}} = 2).

\sphinxAtStartPar
Note that these default values are also those harwired in the code for use with the 4\sphinxhyphen{}pool model ({\hyperref[\detokenize{namelists/jules_soil_biogeochem.nml:JULES_SOIL_BIOGEOCHEM::soil_bgc_model}]{\sphinxcrossref{\sphinxcode{\sphinxupquote{soil\_bgc\_model}}}}} = 2, {\hyperref[\detokenize{namelists/jules_soil_biogeochem.nml:JULES_SOIL_BIOGEOCHEM::l_q10}]{\sphinxcrossref{\sphinxcode{\sphinxupquote{l\_q10}}}}} = F ).

\end{fulllineitems}

\index{decomp\_wrate\_min\_smith (in namelist JULES\_SOIL\_ECOSSE)@\spxentry{decomp\_wrate\_min\_smith}\spxextra{in namelist JULES\_SOIL\_ECOSSE}|spxpagem}

\begin{fulllineitems}
\phantomsection\label{\detokenize{namelists/jules_soil_ecosse.nml:JULES_SOIL_ECOSSE::decomp_wrate_min_smith}}
\pysigstartsignatures
\pysigline{\sphinxcode{\sphinxupquote{JULES\_SOIL\_ECOSSE::}}\sphinxbfcode{\sphinxupquote{decomp\_wrate\_min\_smith}}}
\pysigstopsignatures\begin{quote}\begin{description}
\sphinxlineitem{Type}
\sphinxAtStartPar
REAL

\sphinxlineitem{Default}
\sphinxAtStartPar
0.2

\end{description}\end{quote}

\sphinxAtStartPar
Minimum allowed value of the water rate modifier for decomposition when the 4\sphinxhyphen{}pool form is used ({\hyperref[\detokenize{namelists/jules_soil_ecosse.nml:JULES_SOIL_ECOSSE::water_modifier}]{\sphinxcrossref{\sphinxcode{\sphinxupquote{water\_modifier}}}}} = 2).

\end{fulllineitems}

\index{decomp\_wrate\_min\_clark (in namelist JULES\_SOIL\_ECOSSE)@\spxentry{decomp\_wrate\_min\_clark}\spxextra{in namelist JULES\_SOIL\_ECOSSE}|spxpagem}

\begin{fulllineitems}
\phantomsection\label{\detokenize{namelists/jules_soil_ecosse.nml:JULES_SOIL_ECOSSE::decomp_wrate_min_clark}}
\pysigstartsignatures
\pysigline{\sphinxcode{\sphinxupquote{JULES\_SOIL\_ECOSSE::}}\sphinxbfcode{\sphinxupquote{decomp\_wrate\_min\_clark}}}
\pysigstopsignatures\begin{quote}\begin{description}
\sphinxlineitem{Type}
\sphinxAtStartPar
REAL

\sphinxlineitem{Default}
\sphinxAtStartPar
0.2

\end{description}\end{quote}

\sphinxAtStartPar
Minimum allowed value of the water rate modifier for decomposition when the Clark et al. (2011) form is used ({\hyperref[\detokenize{namelists/jules_soil_ecosse.nml:JULES_SOIL_ECOSSE::water_modifier}]{\sphinxcrossref{\sphinxcode{\sphinxupquote{water\_modifier}}}}} = 1).

\sphinxAtStartPar
Note that this default value is also that harwired in the code for use with the 4\sphinxhyphen{}pool model ({\hyperref[\detokenize{namelists/jules_soil_biogeochem.nml:JULES_SOIL_BIOGEOCHEM::soil_bgc_model}]{\sphinxcrossref{\sphinxcode{\sphinxupquote{soil\_bgc\_model}}}}} = 2).

\end{fulllineitems}


\begin{sphinxadmonition}{note}{Parameters for ECOSSE that are only used if soil N is included (\sphinxstyleliteralintitle{\sphinxupquote{l\_soil\_n}} = TRUE).}
\index{l\_decomp\_slow (in namelist JULES\_SOIL\_ECOSSE)@\spxentry{l\_decomp\_slow}\spxextra{in namelist JULES\_SOIL\_ECOSSE}|spxpagem}

\begin{fulllineitems}
\phantomsection\label{\detokenize{namelists/jules_soil_ecosse.nml:JULES_SOIL_ECOSSE::l_decomp_slow}}
\pysigstartsignatures
\pysigline{\sphinxcode{\sphinxupquote{JULES\_SOIL\_ECOSSE::}}\sphinxbfcode{\sphinxupquote{l\_decomp\_slow}}}
\pysigstopsignatures\begin{quote}\begin{description}
\sphinxlineitem{Type}
\sphinxAtStartPar
logical

\sphinxlineitem{Default}
\sphinxAtStartPar
T

\end{description}\end{quote}

\sphinxAtStartPar
Switch controlling how lack of nitrogen affects soil decomposition.
\begin{description}
\sphinxlineitem{TRUE}
\sphinxAtStartPar
Reduce the decomposition rate.

\sphinxlineitem{FALSE}
\sphinxAtStartPar
Reduce the efficiency of decomposition, so that decomposition results in increased production of CO$_{\text{2}}$ and decreased production of further soil C. This is the approach used in the standalone version of ECOSSE.

\end{description}

\end{fulllineitems}

\index{depo\_nit\_frac (in namelist JULES\_SOIL\_ECOSSE)@\spxentry{depo\_nit\_frac}\spxextra{in namelist JULES\_SOIL\_ECOSSE}|spxpagem}

\begin{fulllineitems}
\phantomsection\label{\detokenize{namelists/jules_soil_ecosse.nml:JULES_SOIL_ECOSSE::depo_nit_frac}}
\pysigstartsignatures
\pysigline{\sphinxcode{\sphinxupquote{JULES\_SOIL\_ECOSSE::}}\sphinxbfcode{\sphinxupquote{depo\_nit\_frac}}}
\pysigstopsignatures\begin{quote}\begin{description}
\sphinxlineitem{Type}
\sphinxAtStartPar
REAL

\sphinxlineitem{Permitted}
\sphinxAtStartPar
0 \textless{}= depo\_nit\_frac \textless{}= 1

\sphinxlineitem{Default}
\sphinxAtStartPar
1.0

\end{description}\end{quote}

\sphinxAtStartPar
The fraction of nitrogen deposition that is aded to the soil nitrate pool. The complement is aded to the ammonium pool.

\end{fulllineitems}

\index{bacteria\_min\_frac (in namelist JULES\_SOIL\_ECOSSE)@\spxentry{bacteria\_min\_frac}\spxextra{in namelist JULES\_SOIL\_ECOSSE}|spxpagem}

\begin{fulllineitems}
\phantomsection\label{\detokenize{namelists/jules_soil_ecosse.nml:JULES_SOIL_ECOSSE::bacteria_min_frac}}
\pysigstartsignatures
\pysigline{\sphinxcode{\sphinxupquote{JULES\_SOIL\_ECOSSE::}}\sphinxbfcode{\sphinxupquote{bacteria\_min\_frac}}}
\pysigstopsignatures\begin{quote}\begin{description}
\sphinxlineitem{Type}
\sphinxAtStartPar
REAL

\sphinxlineitem{Permitted}
\sphinxAtStartPar
0 \textless{}= bacteria\_min\_frac \textless{}= 1

\sphinxlineitem{Default}
\sphinxAtStartPar
0.2

\end{description}\end{quote}

\sphinxAtStartPar
The minimum fraction of the decomposer community that are bacteria.

\end{fulllineitems}

\index{bacteria\_max\_frac (in namelist JULES\_SOIL\_ECOSSE)@\spxentry{bacteria\_max\_frac}\spxextra{in namelist JULES\_SOIL\_ECOSSE}|spxpagem}

\begin{fulllineitems}
\phantomsection\label{\detokenize{namelists/jules_soil_ecosse.nml:JULES_SOIL_ECOSSE::bacteria_max_frac}}
\pysigstartsignatures
\pysigline{\sphinxcode{\sphinxupquote{JULES\_SOIL\_ECOSSE::}}\sphinxbfcode{\sphinxupquote{bacteria\_max\_frac}}}
\pysigstopsignatures\begin{quote}\begin{description}
\sphinxlineitem{Type}
\sphinxAtStartPar
REAL

\sphinxlineitem{Permitted}
\sphinxAtStartPar
bacteria\_min\_frac \textless{}= bacteria\_max\_frac \textless{}= 1

\sphinxlineitem{Default}
\sphinxAtStartPar
0.5

\end{description}\end{quote}

\sphinxAtStartPar
The maximum fraction of the decomposer community that are bacteria.

\end{fulllineitems}

\index{bacteria\_min\_frac\_pH (in namelist JULES\_SOIL\_ECOSSE)@\spxentry{bacteria\_min\_frac\_pH}\spxextra{in namelist JULES\_SOIL\_ECOSSE}|spxpagem}

\begin{fulllineitems}
\phantomsection\label{\detokenize{namelists/jules_soil_ecosse.nml:JULES_SOIL_ECOSSE::bacteria_min_frac_pH}}
\pysigstartsignatures
\pysigline{\sphinxcode{\sphinxupquote{JULES\_SOIL\_ECOSSE::}}\sphinxbfcode{\sphinxupquote{bacteria\_min\_frac\_pH}}}
\pysigstopsignatures\begin{quote}\begin{description}
\sphinxlineitem{Type}
\sphinxAtStartPar
REAL

\sphinxlineitem{Default}
\sphinxAtStartPar
4.0

\end{description}\end{quote}

\sphinxAtStartPar
The soil pH at or below which the fraction of bacteria is at a minimum.

\end{fulllineitems}

\index{bacteria\_max\_frac\_pH (in namelist JULES\_SOIL\_ECOSSE)@\spxentry{bacteria\_max\_frac\_pH}\spxextra{in namelist JULES\_SOIL\_ECOSSE}|spxpagem}

\begin{fulllineitems}
\phantomsection\label{\detokenize{namelists/jules_soil_ecosse.nml:JULES_SOIL_ECOSSE::bacteria_max_frac_pH}}
\pysigstartsignatures
\pysigline{\sphinxcode{\sphinxupquote{JULES\_SOIL\_ECOSSE::}}\sphinxbfcode{\sphinxupquote{bacteria\_max\_frac\_pH}}}
\pysigstopsignatures\begin{quote}\begin{description}
\sphinxlineitem{Type}
\sphinxAtStartPar
REAL

\sphinxlineitem{Permitted}
\sphinxAtStartPar
bacteria\_min\_frac\_pH \textless{}= bacteria\_max\_frac\_pH

\sphinxlineitem{Default}
\sphinxAtStartPar
5.5

\end{description}\end{quote}

\sphinxAtStartPar
The soil pH at or above which the fraction of bacteria is at a maximum.

\end{fulllineitems}

\index{cn\_bacteria (in namelist JULES\_SOIL\_ECOSSE)@\spxentry{cn\_bacteria}\spxextra{in namelist JULES\_SOIL\_ECOSSE}|spxpagem}

\begin{fulllineitems}
\phantomsection\label{\detokenize{namelists/jules_soil_ecosse.nml:JULES_SOIL_ECOSSE::cn_bacteria}}
\pysigstartsignatures
\pysigline{\sphinxcode{\sphinxupquote{JULES\_SOIL\_ECOSSE::}}\sphinxbfcode{\sphinxupquote{cn\_bacteria}}}
\pysigstopsignatures\begin{quote}\begin{description}
\sphinxlineitem{Type}
\sphinxAtStartPar
REAL

\sphinxlineitem{Default}
\sphinxAtStartPar
5.5

\end{description}\end{quote}

\sphinxAtStartPar
The C:N ratio of soil bacteria.

\end{fulllineitems}

\index{cn\_fungi (in namelist JULES\_SOIL\_ECOSSE)@\spxentry{cn\_fungi}\spxextra{in namelist JULES\_SOIL\_ECOSSE}|spxpagem}

\begin{fulllineitems}
\phantomsection\label{\detokenize{namelists/jules_soil_ecosse.nml:JULES_SOIL_ECOSSE::cn_fungi}}
\pysigstartsignatures
\pysigline{\sphinxcode{\sphinxupquote{JULES\_SOIL\_ECOSSE::}}\sphinxbfcode{\sphinxupquote{cn\_fungi}}}
\pysigstopsignatures\begin{quote}\begin{description}
\sphinxlineitem{Type}
\sphinxAtStartPar
REAL

\sphinxlineitem{Default}
\sphinxAtStartPar
5.5

\end{description}\end{quote}

\sphinxAtStartPar
The C:N ratio of soil fungi.

\end{fulllineitems}

\index{depth\_nitrif (in namelist JULES\_SOIL\_ECOSSE)@\spxentry{depth\_nitrif}\spxextra{in namelist JULES\_SOIL\_ECOSSE}|spxpagem}

\begin{fulllineitems}
\phantomsection\label{\detokenize{namelists/jules_soil_ecosse.nml:JULES_SOIL_ECOSSE::depth_nitrif}}
\pysigstartsignatures
\pysigline{\sphinxcode{\sphinxupquote{JULES\_SOIL\_ECOSSE::}}\sphinxbfcode{\sphinxupquote{depth\_nitrif}}}
\pysigstopsignatures\begin{quote}\begin{description}
\sphinxlineitem{Type}
\sphinxAtStartPar
REAL

\sphinxlineitem{Default}
\sphinxAtStartPar
0.25

\end{description}\end{quote}

\sphinxAtStartPar
Greatest depth at which nitrification and denitrification are allowed (m).

\end{fulllineitems}

\index{nitrif\_rate (in namelist JULES\_SOIL\_ECOSSE)@\spxentry{nitrif\_rate}\spxextra{in namelist JULES\_SOIL\_ECOSSE}|spxpagem}

\begin{fulllineitems}
\phantomsection\label{\detokenize{namelists/jules_soil_ecosse.nml:JULES_SOIL_ECOSSE::nitrif_rate}}
\pysigstartsignatures
\pysigline{\sphinxcode{\sphinxupquote{JULES\_SOIL\_ECOSSE::}}\sphinxbfcode{\sphinxupquote{nitrif\_rate}}}
\pysigstopsignatures\begin{quote}\begin{description}
\sphinxlineitem{Type}
\sphinxAtStartPar
REAL

\sphinxlineitem{Default}
\sphinxAtStartPar
9.921e\sphinxhyphen{}7

\end{description}\end{quote}

\sphinxAtStartPar
Rate constant for nitrification (s$^{\text{\sphinxhyphen{}1}}$).

\end{fulllineitems}

\index{nitrif\_wrate\_min (in namelist JULES\_SOIL\_ECOSSE)@\spxentry{nitrif\_wrate\_min}\spxextra{in namelist JULES\_SOIL\_ECOSSE}|spxpagem}

\begin{fulllineitems}
\phantomsection\label{\detokenize{namelists/jules_soil_ecosse.nml:JULES_SOIL_ECOSSE::nitrif_wrate_min}}
\pysigstartsignatures
\pysigline{\sphinxcode{\sphinxupquote{JULES\_SOIL\_ECOSSE::}}\sphinxbfcode{\sphinxupquote{nitrif\_wrate\_min}}}
\pysigstopsignatures\begin{quote}\begin{description}
\sphinxlineitem{Type}
\sphinxAtStartPar
REAL

\sphinxlineitem{Default}
\sphinxAtStartPar
0.6

\end{description}\end{quote}

\sphinxAtStartPar
Minimum allowed value of the water rate modifier for nitrification when 4\sphinxhyphen{}pool form is used. Only used if {\hyperref[\detokenize{namelists/jules_soil_ecosse.nml:JULES_SOIL_ECOSSE::water_modifier}]{\sphinxcrossref{\sphinxcode{\sphinxupquote{water\_modifier}}}}} = 2.).

\end{fulllineitems}

\index{nitrif\_max\_factor (in namelist JULES\_SOIL\_ECOSSE)@\spxentry{nitrif\_max\_factor}\spxextra{in namelist JULES\_SOIL\_ECOSSE}|spxpagem}

\begin{fulllineitems}
\phantomsection\label{\detokenize{namelists/jules_soil_ecosse.nml:JULES_SOIL_ECOSSE::nitrif_max_factor}}
\pysigstartsignatures
\pysigline{\sphinxcode{\sphinxupquote{JULES\_SOIL\_ECOSSE::}}\sphinxbfcode{\sphinxupquote{nitrif\_max\_factor}}}
\pysigstopsignatures\begin{quote}\begin{description}
\sphinxlineitem{Type}
\sphinxAtStartPar
REAL

\sphinxlineitem{Default}
\sphinxAtStartPar
0.1

\end{description}\end{quote}

\sphinxAtStartPar
Shape factor in rate modifier for nitrification (kg m$^{\text{\sphinxhyphen{}3}}$).

\end{fulllineitems}

\index{nitrif\_frac\_n2o\_fc (in namelist JULES\_SOIL\_ECOSSE)@\spxentry{nitrif\_frac\_n2o\_fc}\spxextra{in namelist JULES\_SOIL\_ECOSSE}|spxpagem}

\begin{fulllineitems}
\phantomsection\label{\detokenize{namelists/jules_soil_ecosse.nml:JULES_SOIL_ECOSSE::nitrif_frac_n2o_fc}}
\pysigstartsignatures
\pysigline{\sphinxcode{\sphinxupquote{JULES\_SOIL\_ECOSSE::}}\sphinxbfcode{\sphinxupquote{nitrif\_frac\_n2o\_fc}}}
\pysigstopsignatures\begin{quote}\begin{description}
\sphinxlineitem{Type}
\sphinxAtStartPar
REAL

\sphinxlineitem{Permitted}
\sphinxAtStartPar
0 \textless{}= nitrif\_frac\_n2o\_fc \textless{}= 1

\sphinxlineitem{Default}
\sphinxAtStartPar
0.02

\end{description}\end{quote}

\sphinxAtStartPar
Fraction of nitrification lost as N$_{\text{2}}$O by partial nitrification at field capacity.

\end{fulllineitems}

\index{nitrif\_frac\_gas (in namelist JULES\_SOIL\_ECOSSE)@\spxentry{nitrif\_frac\_gas}\spxextra{in namelist JULES\_SOIL\_ECOSSE}|spxpagem}

\begin{fulllineitems}
\phantomsection\label{\detokenize{namelists/jules_soil_ecosse.nml:JULES_SOIL_ECOSSE::nitrif_frac_gas}}
\pysigstartsignatures
\pysigline{\sphinxcode{\sphinxupquote{JULES\_SOIL\_ECOSSE::}}\sphinxbfcode{\sphinxupquote{nitrif\_frac\_gas}}}
\pysigstopsignatures\begin{quote}\begin{description}
\sphinxlineitem{Type}
\sphinxAtStartPar
REAL

\sphinxlineitem{Permitted}
\sphinxAtStartPar
0 \textless{}= nitrif\_frac\_gas \textless{}= 1

\sphinxlineitem{Default}
\sphinxAtStartPar
0.02

\end{description}\end{quote}

\sphinxAtStartPar
Fraction of nitrification lost as gas through full nitrification.

\end{fulllineitems}

\index{nitrif\_frac\_no (in namelist JULES\_SOIL\_ECOSSE)@\spxentry{nitrif\_frac\_no}\spxextra{in namelist JULES\_SOIL\_ECOSSE}|spxpagem}

\begin{fulllineitems}
\phantomsection\label{\detokenize{namelists/jules_soil_ecosse.nml:JULES_SOIL_ECOSSE::nitrif_frac_no}}
\pysigstartsignatures
\pysigline{\sphinxcode{\sphinxupquote{JULES\_SOIL\_ECOSSE::}}\sphinxbfcode{\sphinxupquote{nitrif\_frac\_no}}}
\pysigstopsignatures\begin{quote}\begin{description}
\sphinxlineitem{Type}
\sphinxAtStartPar
REAL

\sphinxlineitem{Permitted}
\sphinxAtStartPar
0 \textless{}= nitrif\_frac\_no \textless{}= 1

\sphinxlineitem{Default}
\sphinxAtStartPar
0.4

\end{description}\end{quote}

\sphinxAtStartPar
Fraction of nitrification gas loss through full nitrification that is NO.

\end{fulllineitems}

\index{denit50 (in namelist JULES\_SOIL\_ECOSSE)@\spxentry{denit50}\spxextra{in namelist JULES\_SOIL\_ECOSSE}|spxpagem}

\begin{fulllineitems}
\phantomsection\label{\detokenize{namelists/jules_soil_ecosse.nml:JULES_SOIL_ECOSSE::denit50}}
\pysigstartsignatures
\pysigline{\sphinxcode{\sphinxupquote{JULES\_SOIL\_ECOSSE::}}\sphinxbfcode{\sphinxupquote{denit50}}}
\pysigstopsignatures\begin{quote}\begin{description}
\sphinxlineitem{Type}
\sphinxAtStartPar
REAL

\sphinxlineitem{Default}
\sphinxAtStartPar
0.033

\end{description}\end{quote}

\sphinxAtStartPar
Amount of nitrate at which denitrification rate is 50\% of the potential rate (kg m$^{\text{\sphinxhyphen{}3}}$).

\end{fulllineitems}

\index{denit\_bio\_factor (in namelist JULES\_SOIL\_ECOSSE)@\spxentry{denit\_bio\_factor}\spxextra{in namelist JULES\_SOIL\_ECOSSE}|spxpagem}

\begin{fulllineitems}
\phantomsection\label{\detokenize{namelists/jules_soil_ecosse.nml:JULES_SOIL_ECOSSE::denit_bio_factor}}
\pysigstartsignatures
\pysigline{\sphinxcode{\sphinxupquote{JULES\_SOIL\_ECOSSE::}}\sphinxbfcode{\sphinxupquote{denit\_bio\_factor}}}
\pysigstopsignatures\begin{quote}\begin{description}
\sphinxlineitem{Type}
\sphinxAtStartPar
REAL

\sphinxlineitem{Default}
\sphinxAtStartPar
0.005

\end{description}\end{quote}

\sphinxAtStartPar
Factor in denitrification calculation to convert flux of CO$_{\text{2}}$ into a representation of biological activity (m$^{\text{2}}$ kg$^{\text{\sphinxhyphen{}1}}$).

\end{fulllineitems}

\index{denit\_frac\_n2\_fc (in namelist JULES\_SOIL\_ECOSSE)@\spxentry{denit\_frac\_n2\_fc}\spxextra{in namelist JULES\_SOIL\_ECOSSE}|spxpagem}

\begin{fulllineitems}
\phantomsection\label{\detokenize{namelists/jules_soil_ecosse.nml:JULES_SOIL_ECOSSE::denit_frac_n2_fc}}
\pysigstartsignatures
\pysigline{\sphinxcode{\sphinxupquote{JULES\_SOIL\_ECOSSE::}}\sphinxbfcode{\sphinxupquote{denit\_frac\_n2\_fc}}}
\pysigstopsignatures\begin{quote}\begin{description}
\sphinxlineitem{Type}
\sphinxAtStartPar
REAL

\sphinxlineitem{Permitted}
\sphinxAtStartPar
0 \textless{}= denit\_frac\_n2\_fc \textless{}= 1

\sphinxlineitem{Default}
\sphinxAtStartPar
0.55

\end{description}\end{quote}

\sphinxAtStartPar
Proportion of denitrified N that becomes N$_{\text{2}}$ when soil moisture is at field capacity.

\end{fulllineitems}

\index{denit\_nitrate\_equal (in namelist JULES\_SOIL\_ECOSSE)@\spxentry{denit\_nitrate\_equal}\spxextra{in namelist JULES\_SOIL\_ECOSSE}|spxpagem}

\begin{fulllineitems}
\phantomsection\label{\detokenize{namelists/jules_soil_ecosse.nml:JULES_SOIL_ECOSSE::denit_nitrate_equal}}
\pysigstartsignatures
\pysigline{\sphinxcode{\sphinxupquote{JULES\_SOIL\_ECOSSE::}}\sphinxbfcode{\sphinxupquote{denit\_nitrate\_equal}}}
\pysigstopsignatures\begin{quote}\begin{description}
\sphinxlineitem{Type}
\sphinxAtStartPar
REAL

\sphinxlineitem{Default}
\sphinxAtStartPar
0.4

\end{description}\end{quote}

\sphinxAtStartPar
Amount of N in soil nitrate at which denitrified N is released as equal amounts of N$_{\text{2}}$ and N$_{\text{2\textasciigrave{}O (kg m:sup:}}$\sphinxhyphen{}3\textasciigrave{}).

\end{fulllineitems}

\index{denit\_water\_coeff (in namelist JULES\_SOIL\_ECOSSE)@\spxentry{denit\_water\_coeff}\spxextra{in namelist JULES\_SOIL\_ECOSSE}|spxpagem}

\begin{fulllineitems}
\phantomsection\label{\detokenize{namelists/jules_soil_ecosse.nml:JULES_SOIL_ECOSSE::denit_water_coeff}}
\pysigstartsignatures
\pysigline{\sphinxcode{\sphinxupquote{JULES\_SOIL\_ECOSSE::}}\sphinxbfcode{\sphinxupquote{denit\_water\_coeff}}}
\pysigstopsignatures\begin{quote}\begin{description}
\sphinxlineitem{Type}
\sphinxAtStartPar
REAL(3)

\sphinxlineitem{Default}
\sphinxAtStartPar
0.62, 0.38, 1.74

\end{description}\end{quote}

\sphinxAtStartPar
Constants describing water modifier for denitrification.

\end{fulllineitems}

\index{amm\_leach\_min (in namelist JULES\_SOIL\_ECOSSE)@\spxentry{amm\_leach\_min}\spxextra{in namelist JULES\_SOIL\_ECOSSE}|spxpagem}

\begin{fulllineitems}
\phantomsection\label{\detokenize{namelists/jules_soil_ecosse.nml:JULES_SOIL_ECOSSE::amm_leach_min}}
\pysigstartsignatures
\pysigline{\sphinxcode{\sphinxupquote{JULES\_SOIL\_ECOSSE::}}\sphinxbfcode{\sphinxupquote{amm\_leach\_min}}}
\pysigstopsignatures\begin{quote}\begin{description}
\sphinxlineitem{Type}
\sphinxAtStartPar
REAL

\sphinxlineitem{Default}
\sphinxAtStartPar
0.02

\end{description}\end{quote}

\sphinxAtStartPar
Minimum allowed amount of N in soil ammonium after leaching (kg m$^{\text{\sphinxhyphen{}3}}$).

\end{fulllineitems}

\index{n\_inorg\_max\_conc (in namelist JULES\_SOIL\_ECOSSE)@\spxentry{n\_inorg\_max\_conc}\spxextra{in namelist JULES\_SOIL\_ECOSSE}|spxpagem}

\begin{fulllineitems}
\phantomsection\label{\detokenize{namelists/jules_soil_ecosse.nml:JULES_SOIL_ECOSSE::n_inorg_max_conc}}
\pysigstartsignatures
\pysigline{\sphinxcode{\sphinxupquote{JULES\_SOIL\_ECOSSE::}}\sphinxbfcode{\sphinxupquote{n\_inorg\_max\_conc}}}
\pysigstopsignatures\begin{quote}\begin{description}
\sphinxlineitem{Type}
\sphinxAtStartPar
REAL

\sphinxlineitem{Default}
\sphinxAtStartPar
\sphinxhyphen{}1.0

\end{description}\end{quote}

\sphinxAtStartPar
Maximum\sphinxhyphen{}allowed concentration of inorganic N in a layer (kg m$^{\text{\sphinxhyphen{}3}}$). A value less than zero means no maximum concentration is imposed.

\end{fulllineitems}

\end{sphinxadmonition}

\sphinxstepscope


\section{\sphinxstyleliteralintitle{\sphinxupquote{jules\_deposition.nml}}}
\label{\detokenize{namelists/jules_deposition.nml:jules-deposition-nml}}\label{\detokenize{namelists/jules_deposition.nml::doc}}
\sphinxAtStartPar
This file contains options and parameters for modelling of dry deposition of atmospheric trace constituents. It contains a variable number of namelists depending on the required model configuration. The namelist {\hyperref[\detokenize{namelists/jules_deposition.nml:namelist-JULES_DEPOSITION}]{\sphinxcrossref{\sphinxcode{\sphinxupquote{JULES\_DEPOSITION}}}}} is always required, and one or more instances of the namelist {\hyperref[\detokenize{namelists/jules_deposition.nml:namelist-JULES_DEPOSITION_SPECIES}]{\sphinxcrossref{\sphinxcode{\sphinxupquote{JULES\_DEPOSITION\_SPECIES}}}}} is required if dry deposition has been selected.

\begin{sphinxadmonition}{warning}{Warning:}
\sphinxAtStartPar
Atmospheric deposition in JULES is still in development and is far from fully functional in this version \sphinxhyphen{} the code is included to allow further development. Users should not try to activate the deposition code: leave {\hyperref[\detokenize{namelists/jules_deposition.nml:JULES_DEPOSITION::l_deposition}]{\sphinxcrossref{\sphinxcode{\sphinxupquote{l\_deposition}}}}} as FALSE.
\end{sphinxadmonition}


\subsection{\sphinxstyleliteralintitle{\sphinxupquote{JULES\_DEPOSITION}} namelist members}
\label{\detokenize{namelists/jules_deposition.nml:namelist-JULES_DEPOSITION}}\label{\detokenize{namelists/jules_deposition.nml:jules-deposition-namelist-members}}\index{JULES\_DEPOSITION (namelist)@\spxentry{JULES\_DEPOSITION}\spxextra{namelist}|spxpagem}\index{l\_deposition (in namelist JULES\_DEPOSITION)@\spxentry{l\_deposition}\spxextra{in namelist JULES\_DEPOSITION}|spxpagem}

\begin{fulllineitems}
\phantomsection\label{\detokenize{namelists/jules_deposition.nml:JULES_DEPOSITION::l_deposition}}
\pysigstartsignatures
\pysigline{\sphinxcode{\sphinxupquote{JULES\_DEPOSITION::}}\sphinxbfcode{\sphinxupquote{l\_deposition}}}
\pysigstopsignatures\begin{quote}\begin{description}
\sphinxlineitem{Type}
\sphinxAtStartPar
logical

\sphinxlineitem{Default}
\sphinxAtStartPar
FALSE

\end{description}\end{quote}

\sphinxAtStartPar
Switch to activate deposition code in JULES.
\begin{description}
\sphinxlineitem{TRUE}
\sphinxAtStartPar
Model deposition in JULES.

\sphinxlineitem{FALSE}
\sphinxAtStartPar
Do not model deposition in JULES.

\end{description}

\end{fulllineitems}


\begin{sphinxadmonition}{note}{Only used if \sphinxstyleliteralintitle{\sphinxupquote{l\_deposition}} = TRUE.}
\index{dry\_dep\_model (in namelist JULES\_DEPOSITION)@\spxentry{dry\_dep\_model}\spxextra{in namelist JULES\_DEPOSITION}|spxpagem}

\begin{fulllineitems}
\phantomsection\label{\detokenize{namelists/jules_deposition.nml:JULES_DEPOSITION::dry_dep_model}}
\pysigstartsignatures
\pysigline{\sphinxcode{\sphinxupquote{JULES\_DEPOSITION::}}\sphinxbfcode{\sphinxupquote{dry\_dep\_model}}}
\pysigstopsignatures\begin{quote}\begin{description}
\sphinxlineitem{Type}
\sphinxAtStartPar
integer

\sphinxlineitem{Permitted}
\sphinxAtStartPar
1

\sphinxlineitem{Default}
\sphinxAtStartPar
none

\end{description}\end{quote}

\sphinxAtStartPar
Choice for model of dry deposition.

\sphinxAtStartPar
Possible values are:
\begin{enumerate}
\sphinxsetlistlabels{\arabic}{enumi}{enumii}{}{.}%
\item {} 
\begin{DUlineblock}{0em}
\item[] Parameterisation based on that found in UKCA (but now implemented in JULES).
\end{DUlineblock}

\end{enumerate}

\end{fulllineitems}

\index{l\_deposition\_flux (in namelist JULES\_DEPOSITION)@\spxentry{l\_deposition\_flux}\spxextra{in namelist JULES\_DEPOSITION}|spxpagem}

\begin{fulllineitems}
\phantomsection\label{\detokenize{namelists/jules_deposition.nml:JULES_DEPOSITION::l_deposition_flux}}
\pysigstartsignatures
\pysigline{\sphinxcode{\sphinxupquote{JULES\_DEPOSITION::}}\sphinxbfcode{\sphinxupquote{l\_deposition\_flux}}}
\pysigstopsignatures\begin{quote}\begin{description}
\sphinxlineitem{Type}
\sphinxAtStartPar
logical

\sphinxlineitem{Default}
\sphinxAtStartPar
FALSE

\end{description}\end{quote}

\sphinxAtStartPar
Switch for calculation of deposition fluxes as opposed to deposition velocities.
\begin{description}
\sphinxlineitem{TRUE}
\sphinxAtStartPar
Calculate deposition fluxes. This requires that the concentrations of atmopsheric tracer species are provided as prescribed data (see {\hyperref[\detokenize{namelists/prescribed_data.nml:supported-prescribed-variables}]{\sphinxcrossref{\DUrole{std,std-ref}{List of supported variables}}}}).

\sphinxlineitem{FALSE}
\sphinxAtStartPar
Calculate deposition velocities.

\end{description}

\end{fulllineitems}

\end{sphinxadmonition}

\begin{sphinxadmonition}{note}{Only used if \sphinxstyleliteralintitle{\sphinxupquote{l\_deposition}} = TRUE and \sphinxstyleliteralintitle{\sphinxupquote{dry\_dep\_model}} = 1 (UKCA).}
\index{ndry\_dep\_species (in namelist JULES\_DEPOSITION)@\spxentry{ndry\_dep\_species}\spxextra{in namelist JULES\_DEPOSITION}|spxpagem}

\begin{fulllineitems}
\phantomsection\label{\detokenize{namelists/jules_deposition.nml:JULES_DEPOSITION::ndry_dep_species}}
\pysigstartsignatures
\pysigline{\sphinxcode{\sphinxupquote{JULES\_DEPOSITION::}}\sphinxbfcode{\sphinxupquote{ndry\_dep\_species}}}
\pysigstopsignatures\begin{quote}\begin{description}
\sphinxlineitem{Type}
\sphinxAtStartPar
integer

\sphinxlineitem{Permitted}
\sphinxAtStartPar
\textgreater{}= 1

\sphinxlineitem{Default}
\sphinxAtStartPar
none

\end{description}\end{quote}

\sphinxAtStartPar
Number of species for which dry deposition is calculated.

\end{fulllineitems}

\index{l\_ukca\_ddep\_lev1 (in namelist JULES\_DEPOSITION)@\spxentry{l\_ukca\_ddep\_lev1}\spxextra{in namelist JULES\_DEPOSITION}|spxpagem}

\begin{fulllineitems}
\phantomsection\label{\detokenize{namelists/jules_deposition.nml:JULES_DEPOSITION::l_ukca_ddep_lev1}}
\pysigstartsignatures
\pysigline{\sphinxcode{\sphinxupquote{JULES\_DEPOSITION::}}\sphinxbfcode{\sphinxupquote{l\_ukca\_ddep\_lev1}}}
\pysigstopsignatures\begin{quote}\begin{description}
\sphinxlineitem{Type}
\sphinxAtStartPar
logical

\sphinxlineitem{Default}
\sphinxAtStartPar
FALSE

\end{description}\end{quote}

\sphinxAtStartPar
Switch controlling which atmospheric levels are used for dry deposition.
\begin{description}
\sphinxlineitem{TRUE}
\sphinxAtStartPar
Deposition occurs only from the lowest atmospheric level.

\sphinxlineitem{FALSE}
\sphinxAtStartPar
Deposition occurs from all levels within the atmospheric boundary layer.

\end{description}

\end{fulllineitems}

\index{tundra\_s\_limit (in namelist JULES\_DEPOSITION)@\spxentry{tundra\_s\_limit}\spxextra{in namelist JULES\_DEPOSITION}|spxpagem}

\begin{fulllineitems}
\phantomsection\label{\detokenize{namelists/jules_deposition.nml:JULES_DEPOSITION::tundra_s_limit}}
\pysigstartsignatures
\pysigline{\sphinxcode{\sphinxupquote{JULES\_DEPOSITION::}}\sphinxbfcode{\sphinxupquote{tundra\_s\_limit}}}
\pysigstopsignatures\begin{quote}\begin{description}
\sphinxlineitem{Type}
\sphinxAtStartPar
real

\sphinxlineitem{Default}
\sphinxAtStartPar
none

\end{description}\end{quote}

\sphinxAtStartPar
Latitude of southern limit of tundra (degrees). This is used to alter the calculation of deposition of certain species in the tundra region (actually for all points north of this limit). Only used if the list of species (see {\hyperref[\detokenize{namelists/jules_deposition.nml:JULES_DEPOSITION_SPECIES::dep_species_name_io}]{\sphinxcrossref{\sphinxcode{\sphinxupquote{dep\_species\_name\_io}}}}}) includes one or more of ‘CO’, ‘NO2’, ‘O3’, ‘PAN’, ‘PPAN’, ‘MPAN’ or ‘ONITU’.

\end{fulllineitems}

\index{dzl\_const (in namelist JULES\_DEPOSITION)@\spxentry{dzl\_const}\spxextra{in namelist JULES\_DEPOSITION}|spxpagem}

\begin{fulllineitems}
\phantomsection\label{\detokenize{namelists/jules_deposition.nml:JULES_DEPOSITION::dzl_const}}
\pysigstartsignatures
\pysigline{\sphinxcode{\sphinxupquote{JULES\_DEPOSITION::}}\sphinxbfcode{\sphinxupquote{dzl\_const}}}
\pysigstopsignatures\begin{quote}\begin{description}
\sphinxlineitem{Type}
\sphinxAtStartPar
real

\sphinxlineitem{Default}
\sphinxAtStartPar
none

\end{description}\end{quote}

\sphinxAtStartPar
Constant value for separation of boundary layer levels (m). All layer thicknesses are set to this value. This is used as a simple way to prescribe the layer thicknesses in standalone mode. This value can be overriden by prescribed data \sphinxhyphen{} see {\hyperref[\detokenize{namelists/prescribed_data.nml:namelist-JULES_PRESCRIBED}]{\sphinxcrossref{\sphinxcode{\sphinxupquote{JULES\_PRESCRIBED}}}}}. This can be considered as the representative depth for tracer concentration and the depth over which the deposition flux is removed.

\end{fulllineitems}

\end{sphinxadmonition}


\subsection{Notes on the \sphinxstyleliteralintitle{\sphinxupquote{JULES\_DEPOSITION}} namelist}
\label{\detokenize{namelists/jules_deposition.nml:notes-on-the-jules-deposition-namelist}}
\sphinxAtStartPar
The height of the atmospheric boundary layer is required and is set to a default constant value of 1000 m. This value can be overridden via the namelist variable {\hyperref[\detokenize{namelists/drive.nml:JULES_DRIVE::bl_height}]{\sphinxcrossref{\sphinxcode{\sphinxupquote{bl\_height}}}}}, or can be prescribed (i.e. allowed to vary in time and/or space) by providing \sphinxcode{\sphinxupquote{bl\_height}} as prescribed data (see {\hyperref[\detokenize{namelists/prescribed_data.nml:supported-prescribed-variables}]{\sphinxcrossref{\DUrole{std,std-ref}{List of supported variables}}}}).

\sphinxAtStartPar
The number of model levels in the boundary layer is required and is set to a default balues of 1. This can be overridden via the namelist variable {\hyperref[\detokenize{namelists/model_grid.nml:JULES_NLSIZES::bl_levels}]{\sphinxcrossref{\sphinxcode{\sphinxupquote{bl\_levels}}}}}. The number of levels is only used to communicate with the atmospheric model (e.g. UKCA).

\sphinxAtStartPar
The separation of the model levels in the boundary layer is required and is set to a constant value via {\hyperref[\detokenize{namelists/jules_deposition.nml:JULES_DEPOSITION::dzl_const}]{\sphinxcrossref{\sphinxcode{\sphinxupquote{dzl\_const}}}}}. The separation can be prescribed (i.e. allowed to vary in time and/or space) by providing \sphinxcode{\sphinxupquote{level\_separation}} as prescribed data (see {\hyperref[\detokenize{namelists/prescribed_data.nml:supported-prescribed-variables}]{\sphinxcrossref{\DUrole{std,std-ref}{List of supported variables}}}}). This can be considered as the representative depth for tracer concentration and the depth over which the deposition flux is removed.

\sphinxAtStartPar
If deposition fluxes (rather than deposition velocities) are to be calculated (see {\hyperref[\detokenize{namelists/jules_deposition.nml:JULES_DEPOSITION::l_deposition_flux}]{\sphinxcrossref{\sphinxcode{\sphinxupquote{l\_deposition\_flux}}}}}), the concentrations of atmospheric tracer species need to be prescribed (see \sphinxcode{\sphinxupquote{tracer\_field}} in {\hyperref[\detokenize{namelists/prescribed_data.nml:supported-prescribed-variables}]{\sphinxcrossref{\DUrole{std,std-ref}{List of supported variables}}}}).


\subsection{\sphinxstyleliteralintitle{\sphinxupquote{JULES\_DEPOSITION\_SPECIES}} namelist members}
\label{\detokenize{namelists/jules_deposition.nml:namelist-JULES_DEPOSITION_SPECIES}}\label{\detokenize{namelists/jules_deposition.nml:jules-deposition-species-namelist-members}}\index{JULES\_DEPOSITION\_SPECIES (namelist)@\spxentry{JULES\_DEPOSITION\_SPECIES}\spxextra{namelist}|spxpagem}
\sphinxAtStartPar
This namelist should occur {\hyperref[\detokenize{namelists/jules_deposition.nml:JULES_DEPOSITION::ndry_dep_species}]{\sphinxcrossref{\sphinxcode{\sphinxupquote{ndry\_dep\_species}}}}} times, with each occurence containing parameters for an atmospheric tracer species that is to be considered in dry deposition.
\index{dep\_species\_name\_io (in namelist JULES\_DEPOSITION\_SPECIES)@\spxentry{dep\_species\_name\_io}\spxextra{in namelist JULES\_DEPOSITION\_SPECIES}|spxpagem}

\begin{fulllineitems}
\phantomsection\label{\detokenize{namelists/jules_deposition.nml:JULES_DEPOSITION_SPECIES::dep_species_name_io}}
\pysigstartsignatures
\pysigline{\sphinxcode{\sphinxupquote{JULES\_DEPOSITION\_SPECIES::}}\sphinxbfcode{\sphinxupquote{dep\_species\_name\_io}}}
\pysigstopsignatures\begin{quote}\begin{description}
\sphinxlineitem{Type}
\sphinxAtStartPar
character

\sphinxlineitem{Default}
\sphinxAtStartPar
none

\end{description}\end{quote}

\sphinxAtStartPar
Name of an atmospheric tracer species to be included in deposition modelling.

\end{fulllineitems}

\index{diffusion\_coeff\_io (in namelist JULES\_DEPOSITION\_SPECIES)@\spxentry{diffusion\_coeff\_io}\spxextra{in namelist JULES\_DEPOSITION\_SPECIES}|spxpagem}

\begin{fulllineitems}
\phantomsection\label{\detokenize{namelists/jules_deposition.nml:JULES_DEPOSITION_SPECIES::diffusion_coeff_io}}
\pysigstartsignatures
\pysigline{\sphinxcode{\sphinxupquote{JULES\_DEPOSITION\_SPECIES::}}\sphinxbfcode{\sphinxupquote{diffusion\_coeff\_io}}}
\pysigstopsignatures\begin{quote}\begin{description}
\sphinxlineitem{Type}
\sphinxAtStartPar
real

\sphinxlineitem{Default}
\sphinxAtStartPar
none

\end{description}\end{quote}

\sphinxAtStartPar
Diffusion coefficient (m$^{\text{\sphinxhyphen{}2}}$ s$^{\text{\sphinxhyphen{}1}}$).

\end{fulllineitems}

\index{rsurf\_std\_io (in namelist JULES\_DEPOSITION\_SPECIES)@\spxentry{rsurf\_std\_io}\spxextra{in namelist JULES\_DEPOSITION\_SPECIES}|spxpagem}

\begin{fulllineitems}
\phantomsection\label{\detokenize{namelists/jules_deposition.nml:JULES_DEPOSITION_SPECIES::rsurf_std_io}}
\pysigstartsignatures
\pysigline{\sphinxcode{\sphinxupquote{JULES\_DEPOSITION\_SPECIES::}}\sphinxbfcode{\sphinxupquote{rsurf\_std\_io}}}
\pysigstopsignatures\begin{quote}\begin{description}
\sphinxlineitem{Type}
\sphinxAtStartPar
real(ntype)

\sphinxlineitem{Default}
\sphinxAtStartPar
none

\end{description}\end{quote}

\sphinxAtStartPar
Standard value of surface resistance for each surface type (s m$^{\text{\sphinxhyphen{}1}}$).

\end{fulllineitems}


\begin{sphinxadmonition}{note}{Only used if \sphinxstyleliteralintitle{\sphinxupquote{dep\_species\_name\_io}} = \sphinxstyleliteralintitle{\sphinxupquote{NO2}}, \sphinxstyleliteralintitle{\sphinxupquote{O3}}, \sphinxstyleliteralintitle{\sphinxupquote{PAN}}, \sphinxstyleliteralintitle{\sphinxupquote{PPAN}}, \sphinxstyleliteralintitle{\sphinxupquote{MPAN}} or \sphinxstyleliteralintitle{\sphinxupquote{ONITU}}. Values provided for other species will be ignored.}
\index{diffusion\_corr\_io (in namelist JULES\_DEPOSITION\_SPECIES)@\spxentry{diffusion\_corr\_io}\spxextra{in namelist JULES\_DEPOSITION\_SPECIES}|spxpagem}

\begin{fulllineitems}
\phantomsection\label{\detokenize{namelists/jules_deposition.nml:JULES_DEPOSITION_SPECIES::diffusion_corr_io}}
\pysigstartsignatures
\pysigline{\sphinxcode{\sphinxupquote{JULES\_DEPOSITION\_SPECIES::}}\sphinxbfcode{\sphinxupquote{diffusion\_corr\_io}}}
\pysigstopsignatures\begin{quote}\begin{description}
\sphinxlineitem{Type}
\sphinxAtStartPar
real

\sphinxlineitem{Default}
\sphinxAtStartPar
none

\end{description}\end{quote}

\sphinxAtStartPar
Diffusion correction factor for stomatal resistance, accounting for the different diffusivities of water and other species (dimensionless).

\end{fulllineitems}

\end{sphinxadmonition}

\begin{sphinxadmonition}{note}{Only used if \sphinxstyleliteralintitle{\sphinxupquote{dep\_species\_name\_io}} = \sphinxstyleliteralintitle{\sphinxupquote{CO}}, \sphinxstyleliteralintitle{\sphinxupquote{NO2}}, \sphinxstyleliteralintitle{\sphinxupquote{O3}}, \sphinxstyleliteralintitle{\sphinxupquote{PAN}}, \sphinxstyleliteralintitle{\sphinxupquote{PPAN}}, \sphinxstyleliteralintitle{\sphinxupquote{MPAN}} or \sphinxstyleliteralintitle{\sphinxupquote{ONITU}}. Values provided for other species will be ignored.}
\index{r\_tundra\_io (in namelist JULES\_DEPOSITION\_SPECIES)@\spxentry{r\_tundra\_io}\spxextra{in namelist JULES\_DEPOSITION\_SPECIES}|spxpagem}

\begin{fulllineitems}
\phantomsection\label{\detokenize{namelists/jules_deposition.nml:JULES_DEPOSITION_SPECIES::r_tundra_io}}
\pysigstartsignatures
\pysigline{\sphinxcode{\sphinxupquote{JULES\_DEPOSITION\_SPECIES::}}\sphinxbfcode{\sphinxupquote{r\_tundra\_io}}}
\pysigstopsignatures\begin{quote}\begin{description}
\sphinxlineitem{Type}
\sphinxAtStartPar
real

\sphinxlineitem{Default}
\sphinxAtStartPar
none

\end{description}\end{quote}

\sphinxAtStartPar
Surface resistance used in tundra region (s m$^{\text{\sphinxhyphen{}1}}$).

\end{fulllineitems}

\end{sphinxadmonition}

\begin{sphinxadmonition}{note}{Only used if \sphinxstyleliteralintitle{\sphinxupquote{dep\_species\_name\_io}} = \sphinxstyleliteralintitle{\sphinxupquote{CH4}}.}
\index{ch4\_scaling\_io (in namelist JULES\_DEPOSITION\_SPECIES)@\spxentry{ch4\_scaling\_io}\spxextra{in namelist JULES\_DEPOSITION\_SPECIES}|spxpagem}

\begin{fulllineitems}
\phantomsection\label{\detokenize{namelists/jules_deposition.nml:JULES_DEPOSITION_SPECIES::ch4_scaling_io}}
\pysigstartsignatures
\pysigline{\sphinxcode{\sphinxupquote{JULES\_DEPOSITION\_SPECIES::}}\sphinxbfcode{\sphinxupquote{ch4\_scaling\_io}}}
\pysigstopsignatures\begin{quote}\begin{description}
\sphinxlineitem{Type}
\sphinxAtStartPar
real

\sphinxlineitem{Default}
\sphinxAtStartPar
none

\end{description}\end{quote}

\sphinxAtStartPar
Scaling applied to methane soil uptake (dimensionless). (Originally this was used to match the value from the IPCC Third Assessment Report.)

\end{fulllineitems}

\index{ch4dd\_tundra\_io (in namelist JULES\_DEPOSITION\_SPECIES)@\spxentry{ch4dd\_tundra\_io}\spxextra{in namelist JULES\_DEPOSITION\_SPECIES}|spxpagem}

\begin{fulllineitems}
\phantomsection\label{\detokenize{namelists/jules_deposition.nml:JULES_DEPOSITION_SPECIES::ch4dd_tundra_io}}
\pysigstartsignatures
\pysigline{\sphinxcode{\sphinxupquote{JULES\_DEPOSITION\_SPECIES::}}\sphinxbfcode{\sphinxupquote{ch4dd\_tundra\_io}}}
\pysigstopsignatures\begin{quote}\begin{description}
\sphinxlineitem{Type}
\sphinxAtStartPar
real(4)

\sphinxlineitem{Default}
\sphinxAtStartPar
none

\end{description}\end{quote}

\sphinxAtStartPar
Coefficients of cubic polynomial relating methane loss for tundra to temperature. Flux is in units of ug (CH$_{\text{4}}$) m$^{\text{\sphinxhyphen{}2}}$ s$^{\text{\sphinxhyphen{}1}}$.

\end{fulllineitems}

\end{sphinxadmonition}

\begin{sphinxadmonition}{note}{Only used if \sphinxstyleliteralintitle{\sphinxupquote{dep\_species\_name\_io}} = \sphinxstyleliteralintitle{\sphinxupquote{H2}}.}
\index{h2dd\_c\_io (in namelist JULES\_DEPOSITION\_SPECIES)@\spxentry{h2dd\_c\_io}\spxextra{in namelist JULES\_DEPOSITION\_SPECIES}|spxpagem}

\begin{fulllineitems}
\phantomsection\label{\detokenize{namelists/jules_deposition.nml:JULES_DEPOSITION_SPECIES::h2dd_c_io}}
\pysigstartsignatures
\pysigline{\sphinxcode{\sphinxupquote{JULES\_DEPOSITION\_SPECIES::}}\sphinxbfcode{\sphinxupquote{h2dd\_c\_io}}}
\pysigstopsignatures\begin{quote}\begin{description}
\sphinxlineitem{Type}
\sphinxAtStartPar
real(ntype)

\sphinxlineitem{Default}
\sphinxAtStartPar
none

\end{description}\end{quote}

\sphinxAtStartPar
Constant in quadratic function relating hydrogen deposition to soil moisture (s m$^{\text{\sphinxhyphen{}1}}$).

\end{fulllineitems}

\index{h2dd\_m\_io (in namelist JULES\_DEPOSITION\_SPECIES)@\spxentry{h2dd\_m\_io}\spxextra{in namelist JULES\_DEPOSITION\_SPECIES}|spxpagem}

\begin{fulllineitems}
\phantomsection\label{\detokenize{namelists/jules_deposition.nml:JULES_DEPOSITION_SPECIES::h2dd_m_io}}
\pysigstartsignatures
\pysigline{\sphinxcode{\sphinxupquote{JULES\_DEPOSITION\_SPECIES::}}\sphinxbfcode{\sphinxupquote{h2dd\_m\_io}}}
\pysigstopsignatures\begin{quote}\begin{description}
\sphinxlineitem{Type}
\sphinxAtStartPar
real(ntype)

\sphinxlineitem{Default}
\sphinxAtStartPar
none

\end{description}\end{quote}

\sphinxAtStartPar
Coefficient of first order term in quadratic function relating hydrogen deposition to soil moisture (s m$^{\text{\sphinxhyphen{}1}}$).

\end{fulllineitems}

\index{h2dd\_q\_io (in namelist JULES\_DEPOSITION\_SPECIES)@\spxentry{h2dd\_q\_io}\spxextra{in namelist JULES\_DEPOSITION\_SPECIES}|spxpagem}

\begin{fulllineitems}
\phantomsection\label{\detokenize{namelists/jules_deposition.nml:JULES_DEPOSITION_SPECIES::h2dd_q_io}}
\pysigstartsignatures
\pysigline{\sphinxcode{\sphinxupquote{JULES\_DEPOSITION\_SPECIES::}}\sphinxbfcode{\sphinxupquote{h2dd\_q\_io}}}
\pysigstopsignatures\begin{quote}\begin{description}
\sphinxlineitem{Type}
\sphinxAtStartPar
real(ntype)

\sphinxlineitem{Default}
\sphinxAtStartPar
none

\end{description}\end{quote}

\sphinxAtStartPar
Coefficient of second order term in quadratic function relating hydrogen deposition to soil moisture (s m$^{\text{\sphinxhyphen{}1}}$).

\end{fulllineitems}

\end{sphinxadmonition}

\begin{sphinxadmonition}{note}{Only used if \sphinxstyleliteralintitle{\sphinxupquote{dep\_species\_name\_io}} = \sphinxstyleliteralintitle{\sphinxupquote{HNO3}}, \sphinxstyleliteralintitle{\sphinxupquote{HONO2}} or \sphinxstyleliteralintitle{\sphinxupquote{ISON}}.}
\index{dd\_ice\_coeff (in namelist JULES\_DEPOSITION\_SPECIES)@\spxentry{dd\_ice\_coeff}\spxextra{in namelist JULES\_DEPOSITION\_SPECIES}|spxpagem}

\begin{fulllineitems}
\phantomsection\label{\detokenize{namelists/jules_deposition.nml:JULES_DEPOSITION_SPECIES::dd_ice_coeff}}
\pysigstartsignatures
\pysigline{\sphinxcode{\sphinxupquote{JULES\_DEPOSITION\_SPECIES::}}\sphinxbfcode{\sphinxupquote{dd\_ice\_coeff}}}
\pysigstopsignatures\begin{quote}\begin{description}
\sphinxlineitem{Type}
\sphinxAtStartPar
real(3)

\sphinxlineitem{Default}
\sphinxAtStartPar
none

\end{description}\end{quote}

\sphinxAtStartPar
Coefficients in quadratic function relating dry deposition over ice to temperature.

\end{fulllineitems}

\end{sphinxadmonition}

\begin{sphinxadmonition}{note}{Only used if \sphinxstyleliteralintitle{\sphinxupquote{dep\_species\_name\_io}} = \sphinxstyleliteralintitle{\sphinxupquote{O3}}.}
\index{cuticle\_o3\_io (in namelist JULES\_DEPOSITION\_SPECIES)@\spxentry{cuticle\_o3\_io}\spxextra{in namelist JULES\_DEPOSITION\_SPECIES}|spxpagem}

\begin{fulllineitems}
\phantomsection\label{\detokenize{namelists/jules_deposition.nml:JULES_DEPOSITION_SPECIES::cuticle_o3_io}}
\pysigstartsignatures
\pysigline{\sphinxcode{\sphinxupquote{JULES\_DEPOSITION\_SPECIES::}}\sphinxbfcode{\sphinxupquote{cuticle\_o3\_io}}}
\pysigstopsignatures\begin{quote}\begin{description}
\sphinxlineitem{Type}
\sphinxAtStartPar
real

\sphinxlineitem{Default}
\sphinxAtStartPar
none

\end{description}\end{quote}

\sphinxAtStartPar
Constant for calculation of cuticular resistance for ozone (s m$^{\text{\sphinxhyphen{}1}}$).

\end{fulllineitems}

\index{r\_wet\_soil\_o3\_io (in namelist JULES\_DEPOSITION\_SPECIES)@\spxentry{r\_wet\_soil\_o3\_io}\spxextra{in namelist JULES\_DEPOSITION\_SPECIES}|spxpagem}

\begin{fulllineitems}
\phantomsection\label{\detokenize{namelists/jules_deposition.nml:JULES_DEPOSITION_SPECIES::r_wet_soil_o3_io}}
\pysigstartsignatures
\pysigline{\sphinxcode{\sphinxupquote{JULES\_DEPOSITION\_SPECIES::}}\sphinxbfcode{\sphinxupquote{r\_wet\_soil\_o3\_io}}}
\pysigstopsignatures\begin{quote}\begin{description}
\sphinxlineitem{Type}
\sphinxAtStartPar
real

\sphinxlineitem{Default}
\sphinxAtStartPar
none

\end{description}\end{quote}

\sphinxAtStartPar
Wet soil surface resistance for ozone (s m$^{\text{\sphinxhyphen{}1}}$).

\end{fulllineitems}

\end{sphinxadmonition}

\sphinxstepscope


\section{\sphinxstyleliteralintitle{\sphinxupquote{jules\_snow.nml}}}
\label{\detokenize{namelists/jules_snow.nml:jules-snow-nml}}\label{\detokenize{namelists/jules_snow.nml::doc}}
\sphinxAtStartPar
This file sets the snow options and parameters. It contains one namelist called {\hyperref[\detokenize{namelists/jules_snow.nml:namelist-JULES_SNOW}]{\sphinxcrossref{\sphinxcode{\sphinxupquote{JULES\_SNOW}}}}}.


\subsection{\sphinxstyleliteralintitle{\sphinxupquote{JULES\_SNOW}} namelist members}
\label{\detokenize{namelists/jules_snow.nml:jules-snow-namelist-members}}
\sphinxAtStartPar
HCTN30 refers to Hadley Centre technical note 30, available from \sphinxhref{http://www.metoffice.gov.uk/learning/library/publications/science/climate-science-technical-notes}{the Met Office Library}.

\phantomsection\label{\detokenize{namelists/jules_snow.nml:namelist-JULES_SNOW}}\index{JULES\_SNOW (namelist)@\spxentry{JULES\_SNOW}\spxextra{namelist}|spxpagem}\index{nsmax (in namelist JULES\_SNOW)@\spxentry{nsmax}\spxextra{in namelist JULES\_SNOW}|spxpagem}

\begin{fulllineitems}
\phantomsection\label{\detokenize{namelists/jules_snow.nml:JULES_SNOW::nsmax}}
\pysigstartsignatures
\pysigline{\sphinxcode{\sphinxupquote{JULES\_SNOW::}}\sphinxbfcode{\sphinxupquote{nsmax}}}
\pysigstopsignatures\begin{quote}\begin{description}
\sphinxlineitem{Type}
\sphinxAtStartPar
integer

\sphinxlineitem{Permitted}
\sphinxAtStartPar
\textgreater{}= 0

\sphinxlineitem{Default}
\sphinxAtStartPar
0

\end{description}\end{quote}

\sphinxAtStartPar
Maximum possible number of snow layers.
\begin{description}
\sphinxlineitem{0}
\sphinxAtStartPar
A composite soil/snow layer is used. This value gives the behaviour found in JULES2.0 and earlier.

\sphinxlineitem{\textgreater{} 0}
\sphinxAtStartPar
The state of up to \sphinxcode{\sphinxupquote{nsmax}} separate snow layers is modelled. Values of \sphinxcode{\sphinxupquote{nsmax = 3}} or more are recommended.

\end{description}

\end{fulllineitems}

\index{l\_snowdep\_surf (in namelist JULES\_SNOW)@\spxentry{l\_snowdep\_surf}\spxextra{in namelist JULES\_SNOW}|spxpagem}

\begin{fulllineitems}
\phantomsection\label{\detokenize{namelists/jules_snow.nml:JULES_SNOW::l_snowdep_surf}}
\pysigstartsignatures
\pysigline{\sphinxcode{\sphinxupquote{JULES\_SNOW::}}\sphinxbfcode{\sphinxupquote{l\_snowdep\_surf}}}
\pysigstopsignatures\begin{quote}\begin{description}
\sphinxlineitem{Type}
\sphinxAtStartPar
logical

\sphinxlineitem{Default}
\sphinxAtStartPar
F

\end{description}\end{quote}
\begin{description}
\sphinxlineitem{TRUE}
\sphinxAtStartPar
Use equivalent canopy snow depth for surface calculations on surface tiles with a snow canopy.

\sphinxlineitem{FALSE}
\sphinxAtStartPar
No effect.

\end{description}

\end{fulllineitems}

\index{frac\_snow\_subl\_melt (in namelist JULES\_SNOW)@\spxentry{frac\_snow\_subl\_melt}\spxextra{in namelist JULES\_SNOW}|spxpagem}

\begin{fulllineitems}
\phantomsection\label{\detokenize{namelists/jules_snow.nml:JULES_SNOW::frac_snow_subl_melt}}
\pysigstartsignatures
\pysigline{\sphinxcode{\sphinxupquote{JULES\_SNOW::}}\sphinxbfcode{\sphinxupquote{frac\_snow\_subl\_melt}}}
\pysigstopsignatures\begin{quote}\begin{description}
\sphinxlineitem{Type}
\sphinxAtStartPar
integer

\sphinxlineitem{Permitted}
\sphinxAtStartPar
0 or 1

\sphinxlineitem{Default}
\sphinxAtStartPar
0

\end{description}\end{quote}

\sphinxAtStartPar
Switch for use of snow\sphinxhyphen{}cover fraction in the calculation of sublimation and melting.
\begin{enumerate}
\sphinxsetlistlabels{\arabic}{enumi}{enumii}{}{.}%
\setcounter{enumi}{-1}
\item {} 
\sphinxAtStartPar
Off

\item {} 
\sphinxAtStartPar
On

\end{enumerate}

\end{fulllineitems}

\index{graupel\_options (in namelist JULES\_SNOW)@\spxentry{graupel\_options}\spxextra{in namelist JULES\_SNOW}|spxpagem}

\begin{fulllineitems}
\phantomsection\label{\detokenize{namelists/jules_snow.nml:JULES_SNOW::graupel_options}}
\pysigstartsignatures
\pysigline{\sphinxcode{\sphinxupquote{JULES\_SNOW::}}\sphinxbfcode{\sphinxupquote{graupel\_options}}}
\pysigstopsignatures\begin{quote}\begin{description}
\sphinxlineitem{Type}
\sphinxAtStartPar
integer

\sphinxlineitem{Permitted}
\sphinxAtStartPar
0 or 1 or 2

\sphinxlineitem{Default}
\sphinxAtStartPar
0

\end{description}\end{quote}

\sphinxAtStartPar
Switch for treatment of graupel in the snow scheme
\begin{enumerate}
\sphinxsetlistlabels{\arabic}{enumi}{enumii}{}{.}%
\setcounter{enumi}{-1}
\item {} 
\sphinxAtStartPar
Include graupel as snowfall

\item {} 
\sphinxAtStartPar
Ignore graupel in the surface snowfall

\item {} 
\sphinxAtStartPar
Treat graupel separately

\end{enumerate}

\sphinxAtStartPar
Always “Include graupel as snowfall” (option 0) in standalone JULES because
separate snow and graupel driving data are not available.
If graupel is included in the UM surface snowfall diagnostic
then JULES can either include this graupel as snow in the surface scheme (option 0),
ignore this graupel completely, thereby breaking conservation
of water and energy in the coupled land\sphinxhyphen{}atmosphere model (option 1) or
treat graupel seperately (currently this only means allowing graupel to
fall straight through the canopy)

\end{fulllineitems}

\index{dzsnow (in namelist JULES\_SNOW)@\spxentry{dzsnow}\spxextra{in namelist JULES\_SNOW}|spxpagem}

\begin{fulllineitems}
\phantomsection\label{\detokenize{namelists/jules_snow.nml:JULES_SNOW::dzsnow}}
\pysigstartsignatures
\pysigline{\sphinxcode{\sphinxupquote{JULES\_SNOW::}}\sphinxbfcode{\sphinxupquote{dzsnow}}}
\pysigstopsignatures\begin{quote}\begin{description}
\sphinxlineitem{Type}
\sphinxAtStartPar
real(nsmax)

\sphinxlineitem{Default}
\sphinxAtStartPar
None

\end{description}\end{quote}

\sphinxAtStartPar
Prescribed thickness of each snow layer (m).

\sphinxAtStartPar
Only used if {\hyperref[\detokenize{namelists/jules_snow.nml:JULES_SNOW::nsmax}]{\sphinxcrossref{\sphinxcode{\sphinxupquote{nsmax}}}}} \textgreater{} 0.

\sphinxAtStartPar
The interpretation of \sphinxcode{\sphinxupquote{dzsnow}} is slightly complicated and an example of the evolution of the snow layers is given below.

\sphinxAtStartPar
\sphinxcode{\sphinxupquote{dzsnow}} gives the thickness of each layer when it is not the bottom layer.

\sphinxAtStartPar
For the top layer, the minimum thickness is \sphinxcode{\sphinxupquote{dzsnow(1)}} and the maximum thickness is \sphinxcode{\sphinxupquote{2 * dzsnow(1)}}. For all other layers \sphinxcode{\sphinxupquote{iz}}, the minimum thickness is \sphinxcode{\sphinxupquote{dzsnow(iz \sphinxhyphen{} 1)}}, i.e. the given thickness of the previous layer, and the maximum thickness is \sphinxcode{\sphinxupquote{2 * dzsnow(iz)}}, i.e. twice the layer \sphinxcode{\sphinxupquote{dzsnow}} value, except for the last possible layer ({\hyperref[\detokenize{namelists/jules_snow.nml:JULES_SNOW::nsmax}]{\sphinxcrossref{\sphinxcode{\sphinxupquote{nsmax}}}}}) which has no upper limit.

\sphinxAtStartPar
As a snowpack deepens, the bottom layer (closest to the soil; label this as layer \sphinxcode{\sphinxupquote{b}}) thickens until it reaches its maximum allowed thickness, at which point it will split into a layer of depth \sphinxcode{\sphinxupquote{dzsnow(b)}} and a new bottom layer \sphinxcode{\sphinxupquote{b + 1}} is added to hold the remaining snow. If a layer becomes thinner than its value in \sphinxcode{\sphinxupquote{dzsnow}} it is removed and the snow partitioned between the remaining layers. Whenever a layer splits or is removed, the properties of the layer (e.g. temperature) are allocated to the remaining layers.

\sphinxAtStartPar
Note that \sphinxcode{\sphinxupquote{dzsnow(nsmax)}}, the final thickness, is not used but a value must be input.

\end{fulllineitems}

\index{cansnowpft (in namelist JULES\_SNOW)@\spxentry{cansnowpft}\spxextra{in namelist JULES\_SNOW}|spxpagem}

\begin{fulllineitems}
\phantomsection\label{\detokenize{namelists/jules_snow.nml:JULES_SNOW::cansnowpft}}
\pysigstartsignatures
\pysigline{\sphinxcode{\sphinxupquote{JULES\_SNOW::}}\sphinxbfcode{\sphinxupquote{cansnowpft}}}
\pysigstopsignatures\begin{quote}\begin{description}
\sphinxlineitem{Type}
\sphinxAtStartPar
logical(npft)

\sphinxlineitem{Default}
\sphinxAtStartPar
F

\end{description}\end{quote}

\sphinxAtStartPar
Flag indicating whether snow can be held under the canopy of each PFT.

\sphinxAtStartPar
Only used if {\hyperref[\detokenize{namelists/jules_vegetation.nml:JULES_VEGETATION::can_model}]{\sphinxcrossref{\sphinxcode{\sphinxupquote{can\_model}}}}} = 4.

\sphinxAtStartPar
The model of snow under the canopy is currently only suitable for coniferous trees.
\begin{description}
\sphinxlineitem{TRUE}
\sphinxAtStartPar
Snow can be held under the canopy.

\sphinxlineitem{FALSE}
\sphinxAtStartPar
Snow cannot be held under the canopy.

\end{description}

\end{fulllineitems}


\begin{sphinxadmonition}{note}{Radiation parameters}
\index{r0 (in namelist JULES\_SNOW)@\spxentry{r0}\spxextra{in namelist JULES\_SNOW}|spxpagem}

\begin{fulllineitems}
\phantomsection\label{\detokenize{namelists/jules_snow.nml:JULES_SNOW::r0}}
\pysigstartsignatures
\pysigline{\sphinxcode{\sphinxupquote{JULES\_SNOW::}}\sphinxbfcode{\sphinxupquote{r0}}}
\pysigstopsignatures\begin{quote}\begin{description}
\sphinxlineitem{Type}
\sphinxAtStartPar
real

\sphinxlineitem{Default}
\sphinxAtStartPar
50.0

\end{description}\end{quote}

\sphinxAtStartPar
Grain size for fresh snow (μm).

\sphinxAtStartPar
Only used if {\hyperref[\detokenize{namelists/jules_radiation.nml:JULES_RADIATION::l_snow_albedo}]{\sphinxcrossref{\sphinxcode{\sphinxupquote{l\_snow\_albedo}}}}} = TRUE. See HCTN30 Eq.15.

\end{fulllineitems}

\index{rmax (in namelist JULES\_SNOW)@\spxentry{rmax}\spxextra{in namelist JULES\_SNOW}|spxpagem}

\begin{fulllineitems}
\phantomsection\label{\detokenize{namelists/jules_snow.nml:JULES_SNOW::rmax}}
\pysigstartsignatures
\pysigline{\sphinxcode{\sphinxupquote{JULES\_SNOW::}}\sphinxbfcode{\sphinxupquote{rmax}}}
\pysigstopsignatures\begin{quote}\begin{description}
\sphinxlineitem{Type}
\sphinxAtStartPar
real

\sphinxlineitem{Default}
\sphinxAtStartPar
2000.0

\end{description}\end{quote}

\sphinxAtStartPar
Maximum snow grain size (μm).

\sphinxAtStartPar
Only used if {\hyperref[\detokenize{namelists/jules_radiation.nml:JULES_RADIATION::l_snow_albedo}]{\sphinxcrossref{\sphinxcode{\sphinxupquote{l\_snow\_albedo}}}}} = TRUE. See HCTN30 p4.

\end{fulllineitems}

\index{snow\_ggr (in namelist JULES\_SNOW)@\spxentry{snow\_ggr}\spxextra{in namelist JULES\_SNOW}|spxpagem}

\begin{fulllineitems}
\phantomsection\label{\detokenize{namelists/jules_snow.nml:JULES_SNOW::snow_ggr}}
\pysigstartsignatures
\pysigline{\sphinxcode{\sphinxupquote{JULES\_SNOW::}}\sphinxbfcode{\sphinxupquote{snow\_ggr}}}
\pysigstopsignatures\begin{quote}\begin{description}
\sphinxlineitem{Type}
\sphinxAtStartPar
real(3)

\sphinxlineitem{Default}
\sphinxAtStartPar
0.6, 0.06, 0.23e6

\end{description}\end{quote}

\sphinxAtStartPar
Snow grain area growth rates (μm$^{\text{2}}$ s$^{\text{\sphinxhyphen{}1}}$).

\sphinxAtStartPar
Only used if {\hyperref[\detokenize{namelists/jules_radiation.nml:JULES_RADIATION::l_snow_albedo}]{\sphinxcrossref{\sphinxcode{\sphinxupquote{l\_snow\_albedo}}}}} = TRUE. See HCTN30 Eq.16.

\sphinxAtStartPar
The 3 values are for melting snow, cold fresh snow and cold aged snow respectively.

\end{fulllineitems}

\index{amax (in namelist JULES\_SNOW)@\spxentry{amax}\spxextra{in namelist JULES\_SNOW}|spxpagem}

\begin{fulllineitems}
\phantomsection\label{\detokenize{namelists/jules_snow.nml:JULES_SNOW::amax}}
\pysigstartsignatures
\pysigline{\sphinxcode{\sphinxupquote{JULES\_SNOW::}}\sphinxbfcode{\sphinxupquote{amax}}}
\pysigstopsignatures\begin{quote}\begin{description}
\sphinxlineitem{Type}
\sphinxAtStartPar
real(2)

\sphinxlineitem{Default}
\sphinxAtStartPar
0.98, 0.7

\end{description}\end{quote}

\sphinxAtStartPar
Maximum albedo for fresh snow.

\sphinxAtStartPar
Only used if {\hyperref[\detokenize{namelists/jules_radiation.nml:JULES_RADIATION::l_snow_albedo}]{\sphinxcrossref{\sphinxcode{\sphinxupquote{l\_snow\_albedo}}}}} or {\hyperref[\detokenize{namelists/jules_surface.nml:JULES_SURFACE::l_elev_land_ice}]{\sphinxcrossref{\sphinxcode{\sphinxupquote{l\_elev\_land\_ice}}}}}
are true

\sphinxAtStartPar
Values 1 and 2 are for VIS and NIR wavebands respectively.

\end{fulllineitems}

\index{aicemax (in namelist JULES\_SNOW)@\spxentry{aicemax}\spxextra{in namelist JULES\_SNOW}|spxpagem}

\begin{fulllineitems}
\phantomsection\label{\detokenize{namelists/jules_snow.nml:JULES_SNOW::aicemax}}
\pysigstartsignatures
\pysigline{\sphinxcode{\sphinxupquote{JULES\_SNOW::}}\sphinxbfcode{\sphinxupquote{aicemax}}}
\pysigstopsignatures\begin{quote}\begin{description}
\sphinxlineitem{Type}
\sphinxAtStartPar
real(2)

\sphinxlineitem{Default}
\sphinxAtStartPar
0.78, 0.36

\end{description}\end{quote}

\sphinxAtStartPar
Maximum albedo for bare ice

\sphinxAtStartPar
Only used if {\hyperref[\detokenize{namelists/jules_surface.nml:JULES_SURFACE::l_elev_land_ice}]{\sphinxcrossref{\sphinxcode{\sphinxupquote{l\_elev\_land\_ice}}}}} = TRUE. See also \sphinxtitleref{rho\_firn\_albedo}

\sphinxAtStartPar
Values 1 and 2 are for VIS and NIR wavebands respectively.

\end{fulllineitems}

\index{maskd (in namelist JULES\_SNOW)@\spxentry{maskd}\spxextra{in namelist JULES\_SNOW}|spxpagem}

\begin{fulllineitems}
\phantomsection\label{\detokenize{namelists/jules_snow.nml:JULES_SNOW::maskd}}
\pysigstartsignatures
\pysigline{\sphinxcode{\sphinxupquote{JULES\_SNOW::}}\sphinxbfcode{\sphinxupquote{maskd}}}
\pysigstopsignatures\begin{quote}\begin{description}
\sphinxlineitem{Type}
\sphinxAtStartPar
real

\sphinxlineitem{Default}
\sphinxAtStartPar
50.0

\end{description}\end{quote}

\sphinxAtStartPar
Used in exponent of equation weighting snow\sphinxhyphen{}covered and snow\sphinxhyphen{}free albedo.

\end{fulllineitems}

\index{dtland (in namelist JULES\_SNOW)@\spxentry{dtland}\spxextra{in namelist JULES\_SNOW}|spxpagem}

\begin{fulllineitems}
\phantomsection\label{\detokenize{namelists/jules_snow.nml:JULES_SNOW::dtland}}
\pysigstartsignatures
\pysigline{\sphinxcode{\sphinxupquote{JULES\_SNOW::}}\sphinxbfcode{\sphinxupquote{dtland}}}
\pysigstopsignatures\begin{quote}\begin{description}
\sphinxlineitem{Type}
\sphinxAtStartPar
real

\sphinxlineitem{Default}
\sphinxAtStartPar
2.0

\end{description}\end{quote}

\sphinxAtStartPar
Degrees Celsius below zero at which snow albedo equals cold deep snow albedo.

\sphinxAtStartPar
Only used if {\hyperref[\detokenize{namelists/jules_radiation.nml:JULES_RADIATION::l_snow_albedo}]{\sphinxcrossref{\sphinxcode{\sphinxupquote{l\_snow\_albedo}}}}} = FALSE. This is 2.0 in HCTN30 Eq4.

\end{fulllineitems}

\index{kland\_numerator (in namelist JULES\_SNOW)@\spxentry{kland\_numerator}\spxextra{in namelist JULES\_SNOW}|spxpagem}

\begin{fulllineitems}
\phantomsection\label{\detokenize{namelists/jules_snow.nml:JULES_SNOW::kland_numerator}}
\pysigstartsignatures
\pysigline{\sphinxcode{\sphinxupquote{JULES\_SNOW::}}\sphinxbfcode{\sphinxupquote{kland\_numerator}}}
\pysigstopsignatures\begin{quote}\begin{description}
\sphinxlineitem{Type}
\sphinxAtStartPar
real

\sphinxlineitem{Default}
\sphinxAtStartPar
0.3

\end{description}\end{quote}

\sphinxAtStartPar
Used in snow\sphinxhyphen{}ageing effect on albedo.

\sphinxAtStartPar
Only used if {\hyperref[\detokenize{namelists/jules_radiation.nml:JULES_RADIATION::l_snow_albedo}]{\sphinxcrossref{\sphinxcode{\sphinxupquote{l\_snow\_albedo}}}}} = FALSE.

\sphinxAtStartPar
Must not be zero.

\sphinxAtStartPar
\sphinxcode{\sphinxupquote{kland}} is computed by dividing this value by {\hyperref[\detokenize{namelists/jules_snow.nml:JULES_SNOW::dtland}]{\sphinxcrossref{\sphinxcode{\sphinxupquote{dtland}}}}} \sphinxhyphen{} see HCTN30 Eq4.

\end{fulllineitems}

\index{can\_clump (in namelist JULES\_SNOW)@\spxentry{can\_clump}\spxextra{in namelist JULES\_SNOW}|spxpagem}

\begin{fulllineitems}
\phantomsection\label{\detokenize{namelists/jules_snow.nml:JULES_SNOW::can_clump}}
\pysigstartsignatures
\pysigline{\sphinxcode{\sphinxupquote{JULES\_SNOW::}}\sphinxbfcode{\sphinxupquote{can\_clump}}}
\pysigstopsignatures\begin{quote}\begin{description}
\sphinxlineitem{Type}
\sphinxAtStartPar
real(npft)

\sphinxlineitem{Default}
\sphinxAtStartPar
MDI

\end{description}\end{quote}

\sphinxAtStartPar
Clumping parameter for snow on the canopy in calculation of albedo.

\sphinxAtStartPar
Only used if {\hyperref[\detokenize{namelists/jules_vegetation.nml:JULES_VEGETATION::can_model}]{\sphinxcrossref{\sphinxcode{\sphinxupquote{can\_model}}}}} = 4, {\hyperref[\detokenize{namelists/jules_snow.nml:JULES_SNOW::cansnowpft}]{\sphinxcrossref{\sphinxcode{\sphinxupquote{cansnowpft}}}}} = TRUE on that surface tile and {\hyperref[\detokenize{namelists/jules_radiation.nml:JULES_RADIATION::l_embedded_snow}]{\sphinxcrossref{\sphinxcode{\sphinxupquote{l\_embedded\_snow}}}}} = TRUE.

\sphinxAtStartPar
The model of snow under the canopy is currently only suitable
for coniferous trees.

\sphinxAtStartPar
The inverse of this parameter specifies the fraction of the
canopy over which snow is distributed when calculating the albedo.

\end{fulllineitems}

\index{n\_lai\_exposed (in namelist JULES\_SNOW)@\spxentry{n\_lai\_exposed}\spxextra{in namelist JULES\_SNOW}|spxpagem}

\begin{fulllineitems}
\phantomsection\label{\detokenize{namelists/jules_snow.nml:JULES_SNOW::n_lai_exposed}}
\pysigstartsignatures
\pysigline{\sphinxcode{\sphinxupquote{JULES\_SNOW::}}\sphinxbfcode{\sphinxupquote{n\_lai\_exposed}}}
\pysigstopsignatures\begin{quote}\begin{description}
\sphinxlineitem{Type}
\sphinxAtStartPar
real(npft)

\sphinxlineitem{Default}
\sphinxAtStartPar
MDI

\end{description}\end{quote}

\sphinxAtStartPar
LAI distribution parameter for calculation of snow albedo.

\sphinxAtStartPar
A power\sphinxhyphen{}law distribution of leaf area density is assumed within
the canopy for calculating masking of snow by vegetation using
the embedded scheme. Larger values imply greater densities
toward the base of the canopy.

\sphinxAtStartPar
Only used if {\hyperref[\detokenize{namelists/jules_radiation.nml:JULES_RADIATION::l_embedded_snow}]{\sphinxcrossref{\sphinxcode{\sphinxupquote{l\_embedded\_snow}}}}} = TRUE.

\end{fulllineitems}

\index{lai\_alb\_lim\_sn (in namelist JULES\_SNOW)@\spxentry{lai\_alb\_lim\_sn}\spxextra{in namelist JULES\_SNOW}|spxpagem}

\begin{fulllineitems}
\phantomsection\label{\detokenize{namelists/jules_snow.nml:JULES_SNOW::lai_alb_lim_sn}}
\pysigstartsignatures
\pysigline{\sphinxcode{\sphinxupquote{JULES\_SNOW::}}\sphinxbfcode{\sphinxupquote{lai\_alb\_lim\_sn}}}
\pysigstopsignatures\begin{quote}\begin{description}
\sphinxlineitem{Type}
\sphinxAtStartPar
real(npft)

\sphinxlineitem{Default}
\sphinxAtStartPar
MDI

\end{description}\end{quote}

\sphinxAtStartPar
Minimum LAI in calculation of albedo in the presence of snow.

\sphinxAtStartPar
A minimum albedo is imposed when calculating the albedo of
plant canopies (historically 0.5). This parameter allows it
to be set for each PFT in the presence of snow. A separate variable,
{\hyperref[\detokenize{namelists/pft_params.nml:JULES_PFTPARM::lai_alb_lim_io}]{\sphinxcrossref{\sphinxcode{\sphinxupquote{lai\_alb\_lim\_io}}}}} is used in the absence of snow.

\end{fulllineitems}

\end{sphinxadmonition}

\begin{sphinxadmonition}{note}{Other snow parameters}
\index{rho\_snow\_const (in namelist JULES\_SNOW)@\spxentry{rho\_snow\_const}\spxextra{in namelist JULES\_SNOW}|spxpagem}

\begin{fulllineitems}
\phantomsection\label{\detokenize{namelists/jules_snow.nml:JULES_SNOW::rho_snow_const}}
\pysigstartsignatures
\pysigline{\sphinxcode{\sphinxupquote{JULES\_SNOW::}}\sphinxbfcode{\sphinxupquote{rho\_snow\_const}}}
\pysigstopsignatures\begin{quote}\begin{description}
\sphinxlineitem{Type}
\sphinxAtStartPar
real

\sphinxlineitem{Default}
\sphinxAtStartPar
250.0

\end{description}\end{quote}

\sphinxAtStartPar
Constant density of lying snow (kg m$^{\text{\sphinxhyphen{}3}}$).

\sphinxAtStartPar
This value is used if {\hyperref[\detokenize{namelists/jules_snow.nml:JULES_SNOW::nsmax}]{\sphinxcrossref{\sphinxcode{\sphinxupquote{nsmax}}}}} = 0, in which case all
snow is modelled as a single layer of constant density. If
{\hyperref[\detokenize{namelists/jules_snow.nml:JULES_SNOW::nsmax}]{\sphinxcrossref{\sphinxcode{\sphinxupquote{nsmax}}}}} \textgreater{} 0, snow density is prognostic.

\end{fulllineitems}

\index{rho\_snow\_fresh (in namelist JULES\_SNOW)@\spxentry{rho\_snow\_fresh}\spxextra{in namelist JULES\_SNOW}|spxpagem}

\begin{fulllineitems}
\phantomsection\label{\detokenize{namelists/jules_snow.nml:JULES_SNOW::rho_snow_fresh}}
\pysigstartsignatures
\pysigline{\sphinxcode{\sphinxupquote{JULES\_SNOW::}}\sphinxbfcode{\sphinxupquote{rho\_snow\_fresh}}}
\pysigstopsignatures\begin{quote}\begin{description}
\sphinxlineitem{Type}
\sphinxAtStartPar
real

\sphinxlineitem{Default}
\sphinxAtStartPar
100.0

\end{description}\end{quote}

\sphinxAtStartPar
Density of fresh snow (kg m$^{\text{\sphinxhyphen{}3}}$).

\sphinxAtStartPar
This value is only used if {\hyperref[\detokenize{namelists/jules_snow.nml:JULES_SNOW::nsmax}]{\sphinxcrossref{\sphinxcode{\sphinxupquote{nsmax}}}}} \textgreater{} 0.

\end{fulllineitems}

\index{rho\_firn\_albedo (in namelist JULES\_SNOW)@\spxentry{rho\_firn\_albedo}\spxextra{in namelist JULES\_SNOW}|spxpagem}

\begin{fulllineitems}
\phantomsection\label{\detokenize{namelists/jules_snow.nml:JULES_SNOW::rho_firn_albedo}}
\pysigstartsignatures
\pysigline{\sphinxcode{\sphinxupquote{JULES\_SNOW::}}\sphinxbfcode{\sphinxupquote{rho\_firn\_albedo}}}
\pysigstopsignatures\begin{quote}\begin{description}
\sphinxlineitem{Type}
\sphinxAtStartPar
real

\sphinxlineitem{Default}
\sphinxAtStartPar
550.0

\end{description}\end{quote}

\sphinxAtStartPar
If {\hyperref[\detokenize{namelists/jules_surface.nml:JULES_SURFACE::l_elev_land_ice}]{\sphinxcrossref{\sphinxcode{\sphinxupquote{l\_elev\_land\_ice}}}}} = TRUE, this is the threshold density (as measured over the \textasciitilde{}top 10cm, depending
on how the dzsnow layers are specified) at which the grain\sphinxhyphen{}size calculation of prognostic snow albedo will switch to one
dependent on the surface density of the snowpack. Albedo is linearly scaled between \sphinxtitleref{amax} for \sphinxtitleref{rho\_snow\_const} and \sphinxtitleref{aicemax}
for rho\_ice=917 kg/m\textasciicircum{}3.

\end{fulllineitems}

\index{snow\_hcon (in namelist JULES\_SNOW)@\spxentry{snow\_hcon}\spxextra{in namelist JULES\_SNOW}|spxpagem}

\begin{fulllineitems}
\phantomsection\label{\detokenize{namelists/jules_snow.nml:JULES_SNOW::snow_hcon}}
\pysigstartsignatures
\pysigline{\sphinxcode{\sphinxupquote{JULES\_SNOW::}}\sphinxbfcode{\sphinxupquote{snow\_hcon}}}
\pysigstopsignatures\begin{quote}\begin{description}
\sphinxlineitem{Type}
\sphinxAtStartPar
real

\sphinxlineitem{Default}
\sphinxAtStartPar
0.265

\end{description}\end{quote}

\sphinxAtStartPar
Thermal conductivity of lying snow (W m$^{\text{\sphinxhyphen{}1}}$ K$^{\text{\sphinxhyphen{}1}}$).

\sphinxAtStartPar
See HCTN30 Eq.42.

\end{fulllineitems}

\index{snow\_hcap (in namelist JULES\_SNOW)@\spxentry{snow\_hcap}\spxextra{in namelist JULES\_SNOW}|spxpagem}

\begin{fulllineitems}
\phantomsection\label{\detokenize{namelists/jules_snow.nml:JULES_SNOW::snow_hcap}}
\pysigstartsignatures
\pysigline{\sphinxcode{\sphinxupquote{JULES\_SNOW::}}\sphinxbfcode{\sphinxupquote{snow\_hcap}}}
\pysigstopsignatures\begin{quote}\begin{description}
\sphinxlineitem{Type}
\sphinxAtStartPar
real

\sphinxlineitem{Default}
\sphinxAtStartPar
0.63e6

\end{description}\end{quote}

\sphinxAtStartPar
Thermal capacity of lying snow (J K$^{\text{\sphinxhyphen{}1}}$ m$^{\text{\sphinxhyphen{}3}}$).

\end{fulllineitems}

\index{snowliqcap (in namelist JULES\_SNOW)@\spxentry{snowliqcap}\spxextra{in namelist JULES\_SNOW}|spxpagem}

\begin{fulllineitems}
\phantomsection\label{\detokenize{namelists/jules_snow.nml:JULES_SNOW::snowliqcap}}
\pysigstartsignatures
\pysigline{\sphinxcode{\sphinxupquote{JULES\_SNOW::}}\sphinxbfcode{\sphinxupquote{snowliqcap}}}
\pysigstopsignatures\begin{quote}\begin{description}
\sphinxlineitem{Type}
\sphinxAtStartPar
real

\sphinxlineitem{Default}
\sphinxAtStartPar
0.05

\end{description}\end{quote}

\sphinxAtStartPar
Liquid water holding capacity of lying snow, as a fraction of snow mass.

\sphinxAtStartPar
Only used if {\hyperref[\detokenize{namelists/jules_snow.nml:JULES_SNOW::nsmax}]{\sphinxcrossref{\sphinxcode{\sphinxupquote{nsmax}}}}} \textgreater{} 0.

\end{fulllineitems}

\index{snowinterceptfact (in namelist JULES\_SNOW)@\spxentry{snowinterceptfact}\spxextra{in namelist JULES\_SNOW}|spxpagem}

\begin{fulllineitems}
\phantomsection\label{\detokenize{namelists/jules_snow.nml:JULES_SNOW::snowinterceptfact}}
\pysigstartsignatures
\pysigline{\sphinxcode{\sphinxupquote{JULES\_SNOW::}}\sphinxbfcode{\sphinxupquote{snowinterceptfact}}}
\pysigstopsignatures\begin{quote}\begin{description}
\sphinxlineitem{Type}
\sphinxAtStartPar
real

\sphinxlineitem{Default}
\sphinxAtStartPar
0.7

\end{description}\end{quote}

\sphinxAtStartPar
Constant in relationship between mass of intercepted snow and snowfall rate.

\sphinxAtStartPar
Only used if {\hyperref[\detokenize{namelists/jules_vegetation.nml:JULES_VEGETATION::can_model}]{\sphinxcrossref{\sphinxcode{\sphinxupquote{can\_model}}}}} = 4.

\end{fulllineitems}

\index{snowloadlai (in namelist JULES\_SNOW)@\spxentry{snowloadlai}\spxextra{in namelist JULES\_SNOW}|spxpagem}

\begin{fulllineitems}
\phantomsection\label{\detokenize{namelists/jules_snow.nml:JULES_SNOW::snowloadlai}}
\pysigstartsignatures
\pysigline{\sphinxcode{\sphinxupquote{JULES\_SNOW::}}\sphinxbfcode{\sphinxupquote{snowloadlai}}}
\pysigstopsignatures\begin{quote}\begin{description}
\sphinxlineitem{Type}
\sphinxAtStartPar
real

\sphinxlineitem{Default}
\sphinxAtStartPar
4.4

\end{description}\end{quote}

\sphinxAtStartPar
Ratio of maximum canopy snow load to leaf area index (kg m$^{\text{\sphinxhyphen{}2}}$).

\sphinxAtStartPar
Only used if {\hyperref[\detokenize{namelists/jules_vegetation.nml:JULES_VEGETATION::can_model}]{\sphinxcrossref{\sphinxcode{\sphinxupquote{can\_model}}}}} = 4.

\end{fulllineitems}

\index{snowunloadfact (in namelist JULES\_SNOW)@\spxentry{snowunloadfact}\spxextra{in namelist JULES\_SNOW}|spxpagem}

\begin{fulllineitems}
\phantomsection\label{\detokenize{namelists/jules_snow.nml:JULES_SNOW::snowunloadfact}}
\pysigstartsignatures
\pysigline{\sphinxcode{\sphinxupquote{JULES\_SNOW::}}\sphinxbfcode{\sphinxupquote{snowunloadfact}}}
\pysigstopsignatures\begin{quote}\begin{description}
\sphinxlineitem{Type}
\sphinxAtStartPar
real

\sphinxlineitem{Default}
\sphinxAtStartPar
0.4

\end{description}\end{quote}

\sphinxAtStartPar
Constant in relationship between canopy snow unloading and canopy snow melt rate.

\sphinxAtStartPar
Only used if {\hyperref[\detokenize{namelists/jules_vegetation.nml:JULES_VEGETATION::can_model}]{\sphinxcrossref{\sphinxcode{\sphinxupquote{can\_model}}}}} = 4.

\end{fulllineitems}

\index{unload\_rate\_cnst (in namelist JULES\_SNOW)@\spxentry{unload\_rate\_cnst}\spxextra{in namelist JULES\_SNOW}|spxpagem}

\begin{fulllineitems}
\phantomsection\label{\detokenize{namelists/jules_snow.nml:JULES_SNOW::unload_rate_cnst}}
\pysigstartsignatures
\pysigline{\sphinxcode{\sphinxupquote{JULES\_SNOW::}}\sphinxbfcode{\sphinxupquote{unload\_rate\_cnst}}}
\pysigstopsignatures\begin{quote}\begin{description}
\sphinxlineitem{Type}
\sphinxAtStartPar
real(npft)

\sphinxlineitem{Default}
\sphinxAtStartPar
MDI

\end{description}\end{quote}

\sphinxAtStartPar
Constant term in the background unloading rate for snow on the canopy.

\sphinxAtStartPar
Only used if {\hyperref[\detokenize{namelists/jules_vegetation.nml:JULES_VEGETATION::can_model}]{\sphinxcrossref{\sphinxcode{\sphinxupquote{can\_model}}}}} = 4 and {\hyperref[\detokenize{namelists/jules_snow.nml:JULES_SNOW::cansnowpft}]{\sphinxcrossref{\sphinxcode{\sphinxupquote{cansnowpft}}}}} = TRUE on that surface tile.

\end{fulllineitems}

\index{unload\_rate\_u (in namelist JULES\_SNOW)@\spxentry{unload\_rate\_u}\spxextra{in namelist JULES\_SNOW}|spxpagem}

\begin{fulllineitems}
\phantomsection\label{\detokenize{namelists/jules_snow.nml:JULES_SNOW::unload_rate_u}}
\pysigstartsignatures
\pysigline{\sphinxcode{\sphinxupquote{JULES\_SNOW::}}\sphinxbfcode{\sphinxupquote{unload\_rate\_u}}}
\pysigstopsignatures\begin{quote}\begin{description}
\sphinxlineitem{Type}
\sphinxAtStartPar
real(npft)

\sphinxlineitem{Default}
\sphinxAtStartPar
MDI

\end{description}\end{quote}

\sphinxAtStartPar
Term proportional to wind speed in unloading rate for snow on the canopy.

\sphinxAtStartPar
Only used if {\hyperref[\detokenize{namelists/jules_vegetation.nml:JULES_VEGETATION::can_model}]{\sphinxcrossref{\sphinxcode{\sphinxupquote{can\_model}}}}} = 4 and {\hyperref[\detokenize{namelists/jules_snow.nml:JULES_SNOW::cansnowpft}]{\sphinxcrossref{\sphinxcode{\sphinxupquote{cansnowpft}}}}} = TRUE on that surface tile.

\end{fulllineitems}

\index{i\_snow\_cond\_parm (in namelist JULES\_SNOW)@\spxentry{i\_snow\_cond\_parm}\spxextra{in namelist JULES\_SNOW}|spxpagem}

\begin{fulllineitems}
\phantomsection\label{\detokenize{namelists/jules_snow.nml:JULES_SNOW::i_snow_cond_parm}}
\pysigstartsignatures
\pysigline{\sphinxcode{\sphinxupquote{JULES\_SNOW::}}\sphinxbfcode{\sphinxupquote{i\_snow\_cond\_parm}}}
\pysigstopsignatures\begin{quote}\begin{description}
\sphinxlineitem{Type}
\sphinxAtStartPar
integer

\sphinxlineitem{Permitted}
\sphinxAtStartPar
0 or 1

\sphinxlineitem{Default}
\sphinxAtStartPar
MDI

\end{description}\end{quote}

\sphinxAtStartPar
Scheme used to calculate the conductivity of snow

\sphinxAtStartPar
Two parametrizations of snow conductivity are available
taken from the papers of {\hyperref[\detokenize{namelists/jules_snow.nml:references-snow}]{\sphinxcrossref{\DUrole{std,std-ref}{Yen (1981)}}}} and
{\hyperref[\detokenize{namelists/jules_snow.nml:references-snow}]{\sphinxcrossref{\DUrole{std,std-ref}{Calonne et al. (2011)}}}}.

\sphinxAtStartPar
Only used if {\hyperref[\detokenize{namelists/jules_snow.nml:JULES_SNOW::nsmax}]{\sphinxcrossref{\sphinxcode{\sphinxupquote{nsmax}}}}} \textgreater{} 0.


\begin{savenotes}\sphinxattablestart
\centering
\begin{tabulary}{\linewidth}[t]{|T|T|}
\hline

\sphinxAtStartPar
0
&
\sphinxAtStartPar
Yen (1981)
\\
\hline
\sphinxAtStartPar
1
&
\sphinxAtStartPar
Calonne et al. (2011)
\\
\hline
\end{tabulary}
\par
\sphinxattableend\end{savenotes}

\end{fulllineitems}

\index{l\_et\_metamorph (in namelist JULES\_SNOW)@\spxentry{l\_et\_metamorph}\spxextra{in namelist JULES\_SNOW}|spxpagem}

\begin{fulllineitems}
\phantomsection\label{\detokenize{namelists/jules_snow.nml:JULES_SNOW::l_et_metamorph}}
\pysigstartsignatures
\pysigline{\sphinxcode{\sphinxupquote{JULES\_SNOW::}}\sphinxbfcode{\sphinxupquote{l\_et\_metamorph}}}
\pysigstopsignatures\begin{quote}\begin{description}
\sphinxlineitem{Type}
\sphinxAtStartPar
logical

\sphinxlineitem{Default}
\sphinxAtStartPar
F

\end{description}\end{quote}
\begin{description}
\sphinxlineitem{TRUE}
\sphinxAtStartPar
Include the effect of thermal metamorphism on the snow density.

\sphinxlineitem{FALSE}
\sphinxAtStartPar
No effect.

\end{description}

\sphinxAtStartPar
This parametrization follows the form used by eg. Dutra et al. (2010)

\end{fulllineitems}

\index{l\_snow\_infilt (in namelist JULES\_SNOW)@\spxentry{l\_snow\_infilt}\spxextra{in namelist JULES\_SNOW}|spxpagem}

\begin{fulllineitems}
\phantomsection\label{\detokenize{namelists/jules_snow.nml:JULES_SNOW::l_snow_infilt}}
\pysigstartsignatures
\pysigline{\sphinxcode{\sphinxupquote{JULES\_SNOW::}}\sphinxbfcode{\sphinxupquote{l\_snow\_infilt}}}
\pysigstopsignatures\begin{quote}\begin{description}
\sphinxlineitem{Type}
\sphinxAtStartPar
logical

\sphinxlineitem{Default}
\sphinxAtStartPar
F

\end{description}\end{quote}
\begin{description}
\sphinxlineitem{TRUE}
\sphinxAtStartPar
Pass rainfall and melting from the canopy to the snowpack as infiltration.

\sphinxlineitem{FALSE}
\sphinxAtStartPar
No effect.

\end{description}

\end{fulllineitems}

\index{l\_snow\_nocan\_hc (in namelist JULES\_SNOW)@\spxentry{l\_snow\_nocan\_hc}\spxextra{in namelist JULES\_SNOW}|spxpagem}

\begin{fulllineitems}
\phantomsection\label{\detokenize{namelists/jules_snow.nml:JULES_SNOW::l_snow_nocan_hc}}
\pysigstartsignatures
\pysigline{\sphinxcode{\sphinxupquote{JULES\_SNOW::}}\sphinxbfcode{\sphinxupquote{l\_snow\_nocan\_hc}}}
\pysigstopsignatures\begin{quote}\begin{description}
\sphinxlineitem{Type}
\sphinxAtStartPar
logical

\sphinxlineitem{Default}
\sphinxAtStartPar
F

\end{description}\end{quote}
\begin{description}
\sphinxlineitem{TRUE}
\sphinxAtStartPar
Do not include the canopy heat capacity in the surface energy balance at the top of the snow pack on surface tiles without a canopy snow model.

\sphinxlineitem{FALSE}
\sphinxAtStartPar
The canopy heat capacity is include in the surface energy balance at the top of the snow pack.

\end{description}

\end{fulllineitems}

\index{a\_snow\_et (in namelist JULES\_SNOW)@\spxentry{a\_snow\_et}\spxextra{in namelist JULES\_SNOW}|spxpagem}

\begin{fulllineitems}
\phantomsection\label{\detokenize{namelists/jules_snow.nml:JULES_SNOW::a_snow_et}}
\pysigstartsignatures
\pysigline{\sphinxcode{\sphinxupquote{JULES\_SNOW::}}\sphinxbfcode{\sphinxupquote{a\_snow\_et}}}
\pysigstopsignatures\begin{quote}\begin{description}
\sphinxlineitem{Type}
\sphinxAtStartPar
real

\sphinxlineitem{Default}
\sphinxAtStartPar
MDI

\end{description}\end{quote}

\sphinxAtStartPar
Constant in parametrization of thermal metamorphism.

\sphinxAtStartPar
Only used if {\hyperref[\detokenize{namelists/jules_snow.nml:JULES_SNOW::l_et_metamorph}]{\sphinxcrossref{\sphinxcode{\sphinxupquote{l\_et\_metamorph}}}}} = TRUE.

\end{fulllineitems}

\index{b\_snow\_et (in namelist JULES\_SNOW)@\spxentry{b\_snow\_et}\spxextra{in namelist JULES\_SNOW}|spxpagem}

\begin{fulllineitems}
\phantomsection\label{\detokenize{namelists/jules_snow.nml:JULES_SNOW::b_snow_et}}
\pysigstartsignatures
\pysigline{\sphinxcode{\sphinxupquote{JULES\_SNOW::}}\sphinxbfcode{\sphinxupquote{b\_snow\_et}}}
\pysigstopsignatures\begin{quote}\begin{description}
\sphinxlineitem{Type}
\sphinxAtStartPar
real

\sphinxlineitem{Default}
\sphinxAtStartPar
MDI

\end{description}\end{quote}

\sphinxAtStartPar
Constant in parametrization of thermal metamorphism.

\sphinxAtStartPar
Only used if {\hyperref[\detokenize{namelists/jules_snow.nml:JULES_SNOW::l_et_metamorph}]{\sphinxcrossref{\sphinxcode{\sphinxupquote{l\_et\_metamorph}}}}} = TRUE.

\end{fulllineitems}

\index{c\_snow\_et (in namelist JULES\_SNOW)@\spxentry{c\_snow\_et}\spxextra{in namelist JULES\_SNOW}|spxpagem}

\begin{fulllineitems}
\phantomsection\label{\detokenize{namelists/jules_snow.nml:JULES_SNOW::c_snow_et}}
\pysigstartsignatures
\pysigline{\sphinxcode{\sphinxupquote{JULES\_SNOW::}}\sphinxbfcode{\sphinxupquote{c\_snow\_et}}}
\pysigstopsignatures\begin{quote}\begin{description}
\sphinxlineitem{Type}
\sphinxAtStartPar
real

\sphinxlineitem{Default}
\sphinxAtStartPar
MDI

\end{description}\end{quote}

\sphinxAtStartPar
Constant in parametrization of thermal metamorphism.

\sphinxAtStartPar
Only used if {\hyperref[\detokenize{namelists/jules_snow.nml:JULES_SNOW::l_et_metamorph}]{\sphinxcrossref{\sphinxcode{\sphinxupquote{l\_et\_metamorph}}}}} = TRUE.

\end{fulllineitems}

\index{rho\_snow\_et\_crit (in namelist JULES\_SNOW)@\spxentry{rho\_snow\_et\_crit}\spxextra{in namelist JULES\_SNOW}|spxpagem}

\begin{fulllineitems}
\phantomsection\label{\detokenize{namelists/jules_snow.nml:JULES_SNOW::rho_snow_et_crit}}
\pysigstartsignatures
\pysigline{\sphinxcode{\sphinxupquote{JULES\_SNOW::}}\sphinxbfcode{\sphinxupquote{rho\_snow\_et\_crit}}}
\pysigstopsignatures\begin{quote}\begin{description}
\sphinxlineitem{Type}
\sphinxAtStartPar
real

\sphinxlineitem{Default}
\sphinxAtStartPar
MDI

\end{description}\end{quote}

\sphinxAtStartPar
Critical density in parametrization of thermal metamorphism.

\sphinxAtStartPar
Only used if {\hyperref[\detokenize{namelists/jules_snow.nml:JULES_SNOW::l_et_metamorph}]{\sphinxcrossref{\sphinxcode{\sphinxupquote{l\_et\_metamorph}}}}} = TRUE.

\end{fulllineitems}

\index{i\_grain\_growth\_opt (in namelist JULES\_SNOW)@\spxentry{i\_grain\_growth\_opt}\spxextra{in namelist JULES\_SNOW}|spxpagem}

\begin{fulllineitems}
\phantomsection\label{\detokenize{namelists/jules_snow.nml:JULES_SNOW::i_grain_growth_opt}}
\pysigstartsignatures
\pysigline{\sphinxcode{\sphinxupquote{JULES\_SNOW::}}\sphinxbfcode{\sphinxupquote{i\_grain\_growth\_opt}}}
\pysigstopsignatures\begin{quote}\begin{description}
\sphinxlineitem{Type}
\sphinxAtStartPar
integer

\sphinxlineitem{Permitted}
\sphinxAtStartPar
0 or 1

\sphinxlineitem{Default}
\sphinxAtStartPar
0

\end{description}\end{quote}

\sphinxAtStartPar
Scheme used to calculate the rate of growth of snow grains.

\sphinxAtStartPar
Setting this to 0 invokes the original scheme based on Marshall (1989),
with no dependence of the rate of growth of small grains on the
temperature.

\sphinxAtStartPar
Setting it to 1 invokes the scheme for growth of snow grains proposed
by Taillandier et al. (2007) for equitemperature metamorphism. This
is significantly slower than the default scheme at low temperatures.

\end{fulllineitems}

\index{i\_relayer\_opt (in namelist JULES\_SNOW)@\spxentry{i\_relayer\_opt}\spxextra{in namelist JULES\_SNOW}|spxpagem}

\begin{fulllineitems}
\phantomsection\label{\detokenize{namelists/jules_snow.nml:JULES_SNOW::i_relayer_opt}}
\pysigstartsignatures
\pysigline{\sphinxcode{\sphinxupquote{JULES\_SNOW::}}\sphinxbfcode{\sphinxupquote{i\_relayer\_opt}}}
\pysigstopsignatures\begin{quote}\begin{description}
\sphinxlineitem{Type}
\sphinxAtStartPar
integer

\sphinxlineitem{Permitted}
\sphinxAtStartPar
0 or 1

\sphinxlineitem{Default}
\sphinxAtStartPar
0

\end{description}\end{quote}

\sphinxAtStartPar
Scheme used to relayer the snowpack. Setting the option to 0 invokes
the original scheme with relayering of the grain size involving the
grain size itself, while setting it to 1 causes the relayering to be
done using the inverse of the grain size. This is more consistent
with conserving the SSA, though full conservation would require
mass weighting to be invoked during regridding.

\sphinxAtStartPar
Only used if {\hyperref[\detokenize{namelists/jules_snow.nml:JULES_SNOW::nsmax}]{\sphinxcrossref{\sphinxcode{\sphinxupquote{nsmax}}}}} \textgreater{} 0.

\end{fulllineitems}

\index{i\_basal\_melting\_opt (in namelist JULES\_SNOW)@\spxentry{i\_basal\_melting\_opt}\spxextra{in namelist JULES\_SNOW}|spxpagem}

\begin{fulllineitems}
\phantomsection\label{\detokenize{namelists/jules_snow.nml:JULES_SNOW::i_basal_melting_opt}}
\pysigstartsignatures
\pysigline{\sphinxcode{\sphinxupquote{JULES\_SNOW::}}\sphinxbfcode{\sphinxupquote{i\_basal\_melting\_opt}}}
\pysigstopsignatures\begin{quote}\begin{description}
\sphinxlineitem{Type}
\sphinxAtStartPar
integer

\sphinxlineitem{Permitted}
\sphinxAtStartPar
0 or 1

\sphinxlineitem{Default}
\sphinxAtStartPar
0

\end{description}\end{quote}

\sphinxAtStartPar
Option to treat basal melting of the snow pack. When snow falls
on warm ground, it will melt from the base of the snowpack,
where the temperature of the snow will rise to the melting point.
The 0\sphinxhyphen{}layer snow scheme, which is used for thin snow even when the
multilayer scheme is selected, did not represent this process and
included only melting at the surface. This option allows
basal melting to be omitted if it is set to the defaut value of 0,
but offers an alternative setting of 1, which results in basal
melting taking place instantaneously if the temperature of the
first soil layer is above freezing, until the snow is removed or
the temperature of soil layer is reduced to freezing.

\end{fulllineitems}

\end{sphinxadmonition}


\subsection{Example of the evolution of snow layer thickness}
\label{\detokenize{namelists/jules_snow.nml:example-of-the-evolution-of-snow-layer-thickness}}
\sphinxAtStartPar
The table below gives an example of how the number and thickness of snow layers varies with total snow depth for the case of {\hyperref[\detokenize{namelists/jules_snow.nml:JULES_SNOW::nsmax}]{\sphinxcrossref{\sphinxcode{\sphinxupquote{nsmax}}}}} = 3 and \sphinxcode{\sphinxupquote{dzsnow = (0.1, 0.15, 0.2)}}. Note that if the values given by the user for {\hyperref[\detokenize{namelists/jules_snow.nml:JULES_SNOW::dzsnow}]{\sphinxcrossref{\sphinxcode{\sphinxupquote{dzsnow}}}}} are a decreasing series with \sphinxcode{\sphinxupquote{dzsnow(i) \textless{}= 2 * dzsnow(i \sphinxhyphen{} 1)}}, the algorithm will result in layers \sphinxcode{\sphinxupquote{i}} and \sphinxcode{\sphinxupquote{i + 1}} being added at the same time. Don’t panic \sphinxhyphen{} this should not be a problem for the simulation.


\begin{savenotes}\sphinxattablestart
\centering
\begin{tabulary}{\linewidth}[t]{|p{3cm}|p{1.5cm}|p{3cm}|p{7cm}|}
\hline
\sphinxstyletheadfamily 
\sphinxAtStartPar
Snow depth (m)
&\sphinxstyletheadfamily 
\sphinxAtStartPar
Number
of layers
&\sphinxstyletheadfamily 
\sphinxAtStartPar
Layer thickness,
uppermost layer
first (m)
&\sphinxstyletheadfamily 
\sphinxAtStartPar
Comments
\\
\hline
\sphinxAtStartPar
\sphinxcode{\sphinxupquote{d \textless{} 0.1}}
&
\sphinxAtStartPar
0
&&
\sphinxAtStartPar
While the depth of snow is less than \sphinxcode{\sphinxupquote{dzsnow(1)}}, the layer model is not
active and snow and soil are combined in a composite layer.
\\
\hline
\sphinxAtStartPar
\sphinxcode{\sphinxupquote{0.1 \textless{}= d \textless{} 0.2}}
&
\sphinxAtStartPar
1
&
\sphinxAtStartPar
Total snow depth
&
\sphinxAtStartPar
The single layer grows until it is twice as thick as \sphinxcode{\sphinxupquote{dzsnow(1)}}.
\\
\hline
\sphinxAtStartPar
\sphinxcode{\sphinxupquote{0.2 \textless{}= d \textless{} 0.4}}
&
\sphinxAtStartPar
2
&
\sphinxAtStartPar
0.1, remainder
&
\sphinxAtStartPar
Above 0.2m, the single layer splits into a top layer of 0.1m and the remaining
snow in the bottom layer.
\\
\hline
\sphinxAtStartPar
\sphinxcode{\sphinxupquote{\textgreater{}= 0.4}}
&
\sphinxAtStartPar
3
&
\sphinxAtStartPar
0.1, 0.15, remainder
&
\sphinxAtStartPar
At 0.4m depth, layer 2 (which has grown to 0.3m thick, i.e. \sphinxcode{\sphinxupquote{2 * dzsnow(2)}}),
splits into a layer of 0.15m and a new bottom layer holding the the remaining
0.15m. As all layers are now in use, any subsequent deepening of the pack is
dealt with by increasing the thickness in this bottom layer.
\\
\hline
\end{tabulary}
\par
\sphinxattableend\end{savenotes}


\subsection{\sphinxstyleliteralintitle{\sphinxupquote{JULES\_SNOW}} references}
\label{\detokenize{namelists/jules_snow.nml:jules-snow-references}}\label{\detokenize{namelists/jules_snow.nml:references-snow}}\begin{itemize}
\item {} 
\sphinxAtStartPar
Calonne, N., Flin, F., Morin, S., Lesaffre, B., du
Roscoat, S. Rolland, and Geindreau, C. (2011), Numerical and
experimental investigations of the effective thermal conductivity of
snow, Geophys. Res. Lett., 38, L23501,
\sphinxurl{https://doi.org/10.1029/2011GL049234}.

\item {} 
\sphinxAtStartPar
Yen, Y.\sphinxhyphen{}C. (1981). Review of thermal properties of snow, ice and sea
ice. Cold Regions Research and Engineering Laboratory (CRREL) Report
81\sphinxhyphen{}10.  \sphinxurl{https://hdl.handle.net/11681/9469}

\end{itemize}

\sphinxstepscope


\section{\sphinxstyleliteralintitle{\sphinxupquote{jules\_rivers.nml}}}
\label{\detokenize{namelists/jules_rivers.nml:jules-rivers-nml}}\label{\detokenize{namelists/jules_rivers.nml::doc}}
\sphinxAtStartPar
This file sets the river routing options. It contains two namelists called {\hyperref[\detokenize{namelists/jules_rivers.nml:namelist-JULES_RIVERS}]{\sphinxcrossref{\sphinxcode{\sphinxupquote{JULES\_RIVERS}}}}} and {\hyperref[\detokenize{namelists/jules_rivers.nml:namelist-JULES_OVERBANK}]{\sphinxcrossref{\sphinxcode{\sphinxupquote{JULES\_OVERBANK}}}}}.

\sphinxAtStartPar
River routing introduces two more grids to a JULES run: the river routing input grid and the river routing model grid. The river routing input grid must always be specified as a 2D grid in {\hyperref[\detokenize{namelists/ancillaries.nml:namelist-JULES_RIVERS_PROPS}]{\sphinxcrossref{\sphinxcode{\sphinxupquote{JULES\_RIVERS\_PROPS}}}}}. This is not required to be identical to the input or the model grid. Internally the model compresses this to the river routing model grid, which is a 1D grid with length \sphinxcode{\sphinxupquote{np\_rivers}}, which is the number of valid routing points in the river routing input grid. All river routing output will be on the river routing model grid, or will be regridded to the model grid.

\begin{sphinxadmonition}{note}{Note:}
\sphinxAtStartPar
The river routing code in JULES is still in development. Users should ensure that results are as expected, and provide feedback where deficiencies are identified.
\end{sphinxadmonition}


\subsection{\sphinxstyleliteralintitle{\sphinxupquote{JULES\_RIVERS}} namelist members}
\label{\detokenize{namelists/jules_rivers.nml:namelist-JULES_RIVERS}}\label{\detokenize{namelists/jules_rivers.nml:jules-rivers-namelist-members}}\index{JULES\_RIVERS (namelist)@\spxentry{JULES\_RIVERS}\spxextra{namelist}|spxpagem}\index{l\_rivers (in namelist JULES\_RIVERS)@\spxentry{l\_rivers}\spxextra{in namelist JULES\_RIVERS}|spxpagem}

\begin{fulllineitems}
\phantomsection\label{\detokenize{namelists/jules_rivers.nml:JULES_RIVERS::l_rivers}}
\pysigstartsignatures
\pysigline{\sphinxcode{\sphinxupquote{JULES\_RIVERS::}}\sphinxbfcode{\sphinxupquote{l\_rivers}}}
\pysigstopsignatures\begin{quote}\begin{description}
\sphinxlineitem{Type}
\sphinxAtStartPar
logical

\sphinxlineitem{Default}
\sphinxAtStartPar
F

\end{description}\end{quote}

\sphinxAtStartPar
Switch for enabling river routing.
\begin{description}
\sphinxlineitem{TRUE}
\sphinxAtStartPar
Use the river routing algorithm specified by {\hyperref[\detokenize{namelists/jules_rivers.nml:JULES_RIVERS::i_river_vn}]{\sphinxcrossref{\sphinxcode{\sphinxupquote{i\_river\_vn}}}}} to route runoff along river pathways.

\sphinxlineitem{FALSE}
\sphinxAtStartPar
No river routing.

\end{description}

\end{fulllineitems}

\index{i\_river\_vn (in namelist JULES\_RIVERS)@\spxentry{i\_river\_vn}\spxextra{in namelist JULES\_RIVERS}|spxpagem}

\begin{fulllineitems}
\phantomsection\label{\detokenize{namelists/jules_rivers.nml:JULES_RIVERS::i_river_vn}}
\pysigstartsignatures
\pysigline{\sphinxcode{\sphinxupquote{JULES\_RIVERS::}}\sphinxbfcode{\sphinxupquote{i\_river\_vn}}}
\pysigstopsignatures\begin{quote}\begin{description}
\sphinxlineitem{Type}
\sphinxAtStartPar
integer

\sphinxlineitem{Default}
\sphinxAtStartPar
None

\end{description}\end{quote}

\sphinxAtStartPar
Switch to select the river routing algorithm to use for river routing.
\begin{description}
\sphinxlineitem{\sphinxcode{\sphinxupquote{1}}}
\sphinxAtStartPar
Use a UM\sphinxhyphen{}coupled JULES implementation of the TRIP model (see Oki et al. 1999). This value is not allowed in standalone JULES

\sphinxlineitem{\sphinxcode{\sphinxupquote{2}}}
\sphinxAtStartPar
Use a standalone JULES implementation of the RFM kinematic wave model (see Dadson and Bell 2010, Bell et al. 2007).

\sphinxlineitem{\sphinxcode{\sphinxupquote{3}}}
\sphinxAtStartPar
Use a standalone JULES implementation of the TRIP model (see Oki et al. 1999).

\end{description}

\end{fulllineitems}

\index{nstep\_rivers (in namelist JULES\_RIVERS)@\spxentry{nstep\_rivers}\spxextra{in namelist JULES\_RIVERS}|spxpagem}

\begin{fulllineitems}
\phantomsection\label{\detokenize{namelists/jules_rivers.nml:JULES_RIVERS::nstep_rivers}}
\pysigstartsignatures
\pysigline{\sphinxcode{\sphinxupquote{JULES\_RIVERS::}}\sphinxbfcode{\sphinxupquote{nstep\_rivers}}}
\pysigstopsignatures\begin{quote}\begin{description}
\sphinxlineitem{Type}
\sphinxAtStartPar
integer

\sphinxlineitem{Permitted}
\sphinxAtStartPar
\textgreater{} 0

\sphinxlineitem{Default}
\sphinxAtStartPar
None

\end{description}\end{quote}

\sphinxAtStartPar
The number of model timesteps per routing timestep.

\sphinxAtStartPar
For example, {\hyperref[\detokenize{namelists/jules_rivers.nml:JULES_RIVERS::nstep_rivers}]{\sphinxcrossref{\sphinxcode{\sphinxupquote{nstep\_rivers}}}}} = 5 means that runoff will be accumulated for 5 model timesteps before being routed on the 5th timestep.

\end{fulllineitems}


\begin{sphinxadmonition}{warning}{Warning:}
\sphinxAtStartPar
The river routing parameter values can be highly dependent on model resolution, so care is required by the user to ensure that appropriate values are selected, tested and adjusted as required.

\sphinxAtStartPar
Suggested values for global and high\sphinxhyphen{}resolution runs are listed below, however these should be treated as a starting point only.
\end{sphinxadmonition}

\begin{sphinxadmonition}{note}{RFM parameters \sphinxhyphen{} used if \sphinxstyleliteralintitle{\sphinxupquote{i\_river\_vn}} = \sphinxstyleliteralintitle{\sphinxupquote{2}}}
\index{a\_thresh (in namelist JULES\_RIVERS)@\spxentry{a\_thresh}\spxextra{in namelist JULES\_RIVERS}|spxpagem}

\begin{fulllineitems}
\phantomsection\label{\detokenize{namelists/jules_rivers.nml:JULES_RIVERS::a_thresh}}
\pysigstartsignatures
\pysigline{\sphinxcode{\sphinxupquote{JULES\_RIVERS::}}\sphinxbfcode{\sphinxupquote{a\_thresh}}}
\pysigstopsignatures\begin{quote}\begin{description}
\sphinxlineitem{Type}
\sphinxAtStartPar
integer

\sphinxlineitem{Default}
\sphinxAtStartPar
None

\sphinxlineitem{Suggested}
\sphinxAtStartPar
1 (spatial resolution coarser than 20 km gridcells), \textasciitilde{}10 (high\sphinxhyphen{}resolution)

\end{description}\end{quote}

\sphinxAtStartPar
The threshold drainage area (specified in number of cells) draining to a gridbox, above which the grid cell is considered to be a river point (see a\_T in Bell et al. 2007:541).

\sphinxAtStartPar
Remaining points are treated as land (drainage area = 0) or sea (drainage area \textless{} 0). See Bell et al. (2007).

\end{fulllineitems}

\index{cland (in namelist JULES\_RIVERS)@\spxentry{cland}\spxextra{in namelist JULES\_RIVERS}|spxpagem}

\begin{fulllineitems}
\phantomsection\label{\detokenize{namelists/jules_rivers.nml:JULES_RIVERS::cland}}
\pysigstartsignatures
\pysigline{\sphinxcode{\sphinxupquote{JULES\_RIVERS::}}\sphinxbfcode{\sphinxupquote{cland}}}
\pysigstopsignatures\begin{quote}\begin{description}
\sphinxlineitem{Type}
\sphinxAtStartPar
real

\sphinxlineitem{Permitted}
\sphinxAtStartPar
\textgreater{} 0

\sphinxlineitem{Default}
\sphinxAtStartPar
None

\sphinxlineitem{Suggested}
\sphinxAtStartPar
0.20 m/s (global), 0.40 m/s (1 km resolution, Bell et al. 2007)

\end{description}\end{quote}

\sphinxAtStartPar
The land wave speed (kinematic wave speed for surface flow in a land grid box on the river routing grid, m s$^{\text{\sphinxhyphen{}1}}$). This is the speed at which water moves through surface soil in a non\sphinxhyphen{}river grid cell (even without major rivers, there are always minor water courses so these cells do still contribute flow to neighbouring cells).

\end{fulllineitems}

\index{criver (in namelist JULES\_RIVERS)@\spxentry{criver}\spxextra{in namelist JULES\_RIVERS}|spxpagem}

\begin{fulllineitems}
\phantomsection\label{\detokenize{namelists/jules_rivers.nml:JULES_RIVERS::criver}}
\pysigstartsignatures
\pysigline{\sphinxcode{\sphinxupquote{JULES\_RIVERS::}}\sphinxbfcode{\sphinxupquote{criver}}}
\pysigstopsignatures\begin{quote}\begin{description}
\sphinxlineitem{Type}
\sphinxAtStartPar
real

\sphinxlineitem{Permitted}
\sphinxAtStartPar
\textgreater{} 0

\sphinxlineitem{Default}
\sphinxAtStartPar
None

\sphinxlineitem{Suggested}
\sphinxAtStartPar
0.62 m/s (global), 0.50 m/s (1 km resolution, Bell et al. 2007)

\end{description}\end{quote}

\sphinxAtStartPar
The river wave speed (kinematic wave speed for surface flow in a river grid box on the river routing grid, m s$^{\text{\sphinxhyphen{}1}}$). This value should be close to the {\hyperref[\detokenize{namelists/jules_rivers.nml:JULES_RIVERS::rivers_speed}]{\sphinxcrossref{\sphinxcode{\sphinxupquote{rivers\_speed}}}}} used by TRIP, but not identical because RFM makes different assumptions about e.g. meandering.

\end{fulllineitems}

\index{cbland (in namelist JULES\_RIVERS)@\spxentry{cbland}\spxextra{in namelist JULES\_RIVERS}|spxpagem}

\begin{fulllineitems}
\phantomsection\label{\detokenize{namelists/jules_rivers.nml:JULES_RIVERS::cbland}}
\pysigstartsignatures
\pysigline{\sphinxcode{\sphinxupquote{JULES\_RIVERS::}}\sphinxbfcode{\sphinxupquote{cbland}}}
\pysigstopsignatures\begin{quote}\begin{description}
\sphinxlineitem{Type}
\sphinxAtStartPar
real

\sphinxlineitem{Permitted}
\sphinxAtStartPar
\textgreater{} 0

\sphinxlineitem{Default}
\sphinxAtStartPar
None

\sphinxlineitem{Suggested}
\sphinxAtStartPar
\textless{}= {\hyperref[\detokenize{namelists/jules_rivers.nml:JULES_RIVERS::cland}]{\sphinxcrossref{\sphinxcode{\sphinxupquote{cland}}}}}. 0.10 m/s (global), 0.05 m/s (1 km resolution, Bell et al. 2007)

\end{description}\end{quote}

\sphinxAtStartPar
The subsurface land wave speed (kinematic wave speed for subsurface flow in a land grid box on the river routing grid, m s$^{\text{\sphinxhyphen{}1}}$).

\end{fulllineitems}

\index{cbriver (in namelist JULES\_RIVERS)@\spxentry{cbriver}\spxextra{in namelist JULES\_RIVERS}|spxpagem}

\begin{fulllineitems}
\phantomsection\label{\detokenize{namelists/jules_rivers.nml:JULES_RIVERS::cbriver}}
\pysigstartsignatures
\pysigline{\sphinxcode{\sphinxupquote{JULES\_RIVERS::}}\sphinxbfcode{\sphinxupquote{cbriver}}}
\pysigstopsignatures\begin{quote}\begin{description}
\sphinxlineitem{Type}
\sphinxAtStartPar
real

\sphinxlineitem{Permitted}
\sphinxAtStartPar
\textgreater{} 0

\sphinxlineitem{Default}
\sphinxAtStartPar
None

\sphinxlineitem{Suggested}
\sphinxAtStartPar
\textless{}= {\hyperref[\detokenize{namelists/jules_rivers.nml:JULES_RIVERS::criver}]{\sphinxcrossref{\sphinxcode{\sphinxupquote{criver}}}}}. 0.15 m/s (global), 0.05 m/s (1 km resolution, Bell et al. 2007)

\end{description}\end{quote}

\sphinxAtStartPar
The subsurface river wave speed (kinematic wave speed for subsurface flow in a river grid box on the river routing grid, m s$^{\text{\sphinxhyphen{}1}}$).

\end{fulllineitems}

\index{retl (in namelist JULES\_RIVERS)@\spxentry{retl}\spxextra{in namelist JULES\_RIVERS}|spxpagem}

\begin{fulllineitems}
\phantomsection\label{\detokenize{namelists/jules_rivers.nml:JULES_RIVERS::retl}}
\pysigstartsignatures
\pysigline{\sphinxcode{\sphinxupquote{JULES\_RIVERS::}}\sphinxbfcode{\sphinxupquote{retl}}}
\pysigstopsignatures\begin{quote}\begin{description}
\sphinxlineitem{Type}
\sphinxAtStartPar
real

\sphinxlineitem{Permitted}
\sphinxAtStartPar
\sphinxhyphen{}1 to 1

\sphinxlineitem{Default}
\sphinxAtStartPar
None

\sphinxlineitem{Suggested}
\sphinxAtStartPar
0.005 (1 km resolution, Bell et al. 2007)

\end{description}\end{quote}

\sphinxAtStartPar
The (resolution dependent) land return flow fraction. Bell et al. (2007:Table1) suggested value 0.005. On non\sphinxhyphen{}river grid cells in the land mask: if retl\textgreater{}0 then fraction retl of the subsurface flow moves to the surface per routing timestep; if retl\textless{}0 then fraction retl of the surface flow moves to the subsurface per routing timestep.

\end{fulllineitems}

\index{retr (in namelist JULES\_RIVERS)@\spxentry{retr}\spxextra{in namelist JULES\_RIVERS}|spxpagem}

\begin{fulllineitems}
\phantomsection\label{\detokenize{namelists/jules_rivers.nml:JULES_RIVERS::retr}}
\pysigstartsignatures
\pysigline{\sphinxcode{\sphinxupquote{JULES\_RIVERS::}}\sphinxbfcode{\sphinxupquote{retr}}}
\pysigstopsignatures\begin{quote}\begin{description}
\sphinxlineitem{Type}
\sphinxAtStartPar
real

\sphinxlineitem{Permitted}
\sphinxAtStartPar
\sphinxhyphen{}1 to 1

\sphinxlineitem{Default}
\sphinxAtStartPar
None

\sphinxlineitem{Suggested}
\sphinxAtStartPar
0.005 (1 km resolution, Bell et al. 2007)

\end{description}\end{quote}

\sphinxAtStartPar
The (resolution dependent) river return flow fraction. On river grid cells in the land mask: if retr\textgreater{}0 then fraction retr of the subsurface flow moves to the surface per routing timestep; if retr\textless{}0 then fraction retr of the surface flow moves to the subsurface per routing timestep.

\end{fulllineitems}

\index{runoff\_factor (in namelist JULES\_RIVERS)@\spxentry{runoff\_factor}\spxextra{in namelist JULES\_RIVERS}|spxpagem}

\begin{fulllineitems}
\phantomsection\label{\detokenize{namelists/jules_rivers.nml:JULES_RIVERS::runoff_factor}}
\pysigstartsignatures
\pysigline{\sphinxcode{\sphinxupquote{JULES\_RIVERS::}}\sphinxbfcode{\sphinxupquote{runoff\_factor}}}
\pysigstopsignatures\begin{quote}\begin{description}
\sphinxlineitem{Type}
\sphinxAtStartPar
real

\sphinxlineitem{Permitted}
\sphinxAtStartPar
\textgreater{} 0

\sphinxlineitem{Default}
\sphinxAtStartPar
None

\end{description}\end{quote}

\sphinxAtStartPar
Values !=1.0 are generally used to correct biases in precipitation when the model is forced with observed data \sphinxstylestrong{It is highly recommended that this is set to 1.0 (i.e. no runoff adjustment).}

\end{fulllineitems}

\end{sphinxadmonition}

\begin{sphinxadmonition}{note}{TRIP parameters \sphinxhyphen{} used if \sphinxstyleliteralintitle{\sphinxupquote{i\_river\_vn}} = \sphinxstyleliteralintitle{\sphinxupquote{1,3}}}
\index{rivers\_speed (in namelist JULES\_RIVERS)@\spxentry{rivers\_speed}\spxextra{in namelist JULES\_RIVERS}|spxpagem}

\begin{fulllineitems}
\phantomsection\label{\detokenize{namelists/jules_rivers.nml:JULES_RIVERS::rivers_speed}}
\pysigstartsignatures
\pysigline{\sphinxcode{\sphinxupquote{JULES\_RIVERS::}}\sphinxbfcode{\sphinxupquote{rivers\_speed}}}
\pysigstopsignatures\begin{quote}\begin{description}
\sphinxlineitem{Type}
\sphinxAtStartPar
real

\sphinxlineitem{Permitted}
\sphinxAtStartPar
\textgreater{} 0

\sphinxlineitem{Default}
\sphinxAtStartPar
None

\end{description}\end{quote}

\sphinxAtStartPar
The effective river velocity (m s$^{\text{\sphinxhyphen{}1}}$). See Oki et al. (1999). {\hyperref[\detokenize{namelists/jules_rivers.nml:JULES_RIVERS::rivers_speed}]{\sphinxcrossref{\sphinxcode{\sphinxupquote{rivers\_speed}}}}} should equal (river flow velocity / {\hyperref[\detokenize{namelists/jules_rivers.nml:JULES_RIVERS::rivers_meander}]{\sphinxcrossref{\sphinxcode{\sphinxupquote{rivers\_meander}}}}}). A value of 0.4 can be used, while Oki et al. (1999) used a value of 0.5.

\end{fulllineitems}

\index{rivers\_meander (in namelist JULES\_RIVERS)@\spxentry{rivers\_meander}\spxextra{in namelist JULES\_RIVERS}|spxpagem}

\begin{fulllineitems}
\phantomsection\label{\detokenize{namelists/jules_rivers.nml:JULES_RIVERS::rivers_meander}}
\pysigstartsignatures
\pysigline{\sphinxcode{\sphinxupquote{JULES\_RIVERS::}}\sphinxbfcode{\sphinxupquote{rivers\_meander}}}
\pysigstopsignatures\begin{quote}\begin{description}
\sphinxlineitem{Type}
\sphinxAtStartPar
real

\sphinxlineitem{Permitted}
\sphinxAtStartPar
\textgreater{} 0

\sphinxlineitem{Default}
\sphinxAtStartPar
None

\end{description}\end{quote}

\sphinxAtStartPar
The ratio of the actual to calculated river lengths in a river routing gridbox. See Oki et al. (1999). Oki \& Sud (1998) called this the Meandering Ratio r\_M and suggested an average global value of 1.4.

\end{fulllineitems}

\end{sphinxadmonition}


\sphinxstrong{See also:}
\nopagebreak


\sphinxAtStartPar
References:
\begin{itemize}
\item {} 
\sphinxAtStartPar
Arora VK \& Boer GJ (2012). A variable velocity flow routing algorithm for GCMs. Journal of Geophysical Research D 104:30965\sphinxhyphen{}30979.

\item {} 
\sphinxAtStartPar
Bell, V.A. et al. (2007) Development of a high resolution grid\sphinxhyphen{}based river flow model for use with regional climate model output. Hydrology and Earth System Sciences. 11 532\sphinxhyphen{}549

\item {} 
\sphinxAtStartPar
Dadson, S.J. and Bell, V.A. (2010) Comparison of Grid\sphinxhyphen{}2\sphinxhyphen{}Grid and TRIP runoff routing schemes. Centre for Ecology \& Hydrology Internal Report \sphinxurl{http://nora.nerc.ac.uk/10890/1/dadson\_etal\_2010\_g2gtrip.pdf}

\item {} 
\sphinxAtStartPar
Dadson S.J. et al. (2011) Evaluation of a grid\sphinxhyphen{}based river flow model configured for use in a regional climate model. Journal of Hydrology. 411 238\sphinxhyphen{}250

\item {} 
\sphinxAtStartPar
Falloon, P.D. et al (2007) New global river routing scheme in the Unified Model. Hadley Centre Technical Note 72, available from \sphinxhref{http://www.metoffice.gov.uk/learning/library/publications/science/climate-science-technical-notes}{the Met Office Library}.

\item {} 
\sphinxAtStartPar
Jones R., Dadson, S. and Bell, V.A. (2007) Report on European grid\sphinxhyphen{}based river\sphinxhyphen{}flow modelling for application to Regional Climate Models. Met Office Hadley Centre deliverable report.

\item {} 
\sphinxAtStartPar
Oki, T. and Sud, Y.C. (1998) Design of Total Runoff Integrating Pathways (TRIP)—A Global River Channel Network. Earth Interactions, 2: 1\sphinxhyphen{}37.

\item {} 
\sphinxAtStartPar
Oki, T., et al (1999) Assessment of annual runoff from land surface models using Total Runoff Integrating Pathways (TRIP). Journal of the Meteorological Society of Japan. 77 235\sphinxhyphen{}255

\end{itemize}




\subsection{\sphinxstyleliteralintitle{\sphinxupquote{JULES\_OVERBANK}} namelist members}
\label{\detokenize{namelists/jules_rivers.nml:namelist-JULES_OVERBANK}}\label{\detokenize{namelists/jules_rivers.nml:jules-overbank-namelist-members}}\index{JULES\_OVERBANK (namelist)@\spxentry{JULES\_OVERBANK}\spxextra{namelist}|spxpagem}
\begin{sphinxadmonition}{warning}{Warning:}
\sphinxAtStartPar
The overbank inundation parameter values can be highly dependent on model resolution, so care is required by the user to ensure that appropriate values are selected, tested and adjusted as required.

\sphinxAtStartPar
Suggested values for global and high\sphinxhyphen{}resolution runs are listed below, however these should be treated as a starting point only.
\end{sphinxadmonition}
\index{l\_riv\_overbank (in namelist JULES\_OVERBANK)@\spxentry{l\_riv\_overbank}\spxextra{in namelist JULES\_OVERBANK}|spxpagem}

\begin{fulllineitems}
\phantomsection\label{\detokenize{namelists/jules_rivers.nml:JULES_OVERBANK::l_riv_overbank}}
\pysigstartsignatures
\pysigline{\sphinxcode{\sphinxupquote{JULES\_OVERBANK::}}\sphinxbfcode{\sphinxupquote{l\_riv\_overbank}}}
\pysigstopsignatures\begin{quote}\begin{description}
\sphinxlineitem{Type}
\sphinxAtStartPar
logical

\sphinxlineitem{Default}
\sphinxAtStartPar
F

\end{description}\end{quote}

\sphinxAtStartPar
Switch for enabling river overbank inundation. Only used if {\hyperref[\detokenize{namelists/jules_rivers.nml:JULES_RIVERS::l_rivers}]{\sphinxcrossref{\sphinxcode{\sphinxupquote{l\_rivers}}}}} is TRUE.
\begin{description}
\sphinxlineitem{TRUE}
\sphinxAtStartPar
Calculate frac\_fplain\_lp, i.e. overbank inundation area as a fraction of gridcell area.

\sphinxlineitem{FALSE}
\sphinxAtStartPar
No overbank inundation calculations

\end{description}

\end{fulllineitems}


\begin{sphinxadmonition}{note}{Note:}
\sphinxAtStartPar
If {\hyperref[\detokenize{namelists/jules_rivers.nml:JULES_OVERBANK::l_riv_overbank}]{\sphinxcrossref{\sphinxcode{\sphinxupquote{l\_riv\_overbank}}}}} = FALSE, no further variables are needed from this namelist.
\end{sphinxadmonition}
\index{l\_riv\_hypsometry (in namelist JULES\_OVERBANK)@\spxentry{l\_riv\_hypsometry}\spxextra{in namelist JULES\_OVERBANK}|spxpagem}

\begin{fulllineitems}
\phantomsection\label{\detokenize{namelists/jules_rivers.nml:JULES_OVERBANK::l_riv_hypsometry}}
\pysigstartsignatures
\pysigline{\sphinxcode{\sphinxupquote{JULES\_OVERBANK::}}\sphinxbfcode{\sphinxupquote{l\_riv\_hypsometry}}}
\pysigstopsignatures\begin{quote}\begin{description}
\sphinxlineitem{Type}
\sphinxAtStartPar
logical

\sphinxlineitem{Default}
\sphinxAtStartPar
F

\end{description}\end{quote}

\sphinxAtStartPar
Switch for enabling use of a hypsometric integral calculation.
\begin{description}
\sphinxlineitem{TRUE}
\sphinxAtStartPar
Calculate inundated area from a hypsometric integral based on a lognormal area\sphinxhyphen{}altitude distribution (\sphinxstylestrong{recommended}).

\sphinxlineitem{FALSE}
\sphinxAtStartPar
Estimate inundated area from simple river width scaling, ignoring topography (only to be used for testing).

\end{description}

\end{fulllineitems}


\begin{sphinxadmonition}{note}{River depth allometry (used if \sphinxstyleliteralintitle{\sphinxupquote{l\_riv\_hypsometry}} is TRUE or \sphinxstyleliteralintitle{\sphinxupquote{use\_rosgen}} is TRUE)}

\sphinxAtStartPar
Allometry is: (DEPTH in m) = {\hyperref[\detokenize{namelists/jules_rivers.nml:JULES_OVERBANK::riv_c}]{\sphinxcrossref{\sphinxcode{\sphinxupquote{riv\_c}}}}} * ( (SURFACE RIVER INFLOW in m3 s$^{\text{\sphinxhyphen{}1}}$) \textasciicircum{} {\hyperref[\detokenize{namelists/jules_rivers.nml:JULES_OVERBANK::riv_f}]{\sphinxcrossref{\sphinxcode{\sphinxupquote{riv\_f}}}}}) (Leopold \& Maddock 1953:eqn2)
\index{riv\_c (in namelist JULES\_OVERBANK)@\spxentry{riv\_c}\spxextra{in namelist JULES\_OVERBANK}|spxpagem}

\begin{fulllineitems}
\phantomsection\label{\detokenize{namelists/jules_rivers.nml:JULES_OVERBANK::riv_c}}
\pysigstartsignatures
\pysigline{\sphinxcode{\sphinxupquote{JULES\_OVERBANK::}}\sphinxbfcode{\sphinxupquote{riv\_c}}}
\pysigstopsignatures\begin{quote}\begin{description}
\sphinxlineitem{Type}
\sphinxAtStartPar
real

\sphinxlineitem{Default}
\sphinxAtStartPar
none

\sphinxlineitem{Permitted}
\sphinxAtStartPar
\textgreater{}=0 and \textless{}=(1/{\hyperref[\detokenize{namelists/jules_rivers.nml:JULES_OVERBANK::riv_a}]{\sphinxcrossref{\sphinxcode{\sphinxupquote{riv\_a}}}}})

\sphinxlineitem{Suggested}
\sphinxAtStartPar
0.27 (global, from Andreadis et al. 2013)

\end{description}\end{quote}

\sphinxAtStartPar
Coefficient in the allometry for river depth (units are (m / ((m3/s)\textasciicircum{}riv\_f)), i.e. dependent on the value of riv\_f)

\end{fulllineitems}

\index{riv\_f (in namelist JULES\_OVERBANK)@\spxentry{riv\_f}\spxextra{in namelist JULES\_OVERBANK}|spxpagem}

\begin{fulllineitems}
\phantomsection\label{\detokenize{namelists/jules_rivers.nml:JULES_OVERBANK::riv_f}}
\pysigstartsignatures
\pysigline{\sphinxcode{\sphinxupquote{JULES\_OVERBANK::}}\sphinxbfcode{\sphinxupquote{riv\_f}}}
\pysigstopsignatures\begin{quote}\begin{description}
\sphinxlineitem{Type}
\sphinxAtStartPar
real

\sphinxlineitem{Default}
\sphinxAtStartPar
none

\sphinxlineitem{Permitted}
\sphinxAtStartPar
\textgreater{}=0 and \textless{}=(1\sphinxhyphen{}{\hyperref[\detokenize{namelists/jules_rivers.nml:JULES_OVERBANK::riv_b}]{\sphinxcrossref{\sphinxcode{\sphinxupquote{riv\_b}}}}})

\sphinxlineitem{Suggested}
\sphinxAtStartPar
0.30 (global, from Andreadis et al. 2013)

\end{description}\end{quote}

\sphinxAtStartPar
Exponent in the allometry for river depth (dimensionless)

\end{fulllineitems}

\end{sphinxadmonition}

\begin{sphinxadmonition}{note}{River width scaling (used if \sphinxstyleliteralintitle{\sphinxupquote{l\_riv\_hypsometry}} is FALSE)}

\begin{sphinxadmonition}{note}{River width allometry}

\sphinxAtStartPar
Allometry is: (WIDTH in m) = {\hyperref[\detokenize{namelists/jules_rivers.nml:JULES_OVERBANK::riv_a}]{\sphinxcrossref{\sphinxcode{\sphinxupquote{riv\_a}}}}} * ( (SURFACE RIVER INFLOW in m3 s$^{\text{\sphinxhyphen{}1}}$) \textasciicircum{} {\hyperref[\detokenize{namelists/jules_rivers.nml:JULES_OVERBANK::riv_b}]{\sphinxcrossref{\sphinxcode{\sphinxupquote{riv\_b}}}}}) (Leopold \& Maddock 1953:eqn1)
\index{riv\_a (in namelist JULES\_OVERBANK)@\spxentry{riv\_a}\spxextra{in namelist JULES\_OVERBANK}|spxpagem}

\begin{fulllineitems}
\phantomsection\label{\detokenize{namelists/jules_rivers.nml:JULES_OVERBANK::riv_a}}
\pysigstartsignatures
\pysigline{\sphinxcode{\sphinxupquote{JULES\_OVERBANK::}}\sphinxbfcode{\sphinxupquote{riv\_a}}}
\pysigstopsignatures\begin{quote}\begin{description}
\sphinxlineitem{Type}
\sphinxAtStartPar
real

\sphinxlineitem{Default}
\sphinxAtStartPar
none

\sphinxlineitem{Permitted}
\sphinxAtStartPar
\textgreater{}=0 and \textless{}=(1/{\hyperref[\detokenize{namelists/jules_rivers.nml:JULES_OVERBANK::riv_c}]{\sphinxcrossref{\sphinxcode{\sphinxupquote{riv\_c}}}}})

\sphinxlineitem{Suggested}
\sphinxAtStartPar
7.20 (global, from Andreadis et al. 2013)

\end{description}\end{quote}

\sphinxAtStartPar
Coefficient in the allometry for river width (units are (m / ((m3/s)\textasciicircum{}riv\_b)), i.e. dependent on the value of riv\_b)

\end{fulllineitems}

\index{riv\_b (in namelist JULES\_OVERBANK)@\spxentry{riv\_b}\spxextra{in namelist JULES\_OVERBANK}|spxpagem}

\begin{fulllineitems}
\phantomsection\label{\detokenize{namelists/jules_rivers.nml:JULES_OVERBANK::riv_b}}
\pysigstartsignatures
\pysigline{\sphinxcode{\sphinxupquote{JULES\_OVERBANK::}}\sphinxbfcode{\sphinxupquote{riv\_b}}}
\pysigstopsignatures\begin{quote}\begin{description}
\sphinxlineitem{Type}
\sphinxAtStartPar
real

\sphinxlineitem{Default}
\sphinxAtStartPar
none

\sphinxlineitem{Permitted}
\sphinxAtStartPar
\textgreater{}=0 and \textless{}=(1\sphinxhyphen{}{\hyperref[\detokenize{namelists/jules_rivers.nml:JULES_OVERBANK::riv_f}]{\sphinxcrossref{\sphinxcode{\sphinxupquote{riv\_f}}}}})

\sphinxlineitem{Suggested}
\sphinxAtStartPar
0.50 (global, from Andreadis et al. 2013)

\end{description}\end{quote}

\sphinxAtStartPar
Exponent in the allometry for river width (dimensionless)

\end{fulllineitems}

\index{use\_rosgen (in namelist JULES\_OVERBANK)@\spxentry{use\_rosgen}\spxextra{in namelist JULES\_OVERBANK}|spxpagem}

\begin{fulllineitems}
\phantomsection\label{\detokenize{namelists/jules_rivers.nml:JULES_OVERBANK::use_rosgen}}
\pysigstartsignatures
\pysigline{\sphinxcode{\sphinxupquote{JULES\_OVERBANK::}}\sphinxbfcode{\sphinxupquote{use\_rosgen}}}
\pysigstopsignatures\begin{quote}\begin{description}
\sphinxlineitem{Type}
\sphinxAtStartPar
logical

\sphinxlineitem{Default}
\sphinxAtStartPar
F

\end{description}\end{quote}

\sphinxAtStartPar
Switch for applying the Rosgen entrenchment ratio approach to estimate river width
\begin{description}
\sphinxlineitem{TRUE}
\sphinxAtStartPar
When inflow rates are lower than bankfull flow, river width is calculated from the River width allometry (above). However, when higher than bankfull flow, river width is constrained so that when river depth = 2 x bankfull depth then width = {\hyperref[\detokenize{namelists/jules_rivers.nml:JULES_OVERBANK::ent_ratio}]{\sphinxcrossref{\sphinxcode{\sphinxupquote{ent\_ratio}}}}} * bankfull width.

\sphinxlineitem{FALSE}
\sphinxAtStartPar
River width follows the allometry specified above whatever the inflow rate.

\end{description}

\end{fulllineitems}

\end{sphinxadmonition}

\begin{sphinxadmonition}{note}{Bankfull flow allometry (used if \sphinxstyleliteralintitle{\sphinxupquote{use\_rosgen}} is TRUE) (Rosgen 1994)}

\sphinxAtStartPar
Allometry is: (BANKFULL DISCHARGE RATE QBF in m3 s$^{\text{\sphinxhyphen{}1}}$) = {\hyperref[\detokenize{namelists/jules_rivers.nml:JULES_OVERBANK::coef_b}]{\sphinxcrossref{\sphinxcode{\sphinxupquote{coef\_b}}}}} * ( (CONTRIBUTING AREA in km2) \textasciicircum{} {\hyperref[\detokenize{namelists/jules_rivers.nml:JULES_OVERBANK::exp_c}]{\sphinxcrossref{\sphinxcode{\sphinxupquote{exp\_c}}}}} ) (see e.g. Andreadis et al. 2013)
\index{coef\_b (in namelist JULES\_OVERBANK)@\spxentry{coef\_b}\spxextra{in namelist JULES\_OVERBANK}|spxpagem}

\begin{fulllineitems}
\phantomsection\label{\detokenize{namelists/jules_rivers.nml:JULES_OVERBANK::coef_b}}
\pysigstartsignatures
\pysigline{\sphinxcode{\sphinxupquote{JULES\_OVERBANK::}}\sphinxbfcode{\sphinxupquote{coef\_b}}}
\pysigstopsignatures\begin{quote}\begin{description}
\sphinxlineitem{Type}
\sphinxAtStartPar
real

\sphinxlineitem{Default}
\sphinxAtStartPar
none

\sphinxlineitem{Suggested}
\sphinxAtStartPar
0.08 (for “several drainages in western Washington State, USA”, Cragun 2005)

\end{description}\end{quote}

\sphinxAtStartPar
Coefficient in the allometry for bankfull flow (see Sen 2018:eqn3.33).

\end{fulllineitems}

\index{exp\_c (in namelist JULES\_OVERBANK)@\spxentry{exp\_c}\spxextra{in namelist JULES\_OVERBANK}|spxpagem}

\begin{fulllineitems}
\phantomsection\label{\detokenize{namelists/jules_rivers.nml:JULES_OVERBANK::exp_c}}
\pysigstartsignatures
\pysigline{\sphinxcode{\sphinxupquote{JULES\_OVERBANK::}}\sphinxbfcode{\sphinxupquote{exp\_c}}}
\pysigstopsignatures\begin{quote}\begin{description}
\sphinxlineitem{Type}
\sphinxAtStartPar
real

\sphinxlineitem{Default}
\sphinxAtStartPar
none

\sphinxlineitem{Suggested}
\sphinxAtStartPar
0.95 (for “several drainages in western Washington State, USA”, Cragun 2005)

\end{description}\end{quote}

\sphinxAtStartPar
Exponent in the allometry for bankfull flow (see Sen 2018:eqn3.33).

\end{fulllineitems}

\index{ent\_ratio (in namelist JULES\_OVERBANK)@\spxentry{ent\_ratio}\spxextra{in namelist JULES\_OVERBANK}|spxpagem}

\begin{fulllineitems}
\phantomsection\label{\detokenize{namelists/jules_rivers.nml:JULES_OVERBANK::ent_ratio}}
\pysigstartsignatures
\pysigline{\sphinxcode{\sphinxupquote{JULES\_OVERBANK::}}\sphinxbfcode{\sphinxupquote{ent\_ratio}}}
\pysigstopsignatures\begin{quote}\begin{description}
\sphinxlineitem{Type}
\sphinxAtStartPar
real

\sphinxlineitem{Default}
\sphinxAtStartPar
none

\end{description}\end{quote}

\sphinxAtStartPar
The Rosgen entrenchment ratio (single value for all water courses in the simulation): when river depth = 2 x bankfull depth then width = {\hyperref[\detokenize{namelists/jules_rivers.nml:JULES_OVERBANK::ent_ratio}]{\sphinxcrossref{\sphinxcode{\sphinxupquote{ent\_ratio}}}}} * bankfull width (i.e. {\hyperref[\detokenize{namelists/jules_rivers.nml:JULES_OVERBANK::ent_ratio}]{\sphinxcrossref{\sphinxcode{\sphinxupquote{ent\_ratio}}}}} can be used to specify how wide floodplains are allowed to be).

\end{fulllineitems}

\end{sphinxadmonition}
\end{sphinxadmonition}


\sphinxstrong{See also:}
\nopagebreak


\sphinxAtStartPar
References:
\begin{itemize}
\item {} 
\sphinxAtStartPar
Andreadis KM, Schumann GJ \& Pavelsky T (2013). A simple global river bankfull width and depth database. Water Resources Research 49:7164\sphinxhyphen{}7168

\item {} 
\sphinxAtStartPar
Cragun WS (2005). Discharge\sphinxhyphen{}Area relations from Selected Drainages on the Colorado Plateau: A GIS Application. Utah State University, \sphinxurl{http://hydrology.usu.edu/giswr/archive05/scragun/termproject/}

\item {} 
\sphinxAtStartPar
Leopold LB \& Maddock T (1953). The Hydraulic Geometry of Stream Channels and Some Physiographic Implications. United States Geological Survey Professional Papers 252:1\sphinxhyphen{}57

\item {} 
\sphinxAtStartPar
Rosgen DL (1994). A classification of natural rivers. Catena 22:169\sphinxhyphen{}199.

\item {} 
\sphinxAtStartPar
Sen Z (2018). Flood Modeling, Prediction and Mitigation. Springer.

\end{itemize}



\sphinxstepscope


\section{\sphinxstyleliteralintitle{\sphinxupquote{jules\_water\_resources.nml}}}
\label{\detokenize{namelists/jules_water_resources.nml:jules-water-resources-nml}}\label{\detokenize{namelists/jules_water_resources.nml::doc}}
\sphinxAtStartPar
This file sets options for water resource modelling. It contains a single namelist called {\hyperref[\detokenize{namelists/jules_water_resources.nml:namelist-JULES_WATER_RESOURCES}]{\sphinxcrossref{\sphinxcode{\sphinxupquote{JULES\_WATER\_RESOURCES}}}}}.

\begin{sphinxadmonition}{warning}{Warning:}
\sphinxAtStartPar
The water resource code in JULES is still in development. It should only be used by the developers.
\end{sphinxadmonition}


\subsection{\sphinxstyleliteralintitle{\sphinxupquote{JULES\_WATER\_RESOURCES}} namelist members}
\label{\detokenize{namelists/jules_water_resources.nml:namelist-JULES_WATER_RESOURCES}}\label{\detokenize{namelists/jules_water_resources.nml:jules-water-resources-namelist-members}}\index{JULES\_WATER\_RESOURCES (namelist)@\spxentry{JULES\_WATER\_RESOURCES}\spxextra{namelist}|spxpagem}\index{l\_water\_resources (in namelist JULES\_WATER\_RESOURCES)@\spxentry{l\_water\_resources}\spxextra{in namelist JULES\_WATER\_RESOURCES}|spxpagem}

\begin{fulllineitems}
\phantomsection\label{\detokenize{namelists/jules_water_resources.nml:JULES_WATER_RESOURCES::l_water_resources}}
\pysigstartsignatures
\pysigline{\sphinxcode{\sphinxupquote{JULES\_WATER\_RESOURCES::}}\sphinxbfcode{\sphinxupquote{l\_water\_resources}}}
\pysigstopsignatures\begin{quote}\begin{description}
\sphinxlineitem{Type}
\sphinxAtStartPar
logical

\sphinxlineitem{Default}
\sphinxAtStartPar
F

\end{description}\end{quote}

\sphinxAtStartPar
Switch to enable modelling of water resources.
\begin{description}
\sphinxlineitem{TRUE}
\sphinxAtStartPar
Water resources are modelled.
This also requires that river routing is selected ({\hyperref[\detokenize{namelists/jules_rivers.nml:JULES_RIVERS::l_rivers}]{\sphinxcrossref{\sphinxcode{\sphinxupquote{l\_rivers}}}}} = TRUE).

\sphinxlineitem{FALSE}
\sphinxAtStartPar
No water resources. In this case no further values from this namelist are required.

\end{description}

\end{fulllineitems}

\index{nstep\_water\_res (in namelist JULES\_WATER\_RESOURCES)@\spxentry{nstep\_water\_res}\spxextra{in namelist JULES\_WATER\_RESOURCES}|spxpagem}

\begin{fulllineitems}
\phantomsection\label{\detokenize{namelists/jules_water_resources.nml:JULES_WATER_RESOURCES::nstep_water_res}}
\pysigstartsignatures
\pysigline{\sphinxcode{\sphinxupquote{JULES\_WATER\_RESOURCES::}}\sphinxbfcode{\sphinxupquote{nstep\_water\_res}}}
\pysigstopsignatures\begin{quote}\begin{description}
\sphinxlineitem{Type}
\sphinxAtStartPar
integer

\sphinxlineitem{Permitted}
\sphinxAtStartPar
\textgreater{} 0

\sphinxlineitem{Default}
\sphinxAtStartPar
none

\end{description}\end{quote}

\sphinxAtStartPar
The number of model timesteps per water resource timestep. (The main model timestep is given by {\hyperref[\detokenize{namelists/timesteps.nml:JULES_TIME::timestep_len}]{\sphinxcrossref{\sphinxcode{\sphinxupquote{timestep\_len}}}}}.)

\sphinxAtStartPar
For example, {\hyperref[\detokenize{namelists/jules_water_resources.nml:JULES_WATER_RESOURCES::nstep_water_res}]{\sphinxcrossref{\sphinxcode{\sphinxupquote{nstep\_water\_res}}}}} = 12 means that demands for water will be accumulated over 12 model timesteps before the water resource code is called on the 12th timestep.

\sphinxAtStartPar
The water resource and river routing models must be in synchrony (i.e. be called on the same timesteps).

\end{fulllineitems}


\begin{sphinxadmonition}{note}{Switches that control which water sectors are considered.}
\index{l\_water\_domestic (in namelist JULES\_WATER\_RESOURCES)@\spxentry{l\_water\_domestic}\spxextra{in namelist JULES\_WATER\_RESOURCES}|spxpagem}

\begin{fulllineitems}
\phantomsection\label{\detokenize{namelists/jules_water_resources.nml:JULES_WATER_RESOURCES::l_water_domestic}}
\pysigstartsignatures
\pysigline{\sphinxcode{\sphinxupquote{JULES\_WATER\_RESOURCES::}}\sphinxbfcode{\sphinxupquote{l\_water\_domestic}}}
\pysigstopsignatures\begin{quote}\begin{description}
\sphinxlineitem{Type}
\sphinxAtStartPar
logical

\sphinxlineitem{Default}
\sphinxAtStartPar
F

\end{description}\end{quote}

\sphinxAtStartPar
Switch for modelling of water for domestic use.
\begin{description}
\sphinxlineitem{TRUE}
\sphinxAtStartPar
Consider demand for water for domestic use. This requires that the domestic demand is prescribed as an input to the model (see {\hyperref[\detokenize{namelists/prescribed_data.nml::doc}]{\sphinxcrossref{\DUrole{doc}{prescribed\_data.nml}}}}).

\sphinxlineitem{FALSE}
\sphinxAtStartPar
Do not consider domestic demand.

\end{description}

\end{fulllineitems}

\index{l\_water\_environment (in namelist JULES\_WATER\_RESOURCES)@\spxentry{l\_water\_environment}\spxextra{in namelist JULES\_WATER\_RESOURCES}|spxpagem}

\begin{fulllineitems}
\phantomsection\label{\detokenize{namelists/jules_water_resources.nml:JULES_WATER_RESOURCES::l_water_environment}}
\pysigstartsignatures
\pysigline{\sphinxcode{\sphinxupquote{JULES\_WATER\_RESOURCES::}}\sphinxbfcode{\sphinxupquote{l\_water\_environment}}}
\pysigstopsignatures\begin{quote}\begin{description}
\sphinxlineitem{Type}
\sphinxAtStartPar
logical

\sphinxlineitem{Default}
\sphinxAtStartPar
F

\end{description}\end{quote}

\sphinxAtStartPar
Switch for modelling of water for environmental flow requirements.
\begin{description}
\sphinxlineitem{TRUE}
\sphinxAtStartPar
Consider demand for water for environmental flows.

\sphinxlineitem{FALSE}
\sphinxAtStartPar
Do not consider environmental demand.

\end{description}

\end{fulllineitems}

\index{l\_water\_industry (in namelist JULES\_WATER\_RESOURCES)@\spxentry{l\_water\_industry}\spxextra{in namelist JULES\_WATER\_RESOURCES}|spxpagem}

\begin{fulllineitems}
\phantomsection\label{\detokenize{namelists/jules_water_resources.nml:JULES_WATER_RESOURCES::l_water_industry}}
\pysigstartsignatures
\pysigline{\sphinxcode{\sphinxupquote{JULES\_WATER\_RESOURCES::}}\sphinxbfcode{\sphinxupquote{l\_water\_industry}}}
\pysigstopsignatures\begin{quote}\begin{description}
\sphinxlineitem{Type}
\sphinxAtStartPar
logical

\sphinxlineitem{Default}
\sphinxAtStartPar
F

\end{description}\end{quote}

\sphinxAtStartPar
Switch for modelling of water for industrial use. This requires that the industrial demand is prescribed as an input to the model (see {\hyperref[\detokenize{namelists/prescribed_data.nml::doc}]{\sphinxcrossref{\DUrole{doc}{prescribed\_data.nml}}}}).
\begin{description}
\sphinxlineitem{TRUE}
\sphinxAtStartPar
Consider demand for water for industrial use.

\sphinxlineitem{FALSE}
\sphinxAtStartPar
Do not consider industrial demand.

\end{description}

\end{fulllineitems}

\index{l\_water\_irrigation (in namelist JULES\_WATER\_RESOURCES)@\spxentry{l\_water\_irrigation}\spxextra{in namelist JULES\_WATER\_RESOURCES}|spxpagem}

\begin{fulllineitems}
\phantomsection\label{\detokenize{namelists/jules_water_resources.nml:JULES_WATER_RESOURCES::l_water_irrigation}}
\pysigstartsignatures
\pysigline{\sphinxcode{\sphinxupquote{JULES\_WATER\_RESOURCES::}}\sphinxbfcode{\sphinxupquote{l\_water\_irrigation}}}
\pysigstopsignatures\begin{quote}\begin{description}
\sphinxlineitem{Type}
\sphinxAtStartPar
logical

\sphinxlineitem{Default}
\sphinxAtStartPar
F

\end{description}\end{quote}

\sphinxAtStartPar
Switch for modelling of water for irrigation.
\begin{description}
\sphinxlineitem{TRUE}
\sphinxAtStartPar
Consider demand for water for irrigation. This must be used with {\hyperref[\detokenize{namelists/jules_irrig.nml:JULES_IRRIG::l_irrig_dmd}]{\sphinxcrossref{\sphinxcode{\sphinxupquote{l\_irrig\_dmd}}}}} = TRUE (to activate the inclusion of irrigation in other aspects of the model) and {\hyperref[\detokenize{namelists/jules_irrig.nml:JULES_IRRIG::l_irrig_limit}]{\sphinxcrossref{\sphinxcode{\sphinxupquote{l\_irrig\_limit}}}}} = FALSE (to avoid triggering an alternative approach to calculating the water available for irrigation).

\sphinxlineitem{FALSE}
\sphinxAtStartPar
Do not consider irrigation demand.

\end{description}

\end{fulllineitems}

\index{l\_water\_livestock (in namelist JULES\_WATER\_RESOURCES)@\spxentry{l\_water\_livestock}\spxextra{in namelist JULES\_WATER\_RESOURCES}|spxpagem}

\begin{fulllineitems}
\phantomsection\label{\detokenize{namelists/jules_water_resources.nml:JULES_WATER_RESOURCES::l_water_livestock}}
\pysigstartsignatures
\pysigline{\sphinxcode{\sphinxupquote{JULES\_WATER\_RESOURCES::}}\sphinxbfcode{\sphinxupquote{l\_water\_livestock}}}
\pysigstopsignatures\begin{quote}\begin{description}
\sphinxlineitem{Type}
\sphinxAtStartPar
logical

\sphinxlineitem{Default}
\sphinxAtStartPar
F

\end{description}\end{quote}

\sphinxAtStartPar
Switch for modelling of water for livestock.
\begin{description}
\sphinxlineitem{TRUE}
\sphinxAtStartPar
Consider demand for water for livestock. This requires that the livestock demand is prescribed as an input to the model (see {\hyperref[\detokenize{namelists/prescribed_data.nml::doc}]{\sphinxcrossref{\DUrole{doc}{prescribed\_data.nml}}}}).

\sphinxlineitem{FALSE}
\sphinxAtStartPar
Do not consider livestock demand.

\end{description}

\end{fulllineitems}

\index{l\_water\_transfers (in namelist JULES\_WATER\_RESOURCES)@\spxentry{l\_water\_transfers}\spxextra{in namelist JULES\_WATER\_RESOURCES}|spxpagem}

\begin{fulllineitems}
\phantomsection\label{\detokenize{namelists/jules_water_resources.nml:JULES_WATER_RESOURCES::l_water_transfers}}
\pysigstartsignatures
\pysigline{\sphinxcode{\sphinxupquote{JULES\_WATER\_RESOURCES::}}\sphinxbfcode{\sphinxupquote{l\_water\_transfers}}}
\pysigstopsignatures\begin{quote}\begin{description}
\sphinxlineitem{Type}
\sphinxAtStartPar
logical

\sphinxlineitem{Default}
\sphinxAtStartPar
F

\end{description}\end{quote}

\sphinxAtStartPar
Switch for modelling of water for transfers.
\begin{description}
\sphinxlineitem{TRUE}
\sphinxAtStartPar
Consider demand for water for transfers. This requires that the demand for transfers is prescribed as an input to the model (see {\hyperref[\detokenize{namelists/prescribed_data.nml::doc}]{\sphinxcrossref{\DUrole{doc}{prescribed\_data.nml}}}}).

\sphinxlineitem{FALSE}
\sphinxAtStartPar
Do not consider transfers.

\end{description}

\end{fulllineitems}

\end{sphinxadmonition}

\begin{sphinxadmonition}{note}{Switches that control prioritisation of demands.}
\index{l\_prioritise (in namelist JULES\_WATER\_RESOURCES)@\spxentry{l\_prioritise}\spxextra{in namelist JULES\_WATER\_RESOURCES}|spxpagem}

\begin{fulllineitems}
\phantomsection\label{\detokenize{namelists/jules_water_resources.nml:JULES_WATER_RESOURCES::l_prioritise}}
\pysigstartsignatures
\pysigline{\sphinxcode{\sphinxupquote{JULES\_WATER\_RESOURCES::}}\sphinxbfcode{\sphinxupquote{l\_prioritise}}}
\pysigstopsignatures\begin{quote}\begin{description}
\sphinxlineitem{Type}
\sphinxAtStartPar
logical

\sphinxlineitem{Default}
\sphinxAtStartPar
F

\end{description}\end{quote}

\sphinxAtStartPar
Switch controlling prioritisation of demands.
\begin{description}
\sphinxlineitem{TRUE}
\sphinxAtStartPar
Rank demands from sectors in priority order.

\sphinxlineitem{FALSE}
\sphinxAtStartPar
No prioritisation. No further items from this group are required.

\end{description}

\end{fulllineitems}

\index{priority (in namelist JULES\_WATER\_RESOURCES)@\spxentry{priority}\spxextra{in namelist JULES\_WATER\_RESOURCES}|spxpagem}

\begin{fulllineitems}
\phantomsection\label{\detokenize{namelists/jules_water_resources.nml:JULES_WATER_RESOURCES::priority}}
\pysigstartsignatures
\pysigline{\sphinxcode{\sphinxupquote{JULES\_WATER\_RESOURCES::}}\sphinxbfcode{\sphinxupquote{priority}}}
\pysigstopsignatures\begin{quote}\begin{description}
\sphinxlineitem{Type}
\sphinxAtStartPar
character

\sphinxlineitem{Default}
\sphinxAtStartPar
none

\end{description}\end{quote}

\sphinxAtStartPar
A list of water sector names, in order of decreasing priority \sphinxhyphen{} see the table below for valid names. Only used if {\hyperref[\detokenize{namelists/jules_water_resources.nml:JULES_WATER_RESOURCES::l_prioritise}]{\sphinxcrossref{\sphinxcode{\sphinxupquote{l\_prioritise}}}}} = TRUE. All active sectors (as selected by switches such as {\hyperref[\detokenize{namelists/jules_water_resources.nml:JULES_WATER_RESOURCES::l_water_domestic}]{\sphinxcrossref{\sphinxcode{\sphinxupquote{l\_water\_domestic}}}}}) must be represented in this list. The same prioritisation is used for all points in the domain.

\end{fulllineitems}



\begin{savenotes}\sphinxattablestart
\centering
\begin{tabulary}{\linewidth}[t]{|p{2cm}|p{9cm}|}
\hline
\sphinxstyletheadfamily 
\sphinxAtStartPar
Name
&\sphinxstyletheadfamily 
\sphinxAtStartPar
Description
\\
\hline
\sphinxAtStartPar
\sphinxcode{\sphinxupquote{dom}}
&
\sphinxAtStartPar
Domestic use
\\
\hline
\sphinxAtStartPar
\sphinxcode{\sphinxupquote{env}}
&
\sphinxAtStartPar
Environmental flows
\\
\hline
\sphinxAtStartPar
\sphinxcode{\sphinxupquote{ind}}
&
\sphinxAtStartPar
Industrial use
\\
\hline
\sphinxAtStartPar
\sphinxcode{\sphinxupquote{irr}}
&
\sphinxAtStartPar
Irrigation
\\
\hline
\sphinxAtStartPar
\sphinxcode{\sphinxupquote{liv}}
&
\sphinxAtStartPar
Livestock use
\\
\hline
\sphinxAtStartPar
\sphinxcode{\sphinxupquote{tra}}
&
\sphinxAtStartPar
Water transfers
\\
\hline
\end{tabulary}
\par
\sphinxattableend\end{savenotes}
\end{sphinxadmonition}
\index{nr\_gwater\_model (in namelist JULES\_WATER\_RESOURCES)@\spxentry{nr\_gwater\_model}\spxextra{in namelist JULES\_WATER\_RESOURCES}|spxpagem}

\begin{fulllineitems}
\phantomsection\label{\detokenize{namelists/jules_water_resources.nml:JULES_WATER_RESOURCES::nr_gwater_model}}
\pysigstartsignatures
\pysigline{\sphinxcode{\sphinxupquote{JULES\_WATER\_RESOURCES::}}\sphinxbfcode{\sphinxupquote{nr\_gwater\_model}}}
\pysigstopsignatures\begin{quote}\begin{description}
\sphinxlineitem{Type}
\sphinxAtStartPar
integer

\sphinxlineitem{Permitted}
\sphinxAtStartPar
0,1,2

\sphinxlineitem{Default}
\sphinxAtStartPar
none

\end{description}\end{quote}

\sphinxAtStartPar
Choice for the model of non\sphinxhyphen{}renewable groundwater. Non\sphinxhyphen{}renewable groundwater is water that is not otherwise explicitly included in the model. It is an idealised, infinite source of water which is typically intended to allow consideration of pumping of grounwater from deep reserves that are difficult to quantify.

\sphinxAtStartPar
Possible values are:
\begin{enumerate}
\sphinxsetlistlabels{\arabic}{enumi}{enumii}{}{.}%
\setcounter{enumi}{-1}
\item {} 
\begin{DUlineblock}{0em}
\item[] No non\sphinxhyphen{}renewable groundwater is considered.
\end{DUlineblock}

\item {} 
\begin{DUlineblock}{0em}
\item[] Non\sphinxhyphen{}renewable groundwater is used as a last resort, when no other sources of water are available.
\end{DUlineblock}

\item {} 
\begin{DUlineblock}{0em}
\item[] Non\sphinxhyphen{}renewable groundwater is used as as ‘part of the mix’, in conjunction with other sources of water.
\end{DUlineblock}

\end{enumerate}

\end{fulllineitems}

\index{rf\_domestic (in namelist JULES\_WATER\_RESOURCES)@\spxentry{rf\_domestic}\spxextra{in namelist JULES\_WATER\_RESOURCES}|spxpagem}

\begin{fulllineitems}
\phantomsection\label{\detokenize{namelists/jules_water_resources.nml:JULES_WATER_RESOURCES::rf_domestic}}
\pysigstartsignatures
\pysigline{\sphinxcode{\sphinxupquote{JULES\_WATER\_RESOURCES::}}\sphinxbfcode{\sphinxupquote{rf\_domestic}}}
\pysigstopsignatures\begin{quote}\begin{description}
\sphinxlineitem{Type}
\sphinxAtStartPar
real

\sphinxlineitem{Permitted}
\sphinxAtStartPar
0\sphinxhyphen{}1

\sphinxlineitem{Default}
\sphinxAtStartPar
none

\end{description}\end{quote}

\sphinxAtStartPar
The fraction of water that is returned after abstraction for domestic use (via sewage systems etc.). Only used if {\hyperref[\detokenize{namelists/jules_water_resources.nml:JULES_WATER_RESOURCES::l_water_domestic}]{\sphinxcrossref{\sphinxcode{\sphinxupquote{l\_water\_domestic}}}}} = TRUE.

\end{fulllineitems}

\index{rf\_industry (in namelist JULES\_WATER\_RESOURCES)@\spxentry{rf\_industry}\spxextra{in namelist JULES\_WATER\_RESOURCES}|spxpagem}

\begin{fulllineitems}
\phantomsection\label{\detokenize{namelists/jules_water_resources.nml:JULES_WATER_RESOURCES::rf_industry}}
\pysigstartsignatures
\pysigline{\sphinxcode{\sphinxupquote{JULES\_WATER\_RESOURCES::}}\sphinxbfcode{\sphinxupquote{rf\_industry}}}
\pysigstopsignatures\begin{quote}\begin{description}
\sphinxlineitem{Type}
\sphinxAtStartPar
real

\sphinxlineitem{Permitted}
\sphinxAtStartPar
0\sphinxhyphen{}1

\sphinxlineitem{Default}
\sphinxAtStartPar
none

\end{description}\end{quote}

\sphinxAtStartPar
The fraction of water that is returned after abstraction for industrial use. Only used if {\hyperref[\detokenize{namelists/jules_water_resources.nml:JULES_WATER_RESOURCES::l_water_industry}]{\sphinxcrossref{\sphinxcode{\sphinxupquote{l\_water\_industry}}}}} = TRUE.

\end{fulllineitems}

\index{rf\_livestock (in namelist JULES\_WATER\_RESOURCES)@\spxentry{rf\_livestock}\spxextra{in namelist JULES\_WATER\_RESOURCES}|spxpagem}

\begin{fulllineitems}
\phantomsection\label{\detokenize{namelists/jules_water_resources.nml:JULES_WATER_RESOURCES::rf_livestock}}
\pysigstartsignatures
\pysigline{\sphinxcode{\sphinxupquote{JULES\_WATER\_RESOURCES::}}\sphinxbfcode{\sphinxupquote{rf\_livestock}}}
\pysigstopsignatures\begin{quote}\begin{description}
\sphinxlineitem{Type}
\sphinxAtStartPar
real

\sphinxlineitem{Permitted}
\sphinxAtStartPar
0\sphinxhyphen{}1

\sphinxlineitem{Default}
\sphinxAtStartPar
none

\end{description}\end{quote}

\sphinxAtStartPar
The fraction of water that is returned after abstraction for livestock. Only used if {\hyperref[\detokenize{namelists/jules_water_resources.nml:JULES_WATER_RESOURCES::l_water_livestock}]{\sphinxcrossref{\sphinxcode{\sphinxupquote{l\_water\_livestock}}}}} = TRUE.

\end{fulllineitems}


\sphinxstepscope


\section{\sphinxstyleliteralintitle{\sphinxupquote{jules\_irrig.nml}}}
\label{\detokenize{namelists/jules_irrig.nml:jules-irrig-nml}}\label{\detokenize{namelists/jules_irrig.nml::doc}}
\sphinxAtStartPar
This file sets the irrigation options. It contains one namelist called {\hyperref[\detokenize{namelists/jules_irrig.nml:namelist-JULES_IRRIG}]{\sphinxcrossref{\sphinxcode{\sphinxupquote{JULES\_IRRIG}}}}}.


\subsection{\sphinxstyleliteralintitle{\sphinxupquote{JULES\_IRRIG}} namelist members}
\label{\detokenize{namelists/jules_irrig.nml:namelist-JULES_IRRIG}}\label{\detokenize{namelists/jules_irrig.nml:jules-irrig-namelist-members}}\index{JULES\_IRRIG (namelist)@\spxentry{JULES\_IRRIG}\spxextra{namelist}|spxpagem}
\sphinxAtStartPar
This namelist specifies the different options available for setting up the irrigation.

\begin{sphinxadmonition}{note}{Note:}\begin{description}
\sphinxlineitem{Irrigation can be applied at a constant rate in three ways:}\begin{itemize}
\item {} \begin{enumerate}
\sphinxsetlistlabels{\arabic}{enumi}{enumii}{}{.}%
\item {} 
\sphinxAtStartPar
To apply a constant irrigation to all surface tiles the irrigation settings are as follows: frac\_irrig\_all\_tiles=T, set\_irrfrac\_on\_irrtiles=F and set a value for {\hyperref[\detokenize{namelists/ancillaries.nml:JULES_IRRIG_PROPS::const_frac_irr}]{\sphinxcrossref{\sphinxcode{\sphinxupquote{const\_frac\_irr}}}}}.

\end{enumerate}

\item {} \begin{enumerate}
\sphinxsetlistlabels{\arabic}{enumi}{enumii}{}{.}%
\setcounter{enumi}{1}
\item {} 
\sphinxAtStartPar
To apply a constant irrigation to only specific surface tiles the irrigation settings are as follows: frac\_irrig\_all\_tiles=F and set\_irrfrac\_on\_irrtiles=T and set a value for {\hyperref[\detokenize{namelists/ancillaries.nml:JULES_IRRIG_PROPS::const_irrfrac_irrtiles}]{\sphinxcrossref{\sphinxcode{\sphinxupquote{const\_irrfrac\_irrtiles}}}}}.

\end{enumerate}

\item {} \begin{enumerate}
\sphinxsetlistlabels{\arabic}{enumi}{enumii}{}{.}%
\setcounter{enumi}{2}
\item {} 
\sphinxAtStartPar
To apply a constant irrigation to specific surface tiles as an average across the gridbox, which is the way irrigation on specific tiles was done prior to vn5.7, the irrigation settings are as follows: frac\_irrig\_all\_tiles=F, set\_irrfrac\_on\_irrtiles=F and set a value for {\hyperref[\detokenize{namelists/ancillaries.nml:JULES_IRRIG_PROPS::const_frac_irr}]{\sphinxcrossref{\sphinxcode{\sphinxupquote{const\_frac\_irr}}}}}.

\end{enumerate}

\end{itemize}

\end{description}

\sphinxAtStartPar
In both option 2 and 3, {\hyperref[\detokenize{namelists/jules_irrig.nml:JULES_IRRIG::irrigtiles}]{\sphinxcrossref{\sphinxcode{\sphinxupquote{irrigtiles}}}}} is the index of surface tiles you wish to irrigate, the length of which is {\hyperref[\detokenize{namelists/jules_irrig.nml:JULES_IRRIG::nirrtile}]{\sphinxcrossref{\sphinxcode{\sphinxupquote{nirrtile}}}}} e.g. if you include wheat and maize in your run at index 5 and 6 then irrigtiles=5,6 and nirrtile=2.
\end{sphinxadmonition}
\index{l\_irrig\_dmd (in namelist JULES\_IRRIG)@\spxentry{l\_irrig\_dmd}\spxextra{in namelist JULES\_IRRIG}|spxpagem}

\begin{fulllineitems}
\phantomsection\label{\detokenize{namelists/jules_irrig.nml:JULES_IRRIG::l_irrig_dmd}}
\pysigstartsignatures
\pysigline{\sphinxcode{\sphinxupquote{JULES\_IRRIG::}}\sphinxbfcode{\sphinxupquote{l\_irrig\_dmd}}}
\pysigstopsignatures\begin{quote}\begin{description}
\sphinxlineitem{Type}
\sphinxAtStartPar
logical

\sphinxlineitem{Default}
\sphinxAtStartPar
F

\end{description}\end{quote}

\sphinxAtStartPar
Switch controlling the implementation of irrigation demand code.
\begin{description}
\sphinxlineitem{TRUE}
\sphinxAtStartPar
Tiles are irrigated.

\sphinxlineitem{FALSE}
\sphinxAtStartPar
No effect.

\end{description}

\sphinxAtStartPar
This must be set to TRUE if {\hyperref[\detokenize{namelists/jules_water_resources.nml:JULES_WATER_RESOURCES::l_water_irrigation}]{\sphinxcrossref{\sphinxcode{\sphinxupquote{l\_water\_irrigation}}}}} = TRUE.

\end{fulllineitems}

\index{l\_irrig\_limit (in namelist JULES\_IRRIG)@\spxentry{l\_irrig\_limit}\spxextra{in namelist JULES\_IRRIG}|spxpagem}

\begin{fulllineitems}
\phantomsection\label{\detokenize{namelists/jules_irrig.nml:JULES_IRRIG::l_irrig_limit}}
\pysigstartsignatures
\pysigline{\sphinxcode{\sphinxupquote{JULES\_IRRIG::}}\sphinxbfcode{\sphinxupquote{l\_irrig\_limit}}}
\pysigstopsignatures\begin{quote}\begin{description}
\sphinxlineitem{Type}
\sphinxAtStartPar
logical

\sphinxlineitem{Default}
\sphinxAtStartPar
F

\end{description}\end{quote}

\sphinxAtStartPar
Switch controlling whether the amount of water used to irrigate tiles is limited.
\begin{description}
\sphinxlineitem{TRUE}
\sphinxAtStartPar
Water for irrigation is taken first from the deep soil
(groundwater) store, and then from the river storage when the
deep soil store is exhausted. Tiles are irrigated up to the
critical point if the necessary water is available. This option
requires {\hyperref[\detokenize{namelists/jules_irrig.nml:JULES_IRRIG::l_irrig_dmd}]{\sphinxcrossref{\sphinxcode{\sphinxupquote{l\_irrig\_dmd}}}}} = TRUE,
{\hyperref[\detokenize{namelists/jules_hydrology.nml:JULES_HYDROLOGY::l_top}]{\sphinxcrossref{\sphinxcode{\sphinxupquote{l\_top}}}}} = TRUE,
{\hyperref[\detokenize{namelists/jules_rivers.nml:JULES_RIVERS::l_rivers}]{\sphinxcrossref{\sphinxcode{\sphinxupquote{l\_rivers}}}}} = TRUE and
{\hyperref[\detokenize{namelists/jules_rivers.nml:JULES_RIVERS::i_river_vn}]{\sphinxcrossref{\sphinxcode{\sphinxupquote{i\_river\_vn}}}}} = \sphinxcode{\sphinxupquote{1,3}}.

\begin{sphinxadmonition}{warning}{Warning:}
\sphinxAtStartPar
The irrigation supply code in JULES is still in development,
and is available in this release to support beta testing
activities.

\sphinxAtStartPar
Users should ensure that results are as expected, and provide
feedback where deficiencies are identified.
\end{sphinxadmonition}

\sphinxlineitem{FALSE}
\sphinxAtStartPar
Tiles will be irrigated to critical point from an unconstrained water supply.

\end{description}

\sphinxAtStartPar
This must be set to FALSE if {\hyperref[\detokenize{namelists/jules_water_resources.nml:JULES_WATER_RESOURCES::l_water_irrigation}]{\sphinxcrossref{\sphinxcode{\sphinxupquote{l\_water\_irrigation}}}}} = TRUE.

\end{fulllineitems}

\index{irr\_crop (in namelist JULES\_IRRIG)@\spxentry{irr\_crop}\spxextra{in namelist JULES\_IRRIG}|spxpagem}

\begin{fulllineitems}
\phantomsection\label{\detokenize{namelists/jules_irrig.nml:JULES_IRRIG::irr_crop}}
\pysigstartsignatures
\pysigline{\sphinxcode{\sphinxupquote{JULES\_IRRIG::}}\sphinxbfcode{\sphinxupquote{irr\_crop}}}
\pysigstopsignatures\begin{quote}\begin{description}
\sphinxlineitem{Type}
\sphinxAtStartPar
integer

\sphinxlineitem{Permitted}
\sphinxAtStartPar
0, 1 or 2

\sphinxlineitem{Default}
\sphinxAtStartPar
0

\end{description}\end{quote}
\begin{enumerate}
\sphinxsetlistlabels{\arabic}{enumi}{enumii}{}{.}%
\setcounter{enumi}{-1}
\item {} 
\sphinxAtStartPar
Irrigation season (i.e. season in which crops might be growing
on the gridbox) lasts the entire year.

\item {} 
\sphinxAtStartPar
Irrigation season is determined from driving data according to
{\hyperref[\detokenize{namelists/jules_irrig.nml:references-irrig}]{\sphinxcrossref{\DUrole{std,std-ref}{Döll \& Siebert (2002)}}}} method. No irrigation
is applied outside the irrigation season.

\item {} 
\sphinxAtStartPar
Irrigation season is determined by maximum dvi across all
surface tiles. Requires {\hyperref[\detokenize{namelists/jules_surface_types.nml:JULES_SURFACE_TYPES::ncpft}]{\sphinxcrossref{\sphinxcode{\sphinxupquote{ncpft}}}}} \textgreater{} 0. No
irrigation is applied outside the irrigation season.

\end{enumerate}

\end{fulllineitems}

\index{frac\_irrig\_all\_tiles (in namelist JULES\_IRRIG)@\spxentry{frac\_irrig\_all\_tiles}\spxextra{in namelist JULES\_IRRIG}|spxpagem}

\begin{fulllineitems}
\phantomsection\label{\detokenize{namelists/jules_irrig.nml:JULES_IRRIG::frac_irrig_all_tiles}}
\pysigstartsignatures
\pysigline{\sphinxcode{\sphinxupquote{JULES\_IRRIG::}}\sphinxbfcode{\sphinxupquote{frac\_irrig\_all\_tiles}}}
\pysigstopsignatures\begin{quote}\begin{description}
\sphinxlineitem{Type}
\sphinxAtStartPar
logical

\sphinxlineitem{Default}
\sphinxAtStartPar
T

\end{description}\end{quote}

\sphinxAtStartPar
If T, then irrigation fraction is applied to all surface tiles, and F, it is applied only to the surface tiles specified in {\hyperref[\detokenize{namelists/jules_irrig.nml:JULES_IRRIG::irrigtiles}]{\sphinxcrossref{\sphinxcode{\sphinxupquote{irrigtiles}}}}}.

\end{fulllineitems}

\index{set\_irrfrac\_on\_irrtiles (in namelist JULES\_IRRIG)@\spxentry{set\_irrfrac\_on\_irrtiles}\spxextra{in namelist JULES\_IRRIG}|spxpagem}

\begin{fulllineitems}
\phantomsection\label{\detokenize{namelists/jules_irrig.nml:JULES_IRRIG::set_irrfrac_on_irrtiles}}
\pysigstartsignatures
\pysigline{\sphinxcode{\sphinxupquote{JULES\_IRRIG::}}\sphinxbfcode{\sphinxupquote{set\_irrfrac\_on\_irrtiles}}}
\pysigstopsignatures\begin{quote}\begin{description}
\sphinxlineitem{Type}
\sphinxAtStartPar
logical

\sphinxlineitem{Default}
\sphinxAtStartPar
F

\end{description}\end{quote}

\sphinxAtStartPar
If F then irrigation is applied as an average across the gridbox and not to specific surface tiles. If T, then the irrigation fraction is only applied to the surface tile specified in {\hyperref[\detokenize{namelists/jules_irrig.nml:JULES_IRRIG::irrigtiles}]{\sphinxcrossref{\sphinxcode{\sphinxupquote{irrigtiles}}}}}. Both {\hyperref[\detokenize{namelists/jules_irrig.nml:JULES_IRRIG::frac_irrig_all_tiles}]{\sphinxcrossref{\sphinxcode{\sphinxupquote{frac\_irrig\_all\_tiles}}}}} and {\hyperref[\detokenize{namelists/jules_irrig.nml:JULES_IRRIG::set_irrfrac_on_irrtiles}]{\sphinxcrossref{\sphinxcode{\sphinxupquote{set\_irrfrac\_on\_irrtiles}}}}} cannot be set to T.

\end{fulllineitems}

\index{nirrtile (in namelist JULES\_IRRIG)@\spxentry{nirrtile}\spxextra{in namelist JULES\_IRRIG}|spxpagem}

\begin{fulllineitems}
\phantomsection\label{\detokenize{namelists/jules_irrig.nml:JULES_IRRIG::nirrtile}}
\pysigstartsignatures
\pysigline{\sphinxcode{\sphinxupquote{JULES\_IRRIG::}}\sphinxbfcode{\sphinxupquote{nirrtile}}}
\pysigstopsignatures\begin{quote}\begin{description}
\sphinxlineitem{Type}
\sphinxAtStartPar
integer

\sphinxlineitem{Default}
\sphinxAtStartPar
None

\end{description}\end{quote}

\sphinxAtStartPar
The number of surface tiles to be irrigated. Only used if {\hyperref[\detokenize{namelists/jules_irrig.nml:JULES_IRRIG::frac_irrig_all_tiles}]{\sphinxcrossref{\sphinxcode{\sphinxupquote{frac\_irrig\_all\_tiles}}}}} = F.

\end{fulllineitems}

\index{irrigtiles (in namelist JULES\_IRRIG)@\spxentry{irrigtiles}\spxextra{in namelist JULES\_IRRIG}|spxpagem}

\begin{fulllineitems}
\phantomsection\label{\detokenize{namelists/jules_irrig.nml:JULES_IRRIG::irrigtiles}}
\pysigstartsignatures
\pysigline{\sphinxcode{\sphinxupquote{JULES\_IRRIG::}}\sphinxbfcode{\sphinxupquote{irrigtiles}}}
\pysigstopsignatures\begin{quote}\begin{description}
\sphinxlineitem{Type}
\sphinxAtStartPar
integer(nirrtile)

\sphinxlineitem{Default}
\sphinxAtStartPar
None

\end{description}\end{quote}

\sphinxAtStartPar
Indices of the surface tiles to be irrigated. Only used if {\hyperref[\detokenize{namelists/jules_irrig.nml:JULES_IRRIG::frac_irrig_all_tiles}]{\sphinxcrossref{\sphinxcode{\sphinxupquote{frac\_irrig\_all\_tiles}}}}} = F.

\end{fulllineitems}

\index{nstep\_irrig (in namelist JULES\_IRRIG)@\spxentry{nstep\_irrig}\spxextra{in namelist JULES\_IRRIG}|spxpagem}

\begin{fulllineitems}
\phantomsection\label{\detokenize{namelists/jules_irrig.nml:JULES_IRRIG::nstep_irrig}}
\pysigstartsignatures
\pysigline{\sphinxcode{\sphinxupquote{JULES\_IRRIG::}}\sphinxbfcode{\sphinxupquote{nstep\_irrig}}}
\pysigstopsignatures\begin{quote}\begin{description}
\sphinxlineitem{Type}
\sphinxAtStartPar
integer

\sphinxlineitem{Permitted}
\sphinxAtStartPar
\textgreater{} 0

\sphinxlineitem{Default}
\sphinxAtStartPar
86400/{\hyperref[\detokenize{namelists/timesteps.nml:JULES_TIME::timestep_len}]{\sphinxcrossref{\sphinxcode{\sphinxupquote{timestep\_len}}}}}

\end{description}\end{quote}

\sphinxAtStartPar
The number of model timesteps per irrigation update step

\sphinxAtStartPar
Irrigation will be updated every {\hyperref[\detokenize{namelists/jules_irrig.nml:JULES_IRRIG::nstep_irrig}]{\sphinxcrossref{\sphinxcode{\sphinxupquote{nstep\_irrig}}}}} timesteps. For example, with a model timestep of 1 hour, {\hyperref[\detokenize{namelists/jules_irrig.nml:JULES_IRRIG::nstep_irrig}]{\sphinxcrossref{\sphinxcode{\sphinxupquote{nstep\_irrig}}}}} = 24 means that irrigation will be updated on the 24th timestep, i.e. daily updates.

\sphinxAtStartPar
{\hyperref[\detokenize{namelists/jules_irrig.nml:JULES_IRRIG::nstep_irrig}]{\sphinxcrossref{\sphinxcode{\sphinxupquote{nstep\_irrig}}}}} = NINT(frequency of irrigation update (in sec)) / {\hyperref[\detokenize{namelists/timesteps.nml:JULES_TIME::timestep_len}]{\sphinxcrossref{\sphinxcode{\sphinxupquote{timestep\_len}}}}})

\end{fulllineitems}



\subsection{\sphinxstyleliteralintitle{\sphinxupquote{JULES\_IRRIG}} references}
\label{\detokenize{namelists/jules_irrig.nml:jules-irrig-references}}\label{\detokenize{namelists/jules_irrig.nml:references-irrig}}\begin{itemize}
\item {} 
\sphinxAtStartPar
Döll, P., and Siebert, S., Global modeling of irrigation water
requirements, Water Resour. Res., 38(4),
\sphinxurl{https://doi.org/10.1029/2001WR000355}, 2002.

\end{itemize}

\sphinxstepscope


\section{\sphinxstyleliteralintitle{\sphinxupquote{science\_fixes.nml}}}
\label{\detokenize{namelists/science_fixes.nml:science-fixes-nml}}\label{\detokenize{namelists/science_fixes.nml::doc}}
\sphinxAtStartPar
This file contains one namelist called {\hyperref[\detokenize{namelists/science_fixes.nml:namelist-JULES_TEMP_FIXES}]{\sphinxcrossref{\sphinxcode{\sphinxupquote{JULES\_TEMP\_FIXES}}}}}.

\sphinxAtStartPar
This namelist sets ‘short\sphinxhyphen{}term’ temporary logicals used to protect science bug
fixes that lead to alterations in science results. It is expected that these
logicals will be short lived as the preference should be for all configurations
to use the corrected code. However, to maintain short term reproducibility of
results across JULES versions the fixes are protected by logicals until the
fixes become the default in all model configurations at which point the logical
is retired. See module for when the switch is due for review.


\subsection{\sphinxstyleliteralintitle{\sphinxupquote{JULES\_TEMP\_FIXES}} namelist members}
\label{\detokenize{namelists/science_fixes.nml:namelist-JULES_TEMP_FIXES}}\label{\detokenize{namelists/science_fixes.nml:jules-temp-fixes-namelist-members}}\index{JULES\_TEMP\_FIXES (namelist)@\spxentry{JULES\_TEMP\_FIXES}\spxextra{namelist}|spxpagem}\index{ctile\_orog\_fix (in namelist JULES\_TEMP\_FIXES)@\spxentry{ctile\_orog\_fix}\spxextra{in namelist JULES\_TEMP\_FIXES}|spxpagem}

\begin{fulllineitems}
\phantomsection\label{\detokenize{namelists/science_fixes.nml:JULES_TEMP_FIXES::ctile_orog_fix}}
\pysigstartsignatures
\pysigline{\sphinxcode{\sphinxupquote{JULES\_TEMP\_FIXES::}}\sphinxbfcode{\sphinxupquote{ctile\_orog\_fix}}}
\pysigstopsignatures\begin{quote}\begin{description}
\sphinxlineitem{Type}
\sphinxAtStartPar
integer

\sphinxlineitem{Permitted}
\sphinxAtStartPar
0\sphinxhyphen{}2

\sphinxlineitem{Default}
\sphinxAtStartPar
2

\end{description}\end{quote}

\sphinxAtStartPar
If nonzero, corrects the surface exchange calculations in coastally
tiled grid\sphinxhyphen{}boxes, assuming that the lowest level is physically
terrain following and adjusting the temperature of the land/sea
portions in accordance with their relative offset from the grid\sphinxhyphen{}box
mean height using a dry/moist lapse rate where appropriate. Option 2
will only adjust values over the sea.

\end{fulllineitems}

\index{l\_accurate\_rho (in namelist JULES\_TEMP\_FIXES)@\spxentry{l\_accurate\_rho}\spxextra{in namelist JULES\_TEMP\_FIXES}|spxpagem}

\begin{fulllineitems}
\phantomsection\label{\detokenize{namelists/science_fixes.nml:JULES_TEMP_FIXES::l_accurate_rho}}
\pysigstartsignatures
\pysigline{\sphinxcode{\sphinxupquote{JULES\_TEMP\_FIXES::}}\sphinxbfcode{\sphinxupquote{l\_accurate\_rho}}}
\pysigstopsignatures\begin{quote}\begin{description}
\sphinxlineitem{Type}
\sphinxAtStartPar
logical

\sphinxlineitem{Default}
\sphinxAtStartPar
F

\end{description}\end{quote}

\sphinxAtStartPar
This switch improves the calculation of surface air density in the
surface turbulent fluxes.  It includes appropriate use of dry air density
when the atmospheric water vapour is expressed as a mixing ratio
(l\_mr\_physics = .TRUE.), otherwise use the wet air density when
it is expressed as a specific humidity.

\end{fulllineitems}

\index{l\_dtcanfix (in namelist JULES\_TEMP\_FIXES)@\spxentry{l\_dtcanfix}\spxextra{in namelist JULES\_TEMP\_FIXES}|spxpagem}

\begin{fulllineitems}
\phantomsection\label{\detokenize{namelists/science_fixes.nml:JULES_TEMP_FIXES::l_dtcanfix}}
\pysigstartsignatures
\pysigline{\sphinxcode{\sphinxupquote{JULES\_TEMP\_FIXES::}}\sphinxbfcode{\sphinxupquote{l\_dtcanfix}}}
\pysigstopsignatures\begin{quote}\begin{description}
\sphinxlineitem{Type}
\sphinxAtStartPar
logical

\sphinxlineitem{Default}
\sphinxAtStartPar
F

\end{description}\end{quote}

\sphinxAtStartPar
This switch corrects a bug in the evolution of the skin temperature in
the implicit solver,
whereby the change in the skin temperature is
artificially constrained. This generally has a small effect,
but can
cause unphysical results if a canopy with a large heat capacity is
coupled to an underlying substrate with a small heat capacity.

\end{fulllineitems}

\index{l\_fix\_alb\_ice\_thick (in namelist JULES\_TEMP\_FIXES)@\spxentry{l\_fix\_alb\_ice\_thick}\spxextra{in namelist JULES\_TEMP\_FIXES}|spxpagem}

\begin{fulllineitems}
\phantomsection\label{\detokenize{namelists/science_fixes.nml:JULES_TEMP_FIXES::l_fix_alb_ice_thick}}
\pysigstartsignatures
\pysigline{\sphinxcode{\sphinxupquote{JULES\_TEMP\_FIXES::}}\sphinxbfcode{\sphinxupquote{l\_fix\_alb\_ice\_thick}}}
\pysigstopsignatures\begin{quote}\begin{description}
\sphinxlineitem{Type}
\sphinxAtStartPar
logical

\sphinxlineitem{Default}
\sphinxAtStartPar
F

\end{description}\end{quote}

\sphinxAtStartPar
When zero\sphinxhyphen{}layer sea ice is used the thermodynamics is calculated in the
UM through an effective thickness calculated from snow and ice thicknesses
and associated thermal conductivities. With multi\sphinxhyphen{}layer sea ice the
thermodynamics is calculated in the sea ice component of the model, and
the effective thickness is no longer required.  However, it was still
being used erroneously. This fix removes the effective thickness
adjustment when multi\sphinxhyphen{}layer sea ice is used.

\end{fulllineitems}

\index{l\_fix\_albsnow\_ts (in namelist JULES\_TEMP\_FIXES)@\spxentry{l\_fix\_albsnow\_ts}\spxextra{in namelist JULES\_TEMP\_FIXES}|spxpagem}

\begin{fulllineitems}
\phantomsection\label{\detokenize{namelists/science_fixes.nml:JULES_TEMP_FIXES::l_fix_albsnow_ts}}
\pysigstartsignatures
\pysigline{\sphinxcode{\sphinxupquote{JULES\_TEMP\_FIXES::}}\sphinxbfcode{\sphinxupquote{l\_fix\_albsnow\_ts}}}
\pysigstopsignatures\begin{quote}\begin{description}
\sphinxlineitem{Type}
\sphinxAtStartPar
logical

\sphinxlineitem{Default}
\sphinxAtStartPar
F

\end{description}\end{quote}

\sphinxAtStartPar
The original version of the two\sphinxhyphen{}stream scheme to calculate the albedo
of snow in JULES contains a bug in the calculation of the reflection
coefficient that renders very thin layers of snow too reflective.
This logical applies the appropriate correction when it is enabled.

\end{fulllineitems}

\index{l\_fix\_lake\_ice\_temperatures (in namelist JULES\_TEMP\_FIXES)@\spxentry{l\_fix\_lake\_ice\_temperatures}\spxextra{in namelist JULES\_TEMP\_FIXES}|spxpagem}

\begin{fulllineitems}
\phantomsection\label{\detokenize{namelists/science_fixes.nml:JULES_TEMP_FIXES::l_fix_lake_ice_temperatures}}
\pysigstartsignatures
\pysigline{\sphinxcode{\sphinxupquote{JULES\_TEMP\_FIXES::}}\sphinxbfcode{\sphinxupquote{l\_fix\_lake\_ice\_temperatures}}}
\pysigstopsignatures\begin{quote}\begin{description}
\sphinxlineitem{Type}
\sphinxAtStartPar
logical

\sphinxlineitem{Default}
\sphinxAtStartPar
F

\end{description}\end{quote}

\sphinxAtStartPar
If true, allows sea ice temperatures in lakes to evolve over time
for coupled models when the lake is defined as a sea point but is
not coupled to an ocean model.

\end{fulllineitems}

\index{l\_fix\_moruses\_roof\_rad\_coupling (in namelist JULES\_TEMP\_FIXES)@\spxentry{l\_fix\_moruses\_roof\_rad\_coupling}\spxextra{in namelist JULES\_TEMP\_FIXES}|spxpagem}

\begin{fulllineitems}
\phantomsection\label{\detokenize{namelists/science_fixes.nml:JULES_TEMP_FIXES::l_fix_moruses_roof_rad_coupling}}
\pysigstartsignatures
\pysigline{\sphinxcode{\sphinxupquote{JULES\_TEMP\_FIXES::}}\sphinxbfcode{\sphinxupquote{l\_fix\_moruses\_roof\_rad\_coupling}}}
\pysigstopsignatures\begin{quote}\begin{description}
\sphinxlineitem{Type}
\sphinxAtStartPar
logical

\sphinxlineitem{Default}
\sphinxAtStartPar
F

\end{description}\end{quote}

\sphinxAtStartPar
If true, this switch corrects a bug in the surface energy balance
when the MORUSES radiative roof coupling is used
(see {\hyperref[\detokenize{namelists/urban.nml:JULES_URBAN::l_moruses_storage}]{\sphinxcrossref{\sphinxcode{\sphinxupquote{l\_moruses\_storage}}}}}).
If false, the thermal conductivity of the soil (hcons) is erroneously
set to zero, which causes the roof to be effectively uncoupled when
{\hyperref[\detokenize{namelists/jules_vegetation.nml:JULES_VEGETATION::l_vegcan_soilfx}]{\sphinxcrossref{\sphinxcode{\sphinxupquote{l\_vegcan\_soilfx}}}}}.

\end{fulllineitems}

\index{l\_fix\_osa\_chloro (in namelist JULES\_TEMP\_FIXES)@\spxentry{l\_fix\_osa\_chloro}\spxextra{in namelist JULES\_TEMP\_FIXES}|spxpagem}

\begin{fulllineitems}
\phantomsection\label{\detokenize{namelists/science_fixes.nml:JULES_TEMP_FIXES::l_fix_osa_chloro}}
\pysigstartsignatures
\pysigline{\sphinxcode{\sphinxupquote{JULES\_TEMP\_FIXES::}}\sphinxbfcode{\sphinxupquote{l\_fix\_osa\_chloro}}}
\pysigstopsignatures\begin{quote}\begin{description}
\sphinxlineitem{Type}
\sphinxAtStartPar
logical

\sphinxlineitem{Default}
\sphinxAtStartPar
F

\end{description}\end{quote}

\sphinxAtStartPar
When set to false, the chlorophyll content used to determine the optical
properties of water, for the ocean surface albedo, are specified in gm\sphinxhyphen{}3
when the parameterisation they use is defined in mg m\sphinxhyphen{}3.
It is a short term logical until the code becomes the new default.

\end{fulllineitems}

\index{l\_fix\_snow\_frac (in namelist JULES\_TEMP\_FIXES)@\spxentry{l\_fix\_snow\_frac}\spxextra{in namelist JULES\_TEMP\_FIXES}|spxpagem}

\begin{fulllineitems}
\phantomsection\label{\detokenize{namelists/science_fixes.nml:JULES_TEMP_FIXES::l_fix_snow_frac}}
\pysigstartsignatures
\pysigline{\sphinxcode{\sphinxupquote{JULES\_TEMP\_FIXES::}}\sphinxbfcode{\sphinxupquote{l\_fix\_snow\_frac}}}
\pysigstopsignatures\begin{quote}\begin{description}
\sphinxlineitem{Type}
\sphinxAtStartPar
logical

\sphinxlineitem{Default}
\sphinxAtStartPar
F

\end{description}\end{quote}

\sphinxAtStartPar
When set to  false, there is the potential to have small snow mass, but a
zero snow fraction due to machine precision in the calculations. This
prevents sublimation or snow melt from removing the remaining snow mass,
hence small values can persist.
In addition to this there is a conceptual bug in the calculation of
the fraction of potential evaporation because it does not add in canopy
evaporation when the snow fraction is less than one.
When set to true these issues are corrected and in addition the radiation
calculations for snow fraction are also made consistent.

\end{fulllineitems}

\index{l\_fix\_ustar\_dust (in namelist JULES\_TEMP\_FIXES)@\spxentry{l\_fix\_ustar\_dust}\spxextra{in namelist JULES\_TEMP\_FIXES}|spxpagem}

\begin{fulllineitems}
\phantomsection\label{\detokenize{namelists/science_fixes.nml:JULES_TEMP_FIXES::l_fix_ustar_dust}}
\pysigstartsignatures
\pysigline{\sphinxcode{\sphinxupquote{JULES\_TEMP\_FIXES::}}\sphinxbfcode{\sphinxupquote{l\_fix\_ustar\_dust}}}
\pysigstopsignatures\begin{quote}\begin{description}
\sphinxlineitem{Type}
\sphinxAtStartPar
logical

\sphinxlineitem{Default}
\sphinxAtStartPar
F

\end{description}\end{quote}

\sphinxAtStartPar
If true, corrects how ustar is calculated in the exchange
coefficient for dust deposition

\end{fulllineitems}

\index{l\_fix\_wind\_snow (in namelist JULES\_TEMP\_FIXES)@\spxentry{l\_fix\_wind\_snow}\spxextra{in namelist JULES\_TEMP\_FIXES}|spxpagem}

\begin{fulllineitems}
\phantomsection\label{\detokenize{namelists/science_fixes.nml:JULES_TEMP_FIXES::l_fix_wind_snow}}
\pysigstartsignatures
\pysigline{\sphinxcode{\sphinxupquote{JULES\_TEMP\_FIXES::}}\sphinxbfcode{\sphinxupquote{l\_fix\_wind\_snow}}}
\pysigstopsignatures\begin{quote}\begin{description}
\sphinxlineitem{Type}
\sphinxAtStartPar
logical

\sphinxlineitem{Default}
\sphinxAtStartPar
F

\end{description}\end{quote}

\sphinxAtStartPar
If true, ensures that wind speed is calculated for use in snow unloading.
If false, the wind speed for unloading will be zero on timesteps when
10m wind diagnostics are not calculated. This will tend to leave more
snow on the vegetation.
It is a short term logical until the code becomes the new default.

\end{fulllineitems}


\sphinxstepscope


\section{\sphinxstyleliteralintitle{\sphinxupquote{timesteps.nml}}}
\label{\detokenize{namelists/timesteps.nml:timesteps-nml}}\label{\detokenize{namelists/timesteps.nml::doc}}
\sphinxAtStartPar
This file sets the start and end time of the run. It can also be used to specify an optional spin\sphinxhyphen{}up procedure. It contains two namelists called {\hyperref[\detokenize{namelists/timesteps.nml:namelist-JULES_TIME}]{\sphinxcrossref{\sphinxcode{\sphinxupquote{JULES\_TIME}}}}} and {\hyperref[\detokenize{namelists/timesteps.nml:namelist-JULES_SPINUP}]{\sphinxcrossref{\sphinxcode{\sphinxupquote{JULES\_SPINUP}}}}}.

\begin{sphinxadmonition}{warning}{Warning:}
\sphinxAtStartPar
It is recommended that all times use not local time, but Coordinated Universal Time (UTC; known in some countries as Greenwich Mean Time GMT). The correct specification of the time is essential if the options causing the surface albedo to depend on the solar zenith angle are set. If the data are provided in local solar time, {\hyperref[\detokenize{namelists/timesteps.nml:JULES_TIME::l_local_solar_time}]{\sphinxcrossref{\sphinxcode{\sphinxupquote{l\_local\_solar\_time}}}}} should be set to TRUE to assume local solar time throughout the model.
\end{sphinxadmonition}


\subsection{\sphinxstyleliteralintitle{\sphinxupquote{JULES\_TIME}} namelist members}
\label{\detokenize{namelists/timesteps.nml:namelist-JULES_TIME}}\label{\detokenize{namelists/timesteps.nml:jules-time-namelist-members}}\index{JULES\_TIME (namelist)@\spxentry{JULES\_TIME}\spxextra{namelist}|spxpagem}\index{l\_360 (in namelist JULES\_TIME)@\spxentry{l\_360}\spxextra{in namelist JULES\_TIME}|spxpagem}

\begin{fulllineitems}
\phantomsection\label{\detokenize{namelists/timesteps.nml:JULES_TIME::l_360}}
\pysigstartsignatures
\pysigline{\sphinxcode{\sphinxupquote{JULES\_TIME::}}\sphinxbfcode{\sphinxupquote{l\_360}}}
\pysigstopsignatures\begin{quote}\begin{description}
\sphinxlineitem{Type}
\sphinxAtStartPar
logical

\sphinxlineitem{Default}
\sphinxAtStartPar
F

\end{description}\end{quote}

\sphinxAtStartPar
Switch indicating use of 360 day years.
\begin{description}
\sphinxlineitem{TRUE}
\sphinxAtStartPar
Each year consists of 360 days. This is sometimes used for idealised experiments.

\sphinxlineitem{FALSE}
\sphinxAtStartPar
Each year consists of 365 or 366 days.

\end{description}

\end{fulllineitems}

\index{l\_leap (in namelist JULES\_TIME)@\spxentry{l\_leap}\spxextra{in namelist JULES\_TIME}|spxpagem}

\begin{fulllineitems}
\phantomsection\label{\detokenize{namelists/timesteps.nml:JULES_TIME::l_leap}}
\pysigstartsignatures
\pysigline{\sphinxcode{\sphinxupquote{JULES\_TIME::}}\sphinxbfcode{\sphinxupquote{l\_leap}}}
\pysigstopsignatures\begin{quote}\begin{description}
\sphinxlineitem{Type}
\sphinxAtStartPar
logical

\sphinxlineitem{Default}
\sphinxAtStartPar
T

\end{description}\end{quote}

\sphinxAtStartPar
Switch indicating whether the calendar has leap years. This flag is not used if l\_360=T.
\begin{description}
\sphinxlineitem{TRUE}
\sphinxAtStartPar
Leap years are modelled i.e. each year consists of 365 or 366 days.

\sphinxlineitem{FALSE}
\sphinxAtStartPar
Each year consists of 365 days.

\end{description}

\end{fulllineitems}

\index{l\_local\_solar\_time (in namelist JULES\_TIME)@\spxentry{l\_local\_solar\_time}\spxextra{in namelist JULES\_TIME}|spxpagem}

\begin{fulllineitems}
\phantomsection\label{\detokenize{namelists/timesteps.nml:JULES_TIME::l_local_solar_time}}
\pysigstartsignatures
\pysigline{\sphinxcode{\sphinxupquote{JULES\_TIME::}}\sphinxbfcode{\sphinxupquote{l\_local\_solar\_time}}}
\pysigstopsignatures\begin{quote}\begin{description}
\sphinxlineitem{Type}
\sphinxAtStartPar
logical

\sphinxlineitem{Default}
\sphinxAtStartPar
F

\end{description}\end{quote}

\sphinxAtStartPar
Switch indicating whether the time\sphinxhyphen{}stamping of the driving data and throughout the run is to be interpreted as local solar time, not UTC.
\begin{description}
\sphinxlineitem{TRUE}
\sphinxAtStartPar
The driving data and all times within the model are interpreted as being in local solar time, irrespective of any data attributes.

\sphinxlineitem{FALSE}
\sphinxAtStartPar
The time convention applying within the model and the driving data is assumed to be UTC.

\end{description}

\end{fulllineitems}

\index{timestep\_len (in namelist JULES\_TIME)@\spxentry{timestep\_len}\spxextra{in namelist JULES\_TIME}|spxpagem}

\begin{fulllineitems}
\phantomsection\label{\detokenize{namelists/timesteps.nml:JULES_TIME::timestep_len}}
\pysigstartsignatures
\pysigline{\sphinxcode{\sphinxupquote{JULES\_TIME::}}\sphinxbfcode{\sphinxupquote{timestep\_len}}}
\pysigstopsignatures\begin{quote}\begin{description}
\sphinxlineitem{Type}
\sphinxAtStartPar
integer

\sphinxlineitem{Permitted}
\sphinxAtStartPar
\textgreater{}= 1

\sphinxlineitem{Default}
\sphinxAtStartPar
None

\end{description}\end{quote}

\sphinxAtStartPar
Model timestep length in seconds (n.b. ‘special periods’ \sphinxhyphen{}1 (monthly) and \sphinxhyphen{}2 (annual) may not be used).

\sphinxAtStartPar
Typically, 30 or 60 minutes is chosen, depending on the driving data available.

\begin{sphinxadmonition}{warning}{Warning:}
\sphinxAtStartPar
If the timestep is too long, the model becomes numerically unstable.
\end{sphinxadmonition}

\end{fulllineitems}

\index{main\_run\_start (in namelist JULES\_TIME)@\spxentry{main\_run\_start}\spxextra{in namelist JULES\_TIME}|spxpagem}

\begin{fulllineitems}
\phantomsection\label{\detokenize{namelists/timesteps.nml:JULES_TIME::main_run_start}}
\pysigstartsignatures
\pysigline{\sphinxcode{\sphinxupquote{JULES\_TIME::}}\sphinxbfcode{\sphinxupquote{main\_run\_start}}}
\pysigstopsignatures
\end{fulllineitems}

\index{main\_run\_end (in namelist JULES\_TIME)@\spxentry{main\_run\_end}\spxextra{in namelist JULES\_TIME}|spxpagem}

\begin{fulllineitems}
\phantomsection\label{\detokenize{namelists/timesteps.nml:JULES_TIME::main_run_end}}
\pysigstartsignatures
\pysigline{\sphinxcode{\sphinxupquote{JULES\_TIME::}}\sphinxbfcode{\sphinxupquote{main\_run\_end}}}
\pysigstopsignatures\begin{quote}\begin{description}
\sphinxlineitem{Type}
\sphinxAtStartPar
character

\sphinxlineitem{Default}
\sphinxAtStartPar
None

\end{description}\end{quote}

\sphinxAtStartPar
The start and end times for the integration.

\sphinxAtStartPar
Each run of JULES consists of an optional spin\sphinxhyphen{}up period and the ‘main run’ that follows the spin\sphinxhyphen{}up. See below for more about the specification of the spin\sphinxhyphen{}up. These variables specify the start and end times for the ‘main run’.

\sphinxAtStartPar
The times must be given in the format:

\begin{sphinxVerbatim}[commandchars=\\\{\}]
\PYG{l+s+s2}{\PYGZdq{}yyyy\PYGZhy{}mm\PYGZhy{}dd hh:mm:ss\PYGZdq{}}
\end{sphinxVerbatim}

\end{fulllineitems}

\index{print\_step (in namelist JULES\_TIME)@\spxentry{print\_step}\spxextra{in namelist JULES\_TIME}|spxpagem}

\begin{fulllineitems}
\phantomsection\label{\detokenize{namelists/timesteps.nml:JULES_TIME::print_step}}
\pysigstartsignatures
\pysigline{\sphinxcode{\sphinxupquote{JULES\_TIME::}}\sphinxbfcode{\sphinxupquote{print\_step}}}
\pysigstopsignatures\begin{quote}\begin{description}
\sphinxlineitem{Type}
\sphinxAtStartPar
integer

\sphinxlineitem{Permitted}
\sphinxAtStartPar
\textgreater{}= 1

\sphinxlineitem{Default}
\sphinxAtStartPar
1

\end{description}\end{quote}

\sphinxAtStartPar
Number of timesteps between printing timestep information to screen, i.e. if {\hyperref[\detokenize{namelists/timesteps.nml:JULES_TIME::print_step}]{\sphinxcrossref{\sphinxcode{\sphinxupquote{print\_step}}}}} = 48, then the timestep start time will only be printed every 48 timesteps.

\end{fulllineitems}



\subsection{\sphinxstyleliteralintitle{\sphinxupquote{JULES\_SPINUP}} namelist members}
\label{\detokenize{namelists/timesteps.nml:namelist-JULES_SPINUP}}\label{\detokenize{namelists/timesteps.nml:jules-spinup-namelist-members}}\index{JULES\_SPINUP (namelist)@\spxentry{JULES\_SPINUP}\spxextra{namelist}|spxpagem}\index{max\_spinup\_cycles (in namelist JULES\_SPINUP)@\spxentry{max\_spinup\_cycles}\spxextra{in namelist JULES\_SPINUP}|spxpagem}

\begin{fulllineitems}
\phantomsection\label{\detokenize{namelists/timesteps.nml:JULES_SPINUP::max_spinup_cycles}}
\pysigstartsignatures
\pysigline{\sphinxcode{\sphinxupquote{JULES\_SPINUP::}}\sphinxbfcode{\sphinxupquote{max\_spinup\_cycles}}}
\pysigstopsignatures\begin{quote}\begin{description}
\sphinxlineitem{Type}
\sphinxAtStartPar
integer

\sphinxlineitem{Permitted}
\sphinxAtStartPar
\textgreater{}= 0

\sphinxlineitem{Default}
\sphinxAtStartPar
0

\end{description}\end{quote}

\sphinxAtStartPar
The maximum number of times the spin\sphinxhyphen{}up period is to be repeated:
\begin{description}
\sphinxlineitem{0}
\sphinxAtStartPar
No spin\sphinxhyphen{}up.

\sphinxlineitem{\textgreater{} 0}
\sphinxAtStartPar
At least 1 and at most {\hyperref[\detokenize{namelists/timesteps.nml:JULES_SPINUP::max_spinup_cycles}]{\sphinxcrossref{\sphinxcode{\sphinxupquote{max\_spinup\_cycles}}}}} repetitions of spin\sphinxhyphen{}up are used.

\end{description}

\sphinxAtStartPar
After each repetition, the model tests whether the selected variables have changed by more than a specified amount over the last repetition (see {\hyperref[\detokenize{namelists/timesteps.nml:JULES_SPINUP::tolerance}]{\sphinxcrossref{\sphinxcode{\sphinxupquote{tolerance}}}}} below).

\sphinxAtStartPar
If the change is less than this amount, the model is considered to have spun up and the model moves on to the main run.

\end{fulllineitems}

\index{spinup\_start (in namelist JULES\_SPINUP)@\spxentry{spinup\_start}\spxextra{in namelist JULES\_SPINUP}|spxpagem}

\begin{fulllineitems}
\phantomsection\label{\detokenize{namelists/timesteps.nml:JULES_SPINUP::spinup_start}}
\pysigstartsignatures
\pysigline{\sphinxcode{\sphinxupquote{JULES\_SPINUP::}}\sphinxbfcode{\sphinxupquote{spinup\_start}}}
\pysigstopsignatures
\end{fulllineitems}

\index{spinup\_end (in namelist JULES\_SPINUP)@\spxentry{spinup\_end}\spxextra{in namelist JULES\_SPINUP}|spxpagem}

\begin{fulllineitems}
\phantomsection\label{\detokenize{namelists/timesteps.nml:JULES_SPINUP::spinup_end}}
\pysigstartsignatures
\pysigline{\sphinxcode{\sphinxupquote{JULES\_SPINUP::}}\sphinxbfcode{\sphinxupquote{spinup\_end}}}
\pysigstopsignatures\begin{quote}\begin{description}
\sphinxlineitem{Type}
\sphinxAtStartPar
character

\sphinxlineitem{Default}
\sphinxAtStartPar
None

\end{description}\end{quote}

\sphinxAtStartPar
Only used if {\hyperref[\detokenize{namelists/timesteps.nml:JULES_SPINUP::max_spinup_cycles}]{\sphinxcrossref{\sphinxcode{\sphinxupquote{max\_spinup\_cycles}}}}} \textgreater{} 0.

\sphinxAtStartPar
The start and end times for each cycle of spin\sphinxhyphen{}up.

\sphinxAtStartPar
The times must be given in the format:

\begin{sphinxVerbatim}[commandchars=\\\{\}]
\PYG{l+s+s2}{\PYGZdq{}yyyy\PYGZhy{}mm\PYGZhy{}dd hh:mm:ss\PYGZdq{}}
\end{sphinxVerbatim}

\end{fulllineitems}

\index{terminate\_on\_spinup\_fail (in namelist JULES\_SPINUP)@\spxentry{terminate\_on\_spinup\_fail}\spxextra{in namelist JULES\_SPINUP}|spxpagem}

\begin{fulllineitems}
\phantomsection\label{\detokenize{namelists/timesteps.nml:JULES_SPINUP::terminate_on_spinup_fail}}
\pysigstartsignatures
\pysigline{\sphinxcode{\sphinxupquote{JULES\_SPINUP::}}\sphinxbfcode{\sphinxupquote{terminate\_on\_spinup\_fail}}}
\pysigstopsignatures\begin{quote}\begin{description}
\sphinxlineitem{Type}
\sphinxAtStartPar
logical

\sphinxlineitem{Default}
\sphinxAtStartPar
F

\end{description}\end{quote}

\sphinxAtStartPar
Only used if {\hyperref[\detokenize{namelists/timesteps.nml:JULES_SPINUP::max_spinup_cycles}]{\sphinxcrossref{\sphinxcode{\sphinxupquote{max\_spinup\_cycles}}}}} \textgreater{} 0.

\sphinxAtStartPar
Switch controlling behaviour if the model does not pass the spin\sphinxhyphen{}up test after {\hyperref[\detokenize{namelists/timesteps.nml:JULES_SPINUP::max_spinup_cycles}]{\sphinxcrossref{\sphinxcode{\sphinxupquote{max\_spinup\_cycles}}}}} of spin\sphinxhyphen{}up.
\begin{description}
\sphinxlineitem{TRUE}
\sphinxAtStartPar
End the run if model has not spun up.

\sphinxlineitem{FALSE}
\sphinxAtStartPar
Continue the run regardless.

\end{description}

\end{fulllineitems}


\begin{sphinxadmonition}{note}{Variables used to specify spin\sphinxhyphen{}up conditions}
\index{nvars (in namelist JULES\_SPINUP)@\spxentry{nvars}\spxextra{in namelist JULES\_SPINUP}|spxpagem}

\begin{fulllineitems}
\phantomsection\label{\detokenize{namelists/timesteps.nml:JULES_SPINUP::nvars}}
\pysigstartsignatures
\pysigline{\sphinxcode{\sphinxupquote{JULES\_SPINUP::}}\sphinxbfcode{\sphinxupquote{nvars}}}
\pysigstopsignatures\begin{quote}\begin{description}
\sphinxlineitem{Type}
\sphinxAtStartPar
integer

\sphinxlineitem{Permitted}
\sphinxAtStartPar
\textgreater{}= 0

\sphinxlineitem{Default}
\sphinxAtStartPar
0

\end{description}\end{quote}

\sphinxAtStartPar
Only used if {\hyperref[\detokenize{namelists/timesteps.nml:JULES_SPINUP::max_spinup_cycles}]{\sphinxcrossref{\sphinxcode{\sphinxupquote{max\_spinup\_cycles}}}}} \textgreater{} 0.

\sphinxAtStartPar
The number of variables to use to assess if the model has spun up.

\end{fulllineitems}

\index{var (in namelist JULES\_SPINUP)@\spxentry{var}\spxextra{in namelist JULES\_SPINUP}|spxpagem}

\begin{fulllineitems}
\phantomsection\label{\detokenize{namelists/timesteps.nml:JULES_SPINUP::var}}
\pysigstartsignatures
\pysigline{\sphinxcode{\sphinxupquote{JULES\_SPINUP::}}\sphinxbfcode{\sphinxupquote{var}}}
\pysigstopsignatures\begin{quote}\begin{description}
\sphinxlineitem{Type}
\sphinxAtStartPar
character(nvars)

\sphinxlineitem{Default}
\sphinxAtStartPar
None

\end{description}\end{quote}

\sphinxAtStartPar
Only used if {\hyperref[\detokenize{namelists/timesteps.nml:JULES_SPINUP::nvars}]{\sphinxcrossref{\sphinxcode{\sphinxupquote{nvars}}}}} \textgreater{} 0.

\sphinxAtStartPar
List of variables to be used to determine if the model has spun up. Spin\sphinxhyphen{}up can be assessed in terms of soil temperature and soil moisture.

\sphinxAtStartPar
Possible values are:
\begin{description}
\sphinxlineitem{\sphinxcode{\sphinxupquote{c\_soil}}}
\sphinxAtStartPar
Soil carbon in each layer (summed over all pools) (kg m$^{\text{\sphinxhyphen{}2}}$).

\sphinxlineitem{\sphinxcode{\sphinxupquote{c\_veg}}}
\sphinxAtStartPar
Vegetation carbon (summed over all vegetation types) (kg m$^{\text{\sphinxhyphen{}2}}$).

\sphinxlineitem{\sphinxcode{\sphinxupquote{smcl}}}
\sphinxAtStartPar
Moisture content of each soil layer (kg m$^{\text{\sphinxhyphen{}2}}$).

\sphinxlineitem{\sphinxcode{\sphinxupquote{t\_soil}}}
\sphinxAtStartPar
Temperature of each soil layer (K).

\end{description}

\end{fulllineitems}

\index{use\_percent (in namelist JULES\_SPINUP)@\spxentry{use\_percent}\spxextra{in namelist JULES\_SPINUP}|spxpagem}

\begin{fulllineitems}
\phantomsection\label{\detokenize{namelists/timesteps.nml:JULES_SPINUP::use_percent}}
\pysigstartsignatures
\pysigline{\sphinxcode{\sphinxupquote{JULES\_SPINUP::}}\sphinxbfcode{\sphinxupquote{use\_percent}}}
\pysigstopsignatures\begin{quote}\begin{description}
\sphinxlineitem{Type}
\sphinxAtStartPar
logical(nvars)

\sphinxlineitem{Default}
\sphinxAtStartPar
F

\end{description}\end{quote}

\sphinxAtStartPar
Only used if {\hyperref[\detokenize{namelists/timesteps.nml:JULES_SPINUP::nvars}]{\sphinxcrossref{\sphinxcode{\sphinxupquote{nvars}}}}} \textgreater{} 0.

\sphinxAtStartPar
Indicates whether the tolerance for each variable is expressed as a percentage.
\begin{description}
\sphinxlineitem{TRUE}
\sphinxAtStartPar
Tolerance is a percentage.

\sphinxlineitem{FALSE}
\sphinxAtStartPar
Tolerance is an absolute value.

\end{description}

\end{fulllineitems}

\index{tolerance (in namelist JULES\_SPINUP)@\spxentry{tolerance}\spxextra{in namelist JULES\_SPINUP}|spxpagem}

\begin{fulllineitems}
\phantomsection\label{\detokenize{namelists/timesteps.nml:JULES_SPINUP::tolerance}}
\pysigstartsignatures
\pysigline{\sphinxcode{\sphinxupquote{JULES\_SPINUP::}}\sphinxbfcode{\sphinxupquote{tolerance}}}
\pysigstopsignatures\begin{quote}\begin{description}
\sphinxlineitem{Type}
\sphinxAtStartPar
real(nvars)

\sphinxlineitem{Default}
\sphinxAtStartPar
None

\end{description}\end{quote}

\sphinxAtStartPar
Only used if {\hyperref[\detokenize{namelists/timesteps.nml:JULES_SPINUP::nvars}]{\sphinxcrossref{\sphinxcode{\sphinxupquote{nvars}}}}} \textgreater{} 0.

\sphinxAtStartPar
Tolerance for spin\sphinxhyphen{}up test for each variable.

\sphinxAtStartPar
For each spin\sphinxhyphen{}up variable, this is the maximum allowed change over a spin\sphinxhyphen{}up cycle if the variable is to be considered as spun\sphinxhyphen{}up.

\end{fulllineitems}

\end{sphinxadmonition}


\subsection{Note on time conventions}
\label{\detokenize{namelists/timesteps.nml:note-on-time-conventions}}
\sphinxAtStartPar
When specifying start times (e.g. {\hyperref[\detokenize{namelists/timesteps.nml:JULES_TIME::main_run_start}]{\sphinxcrossref{\sphinxcode{\sphinxupquote{main\_run\_start}}}}}, {\hyperref[\detokenize{namelists/timesteps.nml:JULES_SPINUP::spinup_start}]{\sphinxcrossref{\sphinxcode{\sphinxupquote{spinup\_start}}}}}), the time is taken to be the \sphinxstyleemphasis{start of the first timestep}. When specifying end times (e.g. {\hyperref[\detokenize{namelists/timesteps.nml:JULES_TIME::main_run_end}]{\sphinxcrossref{\sphinxcode{\sphinxupquote{main\_run\_end}}}}}, {\hyperref[\detokenize{namelists/timesteps.nml:JULES_SPINUP::spinup_end}]{\sphinxcrossref{\sphinxcode{\sphinxupquote{spinup\_end}}}}}), the time is taken to be the \sphinxstyleemphasis{end of the last timestep}. Also, the time “00:00:00” always refers to midnight at the \sphinxstyleemphasis{start} of the day concerned. Take the following setup:

\begin{sphinxVerbatim}[commandchars=\\\{\}]
\PYG{n+nn}{\PYGZam{}JULES\PYGZus{}TIME}
  \PYG{n+nv}{timestep\PYGZus{}len}   \PYG{o}{=} \PYG{l+m+mi}{3600}\PYG{p}{,}
  \PYG{n+nv}{main\PYGZus{}run\PYGZus{}start} \PYG{o}{=} \PYG{l+s+s2}{\PYGZdq{}1997\PYGZhy{}01\PYGZhy{}01 00:00:00\PYGZdq{}}\PYG{p}{,}
  \PYG{n+nv}{main\PYGZus{}run\PYGZus{}end}   \PYG{o}{=} \PYG{l+s+s2}{\PYGZdq{}1998\PYGZhy{}01\PYGZhy{}01 00:00:00\PYGZdq{}}\PYG{p}{,}

  \PYG{c+c1}{\PYGZsh{} ...}
\PYG{n+nn}{/}
\end{sphinxVerbatim}

\sphinxAtStartPar
With this setup, exactly one whole year of timesteps will be run. The first model timestep begins at 1997\sphinxhyphen{}01\sphinxhyphen{}01 00:00:00, the second at 1997\sphinxhyphen{}01\sphinxhyphen{}01 01:00:00 etc. The final model timestep begins at 1997\sphinxhyphen{}12\sphinxhyphen{}31 23:00:00 and ends at 1998\sphinxhyphen{}01\sphinxhyphen{}01 00:00:00. Note that even though only 1997 is simulated, JULES may nevertheless need to access data file(s) for subsequent timesteps (January 1998 in this example), depending on the interpolation flag settings in {\hyperref[\detokenize{input/temporal-interpolation::doc}]{\sphinxcrossref{\DUrole{doc}{Temporal interpolation}}}}.

\sphinxAtStartPar
For example, if your driving data set extends to the end of 1997 only, then you may need to either stop the simulation 1\sphinxhyphen{}2 data timesteps before the end of 1997 (by modifying {\hyperref[\detokenize{namelists/timesteps.nml:JULES_TIME::main_run_end}]{\sphinxcrossref{\sphinxcode{\sphinxupquote{main\_run\_end}}}}} and {\hyperref[\detokenize{namelists/timesteps.nml:JULES_SPINUP::spinup_end}]{\sphinxcrossref{\sphinxcode{\sphinxupquote{spinup\_end}}}}}; n.b. if these data timesteps are at the end of the year/month then an annual/monthly average will not be calculated) or generate dummy driving data for the start of 1998 (which must be realistic because they will be used for interpolation during the last few model timesteps).

\sphinxAtStartPar
The maximum extent of your driving data should be specified by {\hyperref[\detokenize{namelists/drive.nml:JULES_DRIVE::data_start}]{\sphinxcrossref{\sphinxcode{\sphinxupquote{data\_start}}}}} and {\hyperref[\detokenize{namelists/drive.nml:JULES_DRIVE::data_end}]{\sphinxcrossref{\sphinxcode{\sphinxupquote{data\_end}}}}}. The periods of the main run ({\hyperref[\detokenize{namelists/timesteps.nml:JULES_TIME::main_run_start}]{\sphinxcrossref{\sphinxcode{\sphinxupquote{main\_run\_start}}}}} to {\hyperref[\detokenize{namelists/timesteps.nml:JULES_TIME::main_run_end}]{\sphinxcrossref{\sphinxcode{\sphinxupquote{main\_run\_end}}}}}) and spin\sphinxhyphen{}up ({\hyperref[\detokenize{namelists/timesteps.nml:JULES_SPINUP::spinup_start}]{\sphinxcrossref{\sphinxcode{\sphinxupquote{spinup\_start}}}}} to {\hyperref[\detokenize{namelists/timesteps.nml:JULES_SPINUP::spinup_end}]{\sphinxcrossref{\sphinxcode{\sphinxupquote{spinup\_end}}}}}) must be contained within {\hyperref[\detokenize{namelists/drive.nml:JULES_DRIVE::data_start}]{\sphinxcrossref{\sphinxcode{\sphinxupquote{data\_start}}}}} to {\hyperref[\detokenize{namelists/drive.nml:JULES_DRIVE::data_end}]{\sphinxcrossref{\sphinxcode{\sphinxupquote{data\_end}}}}}.

\sphinxAtStartPar
Note that all times are recommended to be in Coordinated Universal Time (UTC), not local time (see Warning at top of this page). Note also the limitations on timestep mentioned in {\hyperref[\detokenize{input/temporal-interpolation::doc}]{\sphinxcrossref{\DUrole{doc}{Temporal interpolation}}}} and on run length mentioned in {\hyperref[\detokenize{code/known-limitations::doc}]{\sphinxcrossref{\DUrole{doc}{Known limitations of the code}}}}.


\subsection{Note on solar zenith angle}
\label{\detokenize{namelists/timesteps.nml:note-on-solar-zenith-angle}}
\sphinxAtStartPar
When the characteristics of the surface relevant to solar radiation are represented in a simple manner, the cosine of the solar zenith angle itself is not required and all that is needed is the downward shortwave flux provided by the forcing data. In such cases it is sufficient to set {\hyperref[\detokenize{namelists/jules_radiation.nml:JULES_RADIATION::l_cosz}]{\sphinxcrossref{\sphinxcode{\sphinxupquote{l\_cosz}}}}} = FALSE, which will set the cosine of the solar zenith angle to a default value of 1.0.

\sphinxAtStartPar
However, more elaborate treatments of the surface albedo and of solar radiative transfer in plant canopies do depend on the actual value of the cosine of the solar zenith angle, as well as the downward flux, so it is more realistic to set {\hyperref[\detokenize{namelists/jules_radiation.nml:JULES_RADIATION::l_cosz}]{\sphinxcrossref{\sphinxcode{\sphinxupquote{l\_cosz}}}}} = TRUE, and, indeed, the results obtained with {\hyperref[\detokenize{namelists/jules_radiation.nml:JULES_RADIATION::l_cosz}]{\sphinxcrossref{\sphinxcode{\sphinxupquote{l\_cosz}}}}} = FALSE may be significantly in error. To calculate the cosine of the zenith angle, the times of the forcing data must be specified accurately.  The consistent use of UTC is strongly recommended. Nevertheless, certain forcing data, widely used within the land\sphinxhyphen{}surface community, are specified in local solar time, even though the metadata in the file may refer to UTC. In such cases {\hyperref[\detokenize{namelists/timesteps.nml:JULES_TIME::l_local_solar_time}]{\sphinxcrossref{\sphinxcode{\sphinxupquote{l\_local\_solar\_time}}}}} should be set to TRUE when {\hyperref[\detokenize{namelists/jules_radiation.nml:JULES_RADIATION::l_cosz}]{\sphinxcrossref{\sphinxcode{\sphinxupquote{l\_cosz}}}}} = TRUE.


\subsection{Examples}
\label{\detokenize{namelists/timesteps.nml:examples}}

\subsubsection{A run without spin\sphinxhyphen{}up}
\label{\detokenize{namelists/timesteps.nml:a-run-without-spin-up}}
\begin{sphinxVerbatim}[commandchars=\\\{\}]
\PYG{n+nn}{\PYGZam{}JULES\PYGZus{}TIME}
  \PYG{n+nv}{timestep\PYGZus{}len}   \PYG{o}{=} \PYG{l+m+mi}{3600}\PYG{p}{,}
  \PYG{n+nv}{main\PYGZus{}run\PYGZus{}start} \PYG{o}{=} \PYG{l+s+s2}{\PYGZdq{}1997\PYGZhy{}01\PYGZhy{}01 00:00:00\PYGZdq{}}\PYG{p}{,}
  \PYG{n+nv}{main\PYGZus{}run\PYGZus{}end}   \PYG{o}{=} \PYG{l+s+s2}{\PYGZdq{}1999\PYGZhy{}01\PYGZhy{}01 01:00:00\PYGZdq{}}
\PYG{n+nn}{/}

\PYG{n+nn}{\PYGZam{}JULES\PYGZus{}SPINUP}
  \PYG{n+nv}{max\PYGZus{}spinup\PYGZus{}cycles} \PYG{o}{=} \PYG{l+m+mi}{0}
\PYG{n+nn}{/}
\end{sphinxVerbatim}

\sphinxAtStartPar
This specifies a run with a timestep length of one hour. The run will begin at midnight on 1st January 1997 and end at 01:00 UTC on 1st January 1999. {\hyperref[\detokenize{namelists/timesteps.nml:JULES_SPINUP::max_spinup_cycles}]{\sphinxcrossref{\sphinxcode{\sphinxupquote{max\_spinup\_cycles}}}}} = 0 means there is no spin\sphinxhyphen{}up.


\subsubsection{A run with spin\sphinxhyphen{}up over a period that immediately precedes the main run}
\label{\detokenize{namelists/timesteps.nml:a-run-with-spin-up-over-a-period-that-immediately-precedes-the-main-run}}
\begin{sphinxVerbatim}[commandchars=\\\{\}]
\PYG{n+nn}{\PYGZam{}JULES\PYGZus{}TIME}
  \PYG{n+nv}{timestep\PYGZus{}len}   \PYG{o}{=} \PYG{l+m+mi}{3600}\PYG{p}{,}
  \PYG{n+nv}{main\PYGZus{}run\PYGZus{}start} \PYG{o}{=} \PYG{l+s+s2}{\PYGZdq{}1997\PYGZhy{}01\PYGZhy{}01 00:00:00\PYGZdq{}}\PYG{p}{,}
  \PYG{n+nv}{main\PYGZus{}run\PYGZus{}end}   \PYG{o}{=} \PYG{l+s+s2}{\PYGZdq{}1999\PYGZhy{}01\PYGZhy{}01 01:00:00\PYGZdq{}}
\PYG{n+nn}{/}

\PYG{n+nn}{\PYGZam{}JULES\PYGZus{}SPINUP}
  \PYG{n+nv}{max\PYGZus{}spinup\PYGZus{}cycles} \PYG{o}{=} \PYG{l+m+mi}{5}\PYG{p}{,}
  \PYG{n+nv}{spinup\PYGZus{}start}      \PYG{o}{=} \PYG{l+s+s2}{\PYGZdq{}1996\PYGZhy{}01\PYGZhy{}01 00:00:00\PYGZdq{}}\PYG{p}{,}
  \PYG{n+nv}{spinup\PYGZus{}end}        \PYG{o}{=} \PYG{l+s+s2}{\PYGZdq{}1997\PYGZhy{}01\PYGZhy{}01 00:00:00\PYGZdq{}}

  \PYG{c+c1}{\PYGZsh{} \PYGZlt{}spinup variable specification\PYGZgt{}}
\PYG{n+nn}{/}
\end{sphinxVerbatim}

\sphinxAtStartPar
This specifies a spin\sphinxhyphen{}up period from midnight on 1st January 1996 to midnight on 1st January 1997. This spin\sphinxhyphen{}up will be repeated up to 5 times, before the main run from midnight on 1st January 1997 until 01:00 UTC on 1st January 1999.


\subsubsection{A run with spin\sphinxhyphen{}up over a period that overlaps the main run}
\label{\detokenize{namelists/timesteps.nml:a-run-with-spin-up-over-a-period-that-overlaps-the-main-run}}
\begin{sphinxVerbatim}[commandchars=\\\{\}]
\PYG{n+nn}{\PYGZam{}JULES\PYGZus{}TIME}
  \PYG{n+nv}{timestep\PYGZus{}len}   \PYG{o}{=} \PYG{l+m+mi}{3600}\PYG{p}{,}
  \PYG{n+nv}{main\PYGZus{}run\PYGZus{}start} \PYG{o}{=} \PYG{l+s+s2}{\PYGZdq{}1997\PYGZhy{}01\PYGZhy{}01 00:00:00\PYGZdq{}}\PYG{p}{,}
  \PYG{n+nv}{main\PYGZus{}run\PYGZus{}end}   \PYG{o}{=} \PYG{l+s+s2}{\PYGZdq{}1999\PYGZhy{}01\PYGZhy{}01 01:00:00\PYGZdq{}}
\PYG{n+nn}{/}

\PYG{n+nn}{\PYGZam{}JULES\PYGZus{}SPINUP}
  \PYG{n+nv}{max\PYGZus{}spinup\PYGZus{}cycles} \PYG{o}{=} \PYG{l+m+mi}{5}\PYG{p}{,}
  \PYG{n+nv}{spinup\PYGZus{}start}      \PYG{o}{=} \PYG{l+s+s2}{\PYGZdq{}1997\PYGZhy{}01\PYGZhy{}01 00:00:00\PYGZdq{}}\PYG{p}{,}
  \PYG{n+nv}{spinup\PYGZus{}end}        \PYG{o}{=} \PYG{l+s+s2}{\PYGZdq{}1998\PYGZhy{}01\PYGZhy{}01 00:00:00\PYGZdq{}}

  \PYG{c+c1}{\PYGZsh{} \PYGZlt{}spinup variable specification\PYGZgt{}}
\PYG{n+nn}{/}
\end{sphinxVerbatim}

\sphinxAtStartPar
This specifies a spin\sphinxhyphen{}up period from midnight on 1st January 1997 to midnight on 1st January 1998. This spin\sphinxhyphen{}up will be repeated up to 5 times, before the main run from midnight on 1st January 1997 until 01:00 UTC on 1st January 1999.


\subsubsection{Example of specifying requirements for spin\sphinxhyphen{}up}
\label{\detokenize{namelists/timesteps.nml:example-of-specifying-requirements-for-spin-up}}
\begin{sphinxVerbatim}[commandchars=\\\{\}]
\PYG{n+nn}{\PYGZam{}JULES\PYGZus{}SPINUP}
  \PYG{c+c1}{\PYGZsh{} \PYGZlt{}spinup time specification\PYGZgt{}}

  \PYG{n+nv}{terminate\PYGZus{}on\PYGZus{}spinup\PYGZus{}fail} \PYG{o}{=} \PYG{l+s+ss}{T}\PYG{p}{,}

  \PYG{n+nv}{nvars} \PYG{o}{=} \PYG{l+m+mi}{2}\PYG{p}{,}
  \PYG{n+nv}{var}         \PYG{o}{=} \PYG{l+s+s2}{\PYGZdq{}smcl\PYGZdq{}}  \PYG{l+s+s2}{\PYGZdq{}t\PYGZus{}soil\PYGZdq{}}\PYG{p}{,}
  \PYG{n+nv}{use\PYGZus{}percent} \PYG{o}{=}     \PYG{l+s+ss}{F}         \PYG{l+s+ss}{T} \PYG{p}{,}
  \PYG{n+nv}{tolerance}   \PYG{o}{=}   \PYG{l+m+mf}{1.0}       \PYG{l+m+mf}{0.1}
\PYG{n+nn}{/}
\end{sphinxVerbatim}

\sphinxAtStartPar
With this setup, {\hyperref[\detokenize{namelists/timesteps.nml:JULES_SPINUP::terminate_on_spinup_fail}]{\sphinxcrossref{\sphinxcode{\sphinxupquote{terminate\_on\_spinup\_fail}}}}} = TRUE means that if the spin\sphinxhyphen{}up has not ‘converged’ after {\hyperref[\detokenize{namelists/timesteps.nml:JULES_SPINUP::max_spinup_cycles}]{\sphinxcrossref{\sphinxcode{\sphinxupquote{max\_spinup\_cycles}}}}} cycles, the run will end. Convergence is measured using the moisture content and temperature of each soil layer. At every point and in every layer, soil moisture must change by less than 1 kg m$^{\text{\sphinxhyphen{}2}}$, while soil temperature must change by less than 0.1\%.


\subsection{Notes on spin\sphinxhyphen{}up}
\label{\detokenize{namelists/timesteps.nml:notes-on-spin-up}}
\sphinxAtStartPar
Spin\sphinxhyphen{}up is assessed using the difference between instantaneous values at the end of consecutive cycles of spin\sphinxhyphen{}up. For example, if the spin\sphinxhyphen{}up period is from 2005\sphinxhyphen{}01\sphinxhyphen{}01 00:00:00 to 2006\sphinxhyphen{}01\sphinxhyphen{}01 00:00:00 then every time the model gets to the end of 2005 the spin\sphinxhyphen{}up variables are compared with their value at the end of the previous cycle. The model is considered spun\sphinxhyphen{}up when \sphinxstyleemphasis{all} the spin\sphinxhyphen{}up variables are spun\sphinxhyphen{}up at \sphinxstyleemphasis{all} points. A spin\sphinxhyphen{}up variable is considered spun\sphinxhyphen{}up if, at each point, the absolute value of the change (percentage change if {\hyperref[\detokenize{namelists/timesteps.nml:JULES_SPINUP::use_percent}]{\sphinxcrossref{\sphinxcode{\sphinxupquote{use\_percent}}}}} = TRUE) over the spin\sphinxhyphen{}up cycle is less than or equal to the given tolerance.

\sphinxAtStartPar
At present the analysis of whether the model has spun up or not is limited to aspects of the ‘physical’ state of the system, and does not explicitly consider carbon stores, making it less useful for runs with interactive vegetation (the equilibrium mode of TRIFFID is designed to spin\sphinxhyphen{}up TRIFFID) or prognostic soil carbon.

\sphinxAtStartPar
During the spin\sphinxhyphen{}up phase of a run, JULES provides the correct driving data (for example, meteorological data) as the model time ‘cycles’ round over the spin\sphinxhyphen{}up period. Consider the case of a spin\sphinxhyphen{}up from 2005\sphinxhyphen{}01\sphinxhyphen{}01 00:00:00 to 2006\sphinxhyphen{}01\sphinxhyphen{}01 00:00:00. At or near the end of 31st December 2005 during the spin\sphinxhyphen{}up, the driving data will start to adjust to the values for 1st January 2005. The calculated driving data may vary slightly between the start or end of the first cycle and similar times in later cycles, because of the need to match the data at the end of each cycle to that at the start of the next cycle. When the main run begins after a period of spin\sphinxhyphen{}up, the driving data is reset to the start of the main run \sphinxhyphen{} no effort is made to adjust the data for a smooth transition. Generally this does not cause a problem.

\sphinxAtStartPar
Depending upon the details of the input data and any temporal interpolation, the driving data may vary rapidly at the end of a cycle of spin\sphinxhyphen{}up, causing an extreme response from the model. In most cases the model will adjust, possibly with large heat fluxes over a few hours, but the user should be aware that unusual behaviour near the end/start of a spin\sphinxhyphen{}up cycle may be the result of this adjustment. Consider the case of a spin\sphinxhyphen{}up from 2005\sphinxhyphen{}01\sphinxhyphen{}01 00:00:00 to 2006\sphinxhyphen{}01\sphinxhyphen{}01 00:00:00. At or near the end of 31st December 2005 during the spin\sphinxhyphen{}up, the driving data will start to adjust to the values for 1st January 2005, which could be very different from conditions on 31st December 2005. The length of time over which the driving data adjust depends on the frequency of the data, and the choice of temporal interpolation. For example, with 3\sphinxhyphen{}hourly data that is interpolated onto a one hour timestep, the adjustment will take place over 3 hours. However, hourly data and an hourly timestep will force an instantaneous adjustment at the start of 1st January 2005.

\sphinxAtStartPar
Although {\hyperref[\detokenize{namelists/timesteps.nml:JULES_SPINUP::max_spinup_cycles}]{\sphinxcrossref{\sphinxcode{\sphinxupquote{max\_spinup\_cycles}}}}} specifies the maximum number of spin\sphinxhyphen{}up cycles, some of which might not be used if the model is considered to have spun up earlier, it is possible to specify the exact number of cycles that will be performed. This can be done by demanding an impossible level of convergence by setting {\hyperref[\detokenize{namelists/timesteps.nml:JULES_SPINUP::tolerance}]{\sphinxcrossref{\sphinxcode{\sphinxupquote{tolerance}}}}} \textless{} 0 (remember that {\hyperref[\detokenize{namelists/timesteps.nml:JULES_SPINUP::tolerance}]{\sphinxcrossref{\sphinxcode{\sphinxupquote{tolerance}}}}} is compared with the absolute change over a cycle) and setting {\hyperref[\detokenize{namelists/timesteps.nml:JULES_SPINUP::terminate_on_spinup_fail}]{\sphinxcrossref{\sphinxcode{\sphinxupquote{terminate\_on\_spinup\_fail}}}}} = FALSE so that the integration continues when spin\sphinxhyphen{}up is judged to have failed after {\hyperref[\detokenize{namelists/timesteps.nml:JULES_SPINUP::max_spinup_cycles}]{\sphinxcrossref{\sphinxcode{\sphinxupquote{max\_spinup\_cycles}}}}} cycles.

\sphinxAtStartPar
Although it is expected that a spin\sphinxhyphen{}up phase will be followed by the main run in the same integration, it is possible to do the spin\sphinxhyphen{}up and main run in separate integrations. This can be done by demanding an impossible level of convergence by setting {\hyperref[\detokenize{namelists/timesteps.nml:JULES_SPINUP::tolerance}]{\sphinxcrossref{\sphinxcode{\sphinxupquote{tolerance}}}}} \textless{} 0 and setting {\hyperref[\detokenize{namelists/timesteps.nml:JULES_SPINUP::terminate_on_spinup_fail}]{\sphinxcrossref{\sphinxcode{\sphinxupquote{terminate\_on\_spinup\_fail}}}}} = TRUE so that the integration stops when spin\sphinxhyphen{}up is judged to have failed. The final state of the model, after {\hyperref[\detokenize{namelists/timesteps.nml:JULES_SPINUP::max_spinup_cycles}]{\sphinxcrossref{\sphinxcode{\sphinxupquote{max\_spinup\_cycles}}}}} cycles of spin\sphinxhyphen{}up, will be written to the final dump, and a subsequent simulation can be started from this dump.

\sphinxAtStartPar
A limitation of the current code is that it cannot cope with a spin\sphinxhyphen{}up cycle that is short in comparison to the period of any input data. For example, a spin\sphinxhyphen{}up cycle of 1 day cannot use 10\sphinxhyphen{}day vegetation data. The code will likely run but the evolution of the vegetation data will probably not be what the user intended! However, it is unlikely that a user would want to try such a run.

\sphinxAtStartPar
Occasionally, the model fails to diagnose a spun up state when in fact the integration has reached a quasi\sphinxhyphen{}steady state that is not detected by the procedure of assessing spin\sphinxhyphen{}up through comparison of instantaneous values at the end of consecutive cycles of spin\sphinxhyphen{}up. An example of this is ‘period\sphinxhyphen{}2’ behaviour, where the model state repeats itself over a period of 2 cycles. Such behaviour should be apparent in the model output during spin\sphinxhyphen{}up, and the user can opt to repeat the integration over a given number of spin\sphinxhyphen{}up cycles, and not to wait for a spun\sphinxhyphen{}up state to be diagnosed.

\sphinxstepscope


\section{\sphinxstyleliteralintitle{\sphinxupquote{model\_grid.nml}}}
\label{\detokenize{namelists/model_grid.nml:model-grid-nml}}\label{\detokenize{namelists/model_grid.nml::doc}}
\sphinxAtStartPar
This file sets up the grid configuration for the run. It contains seven namelists \sphinxhyphen{} {\hyperref[\detokenize{namelists/model_grid.nml:namelist-JULES_INPUT_GRID}]{\sphinxcrossref{\sphinxcode{\sphinxupquote{JULES\_INPUT\_GRID}}}}}, {\hyperref[\detokenize{namelists/model_grid.nml:namelist-JULES_LATLON}]{\sphinxcrossref{\sphinxcode{\sphinxupquote{JULES\_LATLON}}}}}, {\hyperref[\detokenize{namelists/model_grid.nml:namelist-JULES_LAND_FRAC}]{\sphinxcrossref{\sphinxcode{\sphinxupquote{JULES\_LAND\_FRAC}}}}}, {\hyperref[\detokenize{namelists/model_grid.nml:namelist-JULES_MODEL_GRID}]{\sphinxcrossref{\sphinxcode{\sphinxupquote{JULES\_MODEL\_GRID}}}}}, {\hyperref[\detokenize{namelists/model_grid.nml:namelist-JULES_NLSIZES}]{\sphinxcrossref{\sphinxcode{\sphinxupquote{JULES\_NLSIZES}}}}}, {\hyperref[\detokenize{namelists/model_grid.nml:namelist-JULES_SURF_HGT}]{\sphinxcrossref{\sphinxcode{\sphinxupquote{JULES\_SURF\_HGT}}}}} and {\hyperref[\detokenize{namelists/model_grid.nml:namelist-JULES_Z_LAND}]{\sphinxcrossref{\sphinxcode{\sphinxupquote{JULES\_Z\_LAND}}}}}

\sphinxAtStartPar
Each run of JULES involves two grids: the input grid and the model grid. The input grid is the grid on which all input data are held. The model grid is the set of points on which the model is run. The model grid is the grid of points that will be processed by JULES, and is a subset of the input grid.

\sphinxAtStartPar
As discussed in {\hyperref[\detokenize{input/principles::doc}]{\sphinxcrossref{\DUrole{doc}{General principles}}}}, the input grid consists of three pieces of information:
\begin{enumerate}
\sphinxsetlistlabels{\arabic}{enumi}{enumii}{}{.}%
\item {} 
\sphinxAtStartPar
Whether the grid is 1D or 2D.

\item {} 
\sphinxAtStartPar
The size of each dimension.

\item {} 
\sphinxAtStartPar
The name of each dimension in the input file(s).

\end{enumerate}

\sphinxAtStartPar
The latitude, longitude and land fraction of each point are then read in on the full input grid as specified by the namelists. A subset of the input grid to use as the model grid can then be specified in various ways described below (e.g. land points only, all points within certain latitude/longitude bounds).

\sphinxAtStartPar
In most cases, the model grid will be represented internally as a vector of points, even when the input grid is 2D. Numerically, this makes no difference. The only time that the model grid will be 2D is when the input grid is 2D, {\hyperref[\detokenize{namelists/model_grid.nml:JULES_MODEL_GRID::force_1d_grid}]{\sphinxcrossref{\sphinxcode{\sphinxupquote{force\_1d\_grid}}}}} = F and the model grid is a contiguous rectangular subsection of the input grid.


\subsection{\sphinxstyleliteralintitle{\sphinxupquote{JULES\_INPUT\_GRID}} namelist members}
\label{\detokenize{namelists/model_grid.nml:namelist-JULES_INPUT_GRID}}\label{\detokenize{namelists/model_grid.nml:jules-input-grid-namelist-members}}\index{JULES\_INPUT\_GRID (namelist)@\spxentry{JULES\_INPUT\_GRID}\spxextra{namelist}|spxpagem}
\begin{sphinxadmonition}{warning}{Warning:}
\sphinxAtStartPar
The dimension names specified in this namelist will be used for all input files.
\end{sphinxadmonition}
\index{grid\_is\_1d (in namelist JULES\_INPUT\_GRID)@\spxentry{grid\_is\_1d}\spxextra{in namelist JULES\_INPUT\_GRID}|spxpagem}

\begin{fulllineitems}
\phantomsection\label{\detokenize{namelists/model_grid.nml:JULES_INPUT_GRID::grid_is_1d}}
\pysigstartsignatures
\pysigline{\sphinxcode{\sphinxupquote{JULES\_INPUT\_GRID::}}\sphinxbfcode{\sphinxupquote{grid\_is\_1d}}}
\pysigstopsignatures\begin{quote}\begin{description}
\sphinxlineitem{Type}
\sphinxAtStartPar
logical

\sphinxlineitem{Default}
\sphinxAtStartPar
F

\end{description}\end{quote}

\sphinxAtStartPar
Indicates if the input grid is 1D or 2D.
\begin{description}
\sphinxlineitem{TRUE}
\sphinxAtStartPar
Variables have one grid dimension in the input file(s) \sphinxhyphen{} a points dimensions (e.g. a vector of land points with grid dimension “land”).

\sphinxlineitem{FALSE}
\sphinxAtStartPar
Variables have two grid dimensions in the input file(s) \sphinxhyphen{} an x and a y dimension.

\end{description}

\end{fulllineitems}


\begin{sphinxadmonition}{note}{Only used when \sphinxstyleliteralintitle{\sphinxupquote{grid\_is\_1d}} = TRUE}
\index{grid\_dim\_name (in namelist JULES\_INPUT\_GRID)@\spxentry{grid\_dim\_name}\spxextra{in namelist JULES\_INPUT\_GRID}|spxpagem}

\begin{fulllineitems}
\phantomsection\label{\detokenize{namelists/model_grid.nml:JULES_INPUT_GRID::grid_dim_name}}
\pysigstartsignatures
\pysigline{\sphinxcode{\sphinxupquote{JULES\_INPUT\_GRID::}}\sphinxbfcode{\sphinxupquote{grid\_dim\_name}}}
\pysigstopsignatures\begin{quote}\begin{description}
\sphinxlineitem{Type}
\sphinxAtStartPar
character

\sphinxlineitem{Default}
\sphinxAtStartPar
“land”

\end{description}\end{quote}

\sphinxAtStartPar
The name of the single grid dimension.

\begin{sphinxadmonition}{note}{Note:}
\sphinxAtStartPar
For ASCII files, this can be anything. For NetCDF files, it should the name of the dimension in input file(s).
\end{sphinxadmonition}

\end{fulllineitems}

\index{npoints (in namelist JULES\_INPUT\_GRID)@\spxentry{npoints}\spxextra{in namelist JULES\_INPUT\_GRID}|spxpagem}

\begin{fulllineitems}
\phantomsection\label{\detokenize{namelists/model_grid.nml:JULES_INPUT_GRID::npoints}}
\pysigstartsignatures
\pysigline{\sphinxcode{\sphinxupquote{JULES\_INPUT\_GRID::}}\sphinxbfcode{\sphinxupquote{npoints}}}
\pysigstopsignatures\begin{quote}\begin{description}
\sphinxlineitem{Type}
\sphinxAtStartPar
integer

\sphinxlineitem{Permitted}
\sphinxAtStartPar
\textgreater{}= 1

\sphinxlineitem{Default}
\sphinxAtStartPar
0

\end{description}\end{quote}

\sphinxAtStartPar
The size of the single grid dimension.

\end{fulllineitems}

\end{sphinxadmonition}

\begin{sphinxadmonition}{note}{Only used when \sphinxstyleliteralintitle{\sphinxupquote{grid\_is\_1d}} = FALSE}
\index{x\_dim\_name (in namelist JULES\_INPUT\_GRID)@\spxentry{x\_dim\_name}\spxextra{in namelist JULES\_INPUT\_GRID}|spxpagem}

\begin{fulllineitems}
\phantomsection\label{\detokenize{namelists/model_grid.nml:JULES_INPUT_GRID::x_dim_name}}
\pysigstartsignatures
\pysigline{\sphinxcode{\sphinxupquote{JULES\_INPUT\_GRID::}}\sphinxbfcode{\sphinxupquote{x\_dim\_name}}}
\pysigstopsignatures\begin{quote}\begin{description}
\sphinxlineitem{Type}
\sphinxAtStartPar
character

\sphinxlineitem{Default}
\sphinxAtStartPar
“x”

\end{description}\end{quote}

\sphinxAtStartPar
The name of the x dimension (it may, but does not have to, coincide with {\hyperref[\detokenize{namelists/ancillaries.nml:JULES_RIVERS_PROPS::x_dim_name}]{\sphinxcrossref{\sphinxcode{\sphinxupquote{x\_dim\_name}}}}}).

\begin{sphinxadmonition}{note}{Note:}
\sphinxAtStartPar
For ASCII files, this can be anything. For NetCDF files, it should be the name of the dimension in the input file(s).
\end{sphinxadmonition}

\end{fulllineitems}

\index{y\_dim\_name (in namelist JULES\_INPUT\_GRID)@\spxentry{y\_dim\_name}\spxextra{in namelist JULES\_INPUT\_GRID}|spxpagem}

\begin{fulllineitems}
\phantomsection\label{\detokenize{namelists/model_grid.nml:JULES_INPUT_GRID::y_dim_name}}
\pysigstartsignatures
\pysigline{\sphinxcode{\sphinxupquote{JULES\_INPUT\_GRID::}}\sphinxbfcode{\sphinxupquote{y\_dim\_name}}}
\pysigstopsignatures\begin{quote}\begin{description}
\sphinxlineitem{Type}
\sphinxAtStartPar
character

\sphinxlineitem{Default}
\sphinxAtStartPar
“y”

\end{description}\end{quote}

\sphinxAtStartPar
The name of the y dimension (it may, but does not have to, coincide with {\hyperref[\detokenize{namelists/ancillaries.nml:JULES_RIVERS_PROPS::y_dim_name}]{\sphinxcrossref{\sphinxcode{\sphinxupquote{y\_dim\_name}}}}}).

\begin{sphinxadmonition}{note}{Note:}
\sphinxAtStartPar
For ASCII files, this can be anything. For NetCDF files, it should be the name of the dimension in the input file(s).
\end{sphinxadmonition}

\end{fulllineitems}

\index{nx (in namelist JULES\_INPUT\_GRID)@\spxentry{nx}\spxextra{in namelist JULES\_INPUT\_GRID}|spxpagem}

\begin{fulllineitems}
\phantomsection\label{\detokenize{namelists/model_grid.nml:JULES_INPUT_GRID::nx}}
\pysigstartsignatures
\pysigline{\sphinxcode{\sphinxupquote{JULES\_INPUT\_GRID::}}\sphinxbfcode{\sphinxupquote{nx}}}
\pysigstopsignatures\begin{quote}\begin{description}
\sphinxlineitem{Type}
\sphinxAtStartPar
integer

\sphinxlineitem{Permitted}
\sphinxAtStartPar
\textgreater{}= 1

\sphinxlineitem{Default}
\sphinxAtStartPar
0

\end{description}\end{quote}

\sphinxAtStartPar
The size of the x dimension.

\end{fulllineitems}

\index{ny (in namelist JULES\_INPUT\_GRID)@\spxentry{ny}\spxextra{in namelist JULES\_INPUT\_GRID}|spxpagem}

\begin{fulllineitems}
\phantomsection\label{\detokenize{namelists/model_grid.nml:JULES_INPUT_GRID::ny}}
\pysigstartsignatures
\pysigline{\sphinxcode{\sphinxupquote{JULES\_INPUT\_GRID::}}\sphinxbfcode{\sphinxupquote{ny}}}
\pysigstopsignatures\begin{quote}\begin{description}
\sphinxlineitem{Type}
\sphinxAtStartPar
integer

\sphinxlineitem{Permitted}
\sphinxAtStartPar
\textgreater{}= 1

\sphinxlineitem{Default}
\sphinxAtStartPar
0

\end{description}\end{quote}

\sphinxAtStartPar
The size of the y dimension.

\end{fulllineitems}

\end{sphinxadmonition}
\index{time\_dim\_name (in namelist JULES\_INPUT\_GRID)@\spxentry{time\_dim\_name}\spxextra{in namelist JULES\_INPUT\_GRID}|spxpagem}

\begin{fulllineitems}
\phantomsection\label{\detokenize{namelists/model_grid.nml:JULES_INPUT_GRID::time_dim_name}}
\pysigstartsignatures
\pysigline{\sphinxcode{\sphinxupquote{JULES\_INPUT\_GRID::}}\sphinxbfcode{\sphinxupquote{time\_dim\_name}}}
\pysigstopsignatures\begin{quote}\begin{description}
\sphinxlineitem{Type}
\sphinxAtStartPar
character

\sphinxlineitem{Default}
\sphinxAtStartPar
“time”

\end{description}\end{quote}

\sphinxAtStartPar
The name of the time dimension in any input files containing time varying data.

\begin{sphinxadmonition}{note}{Note:}
\sphinxAtStartPar
For ASCII files, this can be anything. For NetCDF files, it should the name of the dimension in input file(s).
\end{sphinxadmonition}

\end{fulllineitems}

\index{pft\_dim\_name (in namelist JULES\_INPUT\_GRID)@\spxentry{pft\_dim\_name}\spxextra{in namelist JULES\_INPUT\_GRID}|spxpagem}

\begin{fulllineitems}
\phantomsection\label{\detokenize{namelists/model_grid.nml:JULES_INPUT_GRID::pft_dim_name}}
\pysigstartsignatures
\pysigline{\sphinxcode{\sphinxupquote{JULES\_INPUT\_GRID::}}\sphinxbfcode{\sphinxupquote{pft\_dim\_name}}}
\pysigstopsignatures\begin{quote}\begin{description}
\sphinxlineitem{Type}
\sphinxAtStartPar
character

\sphinxlineitem{Default}
\sphinxAtStartPar
“pft”

\end{description}\end{quote}

\sphinxAtStartPar
The dimension name used when variables have an additional dimension of size {\hyperref[\detokenize{namelists/jules_surface_types.nml:JULES_SURFACE_TYPES::npft}]{\sphinxcrossref{\sphinxcode{\sphinxupquote{npft}}}}}.

\begin{sphinxadmonition}{note}{Note:}
\sphinxAtStartPar
For ASCII files, this can be anything. For NetCDF files, it should the name of the dimension in input file(s).
\end{sphinxadmonition}

\end{fulllineitems}

\index{cpft\_dim\_name (in namelist JULES\_INPUT\_GRID)@\spxentry{cpft\_dim\_name}\spxextra{in namelist JULES\_INPUT\_GRID}|spxpagem}

\begin{fulllineitems}
\phantomsection\label{\detokenize{namelists/model_grid.nml:JULES_INPUT_GRID::cpft_dim_name}}
\pysigstartsignatures
\pysigline{\sphinxcode{\sphinxupquote{JULES\_INPUT\_GRID::}}\sphinxbfcode{\sphinxupquote{cpft\_dim\_name}}}
\pysigstopsignatures\begin{quote}\begin{description}
\sphinxlineitem{Type}
\sphinxAtStartPar
character

\sphinxlineitem{Default}
\sphinxAtStartPar
“cpft”

\end{description}\end{quote}

\sphinxAtStartPar
The dimension name used when variables have an additional dimension of size {\hyperref[\detokenize{namelists/jules_surface_types.nml:JULES_SURFACE_TYPES::ncpft}]{\sphinxcrossref{\sphinxcode{\sphinxupquote{ncpft}}}}}.

\begin{sphinxadmonition}{note}{Note:}
\sphinxAtStartPar
For ASCII files, this can be anything. For NetCDF files, it should the name of the dimension in input file(s).
\end{sphinxadmonition}

\end{fulllineitems}

\index{nvg\_dim\_name (in namelist JULES\_INPUT\_GRID)@\spxentry{nvg\_dim\_name}\spxextra{in namelist JULES\_INPUT\_GRID}|spxpagem}

\begin{fulllineitems}
\phantomsection\label{\detokenize{namelists/model_grid.nml:JULES_INPUT_GRID::nvg_dim_name}}
\pysigstartsignatures
\pysigline{\sphinxcode{\sphinxupquote{JULES\_INPUT\_GRID::}}\sphinxbfcode{\sphinxupquote{nvg\_dim\_name}}}
\pysigstopsignatures\begin{quote}\begin{description}
\sphinxlineitem{Type}
\sphinxAtStartPar
character

\sphinxlineitem{Default}
\sphinxAtStartPar
“nvg”

\end{description}\end{quote}

\sphinxAtStartPar
The dimension name used when variables have an additional dimension of size  {\hyperref[\detokenize{namelists/jules_surface_types.nml:JULES_SURFACE_TYPES::nnvg}]{\sphinxcrossref{\sphinxcode{\sphinxupquote{nnvg}}}}}.

\begin{sphinxadmonition}{note}{Note:}
\sphinxAtStartPar
For ASCII files, this can be anything. For NetCDF files, it should the name of the dimension in input file(s).
\end{sphinxadmonition}

\end{fulllineitems}

\index{type\_dim\_name (in namelist JULES\_INPUT\_GRID)@\spxentry{type\_dim\_name}\spxextra{in namelist JULES\_INPUT\_GRID}|spxpagem}

\begin{fulllineitems}
\phantomsection\label{\detokenize{namelists/model_grid.nml:JULES_INPUT_GRID::type_dim_name}}
\pysigstartsignatures
\pysigline{\sphinxcode{\sphinxupquote{JULES\_INPUT\_GRID::}}\sphinxbfcode{\sphinxupquote{type\_dim\_name}}}
\pysigstopsignatures\begin{quote}\begin{description}
\sphinxlineitem{Type}
\sphinxAtStartPar
character

\sphinxlineitem{Default}
\sphinxAtStartPar
“type”

\end{description}\end{quote}

\sphinxAtStartPar
The dimension name used when variables have an additional dimension of size  \sphinxcode{\sphinxupquote{ntype}}.

\begin{sphinxadmonition}{note}{Note:}
\sphinxAtStartPar
For ASCII files, this can be anything. For NetCDF files, it should the name of the dimension in input file(s).
\end{sphinxadmonition}

\end{fulllineitems}

\index{tile\_dim\_name (in namelist JULES\_INPUT\_GRID)@\spxentry{tile\_dim\_name}\spxextra{in namelist JULES\_INPUT\_GRID}|spxpagem}

\begin{fulllineitems}
\phantomsection\label{\detokenize{namelists/model_grid.nml:JULES_INPUT_GRID::tile_dim_name}}
\pysigstartsignatures
\pysigline{\sphinxcode{\sphinxupquote{JULES\_INPUT\_GRID::}}\sphinxbfcode{\sphinxupquote{tile\_dim\_name}}}
\pysigstopsignatures\begin{quote}\begin{description}
\sphinxlineitem{Type}
\sphinxAtStartPar
character

\sphinxlineitem{Default}
\sphinxAtStartPar
“tile”

\end{description}\end{quote}

\sphinxAtStartPar
The dimension name used when variables have an additional dimension of size  \sphinxcode{\sphinxupquote{nsurft}}.

\begin{sphinxadmonition}{note}{Note:}
\sphinxAtStartPar
For ASCII files, this can be anything. For NetCDF files, it should the name of the dimension in input file(s).
\end{sphinxadmonition}

\end{fulllineitems}

\index{soil\_dim\_name (in namelist JULES\_INPUT\_GRID)@\spxentry{soil\_dim\_name}\spxextra{in namelist JULES\_INPUT\_GRID}|spxpagem}

\begin{fulllineitems}
\phantomsection\label{\detokenize{namelists/model_grid.nml:JULES_INPUT_GRID::soil_dim_name}}
\pysigstartsignatures
\pysigline{\sphinxcode{\sphinxupquote{JULES\_INPUT\_GRID::}}\sphinxbfcode{\sphinxupquote{soil\_dim\_name}}}
\pysigstopsignatures\begin{quote}\begin{description}
\sphinxlineitem{Type}
\sphinxAtStartPar
character

\sphinxlineitem{Default}
\sphinxAtStartPar
“soil”

\end{description}\end{quote}

\sphinxAtStartPar
The dimension name used when variables have an additional dimension of size  {\hyperref[\detokenize{namelists/jules_soil.nml:JULES_SOIL::sm_levels}]{\sphinxcrossref{\sphinxcode{\sphinxupquote{sm\_levels}}}}}.

\begin{sphinxadmonition}{note}{Note:}
\sphinxAtStartPar
For ASCII files, this can be anything. For NetCDF files, it should the name of the dimension in input file(s).
\end{sphinxadmonition}

\end{fulllineitems}

\index{snow\_dim\_name (in namelist JULES\_INPUT\_GRID)@\spxentry{snow\_dim\_name}\spxextra{in namelist JULES\_INPUT\_GRID}|spxpagem}

\begin{fulllineitems}
\phantomsection\label{\detokenize{namelists/model_grid.nml:JULES_INPUT_GRID::snow_dim_name}}
\pysigstartsignatures
\pysigline{\sphinxcode{\sphinxupquote{JULES\_INPUT\_GRID::}}\sphinxbfcode{\sphinxupquote{snow\_dim\_name}}}
\pysigstopsignatures\begin{quote}\begin{description}
\sphinxlineitem{Type}
\sphinxAtStartPar
character

\sphinxlineitem{Default}
\sphinxAtStartPar
“snow”

\end{description}\end{quote}

\sphinxAtStartPar
The dimension name used when variables have an additional dimension of size  {\hyperref[\detokenize{namelists/jules_snow.nml:JULES_SNOW::nsmax}]{\sphinxcrossref{\sphinxcode{\sphinxupquote{nsmax}}}}}.

\begin{sphinxadmonition}{note}{Note:}
\sphinxAtStartPar
For ASCII files, this can be anything. For NetCDF files, it should the name of the dimension in input file(s).
\end{sphinxadmonition}

\end{fulllineitems}

\index{sclayer\_dim\_name (in namelist JULES\_INPUT\_GRID)@\spxentry{sclayer\_dim\_name}\spxextra{in namelist JULES\_INPUT\_GRID}|spxpagem}

\begin{fulllineitems}
\phantomsection\label{\detokenize{namelists/model_grid.nml:JULES_INPUT_GRID::sclayer_dim_name}}
\pysigstartsignatures
\pysigline{\sphinxcode{\sphinxupquote{JULES\_INPUT\_GRID::}}\sphinxbfcode{\sphinxupquote{sclayer\_dim\_name}}}
\pysigstopsignatures\begin{quote}\begin{description}
\sphinxlineitem{Type}
\sphinxAtStartPar
character

\sphinxlineitem{Default}
\sphinxAtStartPar
“sclayer”

\end{description}\end{quote}

\sphinxAtStartPar
The dimension name used for the soil biogeochemistry when layered soil is used i.e. {\hyperref[\detokenize{namelists/jules_soil_biogeochem.nml:JULES_SOIL_BIOGEOCHEM::l_layeredc}]{\sphinxcrossref{\sphinxcode{\sphinxupquote{l\_layeredc}}}}} = TRUE. When {\hyperref[\detokenize{namelists/jules_soil_biogeochem.nml:JULES_SOIL_BIOGEOCHEM::l_layeredc}]{\sphinxcrossref{\sphinxcode{\sphinxupquote{l\_layeredc}}}}} = TRUE the soil biogeochemistry has the same number of layers as the soil hydrology ({\hyperref[\detokenize{namelists/jules_soil.nml:JULES_SOIL::sm_levels}]{\sphinxcrossref{\sphinxcode{\sphinxupquote{sm\_levels}}}}}). When {\hyperref[\detokenize{namelists/jules_soil_biogeochem.nml:JULES_SOIL_BIOGEOCHEM::l_layeredc}]{\sphinxcrossref{\sphinxcode{\sphinxupquote{l\_layeredc}}}}} = FALSE the soil biogeochemistry represents a single bulk layer. Despite the similar name, this parameter is unrelated to {\hyperref[\detokenize{namelists/jules_soil_ecosse.nml:JULES_SOIL_ECOSSE::dim_cslayer}]{\sphinxcrossref{\sphinxcode{\sphinxupquote{dim\_cslayer}}}}}.

\begin{sphinxadmonition}{note}{Note:}
\sphinxAtStartPar
For ASCII files, this can be anything. For NetCDF files, it should the name of the dimension in input file(s).
\end{sphinxadmonition}

\end{fulllineitems}

\index{scpool\_dim\_name (in namelist JULES\_INPUT\_GRID)@\spxentry{scpool\_dim\_name}\spxextra{in namelist JULES\_INPUT\_GRID}|spxpagem}

\begin{fulllineitems}
\phantomsection\label{\detokenize{namelists/model_grid.nml:JULES_INPUT_GRID::scpool_dim_name}}
\pysigstartsignatures
\pysigline{\sphinxcode{\sphinxupquote{JULES\_INPUT\_GRID::}}\sphinxbfcode{\sphinxupquote{scpool\_dim\_name}}}
\pysigstopsignatures\begin{quote}\begin{description}
\sphinxlineitem{Type}
\sphinxAtStartPar
character

\sphinxlineitem{Default}
\sphinxAtStartPar
“scpool”

\end{description}\end{quote}

\sphinxAtStartPar
The dimension name used when variables have an additional dimension of size  \sphinxcode{\sphinxupquote{dim\_cs1}}.

\begin{sphinxadmonition}{note}{Note:}
\sphinxAtStartPar
For ASCII files, this can be anything. For NetCDF files, it should the name of the dimension in input file(s).
\end{sphinxadmonition}

\end{fulllineitems}

\index{bedrock\_dim\_name (in namelist JULES\_INPUT\_GRID)@\spxentry{bedrock\_dim\_name}\spxextra{in namelist JULES\_INPUT\_GRID}|spxpagem}

\begin{fulllineitems}
\phantomsection\label{\detokenize{namelists/model_grid.nml:JULES_INPUT_GRID::bedrock_dim_name}}
\pysigstartsignatures
\pysigline{\sphinxcode{\sphinxupquote{JULES\_INPUT\_GRID::}}\sphinxbfcode{\sphinxupquote{bedrock\_dim\_name}}}
\pysigstopsignatures\begin{quote}\begin{description}
\sphinxlineitem{Type}
\sphinxAtStartPar
character

\sphinxlineitem{Default}
\sphinxAtStartPar
“bedrock”

\end{description}\end{quote}

\sphinxAtStartPar
The dimension name used when variables have an additional dimension of size  {\hyperref[\detokenize{namelists/jules_soil.nml:JULES_SOIL::ns_deep}]{\sphinxcrossref{\sphinxcode{\sphinxupquote{ns\_deep}}}}}.

\begin{sphinxadmonition}{note}{Note:}
\sphinxAtStartPar
For ASCII files, this can be anything. For NetCDF files, it should the name of the dimension in input file(s).
\end{sphinxadmonition}

\end{fulllineitems}

\index{tracer\_dim\_name (in namelist JULES\_INPUT\_GRID)@\spxentry{tracer\_dim\_name}\spxextra{in namelist JULES\_INPUT\_GRID}|spxpagem}

\begin{fulllineitems}
\phantomsection\label{\detokenize{namelists/model_grid.nml:JULES_INPUT_GRID::tracer_dim_name}}
\pysigstartsignatures
\pysigline{\sphinxcode{\sphinxupquote{JULES\_INPUT\_GRID::}}\sphinxbfcode{\sphinxupquote{tracer\_dim\_name}}}
\pysigstopsignatures\begin{quote}\begin{description}
\sphinxlineitem{Type}
\sphinxAtStartPar
character

\sphinxlineitem{Default}
\sphinxAtStartPar
“tracer”

\end{description}\end{quote}

\sphinxAtStartPar
The dimension name used when variables have an additional dimension of size {\hyperref[\detokenize{namelists/jules_deposition.nml:JULES_DEPOSITION::ndry_dep_species}]{\sphinxcrossref{\sphinxcode{\sphinxupquote{ndry\_dep\_species}}}}} (e.g. chemical tracers in the atmosphere).

\begin{sphinxadmonition}{note}{Note:}
\sphinxAtStartPar
For ASCII files, this can be anything. For NetCDF files, it should the name of the dimension in input file(s).
\end{sphinxadmonition}

\end{fulllineitems}

\index{bl\_level\_dim\_name (in namelist JULES\_INPUT\_GRID)@\spxentry{bl\_level\_dim\_name}\spxextra{in namelist JULES\_INPUT\_GRID}|spxpagem}

\begin{fulllineitems}
\phantomsection\label{\detokenize{namelists/model_grid.nml:JULES_INPUT_GRID::bl_level_dim_name}}
\pysigstartsignatures
\pysigline{\sphinxcode{\sphinxupquote{JULES\_INPUT\_GRID::}}\sphinxbfcode{\sphinxupquote{bl\_level\_dim\_name}}}
\pysigstopsignatures\begin{quote}\begin{description}
\sphinxlineitem{Type}
\sphinxAtStartPar
character

\sphinxlineitem{Default}
\sphinxAtStartPar
“bllevel”

\end{description}\end{quote}

\sphinxAtStartPar
The dimension name used when variables have an additional dimension of size {\hyperref[\detokenize{namelists/model_grid.nml:JULES_NLSIZES::bl_levels}]{\sphinxcrossref{\sphinxcode{\sphinxupquote{bl\_levels}}}}} (e.g. variables on atmospheric boundary layer levels).

\begin{sphinxadmonition}{note}{Note:}
\sphinxAtStartPar
For ASCII files, this can be anything. For NetCDF files, it should the name of the dimension in input file(s).
\end{sphinxadmonition}

\end{fulllineitems}



\subsection{\sphinxstyleliteralintitle{\sphinxupquote{JULES\_LATLON}} namelist members}
\label{\detokenize{namelists/model_grid.nml:namelist-JULES_LATLON}}\label{\detokenize{namelists/model_grid.nml:jules-latlon-namelist-members}}\index{JULES\_LATLON (namelist)@\spxentry{JULES\_LATLON}\spxextra{namelist}|spxpagem}
\begin{sphinxadmonition}{note}{Members used to determine how gridpoint location variables are set}
\index{read\_from\_dump (in namelist JULES\_LATLON)@\spxentry{read\_from\_dump}\spxextra{in namelist JULES\_LATLON}|spxpagem}

\begin{fulllineitems}
\phantomsection\label{\detokenize{namelists/model_grid.nml:JULES_LATLON::read_from_dump}}
\pysigstartsignatures
\pysigline{\sphinxcode{\sphinxupquote{JULES\_LATLON::}}\sphinxbfcode{\sphinxupquote{read\_from\_dump}}}
\pysigstopsignatures\begin{quote}\begin{description}
\sphinxlineitem{Type}
\sphinxAtStartPar
logical

\sphinxlineitem{Default}
\sphinxAtStartPar
F

\end{description}\end{quote}
\begin{description}
\sphinxlineitem{TRUE}
\sphinxAtStartPar
Populate variables associated with this namelist from the dump file. All other namelist members are ignored.

\sphinxlineitem{FALSE}
\sphinxAtStartPar
Use the other namelist members to determine how to populate variables.

\end{description}

\end{fulllineitems}

\index{l\_coord\_latlon (in namelist JULES\_LATLON)@\spxentry{l\_coord\_latlon}\spxextra{in namelist JULES\_LATLON}|spxpagem}

\begin{fulllineitems}
\phantomsection\label{\detokenize{namelists/model_grid.nml:JULES_LATLON::l_coord_latlon}}
\pysigstartsignatures
\pysigline{\sphinxcode{\sphinxupquote{JULES\_LATLON::}}\sphinxbfcode{\sphinxupquote{l\_coord\_latlon}}}
\pysigstopsignatures\begin{quote}\begin{description}
\sphinxlineitem{Type}
\sphinxAtStartPar
logical

\sphinxlineitem{Default}
\sphinxAtStartPar
F

\end{description}\end{quote}
\begin{description}
\sphinxlineitem{TRUE}
\sphinxAtStartPar
The coordinate system used for the model grid is latitude and longitude.

\sphinxlineitem{FALSE}
\sphinxAtStartPar
The model grid is defined by projection coordinates other than latitude and longitude (e.g. northings and eastings, or a rotated grid).

\end{description}

\end{fulllineitems}

\index{nvars (in namelist JULES\_LATLON)@\spxentry{nvars}\spxextra{in namelist JULES\_LATLON}|spxpagem}

\begin{fulllineitems}
\phantomsection\label{\detokenize{namelists/model_grid.nml:JULES_LATLON::nvars}}
\pysigstartsignatures
\pysigline{\sphinxcode{\sphinxupquote{JULES\_LATLON::}}\sphinxbfcode{\sphinxupquote{nvars}}}
\pysigstopsignatures\begin{quote}\begin{description}
\sphinxlineitem{Type}
\sphinxAtStartPar
integer

\sphinxlineitem{Permitted}
\sphinxAtStartPar
\textgreater{}= 2

\sphinxlineitem{Default}
\sphinxAtStartPar
0

\end{description}\end{quote}

\sphinxAtStartPar
The number of location variables that will be provided (see {\hyperref[\detokenize{namelists/model_grid.nml:list-of-grid-location-params}]{\sphinxcrossref{\DUrole{std,std-ref}{List of grid location properties}}}}).

\end{fulllineitems}

\index{var (in namelist JULES\_LATLON)@\spxentry{var}\spxextra{in namelist JULES\_LATLON}|spxpagem}

\begin{fulllineitems}
\phantomsection\label{\detokenize{namelists/model_grid.nml:JULES_LATLON::var}}
\pysigstartsignatures
\pysigline{\sphinxcode{\sphinxupquote{JULES\_LATLON::}}\sphinxbfcode{\sphinxupquote{var}}}
\pysigstopsignatures\begin{quote}\begin{description}
\sphinxlineitem{Type}
\sphinxAtStartPar
character(nvars)

\sphinxlineitem{Default}
\sphinxAtStartPar
None

\end{description}\end{quote}

\sphinxAtStartPar
List of location variable names as recognised by JULES (see {\hyperref[\detokenize{namelists/model_grid.nml:list-of-grid-location-params}]{\sphinxcrossref{\DUrole{std,std-ref}{List of grid location properties}}}}). Names are case sensitive.

\begin{sphinxadmonition}{note}{Note:}
\sphinxAtStartPar
For ASCII files, variable names must be in the order they appear in the file.
\end{sphinxadmonition}

\end{fulllineitems}

\index{use\_file (in namelist JULES\_LATLON)@\spxentry{use\_file}\spxextra{in namelist JULES\_LATLON}|spxpagem}

\begin{fulllineitems}
\phantomsection\label{\detokenize{namelists/model_grid.nml:JULES_LATLON::use_file}}
\pysigstartsignatures
\pysigline{\sphinxcode{\sphinxupquote{JULES\_LATLON::}}\sphinxbfcode{\sphinxupquote{use\_file}}}
\pysigstopsignatures\begin{quote}\begin{description}
\sphinxlineitem{Type}
\sphinxAtStartPar
logical(nvars)

\sphinxlineitem{Default}
\sphinxAtStartPar
T

\end{description}\end{quote}

\sphinxAtStartPar
For each JULES variable specified in {\hyperref[\detokenize{namelists/model_grid.nml:JULES_LATLON::var}]{\sphinxcrossref{\sphinxcode{\sphinxupquote{var}}}}}, this indicates if it should be read from the specified file or whether a constant value is to be used.
\begin{description}
\sphinxlineitem{TRUE}
\sphinxAtStartPar
The variable will be read from the file.

\sphinxlineitem{FALSE}
\sphinxAtStartPar
The variable will be set to a constant value everywhere using {\hyperref[\detokenize{namelists/model_grid.nml:JULES_LATLON::const_val}]{\sphinxcrossref{\sphinxcode{\sphinxupquote{const\_val}}}}} below.

\end{description}

\end{fulllineitems}

\index{var\_name (in namelist JULES\_LATLON)@\spxentry{var\_name}\spxextra{in namelist JULES\_LATLON}|spxpagem}

\begin{fulllineitems}
\phantomsection\label{\detokenize{namelists/model_grid.nml:JULES_LATLON::var_name}}
\pysigstartsignatures
\pysigline{\sphinxcode{\sphinxupquote{JULES\_LATLON::}}\sphinxbfcode{\sphinxupquote{var\_name}}}
\pysigstopsignatures\begin{quote}\begin{description}
\sphinxlineitem{Type}
\sphinxAtStartPar
character(nvars)

\sphinxlineitem{Default}
\sphinxAtStartPar
‘’ (empty string)

\end{description}\end{quote}

\sphinxAtStartPar
For each JULES variable specified in {\hyperref[\detokenize{namelists/model_grid.nml:JULES_LATLON::var}]{\sphinxcrossref{\sphinxcode{\sphinxupquote{var}}}}} where {\hyperref[\detokenize{namelists/model_grid.nml:JULES_LATLON::use_file}]{\sphinxcrossref{\sphinxcode{\sphinxupquote{use\_file}}}}} = TRUE, this is the name of the variable in the file containing the data.

\sphinxAtStartPar
If the empty string (the default) is given for any variable, then the corresponding value from {\hyperref[\detokenize{namelists/model_grid.nml:JULES_LATLON::var}]{\sphinxcrossref{\sphinxcode{\sphinxupquote{var}}}}} is used instead.

\sphinxAtStartPar
This is not used for variables where {\hyperref[\detokenize{namelists/model_grid.nml:JULES_LATLON::use_file}]{\sphinxcrossref{\sphinxcode{\sphinxupquote{use\_file}}}}} = FALSE, but a placeholder must still be given in that case.

\begin{sphinxadmonition}{note}{Note:}
\sphinxAtStartPar
For ASCII files, this is not used \sphinxhyphen{} only the order in the file matters, as described above.
\end{sphinxadmonition}

\end{fulllineitems}

\index{tpl\_name (in namelist JULES\_LATLON)@\spxentry{tpl\_name}\spxextra{in namelist JULES\_LATLON}|spxpagem}

\begin{fulllineitems}
\phantomsection\label{\detokenize{namelists/model_grid.nml:JULES_LATLON::tpl_name}}
\pysigstartsignatures
\pysigline{\sphinxcode{\sphinxupquote{JULES\_LATLON::}}\sphinxbfcode{\sphinxupquote{tpl\_name}}}
\pysigstopsignatures\begin{quote}\begin{description}
\sphinxlineitem{Type}
\sphinxAtStartPar
character(nvars)

\sphinxlineitem{Default}
\sphinxAtStartPar
None

\end{description}\end{quote}

\sphinxAtStartPar
For each JULES variable specified in {\hyperref[\detokenize{namelists/model_grid.nml:JULES_LATLON::var}]{\sphinxcrossref{\sphinxcode{\sphinxupquote{var}}}}}, this is the string to substitute into the file name in place of the variable name substitution string.

\sphinxAtStartPar
If the file name does not use variable name templating, this is not used.

\end{fulllineitems}

\index{const\_val (in namelist JULES\_LATLON)@\spxentry{const\_val}\spxextra{in namelist JULES\_LATLON}|spxpagem}

\begin{fulllineitems}
\phantomsection\label{\detokenize{namelists/model_grid.nml:JULES_LATLON::const_val}}
\pysigstartsignatures
\pysigline{\sphinxcode{\sphinxupquote{JULES\_LATLON::}}\sphinxbfcode{\sphinxupquote{const\_val}}}
\pysigstopsignatures\begin{quote}\begin{description}
\sphinxlineitem{Type}
\sphinxAtStartPar
real(nvars)

\sphinxlineitem{Default}
\sphinxAtStartPar
None

\end{description}\end{quote}

\sphinxAtStartPar
For each JULES variable specified in {\hyperref[\detokenize{namelists/model_grid.nml:JULES_LATLON::var}]{\sphinxcrossref{\sphinxcode{\sphinxupquote{var}}}}} where {\hyperref[\detokenize{namelists/model_grid.nml:JULES_LATLON::use_file}]{\sphinxcrossref{\sphinxcode{\sphinxupquote{use\_file}}}}} = FALSE, this is a constant value that the variable will be set to at every point.

\sphinxAtStartPar
This is not used for variables where {\hyperref[\detokenize{namelists/model_grid.nml:JULES_LATLON::use_file}]{\sphinxcrossref{\sphinxcode{\sphinxupquote{use\_file}}}}} = TRUE, but a placeholder must still be given in that case.

\end{fulllineitems}

\index{file (in namelist JULES\_LATLON)@\spxentry{file}\spxextra{in namelist JULES\_LATLON}|spxpagem}

\begin{fulllineitems}
\phantomsection\label{\detokenize{namelists/model_grid.nml:JULES_LATLON::file}}
\pysigstartsignatures
\pysigline{\sphinxcode{\sphinxupquote{JULES\_LATLON::}}\sphinxbfcode{\sphinxupquote{file}}}
\pysigstopsignatures\begin{quote}\begin{description}
\sphinxlineitem{Type}
\sphinxAtStartPar
character

\sphinxlineitem{Default}
\sphinxAtStartPar
None

\end{description}\end{quote}

\sphinxAtStartPar
The file to read ancillary properties from.

\sphinxAtStartPar
If {\hyperref[\detokenize{namelists/model_grid.nml:JULES_LATLON::use_file}]{\sphinxcrossref{\sphinxcode{\sphinxupquote{use\_file}}}}} is FALSE for every variable, this will not be used.

\sphinxAtStartPar
This file name can use {\hyperref[\detokenize{input/file-name-templating::doc}]{\sphinxcrossref{\DUrole{doc}{variable name templating}}}}.

\end{fulllineitems}

\end{sphinxadmonition}


\subsubsection{List of grid location properties}
\label{\detokenize{namelists/model_grid.nml:list-of-grid-location-properties}}\label{\detokenize{namelists/model_grid.nml:list-of-grid-location-params}}
\sphinxAtStartPar
The following table summarises ancillary fields that give the location and related characteristics of each point on the grid, specified from an ancillary file if {\hyperref[\detokenize{namelists/model_grid.nml:JULES_LATLON::use_file}]{\sphinxcrossref{\sphinxcode{\sphinxupquote{use\_file}}}}} = TRUE.


\begin{savenotes}\sphinxattablestart
\centering
\begin{tabulary}{\linewidth}[t]{|p{3cm}|L|}
\hline
\sphinxstyletheadfamily 
\sphinxAtStartPar
Name
&\sphinxstyletheadfamily 
\sphinxAtStartPar
Description
\\
\hline
\sphinxAtStartPar
\sphinxcode{\sphinxupquote{latitude}}
&
\sphinxAtStartPar
Latitude of each point. Always required.
\\
\hline
\sphinxAtStartPar
\sphinxcode{\sphinxupquote{longitude}}
&
\sphinxAtStartPar
Longitude of each point. Always required.
Values in the range \sphinxhyphen{}180 to 360 are allowed.
\\
\hline
\sphinxAtStartPar
\sphinxcode{\sphinxupquote{projection\_x\_coord}}
&
\sphinxAtStartPar
Values of the projection coordinate in the x direction.
This is only required if {\hyperref[\detokenize{namelists/model_grid.nml:JULES_LATLON::l_coord_latlon}]{\sphinxcrossref{\sphinxcode{\sphinxupquote{l\_coord\_latlon}}}}} = FALSE.
Note that these can have any valid unit.
\\
\hline
\sphinxAtStartPar
\sphinxcode{\sphinxupquote{projection\_y\_coord}}
&
\sphinxAtStartPar
Values of the projection coordinate in the y direction.
This is only required if {\hyperref[\detokenize{namelists/model_grid.nml:JULES_LATLON::l_coord_latlon}]{\sphinxcrossref{\sphinxcode{\sphinxupquote{l\_coord\_latlon}}}}} = FALSE.
Note that these can have any valid unit.
\\
\hline
\sphinxAtStartPar
\sphinxcode{\sphinxupquote{grid\_area}}
&
\sphinxAtStartPar
The area of each gridbox (m:sup\textasciigrave{}2\textasciigrave{})
This is only requred if irrigation is being modelled with
{\hyperref[\detokenize{namelists/jules_water_resources.nml:JULES_WATER_RESOURCES::l_water_resources}]{\sphinxcrossref{\sphinxcode{\sphinxupquote{l\_water\_resources}}}}} = TRUE and
{\hyperref[\detokenize{namelists/jules_water_resources.nml:JULES_WATER_RESOURCES::l_water_irrigation}]{\sphinxcrossref{\sphinxcode{\sphinxupquote{l\_water\_irrigation}}}}} = TRUE.
\\
\hline
\end{tabulary}
\par
\sphinxattableend\end{savenotes}

\sphinxAtStartPar
Examples of how to specify the model domain using through this namelist are provided at the end of this section.


\subsection{\sphinxstyleliteralintitle{\sphinxupquote{JULES\_LAND\_FRAC}} namelist members}
\label{\detokenize{namelists/model_grid.nml:namelist-JULES_LAND_FRAC}}\label{\detokenize{namelists/model_grid.nml:jules-land-frac-namelist-members}}\index{JULES\_LAND\_FRAC (namelist)@\spxentry{JULES\_LAND\_FRAC}\spxextra{namelist}|spxpagem}
\sphinxAtStartPar
Land fraction is the fraction of each gridbox that is land. Currently, JULES considers any gridbox with land fraction \textgreater{} 0 to be 100\% land, and all others to be 100\% sea (or sea\sphinxhyphen{}ice). Land fraction data can be used to select only land points from the full input grid (see below).

\begin{sphinxadmonition}{warning}{Warning:}
\sphinxAtStartPar
When the input grid consists of a single location (1D and {\hyperref[\detokenize{namelists/model_grid.nml:JULES_INPUT_GRID::npoints}]{\sphinxcrossref{\sphinxcode{\sphinxupquote{npoints}}}}} = 1 or 2D and {\hyperref[\detokenize{namelists/model_grid.nml:JULES_INPUT_GRID::nx}]{\sphinxcrossref{\sphinxcode{\sphinxupquote{nx}}}}} = {\hyperref[\detokenize{namelists/model_grid.nml:JULES_INPUT_GRID::ny}]{\sphinxcrossref{\sphinxcode{\sphinxupquote{ny}}}}} = 1), that single location is assumed to be 100\% land.
\end{sphinxadmonition}

\sphinxAtStartPar
For any input grid with more than a single location, the following are used:
\index{file (in namelist JULES\_LAND\_FRAC)@\spxentry{file}\spxextra{in namelist JULES\_LAND\_FRAC}|spxpagem}

\begin{fulllineitems}
\phantomsection\label{\detokenize{namelists/model_grid.nml:JULES_LAND_FRAC::file}}
\pysigstartsignatures
\pysigline{\sphinxcode{\sphinxupquote{JULES\_LAND\_FRAC::}}\sphinxbfcode{\sphinxupquote{file}}}
\pysigstopsignatures\begin{quote}\begin{description}
\sphinxlineitem{Type}
\sphinxAtStartPar
character

\sphinxlineitem{Default}
\sphinxAtStartPar
None

\end{description}\end{quote}

\sphinxAtStartPar
The name of the file to read land fraction data from.

\end{fulllineitems}

\index{land\_frac\_name (in namelist JULES\_LAND\_FRAC)@\spxentry{land\_frac\_name}\spxextra{in namelist JULES\_LAND\_FRAC}|spxpagem}

\begin{fulllineitems}
\phantomsection\label{\detokenize{namelists/model_grid.nml:JULES_LAND_FRAC::land_frac_name}}
\pysigstartsignatures
\pysigline{\sphinxcode{\sphinxupquote{JULES\_LAND\_FRAC::}}\sphinxbfcode{\sphinxupquote{land\_frac\_name}}}
\pysigstopsignatures\begin{quote}\begin{description}
\sphinxlineitem{Type}
\sphinxAtStartPar
character

\sphinxlineitem{Default}
\sphinxAtStartPar
‘land\_fraction’

\end{description}\end{quote}

\sphinxAtStartPar
The name of the variable containing the land fraction data.

\sphinxAtStartPar
In the file, the variable must have no levels dimensions and no time dimension.

\end{fulllineitems}



\subsection{\sphinxstyleliteralintitle{\sphinxupquote{JULES\_MODEL\_GRID}} namelist members}
\label{\detokenize{namelists/model_grid.nml:namelist-JULES_MODEL_GRID}}\label{\detokenize{namelists/model_grid.nml:jules-model-grid-namelist-members}}\index{JULES\_MODEL\_GRID (namelist)@\spxentry{JULES\_MODEL\_GRID}\spxextra{namelist}|spxpagem}
\sphinxAtStartPar
Members of this namelist are used to select the points to be modelled from the input grid. This can be done in various ways (see the {\hyperref[\detokenize{namelists/model_grid.nml:grid-examples}]{\sphinxcrossref{\DUrole{std,std-ref}{Examples of grid setups}}}}).
\index{land\_only (in namelist JULES\_MODEL\_GRID)@\spxentry{land\_only}\spxextra{in namelist JULES\_MODEL\_GRID}|spxpagem}

\begin{fulllineitems}
\phantomsection\label{\detokenize{namelists/model_grid.nml:JULES_MODEL_GRID::land_only}}
\pysigstartsignatures
\pysigline{\sphinxcode{\sphinxupquote{JULES\_MODEL\_GRID::}}\sphinxbfcode{\sphinxupquote{land\_only}}}
\pysigstopsignatures\begin{quote}\begin{description}
\sphinxlineitem{Type}
\sphinxAtStartPar
logical

\sphinxlineitem{Default}
\sphinxAtStartPar
T

\end{description}\end{quote}
\begin{description}
\sphinxlineitem{TRUE}
\sphinxAtStartPar
Model land points only (from the points that are selected with other options).

\sphinxlineitem{FALSE}
\sphinxAtStartPar
Model all selected points.

\end{description}

\sphinxAtStartPar
If {\hyperref[\detokenize{namelists/model_grid.nml:JULES_MODEL_GRID::use_subgrid}]{\sphinxcrossref{\sphinxcode{\sphinxupquote{use\_subgrid}}}}} = FALSE (see below), the land points will be extracted from the full input grid.

\sphinxAtStartPar
If {\hyperref[\detokenize{namelists/model_grid.nml:JULES_MODEL_GRID::use_subgrid}]{\sphinxcrossref{\sphinxcode{\sphinxupquote{use\_subgrid}}}}} = TRUE, then the subgrid extraction takes place first, and the land points will be extracted from the specified subgrid.

\end{fulllineitems}

\index{force\_1d\_grid (in namelist JULES\_MODEL\_GRID)@\spxentry{force\_1d\_grid}\spxextra{in namelist JULES\_MODEL\_GRID}|spxpagem}

\begin{fulllineitems}
\phantomsection\label{\detokenize{namelists/model_grid.nml:JULES_MODEL_GRID::force_1d_grid}}
\pysigstartsignatures
\pysigline{\sphinxcode{\sphinxupquote{JULES\_MODEL\_GRID::}}\sphinxbfcode{\sphinxupquote{force\_1d\_grid}}}
\pysigstopsignatures\begin{quote}\begin{description}
\sphinxlineitem{Type}
\sphinxAtStartPar
logical

\sphinxlineitem{Default}
\sphinxAtStartPar
F

\end{description}\end{quote}
\begin{description}
\sphinxlineitem{TRUE}
\sphinxAtStartPar
Force the model grid to be 1D, even if it would otherwise have been 2D.

\sphinxlineitem{FALSE}
\sphinxAtStartPar
The model grid takes its default shape.

\end{description}

\end{fulllineitems}

\index{use\_subgrid (in namelist JULES\_MODEL\_GRID)@\spxentry{use\_subgrid}\spxextra{in namelist JULES\_MODEL\_GRID}|spxpagem}

\begin{fulllineitems}
\phantomsection\label{\detokenize{namelists/model_grid.nml:JULES_MODEL_GRID::use_subgrid}}
\pysigstartsignatures
\pysigline{\sphinxcode{\sphinxupquote{JULES\_MODEL\_GRID::}}\sphinxbfcode{\sphinxupquote{use\_subgrid}}}
\pysigstopsignatures\begin{quote}\begin{description}
\sphinxlineitem{Type}
\sphinxAtStartPar
logical

\sphinxlineitem{Default}
\sphinxAtStartPar
F

\end{description}\end{quote}
\begin{description}
\sphinxlineitem{TRUE}
\sphinxAtStartPar
The model grid is a subset of the full input grid, specified using some valid combination of the options below.

\sphinxlineitem{FALSE}
\sphinxAtStartPar
The model grid is the full input grid.

\end{description}

\end{fulllineitems}


\begin{sphinxadmonition}{note}{Only used if \sphinxstyleliteralintitle{\sphinxupquote{use\_subgrid}} = TRUE}
\index{l\_bounds (in namelist JULES\_MODEL\_GRID)@\spxentry{l\_bounds}\spxextra{in namelist JULES\_MODEL\_GRID}|spxpagem}

\begin{fulllineitems}
\phantomsection\label{\detokenize{namelists/model_grid.nml:JULES_MODEL_GRID::l_bounds}}
\pysigstartsignatures
\pysigline{\sphinxcode{\sphinxupquote{JULES\_MODEL\_GRID::}}\sphinxbfcode{\sphinxupquote{l\_bounds}}}
\pysigstopsignatures\begin{quote}\begin{description}
\sphinxlineitem{Type}
\sphinxAtStartPar
logical

\sphinxlineitem{Default}
\sphinxAtStartPar
None

\end{description}\end{quote}
\begin{description}
\sphinxlineitem{TRUE}
\sphinxAtStartPar
Subset of points to model will be selected using bounds for the coordinates variables.

\sphinxlineitem{FALSE}
\sphinxAtStartPar
Subset of points to model will be selected using a list of coordinate pairs for each point.

\end{description}

\sphinxAtStartPar
If {\hyperref[\detokenize{namelists/model_grid.nml:JULES_LATLON::l_coord_latlon}]{\sphinxcrossref{\sphinxcode{\sphinxupquote{l\_coord\_latlon}}}}} = TRUE, the coordinates used here are latitude and longitude.

\sphinxAtStartPar
If {\hyperref[\detokenize{namelists/model_grid.nml:JULES_LATLON::l_coord_latlon}]{\sphinxcrossref{\sphinxcode{\sphinxupquote{l\_coord\_latlon}}}}} = FALSE, the coordinates used here are the values stored in the variables projection\_x\_coord and projection\_y\_coord.

\end{fulllineitems}


\begin{sphinxadmonition}{note}{Only used if \sphinxstyleliteralintitle{\sphinxupquote{l\_bounds}} = TRUE}
\index{y\_bounds (in namelist JULES\_MODEL\_GRID)@\spxentry{y\_bounds}\spxextra{in namelist JULES\_MODEL\_GRID}|spxpagem}

\begin{fulllineitems}
\phantomsection\label{\detokenize{namelists/model_grid.nml:JULES_MODEL_GRID::y_bounds}}
\pysigstartsignatures
\pysigline{\sphinxcode{\sphinxupquote{JULES\_MODEL\_GRID::}}\sphinxbfcode{\sphinxupquote{y\_bounds}}}
\pysigstopsignatures\begin{quote}\begin{description}
\sphinxlineitem{Type}
\sphinxAtStartPar
real(2)

\sphinxlineitem{Default}
\sphinxAtStartPar
None

\end{description}\end{quote}

\sphinxAtStartPar
The lower and upper bounds (in that order) for the y coordinate used to select points. Assuming that the coordinate is latitude (see  {\hyperref[\detokenize{namelists/model_grid.nml:JULES_LATLON::l_coord_latlon}]{\sphinxcrossref{\sphinxcode{\sphinxupquote{l\_coord\_latlon}}}}}) the model grid will comprise the points where \sphinxcode{\sphinxupquote{y\_bounds(1) \textless{}= latitude \textless{}= y\_bounds(2)}}.

\end{fulllineitems}

\index{x\_bounds (in namelist JULES\_MODEL\_GRID)@\spxentry{x\_bounds}\spxextra{in namelist JULES\_MODEL\_GRID}|spxpagem}

\begin{fulllineitems}
\phantomsection\label{\detokenize{namelists/model_grid.nml:JULES_MODEL_GRID::x_bounds}}
\pysigstartsignatures
\pysigline{\sphinxcode{\sphinxupquote{JULES\_MODEL\_GRID::}}\sphinxbfcode{\sphinxupquote{x\_bounds}}}
\pysigstopsignatures\begin{quote}\begin{description}
\sphinxlineitem{Type}
\sphinxAtStartPar
real(2)

\sphinxlineitem{Default}
\sphinxAtStartPar
None

\end{description}\end{quote}

\sphinxAtStartPar
The lower and upper bounds (in that order) for the x coordinate used to select points. Assuming that the coordinate is longitude (see  {\hyperref[\detokenize{namelists/model_grid.nml:JULES_LATLON::l_coord_latlon}]{\sphinxcrossref{\sphinxcode{\sphinxupquote{l\_coord\_latlon}}}}}) the model grid will comprise the points where \sphinxcode{\sphinxupquote{x\_bounds(1) \textless{}= longitude \textless{}= x\_bounds(2)}}.

\sphinxAtStartPar
If the x coordinate is longitude, the values of x\_bounds should lie in the range {[}\sphinxhyphen{}180, 360{]}. A special case is that in which the desired subgrid straddles the edge of a global input grid. For example, if the input grid has longitudes in {[}0, 360{]} and a domain of 20 degrees of longitude centred on 0degE is required, this should be indicated using \sphinxcode{\sphinxupquote{xbounds=\sphinxhyphen{}10,10}} (not xbounds=360,370 because values \textgreater{} 360 are not recognised). In this case the JULES code recognises the cyclic nature of longitude and correctly picks up points in both hemispheres, even though \sphinxhyphen{}10degE is outwith the longitude values in the input grid.

\end{fulllineitems}

\end{sphinxadmonition}

\begin{sphinxadmonition}{note}{Only used if \sphinxstyleliteralintitle{\sphinxupquote{l\_bounds}} = FALSE}
\index{npoints (in namelist JULES\_MODEL\_GRID)@\spxentry{npoints}\spxextra{in namelist JULES\_MODEL\_GRID}|spxpagem}

\begin{fulllineitems}
\phantomsection\label{\detokenize{namelists/model_grid.nml:JULES_MODEL_GRID::npoints}}
\pysigstartsignatures
\pysigline{\sphinxcode{\sphinxupquote{JULES\_MODEL\_GRID::}}\sphinxbfcode{\sphinxupquote{npoints}}}
\pysigstopsignatures\begin{quote}\begin{description}
\sphinxlineitem{Type}
\sphinxAtStartPar
integer

\sphinxlineitem{Permitted}
\sphinxAtStartPar
\textgreater{}= 1

\sphinxlineitem{Default}
\sphinxAtStartPar
0

\end{description}\end{quote}

\sphinxAtStartPar
The number of points to model.

\end{fulllineitems}

\index{points\_file (in namelist JULES\_MODEL\_GRID)@\spxentry{points\_file}\spxextra{in namelist JULES\_MODEL\_GRID}|spxpagem}

\begin{fulllineitems}
\phantomsection\label{\detokenize{namelists/model_grid.nml:JULES_MODEL_GRID::points_file}}
\pysigstartsignatures
\pysigline{\sphinxcode{\sphinxupquote{JULES\_MODEL\_GRID::}}\sphinxbfcode{\sphinxupquote{points\_file}}}
\pysigstopsignatures\begin{quote}\begin{description}
\sphinxlineitem{Type}
\sphinxAtStartPar
character

\sphinxlineitem{Default}
\sphinxAtStartPar
None

\end{description}\end{quote}

\sphinxAtStartPar
The name of the file containing the coordinates for each point.

\sphinxAtStartPar
If {\hyperref[\detokenize{namelists/model_grid.nml:JULES_LATLON::l_coord_latlon}]{\sphinxcrossref{\sphinxcode{\sphinxupquote{l\_coord\_latlon}}}}} = TRUE, the coordinates used here are latitude and longitude. Each line in the file should contain the latitude and longitude (in that order) for a point.

\sphinxAtStartPar
If {\hyperref[\detokenize{namelists/model_grid.nml:JULES_LATLON::l_coord_latlon}]{\sphinxcrossref{\sphinxcode{\sphinxupquote{l\_coord\_latlon}}}}} = FALSE, the coordinates used here are the values stored in the variables projection\_x\_coord and projection\_y\_coord. Each line in the file should contain the value for projection\_y\_coord and projection\_x\_coord (in that order; note this is y then x) for a point.

\sphinxAtStartPar
An error is raised and the run terminates if any coordinate pair does not match to a location in the input grid.

\end{fulllineitems}

\end{sphinxadmonition}
\end{sphinxadmonition}


\subsection{\sphinxstyleliteralintitle{\sphinxupquote{JULES\_NLSIZES}} namelist members}
\label{\detokenize{namelists/model_grid.nml:namelist-JULES_NLSIZES}}\label{\detokenize{namelists/model_grid.nml:jules-nlsizes-namelist-members}}\index{JULES\_NLSIZES (namelist)@\spxentry{JULES\_NLSIZES}\spxextra{namelist}|spxpagem}
\sphinxAtStartPar
This namelist is used to set the number of levels in the boundary layer.
\index{bl\_levels (in namelist JULES\_NLSIZES)@\spxentry{bl\_levels}\spxextra{in namelist JULES\_NLSIZES}|spxpagem}

\begin{fulllineitems}
\phantomsection\label{\detokenize{namelists/model_grid.nml:JULES_NLSIZES::bl_levels}}
\pysigstartsignatures
\pysigline{\sphinxcode{\sphinxupquote{JULES\_NLSIZES::}}\sphinxbfcode{\sphinxupquote{bl\_levels}}}
\pysigstopsignatures\begin{quote}\begin{description}
\sphinxlineitem{Type}
\sphinxAtStartPar
integer

\sphinxlineitem{Default}
\sphinxAtStartPar
1

\end{description}\end{quote}

\sphinxAtStartPar
Number of boundary layer levels. This is only used if atmospheric deposition is selected ({\hyperref[\detokenize{namelists/jules_deposition.nml:JULES_DEPOSITION::l_deposition}]{\sphinxcrossref{\sphinxcode{\sphinxupquote{l\_deposition}}}}} = TRUE) in which case it is used to set the size of input fields.

\end{fulllineitems}



\subsection{\sphinxstyleliteralintitle{\sphinxupquote{JULES\_SURF\_HGT}} namelist members}
\label{\detokenize{namelists/model_grid.nml:namelist-JULES_SURF_HGT}}\label{\detokenize{namelists/model_grid.nml:jules-surf-hgt-namelist-members}}\index{JULES\_SURF\_HGT (namelist)@\spxentry{JULES\_SURF\_HGT}\spxextra{namelist}|spxpagem}
\sphinxAtStartPar
This namelist sets the elevation of each surface tile. Elevations can either be \sphinxstyleemphasis{relative to the gridbox mean} or \sphinxstyleemphasis{have constant elevation bands above sea\sphinxhyphen{}level}.

\sphinxAtStartPar
If tile elevations are set relative to the gridbox mean, then the gridbox mean elevation is not required. The gridbox mean elevation is implicit in the near\sphinxhyphen{}surface meteorological data that are provided (higher locations will tend to be colder, have lower pressure, etc.). The elevation of each tile is used to alter the values of the air temperature and humidity (and possibly downwelling longwave, see {\hyperref[\detokenize{namelists/jules_surface.nml:JULES_SURFACE::l_elev_lw_down}]{\sphinxcrossref{\sphinxcode{\sphinxupquote{l\_elev\_lw\_down}}}}}) over that tile relative to the surface.

\sphinxAtStartPar
If any tile uses absolute heights (i.e. {\hyperref[\detokenize{namelists/model_grid.nml:JULES_SURF_HGT::l_elev_absolute_height}]{\sphinxcrossref{\sphinxcode{\sphinxupquote{l\_elev\_absolute\_height}}}}} has at least one element that is .true.), then the gridbox mean elevation must also be supplied. This is read in from the optional {\hyperref[\detokenize{namelists/model_grid.nml:namelist-JULES_Z_LAND}]{\sphinxcrossref{\sphinxcode{\sphinxupquote{JULES\_Z\_LAND}}}}} namelist which is described below. The model calculates elevations relative to the gridbox mean by taking the difference between the absolute elevation and the gridbox mean.

\sphinxAtStartPar
If any tile uses absolute heights, then tile heights are set constant across a domain, regardless of whether each tile’s height is specified as a relative offset or absolute. This makes it simple to set zero\sphinxhyphen{}offset heights for tiles that should not be considered in the elevation bands. It is no longer possible to have spatially varying tile heights if this option is used.
\index{zero\_height (in namelist JULES\_SURF\_HGT)@\spxentry{zero\_height}\spxextra{in namelist JULES\_SURF\_HGT}|spxpagem}

\begin{fulllineitems}
\phantomsection\label{\detokenize{namelists/model_grid.nml:JULES_SURF_HGT::zero_height}}
\pysigstartsignatures
\pysigline{\sphinxcode{\sphinxupquote{JULES\_SURF\_HGT::}}\sphinxbfcode{\sphinxupquote{zero\_height}}}
\pysigstopsignatures\begin{quote}\begin{description}
\sphinxlineitem{Type}
\sphinxAtStartPar
logical

\sphinxlineitem{Default}
\sphinxAtStartPar
T

\end{description}\end{quote}

\sphinxAtStartPar
Switch used to simplify the initialisation of tile elevation.

\begin{sphinxadmonition}{note}{Note:}
\sphinxAtStartPar
If {\hyperref[\detokenize{namelists/jules_surface.nml:JULES_SURFACE::l_aggregate}]{\sphinxcrossref{\sphinxcode{\sphinxupquote{l\_aggregate}}}}} = TRUE, this switch is also set to TRUE.
\end{sphinxadmonition}
\begin{description}
\sphinxlineitem{TRUE}
\sphinxAtStartPar
Set all surface tile elevations to zero. This is a very common configuration.

\sphinxlineitem{FALSE}
\sphinxAtStartPar
Set surface tile heights using specified data.

\end{description}

\end{fulllineitems}


\begin{sphinxadmonition}{note}{Only used if \sphinxstyleliteralintitle{\sphinxupquote{zero\_height}} = FALSE}
\index{l\_elev\_absolute\_height (in namelist JULES\_SURF\_HGT)@\spxentry{l\_elev\_absolute\_height}\spxextra{in namelist JULES\_SURF\_HGT}|spxpagem}

\begin{fulllineitems}
\phantomsection\label{\detokenize{namelists/model_grid.nml:JULES_SURF_HGT::l_elev_absolute_height}}
\pysigstartsignatures
\pysigline{\sphinxcode{\sphinxupquote{JULES\_SURF\_HGT::}}\sphinxbfcode{\sphinxupquote{l\_elev\_absolute\_height}}}
\pysigstopsignatures\begin{quote}\begin{description}
\sphinxlineitem{Type}
\sphinxAtStartPar
logical(nsurft)

\sphinxlineitem{Default}
\sphinxAtStartPar
F

\end{description}\end{quote}
\begin{description}
\sphinxlineitem{TRUE}
\sphinxAtStartPar
Heights of surface tiles are absolute values above sea\sphinxhyphen{}level. If this option is used, then the elevation of the forcing data must also be provided (see {\hyperref[\detokenize{namelists/model_grid.nml:namelist-JULES_Z_LAND}]{\sphinxcrossref{\sphinxcode{\sphinxupquote{JULES\_Z\_LAND}}}}} namelist below).

\sphinxlineitem{FALSE}
\sphinxAtStartPar
Surface tile heights are relative to the gridbox mean.

\end{description}

\end{fulllineitems}

\index{use\_file (in namelist JULES\_SURF\_HGT)@\spxentry{use\_file}\spxextra{in namelist JULES\_SURF\_HGT}|spxpagem}

\begin{fulllineitems}
\phantomsection\label{\detokenize{namelists/model_grid.nml:JULES_SURF_HGT::use_file}}
\pysigstartsignatures
\pysigline{\sphinxcode{\sphinxupquote{JULES\_SURF\_HGT::}}\sphinxbfcode{\sphinxupquote{use\_file}}}
\pysigstopsignatures\begin{quote}\begin{description}
\sphinxlineitem{Type}
\sphinxAtStartPar
logical

\sphinxlineitem{Default}
\sphinxAtStartPar
T

\end{description}\end{quote}

\sphinxAtStartPar
This indicates if surface tile heights relative to the gridbox mean should be read from a specified file or namelist.
\begin{description}
\sphinxlineitem{TRUE}
\sphinxAtStartPar
The variable will be read from a file if the input grid consists of more than location.

\sphinxlineitem{FALSE}
\sphinxAtStartPar
The variable will be read from a namelist if the input grid is for a single location.

\end{description}

\end{fulllineitems}

\end{sphinxadmonition}

\begin{sphinxadmonition}{note}{Only used if \sphinxstyleliteralintitle{\sphinxupquote{use\_file}} = TRUE}
\index{file (in namelist JULES\_SURF\_HGT)@\spxentry{file}\spxextra{in namelist JULES\_SURF\_HGT}|spxpagem}

\begin{fulllineitems}
\phantomsection\label{\detokenize{namelists/model_grid.nml:JULES_SURF_HGT::file}}
\pysigstartsignatures
\pysigline{\sphinxcode{\sphinxupquote{JULES\_SURF\_HGT::}}\sphinxbfcode{\sphinxupquote{file}}}
\pysigstopsignatures\begin{quote}\begin{description}
\sphinxlineitem{Type}
\sphinxAtStartPar
character

\sphinxlineitem{Default}
\sphinxAtStartPar
None

\end{description}\end{quote}

\sphinxAtStartPar
The name of the file containing surface tile heights relative to the gridbox mean.

\end{fulllineitems}

\index{surf\_hgt\_name (in namelist JULES\_SURF\_HGT)@\spxentry{surf\_hgt\_name}\spxextra{in namelist JULES\_SURF\_HGT}|spxpagem}

\begin{fulllineitems}
\phantomsection\label{\detokenize{namelists/model_grid.nml:JULES_SURF_HGT::surf_hgt_name}}
\pysigstartsignatures
\pysigline{\sphinxcode{\sphinxupquote{JULES\_SURF\_HGT::}}\sphinxbfcode{\sphinxupquote{surf\_hgt\_name}}}
\pysigstopsignatures\begin{quote}\begin{description}
\sphinxlineitem{Type}
\sphinxAtStartPar
character

\sphinxlineitem{Default}
\sphinxAtStartPar
‘surf\_hgt’

\end{description}\end{quote}

\sphinxAtStartPar
The name of the variable containing surface tile heights relative to the gridbox mean. In the file, the variable must have a single levels dimension of size \sphinxcode{\sphinxupquote{nsurft}} called {\hyperref[\detokenize{namelists/model_grid.nml:JULES_INPUT_GRID::tile_dim_name}]{\sphinxcrossref{\sphinxcode{\sphinxupquote{tile\_dim\_name}}}}}.

\end{fulllineitems}

\end{sphinxadmonition}

\begin{sphinxadmonition}{note}{Only used if \sphinxstyleliteralintitle{\sphinxupquote{use\_file}} = FALSE}
\index{surf\_hgt\_io (in namelist JULES\_SURF\_HGT)@\spxentry{surf\_hgt\_io}\spxextra{in namelist JULES\_SURF\_HGT}|spxpagem}

\begin{fulllineitems}
\phantomsection\label{\detokenize{namelists/model_grid.nml:JULES_SURF_HGT::surf_hgt_io}}
\pysigstartsignatures
\pysigline{\sphinxcode{\sphinxupquote{JULES\_SURF\_HGT::}}\sphinxbfcode{\sphinxupquote{surf\_hgt\_io}}}
\pysigstopsignatures\begin{quote}\begin{description}
\sphinxlineitem{Type}
\sphinxAtStartPar
real(nsurft)

\sphinxlineitem{Default}
\sphinxAtStartPar
None

\end{description}\end{quote}

\sphinxAtStartPar
Surface tile heights relative to the gridbox mean for a single location.

\end{fulllineitems}

\end{sphinxadmonition}


\subsection{\sphinxstyleliteralintitle{\sphinxupquote{JULES\_Z\_LAND}} namelist members}
\label{\detokenize{namelists/model_grid.nml:jules-z-land-namelist-members}}
\sphinxAtStartPar
This is an optional namelist and only used if any surface tile has {\hyperref[\detokenize{namelists/model_grid.nml:JULES_SURF_HGT::l_elev_absolute_height}]{\sphinxcrossref{\sphinxcode{\sphinxupquote{l\_elev\_absolute\_height}}}}} = TRUE. The namelist sets values for the elevation bands and reads the elevation of the forcing data.

\phantomsection\label{\detokenize{namelists/model_grid.nml:namelist-JULES_Z_LAND}}\index{JULES\_Z\_LAND (namelist)@\spxentry{JULES\_Z\_LAND}\spxextra{namelist}|spxpagem}\index{surf\_hgt\_band (in namelist JULES\_Z\_LAND)@\spxentry{surf\_hgt\_band}\spxextra{in namelist JULES\_Z\_LAND}|spxpagem}

\begin{fulllineitems}
\phantomsection\label{\detokenize{namelists/model_grid.nml:JULES_Z_LAND::surf_hgt_band}}
\pysigstartsignatures
\pysigline{\sphinxcode{\sphinxupquote{JULES\_Z\_LAND::}}\sphinxbfcode{\sphinxupquote{surf\_hgt\_band}}}
\pysigstopsignatures\begin{quote}\begin{description}
\sphinxlineitem{Type}
\sphinxAtStartPar
real(nsurft)

\sphinxlineitem{Default}
\sphinxAtStartPar
None

\end{description}\end{quote}

\sphinxAtStartPar
Spatially invariant elevation bands for each surface tile. These may be relative to the gridbox mean or absolute elevations above sea\sphinxhyphen{}level depending on {\hyperref[\detokenize{namelists/model_grid.nml:JULES_SURF_HGT::l_elev_absolute_height}]{\sphinxcrossref{\sphinxcode{\sphinxupquote{l\_elev\_absolute\_height}}}}}.

\end{fulllineitems}

\index{use\_file (in namelist JULES\_Z\_LAND)@\spxentry{use\_file}\spxextra{in namelist JULES\_Z\_LAND}|spxpagem}

\begin{fulllineitems}
\phantomsection\label{\detokenize{namelists/model_grid.nml:JULES_Z_LAND::use_file}}
\pysigstartsignatures
\pysigline{\sphinxcode{\sphinxupquote{JULES\_Z\_LAND::}}\sphinxbfcode{\sphinxupquote{use\_file}}}
\pysigstopsignatures\begin{quote}\begin{description}
\sphinxlineitem{Type}
\sphinxAtStartPar
logical

\sphinxlineitem{Default}
\sphinxAtStartPar
T

\end{description}\end{quote}

\sphinxAtStartPar
This indicates if the elevation of the forcing data should be read from a file or from a namelist.
\begin{description}
\sphinxlineitem{TRUE}
\sphinxAtStartPar
The variable will be read from a file if the input grid consists of more than location.

\sphinxlineitem{FALSE}
\sphinxAtStartPar
The variable will be read from a namelist if the input grid is for a single location.

\end{description}

\end{fulllineitems}


\begin{sphinxadmonition}{note}{Used if \sphinxstyleliteralintitle{\sphinxupquote{use\_file}} = TRUE}
\index{file (in namelist JULES\_Z\_LAND)@\spxentry{file}\spxextra{in namelist JULES\_Z\_LAND}|spxpagem}

\begin{fulllineitems}
\phantomsection\label{\detokenize{namelists/model_grid.nml:JULES_Z_LAND::file}}
\pysigstartsignatures
\pysigline{\sphinxcode{\sphinxupquote{JULES\_Z\_LAND::}}\sphinxbfcode{\sphinxupquote{file}}}
\pysigstopsignatures\begin{quote}\begin{description}
\sphinxlineitem{Type}
\sphinxAtStartPar
character

\sphinxlineitem{Default}
\sphinxAtStartPar
None

\end{description}\end{quote}

\sphinxAtStartPar
The name of the file containing the elevation of the forcing data.

\end{fulllineitems}

\index{z\_land\_name (in namelist JULES\_Z\_LAND)@\spxentry{z\_land\_name}\spxextra{in namelist JULES\_Z\_LAND}|spxpagem}

\begin{fulllineitems}
\phantomsection\label{\detokenize{namelists/model_grid.nml:JULES_Z_LAND::z_land_name}}
\pysigstartsignatures
\pysigline{\sphinxcode{\sphinxupquote{JULES\_Z\_LAND::}}\sphinxbfcode{\sphinxupquote{z\_land\_name}}}
\pysigstopsignatures\begin{quote}\begin{description}
\sphinxlineitem{Type}
\sphinxAtStartPar
character

\sphinxlineitem{Default}
\sphinxAtStartPar
‘z\_land’

\end{description}\end{quote}

\sphinxAtStartPar
The name of the variable containing the elevation of the forcing data. In the file, the variable must have no level dimensions and no time dimensions.

\end{fulllineitems}

\end{sphinxadmonition}

\begin{sphinxadmonition}{note}{Used if \sphinxstyleliteralintitle{\sphinxupquote{use\_file}} = FALSE}
\index{z\_land\_io (in namelist JULES\_Z\_LAND)@\spxentry{z\_land\_io}\spxextra{in namelist JULES\_Z\_LAND}|spxpagem}

\begin{fulllineitems}
\phantomsection\label{\detokenize{namelists/model_grid.nml:JULES_Z_LAND::z_land_io}}
\pysigstartsignatures
\pysigline{\sphinxcode{\sphinxupquote{JULES\_Z\_LAND::}}\sphinxbfcode{\sphinxupquote{z\_land\_io}}}
\pysigstopsignatures\begin{quote}\begin{description}
\sphinxlineitem{Type}
\sphinxAtStartPar
real

\sphinxlineitem{Default}
\sphinxAtStartPar
None

\end{description}\end{quote}

\sphinxAtStartPar
Elevation of the forcing data for a single location.

\end{fulllineitems}

\end{sphinxadmonition}


\subsubsection{Example}
\label{\detokenize{namelists/model_grid.nml:example}}
\sphinxAtStartPar
The following gives an example of how you would set up the namelists to use elevation bands above sea\sphinxhyphen{}level.

\begin{sphinxVerbatim}[commandchars=\\\{\}]
\PYG{n+nn}{\PYGZam{}JULES\PYGZus{}SURF\PYGZus{}HGT}

  \PYG{n+nv}{zero\PYGZus{}height}            \PYG{o}{=} \PYG{l+s+ss}{.false.}\PYG{p}{,}

  \PYG{c+c1}{\PYGZsh{} No elevation correction to surface tiles 1 to 6, use elevation bands for surface tiles 7 to 9}
  \PYG{n+nv}{l\PYGZus{}elev\PYGZus{}absolute\PYGZus{}height} \PYG{o}{=} \PYG{l+m+mi}{6}\PYG{o}{*}\PYG{l+s+ss}{.false.}\PYG{p}{,} \PYG{l+m+mi}{3}\PYG{o}{*}\PYG{l+s+ss}{.true.}\PYG{p}{,}

\PYG{n+nn}{/}

\PYG{n+nn}{\PYGZam{}JULES\PYGZus{}Z\PYGZus{}LAND}

  \PYG{c+c1}{\PYGZsh{} Set values for the elevation bands.}
  \PYG{n+nv}{surf\PYGZus{}hgt\PYGZus{}band}          \PYG{o}{=} \PYG{l+m+mi}{6}\PYG{o}{*}\PYG{l+m+mf}{0.0}\PYG{p}{,} \PYG{l+m+mf}{1000.0}\PYG{p}{,} \PYG{l+m+mf}{2000.0}\PYG{p}{,} \PYG{l+m+mf}{3000.0}\PYG{p}{,}

  \PYG{c+c1}{\PYGZsh{} Read the WFDEI forcing data elevation from a file}
  \PYG{n+nv}{use\PYGZus{}file}               \PYG{o}{=} \PYG{l+s+ss}{.true.}\PYG{p}{,}
  \PYG{n+nv}{file}                   \PYG{o}{=} \PYG{l+s+s1}{\PYGZsq{}WFDEI\PYGZhy{}elevation.nc\PYGZsq{}}\PYG{p}{,}
  \PYG{n+nv}{z\PYGZus{}land\PYGZus{}name}            \PYG{o}{=} \PYG{l+s+s1}{\PYGZsq{}elevation\PYGZsq{}}

\PYG{n+nn}{/}
\end{sphinxVerbatim}


\subsection{Examples of grid setups}
\label{\detokenize{namelists/model_grid.nml:examples-of-grid-setups}}\label{\detokenize{namelists/model_grid.nml:grid-examples}}

\subsubsection{A single location}
\label{\detokenize{namelists/model_grid.nml:a-single-location}}
\begin{sphinxVerbatim}[commandchars=\\\{\}]
\PYG{n+nn}{\PYGZam{}JULES\PYGZus{}INPUT\PYGZus{}GRID}
  \PYG{n+nv}{nx} \PYG{o}{=} \PYG{l+m+mi}{1}\PYG{p}{,}
  \PYG{n+nv}{ny} \PYG{o}{=} \PYG{l+m+mi}{1}
\PYG{n+nn}{/}

\PYG{n+nn}{\PYGZam{}JULES\PYGZus{}LATLON}
  \PYG{n+nv}{l\PYGZus{}coord\PYGZus{}latlon} \PYG{o}{=} \PYG{l+s+ss}{T}
  \PYG{n+nv}{nvars}     \PYG{o}{=} \PYG{l+m+mi}{2}\PYG{p}{,}
  \PYG{n+nv}{var}       \PYG{o}{=} \PYG{l+s+s1}{\PYGZsq{}latitude\PYGZsq{}}\PYG{p}{,}\PYG{l+s+s1}{\PYGZsq{}longitude\PYGZsq{}}\PYG{p}{,}
  \PYG{n+nv}{use\PYGZus{}file}  \PYG{o}{=} \PYG{l+s+ss}{.false.}\PYG{p}{,} \PYG{l+s+ss}{.false.}\PYG{p}{,}
  \PYG{n+nv}{const\PYGZus{}val} \PYG{o}{=} \PYG{l+m+mf}{52.168}\PYG{p}{,} \PYG{l+m+mf}{5.744}
\PYG{n+nn}{/}

\PYG{n+nn}{\PYGZam{}JULES\PYGZus{}LAND\PYGZus{}FRAC} \PYG{n+nn}{/}

\PYG{n+nn}{\PYGZam{}JULES\PYGZus{}MODEL\PYGZus{}GRID} \PYG{n+nn}{/}

\PYG{n+nn}{\PYGZam{}JULES\PYGZus{}SURF\PYGZus{}HGT}
  \PYG{n+nv}{zero\PYGZus{}height} \PYG{o}{=} \PYG{l+s+ss}{T}
\PYG{n+nn}{/}
\end{sphinxVerbatim}
\begin{description}
\sphinxlineitem{{\hyperref[\detokenize{namelists/model_grid.nml:namelist-JULES_INPUT_GRID}]{\sphinxcrossref{\sphinxcode{\sphinxupquote{JULES\_INPUT\_GRID}}}}}}
\sphinxAtStartPar
The default value of {\hyperref[\detokenize{namelists/model_grid.nml:JULES_INPUT_GRID::grid_is_1d}]{\sphinxcrossref{\sphinxcode{\sphinxupquote{grid\_is\_1d}}}}}, FALSE, is used. This means the user has to specify the extents, {\hyperref[\detokenize{namelists/model_grid.nml:JULES_INPUT_GRID::nx}]{\sphinxcrossref{\sphinxcode{\sphinxupquote{nx}}}}} and {\hyperref[\detokenize{namelists/model_grid.nml:JULES_INPUT_GRID::ny}]{\sphinxcrossref{\sphinxcode{\sphinxupquote{ny}}}}}, of the input grid. Since all the input data is ASCII, no dimension names are required.

\sphinxlineitem{{\hyperref[\detokenize{namelists/model_grid.nml:namelist-JULES_LATLON}]{\sphinxcrossref{\sphinxcode{\sphinxupquote{JULES\_LATLON}}}}}}
\sphinxAtStartPar
The latitude and longitude of the single location are specified directly in the namelist. {\hyperref[\detokenize{namelists/model_grid.nml:JULES_LATLON::nvars}]{\sphinxcrossref{\sphinxcode{\sphinxupquote{nvars}}}}} = 2 indicates that the two mandatory variables will be provided, and {\hyperref[\detokenize{namelists/model_grid.nml:JULES_LATLON::var}]{\sphinxcrossref{\sphinxcode{\sphinxupquote{var}}}}} = ‘latitude’,’longitude’ confirms that these are the latitude and longitude. {\hyperref[\detokenize{namelists/model_grid.nml:JULES_LATLON::use_file}]{\sphinxcrossref{\sphinxcode{\sphinxupquote{use\_file}}}}} = .false. indicates that the values will be read from the namelist (not from another file) and the values are provided after {\hyperref[\detokenize{namelists/model_grid.nml:JULES_LATLON::const_val}]{\sphinxcrossref{\sphinxcode{\sphinxupquote{const\_val}}}}}.

\sphinxlineitem{{\hyperref[\detokenize{namelists/model_grid.nml:namelist-JULES_LAND_FRAC}]{\sphinxcrossref{\sphinxcode{\sphinxupquote{JULES\_LAND\_FRAC}}}}}}
\sphinxAtStartPar
The land fraction at the single location is assumed to be 100\%, so nothing is required.

\sphinxlineitem{{\hyperref[\detokenize{namelists/model_grid.nml:namelist-JULES_MODEL_GRID}]{\sphinxcrossref{\sphinxcode{\sphinxupquote{JULES\_MODEL\_GRID}}}}}}
\sphinxAtStartPar
Use default options to select the model grid (i.e. land points only from the full input grid). In this case, this leaves the single location as the model grid.

\end{description}


\subsubsection{Examples of gridded runs}
\label{\detokenize{namelists/model_grid.nml:examples-of-gridded-runs}}
\sphinxAtStartPar
All the examples in this section assume gridded NetCDF data.


\paragraph{Specifying a 1D input grid}
\label{\detokenize{namelists/model_grid.nml:specifying-a-1d-input-grid}}
\sphinxAtStartPar
In this example, input files contain data on a vector of land points. The land points dimension is called “land”. The time dimension for time\sphinxhyphen{}varying variables is called “time”. The default dimension names are used for all additional dimensions (e.g. pft, tile).

\begin{sphinxVerbatim}[commandchars=\\\{\}]
\PYG{n+nn}{\PYGZam{}JULES\PYGZus{}INPUT\PYGZus{}GRID}
  \PYG{n+nv}{grid\PYGZus{}is\PYGZus{}1D} \PYG{o}{=} \PYG{l+s+ss}{T}\PYG{p}{,}

  \PYG{n+nv}{npoints} \PYG{o}{=} \PYG{l+m+mi}{15238}\PYG{p}{,}
  \PYG{n+nv}{grid\PYGZus{}dim\PYGZus{}name} \PYG{o}{=} \PYG{l+s+s2}{\PYGZdq{}land\PYGZdq{}}\PYG{p}{,}

  \PYG{n+nv}{time\PYGZus{}dim\PYGZus{}name} \PYG{o}{=} \PYG{l+s+s2}{\PYGZdq{}time\PYGZdq{}}
\PYG{n+nn}{/}
\end{sphinxVerbatim}


\paragraph{Specifying a 2D input grid}
\label{\detokenize{namelists/model_grid.nml:specifying-a-2d-input-grid}}
\sphinxAtStartPar
In this example, input files contain data on a 2D latitude/longitude grid. The x dimension is called “lon” and the y dimension is called “lat”. The time dimension for time\sphinxhyphen{}varying variables is called “time”. Variables with an extra tiles dimension use the dimension name “pseudo” for that dimension. All other additional dimensions use their default names.

\begin{sphinxVerbatim}[commandchars=\\\{\}]
\PYG{n+nn}{\PYGZam{}JULES\PYGZus{}INPUT\PYGZus{}GRID}
  \PYG{n+nv}{grid\PYGZus{}is\PYGZus{}1D} \PYG{o}{=} \PYG{l+s+ss}{F}\PYG{p}{,}

  \PYG{n+nv}{nx} \PYG{o}{=} \PYG{l+m+mi}{96}\PYG{p}{,}
  \PYG{n+nv}{ny} \PYG{o}{=} \PYG{l+m+mi}{56}\PYG{p}{,}

  \PYG{n+nv}{x\PYGZus{}dim\PYGZus{}name} \PYG{o}{=} \PYG{l+s+s2}{\PYGZdq{}lon\PYGZdq{}}\PYG{p}{,}
  \PYG{n+nv}{y\PYGZus{}dim\PYGZus{}name} \PYG{o}{=} \PYG{l+s+s2}{\PYGZdq{}lat\PYGZdq{}}\PYG{p}{,}

  \PYG{n+nv}{tile\PYGZus{}dim\PYGZus{}name} \PYG{o}{=} \PYG{l+s+s2}{\PYGZdq{}pseudo\PYGZdq{}}\PYG{p}{,}

  \PYG{n+nv}{time\PYGZus{}dim\PYGZus{}name} \PYG{o}{=} \PYG{l+s+s2}{\PYGZdq{}time\PYGZdq{}}
\PYG{n+nn}{/}
\end{sphinxVerbatim}


\paragraph{Specifying a subgrid using a given range of latitude and longitude}
\label{\detokenize{namelists/model_grid.nml:specifying-a-subgrid-using-a-given-range-of-latitude-and-longitude}}
\sphinxAtStartPar
This can be used with either a 1D or 2D input grid.

\begin{sphinxVerbatim}[commandchars=\\\{\}]
\PYG{n+nn}{\PYGZam{}JULES\PYGZus{}LATLON}
  \PYG{n+nv}{l\PYGZus{}coord\PYGZus{}latlon} \PYG{o}{=} \PYG{l+s+ss}{T}\PYG{p}{,}
  \PYG{n+nv}{nvars}     \PYG{o}{=} \PYG{l+m+mi}{2}\PYG{p}{,}
  \PYG{n+nv}{var}       \PYG{o}{=} \PYG{l+s+s1}{\PYGZsq{}latitude\PYGZsq{}}\PYG{p}{,}\PYG{l+s+s1}{\PYGZsq{}longitude\PYGZsq{}}\PYG{p}{,}
  \PYG{n+nv}{use\PYGZus{}file}  \PYG{o}{=} \PYG{l+s+ss}{.true.}\PYG{p}{,} \PYG{l+s+ss}{.true.}\PYG{p}{,}
  \PYG{n+nv}{file}      \PYG{o}{=} \PYG{l+s+s1}{\PYGZsq{}lat\PYGZus{}lon.nc\PYGZsq{}}\PYG{p}{,}
\PYG{n+nn}{/}

\PYG{n+nn}{\PYGZam{}JULES\PYGZus{}LAND\PYGZus{}FRAC}
  \PYG{n+nv}{file} \PYG{o}{=} \PYG{l+s+s1}{\PYGZsq{}land\PYGZus{}mask.nc\PYGZsq{}}\PYG{p}{,}

  \PYG{n+nv}{land\PYGZus{}frac\PYGZus{}name} \PYG{o}{=} \PYG{l+s+s1}{\PYGZsq{}land\PYGZus{}frac\PYGZsq{}}
\PYG{n+nn}{/}

\PYG{n+nn}{\PYGZam{}JULES\PYGZus{}MODEL\PYGZus{}GRID}
  \PYG{n+nv}{land\PYGZus{}only} \PYG{o}{=} \PYG{l+s+ss}{F}\PYG{p}{,}

  \PYG{n+nv}{use\PYGZus{}subgrid} \PYG{o}{=} \PYG{l+s+ss}{T}\PYG{p}{,}

  \PYG{n+nv}{l\PYGZus{}bounds} \PYG{o}{=} \PYG{l+s+ss}{T}\PYG{p}{,}

  \PYG{n+nv}{y\PYGZus{}bounds} \PYG{o}{=} \PYG{l+m+mf}{55.0}  \PYG{l+m+mf}{57.0}\PYG{p}{,}
  \PYG{n+nv}{x\PYGZus{}bounds} \PYG{o}{=} \PYG{l+m+mf}{\PYGZhy{}5.0}  \PYG{l+m+mf}{\PYGZhy{}3.0}
\PYG{n+nn}{/}
\end{sphinxVerbatim}

\sphinxAtStartPar
This setup reads latitude, longitude and land fraction for each gridbox in the full input grid (1D or 2D) from the named variables in the specified files.

\sphinxAtStartPar
In {\hyperref[\detokenize{namelists/model_grid.nml:namelist-JULES_MODEL_GRID}]{\sphinxcrossref{\sphinxcode{\sphinxupquote{JULES\_MODEL\_GRID}}}}}, {\hyperref[\detokenize{namelists/model_grid.nml:JULES_MODEL_GRID::use_subgrid}]{\sphinxcrossref{\sphinxcode{\sphinxupquote{use\_subgrid}}}}} indicates that a subset of the input grid will be selected as the model grid. {\hyperref[\detokenize{namelists/model_grid.nml:JULES_MODEL_GRID::l_bounds}]{\sphinxcrossref{\sphinxcode{\sphinxupquote{l\_bounds}}}}} then indicates that latitude and longitude bounds will be used to select the subgrid. {\hyperref[\detokenize{namelists/model_grid.nml:JULES_MODEL_GRID::land_only}]{\sphinxcrossref{\sphinxcode{\sphinxupquote{land\_only}}}}} = FALSE means that sea and sea\sphinxhyphen{}ice points will remain in the model grid if any are selected. The model grid will then be a vector containing the selected points (those that fall within the latitude/longitude bounds), even if those points could be used to form a rectangular region.


\paragraph{Specifying a subgrid using a given range of projection coordinates (not latitude and longitude)}
\label{\detokenize{namelists/model_grid.nml:specifying-a-subgrid-using-a-given-range-of-projection-coordinates-not-latitude-and-longitude}}
\sphinxAtStartPar
This can be used with either a 1D or 2D input grid.

\begin{sphinxVerbatim}[commandchars=\\\{\}]
\PYG{n+nn}{\PYGZam{}JULES\PYGZus{}LATLON}
  \PYG{n+nv}{l\PYGZus{}coord\PYGZus{}latlon} \PYG{o}{=} \PYG{l+s+ss}{F}\PYG{p}{,}
  \PYG{n+nv}{nvars}     \PYG{o}{=} \PYG{l+m+mi}{4}\PYG{p}{,}
  \PYG{n+nv}{var}       \PYG{o}{=} \PYG{l+s+s1}{\PYGZsq{}latitude\PYGZsq{}}\PYG{p}{,}\PYG{l+s+s1}{\PYGZsq{}longitude\PYGZsq{}}\PYG{p}{,}\PYG{l+s+s1}{\PYGZsq{}projection\PYGZus{}x\PYGZus{}coord\PYGZsq{}}\PYG{p}{,}\PYG{l+s+s1}{\PYGZsq{}projection\PYGZus{}y\PYGZus{}coord\PYGZsq{}}
  \PYG{n+nv}{use\PYGZus{}file}  \PYG{o}{=} \PYG{l+s+ss}{.true.}\PYG{p}{,} \PYG{l+s+ss}{.true.}\PYG{p}{,} \PYG{l+s+ss}{.true.}\PYG{p}{,} \PYG{l+s+ss}{.true.}\PYG{p}{,}
  \PYG{n+nv}{file}      \PYG{o}{=} \PYG{l+s+s1}{\PYGZsq{}lat\PYGZus{}lon.nc\PYGZsq{}}\PYG{p}{,}
\PYG{n+nn}{/}

\PYG{n+nn}{\PYGZam{}JULES\PYGZus{}LAND\PYGZus{}FRAC}
  \PYG{n+nv}{file} \PYG{o}{=} \PYG{l+s+s1}{\PYGZsq{}land\PYGZus{}mask.nc\PYGZsq{}}\PYG{p}{,}

  \PYG{n+nv}{land\PYGZus{}frac\PYGZus{}name} \PYG{o}{=} \PYG{l+s+s1}{\PYGZsq{}land\PYGZus{}frac\PYGZsq{}}
\PYG{n+nn}{/}

\PYG{n+nn}{\PYGZam{}JULES\PYGZus{}MODEL\PYGZus{}GRID}
  \PYG{n+nv}{land\PYGZus{}only} \PYG{o}{=} \PYG{l+s+ss}{F}\PYG{p}{,}

  \PYG{n+nv}{use\PYGZus{}subgrid} \PYG{o}{=} \PYG{l+s+ss}{T}\PYG{p}{,}

  \PYG{n+nv}{l\PYGZus{}bounds} \PYG{o}{=} \PYG{l+s+ss}{T}\PYG{p}{,}

  \PYG{n+nv}{y\PYGZus{}bounds} \PYG{o}{=} \PYG{l+m+mf}{500.0}  \PYG{l+m+mf}{40500.0}\PYG{p}{,}
  \PYG{n+nv}{x\PYGZus{}bounds} \PYG{o}{=} \PYG{l+m+mf}{25500.0} \PYG{l+m+mf}{55500.0}
\PYG{n+nn}{/}
\end{sphinxVerbatim}

\sphinxAtStartPar
In this setup {\hyperref[\detokenize{namelists/model_grid.nml:JULES_LATLON::l_coord_latlon}]{\sphinxcrossref{\sphinxcode{\sphinxupquote{l\_coord\_latlon}}}}} = FALSE indicates that data will be read from a grid that is not defined by latitude and longitude \sphinxhyphen{} rather it uses other projection coordinates such as the northings and eastings of the Ordnance Survey (British) National Grid (BNG) OSGB36. The projection coordinates are read via the variables projection\_x\_coord and projection\_y\_coord. Note that the latitude and longitude of each point is also read in; JULES includes these in output files for reference, and they can also be required by the science code (e.g. for solar zenith angle).

\sphinxAtStartPar
In {\hyperref[\detokenize{namelists/model_grid.nml:namelist-JULES_MODEL_GRID}]{\sphinxcrossref{\sphinxcode{\sphinxupquote{JULES\_MODEL\_GRID}}}}}, {\hyperref[\detokenize{namelists/model_grid.nml:JULES_MODEL_GRID::use_subgrid}]{\sphinxcrossref{\sphinxcode{\sphinxupquote{use\_subgrid}}}}} indicates that a subset of the input grid will be selected as the model grid. {\hyperref[\detokenize{namelists/model_grid.nml:JULES_MODEL_GRID::l_bounds}]{\sphinxcrossref{\sphinxcode{\sphinxupquote{l\_bounds}}}}} then indicates that bounding values of the projection coordinates will be used to select the subgrid. {\hyperref[\detokenize{namelists/model_grid.nml:JULES_MODEL_GRID::land_only}]{\sphinxcrossref{\sphinxcode{\sphinxupquote{land\_only}}}}} = FALSE means that sea and sea\sphinxhyphen{}ice points will remain in the model grid if any are selected. The model grid will then be a vector containing the selected points (those that fall within the latitude/longitude bounds), even if those points could be used to form a rectangular region.


\paragraph{Specifying a subgrid using a list of points}
\label{\detokenize{namelists/model_grid.nml:specifying-a-subgrid-using-a-list-of-points}}
\sphinxAtStartPar
This can be used with either a 1D or 2D input grid.

\begin{sphinxVerbatim}[commandchars=\\\{\}]
\PYG{n+nn}{\PYGZam{}JULES\PYGZus{}LATLON}
  \PYG{n+nv}{l\PYGZus{}coord\PYGZus{}latlon} \PYG{o}{=} \PYG{l+s+ss}{T}\PYG{p}{,}
  \PYG{n+nv}{nvars}     \PYG{o}{=} \PYG{l+m+mi}{2}\PYG{p}{,}
  \PYG{n+nv}{var}       \PYG{o}{=} \PYG{l+s+s1}{\PYGZsq{}latitude\PYGZsq{}}\PYG{p}{,}\PYG{l+s+s1}{\PYGZsq{}longitude\PYGZsq{}}\PYG{p}{,}
  \PYG{n+nv}{use\PYGZus{}file}  \PYG{o}{=} \PYG{l+s+ss}{.true.}\PYG{p}{,} \PYG{l+s+ss}{.true.}\PYG{p}{,}
  \PYG{n+nv}{file}      \PYG{o}{=} \PYG{l+s+s1}{\PYGZsq{}lat\PYGZus{}lon.nc\PYGZsq{}}\PYG{p}{,}
\PYG{n+nn}{/}

\PYG{n+nn}{\PYGZam{}JULES\PYGZus{}LAND\PYGZus{}FRAC}
  \PYG{n+nv}{file} \PYG{o}{=} \PYG{l+s+s1}{\PYGZsq{}land\PYGZus{}mask.nc\PYGZsq{}}\PYG{p}{,}

  \PYG{n+nv}{land\PYGZus{}frac\PYGZus{}name} \PYG{o}{=} \PYG{l+s+s1}{\PYGZsq{}land\PYGZus{}frac\PYGZsq{}}
\PYG{n+nn}{/}

\PYG{n+nn}{\PYGZam{}JULES\PYGZus{}MODEL\PYGZus{}GRID}
  \PYG{n+nv}{use\PYGZus{}subgrid} \PYG{o}{=} \PYG{l+s+ss}{T}\PYG{p}{,}

  \PYG{n+nv}{l\PYGZus{}bounds} \PYG{o}{=} \PYG{l+s+ss}{F}\PYG{p}{,}

  \PYG{n+nv}{npoints} \PYG{o}{=} \PYG{l+m+mi}{4}\PYG{p}{,}
  \PYG{n+nv}{points\PYGZus{}file} \PYG{o}{=} \PYG{l+s+s1}{\PYGZsq{}points.txt\PYGZsq{}}
\PYG{n+nn}{/}
\end{sphinxVerbatim}

\sphinxAtStartPar
This setup reads latitude, longitude and land fraction for each gridbox in the full input grid (1D or 2D) from the named variables in the specified files.

\sphinxAtStartPar
In {\hyperref[\detokenize{namelists/model_grid.nml:namelist-JULES_MODEL_GRID}]{\sphinxcrossref{\sphinxcode{\sphinxupquote{JULES\_MODEL\_GRID}}}}}, {\hyperref[\detokenize{namelists/model_grid.nml:JULES_MODEL_GRID::use_subgrid}]{\sphinxcrossref{\sphinxcode{\sphinxupquote{use\_subgrid}}}}} indicates that a subset of the input grid will be selected as the model grid. {\hyperref[\detokenize{namelists/model_grid.nml:JULES_MODEL_GRID::l_bounds}]{\sphinxcrossref{\sphinxcode{\sphinxupquote{l\_bounds}}}}} then indicates that a list of latitudes and longitudes will be used to select the subgrid. {\hyperref[\detokenize{namelists/model_grid.nml:JULES_MODEL_GRID::land_only}]{\sphinxcrossref{\sphinxcode{\sphinxupquote{land\_only}}}}} is not given, meaning it takes its default value, TRUE. This means that any sea or sea\sphinxhyphen{}ice points specified in the list of points will be discarded. The model grid will then be a vector
containing the selected points (those with the given latitude/longitude).

\sphinxAtStartPar
Assuming that the input grid is a 1 degree grid and the latitude and longitude are given at the centre of the gridbox, \sphinxcode{\sphinxupquote{points.txt}} should look like the following:

\begin{sphinxVerbatim}[commandchars=\\\{\}]
\PYG{l+m+mf}{55.5}  \PYG{l+m+mf}{\PYGZhy{}4.5}
\PYG{l+m+mf}{55.5}  \PYG{l+m+mf}{\PYGZhy{}3.5}
\PYG{l+m+mf}{56.5}  \PYG{l+m+mf}{\PYGZhy{}4.5}
\PYG{l+m+mf}{56.5}  \PYG{l+m+mf}{\PYGZhy{}3.5}
\end{sphinxVerbatim}


\paragraph{The only configuration that yields a 2D model grid}
\label{\detokenize{namelists/model_grid.nml:the-only-configuration-that-yields-a-2d-model-grid}}
\begin{sphinxVerbatim}[commandchars=\\\{\}]
\PYG{n+nn}{\PYGZam{}JULES\PYGZus{}INPUT\PYGZus{}GRID}
  \PYG{n+nv}{grid\PYGZus{}is\PYGZus{}1d} \PYG{o}{=} \PYG{l+s+ss}{F}\PYG{p}{,}

  \PYG{n+nv}{nx} \PYG{o}{=} \PYG{l+m+mi}{96}\PYG{p}{,}
  \PYG{n+nv}{ny} \PYG{o}{=} \PYG{l+m+mi}{56}\PYG{p}{,}

  \PYG{c+c1}{\PYGZsh{} ...}
\PYG{n+nn}{/}

\PYG{n+nn}{\PYGZam{}JULES\PYGZus{}LATLON}
  \PYG{c+c1}{\PYGZsh{} \PYGZlt{}specified from file\PYGZgt{}}
\PYG{n+nn}{/}

\PYG{n+nn}{\PYGZam{}JULES\PYGZus{}LAND\PYGZus{}FRAC}
  \PYG{c+c1}{\PYGZsh{} \PYGZlt{}specified from file\PYGZgt{}}
\PYG{n+nn}{/}

\PYG{n+nn}{\PYGZam{}JULES\PYGZus{}MODEL\PYGZus{}GRID}
  \PYG{n+nv}{land\PYGZus{}only} \PYG{o}{=} \PYG{l+s+ss}{F}
\PYG{n+nn}{/}
\end{sphinxVerbatim}

\sphinxAtStartPar
In general, the only configuration that yields a 2D model grid is:
\begin{itemize}
\item {} 
\sphinxAtStartPar
2D input grid

\item {} 
\sphinxAtStartPar
The model grid is the full input grid, including any non\sphinxhyphen{}land points

\end{itemize}

\sphinxAtStartPar
If the input grid is a 2D region where every point is land (i.e. not the whole globe), then {\hyperref[\detokenize{namelists/model_grid.nml:JULES_MODEL_GRID::land_only}]{\sphinxcrossref{\sphinxcode{\sphinxupquote{land\_only}}}}} = TRUE would also yield a 2D model grid. If any options are set that mean some points from the input grid are not modeled, the model grid will be a vector of points. Computationally, this makes no difference.

\sphinxstepscope


\section{\sphinxstyleliteralintitle{\sphinxupquote{ancillaries.nml}}}
\label{\detokenize{namelists/ancillaries.nml:ancillaries-nml}}\label{\detokenize{namelists/ancillaries.nml::doc}}
\sphinxAtStartPar
This file sets up spatially varying ancillary values. It contains the following namelists: {\hyperref[\detokenize{namelists/ancillaries.nml:namelist-JULES_FRAC}]{\sphinxcrossref{\sphinxcode{\sphinxupquote{JULES\_FRAC}}}}}, {\hyperref[\detokenize{namelists/ancillaries.nml:namelist-JULES_VEGETATION_PROPS}]{\sphinxcrossref{\sphinxcode{\sphinxupquote{JULES\_VEGETATION\_PROPS}}}}}, {\hyperref[\detokenize{namelists/ancillaries.nml:namelist-JULES_SOIL_PROPS}]{\sphinxcrossref{\sphinxcode{\sphinxupquote{JULES\_SOIL\_PROPS}}}}}, {\hyperref[\detokenize{namelists/ancillaries.nml:namelist-JULES_TOP}]{\sphinxcrossref{\sphinxcode{\sphinxupquote{JULES\_TOP}}}}}, {\hyperref[\detokenize{namelists/ancillaries.nml:namelist-JULES_PDM}]{\sphinxcrossref{\sphinxcode{\sphinxupquote{JULES\_PDM}}}}}, {\hyperref[\detokenize{namelists/ancillaries.nml:namelist-JULES_AGRIC}]{\sphinxcrossref{\sphinxcode{\sphinxupquote{JULES\_AGRIC}}}}}, {\hyperref[\detokenize{namelists/ancillaries.nml:namelist-JULES_CROP_PROPS}]{\sphinxcrossref{\sphinxcode{\sphinxupquote{JULES\_CROP\_PROPS}}}}}, {\hyperref[\detokenize{namelists/ancillaries.nml:namelist-JULES_IRRIG_PROPS}]{\sphinxcrossref{\sphinxcode{\sphinxupquote{JULES\_IRRIG\_PROPS}}}}}, {\hyperref[\detokenize{namelists/ancillaries.nml:namelist-JULES_RIVERS_PROPS}]{\sphinxcrossref{\sphinxcode{\sphinxupquote{JULES\_RIVERS\_PROPS}}}}}, {\hyperref[\detokenize{namelists/ancillaries.nml:namelist-JULES_OVERBANK_PROPS}]{\sphinxcrossref{\sphinxcode{\sphinxupquote{JULES\_OVERBANK\_PROPS}}}}}, {\hyperref[\detokenize{namelists/ancillaries.nml:namelist-JULES_WATER_RESOURCES_PROPS}]{\sphinxcrossref{\sphinxcode{\sphinxupquote{JULES\_WATER\_RESOURCES\_PROPS}}}}}, {\hyperref[\detokenize{namelists/ancillaries.nml:namelist-URBAN_PROPERTIES}]{\sphinxcrossref{\sphinxcode{\sphinxupquote{URBAN\_PROPERTIES}}}}} , {\hyperref[\detokenize{namelists/ancillaries.nml:namelist-JULES_CO2}]{\sphinxcrossref{\sphinxcode{\sphinxupquote{JULES\_CO2}}}}} and {\hyperref[\detokenize{namelists/ancillaries.nml:namelist-JULES_FLAKE}]{\sphinxcrossref{\sphinxcode{\sphinxupquote{JULES\_FLAKE}}}}}.

\sphinxAtStartPar
Data associated with each of these namelists can optionally be read from the dump file (if present) by setting \sphinxcode{\sphinxupquote{read\_from\_dump}} to true. This functionality provides closer alignment with UM functionality and can help ensure that the correct ancillary data remain associated with the model state.


\subsection{\sphinxstyleliteralintitle{\sphinxupquote{JULES\_FRAC}} namelist members}
\label{\detokenize{namelists/ancillaries.nml:namelist-JULES_FRAC}}\label{\detokenize{namelists/ancillaries.nml:jules-frac-namelist-members}}\index{JULES\_FRAC (namelist)@\spxentry{JULES\_FRAC}\spxextra{namelist}|spxpagem}
\sphinxAtStartPar
This namelist specifies the fraction of the land area in each gridbox that is covered by each of the surface types. If {\hyperref[\detokenize{namelists/jules_vegetation.nml:JULES_VEGETATION::l_veg_compete}]{\sphinxcrossref{\sphinxcode{\sphinxupquote{l\_veg\_compete}}}}} = TRUE, then the fraction of each surface type is modelled and the initial state should be specified in {\hyperref[\detokenize{namelists/initial_conditions.nml:namelist-JULES_INITIAL}]{\sphinxcrossref{\sphinxcode{\sphinxupquote{JULES\_INITIAL}}}}}. In all other cases, it must be read here.

\sphinxAtStartPar
Note that all land points must be either soil points (indicated by values \textgreater{} 0 of the saturated soil moisture content), or land ice points (indicated by the fractional coverage of the ice surface type \sphinxhyphen{} if used \sphinxhyphen{} being one). The fractional cover of the ice surface type in each gridbox must be either zero or one \sphinxhyphen{} there cannot be partial coverage of ice within a gridbox.

\sphinxAtStartPar
If using either URBAN\sphinxhyphen{}2T or MORUSES then the total \sphinxstyleemphasis{urban} fraction can be specified instead of the separate {\hyperref[\detokenize{namelists/jules_surface_types.nml:JULES_SURFACE_TYPES::urban_canyon}]{\sphinxcrossref{\sphinxcode{\sphinxupquote{urban\_canyon}}}}} and {\hyperref[\detokenize{namelists/jules_surface_types.nml:JULES_SURFACE_TYPES::urban_roof}]{\sphinxcrossref{\sphinxcode{\sphinxupquote{urban\_roof}}}}} contributions. When initialising, if the roof fraction is zero, the canyon fraction will be interpreted as the total \sphinxstyleemphasis{urban} fraction and be partitioned according to canyon fraction (W/R, see {\hyperref[\detokenize{namelists/ancillaries.nml:namelist-URBAN_PROPERTIES}]{\sphinxcrossref{\sphinxcode{\sphinxupquote{URBAN\_PROPERTIES}}}}}).

\begin{sphinxadmonition}{note}{Note:}
\sphinxAtStartPar
For runs with dynamic vegetation (TRIFFID, {\hyperref[\detokenize{namelists/jules_vegetation.nml:JULES_VEGETATION::l_triffid}]{\sphinxcrossref{\sphinxcode{\sphinxupquote{l\_triffid}}}}} = TRUE) and {\hyperref[\detokenize{namelists/jules_vegetation.nml:JULES_VEGETATION::l_veg_compete}]{\sphinxcrossref{\sphinxcode{\sphinxupquote{l\_veg\_compete}}}}} = TRUE, then the fraction of each surface type is being modelled and the initial state should be specified in {\hyperref[\detokenize{namelists/initial_conditions.nml:namelist-JULES_INITIAL}]{\sphinxcrossref{\sphinxcode{\sphinxupquote{JULES\_INITIAL}}}}} (which will override any settings given in this section). In all other cases, frac must be read here.
\end{sphinxadmonition}
\index{read\_from\_dump (in namelist JULES\_FRAC)@\spxentry{read\_from\_dump}\spxextra{in namelist JULES\_FRAC}|spxpagem}

\begin{fulllineitems}
\phantomsection\label{\detokenize{namelists/ancillaries.nml:JULES_FRAC::read_from_dump}}
\pysigstartsignatures
\pysigline{\sphinxcode{\sphinxupquote{JULES\_FRAC::}}\sphinxbfcode{\sphinxupquote{read\_from\_dump}}}
\pysigstopsignatures\begin{quote}\begin{description}
\sphinxlineitem{Type}
\sphinxAtStartPar
logical

\sphinxlineitem{Default}
\sphinxAtStartPar
F

\end{description}\end{quote}
\begin{description}
\sphinxlineitem{TRUE}
\sphinxAtStartPar
Populate variables associated with this namelist from the dump file. All other namelist members are ignored.

\sphinxlineitem{FALSE}
\sphinxAtStartPar
Use the other namelist members to determine how to populate variables.

\end{description}

\end{fulllineitems}

\index{file (in namelist JULES\_FRAC)@\spxentry{file}\spxextra{in namelist JULES\_FRAC}|spxpagem}

\begin{fulllineitems}
\phantomsection\label{\detokenize{namelists/ancillaries.nml:JULES_FRAC::file}}
\pysigstartsignatures
\pysigline{\sphinxcode{\sphinxupquote{JULES\_FRAC::}}\sphinxbfcode{\sphinxupquote{file}}}
\pysigstopsignatures\begin{quote}\begin{description}
\sphinxlineitem{Type}
\sphinxAtStartPar
character

\sphinxlineitem{Default}
\sphinxAtStartPar
None

\end{description}\end{quote}

\sphinxAtStartPar
The name of the file to read surface type fractional coverage data from.

\end{fulllineitems}

\index{frac\_name (in namelist JULES\_FRAC)@\spxentry{frac\_name}\spxextra{in namelist JULES\_FRAC}|spxpagem}

\begin{fulllineitems}
\phantomsection\label{\detokenize{namelists/ancillaries.nml:JULES_FRAC::frac_name}}
\pysigstartsignatures
\pysigline{\sphinxcode{\sphinxupquote{JULES\_FRAC::}}\sphinxbfcode{\sphinxupquote{frac\_name}}}
\pysigstopsignatures\begin{quote}\begin{description}
\sphinxlineitem{Type}
\sphinxAtStartPar
character

\sphinxlineitem{Default}
\sphinxAtStartPar
‘frac’

\end{description}\end{quote}

\sphinxAtStartPar
The name of the variable containing the surface type fractional coverage data.

\begin{sphinxadmonition}{note}{Note:}
\sphinxAtStartPar
This is only used for NetCDF files.
For ASCII files, the surface type fractional coverage data is expected to be the first (ideally only) variable in the file.
\end{sphinxadmonition}

\begin{sphinxadmonition}{note}{Note:}
\sphinxAtStartPar
The open water fraction of this array (given by {\hyperref[\detokenize{namelists/jules_surface_types.nml:JULES_SURFACE_TYPES::lake}]{\sphinxcrossref{\sphinxcode{\sphinxupquote{lake}}}}}) should contain permanent water, and wetland extents that are not being otherwise simulated.
If groundwater inundation is being simulated (i.e. TOPMODEL is active {\hyperref[\detokenize{namelists/jules_hydrology.nml:JULES_HYDROLOGY::l_top}]{\sphinxcrossref{\sphinxcode{\sphinxupquote{l\_top}}}}} = TRUE and therefore fsat is being calculated) then all groundwater\sphinxhyphen{}maintained wetlands must be excluded from {\hyperref[\detokenize{namelists/ancillaries.nml:JULES_FRAC::frac_name}]{\sphinxcrossref{\sphinxcode{\sphinxupquote{frac\_name}}}}}.
If overbank inundation is being simulated (i.e. {\hyperref[\detokenize{namelists/jules_rivers.nml:JULES_OVERBANK::l_riv_overbank}]{\sphinxcrossref{\sphinxcode{\sphinxupquote{l\_riv\_overbank}}}}} = TRUE) then all fluvial wetlands must be excluded from {\hyperref[\detokenize{namelists/ancillaries.nml:JULES_FRAC::frac_name}]{\sphinxcrossref{\sphinxcode{\sphinxupquote{frac\_name}}}}}.
Finally, note that simulation of a potential future climate scenario with greatly reduced areas for lakes that are currently ‘permanent’ would require suitable modification of the ancillary provided here.
\end{sphinxadmonition}

\sphinxAtStartPar
In the file, the variable must have a single levels dimension of size \sphinxcode{\sphinxupquote{ntype}} called {\hyperref[\detokenize{namelists/model_grid.nml:JULES_INPUT_GRID::type_dim_name}]{\sphinxcrossref{\sphinxcode{\sphinxupquote{type\_dim\_name}}}}}, and should not have a time dimension.

\end{fulllineitems}



\subsection{\sphinxstyleliteralintitle{\sphinxupquote{JULES\_VEGETATION\_PROPS}} namelist members}
\label{\detokenize{namelists/ancillaries.nml:namelist-JULES_VEGETATION_PROPS}}\label{\detokenize{namelists/ancillaries.nml:jules-vegetation-props-namelist-members}}\index{JULES\_VEGETATION\_PROPS (namelist)@\spxentry{JULES\_VEGETATION\_PROPS}\spxextra{namelist}|spxpagem}
\sphinxAtStartPar
This namelist specifies how spatially\sphinxhyphen{}varying  properties of vegetation should be set.

\sphinxAtStartPar
At present only one variable \sphinxhyphen{} \sphinxcode{\sphinxupquote{t\_home\_gb}} \sphinxhyphen{} can be specified via this namelist, and this is only required if thermal adaptation of photosynthetic capacity is selected ({\hyperref[\detokenize{namelists/jules_vegetation.nml:JULES_VEGETATION::photo_acclim_model}]{\sphinxcrossref{\sphinxcode{\sphinxupquote{photo\_acclim\_model}}}}} = 1 or 3).

\sphinxAtStartPar
Note that Leaf Area Index and vegetation height are specified elsewhere \sphinxhyphen{} see {\hyperref[\detokenize{namelists/prescribed_data.nml:namelist-JULES_PRESCRIBED}]{\sphinxcrossref{\sphinxcode{\sphinxupquote{JULES\_PRESCRIBED}}}}}.
\index{read\_from\_dump (in namelist JULES\_VEGETATION\_PROPS)@\spxentry{read\_from\_dump}\spxextra{in namelist JULES\_VEGETATION\_PROPS}|spxpagem}

\begin{fulllineitems}
\phantomsection\label{\detokenize{namelists/ancillaries.nml:JULES_VEGETATION_PROPS::read_from_dump}}
\pysigstartsignatures
\pysigline{\sphinxcode{\sphinxupquote{JULES\_VEGETATION\_PROPS::}}\sphinxbfcode{\sphinxupquote{read\_from\_dump}}}
\pysigstopsignatures\begin{quote}\begin{description}
\sphinxlineitem{Type}
\sphinxAtStartPar
logical

\sphinxlineitem{Default}
\sphinxAtStartPar
F

\end{description}\end{quote}
\begin{description}
\sphinxlineitem{TRUE}
\sphinxAtStartPar
Populate variables associated with this namelist from the dump file. All other namelist members are ignored.

\sphinxlineitem{FALSE}
\sphinxAtStartPar
Use the other namelist members to determine how to populate variables.

\end{description}

\end{fulllineitems}

\index{file (in namelist JULES\_VEGETATION\_PROPS)@\spxentry{file}\spxextra{in namelist JULES\_VEGETATION\_PROPS}|spxpagem}

\begin{fulllineitems}
\phantomsection\label{\detokenize{namelists/ancillaries.nml:JULES_VEGETATION_PROPS::file}}
\pysigstartsignatures
\pysigline{\sphinxcode{\sphinxupquote{JULES\_VEGETATION\_PROPS::}}\sphinxbfcode{\sphinxupquote{file}}}
\pysigstopsignatures\begin{quote}\begin{description}
\sphinxlineitem{Type}
\sphinxAtStartPar
character

\sphinxlineitem{Default}
\sphinxAtStartPar
None

\end{description}\end{quote}

\sphinxAtStartPar
The file to read vegetation properties from.

\sphinxAtStartPar
If {\hyperref[\detokenize{namelists/ancillaries.nml:JULES_VEGETATION_PROPS::use_file}]{\sphinxcrossref{\sphinxcode{\sphinxupquote{use\_file}}}}} is FALSE for every variable, this will not be used.

\sphinxAtStartPar
This file name can use {\hyperref[\detokenize{input/file-name-templating::doc}]{\sphinxcrossref{\DUrole{doc}{variable name templating}}}}.

\end{fulllineitems}

\index{nvars (in namelist JULES\_VEGETATION\_PROPS)@\spxentry{nvars}\spxextra{in namelist JULES\_VEGETATION\_PROPS}|spxpagem}

\begin{fulllineitems}
\phantomsection\label{\detokenize{namelists/ancillaries.nml:JULES_VEGETATION_PROPS::nvars}}
\pysigstartsignatures
\pysigline{\sphinxcode{\sphinxupquote{JULES\_VEGETATION\_PROPS::}}\sphinxbfcode{\sphinxupquote{nvars}}}
\pysigstopsignatures\begin{quote}\begin{description}
\sphinxlineitem{Type}
\sphinxAtStartPar
integer

\sphinxlineitem{Permitted}
\sphinxAtStartPar
\textgreater{}= 0

\sphinxlineitem{Default}
\sphinxAtStartPar
0

\end{description}\end{quote}

\sphinxAtStartPar
The number of vegetation property variables that will be provided (see {\hyperref[\detokenize{namelists/ancillaries.nml:list-of-vegetation-params}]{\sphinxcrossref{\DUrole{std,std-ref}{List of vegetation parameters}}}}).

\end{fulllineitems}

\index{var (in namelist JULES\_VEGETATION\_PROPS)@\spxentry{var}\spxextra{in namelist JULES\_VEGETATION\_PROPS}|spxpagem}

\begin{fulllineitems}
\phantomsection\label{\detokenize{namelists/ancillaries.nml:JULES_VEGETATION_PROPS::var}}
\pysigstartsignatures
\pysigline{\sphinxcode{\sphinxupquote{JULES\_VEGETATION\_PROPS::}}\sphinxbfcode{\sphinxupquote{var}}}
\pysigstopsignatures\begin{quote}\begin{description}
\sphinxlineitem{Type}
\sphinxAtStartPar
character(nvars)

\sphinxlineitem{Default}
\sphinxAtStartPar
None

\end{description}\end{quote}

\sphinxAtStartPar
List of vegetation variable names as recognised by JULES (see {\hyperref[\detokenize{namelists/ancillaries.nml:list-of-vegetation-params}]{\sphinxcrossref{\DUrole{std,std-ref}{List of vegetation parameters}}}}). Names are case sensitive.

\begin{sphinxadmonition}{note}{Note:}
\sphinxAtStartPar
For ASCII files, variable names must be in the order they appear in the file.
\end{sphinxadmonition}

\end{fulllineitems}

\index{use\_file (in namelist JULES\_VEGETATION\_PROPS)@\spxentry{use\_file}\spxextra{in namelist JULES\_VEGETATION\_PROPS}|spxpagem}

\begin{fulllineitems}
\phantomsection\label{\detokenize{namelists/ancillaries.nml:JULES_VEGETATION_PROPS::use_file}}
\pysigstartsignatures
\pysigline{\sphinxcode{\sphinxupquote{JULES\_VEGETATION\_PROPS::}}\sphinxbfcode{\sphinxupquote{use\_file}}}
\pysigstopsignatures\begin{quote}\begin{description}
\sphinxlineitem{Type}
\sphinxAtStartPar
logical(nvars)

\sphinxlineitem{Default}
\sphinxAtStartPar
T

\end{description}\end{quote}

\sphinxAtStartPar
For each JULES variable specified in {\hyperref[\detokenize{namelists/ancillaries.nml:JULES_VEGETATION_PROPS::var}]{\sphinxcrossref{\sphinxcode{\sphinxupquote{var}}}}}, this indicates if it should be read from the specified file or whether a constant value is to be used.
\begin{description}
\sphinxlineitem{TRUE}
\sphinxAtStartPar
The variable will be read from the file.

\sphinxlineitem{FALSE}
\sphinxAtStartPar
The variable will be set to a constant value everywhere using {\hyperref[\detokenize{namelists/ancillaries.nml:JULES_VEGETATION_PROPS::const_val}]{\sphinxcrossref{\sphinxcode{\sphinxupquote{const\_val}}}}} below.

\end{description}

\end{fulllineitems}

\index{var\_name (in namelist JULES\_VEGETATION\_PROPS)@\spxentry{var\_name}\spxextra{in namelist JULES\_VEGETATION\_PROPS}|spxpagem}

\begin{fulllineitems}
\phantomsection\label{\detokenize{namelists/ancillaries.nml:JULES_VEGETATION_PROPS::var_name}}
\pysigstartsignatures
\pysigline{\sphinxcode{\sphinxupquote{JULES\_VEGETATION\_PROPS::}}\sphinxbfcode{\sphinxupquote{var\_name}}}
\pysigstopsignatures\begin{quote}\begin{description}
\sphinxlineitem{Type}
\sphinxAtStartPar
character(nvars)

\sphinxlineitem{Default}
\sphinxAtStartPar
‘’ (empty string)

\end{description}\end{quote}

\sphinxAtStartPar
For each JULES variable specified in {\hyperref[\detokenize{namelists/ancillaries.nml:JULES_VEGETATION_PROPS::var}]{\sphinxcrossref{\sphinxcode{\sphinxupquote{var}}}}} where {\hyperref[\detokenize{namelists/ancillaries.nml:JULES_VEGETATION_PROPS::use_file}]{\sphinxcrossref{\sphinxcode{\sphinxupquote{use\_file}}}}} = TRUE, this is the name of the variable in the file containing the data.

\sphinxAtStartPar
If the empty string (the default) is given for any variable, then the corresponding value from {\hyperref[\detokenize{namelists/ancillaries.nml:JULES_VEGETATION_PROPS::var}]{\sphinxcrossref{\sphinxcode{\sphinxupquote{var}}}}} is used instead.

\sphinxAtStartPar
This is not used for variables where {\hyperref[\detokenize{namelists/ancillaries.nml:JULES_VEGETATION_PROPS::use_file}]{\sphinxcrossref{\sphinxcode{\sphinxupquote{use\_file}}}}} = FALSE, but a placeholder must still be given in that case.

\begin{sphinxadmonition}{note}{Note:}
\sphinxAtStartPar
For ASCII files, this is not used \sphinxhyphen{} only the order in the file matters, as described above.
\end{sphinxadmonition}

\end{fulllineitems}

\index{tpl\_name (in namelist JULES\_VEGETATION\_PROPS)@\spxentry{tpl\_name}\spxextra{in namelist JULES\_VEGETATION\_PROPS}|spxpagem}

\begin{fulllineitems}
\phantomsection\label{\detokenize{namelists/ancillaries.nml:JULES_VEGETATION_PROPS::tpl_name}}
\pysigstartsignatures
\pysigline{\sphinxcode{\sphinxupquote{JULES\_VEGETATION\_PROPS::}}\sphinxbfcode{\sphinxupquote{tpl\_name}}}
\pysigstopsignatures\begin{quote}\begin{description}
\sphinxlineitem{Type}
\sphinxAtStartPar
character(nvars)

\sphinxlineitem{Default}
\sphinxAtStartPar
None

\end{description}\end{quote}

\sphinxAtStartPar
For each JULES variable specified in {\hyperref[\detokenize{namelists/ancillaries.nml:JULES_VEGETATION_PROPS::var}]{\sphinxcrossref{\sphinxcode{\sphinxupquote{var}}}}}, this is the string to substitute into the file name in place of the variable name substitution string.

\sphinxAtStartPar
If the file name does not use variable name templating, this is not used.

\end{fulllineitems}

\index{const\_val (in namelist JULES\_VEGETATION\_PROPS)@\spxentry{const\_val}\spxextra{in namelist JULES\_VEGETATION\_PROPS}|spxpagem}

\begin{fulllineitems}
\phantomsection\label{\detokenize{namelists/ancillaries.nml:JULES_VEGETATION_PROPS::const_val}}
\pysigstartsignatures
\pysigline{\sphinxcode{\sphinxupquote{JULES\_VEGETATION\_PROPS::}}\sphinxbfcode{\sphinxupquote{const\_val}}}
\pysigstopsignatures\begin{quote}\begin{description}
\sphinxlineitem{Type}
\sphinxAtStartPar
real(nvars)

\sphinxlineitem{Default}
\sphinxAtStartPar
None

\end{description}\end{quote}

\sphinxAtStartPar
For each JULES variable specified in {\hyperref[\detokenize{namelists/ancillaries.nml:JULES_VEGETATION_PROPS::var}]{\sphinxcrossref{\sphinxcode{\sphinxupquote{var}}}}} where {\hyperref[\detokenize{namelists/ancillaries.nml:JULES_VEGETATION_PROPS::use_file}]{\sphinxcrossref{\sphinxcode{\sphinxupquote{use\_file}}}}} = FALSE, this is a constant value that the variable will be set to at every point.

\sphinxAtStartPar
This is not used for variables where {\hyperref[\detokenize{namelists/ancillaries.nml:JULES_VEGETATION_PROPS::use_file}]{\sphinxcrossref{\sphinxcode{\sphinxupquote{use\_file}}}}} = TRUE, but a placeholder must still be given in that case.

\end{fulllineitems}



\subsubsection{List of vegetation parameters}
\label{\detokenize{namelists/ancillaries.nml:list-of-vegetation-parameters}}\label{\detokenize{namelists/ancillaries.nml:list-of-vegetation-params}}

\begin{savenotes}\sphinxattablestart
\centering
\begin{tabulary}{\linewidth}[t]{|p{2cm}|p{9cm}|}
\hline
\sphinxstyletheadfamily 
\sphinxAtStartPar
Name
&\sphinxstyletheadfamily 
\sphinxAtStartPar
Description
\\
\hline
\sphinxAtStartPar
\sphinxcode{\sphinxupquote{t\_home\_gb}}
&
\sphinxAtStartPar
Average temperature (home temperature) for thermal adaptation of photosynthetic
capacity (K), e.g. a multi\sphinxhyphen{}decadal average or pre\sphinxhyphen{}industrial temperature.
Suggestions as to how to calculate a suitable temperature can be found in
{\hyperref[\detokenize{namelists/ancillaries.nml:references-ancillaries}]{\sphinxcrossref{\DUrole{std,std-ref}{Kattge and Knorr (2007)}}}} or
{\hyperref[\detokenize{namelists/ancillaries.nml:references-ancillaries}]{\sphinxcrossref{\DUrole{std,std-ref}{Kumarathunge et al (2019)}}}}. This variable should not
have a time dimension nor any “levels” dimension.
\\
\hline
\end{tabulary}
\par
\sphinxattableend\end{savenotes}


\subsection{\sphinxstyleliteralintitle{\sphinxupquote{JULES\_SOIL\_PROPS}} namelist members}
\label{\detokenize{namelists/ancillaries.nml:namelist-JULES_SOIL_PROPS}}\label{\detokenize{namelists/ancillaries.nml:jules-soil-props-namelist-members}}\index{JULES\_SOIL\_PROPS (namelist)@\spxentry{JULES\_SOIL\_PROPS}\spxextra{namelist}|spxpagem}
\sphinxAtStartPar
This namelist specifies how spatially varying soil properties should be set.
\index{read\_from\_dump (in namelist JULES\_SOIL\_PROPS)@\spxentry{read\_from\_dump}\spxextra{in namelist JULES\_SOIL\_PROPS}|spxpagem}

\begin{fulllineitems}
\phantomsection\label{\detokenize{namelists/ancillaries.nml:JULES_SOIL_PROPS::read_from_dump}}
\pysigstartsignatures
\pysigline{\sphinxcode{\sphinxupquote{JULES\_SOIL\_PROPS::}}\sphinxbfcode{\sphinxupquote{read\_from\_dump}}}
\pysigstopsignatures\begin{quote}\begin{description}
\sphinxlineitem{Type}
\sphinxAtStartPar
logical

\sphinxlineitem{Default}
\sphinxAtStartPar
F

\end{description}\end{quote}
\begin{description}
\sphinxlineitem{TRUE}
\sphinxAtStartPar
Populate variables associated with this namelist from the dump file. All other namelist members are ignored.

\sphinxlineitem{FALSE}
\sphinxAtStartPar
Use the other namelist members to determine how to populate variables.

\end{description}

\end{fulllineitems}

\index{const\_z (in namelist JULES\_SOIL\_PROPS)@\spxentry{const\_z}\spxextra{in namelist JULES\_SOIL\_PROPS}|spxpagem}

\begin{fulllineitems}
\phantomsection\label{\detokenize{namelists/ancillaries.nml:JULES_SOIL_PROPS::const_z}}
\pysigstartsignatures
\pysigline{\sphinxcode{\sphinxupquote{JULES\_SOIL\_PROPS::}}\sphinxbfcode{\sphinxupquote{const\_z}}}
\pysigstopsignatures\begin{quote}\begin{description}
\sphinxlineitem{Type}
\sphinxAtStartPar
logical

\sphinxlineitem{Default}
\sphinxAtStartPar
F

\end{description}\end{quote}

\sphinxAtStartPar
Switch indicating if soil properties are to be uniform with depth.
\begin{description}
\sphinxlineitem{TRUE}
\sphinxAtStartPar
Soil characteristics do not vary with depth.

\sphinxlineitem{FALSE}
\sphinxAtStartPar
Soil characteristics vary with depth. For any variable this is ignored if a constant value is to be used (see {\hyperref[\detokenize{namelists/ancillaries.nml:JULES_SOIL_PROPS::const_val}]{\sphinxcrossref{\sphinxcode{\sphinxupquote{const\_val}}}}}).

\end{description}

\end{fulllineitems}

\index{file (in namelist JULES\_SOIL\_PROPS)@\spxentry{file}\spxextra{in namelist JULES\_SOIL\_PROPS}|spxpagem}

\begin{fulllineitems}
\phantomsection\label{\detokenize{namelists/ancillaries.nml:JULES_SOIL_PROPS::file}}
\pysigstartsignatures
\pysigline{\sphinxcode{\sphinxupquote{JULES\_SOIL\_PROPS::}}\sphinxbfcode{\sphinxupquote{file}}}
\pysigstopsignatures\begin{quote}\begin{description}
\sphinxlineitem{Type}
\sphinxAtStartPar
character

\sphinxlineitem{Default}
\sphinxAtStartPar
None

\end{description}\end{quote}

\sphinxAtStartPar
The file to read soil properties from.

\sphinxAtStartPar
If {\hyperref[\detokenize{namelists/ancillaries.nml:JULES_SOIL_PROPS::use_file}]{\sphinxcrossref{\sphinxcode{\sphinxupquote{use\_file}}}}} is FALSE for every variable, this will not be used.

\sphinxAtStartPar
This file name can use {\hyperref[\detokenize{input/file-name-templating::doc}]{\sphinxcrossref{\DUrole{doc}{variable name templating}}}}.

\end{fulllineitems}

\index{nvars (in namelist JULES\_SOIL\_PROPS)@\spxentry{nvars}\spxextra{in namelist JULES\_SOIL\_PROPS}|spxpagem}

\begin{fulllineitems}
\phantomsection\label{\detokenize{namelists/ancillaries.nml:JULES_SOIL_PROPS::nvars}}
\pysigstartsignatures
\pysigline{\sphinxcode{\sphinxupquote{JULES\_SOIL\_PROPS::}}\sphinxbfcode{\sphinxupquote{nvars}}}
\pysigstopsignatures\begin{quote}\begin{description}
\sphinxlineitem{Type}
\sphinxAtStartPar
integer

\sphinxlineitem{Permitted}
\sphinxAtStartPar
\textgreater{}= 0

\sphinxlineitem{Default}
\sphinxAtStartPar
0

\end{description}\end{quote}

\sphinxAtStartPar
The number of soil property variables that will be provided (see {\hyperref[\detokenize{namelists/ancillaries.nml:list-of-soil-params}]{\sphinxcrossref{\DUrole{std,std-ref}{List of soil parameters}}}}).

\end{fulllineitems}

\index{var (in namelist JULES\_SOIL\_PROPS)@\spxentry{var}\spxextra{in namelist JULES\_SOIL\_PROPS}|spxpagem}

\begin{fulllineitems}
\phantomsection\label{\detokenize{namelists/ancillaries.nml:JULES_SOIL_PROPS::var}}
\pysigstartsignatures
\pysigline{\sphinxcode{\sphinxupquote{JULES\_SOIL\_PROPS::}}\sphinxbfcode{\sphinxupquote{var}}}
\pysigstopsignatures\begin{quote}\begin{description}
\sphinxlineitem{Type}
\sphinxAtStartPar
character(nvars)

\sphinxlineitem{Default}
\sphinxAtStartPar
None

\end{description}\end{quote}

\sphinxAtStartPar
List of soil variable names as recognised by JULES (see {\hyperref[\detokenize{namelists/ancillaries.nml:list-of-soil-params}]{\sphinxcrossref{\DUrole{std,std-ref}{List of soil parameters}}}}). Names are case sensitive.

\begin{sphinxadmonition}{note}{Note:}
\sphinxAtStartPar
For ASCII files, variable names must be in the order they appear in the file.
\end{sphinxadmonition}

\end{fulllineitems}

\index{use\_file (in namelist JULES\_SOIL\_PROPS)@\spxentry{use\_file}\spxextra{in namelist JULES\_SOIL\_PROPS}|spxpagem}

\begin{fulllineitems}
\phantomsection\label{\detokenize{namelists/ancillaries.nml:JULES_SOIL_PROPS::use_file}}
\pysigstartsignatures
\pysigline{\sphinxcode{\sphinxupquote{JULES\_SOIL\_PROPS::}}\sphinxbfcode{\sphinxupquote{use\_file}}}
\pysigstopsignatures\begin{quote}\begin{description}
\sphinxlineitem{Type}
\sphinxAtStartPar
logical(nvars)

\sphinxlineitem{Default}
\sphinxAtStartPar
T

\end{description}\end{quote}

\sphinxAtStartPar
For each JULES variable specified in {\hyperref[\detokenize{namelists/ancillaries.nml:JULES_SOIL_PROPS::var}]{\sphinxcrossref{\sphinxcode{\sphinxupquote{var}}}}}, this indicates if it should be read from the specified file or whether a constant value is to be used.
\begin{description}
\sphinxlineitem{TRUE}
\sphinxAtStartPar
The variable will be read from the file.

\sphinxlineitem{FALSE}
\sphinxAtStartPar
The variable will be set to a constant value everywhere using {\hyperref[\detokenize{namelists/ancillaries.nml:JULES_SOIL_PROPS::const_val}]{\sphinxcrossref{\sphinxcode{\sphinxupquote{const\_val}}}}} below.

\end{description}

\end{fulllineitems}

\index{var\_name (in namelist JULES\_SOIL\_PROPS)@\spxentry{var\_name}\spxextra{in namelist JULES\_SOIL\_PROPS}|spxpagem}

\begin{fulllineitems}
\phantomsection\label{\detokenize{namelists/ancillaries.nml:JULES_SOIL_PROPS::var_name}}
\pysigstartsignatures
\pysigline{\sphinxcode{\sphinxupquote{JULES\_SOIL\_PROPS::}}\sphinxbfcode{\sphinxupquote{var\_name}}}
\pysigstopsignatures\begin{quote}\begin{description}
\sphinxlineitem{Type}
\sphinxAtStartPar
character(nvars)

\sphinxlineitem{Default}
\sphinxAtStartPar
‘’ (empty string)

\end{description}\end{quote}

\sphinxAtStartPar
For each JULES variable specified in {\hyperref[\detokenize{namelists/ancillaries.nml:JULES_SOIL_PROPS::var}]{\sphinxcrossref{\sphinxcode{\sphinxupquote{var}}}}} where {\hyperref[\detokenize{namelists/ancillaries.nml:JULES_SOIL_PROPS::use_file}]{\sphinxcrossref{\sphinxcode{\sphinxupquote{use\_file}}}}} = TRUE, this is the name of the variable in the file containing the data.

\sphinxAtStartPar
If the empty string (the default) is given for any variable, then the corresponding value from {\hyperref[\detokenize{namelists/ancillaries.nml:JULES_SOIL_PROPS::var}]{\sphinxcrossref{\sphinxcode{\sphinxupquote{var}}}}} is used instead.

\sphinxAtStartPar
This is not used for variables where {\hyperref[\detokenize{namelists/ancillaries.nml:JULES_SOIL_PROPS::use_file}]{\sphinxcrossref{\sphinxcode{\sphinxupquote{use\_file}}}}} = FALSE, but a placeholder must still be given in that case.

\begin{sphinxadmonition}{note}{Note:}
\sphinxAtStartPar
For ASCII files, this is not used \sphinxhyphen{} only the order in the file matters, as described above.
\end{sphinxadmonition}

\end{fulllineitems}

\index{tpl\_name (in namelist JULES\_SOIL\_PROPS)@\spxentry{tpl\_name}\spxextra{in namelist JULES\_SOIL\_PROPS}|spxpagem}

\begin{fulllineitems}
\phantomsection\label{\detokenize{namelists/ancillaries.nml:JULES_SOIL_PROPS::tpl_name}}
\pysigstartsignatures
\pysigline{\sphinxcode{\sphinxupquote{JULES\_SOIL\_PROPS::}}\sphinxbfcode{\sphinxupquote{tpl\_name}}}
\pysigstopsignatures\begin{quote}\begin{description}
\sphinxlineitem{Type}
\sphinxAtStartPar
character(nvars)

\sphinxlineitem{Default}
\sphinxAtStartPar
None

\end{description}\end{quote}

\sphinxAtStartPar
For each JULES variable specified in {\hyperref[\detokenize{namelists/ancillaries.nml:JULES_SOIL_PROPS::var}]{\sphinxcrossref{\sphinxcode{\sphinxupquote{var}}}}}, this is the string to substitute into the file name in place of the variable name substitution string.

\sphinxAtStartPar
If the file name does not use variable name templating, this is not used.

\end{fulllineitems}

\index{const\_val (in namelist JULES\_SOIL\_PROPS)@\spxentry{const\_val}\spxextra{in namelist JULES\_SOIL\_PROPS}|spxpagem}

\begin{fulllineitems}
\phantomsection\label{\detokenize{namelists/ancillaries.nml:JULES_SOIL_PROPS::const_val}}
\pysigstartsignatures
\pysigline{\sphinxcode{\sphinxupquote{JULES\_SOIL\_PROPS::}}\sphinxbfcode{\sphinxupquote{const\_val}}}
\pysigstopsignatures\begin{quote}\begin{description}
\sphinxlineitem{Type}
\sphinxAtStartPar
real(nvars)

\sphinxlineitem{Default}
\sphinxAtStartPar
None

\end{description}\end{quote}

\sphinxAtStartPar
For each JULES variable specified in {\hyperref[\detokenize{namelists/ancillaries.nml:JULES_SOIL_PROPS::var}]{\sphinxcrossref{\sphinxcode{\sphinxupquote{var}}}}} where {\hyperref[\detokenize{namelists/ancillaries.nml:JULES_SOIL_PROPS::use_file}]{\sphinxcrossref{\sphinxcode{\sphinxupquote{use\_file}}}}} = FALSE, this is a constant value that the variable will be set to at every point in every layer (overriding {\hyperref[\detokenize{namelists/ancillaries.nml:JULES_SOIL_PROPS::const_z}]{\sphinxcrossref{\sphinxcode{\sphinxupquote{const\_z}}}}} = FALSE).

\sphinxAtStartPar
This is not used for variables where {\hyperref[\detokenize{namelists/ancillaries.nml:JULES_SOIL_PROPS::use_file}]{\sphinxcrossref{\sphinxcode{\sphinxupquote{use\_file}}}}} = TRUE, but a placeholder must still be given in that case.

\end{fulllineitems}



\subsubsection{List of soil parameters}
\label{\detokenize{namelists/ancillaries.nml:list-of-soil-parameters}}\label{\detokenize{namelists/ancillaries.nml:list-of-soil-params}}
\sphinxAtStartPar
If {\hyperref[\detokenize{namelists/ancillaries.nml:JULES_SOIL_PROPS::const_z}]{\sphinxcrossref{\sphinxcode{\sphinxupquote{const\_z}}}}} = FALSE, variables read from file must have a single levels dimension. For most variables this dimension must be of size {\hyperref[\detokenize{namelists/jules_soil.nml:JULES_SOIL::sm_levels}]{\sphinxcrossref{\sphinxcode{\sphinxupquote{sm\_levels}}}}} and called {\hyperref[\detokenize{namelists/model_grid.nml:JULES_INPUT_GRID::soil_dim_name}]{\sphinxcrossref{\sphinxcode{\sphinxupquote{soil\_dim\_name}}}}}; exceptions to this rule are indicated in the table below.

\sphinxAtStartPar
If {\hyperref[\detokenize{namelists/ancillaries.nml:JULES_SOIL_PROPS::const_z}]{\sphinxcrossref{\sphinxcode{\sphinxupquote{const\_z}}}}} = TRUE, variables read from file must have no levels dimensions.

\sphinxAtStartPar
If soil tiling is selected ({\hyperref[\detokenize{namelists/jules_soil.nml:JULES_SOIL::l_tile_soil}]{\sphinxcrossref{\sphinxcode{\sphinxupquote{l\_tile\_soil}}}}} = TRUE), ancillary fields can be specified for each soil tile ({\hyperref[\detokenize{namelists/jules_soil.nml:JULES_SOIL::l_broadcast_ancils}]{\sphinxcrossref{\sphinxcode{\sphinxupquote{l\_broadcast\_ancils}}}}} = FALSE), or values can be read for one soil tile and copied to all tiles ({\hyperref[\detokenize{namelists/jules_soil.nml:JULES_SOIL::l_broadcast_ancils}]{\sphinxcrossref{\sphinxcode{\sphinxupquote{l\_broadcast\_ancils}}}}} = TRUE).

\sphinxAtStartPar
In all cases, the variables must have no time dimension.


\begin{savenotes}\sphinxattablestart
\centering
\begin{tabulary}{\linewidth}[t]{|p{2cm}|p{9cm}|p{3.5cm}|}
\hline
\sphinxstyletheadfamily 
\sphinxAtStartPar
Name
&\sphinxstyletheadfamily 
\sphinxAtStartPar
Description
&\sphinxstyletheadfamily 
\sphinxAtStartPar
Levels name
\\
\hline
\sphinxAtStartPar
\sphinxcode{\sphinxupquote{albsoil}}
&
\sphinxAtStartPar
Soil albedo. A single (averaged) waveband is used.
&
\sphinxAtStartPar
None
\\
\hline
\sphinxAtStartPar
\sphinxcode{\sphinxupquote{b}}
&\begin{description}
\sphinxlineitem{Exponent in soil hydraulic characteristics.}
\sphinxAtStartPar
n.b. Related to the Brooks \& Corey parameter lambda by b=1/lambda
and to the van Genuchten\sphinxhyphen{}Mualem parameter n by b=1/(n\sphinxhyphen{}1)

\end{description}
&
\sphinxAtStartPar
{\hyperref[\detokenize{namelists/model_grid.nml:JULES_INPUT_GRID::soil_dim_name}]{\sphinxcrossref{\sphinxcode{\sphinxupquote{soil\_dim\_name}}}}}
\\
\hline
\sphinxAtStartPar
\sphinxcode{\sphinxupquote{hcap}}
&
\sphinxAtStartPar
Dry heat capacity (J m$^{\text{\sphinxhyphen{}3}}$ K$^{\text{\sphinxhyphen{}1}}$).
&
\sphinxAtStartPar
{\hyperref[\detokenize{namelists/model_grid.nml:JULES_INPUT_GRID::soil_dim_name}]{\sphinxcrossref{\sphinxcode{\sphinxupquote{soil\_dim\_name}}}}}
\\
\hline
\sphinxAtStartPar
\sphinxcode{\sphinxupquote{hcon}}
&
\sphinxAtStartPar
Dry thermal conductivity (W m$^{\text{\sphinxhyphen{}1}}$ K$^{\text{\sphinxhyphen{}1}}$).
&
\sphinxAtStartPar
{\hyperref[\detokenize{namelists/model_grid.nml:JULES_INPUT_GRID::soil_dim_name}]{\sphinxcrossref{\sphinxcode{\sphinxupquote{soil\_dim\_name}}}}}
\\
\hline
\sphinxAtStartPar
\sphinxcode{\sphinxupquote{satcon}}
&
\sphinxAtStartPar
Hydraulic conductivity at saturation (kg m$^{\text{\sphinxhyphen{}2}}$ s$^{\text{\sphinxhyphen{}1}}$).
&
\sphinxAtStartPar
{\hyperref[\detokenize{namelists/model_grid.nml:JULES_INPUT_GRID::soil_dim_name}]{\sphinxcrossref{\sphinxcode{\sphinxupquote{soil\_dim\_name}}}}}
\\
\hline
\sphinxAtStartPar
\sphinxcode{\sphinxupquote{sathh}}
&
\sphinxAtStartPar
If {\hyperref[\detokenize{namelists/jules_soil.nml:JULES_SOIL::l_vg_soil}]{\sphinxcrossref{\sphinxcode{\sphinxupquote{l\_vg\_soil}}}}} = TRUE (i.e. using van Genuchten model),
\sphinxcode{\sphinxupquote{sathh}} = 1 / \sphinxcode{\sphinxupquote{alpha}}, where \sphinxcode{\sphinxupquote{alpha}} (m$^{\text{\sphinxhyphen{}1}}$) is a parameter of the van
Genuchten model.

\sphinxAtStartPar
If {\hyperref[\detokenize{namelists/jules_soil.nml:JULES_SOIL::l_vg_soil}]{\sphinxcrossref{\sphinxcode{\sphinxupquote{l\_vg\_soil}}}}} = FALSE (using Brooks and Corey model), \sphinxcode{\sphinxupquote{sathh}}
is the soil matric suction at saturation (in pressure head units, m), i.e. the
absolute value of the soil matric potential at saturation.
&
\sphinxAtStartPar
{\hyperref[\detokenize{namelists/model_grid.nml:JULES_INPUT_GRID::soil_dim_name}]{\sphinxcrossref{\sphinxcode{\sphinxupquote{soil\_dim\_name}}}}}
\\
\hline
\sphinxAtStartPar
\sphinxcode{\sphinxupquote{sm\_crit}}
&
\sphinxAtStartPar
Volumetric soil moisture content at the critical point (m$^{\text{3}}$ water per
m$^{\text{3}}$ soil). If {\hyperref[\detokenize{namelists/jules_vegetation.nml:JULES_VEGETATION::l_use_pft_psi}]{\sphinxcrossref{\sphinxcode{\sphinxupquote{l\_use\_pft\_psi}}}}} = F,
the point at which soil moisture stress starts to restrict
transpiration is a function of \sphinxcode{\sphinxupquote{sm\_crit}}, \sphinxcode{\sphinxupquote{sm\_sat}} and the pft\sphinxhyphen{}dependent
parameter {\hyperref[\detokenize{namelists/pft_params.nml:JULES_PFTPARM::fsmc_p0_io}]{\sphinxcrossref{\sphinxcode{\sphinxupquote{fsmc\_p0\_io}}}}} .

\sphinxAtStartPar
\sphinxcode{\sphinxupquote{sm\_crit}} is also used to calculate the surface conductance of bare soil.
&
\sphinxAtStartPar
{\hyperref[\detokenize{namelists/model_grid.nml:JULES_INPUT_GRID::soil_dim_name}]{\sphinxcrossref{\sphinxcode{\sphinxupquote{soil\_dim\_name}}}}}
\\
\hline
\sphinxAtStartPar
\sphinxcode{\sphinxupquote{sm\_sat}}
&
\sphinxAtStartPar
Volumetric soil moisture content at saturation (m$^{\text{3}}$ water per m$^{\text{3}}$
soil).

\begin{sphinxadmonition}{note}{Note:}
\sphinxAtStartPar
This field is used to distinguish between soil points and land ice points.

\sphinxAtStartPar
\sphinxcode{\sphinxupquote{sm\_sat \textgreater{} 0}} indicates a soil point.
\end{sphinxadmonition}
&
\sphinxAtStartPar
{\hyperref[\detokenize{namelists/model_grid.nml:JULES_INPUT_GRID::soil_dim_name}]{\sphinxcrossref{\sphinxcode{\sphinxupquote{soil\_dim\_name}}}}}
\\
\hline
\sphinxAtStartPar
\sphinxcode{\sphinxupquote{sm\_wilt}}
&
\sphinxAtStartPar
Volumetric soil moisture content at the wilting point (m$^{\text{3}}$ water
per m$^{\text{3}}$ soil). If {\hyperref[\detokenize{namelists/jules_vegetation.nml:JULES_VEGETATION::l_use_pft_psi}]{\sphinxcrossref{\sphinxcode{\sphinxupquote{l\_use\_pft\_psi}}}}} = F,
\sphinxcode{\sphinxupquote{sm\_wilt}} is the limit where soil moisture stress
completely prevents transpiration.

\sphinxAtStartPar
\sphinxcode{\sphinxupquote{sm\_wilt}} is also used to calculate soil respiration.
&
\sphinxAtStartPar
{\hyperref[\detokenize{namelists/model_grid.nml:JULES_INPUT_GRID::soil_dim_name}]{\sphinxcrossref{\sphinxcode{\sphinxupquote{soil\_dim\_name}}}}}
\\
\hline
\sphinxAtStartPar
\sphinxcode{\sphinxupquote{clay}}
&
\sphinxAtStartPar
Soil clay content (g clay per g soil). Only required for the 4\sphinxhyphen{}pool and ECOSSE soil
carbon models ({\hyperref[\detokenize{namelists/jules_soil_biogeochem.nml:JULES_SOIL_BIOGEOCHEM::soil_bgc_model}]{\sphinxcrossref{\sphinxcode{\sphinxupquote{soil\_bgc\_model}}}}} = 2 or 3).

\begin{sphinxadmonition}{note}{Note:}
\sphinxAtStartPar
To allow backwards compatibility when using the 4\sphinxhyphen{}pool model
({\hyperref[\detokenize{namelists/jules_soil_biogeochem.nml:JULES_SOIL_BIOGEOCHEM::soil_bgc_model}]{\sphinxcrossref{\sphinxcode{\sphinxupquote{soil\_bgc\_model}}}}} = 2), if the clay content is
not available it is set to 0.0 in the code.

\sphinxAtStartPar
However, this is wrong \sphinxhyphen{} if it is not available it should be set to 0.23.
\end{sphinxadmonition}
&
\sphinxAtStartPar
{\hyperref[\detokenize{namelists/model_grid.nml:JULES_INPUT_GRID::sclayer_dim_name}]{\sphinxcrossref{\sphinxcode{\sphinxupquote{sclayer\_dim\_name}}}}}
\\
\hline
\sphinxAtStartPar
\sphinxcode{\sphinxupquote{soil\_ph}}
&
\sphinxAtStartPar
Soil pH. Only required for the ECOSSE soil carbon model
({\hyperref[\detokenize{namelists/jules_soil_biogeochem.nml:JULES_SOIL_BIOGEOCHEM::soil_bgc_model}]{\sphinxcrossref{\sphinxcode{\sphinxupquote{soil\_bgc\_model}}}}} = 3).
&
\sphinxAtStartPar
{\hyperref[\detokenize{namelists/model_grid.nml:JULES_INPUT_GRID::sclayer_dim_name}]{\sphinxcrossref{\sphinxcode{\sphinxupquote{sclayer\_dim\_name}}}}}
\\
\hline
\end{tabulary}
\par
\sphinxattableend\end{savenotes}


\subsection{\sphinxstyleliteralintitle{\sphinxupquote{JULES\_TOP}} namelist members}
\label{\detokenize{namelists/ancillaries.nml:namelist-JULES_TOP}}\label{\detokenize{namelists/ancillaries.nml:jules-top-namelist-members}}\index{JULES\_TOP (namelist)@\spxentry{JULES\_TOP}\spxextra{namelist}|spxpagem}
\sphinxAtStartPar
This namelist reads spatially varying parameter values for the TOPMODEL\sphinxhyphen{}type parameterisation of runoff. The values are only used if {\hyperref[\detokenize{namelists/jules_hydrology.nml:JULES_HYDROLOGY::l_top}]{\sphinxcrossref{\sphinxcode{\sphinxupquote{l\_top}}}}} = TRUE. The description below is very brief. For further details, see the references under {\hyperref[\detokenize{namelists/jules_hydrology.nml:JULES_HYDROLOGY::l_top}]{\sphinxcrossref{\sphinxcode{\sphinxupquote{l\_top}}}}}.
\index{read\_from\_dump (in namelist JULES\_TOP)@\spxentry{read\_from\_dump}\spxextra{in namelist JULES\_TOP}|spxpagem}

\begin{fulllineitems}
\phantomsection\label{\detokenize{namelists/ancillaries.nml:JULES_TOP::read_from_dump}}
\pysigstartsignatures
\pysigline{\sphinxcode{\sphinxupquote{JULES\_TOP::}}\sphinxbfcode{\sphinxupquote{read\_from\_dump}}}
\pysigstopsignatures\begin{quote}\begin{description}
\sphinxlineitem{Type}
\sphinxAtStartPar
logical

\sphinxlineitem{Default}
\sphinxAtStartPar
F

\end{description}\end{quote}
\begin{description}
\sphinxlineitem{TRUE}
\sphinxAtStartPar
Populate variables associated with this namelist from the dump file. All other namelist members are ignored.

\sphinxlineitem{FALSE}
\sphinxAtStartPar
Use the other namelist members to determine how to populate variables.

\end{description}

\end{fulllineitems}

\index{file (in namelist JULES\_TOP)@\spxentry{file}\spxextra{in namelist JULES\_TOP}|spxpagem}

\begin{fulllineitems}
\phantomsection\label{\detokenize{namelists/ancillaries.nml:JULES_TOP::file}}
\pysigstartsignatures
\pysigline{\sphinxcode{\sphinxupquote{JULES\_TOP::}}\sphinxbfcode{\sphinxupquote{file}}}
\pysigstopsignatures\begin{quote}\begin{description}
\sphinxlineitem{Type}
\sphinxAtStartPar
character

\sphinxlineitem{Default}
\sphinxAtStartPar
None

\end{description}\end{quote}

\sphinxAtStartPar
The file to read TOPMODEL properties from.

\sphinxAtStartPar
If {\hyperref[\detokenize{namelists/ancillaries.nml:JULES_TOP::use_file}]{\sphinxcrossref{\sphinxcode{\sphinxupquote{use\_file}}}}} is FALSE for every variable, this will not be used.

\sphinxAtStartPar
This file name can use {\hyperref[\detokenize{input/file-name-templating::doc}]{\sphinxcrossref{\DUrole{doc}{variable name templating}}}}.

\end{fulllineitems}

\index{nvars (in namelist JULES\_TOP)@\spxentry{nvars}\spxextra{in namelist JULES\_TOP}|spxpagem}

\begin{fulllineitems}
\phantomsection\label{\detokenize{namelists/ancillaries.nml:JULES_TOP::nvars}}
\pysigstartsignatures
\pysigline{\sphinxcode{\sphinxupquote{JULES\_TOP::}}\sphinxbfcode{\sphinxupquote{nvars}}}
\pysigstopsignatures\begin{quote}\begin{description}
\sphinxlineitem{Type}
\sphinxAtStartPar
integer

\sphinxlineitem{Permitted}
\sphinxAtStartPar
\textgreater{}= 0

\sphinxlineitem{Default}
\sphinxAtStartPar
0

\end{description}\end{quote}

\sphinxAtStartPar
The number of TOPMODEL property variables that will be provided. At present, all variables are required for runs using TOPMODEL (see {\hyperref[\detokenize{namelists/ancillaries.nml:list-of-topmodel-params}]{\sphinxcrossref{\DUrole{std,std-ref}{List of TOPMODEL parameters}}}}).

\end{fulllineitems}

\index{var (in namelist JULES\_TOP)@\spxentry{var}\spxextra{in namelist JULES\_TOP}|spxpagem}

\begin{fulllineitems}
\phantomsection\label{\detokenize{namelists/ancillaries.nml:JULES_TOP::var}}
\pysigstartsignatures
\pysigline{\sphinxcode{\sphinxupquote{JULES\_TOP::}}\sphinxbfcode{\sphinxupquote{var}}}
\pysigstopsignatures\begin{quote}\begin{description}
\sphinxlineitem{Type}
\sphinxAtStartPar
character(nvars)

\sphinxlineitem{Default}
\sphinxAtStartPar
None

\end{description}\end{quote}

\sphinxAtStartPar
List of TOPMODEL variable names as recognised by JULES (see {\hyperref[\detokenize{namelists/ancillaries.nml:list-of-topmodel-params}]{\sphinxcrossref{\DUrole{std,std-ref}{List of TOPMODEL parameters}}}}). Names are case sensitive.

\begin{sphinxadmonition}{note}{Note:}
\sphinxAtStartPar
For ASCII files, variable names must be in the order they appear in the file.
\end{sphinxadmonition}

\end{fulllineitems}

\index{use\_file (in namelist JULES\_TOP)@\spxentry{use\_file}\spxextra{in namelist JULES\_TOP}|spxpagem}

\begin{fulllineitems}
\phantomsection\label{\detokenize{namelists/ancillaries.nml:JULES_TOP::use_file}}
\pysigstartsignatures
\pysigline{\sphinxcode{\sphinxupquote{JULES\_TOP::}}\sphinxbfcode{\sphinxupquote{use\_file}}}
\pysigstopsignatures\begin{quote}\begin{description}
\sphinxlineitem{Type}
\sphinxAtStartPar
logical(nvars)

\sphinxlineitem{Default}
\sphinxAtStartPar
T

\end{description}\end{quote}

\sphinxAtStartPar
For each JULES variable specified in {\hyperref[\detokenize{namelists/ancillaries.nml:JULES_TOP::var}]{\sphinxcrossref{\sphinxcode{\sphinxupquote{var}}}}}, this indicates if it should be read from the specified file or whether a constant value is to be used.
\begin{description}
\sphinxlineitem{TRUE}
\sphinxAtStartPar
The variable will be read from the file.

\sphinxlineitem{FALSE}
\sphinxAtStartPar
The variable will be set to a constant value everywhere using {\hyperref[\detokenize{namelists/ancillaries.nml:JULES_TOP::const_val}]{\sphinxcrossref{\sphinxcode{\sphinxupquote{const\_val}}}}} below.

\end{description}

\end{fulllineitems}

\index{var\_name (in namelist JULES\_TOP)@\spxentry{var\_name}\spxextra{in namelist JULES\_TOP}|spxpagem}

\begin{fulllineitems}
\phantomsection\label{\detokenize{namelists/ancillaries.nml:JULES_TOP::var_name}}
\pysigstartsignatures
\pysigline{\sphinxcode{\sphinxupquote{JULES\_TOP::}}\sphinxbfcode{\sphinxupquote{var\_name}}}
\pysigstopsignatures\begin{quote}\begin{description}
\sphinxlineitem{Type}
\sphinxAtStartPar
character(nvars)

\sphinxlineitem{Default}
\sphinxAtStartPar
‘’ (empty string)

\end{description}\end{quote}

\sphinxAtStartPar
For each JULES variable specified in {\hyperref[\detokenize{namelists/ancillaries.nml:JULES_TOP::var}]{\sphinxcrossref{\sphinxcode{\sphinxupquote{var}}}}} where {\hyperref[\detokenize{namelists/ancillaries.nml:JULES_TOP::use_file}]{\sphinxcrossref{\sphinxcode{\sphinxupquote{use\_file}}}}} = TRUE, this is the name of the variable in the file containing the data.

\sphinxAtStartPar
If the empty string (the default) is given for any variable, then the corresponding value from {\hyperref[\detokenize{namelists/ancillaries.nml:JULES_TOP::var}]{\sphinxcrossref{\sphinxcode{\sphinxupquote{var}}}}} is used instead.

\sphinxAtStartPar
This is not used for variables where {\hyperref[\detokenize{namelists/ancillaries.nml:JULES_TOP::use_file}]{\sphinxcrossref{\sphinxcode{\sphinxupquote{use\_file}}}}} = FALSE, but a placeholder must still be given in that case.

\begin{sphinxadmonition}{note}{Note:}
\sphinxAtStartPar
For ASCII files, this is not used \sphinxhyphen{} only the order in the file matters, as described above.
\end{sphinxadmonition}

\end{fulllineitems}

\index{tpl\_name (in namelist JULES\_TOP)@\spxentry{tpl\_name}\spxextra{in namelist JULES\_TOP}|spxpagem}

\begin{fulllineitems}
\phantomsection\label{\detokenize{namelists/ancillaries.nml:JULES_TOP::tpl_name}}
\pysigstartsignatures
\pysigline{\sphinxcode{\sphinxupquote{JULES\_TOP::}}\sphinxbfcode{\sphinxupquote{tpl\_name}}}
\pysigstopsignatures\begin{quote}\begin{description}
\sphinxlineitem{Type}
\sphinxAtStartPar
character(nvars)

\sphinxlineitem{Default}
\sphinxAtStartPar
None

\end{description}\end{quote}

\sphinxAtStartPar
For each JULES variable specified in {\hyperref[\detokenize{namelists/ancillaries.nml:JULES_TOP::var}]{\sphinxcrossref{\sphinxcode{\sphinxupquote{var}}}}}, this is the string to substitute into the file name in place of the variable name substitution string.

\sphinxAtStartPar
If the file name does not use variable name templating, this is not used.

\end{fulllineitems}

\index{const\_val (in namelist JULES\_TOP)@\spxentry{const\_val}\spxextra{in namelist JULES\_TOP}|spxpagem}

\begin{fulllineitems}
\phantomsection\label{\detokenize{namelists/ancillaries.nml:JULES_TOP::const_val}}
\pysigstartsignatures
\pysigline{\sphinxcode{\sphinxupquote{JULES\_TOP::}}\sphinxbfcode{\sphinxupquote{const\_val}}}
\pysigstopsignatures\begin{quote}\begin{description}
\sphinxlineitem{Type}
\sphinxAtStartPar
real(nvars)

\sphinxlineitem{Default}
\sphinxAtStartPar
None

\end{description}\end{quote}

\sphinxAtStartPar
For each JULES variable specified in {\hyperref[\detokenize{namelists/ancillaries.nml:JULES_TOP::var}]{\sphinxcrossref{\sphinxcode{\sphinxupquote{var}}}}} where {\hyperref[\detokenize{namelists/ancillaries.nml:JULES_TOP::use_file}]{\sphinxcrossref{\sphinxcode{\sphinxupquote{use\_file}}}}} = FALSE, this is a constant value that the variable will be set to at every point in every layer.

\sphinxAtStartPar
This is not used for variables where {\hyperref[\detokenize{namelists/ancillaries.nml:JULES_TOP::use_file}]{\sphinxcrossref{\sphinxcode{\sphinxupquote{use\_file}}}}} = TRUE, but a placeholder must still be given in that case.

\end{fulllineitems}



\subsubsection{List of TOPMODEL parameters}
\label{\detokenize{namelists/ancillaries.nml:list-of-topmodel-parameters}}\label{\detokenize{namelists/ancillaries.nml:list-of-topmodel-params}}
\sphinxAtStartPar
All of the TOPMODEL variables listed below are expected to have no levels dimensions and no time dimension.


\begin{savenotes}\sphinxattablestart
\centering
\begin{tabulary}{\linewidth}[t]{|p{2.5cm}|L|}
\hline
\sphinxstyletheadfamily 
\sphinxAtStartPar
Name
&\sphinxstyletheadfamily 
\sphinxAtStartPar
Description
\\
\hline
\sphinxAtStartPar
\sphinxcode{\sphinxupquote{fexp}}
&
\sphinxAtStartPar
Decay factor describing how the saturated hydraulic conductivity decreases with depth below the standard
soil column (m$^{\text{\sphinxhyphen{}1}}$).

\sphinxAtStartPar
Routinely set between 2 and 3 m$^{\text{\sphinxhyphen{}1}}$. Gedney \& Cox (2003, J Hydromet) used value 0.5 m$^{\text{\sphinxhyphen{}1}}$;
Niu \& Yang (2003, Global \& Planet. Change) suggested a global mean value of 2.0 m$^{\text{\sphinxhyphen{}1}}$.
\\
\hline
\sphinxAtStartPar
\sphinxcode{\sphinxupquote{ti\_mean}}
&
\sphinxAtStartPar
(Spatial, not temporal) mean value of the topographic index in each gridbox.
Value 5.99 is the global mean given in Marthews et al. (2015, HESS)
\\
\hline
\sphinxAtStartPar
\sphinxcode{\sphinxupquote{ti\_sig}}
&
\sphinxAtStartPar
(Spatial, not temporal) standard deviation of the topographic index in each gridbox.
Values \textless{}0.5 are updated to =0.5 internally to allow at least some variability
\\
\hline
\end{tabulary}
\par
\sphinxattableend\end{savenotes}


\subsection{\sphinxstyleliteralintitle{\sphinxupquote{JULES\_PDM}} namelist members}
\label{\detokenize{namelists/ancillaries.nml:namelist-JULES_PDM}}\label{\detokenize{namelists/ancillaries.nml:jules-pdm-namelist-members}}\index{JULES\_PDM (namelist)@\spxentry{JULES\_PDM}\spxextra{namelist}|spxpagem}
\sphinxAtStartPar
This namelist reads spatially varying parameter values for the PDM\sphinxhyphen{}type parameterisation of runoff. The values are only used if {\hyperref[\detokenize{namelists/jules_hydrology.nml:JULES_HYDROLOGY::l_pdm}]{\sphinxcrossref{\sphinxcode{\sphinxupquote{l\_pdm}}}}} = TRUE. The description below is very brief. For further details, see the references under {\hyperref[\detokenize{namelists/jules_hydrology.nml:JULES_HYDROLOGY::l_pdm}]{\sphinxcrossref{\sphinxcode{\sphinxupquote{l\_pdm}}}}}.
\index{file (in namelist JULES\_PDM)@\spxentry{file}\spxextra{in namelist JULES\_PDM}|spxpagem}

\begin{fulllineitems}
\phantomsection\label{\detokenize{namelists/ancillaries.nml:JULES_PDM::file}}
\pysigstartsignatures
\pysigline{\sphinxcode{\sphinxupquote{JULES\_PDM::}}\sphinxbfcode{\sphinxupquote{file}}}
\pysigstopsignatures\begin{quote}\begin{description}
\sphinxlineitem{Type}
\sphinxAtStartPar
character

\sphinxlineitem{Default}
\sphinxAtStartPar
None

\end{description}\end{quote}

\sphinxAtStartPar
The file to read PDM properties from.

\sphinxAtStartPar
If {\hyperref[\detokenize{namelists/ancillaries.nml:JULES_PDM::use_file}]{\sphinxcrossref{\sphinxcode{\sphinxupquote{use\_file}}}}} is FALSE for every variable, this will not be used.

\sphinxAtStartPar
This file name can use {\hyperref[\detokenize{input/file-name-templating::doc}]{\sphinxcrossref{\DUrole{doc}{variable name templating}}}}.

\end{fulllineitems}

\index{nvars (in namelist JULES\_PDM)@\spxentry{nvars}\spxextra{in namelist JULES\_PDM}|spxpagem}

\begin{fulllineitems}
\phantomsection\label{\detokenize{namelists/ancillaries.nml:JULES_PDM::nvars}}
\pysigstartsignatures
\pysigline{\sphinxcode{\sphinxupquote{JULES\_PDM::}}\sphinxbfcode{\sphinxupquote{nvars}}}
\pysigstopsignatures\begin{quote}\begin{description}
\sphinxlineitem{Type}
\sphinxAtStartPar
integer

\sphinxlineitem{Permitted}
\sphinxAtStartPar
\textgreater{}= 0

\sphinxlineitem{Default}
\sphinxAtStartPar
0

\end{description}\end{quote}

\sphinxAtStartPar
The number of PDM property variables that will be provided (see {\hyperref[\detokenize{namelists/ancillaries.nml:list-of-pdm-params}]{\sphinxcrossref{\DUrole{std,std-ref}{List of PDM parameters}}}}). At present, only the topographic slope can be provided.

\end{fulllineitems}

\index{var (in namelist JULES\_PDM)@\spxentry{var}\spxextra{in namelist JULES\_PDM}|spxpagem}

\begin{fulllineitems}
\phantomsection\label{\detokenize{namelists/ancillaries.nml:JULES_PDM::var}}
\pysigstartsignatures
\pysigline{\sphinxcode{\sphinxupquote{JULES\_PDM::}}\sphinxbfcode{\sphinxupquote{var}}}
\pysigstopsignatures\begin{quote}\begin{description}
\sphinxlineitem{Type}
\sphinxAtStartPar
character(nvars)

\sphinxlineitem{Default}
\sphinxAtStartPar
None

\end{description}\end{quote}

\sphinxAtStartPar
List of PDM variable names as recognised by JULES (see {\hyperref[\detokenize{namelists/ancillaries.nml:list-of-pdm-params}]{\sphinxcrossref{\DUrole{std,std-ref}{List of PDM parameters}}}}). Names are case sensitive.

\begin{sphinxadmonition}{note}{Note:}
\sphinxAtStartPar
For ASCII files, variable names must be in the order they appear in the file.
\end{sphinxadmonition}

\end{fulllineitems}

\index{use\_file (in namelist JULES\_PDM)@\spxentry{use\_file}\spxextra{in namelist JULES\_PDM}|spxpagem}

\begin{fulllineitems}
\phantomsection\label{\detokenize{namelists/ancillaries.nml:JULES_PDM::use_file}}
\pysigstartsignatures
\pysigline{\sphinxcode{\sphinxupquote{JULES\_PDM::}}\sphinxbfcode{\sphinxupquote{use\_file}}}
\pysigstopsignatures\begin{quote}\begin{description}
\sphinxlineitem{Type}
\sphinxAtStartPar
logical(nvars)

\sphinxlineitem{Default}
\sphinxAtStartPar
T

\end{description}\end{quote}

\sphinxAtStartPar
For each JULES variable specified in {\hyperref[\detokenize{namelists/ancillaries.nml:JULES_PDM::var}]{\sphinxcrossref{\sphinxcode{\sphinxupquote{var}}}}}, this indicates if it should be read from the specified file or whether a constant value is to be used.
\begin{description}
\sphinxlineitem{TRUE}
\sphinxAtStartPar
The variable will be read from the file.

\sphinxlineitem{FALSE}
\sphinxAtStartPar
The variable will be set to a constant value everywhere using {\hyperref[\detokenize{namelists/ancillaries.nml:JULES_PDM::const_val}]{\sphinxcrossref{\sphinxcode{\sphinxupquote{const\_val}}}}} below.

\end{description}

\end{fulllineitems}

\index{var\_name (in namelist JULES\_PDM)@\spxentry{var\_name}\spxextra{in namelist JULES\_PDM}|spxpagem}

\begin{fulllineitems}
\phantomsection\label{\detokenize{namelists/ancillaries.nml:JULES_PDM::var_name}}
\pysigstartsignatures
\pysigline{\sphinxcode{\sphinxupquote{JULES\_PDM::}}\sphinxbfcode{\sphinxupquote{var\_name}}}
\pysigstopsignatures\begin{quote}\begin{description}
\sphinxlineitem{Type}
\sphinxAtStartPar
character(nvars)

\sphinxlineitem{Default}
\sphinxAtStartPar
None

\end{description}\end{quote}

\sphinxAtStartPar
For each JULES variable specified in {\hyperref[\detokenize{namelists/ancillaries.nml:JULES_PDM::var}]{\sphinxcrossref{\sphinxcode{\sphinxupquote{var}}}}} where {\hyperref[\detokenize{namelists/ancillaries.nml:JULES_PDM::use_file}]{\sphinxcrossref{\sphinxcode{\sphinxupquote{use\_file}}}}} = TRUE, this is the name of the variable in the file containing the data.

\sphinxAtStartPar
This is not used for variables where {\hyperref[\detokenize{namelists/ancillaries.nml:JULES_PDM::use_file}]{\sphinxcrossref{\sphinxcode{\sphinxupquote{use\_file}}}}} = FALSE, but a placeholder must still be given.

\begin{sphinxadmonition}{note}{Note:}
\sphinxAtStartPar
For ASCII files, this is not used \sphinxhyphen{} only the order in the file matters, as described above.
\end{sphinxadmonition}

\end{fulllineitems}

\index{tpl\_name (in namelist JULES\_PDM)@\spxentry{tpl\_name}\spxextra{in namelist JULES\_PDM}|spxpagem}

\begin{fulllineitems}
\phantomsection\label{\detokenize{namelists/ancillaries.nml:JULES_PDM::tpl_name}}
\pysigstartsignatures
\pysigline{\sphinxcode{\sphinxupquote{JULES\_PDM::}}\sphinxbfcode{\sphinxupquote{tpl\_name}}}
\pysigstopsignatures\begin{quote}\begin{description}
\sphinxlineitem{Type}
\sphinxAtStartPar
character(nvars)

\sphinxlineitem{Default}
\sphinxAtStartPar
None

\end{description}\end{quote}

\sphinxAtStartPar
For each JULES variable specified in {\hyperref[\detokenize{namelists/ancillaries.nml:JULES_PDM::var}]{\sphinxcrossref{\sphinxcode{\sphinxupquote{var}}}}}, this is the string to substitute into the file name in place of the variable name substitution string.

\sphinxAtStartPar
If the file name does not use variable name templating, this is not used.

\end{fulllineitems}

\index{const\_val (in namelist JULES\_PDM)@\spxentry{const\_val}\spxextra{in namelist JULES\_PDM}|spxpagem}

\begin{fulllineitems}
\phantomsection\label{\detokenize{namelists/ancillaries.nml:JULES_PDM::const_val}}
\pysigstartsignatures
\pysigline{\sphinxcode{\sphinxupquote{JULES\_PDM::}}\sphinxbfcode{\sphinxupquote{const\_val}}}
\pysigstopsignatures\begin{quote}\begin{description}
\sphinxlineitem{Type}
\sphinxAtStartPar
real(nvars)

\sphinxlineitem{Default}
\sphinxAtStartPar
None

\end{description}\end{quote}

\sphinxAtStartPar
For each JULES variable specified in {\hyperref[\detokenize{namelists/ancillaries.nml:JULES_PDM::var}]{\sphinxcrossref{\sphinxcode{\sphinxupquote{var}}}}} where {\hyperref[\detokenize{namelists/ancillaries.nml:JULES_PDM::use_file}]{\sphinxcrossref{\sphinxcode{\sphinxupquote{use\_file}}}}} = FALSE, this is a constant value that the variable will be set to at every point in every layer.
make html
This is not used for variables where {\hyperref[\detokenize{namelists/ancillaries.nml:JULES_PDM::use_file}]{\sphinxcrossref{\sphinxcode{\sphinxupquote{use\_file}}}}} = TRUE, but a placeholder must still be given.

\end{fulllineitems}



\subsubsection{List of PDM parameters}
\label{\detokenize{namelists/ancillaries.nml:list-of-pdm-parameters}}\label{\detokenize{namelists/ancillaries.nml:list-of-pdm-params}}
\sphinxAtStartPar
All of the PDM variables listed below are expected to have no levels dimensions and no time dimension.


\begin{savenotes}\sphinxattablestart
\centering
\begin{tabulary}{\linewidth}[t]{|p{2.5cm}|L|}
\hline
\sphinxstyletheadfamily 
\sphinxAtStartPar
Name
&\sphinxstyletheadfamily 
\sphinxAtStartPar
Description
\\
\hline
\sphinxAtStartPar
\sphinxcode{\sphinxupquote{slope}}
&
\sphinxAtStartPar
Mean value of the topographic slope in the gridbox (deg).
\\
\hline
\end{tabulary}
\par
\sphinxattableend\end{savenotes}


\subsection{\sphinxstyleliteralintitle{\sphinxupquote{JULES\_AGRIC}} namelist members}
\label{\detokenize{namelists/ancillaries.nml:namelist-JULES_AGRIC}}\label{\detokenize{namelists/ancillaries.nml:jules-agric-namelist-members}}\index{JULES\_AGRIC (namelist)@\spxentry{JULES\_AGRIC}\spxextra{namelist}|spxpagem}
\sphinxAtStartPar
If the TRIFFID vegetation model is used, the fractional area of agricultural land in each gridbox is specified using this namelist. Otherwise, the values in this namelist are not used.
\index{read\_from\_dump (in namelist JULES\_AGRIC)@\spxentry{read\_from\_dump}\spxextra{in namelist JULES\_AGRIC}|spxpagem}

\begin{fulllineitems}
\phantomsection\label{\detokenize{namelists/ancillaries.nml:JULES_AGRIC::read_from_dump}}
\pysigstartsignatures
\pysigline{\sphinxcode{\sphinxupquote{JULES\_AGRIC::}}\sphinxbfcode{\sphinxupquote{read\_from\_dump}}}
\pysigstopsignatures\begin{quote}\begin{description}
\sphinxlineitem{Type}
\sphinxAtStartPar
logical

\sphinxlineitem{Default}
\sphinxAtStartPar
F

\end{description}\end{quote}
\begin{description}
\sphinxlineitem{TRUE}
\sphinxAtStartPar
Populate frac\_agr, frac\_past, and frac\_biocrop from the dump file. All other namelist members are ignored.

\sphinxlineitem{FALSE}
\sphinxAtStartPar
Use the other namelist members to determine how to populate variables.

\end{description}

\end{fulllineitems}

\index{zero\_agric (in namelist JULES\_AGRIC)@\spxentry{zero\_agric}\spxextra{in namelist JULES\_AGRIC}|spxpagem}

\begin{fulllineitems}
\phantomsection\label{\detokenize{namelists/ancillaries.nml:JULES_AGRIC::zero_agric}}
\pysigstartsignatures
\pysigline{\sphinxcode{\sphinxupquote{JULES\_AGRIC::}}\sphinxbfcode{\sphinxupquote{zero\_agric}}}
\pysigstopsignatures\begin{quote}\begin{description}
\sphinxlineitem{Type}
\sphinxAtStartPar
logical

\sphinxlineitem{Default}
\sphinxAtStartPar
T

\end{description}\end{quote}

\sphinxAtStartPar
Switch used to simplify the initialisation of agricultural fraction.
\begin{description}
\sphinxlineitem{TRUE}
\sphinxAtStartPar
Set agricultural fraction at all points to zero.

\sphinxlineitem{FALSE}
\sphinxAtStartPar
Set agricultural fraction using specified data.

\end{description}

\end{fulllineitems}


\begin{sphinxadmonition}{note}{Used if \sphinxstyleliteralintitle{\sphinxupquote{zero\_agric}} = FALSE and the input grid consists of a single location}
\index{frac\_agr (in namelist JULES\_AGRIC)@\spxentry{frac\_agr}\spxextra{in namelist JULES\_AGRIC}|spxpagem}

\begin{fulllineitems}
\phantomsection\label{\detokenize{namelists/ancillaries.nml:JULES_AGRIC::frac_agr}}
\pysigstartsignatures
\pysigline{\sphinxcode{\sphinxupquote{JULES\_AGRIC::}}\sphinxbfcode{\sphinxupquote{frac\_agr}}}
\pysigstopsignatures\begin{quote}\begin{description}
\sphinxlineitem{Type}
\sphinxAtStartPar
real

\sphinxlineitem{Default}
\sphinxAtStartPar
None

\end{description}\end{quote}

\sphinxAtStartPar
The agricultural fraction for the single location.

\end{fulllineitems}

\end{sphinxadmonition}

\begin{sphinxadmonition}{note}{Used if \sphinxstyleliteralintitle{\sphinxupquote{zero\_agric}} = FALSE and the input grid consists of more than one location}
\index{file (in namelist JULES\_AGRIC)@\spxentry{file}\spxextra{in namelist JULES\_AGRIC}|spxpagem}

\begin{fulllineitems}
\phantomsection\label{\detokenize{namelists/ancillaries.nml:JULES_AGRIC::file}}
\pysigstartsignatures
\pysigline{\sphinxcode{\sphinxupquote{JULES\_AGRIC::}}\sphinxbfcode{\sphinxupquote{file}}}
\pysigstopsignatures\begin{quote}\begin{description}
\sphinxlineitem{Type}
\sphinxAtStartPar
character

\sphinxlineitem{Default}
\sphinxAtStartPar
None

\end{description}\end{quote}

\sphinxAtStartPar
The name of the file to read agricultural fraction data from.

\end{fulllineitems}

\index{agric\_name (in namelist JULES\_AGRIC)@\spxentry{agric\_name}\spxextra{in namelist JULES\_AGRIC}|spxpagem}

\begin{fulllineitems}
\phantomsection\label{\detokenize{namelists/ancillaries.nml:JULES_AGRIC::agric_name}}
\pysigstartsignatures
\pysigline{\sphinxcode{\sphinxupquote{JULES\_AGRIC::}}\sphinxbfcode{\sphinxupquote{agric\_name}}}
\pysigstopsignatures\begin{quote}\begin{description}
\sphinxlineitem{Type}
\sphinxAtStartPar
character

\sphinxlineitem{Default}
\sphinxAtStartPar
‘frac\_agr’

\end{description}\end{quote}

\sphinxAtStartPar
The name of the variable containing the agricultural fraction data.

\sphinxAtStartPar
In the file, the variable must have no levels dimensions and no time dimension.

\end{fulllineitems}

\end{sphinxadmonition}
\index{zero\_past (in namelist JULES\_AGRIC)@\spxentry{zero\_past}\spxextra{in namelist JULES\_AGRIC}|spxpagem}

\begin{fulllineitems}
\phantomsection\label{\detokenize{namelists/ancillaries.nml:JULES_AGRIC::zero_past}}
\pysigstartsignatures
\pysigline{\sphinxcode{\sphinxupquote{JULES\_AGRIC::}}\sphinxbfcode{\sphinxupquote{zero\_past}}}
\pysigstopsignatures\begin{quote}\begin{description}
\sphinxlineitem{Type}
\sphinxAtStartPar
logical

\sphinxlineitem{Default}
\sphinxAtStartPar
T

\end{description}\end{quote}

\sphinxAtStartPar
Switch used to simplify the initialisation of pasture fraction. Pasture fraction can only be used if {\hyperref[\detokenize{namelists/jules_vegetation.nml:JULES_VEGETATION::l_trif_crop}]{\sphinxcrossref{\sphinxcode{\sphinxupquote{l\_trif\_crop}}}}} is TRUE.
\begin{description}
\sphinxlineitem{TRUE}
\sphinxAtStartPar
Set pasture fraction at all points to zero.

\sphinxlineitem{FALSE}
\sphinxAtStartPar
Set pasture fraction using specified data.

\end{description}

\end{fulllineitems}


\begin{sphinxadmonition}{note}{Used if \sphinxstyleliteralintitle{\sphinxupquote{zero\_past}} = FALSE and the input grid consists of a single location}
\index{frac\_past (in namelist JULES\_AGRIC)@\spxentry{frac\_past}\spxextra{in namelist JULES\_AGRIC}|spxpagem}

\begin{fulllineitems}
\phantomsection\label{\detokenize{namelists/ancillaries.nml:JULES_AGRIC::frac_past}}
\pysigstartsignatures
\pysigline{\sphinxcode{\sphinxupquote{JULES\_AGRIC::}}\sphinxbfcode{\sphinxupquote{frac\_past}}}
\pysigstopsignatures\begin{quote}\begin{description}
\sphinxlineitem{Type}
\sphinxAtStartPar
real

\sphinxlineitem{Default}
\sphinxAtStartPar
None

\end{description}\end{quote}

\sphinxAtStartPar
The pasture fraction for the single location.

\end{fulllineitems}

\end{sphinxadmonition}

\begin{sphinxadmonition}{note}{Used if \sphinxstyleliteralintitle{\sphinxupquote{zero\_past}} = FALSE and the input grid consists of more than one location}
\index{file\_past (in namelist JULES\_AGRIC)@\spxentry{file\_past}\spxextra{in namelist JULES\_AGRIC}|spxpagem}

\begin{fulllineitems}
\phantomsection\label{\detokenize{namelists/ancillaries.nml:JULES_AGRIC::file_past}}
\pysigstartsignatures
\pysigline{\sphinxcode{\sphinxupquote{JULES\_AGRIC::}}\sphinxbfcode{\sphinxupquote{file\_past}}}
\pysigstopsignatures\begin{quote}\begin{description}
\sphinxlineitem{Type}
\sphinxAtStartPar
character

\sphinxlineitem{Default}
\sphinxAtStartPar
None

\end{description}\end{quote}

\sphinxAtStartPar
The name of the file to read pasture fraction data from.

\end{fulllineitems}

\index{pasture\_name (in namelist JULES\_AGRIC)@\spxentry{pasture\_name}\spxextra{in namelist JULES\_AGRIC}|spxpagem}

\begin{fulllineitems}
\phantomsection\label{\detokenize{namelists/ancillaries.nml:JULES_AGRIC::pasture_name}}
\pysigstartsignatures
\pysigline{\sphinxcode{\sphinxupquote{JULES\_AGRIC::}}\sphinxbfcode{\sphinxupquote{pasture\_name}}}
\pysigstopsignatures\begin{quote}\begin{description}
\sphinxlineitem{Type}
\sphinxAtStartPar
character

\sphinxlineitem{Default}
\sphinxAtStartPar
‘frac\_past’

\end{description}\end{quote}

\sphinxAtStartPar
The name of the variable containing the pasture fraction data.

\sphinxAtStartPar
In the file, the variable must have no levels dimensions and no time dimension.

\end{fulllineitems}

\end{sphinxadmonition}
\index{zero\_biocrop (in namelist JULES\_AGRIC)@\spxentry{zero\_biocrop}\spxextra{in namelist JULES\_AGRIC}|spxpagem}

\begin{fulllineitems}
\phantomsection\label{\detokenize{namelists/ancillaries.nml:JULES_AGRIC::zero_biocrop}}
\pysigstartsignatures
\pysigline{\sphinxcode{\sphinxupquote{JULES\_AGRIC::}}\sphinxbfcode{\sphinxupquote{zero\_biocrop}}}
\pysigstopsignatures\begin{quote}\begin{description}
\sphinxlineitem{Type}
\sphinxAtStartPar
logical

\sphinxlineitem{Default}
\sphinxAtStartPar
T

\end{description}\end{quote}

\sphinxAtStartPar
Switch used to simplify the initialisation of bioenergy fraction. Bioenergy fraction can only be used if {\hyperref[\detokenize{namelists/jules_vegetation.nml:JULES_VEGETATION::l_trif_biocrop}]{\sphinxcrossref{\sphinxcode{\sphinxupquote{l\_trif\_biocrop}}}}} is TRUE.
\begin{description}
\sphinxlineitem{TRUE}
\sphinxAtStartPar
Set bioenergy fraction at all points to zero.

\sphinxlineitem{FALSE}
\sphinxAtStartPar
Set bioenergy fraction using specified data.

\end{description}

\end{fulllineitems}


\begin{sphinxadmonition}{note}{Used if \sphinxstyleliteralintitle{\sphinxupquote{zero\_biocrop}} = FALSE and the input grid consists of a single location}
\index{frac\_biocrop (in namelist JULES\_AGRIC)@\spxentry{frac\_biocrop}\spxextra{in namelist JULES\_AGRIC}|spxpagem}

\begin{fulllineitems}
\phantomsection\label{\detokenize{namelists/ancillaries.nml:JULES_AGRIC::frac_biocrop}}
\pysigstartsignatures
\pysigline{\sphinxcode{\sphinxupquote{JULES\_AGRIC::}}\sphinxbfcode{\sphinxupquote{frac\_biocrop}}}
\pysigstopsignatures\begin{quote}\begin{description}
\sphinxlineitem{Type}
\sphinxAtStartPar
real

\sphinxlineitem{Default}
\sphinxAtStartPar
None

\end{description}\end{quote}

\sphinxAtStartPar
The bioenergy fraction for the single location.

\end{fulllineitems}

\end{sphinxadmonition}

\begin{sphinxadmonition}{note}{Used if \sphinxstyleliteralintitle{\sphinxupquote{zero\_biocrop}} = FALSE and the input grid consists of more than one location}
\index{file\_biocrop (in namelist JULES\_AGRIC)@\spxentry{file\_biocrop}\spxextra{in namelist JULES\_AGRIC}|spxpagem}

\begin{fulllineitems}
\phantomsection\label{\detokenize{namelists/ancillaries.nml:JULES_AGRIC::file_biocrop}}
\pysigstartsignatures
\pysigline{\sphinxcode{\sphinxupquote{JULES\_AGRIC::}}\sphinxbfcode{\sphinxupquote{file\_biocrop}}}
\pysigstopsignatures\begin{quote}\begin{description}
\sphinxlineitem{Type}
\sphinxAtStartPar
character

\sphinxlineitem{Default}
\sphinxAtStartPar
None

\end{description}\end{quote}

\sphinxAtStartPar
The name of the file to read bioenergy fraction data from.

\end{fulllineitems}

\index{biocrop\_name (in namelist JULES\_AGRIC)@\spxentry{biocrop\_name}\spxextra{in namelist JULES\_AGRIC}|spxpagem}

\begin{fulllineitems}
\phantomsection\label{\detokenize{namelists/ancillaries.nml:JULES_AGRIC::biocrop_name}}
\pysigstartsignatures
\pysigline{\sphinxcode{\sphinxupquote{JULES\_AGRIC::}}\sphinxbfcode{\sphinxupquote{biocrop\_name}}}
\pysigstopsignatures\begin{quote}\begin{description}
\sphinxlineitem{Type}
\sphinxAtStartPar
character

\sphinxlineitem{Default}
\sphinxAtStartPar
‘frac\_biocrop’

\end{description}\end{quote}

\sphinxAtStartPar
The name of the variable containing the bioenergy fraction data.
\begin{quote}

\sphinxAtStartPar
In the file, the variable must have no levels dimensions and no time dimension.
\end{quote}

\end{fulllineitems}

\end{sphinxadmonition}

\begin{sphinxadmonition}{note}{Specify the day of year on which harvesting occurs. Only used if \sphinxstyleliteralintitle{\sphinxupquote{l\_trif\_biocrop}} = TRUE. A placeholder value must be set for all PFTs, though will only be used for PFTs with \sphinxstyleliteralintitle{\sphinxupquote{harvest\_type\_io}} = 2.}
\index{read\_harvest\_doy\_from\_dump (in namelist JULES\_AGRIC)@\spxentry{read\_harvest\_doy\_from\_dump}\spxextra{in namelist JULES\_AGRIC}|spxpagem}

\begin{fulllineitems}
\phantomsection\label{\detokenize{namelists/ancillaries.nml:JULES_AGRIC::read_harvest_doy_from_dump}}
\pysigstartsignatures
\pysigline{\sphinxcode{\sphinxupquote{JULES\_AGRIC::}}\sphinxbfcode{\sphinxupquote{read\_harvest\_doy\_from\_dump}}}
\pysigstopsignatures\begin{quote}\begin{description}
\sphinxlineitem{Type}
\sphinxAtStartPar
logical

\sphinxlineitem{Default}
\sphinxAtStartPar
F

\end{description}\end{quote}
\begin{description}
\sphinxlineitem{TRUE}
\sphinxAtStartPar
Populate harvest\_doy from the dump file. All other namelist members are ignored.

\sphinxlineitem{FALSE}
\sphinxAtStartPar
Use the other namelist members to determine how to populate variables.

\end{description}

\end{fulllineitems}

\index{file\_harvest\_doy (in namelist JULES\_AGRIC)@\spxentry{file\_harvest\_doy}\spxextra{in namelist JULES\_AGRIC}|spxpagem}

\begin{fulllineitems}
\phantomsection\label{\detokenize{namelists/ancillaries.nml:JULES_AGRIC::file_harvest_doy}}
\pysigstartsignatures
\pysigline{\sphinxcode{\sphinxupquote{JULES\_AGRIC::}}\sphinxbfcode{\sphinxupquote{file\_harvest\_doy}}}
\pysigstopsignatures\begin{quote}\begin{description}
\sphinxlineitem{Type}
\sphinxAtStartPar
character

\sphinxlineitem{Default}
\sphinxAtStartPar
None

\end{description}\end{quote}

\sphinxAtStartPar
The name of the file to read harvest day\sphinxhyphen{}of\sphinxhyphen{}year data from.

\end{fulllineitems}

\index{harvest\_doy\_name (in namelist JULES\_AGRIC)@\spxentry{harvest\_doy\_name}\spxextra{in namelist JULES\_AGRIC}|spxpagem}

\begin{fulllineitems}
\phantomsection\label{\detokenize{namelists/ancillaries.nml:JULES_AGRIC::harvest_doy_name}}
\pysigstartsignatures
\pysigline{\sphinxcode{\sphinxupquote{JULES\_AGRIC::}}\sphinxbfcode{\sphinxupquote{harvest\_doy\_name}}}
\pysigstopsignatures\begin{quote}\begin{description}
\sphinxlineitem{Type}
\sphinxAtStartPar
character

\sphinxlineitem{Default}
\sphinxAtStartPar
‘harvest\_doy’

\end{description}\end{quote}

\sphinxAtStartPar
The name of the variable containing the harvest day\sphinxhyphen{}of\sphinxhyphen{}year data.

\begin{sphinxadmonition}{note}{Note:}
\sphinxAtStartPar
This is only used for NetCDF files.
For ASCII files, the harvest day\sphinxhyphen{}of\sphinxhyphen{}year data is expected to be the first (ideally only) variable in the file.
\end{sphinxadmonition}

\end{fulllineitems}

\end{sphinxadmonition}


\subsection{\sphinxstyleliteralintitle{\sphinxupquote{JULES\_CROP\_PROPS}} namelist members}
\label{\detokenize{namelists/ancillaries.nml:namelist-JULES_CROP_PROPS}}\label{\detokenize{namelists/ancillaries.nml:jules-crop-props-namelist-members}}\index{JULES\_CROP\_PROPS (namelist)@\spxentry{JULES\_CROP\_PROPS}\spxextra{namelist}|spxpagem}\index{read\_from\_dump (in namelist JULES\_CROP\_PROPS)@\spxentry{read\_from\_dump}\spxextra{in namelist JULES\_CROP\_PROPS}|spxpagem}

\begin{fulllineitems}
\phantomsection\label{\detokenize{namelists/ancillaries.nml:JULES_CROP_PROPS::read_from_dump}}
\pysigstartsignatures
\pysigline{\sphinxcode{\sphinxupquote{JULES\_CROP\_PROPS::}}\sphinxbfcode{\sphinxupquote{read\_from\_dump}}}
\pysigstopsignatures\begin{quote}\begin{description}
\sphinxlineitem{Type}
\sphinxAtStartPar
logical

\sphinxlineitem{Default}
\sphinxAtStartPar
F

\end{description}\end{quote}
\begin{description}
\sphinxlineitem{TRUE}
\sphinxAtStartPar
Populate variables associated with this namelist from the dump file. All other namelist members are ignored.

\sphinxlineitem{FALSE}
\sphinxAtStartPar
Use the other namelist members to determine how to populate variables.

\end{description}

\end{fulllineitems}

\index{file (in namelist JULES\_CROP\_PROPS)@\spxentry{file}\spxextra{in namelist JULES\_CROP\_PROPS}|spxpagem}

\begin{fulllineitems}
\phantomsection\label{\detokenize{namelists/ancillaries.nml:JULES_CROP_PROPS::file}}
\pysigstartsignatures
\pysigline{\sphinxcode{\sphinxupquote{JULES\_CROP\_PROPS::}}\sphinxbfcode{\sphinxupquote{file}}}
\pysigstopsignatures\begin{quote}\begin{description}
\sphinxlineitem{Type}
\sphinxAtStartPar
character

\sphinxlineitem{Default}
\sphinxAtStartPar
None

\end{description}\end{quote}

\sphinxAtStartPar
The file from which crop properties are read.

\sphinxAtStartPar
If {\hyperref[\detokenize{namelists/ancillaries.nml:JULES_CROP_PROPS::use_file}]{\sphinxcrossref{\sphinxcode{\sphinxupquote{use\_file}}}}} is FALSE for every variable, this will not be used.

\sphinxAtStartPar
This file name can use {\hyperref[\detokenize{input/file-name-templating::doc}]{\sphinxcrossref{\DUrole{doc}{variable name templating}}}}.

\end{fulllineitems}

\index{nvars (in namelist JULES\_CROP\_PROPS)@\spxentry{nvars}\spxextra{in namelist JULES\_CROP\_PROPS}|spxpagem}

\begin{fulllineitems}
\phantomsection\label{\detokenize{namelists/ancillaries.nml:JULES_CROP_PROPS::nvars}}
\pysigstartsignatures
\pysigline{\sphinxcode{\sphinxupquote{JULES\_CROP\_PROPS::}}\sphinxbfcode{\sphinxupquote{nvars}}}
\pysigstopsignatures\begin{quote}\begin{description}
\sphinxlineitem{Type}
\sphinxAtStartPar
integer

\sphinxlineitem{Permitted}
\sphinxAtStartPar
\textgreater{}= 0

\sphinxlineitem{Default}
\sphinxAtStartPar
0

\end{description}\end{quote}

\sphinxAtStartPar
The number of crop property variables that will be provided (see {\hyperref[\detokenize{namelists/ancillaries.nml:list-of-spatially-varying-crop-properties}]{\sphinxcrossref{\DUrole{std,std-ref}{List of spatially\sphinxhyphen{}varying crop properties}}}}).

\end{fulllineitems}

\index{var (in namelist JULES\_CROP\_PROPS)@\spxentry{var}\spxextra{in namelist JULES\_CROP\_PROPS}|spxpagem}

\begin{fulllineitems}
\phantomsection\label{\detokenize{namelists/ancillaries.nml:JULES_CROP_PROPS::var}}
\pysigstartsignatures
\pysigline{\sphinxcode{\sphinxupquote{JULES\_CROP\_PROPS::}}\sphinxbfcode{\sphinxupquote{var}}}
\pysigstopsignatures\begin{quote}\begin{description}
\sphinxlineitem{Type}
\sphinxAtStartPar
character(nvars)

\sphinxlineitem{Default}
\sphinxAtStartPar
None

\end{description}\end{quote}

\sphinxAtStartPar
List of variable names for spatially\sphinxhyphen{}varying crop properties as recognised by JULES (see {\hyperref[\detokenize{namelists/ancillaries.nml:list-of-spatially-varying-crop-properties}]{\sphinxcrossref{\DUrole{std,std-ref}{List of spatially\sphinxhyphen{}varying crop properties}}}}). Names are case sensitive.

\begin{sphinxadmonition}{note}{Note:}
\sphinxAtStartPar
For ASCII files, variable names must be in the order they appear in the file.
\end{sphinxadmonition}

\end{fulllineitems}

\index{use\_file (in namelist JULES\_CROP\_PROPS)@\spxentry{use\_file}\spxextra{in namelist JULES\_CROP\_PROPS}|spxpagem}

\begin{fulllineitems}
\phantomsection\label{\detokenize{namelists/ancillaries.nml:JULES_CROP_PROPS::use_file}}
\pysigstartsignatures
\pysigline{\sphinxcode{\sphinxupquote{JULES\_CROP\_PROPS::}}\sphinxbfcode{\sphinxupquote{use\_file}}}
\pysigstopsignatures\begin{quote}\begin{description}
\sphinxlineitem{Type}
\sphinxAtStartPar
logical(nvars)

\sphinxlineitem{Default}
\sphinxAtStartPar
T

\end{description}\end{quote}

\sphinxAtStartPar
For each JULES variable specified in {\hyperref[\detokenize{namelists/ancillaries.nml:JULES_CROP_PROPS::var}]{\sphinxcrossref{\sphinxcode{\sphinxupquote{var}}}}}, this indicates if it should be read from the specified file or whether a constant value is to be used.
\begin{description}
\sphinxlineitem{TRUE}
\sphinxAtStartPar
The variable will be read from the file.

\sphinxlineitem{FALSE}
\sphinxAtStartPar
The variable will be set to a constant value everywhere using {\hyperref[\detokenize{namelists/ancillaries.nml:JULES_CROP_PROPS::const_val}]{\sphinxcrossref{\sphinxcode{\sphinxupquote{const\_val}}}}} below.

\end{description}

\end{fulllineitems}

\index{var\_name (in namelist JULES\_CROP\_PROPS)@\spxentry{var\_name}\spxextra{in namelist JULES\_CROP\_PROPS}|spxpagem}

\begin{fulllineitems}
\phantomsection\label{\detokenize{namelists/ancillaries.nml:JULES_CROP_PROPS::var_name}}
\pysigstartsignatures
\pysigline{\sphinxcode{\sphinxupquote{JULES\_CROP\_PROPS::}}\sphinxbfcode{\sphinxupquote{var\_name}}}
\pysigstopsignatures\begin{quote}\begin{description}
\sphinxlineitem{Type}
\sphinxAtStartPar
character(nvars)

\sphinxlineitem{Default}
\sphinxAtStartPar
‘’ (empty string)

\end{description}\end{quote}

\sphinxAtStartPar
For each JULES variable specified in {\hyperref[\detokenize{namelists/ancillaries.nml:JULES_CROP_PROPS::var}]{\sphinxcrossref{\sphinxcode{\sphinxupquote{var}}}}} where {\hyperref[\detokenize{namelists/ancillaries.nml:JULES_CROP_PROPS::use_file}]{\sphinxcrossref{\sphinxcode{\sphinxupquote{use\_file}}}}} = TRUE, this is the name of the variable in the file containing the data.

\sphinxAtStartPar
If the empty string (the default) is given for any variable, then the corresponding value from {\hyperref[\detokenize{namelists/ancillaries.nml:JULES_CROP_PROPS::var}]{\sphinxcrossref{\sphinxcode{\sphinxupquote{var}}}}} is used instead.

\sphinxAtStartPar
This is not used for variables where {\hyperref[\detokenize{namelists/ancillaries.nml:JULES_CROP_PROPS::use_file}]{\sphinxcrossref{\sphinxcode{\sphinxupquote{use\_file}}}}} = FALSE, but a placeholder must still be given in that case.

\begin{sphinxadmonition}{note}{Note:}
\sphinxAtStartPar
For ASCII files, this is not used \sphinxhyphen{} only the order in the file matters, as described above.
\end{sphinxadmonition}

\end{fulllineitems}

\index{tpl\_name (in namelist JULES\_CROP\_PROPS)@\spxentry{tpl\_name}\spxextra{in namelist JULES\_CROP\_PROPS}|spxpagem}

\begin{fulllineitems}
\phantomsection\label{\detokenize{namelists/ancillaries.nml:JULES_CROP_PROPS::tpl_name}}
\pysigstartsignatures
\pysigline{\sphinxcode{\sphinxupquote{JULES\_CROP\_PROPS::}}\sphinxbfcode{\sphinxupquote{tpl\_name}}}
\pysigstopsignatures\begin{quote}\begin{description}
\sphinxlineitem{Type}
\sphinxAtStartPar
character(nvars)

\sphinxlineitem{Default}
\sphinxAtStartPar
None

\end{description}\end{quote}

\sphinxAtStartPar
For each JULES variable specified in {\hyperref[\detokenize{namelists/ancillaries.nml:JULES_CROP_PROPS::var}]{\sphinxcrossref{\sphinxcode{\sphinxupquote{var}}}}}, this is the string to substitute into the file name in place of the variable name substitution string.

\sphinxAtStartPar
If the file name does not use variable name templating, this is not used.

\end{fulllineitems}

\index{const\_val (in namelist JULES\_CROP\_PROPS)@\spxentry{const\_val}\spxextra{in namelist JULES\_CROP\_PROPS}|spxpagem}

\begin{fulllineitems}
\phantomsection\label{\detokenize{namelists/ancillaries.nml:JULES_CROP_PROPS::const_val}}
\pysigstartsignatures
\pysigline{\sphinxcode{\sphinxupquote{JULES\_CROP\_PROPS::}}\sphinxbfcode{\sphinxupquote{const\_val}}}
\pysigstopsignatures\begin{quote}\begin{description}
\sphinxlineitem{Type}
\sphinxAtStartPar
real(nvars)

\sphinxlineitem{Default}
\sphinxAtStartPar
None

\end{description}\end{quote}

\sphinxAtStartPar
For each JULES variable specified in {\hyperref[\detokenize{namelists/ancillaries.nml:JULES_CROP_PROPS::var}]{\sphinxcrossref{\sphinxcode{\sphinxupquote{var}}}}} where {\hyperref[\detokenize{namelists/ancillaries.nml:JULES_CROP_PROPS::use_file}]{\sphinxcrossref{\sphinxcode{\sphinxupquote{use\_file}}}}} = FALSE, this is a constant value that the variable will be set to at every point in every layer.

\sphinxAtStartPar
This is not used for variables where {\hyperref[\detokenize{namelists/ancillaries.nml:JULES_CROP_PROPS::use_file}]{\sphinxcrossref{\sphinxcode{\sphinxupquote{use\_file}}}}} = TRUE, but a placeholder must still be given in that case.

\end{fulllineitems}



\subsubsection{List of spatially\sphinxhyphen{}varying crop properties}
\label{\detokenize{namelists/ancillaries.nml:list-of-spatially-varying-crop-properties}}\label{\detokenize{namelists/ancillaries.nml:id1}}
\sphinxAtStartPar
All of the crop variables listed below are expected to have a single levels dimension of size {\hyperref[\detokenize{namelists/jules_surface_types.nml:JULES_SURFACE_TYPES::ncpft}]{\sphinxcrossref{\sphinxcode{\sphinxupquote{ncpft}}}}} called {\hyperref[\detokenize{namelists/model_grid.nml:JULES_INPUT_GRID::cpft_dim_name}]{\sphinxcrossref{\sphinxcode{\sphinxupquote{cpft\_dim\_name}}}}}.


\begin{savenotes}\sphinxattablestart
\centering
\begin{tabulary}{\linewidth}[t]{|p{3.75cm}|p{11cm}|}
\hline
\sphinxstyletheadfamily 
\sphinxAtStartPar
Name
&\sphinxstyletheadfamily 
\sphinxAtStartPar
Description
\\
\hline
\sphinxAtStartPar
\sphinxcode{\sphinxupquote{cropsowdate}}
&
\sphinxAtStartPar
The sowing date for each crop.

\sphinxAtStartPar
The sowing date should be a real number, with \sphinxcode{\sphinxupquote{0 \textless{} nint(sowing\_date) \textless{} number of days in year}}.
For example, for a 365 day year, sow\_date = 1.0 is Jan 1st and sow\_date = 365.0 is Dec 31st.

\sphinxAtStartPar
If a crop requires two sowing dates per year, it should be treated as two separate crops with identical
parameters apart from the sowing date.

\begin{sphinxadmonition}{note}{Note:}
\sphinxAtStartPar
Only required if {\hyperref[\detokenize{namelists/jules_vegetation.nml:JULES_VEGETATION::l_prescsow}]{\sphinxcrossref{\sphinxcode{\sphinxupquote{l\_prescsow}}}}} = TRUE.
\end{sphinxadmonition}
\\
\hline
\sphinxAtStartPar
\sphinxcode{\sphinxupquote{cropttveg}}
&
\sphinxAtStartPar
Thermal time between emergence and flowering (degree days).
\\
\hline
\sphinxAtStartPar
\sphinxcode{\sphinxupquote{cropttrep}}
&
\sphinxAtStartPar
Thermal time between flowering and maturity/harvest (degree days).
\\
\hline
\sphinxAtStartPar
\sphinxcode{\sphinxupquote{croplatestharvdate}}
&
\sphinxAtStartPar
The latest possible harvest date for each crop.
croplatestharvdate is only a required variable when
{\hyperref[\detokenize{namelists/jules_vegetation.nml:JULES_VEGETATION::l_croprotate}]{\sphinxcrossref{\sphinxcode{\sphinxupquote{l\_croprotate}}}}} = TRUE and
{\hyperref[\detokenize{namelists/jules_vegetation.nml:JULES_VEGETATION::l_prescsow}]{\sphinxcrossref{\sphinxcode{\sphinxupquote{l\_prescsow}}}}} = TRUE.

\sphinxAtStartPar
croplatestharvdate is not a required variable when
{\hyperref[\detokenize{namelists/jules_vegetation.nml:JULES_VEGETATION::l_croprotate}]{\sphinxcrossref{\sphinxcode{\sphinxupquote{l\_croprotate}}}}} = FALSE and
{\hyperref[\detokenize{namelists/jules_vegetation.nml:JULES_VEGETATION::l_prescsow}]{\sphinxcrossref{\sphinxcode{\sphinxupquote{l\_prescsow}}}}} = TRUE but will be used if provided in the ancillary file

\sphinxAtStartPar
croplatestharvdate is not a required variable and is only used if provided as an ancillary when
{\hyperref[\detokenize{namelists/jules_vegetation.nml:JULES_VEGETATION::l_prescsow}]{\sphinxcrossref{\sphinxcode{\sphinxupquote{l\_prescsow}}}}} = TRUE.
\\
\hline
\end{tabulary}
\par
\sphinxattableend\end{savenotes}


\sphinxstrong{See also:}
\nopagebreak


\sphinxAtStartPar
References:
\begin{itemize}
\item {} 
\sphinxAtStartPar
Osborne et al, \sphinxhref{http://www.geosci-model-dev.net/8/1139/2015/gmd-8-1139-2015.html}{JULES\sphinxhyphen{}crop: a parametrisation of crops in the Joint UK Land Environment Simulator}, Geosci. Model Dev., 8, 1139\sphinxhyphen{}1155, 2015.

\item {} 
\sphinxAtStartPar
Mathison et al, ‘Developing a sequential cropping capability in the JULESvn5.2 land\textendash{}surface model’, Geosci. Model Dev. Discuss., \sphinxurl{https://doi.org/10.5194/gmd-2019-85}, in review, 2019

\end{itemize}




\subsection{\sphinxstyleliteralintitle{\sphinxupquote{JULES\_IRRIG\_PROPS}} namelist members}
\label{\detokenize{namelists/ancillaries.nml:namelist-JULES_IRRIG_PROPS}}\label{\detokenize{namelists/ancillaries.nml:jules-irrig-props-namelist-members}}\index{JULES\_IRRIG\_PROPS (namelist)@\spxentry{JULES\_IRRIG\_PROPS}\spxextra{namelist}|spxpagem}
\sphinxAtStartPar
This namelist specifies the options available for initialising irrigated fraction.
\index{read\_from\_dump (in namelist JULES\_IRRIG\_PROPS)@\spxentry{read\_from\_dump}\spxextra{in namelist JULES\_IRRIG\_PROPS}|spxpagem}

\begin{fulllineitems}
\phantomsection\label{\detokenize{namelists/ancillaries.nml:JULES_IRRIG_PROPS::read_from_dump}}
\pysigstartsignatures
\pysigline{\sphinxcode{\sphinxupquote{JULES\_IRRIG\_PROPS::}}\sphinxbfcode{\sphinxupquote{read\_from\_dump}}}
\pysigstopsignatures\begin{quote}\begin{description}
\sphinxlineitem{Type}
\sphinxAtStartPar
logical

\sphinxlineitem{Default}
\sphinxAtStartPar
F

\end{description}\end{quote}
\begin{description}
\sphinxlineitem{TRUE}
\sphinxAtStartPar
Populate variables associated with this namelist from the dump file. All other namelist members are ignored.

\sphinxlineitem{FALSE}
\sphinxAtStartPar
Use the other namelist members to determine how to populate variables.

\end{description}

\end{fulllineitems}

\index{read\_file (in namelist JULES\_IRRIG\_PROPS)@\spxentry{read\_file}\spxextra{in namelist JULES\_IRRIG\_PROPS}|spxpagem}

\begin{fulllineitems}
\phantomsection\label{\detokenize{namelists/ancillaries.nml:JULES_IRRIG_PROPS::read_file}}
\pysigstartsignatures
\pysigline{\sphinxcode{\sphinxupquote{JULES\_IRRIG\_PROPS::}}\sphinxbfcode{\sphinxupquote{read\_file}}}
\pysigstopsignatures\begin{quote}\begin{description}
\sphinxlineitem{Type}
\sphinxAtStartPar
logical

\sphinxlineitem{Default}
\sphinxAtStartPar
T

\end{description}\end{quote}

\sphinxAtStartPar
Indicates if irrigated fraction is to be read from file.
\begin{description}
\sphinxlineitem{TRUE}
\sphinxAtStartPar
Irrigated fraction is read from the file specified in {\hyperref[\detokenize{namelists/ancillaries.nml:JULES_IRRIG_PROPS::irrig_frac_file}]{\sphinxcrossref{\sphinxcode{\sphinxupquote{irrig\_frac\_file}}}}}.

\sphinxlineitem{FALSE}
\sphinxAtStartPar
Irrigated fraction is set to the constant value specified in {\hyperref[\detokenize{namelists/ancillaries.nml:JULES_IRRIG_PROPS::const_frac_irr}]{\sphinxcrossref{\sphinxcode{\sphinxupquote{const\_frac\_irr}}}}}.

\end{description}

\end{fulllineitems}

\index{irrig\_frac\_file (in namelist JULES\_IRRIG\_PROPS)@\spxentry{irrig\_frac\_file}\spxextra{in namelist JULES\_IRRIG\_PROPS}|spxpagem}

\begin{fulllineitems}
\phantomsection\label{\detokenize{namelists/ancillaries.nml:JULES_IRRIG_PROPS::irrig_frac_file}}
\pysigstartsignatures
\pysigline{\sphinxcode{\sphinxupquote{JULES\_IRRIG\_PROPS::}}\sphinxbfcode{\sphinxupquote{irrig\_frac\_file}}}
\pysigstopsignatures\begin{quote}\begin{description}
\sphinxlineitem{Type}
\sphinxAtStartPar
character

\sphinxlineitem{Default}
\sphinxAtStartPar
None

\end{description}\end{quote}

\sphinxAtStartPar
The file from which irrigation fractions are read, including path.

\end{fulllineitems}

\index{var\_name (in namelist JULES\_IRRIG\_PROPS)@\spxentry{var\_name}\spxextra{in namelist JULES\_IRRIG\_PROPS}|spxpagem}

\begin{fulllineitems}
\phantomsection\label{\detokenize{namelists/ancillaries.nml:JULES_IRRIG_PROPS::var_name}}
\pysigstartsignatures
\pysigline{\sphinxcode{\sphinxupquote{JULES\_IRRIG\_PROPS::}}\sphinxbfcode{\sphinxupquote{var\_name}}}
\pysigstopsignatures\begin{quote}\begin{description}
\sphinxlineitem{Type}
\sphinxAtStartPar
character

\sphinxlineitem{Default}
\sphinxAtStartPar
‘frac\_irig’

\end{description}\end{quote}

\sphinxAtStartPar
The name of the variable containing the irrigated fraction data.

\begin{sphinxadmonition}{note}{Note:}
\sphinxAtStartPar
This is only used for NetCDF files.
For ASCII files, the irrigated fraction data is expected to be the first (ideally only) variable in the file.
\end{sphinxadmonition}

\sphinxAtStartPar
In the file, the variable must have no levels or time dimensions.

\end{fulllineitems}

\index{const\_frac\_irr (in namelist JULES\_IRRIG\_PROPS)@\spxentry{const\_frac\_irr}\spxextra{in namelist JULES\_IRRIG\_PROPS}|spxpagem}

\begin{fulllineitems}
\phantomsection\label{\detokenize{namelists/ancillaries.nml:JULES_IRRIG_PROPS::const_frac_irr}}
\pysigstartsignatures
\pysigline{\sphinxcode{\sphinxupquote{JULES\_IRRIG\_PROPS::}}\sphinxbfcode{\sphinxupquote{const\_frac\_irr}}}
\pysigstopsignatures\begin{quote}\begin{description}
\sphinxlineitem{Type}
\sphinxAtStartPar
real

\sphinxlineitem{Default}
\sphinxAtStartPar
none

\end{description}\end{quote}

\sphinxAtStartPar
The constant irrigated fraction to be applied to all grid points.

\end{fulllineitems}

\index{const\_irrfrac\_irrtiles (in namelist JULES\_IRRIG\_PROPS)@\spxentry{const\_irrfrac\_irrtiles}\spxextra{in namelist JULES\_IRRIG\_PROPS}|spxpagem}

\begin{fulllineitems}
\phantomsection\label{\detokenize{namelists/ancillaries.nml:JULES_IRRIG_PROPS::const_irrfrac_irrtiles}}
\pysigstartsignatures
\pysigline{\sphinxcode{\sphinxupquote{JULES\_IRRIG\_PROPS::}}\sphinxbfcode{\sphinxupquote{const\_irrfrac\_irrtiles}}}
\pysigstopsignatures\begin{quote}\begin{description}
\sphinxlineitem{Type}
\sphinxAtStartPar
real

\sphinxlineitem{Default}
\sphinxAtStartPar
none

\end{description}\end{quote}

\sphinxAtStartPar
The constant irrigated fraction to be applied to specific surface tiles given in {\hyperref[\detokenize{namelists/jules_irrig.nml:JULES_IRRIG::irrigtiles}]{\sphinxcrossref{\sphinxcode{\sphinxupquote{irrigtiles}}}}}.

\end{fulllineitems}



\subsection{\sphinxstyleliteralintitle{\sphinxupquote{JULES\_RIVERS\_PROPS}} namelist members}
\label{\detokenize{namelists/ancillaries.nml:namelist-JULES_RIVERS_PROPS}}\label{\detokenize{namelists/ancillaries.nml:jules-rivers-props-namelist-members}}\index{JULES\_RIVERS\_PROPS (namelist)@\spxentry{JULES\_RIVERS\_PROPS}\spxextra{namelist}|spxpagem}
\sphinxAtStartPar
This namelist specifies how spatially varying river routing properties should be set.

\begin{sphinxadmonition}{note}{Note:}
\sphinxAtStartPar
\sphinxcode{\sphinxupquote{read\_from\_dump}} is not currently implemented for this namelist, although initial condition variables can be read from a dump file if {\hyperref[\detokenize{namelists/jules_rivers.nml:JULES_RIVERS::i_river_vn}]{\sphinxcrossref{\sphinxcode{\sphinxupquote{i\_river\_vn}}}}} is \sphinxcode{\sphinxupquote{1}}, \sphinxcode{\sphinxupquote{2}},  or \sphinxcode{\sphinxupquote{3}} (see {\hyperref[\detokenize{namelists/initial_conditions.nml:namelist-JULES_INITIAL}]{\sphinxcrossref{\sphinxcode{\sphinxupquote{JULES\_INITIAL}}}}}).
\end{sphinxadmonition}

\begin{sphinxadmonition}{note}{Note:}
\sphinxAtStartPar
The river routing code in JULES is still in development. Users should ensure that results are as expected, and provide feedback where deficiencies are identified.
\end{sphinxadmonition}

\begin{sphinxadmonition}{note}{Note:}
\sphinxAtStartPar
The grid on which the river routing will run, and on which river routing ancillaries must be provided, could potentially differ from the input/model grid specified in {\hyperref[\detokenize{namelists/model_grid.nml::doc}]{\sphinxcrossref{\DUrole{doc}{model\_grid.nml}}}}.

\sphinxAtStartPar
For the duration of this document, the following nomenclature will be used:
\begin{itemize}
\item {} 
\sphinxAtStartPar
\sphinxstylestrong{Model input grid} \sphinxhyphen{} The full JULES input grid specified in {\hyperref[\detokenize{namelists/model_grid.nml:namelist-JULES_INPUT_GRID}]{\sphinxcrossref{\sphinxcode{\sphinxupquote{JULES\_INPUT\_GRID}}}}}.

\item {} 
\sphinxAtStartPar
\sphinxstylestrong{River routing input grid} \sphinxhyphen{} The grid on which river routing ancillaries will be provided

\end{itemize}

\sphinxAtStartPar
Currently, information about the river routing input grid and its relationship to the model input grid is specified in {\hyperref[\detokenize{namelists/ancillaries.nml:namelist-JULES_RIVERS_PROPS}]{\sphinxcrossref{\sphinxcode{\sphinxupquote{JULES\_RIVERS\_PROPS}}}}}.

\sphinxAtStartPar
While the model input can be defined on a 1D grid, the river routing input grid must be defined on a 2D grid, as defined through the x and y dimensions of the rivers ancillary file. See {\hyperref[\detokenize{namelists/ancillaries.nml:JULES_RIVERS_PROPS::x_dim_name}]{\sphinxcrossref{\sphinxcode{\sphinxupquote{x\_dim\_name}}}}} and {\hyperref[\detokenize{namelists/ancillaries.nml:JULES_RIVERS_PROPS::y_dim_name}]{\sphinxcrossref{\sphinxcode{\sphinxupquote{y\_dim\_name}}}}} for further details. If a non\sphinxhyphen{}regular model and river routing input grid is used, both the x and y dimensions and corresponding latitude and longitude values must be specified for each grid point.

\sphinxAtStartPar
However, internally JULES converts the river routing input grid to a 1D river routing model grid, with length \sphinxcode{\sphinxupquote{np\_rivers}}, which is the number of valid routing points in the river routing ancillaries. All river routing output is either defined on the 1D river routing model grid or is regridded to the model grid.

\sphinxAtStartPar
For some applications, the model input and river routing input grids may not be coincident. Note that functionality only currently exists to regrid between regular (but non\sphinxhyphen{}identical) model input and river routing input grids. If a non\sphinxhyphen{}regular model input grid is specified, it is assumed that the model input and river routing input grids will be coincident.
\end{sphinxadmonition}

\begin{sphinxadmonition}{note}{Members used to define the river routing input grid}
\index{rivers\_reglatlon (in namelist JULES\_RIVERS\_PROPS)@\spxentry{rivers\_reglatlon}\spxextra{in namelist JULES\_RIVERS\_PROPS}|spxpagem}

\begin{fulllineitems}
\phantomsection\label{\detokenize{namelists/ancillaries.nml:JULES_RIVERS_PROPS::rivers_reglatlon}}
\pysigstartsignatures
\pysigline{\sphinxcode{\sphinxupquote{JULES\_RIVERS\_PROPS::}}\sphinxbfcode{\sphinxupquote{rivers\_reglatlon}}}
\pysigstopsignatures\begin{quote}\begin{description}
\sphinxlineitem{Type}
\sphinxAtStartPar
logical

\sphinxlineitem{Default}
\sphinxAtStartPar
T

\end{description}\end{quote}

\sphinxAtStartPar
Flag indicating if the river routing input grid is regular in latitude and longitude.
\begin{description}
\sphinxlineitem{TRUE}
\sphinxAtStartPar
River routing input grid is regular in latitude and longitude.

\sphinxlineitem{FALSE}
\sphinxAtStartPar
River routing input grid is not regular in latitude and longitude (e.g. grid defined relative to a rotated pole, Ordnance Survey (British) National Grid (BNG) OSGB36, etc). Only {\hyperref[\detokenize{namelists/jules_rivers.nml:JULES_RIVERS::i_river_vn}]{\sphinxcrossref{\sphinxcode{\sphinxupquote{i\_river\_vn}}}}} = \sphinxcode{\sphinxupquote{2}} should be used in this case. Note that the model input and river routing input grids must be coincident in this case and {\hyperref[\detokenize{namelists/ancillaries.nml:JULES_RIVERS_PROPS::rivers_regrid}]{\sphinxcrossref{\sphinxcode{\sphinxupquote{rivers\_regrid}}}}} must also be set to false.

\end{description}

\end{fulllineitems}

\index{coordinate\_file (in namelist JULES\_RIVERS\_PROPS)@\spxentry{coordinate\_file}\spxextra{in namelist JULES\_RIVERS\_PROPS}|spxpagem}

\begin{fulllineitems}
\phantomsection\label{\detokenize{namelists/ancillaries.nml:JULES_RIVERS_PROPS::coordinate_file}}
\pysigstartsignatures
\pysigline{\sphinxcode{\sphinxupquote{JULES\_RIVERS\_PROPS::}}\sphinxbfcode{\sphinxupquote{coordinate\_file}}}
\pysigstopsignatures\begin{quote}\begin{description}
\sphinxlineitem{Type}
\sphinxAtStartPar
character

\sphinxlineitem{Default}
\sphinxAtStartPar
None

\end{description}\end{quote}

\sphinxAtStartPar
The file from which to read coordinate information for the river routing input grid. This is only used when {\hyperref[\detokenize{namelists/ancillaries.nml:JULES_RIVERS_PROPS::file}]{\sphinxcrossref{\sphinxcode{\sphinxupquote{file}}}}} includes {\hyperref[\detokenize{input/file-name-templating::doc}]{\sphinxcrossref{\DUrole{doc}{variable\sphinxhyphen{}name templating}}}}, i.e. it is only used when ancillary variables will come from multiple files, in which case this variable is used to provide clarity as to where the coordinates are read from.

\end{fulllineitems}

\index{x\_dim\_name (in namelist JULES\_RIVERS\_PROPS)@\spxentry{x\_dim\_name}\spxextra{in namelist JULES\_RIVERS\_PROPS}|spxpagem}

\begin{fulllineitems}
\phantomsection\label{\detokenize{namelists/ancillaries.nml:JULES_RIVERS_PROPS::x_dim_name}}
\pysigstartsignatures
\pysigline{\sphinxcode{\sphinxupquote{JULES\_RIVERS\_PROPS::}}\sphinxbfcode{\sphinxupquote{x\_dim\_name}}}
\pysigstopsignatures\begin{quote}\begin{description}
\sphinxlineitem{Type}
\sphinxAtStartPar
character

\sphinxlineitem{Default}
\sphinxAtStartPar
None

\end{description}\end{quote}

\sphinxAtStartPar
The name of the x dimension for the river routing input grid (it may, but does not have to, coincide with {\hyperref[\detokenize{namelists/model_grid.nml:JULES_INPUT_GRID::x_dim_name}]{\sphinxcrossref{\sphinxcode{\sphinxupquote{x\_dim\_name}}}}}).

\begin{sphinxadmonition}{note}{Note:}
\sphinxAtStartPar
For ASCII files, this can be anything. For NetCDF files, it should be the name of the dimension in {\hyperref[\detokenize{namelists/ancillaries.nml:JULES_RIVERS_PROPS::file}]{\sphinxcrossref{\sphinxcode{\sphinxupquote{file}}}}} (if that does not include {\hyperref[\detokenize{input/file-name-templating::doc}]{\sphinxcrossref{\DUrole{doc}{variable\sphinxhyphen{}name templating}}}}) or in {\hyperref[\detokenize{namelists/ancillaries.nml:JULES_RIVERS_PROPS::coordinate_file}]{\sphinxcrossref{\sphinxcode{\sphinxupquote{coordinate\_file}}}}} (if {\hyperref[\detokenize{namelists/ancillaries.nml:JULES_RIVERS_PROPS::file}]{\sphinxcrossref{\sphinxcode{\sphinxupquote{file}}}}} includes templating).
\end{sphinxadmonition}

\begin{sphinxadmonition}{warning}{Warning:}
\sphinxAtStartPar
Values for the x dimension of the river routing input grid will need to be read from the input file to define the grid, so it is assumed that the file contains a variable of the same name. If a non\sphinxhyphen{}regular river routing input grid is used, a 2D longitude field will also be needed to define the x\sphinxhyphen{}location of each grid point, read in via the longitude\_2d ancillary field.
\end{sphinxadmonition}

\end{fulllineitems}

\index{y\_dim\_name (in namelist JULES\_RIVERS\_PROPS)@\spxentry{y\_dim\_name}\spxextra{in namelist JULES\_RIVERS\_PROPS}|spxpagem}

\begin{fulllineitems}
\phantomsection\label{\detokenize{namelists/ancillaries.nml:JULES_RIVERS_PROPS::y_dim_name}}
\pysigstartsignatures
\pysigline{\sphinxcode{\sphinxupquote{JULES\_RIVERS\_PROPS::}}\sphinxbfcode{\sphinxupquote{y\_dim\_name}}}
\pysigstopsignatures\begin{quote}\begin{description}
\sphinxlineitem{Type}
\sphinxAtStartPar
character

\sphinxlineitem{Default}
\sphinxAtStartPar
None

\end{description}\end{quote}

\sphinxAtStartPar
The name of the y dimension for the river routing input grid (it may, but does not have to, coincide with {\hyperref[\detokenize{namelists/model_grid.nml:JULES_INPUT_GRID::y_dim_name}]{\sphinxcrossref{\sphinxcode{\sphinxupquote{y\_dim\_name}}}}}).

\begin{sphinxadmonition}{note}{Note:}
\sphinxAtStartPar
For ASCII files, this can be anything. For NetCDF files, it should be the name of the dimension in {\hyperref[\detokenize{namelists/ancillaries.nml:JULES_RIVERS_PROPS::file}]{\sphinxcrossref{\sphinxcode{\sphinxupquote{file}}}}} (if that does not include {\hyperref[\detokenize{input/file-name-templating::doc}]{\sphinxcrossref{\DUrole{doc}{variable\sphinxhyphen{}name templating}}}}) or in {\hyperref[\detokenize{namelists/ancillaries.nml:JULES_RIVERS_PROPS::coordinate_file}]{\sphinxcrossref{\sphinxcode{\sphinxupquote{coordinate\_file}}}}} (if {\hyperref[\detokenize{namelists/ancillaries.nml:JULES_RIVERS_PROPS::file}]{\sphinxcrossref{\sphinxcode{\sphinxupquote{file}}}}} includes templating).
\end{sphinxadmonition}

\begin{sphinxadmonition}{warning}{Warning:}
\sphinxAtStartPar
Values for the y dimension of the river routing input grid will need to be read from the input file to define the grid, so it is assumed that the file contains a variable of the same name. If a non\sphinxhyphen{}regular river routing input grid is used, a 2D latitude field will also be needed to define the y\sphinxhyphen{}location of each grid point, read in via the latitude\_2d ancillary field.
\end{sphinxadmonition}

\end{fulllineitems}

\index{nx (in namelist JULES\_RIVERS\_PROPS)@\spxentry{nx}\spxextra{in namelist JULES\_RIVERS\_PROPS}|spxpagem}

\begin{fulllineitems}
\phantomsection\label{\detokenize{namelists/ancillaries.nml:JULES_RIVERS_PROPS::nx}}
\pysigstartsignatures
\pysigline{\sphinxcode{\sphinxupquote{JULES\_RIVERS\_PROPS::}}\sphinxbfcode{\sphinxupquote{nx}}}
\pysigstopsignatures\begin{quote}\begin{description}
\sphinxlineitem{Type}
\sphinxAtStartPar
integer

\sphinxlineitem{Permitted}
\sphinxAtStartPar
\textgreater{}= 1

\sphinxlineitem{Default}
\sphinxAtStartPar
None

\end{description}\end{quote}

\sphinxAtStartPar
The size of the x dimension of the river routing input grid.

\end{fulllineitems}

\index{ny (in namelist JULES\_RIVERS\_PROPS)@\spxentry{ny}\spxextra{in namelist JULES\_RIVERS\_PROPS}|spxpagem}

\begin{fulllineitems}
\phantomsection\label{\detokenize{namelists/ancillaries.nml:JULES_RIVERS_PROPS::ny}}
\pysigstartsignatures
\pysigline{\sphinxcode{\sphinxupquote{JULES\_RIVERS\_PROPS::}}\sphinxbfcode{\sphinxupquote{ny}}}
\pysigstopsignatures\begin{quote}\begin{description}
\sphinxlineitem{Type}
\sphinxAtStartPar
integer

\sphinxlineitem{Permitted}
\sphinxAtStartPar
\textgreater{}= 1

\sphinxlineitem{Default}
\sphinxAtStartPar
None

\end{description}\end{quote}

\sphinxAtStartPar
The size of the y dimension of the river routing input grid.

\end{fulllineitems}

\end{sphinxadmonition}

\begin{sphinxadmonition}{note}{Members used to define the relationship between the model input grid and the river routing input grid}
\index{rivers\_regrid (in namelist JULES\_RIVERS\_PROPS)@\spxentry{rivers\_regrid}\spxextra{in namelist JULES\_RIVERS\_PROPS}|spxpagem}

\begin{fulllineitems}
\phantomsection\label{\detokenize{namelists/ancillaries.nml:JULES_RIVERS_PROPS::rivers_regrid}}
\pysigstartsignatures
\pysigline{\sphinxcode{\sphinxupquote{JULES\_RIVERS\_PROPS::}}\sphinxbfcode{\sphinxupquote{rivers\_regrid}}}
\pysigstopsignatures\begin{quote}\begin{description}
\sphinxlineitem{Type}
\sphinxAtStartPar
logical

\sphinxlineitem{Default}
\sphinxAtStartPar
T

\end{description}\end{quote}

\sphinxAtStartPar
Flag indicating if the model input and river routing input grids are identical, i.e. whether regridding of variables to and from the river routing input grid is required. Note this is only currently possible if {\hyperref[\detokenize{namelists/ancillaries.nml:JULES_RIVERS_PROPS::rivers_reglatlon}]{\sphinxcrossref{\sphinxcode{\sphinxupquote{rivers\_reglatlon}}}}} is TRUE.
\begin{description}
\sphinxlineitem{TRUE}
\sphinxAtStartPar
River routing input and model input grids differ and regridding is required.

\sphinxlineitem{FALSE}
\sphinxAtStartPar
River routing input and model input grids are identical.

\end{description}

\end{fulllineitems}


\begin{sphinxadmonition}{warning}{Warning:}
\sphinxAtStartPar
Currently, regridding between model input and river routing input grids must only be used with regular lat/lon model input and river routing input grids.
\begin{itemize}
\item {} 
\sphinxAtStartPar
If a 1D model input grid is specified in {\hyperref[\detokenize{namelists/model_grid.nml:namelist-JULES_INPUT_GRID}]{\sphinxcrossref{\sphinxcode{\sphinxupquote{JULES\_INPUT\_GRID}}}}}, it must be possible to define a 2D regular lat/lon grid containing all the points in the model input grid. This is done using the variables below.

\end{itemize}

\sphinxAtStartPar
An example with the GSWP2 (land points only) forcing data is given below.
\end{sphinxadmonition}

\begin{sphinxadmonition}{note}{Only used when \sphinxstyleliteralintitle{\sphinxupquote{JULES\_INPUT\_GRID::grid\_is\_1d}} = TRUE or for a parallel standalone run.}
\index{nx\_grid (in namelist JULES\_RIVERS\_PROPS)@\spxentry{nx\_grid}\spxextra{in namelist JULES\_RIVERS\_PROPS}|spxpagem}

\begin{fulllineitems}
\phantomsection\label{\detokenize{namelists/ancillaries.nml:JULES_RIVERS_PROPS::nx_grid}}
\pysigstartsignatures
\pysigline{\sphinxcode{\sphinxupquote{JULES\_RIVERS\_PROPS::}}\sphinxbfcode{\sphinxupquote{nx\_grid}}}
\pysigstopsignatures\begin{quote}\begin{description}
\sphinxlineitem{Type}
\sphinxAtStartPar
integer

\sphinxlineitem{Permitted}
\sphinxAtStartPar
\textgreater{}= 1

\sphinxlineitem{Default}
\sphinxAtStartPar
{\hyperref[\detokenize{namelists/ancillaries.nml:JULES_RIVERS_PROPS::nx}]{\sphinxcrossref{\sphinxcode{\sphinxupquote{JULES\_RIVERS\_PROPS::nx}}}}}

\end{description}\end{quote}

\sphinxAtStartPar
The size of the x dimension of the 2D regular lat/lon grid containing the model input grid.

\end{fulllineitems}

\index{ny\_grid (in namelist JULES\_RIVERS\_PROPS)@\spxentry{ny\_grid}\spxextra{in namelist JULES\_RIVERS\_PROPS}|spxpagem}

\begin{fulllineitems}
\phantomsection\label{\detokenize{namelists/ancillaries.nml:JULES_RIVERS_PROPS::ny_grid}}
\pysigstartsignatures
\pysigline{\sphinxcode{\sphinxupquote{JULES\_RIVERS\_PROPS::}}\sphinxbfcode{\sphinxupquote{ny\_grid}}}
\pysigstopsignatures\begin{quote}\begin{description}
\sphinxlineitem{Type}
\sphinxAtStartPar
integer

\sphinxlineitem{Permitted}
\sphinxAtStartPar
\textgreater{}= 1

\sphinxlineitem{Default}
\sphinxAtStartPar
{\hyperref[\detokenize{namelists/ancillaries.nml:JULES_RIVERS_PROPS::ny}]{\sphinxcrossref{\sphinxcode{\sphinxupquote{JULES\_RIVERS\_PROPS::ny}}}}}

\end{description}\end{quote}

\sphinxAtStartPar
The size of the y dimension of the 2D regular lat/lon grid containing the model input grid.

\end{fulllineitems}

\index{reg\_lat1 (in namelist JULES\_RIVERS\_PROPS)@\spxentry{reg\_lat1}\spxextra{in namelist JULES\_RIVERS\_PROPS}|spxpagem}

\begin{fulllineitems}
\phantomsection\label{\detokenize{namelists/ancillaries.nml:JULES_RIVERS_PROPS::reg_lat1}}
\pysigstartsignatures
\pysigline{\sphinxcode{\sphinxupquote{JULES\_RIVERS\_PROPS::}}\sphinxbfcode{\sphinxupquote{reg\_lat1}}}
\pysigstopsignatures\begin{quote}\begin{description}
\sphinxlineitem{Type}
\sphinxAtStartPar
real

\sphinxlineitem{Default}
\sphinxAtStartPar
Latitude of lower\sphinxhyphen{}left corner of river routing input grid

\end{description}\end{quote}

\sphinxAtStartPar
The latitude of the lower\sphinxhyphen{}left corner of the 2D regular lat\sphinxhyphen{}lon grid containing the model input grid.

\end{fulllineitems}

\index{reg\_lon1 (in namelist JULES\_RIVERS\_PROPS)@\spxentry{reg\_lon1}\spxextra{in namelist JULES\_RIVERS\_PROPS}|spxpagem}

\begin{fulllineitems}
\phantomsection\label{\detokenize{namelists/ancillaries.nml:JULES_RIVERS_PROPS::reg_lon1}}
\pysigstartsignatures
\pysigline{\sphinxcode{\sphinxupquote{JULES\_RIVERS\_PROPS::}}\sphinxbfcode{\sphinxupquote{reg\_lon1}}}
\pysigstopsignatures\begin{quote}\begin{description}
\sphinxlineitem{Type}
\sphinxAtStartPar
real

\sphinxlineitem{Default}
\sphinxAtStartPar
Longitude of lower\sphinxhyphen{}left corner of river routing input grid

\end{description}\end{quote}

\sphinxAtStartPar
The longitude of the lower\sphinxhyphen{}left corner of the 2D regular lat/lon grid containing the model input grid.

\end{fulllineitems}

\index{reg\_dlat (in namelist JULES\_RIVERS\_PROPS)@\spxentry{reg\_dlat}\spxextra{in namelist JULES\_RIVERS\_PROPS}|spxpagem}

\begin{fulllineitems}
\phantomsection\label{\detokenize{namelists/ancillaries.nml:JULES_RIVERS_PROPS::reg_dlat}}
\pysigstartsignatures
\pysigline{\sphinxcode{\sphinxupquote{JULES\_RIVERS\_PROPS::}}\sphinxbfcode{\sphinxupquote{reg\_dlat}}}
\pysigstopsignatures\begin{quote}\begin{description}
\sphinxlineitem{Type}
\sphinxAtStartPar
real

\sphinxlineitem{Default}
\sphinxAtStartPar
Latitude spacing of river routing input grid

\end{description}\end{quote}

\sphinxAtStartPar
The latitude spacing of the 2D regular lat/lon grid containing the model input grid.

\end{fulllineitems}

\index{reg\_dlon (in namelist JULES\_RIVERS\_PROPS)@\spxentry{reg\_dlon}\spxextra{in namelist JULES\_RIVERS\_PROPS}|spxpagem}

\begin{fulllineitems}
\phantomsection\label{\detokenize{namelists/ancillaries.nml:JULES_RIVERS_PROPS::reg_dlon}}
\pysigstartsignatures
\pysigline{\sphinxcode{\sphinxupquote{JULES\_RIVERS\_PROPS::}}\sphinxbfcode{\sphinxupquote{reg\_dlon}}}
\pysigstopsignatures\begin{quote}\begin{description}
\sphinxlineitem{Type}
\sphinxAtStartPar
real

\sphinxlineitem{Default}
\sphinxAtStartPar
Longitude spacing of river routing input grid

\end{description}\end{quote}

\sphinxAtStartPar
The longitude spacing of the 2D regular lat/lon grid containing the model input grid.

\end{fulllineitems}

\end{sphinxadmonition}
\end{sphinxadmonition}

\begin{sphinxadmonition}{note}{Only used when \sphinxstyleliteralintitle{\sphinxupquote{rivers\_reglatlon}} = FALSE}
\index{rivers\_dx (in namelist JULES\_RIVERS\_PROPS)@\spxentry{rivers\_dx}\spxextra{in namelist JULES\_RIVERS\_PROPS}|spxpagem}

\begin{fulllineitems}
\phantomsection\label{\detokenize{namelists/ancillaries.nml:JULES_RIVERS_PROPS::rivers_dx}}
\pysigstartsignatures
\pysigline{\sphinxcode{\sphinxupquote{JULES\_RIVERS\_PROPS::}}\sphinxbfcode{\sphinxupquote{rivers\_dx}}}
\pysigstopsignatures\begin{quote}\begin{description}
\sphinxlineitem{Type}
\sphinxAtStartPar
real

\sphinxlineitem{Permitted}
\sphinxAtStartPar
\textgreater{} 0

\sphinxlineitem{Default}
\sphinxAtStartPar
0

\end{description}\end{quote}

\sphinxAtStartPar
The constant size of the rivers grid (in m) if non\sphinxhyphen{}regular in latitude/longitude (e.g. if defined in Ordnance Survey (British) National Grid (BNG) OSGB36 coordinates).

\end{fulllineitems}

\end{sphinxadmonition}

\begin{sphinxadmonition}{note}{Members used to determine how river routing variables are set}
\index{file (in namelist JULES\_RIVERS\_PROPS)@\spxentry{file}\spxextra{in namelist JULES\_RIVERS\_PROPS}|spxpagem}

\begin{fulllineitems}
\phantomsection\label{\detokenize{namelists/ancillaries.nml:JULES_RIVERS_PROPS::file}}
\pysigstartsignatures
\pysigline{\sphinxcode{\sphinxupquote{JULES\_RIVERS\_PROPS::}}\sphinxbfcode{\sphinxupquote{file}}}
\pysigstopsignatures\begin{quote}\begin{description}
\sphinxlineitem{Type}
\sphinxAtStartPar
character

\sphinxlineitem{Default}
\sphinxAtStartPar
None

\end{description}\end{quote}

\sphinxAtStartPar
The file to read river routing properties from.

\sphinxAtStartPar
If {\hyperref[\detokenize{namelists/ancillaries.nml:JULES_RIVERS_PROPS::use_file}]{\sphinxcrossref{\sphinxcode{\sphinxupquote{use\_file}}}}} is FALSE for every variable, this will not be used.

\sphinxAtStartPar
This file name can use {\hyperref[\detokenize{input/file-name-templating::doc}]{\sphinxcrossref{\DUrole{doc}{variable name templating}}}}.

\end{fulllineitems}

\index{nvars (in namelist JULES\_RIVERS\_PROPS)@\spxentry{nvars}\spxextra{in namelist JULES\_RIVERS\_PROPS}|spxpagem}

\begin{fulllineitems}
\phantomsection\label{\detokenize{namelists/ancillaries.nml:JULES_RIVERS_PROPS::nvars}}
\pysigstartsignatures
\pysigline{\sphinxcode{\sphinxupquote{JULES\_RIVERS\_PROPS::}}\sphinxbfcode{\sphinxupquote{nvars}}}
\pysigstopsignatures\begin{quote}\begin{description}
\sphinxlineitem{Type}
\sphinxAtStartPar
integer

\sphinxlineitem{Permitted}
\sphinxAtStartPar
\textgreater{}= 0

\sphinxlineitem{Default}
\sphinxAtStartPar
0

\end{description}\end{quote}

\sphinxAtStartPar
The number of river routing property variables that will be provided (see {\hyperref[\detokenize{namelists/ancillaries.nml:list-of-rivers-params}]{\sphinxcrossref{\DUrole{std,std-ref}{List of rivers properties}}}}).
\begin{itemize}
\item {} 
\sphinxAtStartPar
For RFM, at least direction is currently required

\item {} 
\sphinxAtStartPar
For TRIP, at least direction and sequence are required

\end{itemize}

\end{fulllineitems}

\index{var (in namelist JULES\_RIVERS\_PROPS)@\spxentry{var}\spxextra{in namelist JULES\_RIVERS\_PROPS}|spxpagem}

\begin{fulllineitems}
\phantomsection\label{\detokenize{namelists/ancillaries.nml:JULES_RIVERS_PROPS::var}}
\pysigstartsignatures
\pysigline{\sphinxcode{\sphinxupquote{JULES\_RIVERS\_PROPS::}}\sphinxbfcode{\sphinxupquote{var}}}
\pysigstopsignatures\begin{quote}\begin{description}
\sphinxlineitem{Type}
\sphinxAtStartPar
character(nvars)

\sphinxlineitem{Default}
\sphinxAtStartPar
None

\end{description}\end{quote}

\sphinxAtStartPar
List of river routing variable names as recognised by JULES (see {\hyperref[\detokenize{namelists/ancillaries.nml:list-of-rivers-params}]{\sphinxcrossref{\DUrole{std,std-ref}{List of rivers properties}}}}). Names are case sensitive.

\begin{sphinxadmonition}{note}{Note:}
\sphinxAtStartPar
For ASCII files, variable names must be in the order they appear in the file.
\end{sphinxadmonition}

\end{fulllineitems}

\index{use\_file (in namelist JULES\_RIVERS\_PROPS)@\spxentry{use\_file}\spxextra{in namelist JULES\_RIVERS\_PROPS}|spxpagem}

\begin{fulllineitems}
\phantomsection\label{\detokenize{namelists/ancillaries.nml:JULES_RIVERS_PROPS::use_file}}
\pysigstartsignatures
\pysigline{\sphinxcode{\sphinxupquote{JULES\_RIVERS\_PROPS::}}\sphinxbfcode{\sphinxupquote{use\_file}}}
\pysigstopsignatures\begin{quote}\begin{description}
\sphinxlineitem{Type}
\sphinxAtStartPar
logical(nvars)

\sphinxlineitem{Default}
\sphinxAtStartPar
T

\end{description}\end{quote}

\sphinxAtStartPar
For each JULES variable specified in {\hyperref[\detokenize{namelists/ancillaries.nml:JULES_RIVERS_PROPS::var}]{\sphinxcrossref{\sphinxcode{\sphinxupquote{var}}}}}, this indicates if it should be read from the specified file or whether a constant value is to be used.
\begin{description}
\sphinxlineitem{TRUE}
\sphinxAtStartPar
The variable will be read from the file.

\sphinxlineitem{FALSE}
\sphinxAtStartPar
The variable will be set to a constant value everywhere using {\hyperref[\detokenize{namelists/ancillaries.nml:JULES_RIVERS_PROPS::const_val}]{\sphinxcrossref{\sphinxcode{\sphinxupquote{const\_val}}}}} below.

\end{description}

\end{fulllineitems}

\index{var\_name (in namelist JULES\_RIVERS\_PROPS)@\spxentry{var\_name}\spxextra{in namelist JULES\_RIVERS\_PROPS}|spxpagem}

\begin{fulllineitems}
\phantomsection\label{\detokenize{namelists/ancillaries.nml:JULES_RIVERS_PROPS::var_name}}
\pysigstartsignatures
\pysigline{\sphinxcode{\sphinxupquote{JULES\_RIVERS\_PROPS::}}\sphinxbfcode{\sphinxupquote{var\_name}}}
\pysigstopsignatures\begin{quote}\begin{description}
\sphinxlineitem{Type}
\sphinxAtStartPar
character(nvars)

\sphinxlineitem{Default}
\sphinxAtStartPar
‘’ (empty string)

\end{description}\end{quote}

\sphinxAtStartPar
For each JULES variable specified in {\hyperref[\detokenize{namelists/ancillaries.nml:JULES_RIVERS_PROPS::var}]{\sphinxcrossref{\sphinxcode{\sphinxupquote{var}}}}} where {\hyperref[\detokenize{namelists/ancillaries.nml:JULES_RIVERS_PROPS::use_file}]{\sphinxcrossref{\sphinxcode{\sphinxupquote{use\_file}}}}} = TRUE, this is the name of the variable in the file containing the data.

\sphinxAtStartPar
If the empty string (the default) is given for any variable, then the corresponding value from {\hyperref[\detokenize{namelists/ancillaries.nml:JULES_RIVERS_PROPS::var}]{\sphinxcrossref{\sphinxcode{\sphinxupquote{var}}}}} is used instead.

\sphinxAtStartPar
This is not used for variables where {\hyperref[\detokenize{namelists/ancillaries.nml:JULES_RIVERS_PROPS::use_file}]{\sphinxcrossref{\sphinxcode{\sphinxupquote{use\_file}}}}} = FALSE, but a placeholder must still be given in that case.

\begin{sphinxadmonition}{note}{Note:}
\sphinxAtStartPar
For ASCII files, this is not used \sphinxhyphen{} only the order in the file matters, as described above.
\end{sphinxadmonition}

\end{fulllineitems}

\index{tpl\_name (in namelist JULES\_RIVERS\_PROPS)@\spxentry{tpl\_name}\spxextra{in namelist JULES\_RIVERS\_PROPS}|spxpagem}

\begin{fulllineitems}
\phantomsection\label{\detokenize{namelists/ancillaries.nml:JULES_RIVERS_PROPS::tpl_name}}
\pysigstartsignatures
\pysigline{\sphinxcode{\sphinxupquote{JULES\_RIVERS\_PROPS::}}\sphinxbfcode{\sphinxupquote{tpl\_name}}}
\pysigstopsignatures\begin{quote}\begin{description}
\sphinxlineitem{Type}
\sphinxAtStartPar
character(nvars)

\sphinxlineitem{Default}
\sphinxAtStartPar
None

\end{description}\end{quote}

\sphinxAtStartPar
For each JULES variable specified in {\hyperref[\detokenize{namelists/ancillaries.nml:JULES_RIVERS_PROPS::var}]{\sphinxcrossref{\sphinxcode{\sphinxupquote{var}}}}}, this is the string to substitute into the file name in place of the variable name substitution string.

\sphinxAtStartPar
If the file name does not use variable name templating, this is not used.

\end{fulllineitems}

\index{const\_val (in namelist JULES\_RIVERS\_PROPS)@\spxentry{const\_val}\spxextra{in namelist JULES\_RIVERS\_PROPS}|spxpagem}

\begin{fulllineitems}
\phantomsection\label{\detokenize{namelists/ancillaries.nml:JULES_RIVERS_PROPS::const_val}}
\pysigstartsignatures
\pysigline{\sphinxcode{\sphinxupquote{JULES\_RIVERS\_PROPS::}}\sphinxbfcode{\sphinxupquote{const\_val}}}
\pysigstopsignatures\begin{quote}\begin{description}
\sphinxlineitem{Type}
\sphinxAtStartPar
real(nvars)

\sphinxlineitem{Default}
\sphinxAtStartPar
None

\end{description}\end{quote}

\sphinxAtStartPar
For each JULES variable specified in {\hyperref[\detokenize{namelists/ancillaries.nml:JULES_RIVERS_PROPS::var}]{\sphinxcrossref{\sphinxcode{\sphinxupquote{var}}}}} where {\hyperref[\detokenize{namelists/ancillaries.nml:JULES_RIVERS_PROPS::use_file}]{\sphinxcrossref{\sphinxcode{\sphinxupquote{use\_file}}}}} = FALSE, this is a constant value that the variable will be set to at every point.

\sphinxAtStartPar
This is not used for variables where {\hyperref[\detokenize{namelists/ancillaries.nml:JULES_RIVERS_PROPS::use_file}]{\sphinxcrossref{\sphinxcode{\sphinxupquote{use\_file}}}}} = TRUE, but a placeholder must still be given in that case.

\end{fulllineitems}

\end{sphinxadmonition}


\subsubsection{Example}
\label{\detokenize{namelists/ancillaries.nml:example}}
\sphinxAtStartPar
The following gives an example of how you would set up the namelists to use routing with the GSWP2 forcing data.

\sphinxAtStartPar
The model input grid is the GSWP2 grid, i.e. a land\sphinxhyphen{}points\sphinxhyphen{}only, 1D grid where points lie on a 1° x 1° grid. The river routing input grid is a 2D 1° x 1° grid.

\sphinxAtStartPar
Since both grids are 1° x 1°, we define the 2D regular lat\sphinxhyphen{}lon grid containing the model input grid to be the river routing input grid, which means we don’t need any regridding of variables.

\begin{sphinxVerbatim}[commandchars=\\\{\}]
\PYG{n+nn}{\PYGZam{}JULES\PYGZus{}INPUT\PYGZus{}GRID}
  \PYG{n+nv}{grid\PYGZus{}is\PYGZus{}1d}    \PYG{o}{=} \PYG{l+s+ss}{T}\PYG{p}{,}
  \PYG{n+nv}{npoints}       \PYG{o}{=} \PYG{l+m+mi}{15238}\PYG{p}{,}
  \PYG{n+nv}{grid\PYGZus{}dim\PYGZus{}name} \PYG{o}{=} \PYG{l+s+s2}{\PYGZdq{}land\PYGZdq{}}
  \PYG{c+c1}{\PYGZsh{} ...}
\PYG{n+nn}{/}

\PYG{c+c1}{\PYGZsh{} ...}

\PYG{n+nn}{\PYGZam{}JULES\PYGZus{}RIVERS\PYGZus{}PROPS}
  \PYG{c+c1}{\PYGZsh{} Define the river routing input grid to be a 2D regular lat\PYGZhy{}lon grid}
  \PYG{n+nv}{rivers\PYGZus{}reglatlon} \PYG{o}{=} \PYG{l+s+ss}{T}\PYG{p}{,}
  \PYG{n+nv}{x\PYGZus{}dim\PYGZus{}name} \PYG{o}{=} \PYG{l+s+s2}{\PYGZdq{}longitude\PYGZdq{}}\PYG{p}{,}
  \PYG{n+nv}{nx}         \PYG{o}{=} \PYG{l+m+mi}{360}\PYG{p}{,}
  \PYG{n+nv}{y\PYGZus{}dim\PYGZus{}name} \PYG{o}{=} \PYG{l+s+s2}{\PYGZdq{}latitude\PYGZdq{}}\PYG{p}{,}
  \PYG{n+nv}{ny}         \PYG{o}{=} \PYG{l+m+mi}{180}\PYG{p}{,}

  \PYG{c+c1}{\PYGZsh{} Define the 2D regular lat\PYGZhy{}lon grid containing the model input grid to be a 2D 1\PYGZbs{} |deg| x 1\PYGZbs{} |deg| grid}
  \PYG{n+nv}{nx\PYGZus{}grid}  \PYG{o}{=} \PYG{l+m+mi}{360}\PYG{p}{,}
  \PYG{n+nv}{ny\PYGZus{}grid}  \PYG{o}{=} \PYG{l+m+mi}{180}\PYG{p}{,}
  \PYG{n+nv}{reg\PYGZus{}lat1} \PYG{o}{=} \PYG{l+m+mf}{\PYGZhy{}89.5}\PYG{p}{,}
  \PYG{n+nv}{reg\PYGZus{}lon1} \PYG{o}{=} \PYG{l+m+mf}{\PYGZhy{}179.5}\PYG{p}{,}
  \PYG{n+nv}{reg\PYGZus{}dlat} \PYG{o}{=} \PYG{l+m+mf}{1.0}\PYG{p}{,}
  \PYG{n+nv}{reg\PYGZus{}dlon} \PYG{o}{=} \PYG{l+m+mf}{1.0}\PYG{p}{,}

  \PYG{c+c1}{\PYGZsh{} No regridding required since the river routing input grid is the same as the 2D regular lat\PYGZhy{}lon grid containing the model input grid}
  \PYG{n+nv}{rivers\PYGZus{}regrid} \PYG{o}{=} \PYG{l+s+ss}{F}
\PYG{n+nn}{/}
\end{sphinxVerbatim}


\subsubsection{List of rivers properties}
\label{\detokenize{namelists/ancillaries.nml:list-of-rivers-properties}}\label{\detokenize{namelists/ancillaries.nml:list-of-rivers-params}}
\sphinxAtStartPar
The following table summarises river routing input grid properties required to run RFM or TRIP river routing algorithms, specified from an ancillary file if {\hyperref[\detokenize{namelists/ancillaries.nml:JULES_RIVERS_PROPS::use_file}]{\sphinxcrossref{\sphinxcode{\sphinxupquote{use\_file}}}}} = TRUE.


\begin{savenotes}\sphinxattablestart
\centering
\begin{tabulary}{\linewidth}[t]{|p{2.5cm}|L|}
\hline
\sphinxstyletheadfamily 
\sphinxAtStartPar
Name
&\sphinxstyletheadfamily 
\sphinxAtStartPar
Description
\\
\hline
\sphinxAtStartPar
\sphinxcode{\sphinxupquote{area}}
&
\sphinxAtStartPar
Drainage area (in number of grid boxes) draining into a given grid box.

\sphinxAtStartPar
This is used if {\hyperref[\detokenize{namelists/jules_rivers.nml:JULES_RIVERS::i_river_vn}]{\sphinxcrossref{\sphinxcode{\sphinxupquote{i\_river\_vn}}}}} = \sphinxcode{\sphinxupquote{2}} to distinguish between river and land points
using the {\hyperref[\detokenize{namelists/jules_rivers.nml:JULES_RIVERS::a_thresh}]{\sphinxcrossref{\sphinxcode{\sphinxupquote{a\_thresh}}}}} parameter. Points with drainage area \textgreater{}
{\hyperref[\detokenize{namelists/jules_rivers.nml:JULES_RIVERS::a_thresh}]{\sphinxcrossref{\sphinxcode{\sphinxupquote{a\_thresh}}}}} are treated as rivers, all others as land. These two classes of points
use different wave speeds (e.g. {\hyperref[\detokenize{namelists/jules_rivers.nml:JULES_RIVERS::cland}]{\sphinxcrossref{\sphinxcode{\sphinxupquote{cland}}}}} and {\hyperref[\detokenize{namelists/jules_rivers.nml:JULES_RIVERS::criver}]{\sphinxcrossref{\sphinxcode{\sphinxupquote{criver}}}}}).

\sphinxAtStartPar
If this field is not available, all points are treated as rivers.

\sphinxAtStartPar
The drainage area does not include the grid point itself, and so an extra point must be added where
catchment area calculations are required.
\\
\hline
\sphinxAtStartPar
\sphinxcode{\sphinxupquote{direction}}
&
\sphinxAtStartPar
Flow direction for each river routing grid box, defining the next grid box into which surface or
sub\sphinxhyphen{}surface water will be routed.

\sphinxAtStartPar
This is specified as an integer according to \sphinxcode{\sphinxupquote{{[}1 = N, 2 = NE, 3 = E, 4 = SE, 5 = S, 6 = SW, 7 = W,
8 = NW{]}}}.

\sphinxAtStartPar
Although these are referred to via compass directions, they are used as “grid\sphinxhyphen{}relative” directions.
e.g. “N” means “same column, one row up”, “E” means “one column over, same row”.
Thus for a rotated grid (columns do not run S\sphinxhyphen{}N on the Earth), the point “same column, one row up” does
does not lie immediately N.

\sphinxAtStartPar
Additionally,

\sphinxAtStartPar
9: river mouth (outflow to sea)

\sphinxAtStartPar
10: inland drainage point (an endorheic catchment; no outflow from grid box)

\sphinxAtStartPar
All other values (\textless{}1 or \textgreater{}10) are excluded from the river calculations (effectively treated as sea).

\sphinxAtStartPar
Note that at present any river flow at an inland drainage point is NOT added to the soil moisture (in
standalone JULES).
\\
\hline
\sphinxAtStartPar
\sphinxcode{\sphinxupquote{sequence}}
&
\sphinxAtStartPar
River routing network pathway number.

\sphinxAtStartPar
Used by TRIP river routing only (i.e. {\hyperref[\detokenize{namelists/jules_rivers.nml:JULES_RIVERS::i_river_vn}]{\sphinxcrossref{\sphinxcode{\sphinxupquote{i\_river\_vn}}}}} = \sphinxcode{\sphinxupquote{1,3}}).
See Oki et al. (1999) for details.
\\
\hline
\sphinxAtStartPar
\sphinxcode{\sphinxupquote{latitude\_2d}}
&
\sphinxAtStartPar
If {\hyperref[\detokenize{namelists/ancillaries.nml:JULES_RIVERS_PROPS::rivers_reglatlon}]{\sphinxcrossref{\sphinxcode{\sphinxupquote{rivers\_reglatlon}}}}} = FALSE, the unique 2D location of each river grid point must be specified.
\\
\hline
\sphinxAtStartPar
\sphinxcode{\sphinxupquote{longitude\_2d}}
&
\sphinxAtStartPar
If {\hyperref[\detokenize{namelists/ancillaries.nml:JULES_RIVERS_PROPS::rivers_reglatlon}]{\sphinxcrossref{\sphinxcode{\sphinxupquote{rivers\_reglatlon}}}}} = FALSE, the unique 2D location of each river grid point must be specified.
\\
\hline
\end{tabulary}
\par
\sphinxattableend\end{savenotes}


\sphinxstrong{See also:}
\nopagebreak


\sphinxAtStartPar
References:
\begin{itemize}
\item {} 
\sphinxAtStartPar
Bell, V.A. et al. (2007) Development of a high resolution grid\sphinxhyphen{}based river flow model for use with regional climate model output. Hydrology and Earth System Sciences. 11 532\sphinxhyphen{}549

\item {} 
\sphinxAtStartPar
Oki, T., et al (1999) Assessment of annual runoff from land surface models using Total Runoff Integrating Pathways (TRIP). Journal of the Meteorological Society of Japan. 77 235\sphinxhyphen{}255

\end{itemize}




\subsection{\sphinxstyleliteralintitle{\sphinxupquote{JULES\_OVERBANK\_PROPS}} namelist members}
\label{\detokenize{namelists/ancillaries.nml:namelist-JULES_OVERBANK_PROPS}}\label{\detokenize{namelists/ancillaries.nml:jules-overbank-props-namelist-members}}\index{JULES\_OVERBANK\_PROPS (namelist)@\spxentry{JULES\_OVERBANK\_PROPS}\spxextra{namelist}|spxpagem}
\sphinxAtStartPar
This namelist specifies how the river overbank inundation properties should be set.

\begin{sphinxadmonition}{note}{Note:}
\sphinxAtStartPar
\sphinxcode{\sphinxupquote{read\_from\_dump}} is not currently implemented for this namelist.
\end{sphinxadmonition}

\begin{sphinxadmonition}{note}{Note:}
\sphinxAtStartPar
The grid here MUST coincide exactly with the river routing input grid specified in {\hyperref[\detokenize{namelists/ancillaries.nml:namelist-JULES_RIVERS_PROPS}]{\sphinxcrossref{\sphinxcode{\sphinxupquote{JULES\_RIVERS\_PROPS}}}}}.
\end{sphinxadmonition}

\begin{sphinxadmonition}{note}{Members used to determine how overbank inundation variables are set}
\index{file (in namelist JULES\_OVERBANK\_PROPS)@\spxentry{file}\spxextra{in namelist JULES\_OVERBANK\_PROPS}|spxpagem}

\begin{fulllineitems}
\phantomsection\label{\detokenize{namelists/ancillaries.nml:JULES_OVERBANK_PROPS::file}}
\pysigstartsignatures
\pysigline{\sphinxcode{\sphinxupquote{JULES\_OVERBANK\_PROPS::}}\sphinxbfcode{\sphinxupquote{file}}}
\pysigstopsignatures\begin{quote}\begin{description}
\sphinxlineitem{Type}
\sphinxAtStartPar
character

\sphinxlineitem{Default}
\sphinxAtStartPar
None

\end{description}\end{quote}

\sphinxAtStartPar
The file to read overbank inundation properties from (can be the same file as specified in {\hyperref[\detokenize{namelists/ancillaries.nml:namelist-JULES_RIVERS_PROPS}]{\sphinxcrossref{\sphinxcode{\sphinxupquote{JULES\_RIVERS\_PROPS}}}}}).

\sphinxAtStartPar
If {\hyperref[\detokenize{namelists/ancillaries.nml:JULES_OVERBANK_PROPS::use_file}]{\sphinxcrossref{\sphinxcode{\sphinxupquote{use\_file}}}}} is FALSE for every variable, this will not be used.

\sphinxAtStartPar
This file name can use {\hyperref[\detokenize{input/file-name-templating::doc}]{\sphinxcrossref{\DUrole{doc}{variable name templating}}}}.

\end{fulllineitems}

\index{nvars (in namelist JULES\_OVERBANK\_PROPS)@\spxentry{nvars}\spxextra{in namelist JULES\_OVERBANK\_PROPS}|spxpagem}

\begin{fulllineitems}
\phantomsection\label{\detokenize{namelists/ancillaries.nml:JULES_OVERBANK_PROPS::nvars}}
\pysigstartsignatures
\pysigline{\sphinxcode{\sphinxupquote{JULES\_OVERBANK\_PROPS::}}\sphinxbfcode{\sphinxupquote{nvars}}}
\pysigstopsignatures\begin{quote}\begin{description}
\sphinxlineitem{Type}
\sphinxAtStartPar
integer

\sphinxlineitem{Permitted}
\sphinxAtStartPar
\textgreater{}= 0

\sphinxlineitem{Default}
\sphinxAtStartPar
0

\end{description}\end{quote}

\sphinxAtStartPar
The number of overbank inundation property variables that will be provided (see {\hyperref[\detokenize{namelists/ancillaries.nml:list-of-overbank-params}]{\sphinxcrossref{\DUrole{std,std-ref}{List of overbank inundation properties}}}}).

\end{fulllineitems}

\index{var (in namelist JULES\_OVERBANK\_PROPS)@\spxentry{var}\spxextra{in namelist JULES\_OVERBANK\_PROPS}|spxpagem}

\begin{fulllineitems}
\phantomsection\label{\detokenize{namelists/ancillaries.nml:JULES_OVERBANK_PROPS::var}}
\pysigstartsignatures
\pysigline{\sphinxcode{\sphinxupquote{JULES\_OVERBANK\_PROPS::}}\sphinxbfcode{\sphinxupquote{var}}}
\pysigstopsignatures\begin{quote}\begin{description}
\sphinxlineitem{Type}
\sphinxAtStartPar
character(nvars)

\sphinxlineitem{Default}
\sphinxAtStartPar
None

\end{description}\end{quote}

\sphinxAtStartPar
List of overbank inundation variable names as recognised by JULES (see {\hyperref[\detokenize{namelists/ancillaries.nml:list-of-overbank-params}]{\sphinxcrossref{\DUrole{std,std-ref}{List of overbank inundation properties}}}}). Names are case sensitive.

\begin{sphinxadmonition}{note}{Note:}
\sphinxAtStartPar
For ASCII files, variable names must be in the order they appear in the file.
\end{sphinxadmonition}

\end{fulllineitems}

\index{use\_file (in namelist JULES\_OVERBANK\_PROPS)@\spxentry{use\_file}\spxextra{in namelist JULES\_OVERBANK\_PROPS}|spxpagem}

\begin{fulllineitems}
\phantomsection\label{\detokenize{namelists/ancillaries.nml:JULES_OVERBANK_PROPS::use_file}}
\pysigstartsignatures
\pysigline{\sphinxcode{\sphinxupquote{JULES\_OVERBANK\_PROPS::}}\sphinxbfcode{\sphinxupquote{use\_file}}}
\pysigstopsignatures\begin{quote}\begin{description}
\sphinxlineitem{Type}
\sphinxAtStartPar
logical(nvars)

\sphinxlineitem{Default}
\sphinxAtStartPar
T

\end{description}\end{quote}

\sphinxAtStartPar
For each JULES variable specified in {\hyperref[\detokenize{namelists/ancillaries.nml:JULES_OVERBANK_PROPS::var}]{\sphinxcrossref{\sphinxcode{\sphinxupquote{var}}}}}, this indicates if it should be read from the specified file or whether a constant value is to be used.
\begin{description}
\sphinxlineitem{TRUE}
\sphinxAtStartPar
The variable will be read from the file.

\sphinxlineitem{FALSE}
\sphinxAtStartPar
The variable will be set to a constant value everywhere using {\hyperref[\detokenize{namelists/ancillaries.nml:JULES_OVERBANK_PROPS::const_val}]{\sphinxcrossref{\sphinxcode{\sphinxupquote{const\_val}}}}} below.

\end{description}

\end{fulllineitems}

\index{var\_name (in namelist JULES\_OVERBANK\_PROPS)@\spxentry{var\_name}\spxextra{in namelist JULES\_OVERBANK\_PROPS}|spxpagem}

\begin{fulllineitems}
\phantomsection\label{\detokenize{namelists/ancillaries.nml:JULES_OVERBANK_PROPS::var_name}}
\pysigstartsignatures
\pysigline{\sphinxcode{\sphinxupquote{JULES\_OVERBANK\_PROPS::}}\sphinxbfcode{\sphinxupquote{var\_name}}}
\pysigstopsignatures\begin{quote}\begin{description}
\sphinxlineitem{Type}
\sphinxAtStartPar
character(nvars)

\sphinxlineitem{Default}
\sphinxAtStartPar
‘’ (empty string)

\end{description}\end{quote}

\sphinxAtStartPar
For each JULES variable specified in {\hyperref[\detokenize{namelists/ancillaries.nml:JULES_OVERBANK_PROPS::var}]{\sphinxcrossref{\sphinxcode{\sphinxupquote{var}}}}} where {\hyperref[\detokenize{namelists/ancillaries.nml:JULES_OVERBANK_PROPS::use_file}]{\sphinxcrossref{\sphinxcode{\sphinxupquote{use\_file}}}}} = TRUE, this is the name of the variable in the file containing the data.

\sphinxAtStartPar
If the empty string (the default) is given for any variable, then the corresponding value from {\hyperref[\detokenize{namelists/ancillaries.nml:JULES_OVERBANK_PROPS::var}]{\sphinxcrossref{\sphinxcode{\sphinxupquote{var}}}}} is used instead.

\sphinxAtStartPar
This is not used for variables where {\hyperref[\detokenize{namelists/ancillaries.nml:JULES_OVERBANK_PROPS::use_file}]{\sphinxcrossref{\sphinxcode{\sphinxupquote{use\_file}}}}} = FALSE, but a placeholder must still be given in that case.

\begin{sphinxadmonition}{note}{Note:}
\sphinxAtStartPar
For ASCII files, this is not used \sphinxhyphen{} only the order in the file matters, as described above.
\end{sphinxadmonition}

\end{fulllineitems}

\index{tpl\_name (in namelist JULES\_OVERBANK\_PROPS)@\spxentry{tpl\_name}\spxextra{in namelist JULES\_OVERBANK\_PROPS}|spxpagem}

\begin{fulllineitems}
\phantomsection\label{\detokenize{namelists/ancillaries.nml:JULES_OVERBANK_PROPS::tpl_name}}
\pysigstartsignatures
\pysigline{\sphinxcode{\sphinxupquote{JULES\_OVERBANK\_PROPS::}}\sphinxbfcode{\sphinxupquote{tpl\_name}}}
\pysigstopsignatures\begin{quote}\begin{description}
\sphinxlineitem{Type}
\sphinxAtStartPar
character(nvars)

\sphinxlineitem{Default}
\sphinxAtStartPar
None

\end{description}\end{quote}

\sphinxAtStartPar
For each JULES variable specified in {\hyperref[\detokenize{namelists/ancillaries.nml:JULES_OVERBANK_PROPS::var}]{\sphinxcrossref{\sphinxcode{\sphinxupquote{var}}}}}, this is the string to substitute into the file name in place of the variable name substitution string.

\sphinxAtStartPar
If the file name does not use variable name templating, this is not used.

\end{fulllineitems}

\index{const\_val (in namelist JULES\_OVERBANK\_PROPS)@\spxentry{const\_val}\spxextra{in namelist JULES\_OVERBANK\_PROPS}|spxpagem}

\begin{fulllineitems}
\phantomsection\label{\detokenize{namelists/ancillaries.nml:JULES_OVERBANK_PROPS::const_val}}
\pysigstartsignatures
\pysigline{\sphinxcode{\sphinxupquote{JULES\_OVERBANK\_PROPS::}}\sphinxbfcode{\sphinxupquote{const\_val}}}
\pysigstopsignatures\begin{quote}\begin{description}
\sphinxlineitem{Type}
\sphinxAtStartPar
real(nvars)

\sphinxlineitem{Default}
\sphinxAtStartPar
None

\end{description}\end{quote}

\sphinxAtStartPar
For each JULES variable specified in {\hyperref[\detokenize{namelists/ancillaries.nml:JULES_OVERBANK_PROPS::var}]{\sphinxcrossref{\sphinxcode{\sphinxupquote{var}}}}} where {\hyperref[\detokenize{namelists/ancillaries.nml:JULES_OVERBANK_PROPS::use_file}]{\sphinxcrossref{\sphinxcode{\sphinxupquote{use\_file}}}}} = FALSE, this is a constant value that the variable will be set to at every point.

\sphinxAtStartPar
This is not used for variables where {\hyperref[\detokenize{namelists/ancillaries.nml:JULES_OVERBANK_PROPS::use_file}]{\sphinxcrossref{\sphinxcode{\sphinxupquote{use\_file}}}}} = TRUE, but a placeholder must still be given in that case.

\end{fulllineitems}

\end{sphinxadmonition}


\subsubsection{List of overbank inundation properties}
\label{\detokenize{namelists/ancillaries.nml:list-of-overbank-inundation-properties}}\label{\detokenize{namelists/ancillaries.nml:list-of-overbank-params}}
\sphinxAtStartPar
The following table summarises overbank inundation grid properties, specified from an ancillary file if {\hyperref[\detokenize{namelists/ancillaries.nml:JULES_OVERBANK_PROPS::use_file}]{\sphinxcrossref{\sphinxcode{\sphinxupquote{use\_file}}}}} = TRUE.


\begin{savenotes}\sphinxattablestart
\centering
\begin{tabulary}{\linewidth}[t]{|p{2.5cm}|L|}
\hline
\sphinxstyletheadfamily 
\sphinxAtStartPar
Name
&\sphinxstyletheadfamily 
\sphinxAtStartPar
Description
\\
\hline
\sphinxAtStartPar
\sphinxcode{\sphinxupquote{logn\_mean}}
&
\sphinxAtStartPar
Mean of ln(elevation\sphinxhyphen{}elev\_min) for each grid cell (in units ln(m))

\sphinxAtStartPar
This is only used if {\hyperref[\detokenize{namelists/jules_rivers.nml:JULES_OVERBANK::l_riv_hypsometry}]{\sphinxcrossref{\sphinxcode{\sphinxupquote{l\_riv\_hypsometry}}}}} = TRUE

\sphinxAtStartPar
Note that elev\_min is DEM minimum, not river/lake bed level (therefore large values close to water
bodies can occur in floodplain gridcells)
\\
\hline
\sphinxAtStartPar
\sphinxcode{\sphinxupquote{logn\_stdev}}
&
\sphinxAtStartPar
Standard deviation of ln(elevation\sphinxhyphen{}elev\_min) for each grid cell (in units ln(m))

\sphinxAtStartPar
This is only used if {\hyperref[\detokenize{namelists/jules_rivers.nml:JULES_OVERBANK::l_riv_hypsometry}]{\sphinxcrossref{\sphinxcode{\sphinxupquote{l\_riv\_hypsometry}}}}} = TRUE
\\
\hline
\end{tabulary}
\par
\sphinxattableend\end{savenotes}


\sphinxstrong{See also:}
\nopagebreak


\sphinxAtStartPar
References:
\begin{itemize}
\item {} 
\sphinxAtStartPar
Appx B of Lewis HW, Castillo Sanchez JM, Graham J, Saulter A, Bornemann J, Arnold A, Fallmann J, Harris C, Pearson D, Ramsdale S, Martínez de la Torre A, Bricheno L, Blyth E, Bell VA, Davies H, Marthews TR, O’Neill C, Rumbold H, O’Dea E, Brereton A, Guihou K, Hines A, Butenschon M, Dadson SJ, Palmer T, Holt J, Reynard N, Best M, Edwards J \& Siddorn J (2018). The UKC2 regional coupled environmental prediction system. Geoscientific Model Development 11:1\sphinxhyphen{}42.

\end{itemize}




\subsection{\sphinxstyleliteralintitle{\sphinxupquote{JULES\_WATER\_RESOURCES\_PROPS}} namelist members}
\label{\detokenize{namelists/ancillaries.nml:namelist-JULES_WATER_RESOURCES_PROPS}}\label{\detokenize{namelists/ancillaries.nml:jules-water-resources-props-namelist-members}}\index{JULES\_WATER\_RESOURCES\_PROPS (namelist)@\spxentry{JULES\_WATER\_RESOURCES\_PROPS}\spxextra{namelist}|spxpagem}
\sphinxAtStartPar
This namelist specifies how the water resource ancillary properties should be set.

\begin{sphinxadmonition}{note}{Members used to determine how water resource variables are set}
\index{read\_from\_dump (in namelist JULES\_WATER\_RESOURCES\_PROPS)@\spxentry{read\_from\_dump}\spxextra{in namelist JULES\_WATER\_RESOURCES\_PROPS}|spxpagem}

\begin{fulllineitems}
\phantomsection\label{\detokenize{namelists/ancillaries.nml:JULES_WATER_RESOURCES_PROPS::read_from_dump}}
\pysigstartsignatures
\pysigline{\sphinxcode{\sphinxupquote{JULES\_WATER\_RESOURCES\_PROPS::}}\sphinxbfcode{\sphinxupquote{read\_from\_dump}}}
\pysigstopsignatures\begin{quote}\begin{description}
\sphinxlineitem{Type}
\sphinxAtStartPar
logical

\sphinxlineitem{Default}
\sphinxAtStartPar
F

\end{description}\end{quote}
\begin{description}
\sphinxlineitem{TRUE}
\sphinxAtStartPar
Populate variables associated with this namelist from the dump file. All other namelist members are ignored.

\sphinxlineitem{FALSE}
\sphinxAtStartPar
Use the other namelist members to determine how to populate variables.

\end{description}

\end{fulllineitems}

\index{file (in namelist JULES\_WATER\_RESOURCES\_PROPS)@\spxentry{file}\spxextra{in namelist JULES\_WATER\_RESOURCES\_PROPS}|spxpagem}

\begin{fulllineitems}
\phantomsection\label{\detokenize{namelists/ancillaries.nml:JULES_WATER_RESOURCES_PROPS::file}}
\pysigstartsignatures
\pysigline{\sphinxcode{\sphinxupquote{JULES\_WATER\_RESOURCES\_PROPS::}}\sphinxbfcode{\sphinxupquote{file}}}
\pysigstopsignatures\begin{quote}\begin{description}
\sphinxlineitem{Type}
\sphinxAtStartPar
character

\sphinxlineitem{Default}
\sphinxAtStartPar
None

\end{description}\end{quote}

\sphinxAtStartPar
The file to read water resource ancillary properties from.

\sphinxAtStartPar
If {\hyperref[\detokenize{namelists/ancillaries.nml:JULES_WATER_RESOURCES_PROPS::use_file}]{\sphinxcrossref{\sphinxcode{\sphinxupquote{use\_file}}}}} is FALSE for every variable, this will not be used.

\sphinxAtStartPar
This file name can use {\hyperref[\detokenize{input/file-name-templating::doc}]{\sphinxcrossref{\DUrole{doc}{variable name templating}}}}.

\end{fulllineitems}

\index{nvars (in namelist JULES\_WATER\_RESOURCES\_PROPS)@\spxentry{nvars}\spxextra{in namelist JULES\_WATER\_RESOURCES\_PROPS}|spxpagem}

\begin{fulllineitems}
\phantomsection\label{\detokenize{namelists/ancillaries.nml:JULES_WATER_RESOURCES_PROPS::nvars}}
\pysigstartsignatures
\pysigline{\sphinxcode{\sphinxupquote{JULES\_WATER\_RESOURCES\_PROPS::}}\sphinxbfcode{\sphinxupquote{nvars}}}
\pysigstopsignatures\begin{quote}\begin{description}
\sphinxlineitem{Type}
\sphinxAtStartPar
integer

\sphinxlineitem{Permitted}
\sphinxAtStartPar
\textgreater{}= 0

\sphinxlineitem{Default}
\sphinxAtStartPar
0

\end{description}\end{quote}

\sphinxAtStartPar
The number of water resource property variables that will be provided (see {\hyperref[\detokenize{namelists/ancillaries.nml:list-of-water-resources-params}]{\sphinxcrossref{\DUrole{std,std-ref}{List of water resources properties}}}}).

\end{fulllineitems}

\index{var (in namelist JULES\_WATER\_RESOURCES\_PROPS)@\spxentry{var}\spxextra{in namelist JULES\_WATER\_RESOURCES\_PROPS}|spxpagem}

\begin{fulllineitems}
\phantomsection\label{\detokenize{namelists/ancillaries.nml:JULES_WATER_RESOURCES_PROPS::var}}
\pysigstartsignatures
\pysigline{\sphinxcode{\sphinxupquote{JULES\_WATER\_RESOURCES\_PROPS::}}\sphinxbfcode{\sphinxupquote{var}}}
\pysigstopsignatures\begin{quote}\begin{description}
\sphinxlineitem{Type}
\sphinxAtStartPar
character(nvars)

\sphinxlineitem{Default}
\sphinxAtStartPar
None

\end{description}\end{quote}

\sphinxAtStartPar
List of water resource variable names as recognised by JULES (see {\hyperref[\detokenize{namelists/ancillaries.nml:list-of-water-resources-params}]{\sphinxcrossref{\DUrole{std,std-ref}{List of water resources properties}}}}). Names are case sensitive.

\begin{sphinxadmonition}{note}{Note:}
\sphinxAtStartPar
For ASCII files, variable names must be in the order they appear in the file.
\end{sphinxadmonition}

\end{fulllineitems}

\index{use\_file (in namelist JULES\_WATER\_RESOURCES\_PROPS)@\spxentry{use\_file}\spxextra{in namelist JULES\_WATER\_RESOURCES\_PROPS}|spxpagem}

\begin{fulllineitems}
\phantomsection\label{\detokenize{namelists/ancillaries.nml:JULES_WATER_RESOURCES_PROPS::use_file}}
\pysigstartsignatures
\pysigline{\sphinxcode{\sphinxupquote{JULES\_WATER\_RESOURCES\_PROPS::}}\sphinxbfcode{\sphinxupquote{use\_file}}}
\pysigstopsignatures\begin{quote}\begin{description}
\sphinxlineitem{Type}
\sphinxAtStartPar
logical(nvars)

\sphinxlineitem{Default}
\sphinxAtStartPar
T

\end{description}\end{quote}

\sphinxAtStartPar
For each JULES variable specified in {\hyperref[\detokenize{namelists/ancillaries.nml:JULES_WATER_RESOURCES_PROPS::var}]{\sphinxcrossref{\sphinxcode{\sphinxupquote{var}}}}}, this indicates if it should be read from the specified file or whether a constant value is to be used.
\begin{description}
\sphinxlineitem{TRUE}
\sphinxAtStartPar
The variable will be read from the file.

\sphinxlineitem{FALSE}
\sphinxAtStartPar
The variable will be set to a constant value everywhere using {\hyperref[\detokenize{namelists/ancillaries.nml:JULES_WATER_RESOURCES_PROPS::const_val}]{\sphinxcrossref{\sphinxcode{\sphinxupquote{const\_val}}}}} below.

\end{description}

\end{fulllineitems}

\index{var\_name (in namelist JULES\_WATER\_RESOURCES\_PROPS)@\spxentry{var\_name}\spxextra{in namelist JULES\_WATER\_RESOURCES\_PROPS}|spxpagem}

\begin{fulllineitems}
\phantomsection\label{\detokenize{namelists/ancillaries.nml:JULES_WATER_RESOURCES_PROPS::var_name}}
\pysigstartsignatures
\pysigline{\sphinxcode{\sphinxupquote{JULES\_WATER\_RESOURCES\_PROPS::}}\sphinxbfcode{\sphinxupquote{var\_name}}}
\pysigstopsignatures\begin{quote}\begin{description}
\sphinxlineitem{Type}
\sphinxAtStartPar
character(nvars)

\sphinxlineitem{Default}
\sphinxAtStartPar
‘’ (empty string)

\end{description}\end{quote}

\sphinxAtStartPar
For each JULES variable specified in {\hyperref[\detokenize{namelists/ancillaries.nml:JULES_WATER_RESOURCES_PROPS::var}]{\sphinxcrossref{\sphinxcode{\sphinxupquote{var}}}}} where {\hyperref[\detokenize{namelists/ancillaries.nml:JULES_WATER_RESOURCES_PROPS::use_file}]{\sphinxcrossref{\sphinxcode{\sphinxupquote{use\_file}}}}} = TRUE, this is the name of the variable in the file containing the data.

\sphinxAtStartPar
If the empty string (the default) is given for any variable, then the corresponding value from {\hyperref[\detokenize{namelists/ancillaries.nml:JULES_WATER_RESOURCES_PROPS::var}]{\sphinxcrossref{\sphinxcode{\sphinxupquote{var}}}}} is used instead.

\sphinxAtStartPar
This is not used for variables where {\hyperref[\detokenize{namelists/ancillaries.nml:JULES_WATER_RESOURCES_PROPS::use_file}]{\sphinxcrossref{\sphinxcode{\sphinxupquote{use\_file}}}}} = FALSE, but a placeholder must still be given in that case.

\begin{sphinxadmonition}{note}{Note:}
\sphinxAtStartPar
For ASCII files, this is not used \sphinxhyphen{} only the order in the file matters, as described above.
\end{sphinxadmonition}

\end{fulllineitems}

\index{tpl\_name (in namelist JULES\_WATER\_RESOURCES\_PROPS)@\spxentry{tpl\_name}\spxextra{in namelist JULES\_WATER\_RESOURCES\_PROPS}|spxpagem}

\begin{fulllineitems}
\phantomsection\label{\detokenize{namelists/ancillaries.nml:JULES_WATER_RESOURCES_PROPS::tpl_name}}
\pysigstartsignatures
\pysigline{\sphinxcode{\sphinxupquote{JULES\_WATER\_RESOURCES\_PROPS::}}\sphinxbfcode{\sphinxupquote{tpl\_name}}}
\pysigstopsignatures\begin{quote}\begin{description}
\sphinxlineitem{Type}
\sphinxAtStartPar
character(nvars)

\sphinxlineitem{Default}
\sphinxAtStartPar
None

\end{description}\end{quote}

\sphinxAtStartPar
For each JULES variable specified in {\hyperref[\detokenize{namelists/ancillaries.nml:JULES_WATER_RESOURCES_PROPS::var}]{\sphinxcrossref{\sphinxcode{\sphinxupquote{var}}}}}, this is the string to substitute into the file name in place of the variable name substitution string.

\sphinxAtStartPar
If the file name does not use variable name templating, this is not used.

\end{fulllineitems}

\index{const\_val (in namelist JULES\_WATER\_RESOURCES\_PROPS)@\spxentry{const\_val}\spxextra{in namelist JULES\_WATER\_RESOURCES\_PROPS}|spxpagem}

\begin{fulllineitems}
\phantomsection\label{\detokenize{namelists/ancillaries.nml:JULES_WATER_RESOURCES_PROPS::const_val}}
\pysigstartsignatures
\pysigline{\sphinxcode{\sphinxupquote{JULES\_WATER\_RESOURCES\_PROPS::}}\sphinxbfcode{\sphinxupquote{const\_val}}}
\pysigstopsignatures\begin{quote}\begin{description}
\sphinxlineitem{Type}
\sphinxAtStartPar
real(nvars)

\sphinxlineitem{Default}
\sphinxAtStartPar
None

\end{description}\end{quote}

\sphinxAtStartPar
For each JULES variable specified in {\hyperref[\detokenize{namelists/ancillaries.nml:JULES_WATER_RESOURCES_PROPS::var}]{\sphinxcrossref{\sphinxcode{\sphinxupquote{var}}}}} where {\hyperref[\detokenize{namelists/ancillaries.nml:JULES_WATER_RESOURCES_PROPS::use_file}]{\sphinxcrossref{\sphinxcode{\sphinxupquote{use\_file}}}}} = FALSE, this is a constant value that the variable will be set to at every point.

\sphinxAtStartPar
This is not used for variables where {\hyperref[\detokenize{namelists/ancillaries.nml:JULES_WATER_RESOURCES_PROPS::use_file}]{\sphinxcrossref{\sphinxcode{\sphinxupquote{use\_file}}}}} = TRUE, but a placeholder must still be given in that case.

\end{fulllineitems}

\end{sphinxadmonition}


\subsubsection{List of water resources properties}
\label{\detokenize{namelists/ancillaries.nml:list-of-water-resources-properties}}\label{\detokenize{namelists/ancillaries.nml:list-of-water-resources-params}}
\sphinxAtStartPar
The following table summarises ancillary fields for the water resources code, specified from an ancillary file if {\hyperref[\detokenize{namelists/ancillaries.nml:JULES_WATER_RESOURCES_PROPS::use_file}]{\sphinxcrossref{\sphinxcode{\sphinxupquote{use\_file}}}}} = TRUE.


\begin{savenotes}\sphinxattablestart
\centering
\begin{tabulary}{\linewidth}[t]{|p{3cm}|L|}
\hline
\sphinxstyletheadfamily 
\sphinxAtStartPar
Name
&\sphinxstyletheadfamily 
\sphinxAtStartPar
Description
\\
\hline
\sphinxAtStartPar
\sphinxcode{\sphinxupquote{conveyance\_loss}}
&
\sphinxAtStartPar
Fraction of water that is lost during conveyance from source to user.
\\
\hline
\sphinxAtStartPar
\sphinxcode{\sphinxupquote{irrig\_eff}}
&
\sphinxAtStartPar
Irrigation efficiency.
This is only used if {\hyperref[\detokenize{namelists/jules_water_resources.nml:JULES_WATER_RESOURCES::l_water_irrigation}]{\sphinxcrossref{\sphinxcode{\sphinxupquote{l\_water\_irrigation}}}}} = TRUE.
\\
\hline
\sphinxAtStartPar
\sphinxcode{\sphinxupquote{sfc\_water\_frac}}
&
\sphinxAtStartPar
Target for the fraction of demand that will be met from surface water (rather than groundwater).
\\
\hline
\end{tabulary}
\par
\sphinxattableend\end{savenotes}


\subsection{\sphinxstyleliteralintitle{\sphinxupquote{URBAN\_PROPERTIES}} namelist members}
\label{\detokenize{namelists/ancillaries.nml:namelist-URBAN_PROPERTIES}}\label{\detokenize{namelists/ancillaries.nml:urban-properties-namelist-members}}\index{URBAN\_PROPERTIES (namelist)@\spxentry{URBAN\_PROPERTIES}\spxextra{namelist}|spxpagem}\index{file (in namelist URBAN\_PROPERTIES)@\spxentry{file}\spxextra{in namelist URBAN\_PROPERTIES}|spxpagem}

\begin{fulllineitems}
\phantomsection\label{\detokenize{namelists/ancillaries.nml:URBAN_PROPERTIES::file}}
\pysigstartsignatures
\pysigline{\sphinxcode{\sphinxupquote{URBAN\_PROPERTIES::}}\sphinxbfcode{\sphinxupquote{file}}}
\pysigstopsignatures\begin{quote}\begin{description}
\sphinxlineitem{Type}
\sphinxAtStartPar
character

\sphinxlineitem{Default}
\sphinxAtStartPar
None

\end{description}\end{quote}

\sphinxAtStartPar
The file to read urban properties from.

\sphinxAtStartPar
If {\hyperref[\detokenize{namelists/ancillaries.nml:URBAN_PROPERTIES::use_file}]{\sphinxcrossref{\sphinxcode{\sphinxupquote{use\_file}}}}} (see below) is FALSE for every variable, this will not be used.

\sphinxAtStartPar
This file name can use {\hyperref[\detokenize{input/file-name-templating::doc}]{\sphinxcrossref{\DUrole{doc}{variable name templating}}}}.

\end{fulllineitems}

\index{nvars (in namelist URBAN\_PROPERTIES)@\spxentry{nvars}\spxextra{in namelist URBAN\_PROPERTIES}|spxpagem}

\begin{fulllineitems}
\phantomsection\label{\detokenize{namelists/ancillaries.nml:URBAN_PROPERTIES::nvars}}
\pysigstartsignatures
\pysigline{\sphinxcode{\sphinxupquote{URBAN\_PROPERTIES::}}\sphinxbfcode{\sphinxupquote{nvars}}}
\pysigstopsignatures\begin{quote}\begin{description}
\sphinxlineitem{Type}
\sphinxAtStartPar
integer

\sphinxlineitem{Permitted}
\sphinxAtStartPar
\textgreater{}= 0

\sphinxlineitem{Default}
\sphinxAtStartPar
0

\end{description}\end{quote}

\sphinxAtStartPar
The number of urban property variables that will be provided.

\sphinxAtStartPar
The required variables depend on whether MORUSES is used or not:
\begin{itemize}
\item {} 
\sphinxAtStartPar
If MORUSES is on, all variables must be given. However, depending on the configuration of MORUSES, not all given variables will be used. Those that will not be used could be set to constant values to avoid setting them from file.

\item {} 
\sphinxAtStartPar
If MORUSES is \sphinxstyleemphasis{not} on, only \sphinxcode{\sphinxupquote{wrr}} is required.

\end{itemize}

\end{fulllineitems}

\index{var (in namelist URBAN\_PROPERTIES)@\spxentry{var}\spxextra{in namelist URBAN\_PROPERTIES}|spxpagem}

\begin{fulllineitems}
\phantomsection\label{\detokenize{namelists/ancillaries.nml:URBAN_PROPERTIES::var}}
\pysigstartsignatures
\pysigline{\sphinxcode{\sphinxupquote{URBAN\_PROPERTIES::}}\sphinxbfcode{\sphinxupquote{var}}}
\pysigstopsignatures\begin{quote}\begin{description}
\sphinxlineitem{Type}
\sphinxAtStartPar
character(nvars)

\sphinxlineitem{Default}
\sphinxAtStartPar
None

\end{description}\end{quote}

\sphinxAtStartPar
List of urban property variable names as recognised by JULES (see {\hyperref[\detokenize{namelists/ancillaries.nml:list-of-urban-properties}]{\sphinxcrossref{\DUrole{std,std-ref}{List of urban properties}}}}). Names are case sensitive.

\begin{sphinxadmonition}{note}{Note:}
\sphinxAtStartPar
For ASCII files, variable names must be in the order they appear in the file.
\end{sphinxadmonition}

\end{fulllineitems}

\index{use\_file (in namelist URBAN\_PROPERTIES)@\spxentry{use\_file}\spxextra{in namelist URBAN\_PROPERTIES}|spxpagem}

\begin{fulllineitems}
\phantomsection\label{\detokenize{namelists/ancillaries.nml:URBAN_PROPERTIES::use_file}}
\pysigstartsignatures
\pysigline{\sphinxcode{\sphinxupquote{URBAN\_PROPERTIES::}}\sphinxbfcode{\sphinxupquote{use\_file}}}
\pysigstopsignatures\begin{quote}\begin{description}
\sphinxlineitem{Type}
\sphinxAtStartPar
logical(nvars)

\sphinxlineitem{Default}
\sphinxAtStartPar
T

\end{description}\end{quote}

\sphinxAtStartPar
For each JULES variable specified in {\hyperref[\detokenize{namelists/ancillaries.nml:URBAN_PROPERTIES::var}]{\sphinxcrossref{\sphinxcode{\sphinxupquote{var}}}}}, this indicates if it should be read from the specified file or whether a constant value is to be used.
\begin{description}
\sphinxlineitem{TRUE}
\sphinxAtStartPar
The variable will be read from the file.

\sphinxlineitem{FALSE}
\sphinxAtStartPar
The variable will be set to a constant value everywhere using {\hyperref[\detokenize{namelists/ancillaries.nml:URBAN_PROPERTIES::const_val}]{\sphinxcrossref{\sphinxcode{\sphinxupquote{const\_val}}}}} below.

\end{description}

\end{fulllineitems}

\index{var\_name (in namelist URBAN\_PROPERTIES)@\spxentry{var\_name}\spxextra{in namelist URBAN\_PROPERTIES}|spxpagem}

\begin{fulllineitems}
\phantomsection\label{\detokenize{namelists/ancillaries.nml:URBAN_PROPERTIES::var_name}}
\pysigstartsignatures
\pysigline{\sphinxcode{\sphinxupquote{URBAN\_PROPERTIES::}}\sphinxbfcode{\sphinxupquote{var\_name}}}
\pysigstopsignatures\begin{quote}\begin{description}
\sphinxlineitem{Type}
\sphinxAtStartPar
character(nvars)

\sphinxlineitem{Default}
\sphinxAtStartPar
‘’ (empty string)

\end{description}\end{quote}

\sphinxAtStartPar
For each JULES variable specified in {\hyperref[\detokenize{namelists/ancillaries.nml:URBAN_PROPERTIES::var}]{\sphinxcrossref{\sphinxcode{\sphinxupquote{var}}}}} where {\hyperref[\detokenize{namelists/ancillaries.nml:URBAN_PROPERTIES::use_file}]{\sphinxcrossref{\sphinxcode{\sphinxupquote{use\_file}}}}} = TRUE, this is the name of the variable in the file containing the data.

\sphinxAtStartPar
If the empty string (the default) is given for any variable, then the corresponding value from {\hyperref[\detokenize{namelists/ancillaries.nml:URBAN_PROPERTIES::var}]{\sphinxcrossref{\sphinxcode{\sphinxupquote{var}}}}} is used instead.

\sphinxAtStartPar
This is not used for variables where {\hyperref[\detokenize{namelists/ancillaries.nml:URBAN_PROPERTIES::use_file}]{\sphinxcrossref{\sphinxcode{\sphinxupquote{use\_file}}}}} = FALSE, but a placeholder must still be given.

\begin{sphinxadmonition}{note}{Note:}
\sphinxAtStartPar
For ASCII files, this is not used \sphinxhyphen{} only the order in the file matters, as described above.
\end{sphinxadmonition}

\end{fulllineitems}

\index{tpl\_name (in namelist URBAN\_PROPERTIES)@\spxentry{tpl\_name}\spxextra{in namelist URBAN\_PROPERTIES}|spxpagem}

\begin{fulllineitems}
\phantomsection\label{\detokenize{namelists/ancillaries.nml:URBAN_PROPERTIES::tpl_name}}
\pysigstartsignatures
\pysigline{\sphinxcode{\sphinxupquote{URBAN\_PROPERTIES::}}\sphinxbfcode{\sphinxupquote{tpl\_name}}}
\pysigstopsignatures\begin{quote}\begin{description}
\sphinxlineitem{Type}
\sphinxAtStartPar
character(nvars)

\sphinxlineitem{Default}
\sphinxAtStartPar
None

\end{description}\end{quote}

\sphinxAtStartPar
For each JULES variable specified in {\hyperref[\detokenize{namelists/ancillaries.nml:URBAN_PROPERTIES::var}]{\sphinxcrossref{\sphinxcode{\sphinxupquote{var}}}}}, this is the string to substitute into the file name in place of the variable name substitution string.

\sphinxAtStartPar
If the file name does not use variable name templating, this is not used.

\end{fulllineitems}

\index{const\_val (in namelist URBAN\_PROPERTIES)@\spxentry{const\_val}\spxextra{in namelist URBAN\_PROPERTIES}|spxpagem}

\begin{fulllineitems}
\phantomsection\label{\detokenize{namelists/ancillaries.nml:URBAN_PROPERTIES::const_val}}
\pysigstartsignatures
\pysigline{\sphinxcode{\sphinxupquote{URBAN\_PROPERTIES::}}\sphinxbfcode{\sphinxupquote{const\_val}}}
\pysigstopsignatures\begin{quote}\begin{description}
\sphinxlineitem{Type}
\sphinxAtStartPar
real(nvars)

\sphinxlineitem{Default}
\sphinxAtStartPar
None

\end{description}\end{quote}

\sphinxAtStartPar
For each JULES variable specified in {\hyperref[\detokenize{namelists/ancillaries.nml:URBAN_PROPERTIES::var}]{\sphinxcrossref{\sphinxcode{\sphinxupquote{var}}}}} where {\hyperref[\detokenize{namelists/ancillaries.nml:URBAN_PROPERTIES::use_file}]{\sphinxcrossref{\sphinxcode{\sphinxupquote{use\_file}}}}} = FALSE, this is a constant value that the variable will be set to at every point in every layer.

\sphinxAtStartPar
This is not used for variables where {\hyperref[\detokenize{namelists/ancillaries.nml:URBAN_PROPERTIES::use_file}]{\sphinxcrossref{\sphinxcode{\sphinxupquote{use\_file}}}}} = TRUE, but a placeholder must still be given.

\end{fulllineitems}



\subsubsection{List of urban properties}
\label{\detokenize{namelists/ancillaries.nml:list-of-urban-properties}}\label{\detokenize{namelists/ancillaries.nml:id2}}
\sphinxAtStartPar
All of the urban property variables listed below are expected to have no levels dimensions and no time dimension.


\begin{savenotes}\sphinxattablestart
\centering
\begin{tabulary}{\linewidth}[t]{|T|T|T|}
\hline
\sphinxstyletheadfamily 
\sphinxAtStartPar
Variable name
&\sphinxstyletheadfamily 
\sphinxAtStartPar
Description \sphinxfootnotemark[1]
&\sphinxstyletheadfamily 
\sphinxAtStartPar
Notes
\\
\hline%
\begin{footnotetext}[1]\sphinxAtStartFootnote
For more information on the urban geometry used please see the JULES technical documentation.
%
\end{footnotetext}\ignorespaces 
\sphinxAtStartPar
\sphinxcode{\sphinxupquote{wrr}}
&
\sphinxAtStartPar
Repeating width ratio (or canyon
fraction, W/R)
&
\sphinxAtStartPar
If {\hyperref[\detokenize{namelists/urban.nml:JULES_URBAN::l_urban_empirical}]{\sphinxcrossref{\sphinxcode{\sphinxupquote{l\_urban\_empirical}}}}} = TRUE
then this is updated with calculated values.
\\
\hline\sphinxstartmulticolumn{3}%
\begin{varwidth}[t]{\sphinxcolwidth{3}{3}}
\sphinxAtStartPar
\sphinxstylestrong{The following apply to MORUSES only}
\par
\vskip-\baselineskip\vbox{\hbox{\strut}}\end{varwidth}%
\sphinxstopmulticolumn
\\
\hline
\sphinxAtStartPar
\sphinxcode{\sphinxupquote{hwr}}
&
\sphinxAtStartPar
Height\sphinxhyphen{}to\sphinxhyphen{}width ratio (H/W)
&
\sphinxAtStartPar
See for \sphinxcode{\sphinxupquote{wrr}} above.
\\
\hline
\sphinxAtStartPar
\sphinxcode{\sphinxupquote{hgt}}
&
\sphinxAtStartPar
Building height (H)
&
\sphinxAtStartPar
See for \sphinxcode{\sphinxupquote{wrr}} above.
\\
\hline
\sphinxAtStartPar
\sphinxcode{\sphinxupquote{ztm}}
&
\sphinxAtStartPar
Effective roughness length of
urban areas
&
\sphinxAtStartPar
If {\hyperref[\detokenize{namelists/urban.nml:JULES_URBAN::l_moruses_macdonald}]{\sphinxcrossref{\sphinxcode{\sphinxupquote{l\_moruses\_macdonald}}}}} = TRUE
then this is updated with calculated values.
\\
\hline
\sphinxAtStartPar
\sphinxcode{\sphinxupquote{disp}}
&
\sphinxAtStartPar
Displacement height
&
\sphinxAtStartPar
See for \sphinxcode{\sphinxupquote{ztm}} above.
\\
\hline
\sphinxAtStartPar
\sphinxcode{\sphinxupquote{albwl}}
&
\sphinxAtStartPar
Wall albedo
&
\sphinxAtStartPar
Data only used if {\hyperref[\detokenize{namelists/urban.nml:JULES_URBAN::l_moruses_albedo}]{\sphinxcrossref{\sphinxcode{\sphinxupquote{l\_moruses\_albedo}}}}} = TRUE.
\\
\hline
\sphinxAtStartPar
\sphinxcode{\sphinxupquote{albrd}}
&
\sphinxAtStartPar
Road albedo
&
\sphinxAtStartPar
See for \sphinxcode{\sphinxupquote{albwl}} above.
\\
\hline
\sphinxAtStartPar
\sphinxcode{\sphinxupquote{emisw}}
&
\sphinxAtStartPar
Wall emissivity
&
\sphinxAtStartPar
Data only used if {\hyperref[\detokenize{namelists/urban.nml:JULES_URBAN::l_moruses_emissivity}]{\sphinxcrossref{\sphinxcode{\sphinxupquote{l\_moruses\_emissivity}}}}} = TRUE.
\\
\hline
\sphinxAtStartPar
\sphinxcode{\sphinxupquote{emisr}}
&
\sphinxAtStartPar
Road emissivity
&
\sphinxAtStartPar
See for \sphinxcode{\sphinxupquote{emisw}} above.
\\
\hline
\end{tabulary}
\par
\sphinxattableend\end{savenotes}


\subsection{\sphinxstyleliteralintitle{\sphinxupquote{JULES\_CO2}} namelist members}
\label{\detokenize{namelists/ancillaries.nml:namelist-JULES_CO2}}\label{\detokenize{namelists/ancillaries.nml:jules-co2-namelist-members}}\index{JULES\_CO2 (namelist)@\spxentry{JULES\_CO2}\spxextra{namelist}|spxpagem}\index{read\_from\_dump (in namelist JULES\_CO2)@\spxentry{read\_from\_dump}\spxextra{in namelist JULES\_CO2}|spxpagem}

\begin{fulllineitems}
\phantomsection\label{\detokenize{namelists/ancillaries.nml:JULES_CO2::read_from_dump}}
\pysigstartsignatures
\pysigline{\sphinxcode{\sphinxupquote{JULES\_CO2::}}\sphinxbfcode{\sphinxupquote{read\_from\_dump}}}
\pysigstopsignatures\begin{quote}\begin{description}
\sphinxlineitem{Type}
\sphinxAtStartPar
logical

\sphinxlineitem{Default}
\sphinxAtStartPar
F

\end{description}\end{quote}
\begin{description}
\sphinxlineitem{TRUE}
\sphinxAtStartPar
Populate variables associated with this namelist from the dump file. All other namelist members are ignored.

\sphinxlineitem{FALSE}
\sphinxAtStartPar
Use the other namelist members to determine how to populate variables.

\end{description}

\end{fulllineitems}

\index{co2\_mmr (in namelist JULES\_CO2)@\spxentry{co2\_mmr}\spxextra{in namelist JULES\_CO2}|spxpagem}

\begin{fulllineitems}
\phantomsection\label{\detokenize{namelists/ancillaries.nml:JULES_CO2::co2_mmr}}
\pysigstartsignatures
\pysigline{\sphinxcode{\sphinxupquote{JULES\_CO2::}}\sphinxbfcode{\sphinxupquote{co2\_mmr}}}
\pysigstopsignatures\begin{quote}\begin{description}
\sphinxlineitem{Type}
\sphinxAtStartPar
real

\sphinxlineitem{Default}
\sphinxAtStartPar
5.241e\sphinxhyphen{}4

\end{description}\end{quote}

\sphinxAtStartPar
Concentration of atmospheric CO2, expressed as a mass mixing ratio.

\end{fulllineitems}



\subsection{\sphinxstyleliteralintitle{\sphinxupquote{JULES\_FLAKE}} namelist members}
\label{\detokenize{namelists/ancillaries.nml:namelist-JULES_FLAKE}}\label{\detokenize{namelists/ancillaries.nml:jules-flake-namelist-members}}\index{JULES\_FLAKE (namelist)@\spxentry{JULES\_FLAKE}\spxextra{namelist}|spxpagem}\index{read\_from\_dump (in namelist JULES\_FLAKE)@\spxentry{read\_from\_dump}\spxextra{in namelist JULES\_FLAKE}|spxpagem}

\begin{fulllineitems}
\phantomsection\label{\detokenize{namelists/ancillaries.nml:JULES_FLAKE::read_from_dump}}
\pysigstartsignatures
\pysigline{\sphinxcode{\sphinxupquote{JULES\_FLAKE::}}\sphinxbfcode{\sphinxupquote{read\_from\_dump}}}
\pysigstopsignatures\begin{quote}\begin{description}
\sphinxlineitem{Type}
\sphinxAtStartPar
logical

\sphinxlineitem{Default}
\sphinxAtStartPar
F

\end{description}\end{quote}
\begin{description}
\sphinxlineitem{TRUE}
\sphinxAtStartPar
Populate variables associated with this namelist from the dump file. All other namelist members are ignored.

\sphinxlineitem{FALSE}
\sphinxAtStartPar
Use the other namelist members to determine how to populate variables.

\end{description}

\end{fulllineitems}

\index{file (in namelist JULES\_FLAKE)@\spxentry{file}\spxextra{in namelist JULES\_FLAKE}|spxpagem}

\begin{fulllineitems}
\phantomsection\label{\detokenize{namelists/ancillaries.nml:JULES_FLAKE::file}}
\pysigstartsignatures
\pysigline{\sphinxcode{\sphinxupquote{JULES\_FLAKE::}}\sphinxbfcode{\sphinxupquote{file}}}
\pysigstopsignatures\begin{quote}\begin{description}
\sphinxlineitem{Type}
\sphinxAtStartPar
character

\sphinxlineitem{Default}
\sphinxAtStartPar
None

\end{description}\end{quote}

\sphinxAtStartPar
The file to read the FLake parameters from.

\end{fulllineitems}

\index{nvars (in namelist JULES\_FLAKE)@\spxentry{nvars}\spxextra{in namelist JULES\_FLAKE}|spxpagem}

\begin{fulllineitems}
\phantomsection\label{\detokenize{namelists/ancillaries.nml:JULES_FLAKE::nvars}}
\pysigstartsignatures
\pysigline{\sphinxcode{\sphinxupquote{JULES\_FLAKE::}}\sphinxbfcode{\sphinxupquote{nvars}}}
\pysigstopsignatures\begin{quote}\begin{description}
\sphinxlineitem{Type}
\sphinxAtStartPar
integer

\sphinxlineitem{Permitted}
\sphinxAtStartPar
\textgreater{}= 0

\sphinxlineitem{Default}
\sphinxAtStartPar
0

\end{description}\end{quote}

\sphinxAtStartPar
The number of FLake variables that will be provided.
(At the moment lake depth is the only variable that needs to be provided).

\end{fulllineitems}

\index{var (in namelist JULES\_FLAKE)@\spxentry{var}\spxextra{in namelist JULES\_FLAKE}|spxpagem}

\begin{fulllineitems}
\phantomsection\label{\detokenize{namelists/ancillaries.nml:JULES_FLAKE::var}}
\pysigstartsignatures
\pysigline{\sphinxcode{\sphinxupquote{JULES\_FLAKE::}}\sphinxbfcode{\sphinxupquote{var}}}
\pysigstopsignatures\begin{quote}\begin{description}
\sphinxlineitem{Type}
\sphinxAtStartPar
character(nvars)

\sphinxlineitem{Default}
\sphinxAtStartPar
None

\end{description}\end{quote}

\sphinxAtStartPar
List of FLake parameter variable names as recognised by JULES (see {\hyperref[\detokenize{namelists/ancillaries.nml:list-of-flake-params}]{\sphinxcrossref{\DUrole{std,std-ref}{List of FLake parameters}}}}). Names are case sensitive.

\begin{sphinxadmonition}{note}{Note:}
\sphinxAtStartPar
For ASCII files, variable names must be in the order they appear in the file.
\end{sphinxadmonition}

\end{fulllineitems}

\index{var\_name (in namelist JULES\_FLAKE)@\spxentry{var\_name}\spxextra{in namelist JULES\_FLAKE}|spxpagem}

\begin{fulllineitems}
\phantomsection\label{\detokenize{namelists/ancillaries.nml:JULES_FLAKE::var_name}}
\pysigstartsignatures
\pysigline{\sphinxcode{\sphinxupquote{JULES\_FLAKE::}}\sphinxbfcode{\sphinxupquote{var\_name}}}
\pysigstopsignatures\begin{quote}\begin{description}
\sphinxlineitem{Type}
\sphinxAtStartPar
character(nvars)

\sphinxlineitem{Default}
\sphinxAtStartPar
‘’ (empty string)

\end{description}\end{quote}

\sphinxAtStartPar
For each JULES variable specified in {\hyperref[\detokenize{namelists/ancillaries.nml:JULES_FLAKE::var}]{\sphinxcrossref{\sphinxcode{\sphinxupquote{var}}}}} where {\hyperref[\detokenize{namelists/ancillaries.nml:JULES_FLAKE::use_file}]{\sphinxcrossref{\sphinxcode{\sphinxupquote{use\_file}}}}} = TRUE, this is the name of the variable in the file containing the data.

\sphinxAtStartPar
If the empty string (the default) is given for any variable, then the corresponding value from {\hyperref[\detokenize{namelists/ancillaries.nml:JULES_FLAKE::var}]{\sphinxcrossref{\sphinxcode{\sphinxupquote{var}}}}} is used instead.

\sphinxAtStartPar
This is not used for variables where {\hyperref[\detokenize{namelists/ancillaries.nml:JULES_FLAKE::use_file}]{\sphinxcrossref{\sphinxcode{\sphinxupquote{use\_file}}}}} = FALSE, but a placeholder must still be given in that case.

\begin{sphinxadmonition}{note}{Note:}
\sphinxAtStartPar
For ASCII files, this is not used \sphinxhyphen{} only the order in the file matters, as described above.
\end{sphinxadmonition}

\end{fulllineitems}

\index{tpl\_name (in namelist JULES\_FLAKE)@\spxentry{tpl\_name}\spxextra{in namelist JULES\_FLAKE}|spxpagem}

\begin{fulllineitems}
\phantomsection\label{\detokenize{namelists/ancillaries.nml:JULES_FLAKE::tpl_name}}
\pysigstartsignatures
\pysigline{\sphinxcode{\sphinxupquote{JULES\_FLAKE::}}\sphinxbfcode{\sphinxupquote{tpl\_name}}}
\pysigstopsignatures\begin{quote}\begin{description}
\sphinxlineitem{Type}
\sphinxAtStartPar
character(nvars)

\sphinxlineitem{Default}
\sphinxAtStartPar
None

\end{description}\end{quote}

\sphinxAtStartPar
For each JULES variable specified in {\hyperref[\detokenize{namelists/ancillaries.nml:JULES_FLAKE::var}]{\sphinxcrossref{\sphinxcode{\sphinxupquote{var}}}}}, this is the string to substitute into the file name in place of the variable name substitution string.

\sphinxAtStartPar
If the file name does not use variable name templating, this is not used.

\end{fulllineitems}

\index{use\_file (in namelist JULES\_FLAKE)@\spxentry{use\_file}\spxextra{in namelist JULES\_FLAKE}|spxpagem}

\begin{fulllineitems}
\phantomsection\label{\detokenize{namelists/ancillaries.nml:JULES_FLAKE::use_file}}
\pysigstartsignatures
\pysigline{\sphinxcode{\sphinxupquote{JULES\_FLAKE::}}\sphinxbfcode{\sphinxupquote{use\_file}}}
\pysigstopsignatures\begin{quote}\begin{description}
\sphinxlineitem{Type}
\sphinxAtStartPar
logical(nvars)

\sphinxlineitem{Default}
\sphinxAtStartPar
T

\end{description}\end{quote}

\sphinxAtStartPar
For each JULES variable specified in {\hyperref[\detokenize{namelists/ancillaries.nml:JULES_FLAKE::var}]{\sphinxcrossref{\sphinxcode{\sphinxupquote{var}}}}}, this indicates if it should be read from the specified file or whether a constant value is to be used.
\begin{description}
\sphinxlineitem{TRUE}
\sphinxAtStartPar
The variable will be read from the file.

\sphinxlineitem{FALSE}
\sphinxAtStartPar
The variable will be set to a constant value everywhere using {\hyperref[\detokenize{namelists/ancillaries.nml:JULES_FLAKE::const_val}]{\sphinxcrossref{\sphinxcode{\sphinxupquote{const\_val}}}}} below.

\end{description}

\end{fulllineitems}

\index{const\_val (in namelist JULES\_FLAKE)@\spxentry{const\_val}\spxextra{in namelist JULES\_FLAKE}|spxpagem}

\begin{fulllineitems}
\phantomsection\label{\detokenize{namelists/ancillaries.nml:JULES_FLAKE::const_val}}
\pysigstartsignatures
\pysigline{\sphinxcode{\sphinxupquote{JULES\_FLAKE::}}\sphinxbfcode{\sphinxupquote{const\_val}}}
\pysigstopsignatures\begin{quote}\begin{description}
\sphinxlineitem{Type}
\sphinxAtStartPar
real(nvars)

\sphinxlineitem{Default}
\sphinxAtStartPar
5.0m

\end{description}\end{quote}

\sphinxAtStartPar
For each JULES variable specified in {\hyperref[\detokenize{namelists/ancillaries.nml:JULES_FLAKE::var}]{\sphinxcrossref{\sphinxcode{\sphinxupquote{var}}}}} where {\hyperref[\detokenize{namelists/ancillaries.nml:JULES_FLAKE::use_file}]{\sphinxcrossref{\sphinxcode{\sphinxupquote{use\_file}}}}} = FALSE, this is a constant value that the variable will be set to for every point or gridbox.

\sphinxAtStartPar
This is not used for variables where {\hyperref[\detokenize{namelists/ancillaries.nml:JULES_FLAKE::use_file}]{\sphinxcrossref{\sphinxcode{\sphinxupquote{use\_file}}}}} = TRUE, but a placeholder must still be given.

\end{fulllineitems}



\subsubsection{List of FLake parameters}
\label{\detokenize{namelists/ancillaries.nml:list-of-flake-parameters}}\label{\detokenize{namelists/ancillaries.nml:list-of-flake-params}}

\begin{savenotes}\sphinxattablestart
\centering
\begin{tabulary}{\linewidth}[t]{|p{2.5cm}|L|}
\hline
\sphinxstyletheadfamily 
\sphinxAtStartPar
Name
&\sphinxstyletheadfamily 
\sphinxAtStartPar
Description
\\
\hline
\sphinxAtStartPar
\sphinxcode{\sphinxupquote{lake\_depth}}
&
\sphinxAtStartPar
For each gridbox, the depth of the lakes should be provided in meters.

\sphinxAtStartPar
Note that for deep lakes FLake will assume an artificial lake bottom at 50m depth.
\\
\hline
\end{tabulary}
\par
\sphinxattableend\end{savenotes}


\subsection{References for ancillaries}
\label{\detokenize{namelists/ancillaries.nml:references-for-ancillaries}}\label{\detokenize{namelists/ancillaries.nml:references-ancillaries}}\begin{itemize}
\item {} 
\sphinxAtStartPar
Kattge, J. and Knorr, W., 2007,
Temperature acclimation in a biochemical model of photosynthesis:
a reanalysis of data from 36 species,
Plant, Cell and Environment, 30: 1176\textendash{}1190,
\sphinxurl{https://doi.org/10.1111/j.1365-3040.2007.01690.x}.

\end{itemize}

\sphinxstepscope


\section{\sphinxstyleliteralintitle{\sphinxupquote{cable\_prognostics.nml}}}
\label{\detokenize{namelists/cable_prognostics.nml:cable-prognostics-nml}}\label{\detokenize{namelists/cable_prognostics.nml::doc}}
\sphinxAtStartPar
This file contains a single namelist called {\hyperref[\detokenize{namelists/cable_prognostics.nml:namelist-CABLE_PROGS}]{\sphinxcrossref{\sphinxcode{\sphinxupquote{CABLE\_PROGS}}}}} that is used to set up the initial state of prognostic variables.


\subsection{\sphinxstyleliteralintitle{\sphinxupquote{CABLE\_PROGS}} namelist members}
\label{\detokenize{namelists/cable_prognostics.nml:namelist-CABLE_PROGS}}\label{\detokenize{namelists/cable_prognostics.nml:cable-progs-namelist-members}}\index{CABLE\_PROGS (namelist)@\spxentry{CABLE\_PROGS}\spxextra{namelist}|spxpagem}
\sphinxAtStartPar
The values of all prognostic variables must be set at the start of a run. This initial state, or initial condition, can be read from a file. Another option is to prescribe a simple or idealised initial state by giving constant values for the prognostic variables directly in the namelist. It is also possible to set some fields using values from a file but to set others to constants given in the namelist.
\index{file (in namelist CABLE\_PROGS)@\spxentry{file}\spxextra{in namelist CABLE\_PROGS}|spxpagem}

\begin{fulllineitems}
\phantomsection\label{\detokenize{namelists/cable_prognostics.nml:CABLE_PROGS::file}}
\pysigstartsignatures
\pysigline{\sphinxcode{\sphinxupquote{CABLE\_PROGS::}}\sphinxbfcode{\sphinxupquote{file}}}
\pysigstopsignatures\begin{quote}\begin{description}
\sphinxlineitem{Type}
\sphinxAtStartPar
character

\sphinxlineitem{Default}
\sphinxAtStartPar
None

\end{description}\end{quote}

\sphinxAtStartPar
The file to read initial values for CABLE prognostic variables from.

\sphinxAtStartPar
If {\hyperref[\detokenize{namelists/cable_prognostics.nml:CABLE_PROGS::use_file}]{\sphinxcrossref{\sphinxcode{\sphinxupquote{use\_file}}}}} is FALSE for every variable, this will not be used.

\sphinxAtStartPar
This file name can use {\hyperref[\detokenize{input/file-name-templating::doc}]{\sphinxcrossref{\DUrole{doc}{variable name templating}}}}.

\end{fulllineitems}

\index{nvars (in namelist CABLE\_PROGS)@\spxentry{nvars}\spxextra{in namelist CABLE\_PROGS}|spxpagem}

\begin{fulllineitems}
\phantomsection\label{\detokenize{namelists/cable_prognostics.nml:CABLE_PROGS::nvars}}
\pysigstartsignatures
\pysigline{\sphinxcode{\sphinxupquote{CABLE\_PROGS::}}\sphinxbfcode{\sphinxupquote{nvars}}}
\pysigstopsignatures\begin{quote}\begin{description}
\sphinxlineitem{Type}
\sphinxAtStartPar
integer

\sphinxlineitem{Permitted}
\sphinxAtStartPar
\textgreater{}= 0

\sphinxlineitem{Default}
\sphinxAtStartPar
0

\end{description}\end{quote}

\sphinxAtStartPar
The number of prognostic variables that will be provided (see {\hyperref[\detokenize{namelists/cable_prognostics.nml:list-of-cable-prognostic-variables}]{\sphinxcrossref{\DUrole{std,std-ref}{List of CABLE prognostic variables}}}}).

\end{fulllineitems}

\index{var (in namelist CABLE\_PROGS)@\spxentry{var}\spxextra{in namelist CABLE\_PROGS}|spxpagem}

\begin{fulllineitems}
\phantomsection\label{\detokenize{namelists/cable_prognostics.nml:CABLE_PROGS::var}}
\pysigstartsignatures
\pysigline{\sphinxcode{\sphinxupquote{CABLE\_PROGS::}}\sphinxbfcode{\sphinxupquote{var}}}
\pysigstopsignatures\begin{quote}\begin{description}
\sphinxlineitem{Type}
\sphinxAtStartPar
character(nvars)

\sphinxlineitem{Default}
\sphinxAtStartPar
None

\end{description}\end{quote}

\sphinxAtStartPar
List of CABLE prognostic variable names as recognised by CABLE (see {\hyperref[\detokenize{namelists/cable_prognostics.nml:list-of-cable-prognostic-variables}]{\sphinxcrossref{\DUrole{std,std-ref}{List of CABLE prognostic variables}}}}). Names are case sensitive.

\begin{sphinxadmonition}{note}{Note:}
\sphinxAtStartPar
For ASCII files, variable names must be in the order they appear in the file.
\end{sphinxadmonition}

\end{fulllineitems}

\index{tpl\_name (in namelist CABLE\_PROGS)@\spxentry{tpl\_name}\spxextra{in namelist CABLE\_PROGS}|spxpagem}

\begin{fulllineitems}
\phantomsection\label{\detokenize{namelists/cable_prognostics.nml:CABLE_PROGS::tpl_name}}
\pysigstartsignatures
\pysigline{\sphinxcode{\sphinxupquote{CABLE\_PROGS::}}\sphinxbfcode{\sphinxupquote{tpl\_name}}}
\pysigstopsignatures\begin{quote}\begin{description}
\sphinxlineitem{Type}
\sphinxAtStartPar
character(nvars)

\sphinxlineitem{Default}
\sphinxAtStartPar
None

\end{description}\end{quote}

\sphinxAtStartPar
For each CABLE variable specified in {\hyperref[\detokenize{namelists/cable_prognostics.nml:CABLE_PROGS::var}]{\sphinxcrossref{\sphinxcode{\sphinxupquote{var}}}}}, this is the string to substitute into the file name in place of the variable name substitution string.

\sphinxAtStartPar
If the file name does not use variable name templating, this is not used.

\end{fulllineitems}

\index{use\_file (in namelist CABLE\_PROGS)@\spxentry{use\_file}\spxextra{in namelist CABLE\_PROGS}|spxpagem}

\begin{fulllineitems}
\phantomsection\label{\detokenize{namelists/cable_prognostics.nml:CABLE_PROGS::use_file}}
\pysigstartsignatures
\pysigline{\sphinxcode{\sphinxupquote{CABLE\_PROGS::}}\sphinxbfcode{\sphinxupquote{use\_file}}}
\pysigstopsignatures\begin{quote}\begin{description}
\sphinxlineitem{Type}
\sphinxAtStartPar
logical(nvars)

\sphinxlineitem{Default}
\sphinxAtStartPar
T

\end{description}\end{quote}

\sphinxAtStartPar
For each CABLE variable specified in {\hyperref[\detokenize{namelists/cable_prognostics.nml:CABLE_PROGS::var}]{\sphinxcrossref{\sphinxcode{\sphinxupquote{var}}}}}, this indicates if it should be read from the specified file or whether a constant value is to be used.
\begin{description}
\sphinxlineitem{TRUE}
\sphinxAtStartPar
The variable will be read from the file.

\sphinxlineitem{FALSE}
\sphinxAtStartPar
The variable will be set to a constant value everywhere using {\hyperref[\detokenize{namelists/cable_prognostics.nml:CABLE_PROGS::const_val}]{\sphinxcrossref{\sphinxcode{\sphinxupquote{const\_val}}}}} below.

\end{description}

\end{fulllineitems}

\index{var\_name (in namelist CABLE\_PROGS)@\spxentry{var\_name}\spxextra{in namelist CABLE\_PROGS}|spxpagem}

\begin{fulllineitems}
\phantomsection\label{\detokenize{namelists/cable_prognostics.nml:CABLE_PROGS::var_name}}
\pysigstartsignatures
\pysigline{\sphinxcode{\sphinxupquote{CABLE\_PROGS::}}\sphinxbfcode{\sphinxupquote{var\_name}}}
\pysigstopsignatures\begin{quote}\begin{description}
\sphinxlineitem{Type}
\sphinxAtStartPar
character(nvars)

\sphinxlineitem{Default}
\sphinxAtStartPar
‘’ (empty string)

\end{description}\end{quote}

\sphinxAtStartPar
For each CABLE variable specified in {\hyperref[\detokenize{namelists/cable_prognostics.nml:CABLE_PROGS::var}]{\sphinxcrossref{\sphinxcode{\sphinxupquote{var}}}}} where {\hyperref[\detokenize{namelists/cable_prognostics.nml:CABLE_PROGS::use_file}]{\sphinxcrossref{\sphinxcode{\sphinxupquote{use\_file}}}}} = TRUE, this is the name of the variable in the file containing the data.

\sphinxAtStartPar
If the empty string (the default) is given for any variable, then the corresponding value from {\hyperref[\detokenize{namelists/cable_prognostics.nml:CABLE_PROGS::var}]{\sphinxcrossref{\sphinxcode{\sphinxupquote{var}}}}} is used instead.

\sphinxAtStartPar
This is not used for variables where {\hyperref[\detokenize{namelists/cable_prognostics.nml:CABLE_PROGS::use_file}]{\sphinxcrossref{\sphinxcode{\sphinxupquote{use\_file}}}}} = FALSE, but a placeholder must still be given in that case.

\begin{sphinxadmonition}{note}{Note:}
\sphinxAtStartPar
For ASCII files, this is not used \sphinxhyphen{} only the order in the file matters, as described above.
\end{sphinxadmonition}

\end{fulllineitems}

\index{const\_val (in namelist CABLE\_PROGS)@\spxentry{const\_val}\spxextra{in namelist CABLE\_PROGS}|spxpagem}

\begin{fulllineitems}
\phantomsection\label{\detokenize{namelists/cable_prognostics.nml:CABLE_PROGS::const_val}}
\pysigstartsignatures
\pysigline{\sphinxcode{\sphinxupquote{CABLE\_PROGS::}}\sphinxbfcode{\sphinxupquote{const\_val}}}
\pysigstopsignatures\begin{quote}\begin{description}
\sphinxlineitem{Type}
\sphinxAtStartPar
real(nvars)

\sphinxlineitem{Default}
\sphinxAtStartPar
None

\end{description}\end{quote}

\sphinxAtStartPar
For each CABLE variable specified in {\hyperref[\detokenize{namelists/cable_prognostics.nml:CABLE_PROGS::var}]{\sphinxcrossref{\sphinxcode{\sphinxupquote{var}}}}} where {\hyperref[\detokenize{namelists/cable_prognostics.nml:CABLE_PROGS::use_file}]{\sphinxcrossref{\sphinxcode{\sphinxupquote{use\_file}}}}} = FALSE, this is a constant value that the variable will be set to at every point in every layer.

\sphinxAtStartPar
This is not used for variables where {\hyperref[\detokenize{namelists/cable_prognostics.nml:CABLE_PROGS::use_file}]{\sphinxcrossref{\sphinxcode{\sphinxupquote{use\_file}}}}} = TRUE, but a placeholder must still be given in that case.

\end{fulllineitems}



\subsubsection{List of CABLE prognostic variables}
\label{\detokenize{namelists/cable_prognostics.nml:list-of-cable-prognostic-variables}}\label{\detokenize{namelists/cable_prognostics.nml:id1}}
\sphinxAtStartPar
Values are set for each tile of each grid point and for each layer of soil or snow.


\begin{savenotes}\sphinxattablestart
\centering
\begin{tabulary}{\linewidth}[t]{|p{5cm}|p{10cm}|}
\hline
\sphinxstyletheadfamily 
\sphinxAtStartPar
Name
&\sphinxstyletheadfamily 
\sphinxAtStartPar
Description
\\
\hline
\sphinxAtStartPar
\sphinxcode{\sphinxupquote{SoilTemp\_CABLE}}
&
\sphinxAtStartPar
Temperature of each soil layer (K).
\\
\hline
\sphinxAtStartPar
\sphinxcode{\sphinxupquote{SoilMoisture\_CABLE}}
&
\sphinxAtStartPar
Soil moisture content of each soil layer (kg m$^{\text{\sphinxhyphen{}2}}$).
\\
\hline
\sphinxAtStartPar
\sphinxcode{\sphinxupquote{FrozenSoilFrac\_CABLE}}
&
\sphinxAtStartPar
Frozen soil moisture content of each soil layer as a fraction of saturation.
\\
\hline
\sphinxAtStartPar
\sphinxcode{\sphinxupquote{SnowDepth\_CABLE}}
&
\sphinxAtStartPar
Depth of each snow level (m).
\\
\hline
\sphinxAtStartPar
\sphinxcode{\sphinxupquote{SnowMass\_CABLE}}
&
\sphinxAtStartPar
Mass of each each snow level (kg).
\\
\hline
\sphinxAtStartPar
\sphinxcode{\sphinxupquote{SnowTemp\_CABLE}}
&
\sphinxAtStartPar
Temperature for each snow layer (K).
\\
\hline
\sphinxAtStartPar
\sphinxcode{\sphinxupquote{SnowDensity\_CABLE}}
&
\sphinxAtStartPar
Density for each snow layer (kg m$^{\text{\sphinxhyphen{}3}}$).
\\
\hline
\sphinxAtStartPar
\sphinxcode{\sphinxupquote{SnowAge\_CABLE}}
&
\sphinxAtStartPar
Age of each snow layer
\\
\hline
\sphinxAtStartPar
\sphinxcode{\sphinxupquote{OneLyrSnowDensity\_CABLE}}
&
\sphinxAtStartPar
Snow density when all snow treated as one layer. (kg m$^{\text{\sphinxhyphen{}3}}$)
\\
\hline
\sphinxAtStartPar
\sphinxcode{\sphinxupquote{ThreeLayerSnowFlag\_CABLE}}
&
\sphinxAtStartPar
Flag for 3 layer snow pack (0 \sphinxhyphen{} false, 1 \sphinxhyphen{} true)
\\
\hline
\end{tabulary}
\par
\sphinxattableend\end{savenotes}

\sphinxstepscope


\section{\sphinxstyleliteralintitle{\sphinxupquote{pft\_params.nml}}}
\label{\detokenize{namelists/pft_params.nml:pft-params-nml}}\label{\detokenize{namelists/pft_params.nml::doc}}
\sphinxAtStartPar
This file sets the time and space\sphinxhyphen{}invariant parameters for plant functional types for the JULES land surface model. It contains one namelist called {\hyperref[\detokenize{namelists/pft_params.nml:namelist-JULES_PFTPARM}]{\sphinxcrossref{\sphinxcode{\sphinxupquote{JULES\_PFTPARM}}}}}.

\begin{sphinxadmonition}{note}{Note:}
\sphinxAtStartPar
If the crop model is on (i.e. {\hyperref[\detokenize{namelists/jules_surface_types.nml:JULES_SURFACE_TYPES::ncpft}]{\sphinxcrossref{\sphinxcode{\sphinxupquote{ncpft}}}}} \textgreater{} 0), the order of PFTs must be natural PFTs followed by crop PFTs.
\end{sphinxadmonition}


\subsection{\sphinxstyleliteralintitle{\sphinxupquote{JULES\_PFTPARM}} namelist members}
\label{\detokenize{namelists/pft_params.nml:namelist-JULES_PFTPARM}}\label{\detokenize{namelists/pft_params.nml:jules-pftparm-namelist-members}}\index{JULES\_PFTPARM (namelist)@\spxentry{JULES\_PFTPARM}\spxextra{namelist}|spxpagem}
\sphinxAtStartPar
This namelist reads the values of parameters for each of the plant functional types (PFTs) if the JULES land surface model is being used. These parameters are a function of PFT only. Parameters that also vary with time and location can be prescribed in {\hyperref[\detokenize{namelists/prescribed_data.nml::doc}]{\sphinxcrossref{\DUrole{doc}{prescribed\_data.nml}}}}. Parameters that are only required if the dynamic vegetation (TRIFFID) or phenology sections are requested are read separately in {\hyperref[\detokenize{namelists/triffid_params.nml::doc}]{\sphinxcrossref{\DUrole{doc}{triffid\_params.nml}}}}. Every member must be given a value for every run.

\sphinxAtStartPar
HCTN24 and 30 refer to Hadley Centre technical notes 24 and 30,
available from \sphinxhref{http://www.metoffice.gov.uk/learning/library/publications/science/climate-science-technical-notes}{the Met Office Library}. For
ease the direct links to these documents are:
\begin{itemize}
\item {} 
\sphinxAtStartPar
\sphinxhref{https://digital.nmla.metoffice.gov.uk/IO\_cc8f146a-d524-4243-88fc-e3a3bcd782e7}{HCTN24 “Description of the “TRIFFID” Dynamic Global Vegetation Model”}

\item {} 
\sphinxAtStartPar
\sphinxhref{https://digital.nmla.metoffice.gov.uk/IO\_7f434aa4-338e-497c-8e66-23488d2e1bd3}{HCTN30 “MOSES 2.2 technical documentation”}

\end{itemize}
\index{canht\_ft\_io (in namelist JULES\_PFTPARM)@\spxentry{canht\_ft\_io}\spxextra{in namelist JULES\_PFTPARM}|spxpagem}

\begin{fulllineitems}
\phantomsection\label{\detokenize{namelists/pft_params.nml:JULES_PFTPARM::canht_ft_io}}
\pysigstartsignatures
\pysigline{\sphinxcode{\sphinxupquote{JULES\_PFTPARM::}}\sphinxbfcode{\sphinxupquote{canht\_ft\_io}}}
\pysigstopsignatures\begin{quote}\begin{description}
\sphinxlineitem{Type}
\sphinxAtStartPar
real(npft)

\sphinxlineitem{Default}
\sphinxAtStartPar
None

\end{description}\end{quote}

\sphinxAtStartPar
The height of each PFT (m), also known as the canopy height.

\sphinxAtStartPar
The value read here is only used if TRIFFID is not active ({\hyperref[\detokenize{namelists/jules_vegetation.nml:JULES_VEGETATION::l_triffid}]{\sphinxcrossref{\sphinxcode{\sphinxupquote{l\_triffid}}}}} = FALSE).

\begin{sphinxadmonition}{note}{Note:}
\sphinxAtStartPar
If TRIFFID is active, canopy height is a prognostic variable and its initial value is read in {\hyperref[\detokenize{namelists/initial_conditions.nml::doc}]{\sphinxcrossref{\DUrole{doc}{initial\_conditions.nml}}}}.
\end{sphinxadmonition}

\end{fulllineitems}

\index{lai\_io (in namelist JULES\_PFTPARM)@\spxentry{lai\_io}\spxextra{in namelist JULES\_PFTPARM}|spxpagem}

\begin{fulllineitems}
\phantomsection\label{\detokenize{namelists/pft_params.nml:JULES_PFTPARM::lai_io}}
\pysigstartsignatures
\pysigline{\sphinxcode{\sphinxupquote{JULES\_PFTPARM::}}\sphinxbfcode{\sphinxupquote{lai\_io}}}
\pysigstopsignatures\begin{quote}\begin{description}
\sphinxlineitem{Type}
\sphinxAtStartPar
real(npft)

\sphinxlineitem{Default}
\sphinxAtStartPar
None

\end{description}\end{quote}

\sphinxAtStartPar
The leaf area index (LAI) of each PFT.

\sphinxAtStartPar
The value read here is only used if neither phenology nor TRIFFID is active ({\hyperref[\detokenize{namelists/jules_vegetation.nml:JULES_VEGETATION::l_phenol}]{\sphinxcrossref{\sphinxcode{\sphinxupquote{l\_phenol}}}}} = FALSE and {\hyperref[\detokenize{namelists/jules_vegetation.nml:JULES_VEGETATION::l_triffid}]{\sphinxcrossref{\sphinxcode{\sphinxupquote{l\_triffid}}}}} = FALSE).

\begin{sphinxadmonition}{note}{Note:}
\sphinxAtStartPar
If phenology is active, LAI is a prognostic variable and its initial value is read in {\hyperref[\detokenize{namelists/initial_conditions.nml::doc}]{\sphinxcrossref{\DUrole{doc}{initial\_conditions.nml}}}}. When TRIFFID is active but phenology is not active (not recommended), LAI is calculated from the canopy height (meaning that the seasonal cycle of LAI will not be correctly represented).
\end{sphinxadmonition}

\end{fulllineitems}

\index{c3\_io (in namelist JULES\_PFTPARM)@\spxentry{c3\_io}\spxextra{in namelist JULES\_PFTPARM}|spxpagem}

\begin{fulllineitems}
\phantomsection\label{\detokenize{namelists/pft_params.nml:JULES_PFTPARM::c3_io}}
\pysigstartsignatures
\pysigline{\sphinxcode{\sphinxupquote{JULES\_PFTPARM::}}\sphinxbfcode{\sphinxupquote{c3\_io}}}
\pysigstopsignatures\begin{quote}\begin{description}
\sphinxlineitem{Type}
\sphinxAtStartPar
integer(npft)

\sphinxlineitem{Default}
\sphinxAtStartPar
None

\end{description}\end{quote}

\sphinxAtStartPar
Flag indicating whether PFT is C3 type.
\begin{enumerate}
\sphinxsetlistlabels{\arabic}{enumi}{enumii}{}{.}%
\setcounter{enumi}{-1}
\item {} 
\sphinxAtStartPar
Not C3 (i.e. C4).

\item {} 
\sphinxAtStartPar
C3.

\end{enumerate}

\end{fulllineitems}

\index{orient\_io (in namelist JULES\_PFTPARM)@\spxentry{orient\_io}\spxextra{in namelist JULES\_PFTPARM}|spxpagem}

\begin{fulllineitems}
\phantomsection\label{\detokenize{namelists/pft_params.nml:JULES_PFTPARM::orient_io}}
\pysigstartsignatures
\pysigline{\sphinxcode{\sphinxupquote{JULES\_PFTPARM::}}\sphinxbfcode{\sphinxupquote{orient\_io}}}
\pysigstopsignatures\begin{quote}\begin{description}
\sphinxlineitem{Type}
\sphinxAtStartPar
integer(npft)

\sphinxlineitem{Default}
\sphinxAtStartPar
None

\end{description}\end{quote}

\sphinxAtStartPar
Flag indicating leaf angle distribution.
\begin{enumerate}
\sphinxsetlistlabels{\arabic}{enumi}{enumii}{}{.}%
\setcounter{enumi}{-1}
\item {} 
\sphinxAtStartPar
Spherical.

\item {} 
\sphinxAtStartPar
Horizontal.

\end{enumerate}

\end{fulllineitems}

\index{can\_struct\_a\_io (in namelist JULES\_PFTPARM)@\spxentry{can\_struct\_a\_io}\spxextra{in namelist JULES\_PFTPARM}|spxpagem}

\begin{fulllineitems}
\phantomsection\label{\detokenize{namelists/pft_params.nml:JULES_PFTPARM::can_struct_a_io}}
\pysigstartsignatures
\pysigline{\sphinxcode{\sphinxupquote{JULES\_PFTPARM::}}\sphinxbfcode{\sphinxupquote{can\_struct\_a\_io}}}
\pysigstopsignatures\begin{quote}\begin{description}
\sphinxlineitem{Type}
\sphinxAtStartPar
real(npft)

\sphinxlineitem{Default}
\sphinxAtStartPar
None

\end{description}\end{quote}

\sphinxAtStartPar
Canopy structure factor (dimensionless). can\_struct\_a\_io=1.0 indicates a structurally homogeneous canopy. Corresponds to the structure factor Zeta in Pinty et al 2006 except assumed not to vary with zenith angle i.e. b=0. The canopy structure factor has no effect if {\hyperref[\detokenize{namelists/jules_vegetation.nml:JULES_VEGETATION::can_rad_mod}]{\sphinxcrossref{\sphinxcode{\sphinxupquote{can\_rad\_mod}}}}} = 1.

\end{fulllineitems}

\index{a\_wl\_io (in namelist JULES\_PFTPARM)@\spxentry{a\_wl\_io}\spxextra{in namelist JULES\_PFTPARM}|spxpagem}

\begin{fulllineitems}
\phantomsection\label{\detokenize{namelists/pft_params.nml:JULES_PFTPARM::a_wl_io}}
\pysigstartsignatures
\pysigline{\sphinxcode{\sphinxupquote{JULES\_PFTPARM::}}\sphinxbfcode{\sphinxupquote{a\_wl\_io}}}
\pysigstopsignatures\begin{quote}\begin{description}
\sphinxlineitem{Type}
\sphinxAtStartPar
real(npft)

\sphinxlineitem{Default}
\sphinxAtStartPar
None

\end{description}\end{quote}

\sphinxAtStartPar
Allometric coefficient relating the target woody biomass to the
leaf area index (kg carbon m$^{\text{\sphinxhyphen{}2}}$)  (Clark et al., 2011; Table 7)

\end{fulllineitems}

\index{a\_ws\_io (in namelist JULES\_PFTPARM)@\spxentry{a\_ws\_io}\spxextra{in namelist JULES\_PFTPARM}|spxpagem}

\begin{fulllineitems}
\phantomsection\label{\detokenize{namelists/pft_params.nml:JULES_PFTPARM::a_ws_io}}
\pysigstartsignatures
\pysigline{\sphinxcode{\sphinxupquote{JULES\_PFTPARM::}}\sphinxbfcode{\sphinxupquote{a\_ws\_io}}}
\pysigstopsignatures\begin{quote}\begin{description}
\sphinxlineitem{Type}
\sphinxAtStartPar
real(npft)

\sphinxlineitem{Default}
\sphinxAtStartPar
None

\end{description}\end{quote}

\sphinxAtStartPar
Woody biomass as a multiple of live stem biomass (Clark et al., 2011; Table 7).

\end{fulllineitems}

\index{albsnc\_max\_io (in namelist JULES\_PFTPARM)@\spxentry{albsnc\_max\_io}\spxextra{in namelist JULES\_PFTPARM}|spxpagem}

\begin{fulllineitems}
\phantomsection\label{\detokenize{namelists/pft_params.nml:JULES_PFTPARM::albsnc_max_io}}
\pysigstartsignatures
\pysigline{\sphinxcode{\sphinxupquote{JULES\_PFTPARM::}}\sphinxbfcode{\sphinxupquote{albsnc\_max\_io}}}
\pysigstopsignatures\begin{quote}\begin{description}
\sphinxlineitem{Type}
\sphinxAtStartPar
real(npft)

\sphinxlineitem{Default}
\sphinxAtStartPar
None

\end{description}\end{quote}

\sphinxAtStartPar
Snow\sphinxhyphen{}covered albedo for large leaf area index.

\sphinxAtStartPar
Only used if {\hyperref[\detokenize{namelists/jules_radiation.nml:JULES_RADIATION::l_snow_albedo}]{\sphinxcrossref{\sphinxcode{\sphinxupquote{l\_snow\_albedo}}}}} = FALSE. See HCTN30 Eq.2.

\end{fulllineitems}

\index{albsnc\_min\_io (in namelist JULES\_PFTPARM)@\spxentry{albsnc\_min\_io}\spxextra{in namelist JULES\_PFTPARM}|spxpagem}

\begin{fulllineitems}
\phantomsection\label{\detokenize{namelists/pft_params.nml:JULES_PFTPARM::albsnc_min_io}}
\pysigstartsignatures
\pysigline{\sphinxcode{\sphinxupquote{JULES\_PFTPARM::}}\sphinxbfcode{\sphinxupquote{albsnc\_min\_io}}}
\pysigstopsignatures\begin{quote}\begin{description}
\sphinxlineitem{Type}
\sphinxAtStartPar
real(npft)

\sphinxlineitem{Default}
\sphinxAtStartPar
None

\end{description}\end{quote}

\sphinxAtStartPar
Snow\sphinxhyphen{}covered albedo for zero leaf area index.

\sphinxAtStartPar
Only used if {\hyperref[\detokenize{namelists/jules_radiation.nml:JULES_RADIATION::l_snow_albedo}]{\sphinxcrossref{\sphinxcode{\sphinxupquote{l\_snow\_albedo}}}}} = FALSE. See HCTN30 Eq.2.

\end{fulllineitems}

\index{albsnf\_max\_io (in namelist JULES\_PFTPARM)@\spxentry{albsnf\_max\_io}\spxextra{in namelist JULES\_PFTPARM}|spxpagem}

\begin{fulllineitems}
\phantomsection\label{\detokenize{namelists/pft_params.nml:JULES_PFTPARM::albsnf_max_io}}
\pysigstartsignatures
\pysigline{\sphinxcode{\sphinxupquote{JULES\_PFTPARM::}}\sphinxbfcode{\sphinxupquote{albsnf\_max\_io}}}
\pysigstopsignatures\begin{quote}\begin{description}
\sphinxlineitem{Type}
\sphinxAtStartPar
real(npft)

\sphinxlineitem{Default}
\sphinxAtStartPar
None

\end{description}\end{quote}

\sphinxAtStartPar
Snow\sphinxhyphen{}free albedo for large LAI.

\sphinxAtStartPar
Only used if {\hyperref[\detokenize{namelists/jules_radiation.nml:JULES_RADIATION::l_spec_albedo}]{\sphinxcrossref{\sphinxcode{\sphinxupquote{l\_spec\_albedo}}}}} = FALSE. See HCTN30 Eq.1.

\end{fulllineitems}

\index{albsnf\_maxu\_io (in namelist JULES\_PFTPARM)@\spxentry{albsnf\_maxu\_io}\spxextra{in namelist JULES\_PFTPARM}|spxpagem}

\begin{fulllineitems}
\phantomsection\label{\detokenize{namelists/pft_params.nml:JULES_PFTPARM::albsnf_maxu_io}}
\pysigstartsignatures
\pysigline{\sphinxcode{\sphinxupquote{JULES\_PFTPARM::}}\sphinxbfcode{\sphinxupquote{albsnf\_maxu\_io}}}
\pysigstopsignatures\begin{quote}\begin{description}
\sphinxlineitem{Type}
\sphinxAtStartPar
real(npft)

\sphinxlineitem{Default}
\sphinxAtStartPar
None

\end{description}\end{quote}

\sphinxAtStartPar
Upper bound for the snow\sphinxhyphen{}free albedo for large LAI, when scaled to match input obs.

\sphinxAtStartPar
Only used if {\hyperref[\detokenize{namelists/jules_radiation.nml:JULES_RADIATION::l_spec_albedo}]{\sphinxcrossref{\sphinxcode{\sphinxupquote{l\_spec\_albedo}}}}} = FALSE and {\hyperref[\detokenize{namelists/jules_radiation.nml:JULES_RADIATION::l_albedo_obs}]{\sphinxcrossref{\sphinxcode{\sphinxupquote{l\_albedo\_obs}}}}} = TRUE.

\end{fulllineitems}

\index{albsnf\_maxl\_io (in namelist JULES\_PFTPARM)@\spxentry{albsnf\_maxl\_io}\spxextra{in namelist JULES\_PFTPARM}|spxpagem}

\begin{fulllineitems}
\phantomsection\label{\detokenize{namelists/pft_params.nml:JULES_PFTPARM::albsnf_maxl_io}}
\pysigstartsignatures
\pysigline{\sphinxcode{\sphinxupquote{JULES\_PFTPARM::}}\sphinxbfcode{\sphinxupquote{albsnf\_maxl\_io}}}
\pysigstopsignatures\begin{quote}\begin{description}
\sphinxlineitem{Type}
\sphinxAtStartPar
real(npft)

\sphinxlineitem{Default}
\sphinxAtStartPar
None

\end{description}\end{quote}

\sphinxAtStartPar
Lower bound for the snow\sphinxhyphen{}free albedo for large LAI, when scaled to match input obs.

\sphinxAtStartPar
Only used if {\hyperref[\detokenize{namelists/jules_radiation.nml:JULES_RADIATION::l_spec_albedo}]{\sphinxcrossref{\sphinxcode{\sphinxupquote{l\_spec\_albedo}}}}} = FALSE and {\hyperref[\detokenize{namelists/jules_radiation.nml:JULES_RADIATION::l_albedo_obs}]{\sphinxcrossref{\sphinxcode{\sphinxupquote{l\_albedo\_obs}}}}} = TRUE.

\end{fulllineitems}

\index{alpha\_io (in namelist JULES\_PFTPARM)@\spxentry{alpha\_io}\spxextra{in namelist JULES\_PFTPARM}|spxpagem}

\begin{fulllineitems}
\phantomsection\label{\detokenize{namelists/pft_params.nml:JULES_PFTPARM::alpha_io}}
\pysigstartsignatures
\pysigline{\sphinxcode{\sphinxupquote{JULES\_PFTPARM::}}\sphinxbfcode{\sphinxupquote{alpha\_io}}}
\pysigstopsignatures\begin{quote}\begin{description}
\sphinxlineitem{Type}
\sphinxAtStartPar
real(npft)

\sphinxlineitem{Default}
\sphinxAtStartPar
None

\end{description}\end{quote}

\sphinxAtStartPar
Quantum efficiency of photosynthesis (mol CO$_{\text{2}}$ per mol PAR photons).

\end{fulllineitems}

\index{alnir\_io (in namelist JULES\_PFTPARM)@\spxentry{alnir\_io}\spxextra{in namelist JULES\_PFTPARM}|spxpagem}

\begin{fulllineitems}
\phantomsection\label{\detokenize{namelists/pft_params.nml:JULES_PFTPARM::alnir_io}}
\pysigstartsignatures
\pysigline{\sphinxcode{\sphinxupquote{JULES\_PFTPARM::}}\sphinxbfcode{\sphinxupquote{alnir\_io}}}
\pysigstopsignatures\begin{quote}\begin{description}
\sphinxlineitem{Type}
\sphinxAtStartPar
real(npft)

\sphinxlineitem{Default}
\sphinxAtStartPar
None

\end{description}\end{quote}

\sphinxAtStartPar
Leaf reflection coefficient for NIR. See HCTN30 Table 3.

\sphinxAtStartPar
Always used unless {\hyperref[\detokenize{namelists/jules_vegetation.nml:JULES_VEGETATION::can_rad_mod}]{\sphinxcrossref{\sphinxcode{\sphinxupquote{can\_rad\_mod}}}}} = 1 and
{\hyperref[\detokenize{namelists/jules_radiation.nml:JULES_RADIATION::l_spec_albedo}]{\sphinxcrossref{\sphinxcode{\sphinxupquote{l\_spec\_albedo}}}}} = FALSE.

\end{fulllineitems}

\index{alniru\_io (in namelist JULES\_PFTPARM)@\spxentry{alniru\_io}\spxextra{in namelist JULES\_PFTPARM}|spxpagem}

\begin{fulllineitems}
\phantomsection\label{\detokenize{namelists/pft_params.nml:JULES_PFTPARM::alniru_io}}
\pysigstartsignatures
\pysigline{\sphinxcode{\sphinxupquote{JULES\_PFTPARM::}}\sphinxbfcode{\sphinxupquote{alniru\_io}}}
\pysigstopsignatures\begin{quote}\begin{description}
\sphinxlineitem{Type}
\sphinxAtStartPar
real(npft)

\sphinxlineitem{Default}
\sphinxAtStartPar
None

\end{description}\end{quote}

\sphinxAtStartPar
Upper limit for the leaf reflection coefficient for NIR, when
{\hyperref[\detokenize{namelists/jules_radiation.nml:JULES_RADIATION::l_albedo_obs}]{\sphinxcrossref{\sphinxcode{\sphinxupquote{l\_albedo\_obs}}}}} = TRUE and when
{\hyperref[\detokenize{namelists/pft_params.nml:JULES_PFTPARM::alnir_io}]{\sphinxcrossref{\sphinxcode{\sphinxupquote{alnir\_io}}}}} is used.

\end{fulllineitems}

\index{alnirl\_io (in namelist JULES\_PFTPARM)@\spxentry{alnirl\_io}\spxextra{in namelist JULES\_PFTPARM}|spxpagem}

\begin{fulllineitems}
\phantomsection\label{\detokenize{namelists/pft_params.nml:JULES_PFTPARM::alnirl_io}}
\pysigstartsignatures
\pysigline{\sphinxcode{\sphinxupquote{JULES\_PFTPARM::}}\sphinxbfcode{\sphinxupquote{alnirl\_io}}}
\pysigstopsignatures\begin{quote}\begin{description}
\sphinxlineitem{Type}
\sphinxAtStartPar
real(npft)

\sphinxlineitem{Default}
\sphinxAtStartPar
None

\end{description}\end{quote}

\sphinxAtStartPar
Lower limit for the leaf reflection coefficient for NIR, when
{\hyperref[\detokenize{namelists/jules_radiation.nml:JULES_RADIATION::l_albedo_obs}]{\sphinxcrossref{\sphinxcode{\sphinxupquote{l\_albedo\_obs}}}}} = TRUE and when
{\hyperref[\detokenize{namelists/pft_params.nml:JULES_PFTPARM::alnir_io}]{\sphinxcrossref{\sphinxcode{\sphinxupquote{alnir\_io}}}}} is used.

\end{fulllineitems}

\index{alpar\_io (in namelist JULES\_PFTPARM)@\spxentry{alpar\_io}\spxextra{in namelist JULES\_PFTPARM}|spxpagem}

\begin{fulllineitems}
\phantomsection\label{\detokenize{namelists/pft_params.nml:JULES_PFTPARM::alpar_io}}
\pysigstartsignatures
\pysigline{\sphinxcode{\sphinxupquote{JULES\_PFTPARM::}}\sphinxbfcode{\sphinxupquote{alpar\_io}}}
\pysigstopsignatures\begin{quote}\begin{description}
\sphinxlineitem{Type}
\sphinxAtStartPar
real(npft)

\sphinxlineitem{Default}
\sphinxAtStartPar
None

\end{description}\end{quote}

\sphinxAtStartPar
Leaf reflection coefficient for VIS (photosyntehtically active radiation). See HCTN30 Table 3.

\sphinxAtStartPar
Always used unless {\hyperref[\detokenize{namelists/jules_vegetation.nml:JULES_VEGETATION::can_rad_mod}]{\sphinxcrossref{\sphinxcode{\sphinxupquote{can\_rad\_mod}}}}} = 1 and
{\hyperref[\detokenize{namelists/jules_radiation.nml:JULES_RADIATION::l_spec_albedo}]{\sphinxcrossref{\sphinxcode{\sphinxupquote{l\_spec\_albedo}}}}} = FALSE.

\end{fulllineitems}

\index{alparu\_io (in namelist JULES\_PFTPARM)@\spxentry{alparu\_io}\spxextra{in namelist JULES\_PFTPARM}|spxpagem}

\begin{fulllineitems}
\phantomsection\label{\detokenize{namelists/pft_params.nml:JULES_PFTPARM::alparu_io}}
\pysigstartsignatures
\pysigline{\sphinxcode{\sphinxupquote{JULES\_PFTPARM::}}\sphinxbfcode{\sphinxupquote{alparu\_io}}}
\pysigstopsignatures\begin{quote}\begin{description}
\sphinxlineitem{Type}
\sphinxAtStartPar
real(npft)

\sphinxlineitem{Default}
\sphinxAtStartPar
None

\end{description}\end{quote}

\sphinxAtStartPar
Upper limit for the leaf reflection coefficient for VIS, when
{\hyperref[\detokenize{namelists/jules_radiation.nml:JULES_RADIATION::l_albedo_obs}]{\sphinxcrossref{\sphinxcode{\sphinxupquote{l\_albedo\_obs}}}}} = TRUE and when
{\hyperref[\detokenize{namelists/pft_params.nml:JULES_PFTPARM::alpar_io}]{\sphinxcrossref{\sphinxcode{\sphinxupquote{alpar\_io}}}}} is used.

\end{fulllineitems}

\index{alparl\_io (in namelist JULES\_PFTPARM)@\spxentry{alparl\_io}\spxextra{in namelist JULES\_PFTPARM}|spxpagem}

\begin{fulllineitems}
\phantomsection\label{\detokenize{namelists/pft_params.nml:JULES_PFTPARM::alparl_io}}
\pysigstartsignatures
\pysigline{\sphinxcode{\sphinxupquote{JULES\_PFTPARM::}}\sphinxbfcode{\sphinxupquote{alparl\_io}}}
\pysigstopsignatures\begin{quote}\begin{description}
\sphinxlineitem{Type}
\sphinxAtStartPar
real(npft)

\sphinxlineitem{Default}
\sphinxAtStartPar
None

\end{description}\end{quote}

\sphinxAtStartPar
Lower limit for the leaf reflection coefficient for VIS, when
{\hyperref[\detokenize{namelists/jules_radiation.nml:JULES_RADIATION::l_albedo_obs}]{\sphinxcrossref{\sphinxcode{\sphinxupquote{l\_albedo\_obs}}}}} = TRUE and when
{\hyperref[\detokenize{namelists/pft_params.nml:JULES_PFTPARM::alpar_io}]{\sphinxcrossref{\sphinxcode{\sphinxupquote{alpar\_io}}}}} is used.

\end{fulllineitems}

\index{b\_wl\_io (in namelist JULES\_PFTPARM)@\spxentry{b\_wl\_io}\spxextra{in namelist JULES\_PFTPARM}|spxpagem}

\begin{fulllineitems}
\phantomsection\label{\detokenize{namelists/pft_params.nml:JULES_PFTPARM::b_wl_io}}
\pysigstartsignatures
\pysigline{\sphinxcode{\sphinxupquote{JULES\_PFTPARM::}}\sphinxbfcode{\sphinxupquote{b\_wl\_io}}}
\pysigstopsignatures\begin{quote}\begin{description}
\sphinxlineitem{Type}
\sphinxAtStartPar
real(npft)

\sphinxlineitem{Default}
\sphinxAtStartPar
None

\end{description}\end{quote}

\sphinxAtStartPar
Allometric exponent relating the target woody biomass to the leaf
area index. This is 5/3 in HCTN24 Eq.8. See also Clark et
al. (2011, Table 7).

\end{fulllineitems}

\index{catch0\_io (in namelist JULES\_PFTPARM)@\spxentry{catch0\_io}\spxextra{in namelist JULES\_PFTPARM}|spxpagem}

\begin{fulllineitems}
\phantomsection\label{\detokenize{namelists/pft_params.nml:JULES_PFTPARM::catch0_io}}
\pysigstartsignatures
\pysigline{\sphinxcode{\sphinxupquote{JULES\_PFTPARM::}}\sphinxbfcode{\sphinxupquote{catch0\_io}}}
\pysigstopsignatures\begin{quote}\begin{description}
\sphinxlineitem{Type}
\sphinxAtStartPar
real(npft)

\sphinxlineitem{Default}
\sphinxAtStartPar
None

\end{description}\end{quote}

\sphinxAtStartPar
Minimum canopy capacity (kg m$^{\text{\sphinxhyphen{}2}}$).

\sphinxAtStartPar
This is the minimum amount of water that can be held on the canopy. See HCTN30 p7.

\end{fulllineitems}

\index{dcatch\_dlai\_io (in namelist JULES\_PFTPARM)@\spxentry{dcatch\_dlai\_io}\spxextra{in namelist JULES\_PFTPARM}|spxpagem}

\begin{fulllineitems}
\phantomsection\label{\detokenize{namelists/pft_params.nml:JULES_PFTPARM::dcatch_dlai_io}}
\pysigstartsignatures
\pysigline{\sphinxcode{\sphinxupquote{JULES\_PFTPARM::}}\sphinxbfcode{\sphinxupquote{dcatch\_dlai\_io}}}
\pysigstopsignatures\begin{quote}\begin{description}
\sphinxlineitem{Type}
\sphinxAtStartPar
real(npft)

\sphinxlineitem{Default}
\sphinxAtStartPar
None

\end{description}\end{quote}

\sphinxAtStartPar
Rate of change of canopy capacity with LAI (kg m$^{\text{\sphinxhyphen{}2}}$).

\sphinxAtStartPar
Canopy capacity is calculated as \sphinxcode{\sphinxupquote{catch0 + dcatch\_dlai*lai}}. See HCTN30 p7.

\end{fulllineitems}

\index{dgl\_dm\_io (in namelist JULES\_PFTPARM)@\spxentry{dgl\_dm\_io}\spxextra{in namelist JULES\_PFTPARM}|spxpagem}

\begin{fulllineitems}
\phantomsection\label{\detokenize{namelists/pft_params.nml:JULES_PFTPARM::dgl_dm_io}}
\pysigstartsignatures
\pysigline{\sphinxcode{\sphinxupquote{JULES\_PFTPARM::}}\sphinxbfcode{\sphinxupquote{dgl\_dm\_io}}}
\pysigstopsignatures\begin{quote}\begin{description}
\sphinxlineitem{Type}
\sphinxAtStartPar
real(npft)

\sphinxlineitem{Default}
\sphinxAtStartPar
None

\end{description}\end{quote}

\sphinxAtStartPar
Rate of change of leaf turnover rate with moisture availability.

\end{fulllineitems}

\index{dgl\_dt\_io (in namelist JULES\_PFTPARM)@\spxentry{dgl\_dt\_io}\spxextra{in namelist JULES\_PFTPARM}|spxpagem}

\begin{fulllineitems}
\phantomsection\label{\detokenize{namelists/pft_params.nml:JULES_PFTPARM::dgl_dt_io}}
\pysigstartsignatures
\pysigline{\sphinxcode{\sphinxupquote{JULES\_PFTPARM::}}\sphinxbfcode{\sphinxupquote{dgl\_dt\_io}}}
\pysigstopsignatures\begin{quote}\begin{description}
\sphinxlineitem{Type}
\sphinxAtStartPar
real(npft)

\sphinxlineitem{Default}
\sphinxAtStartPar
None

\end{description}\end{quote}

\sphinxAtStartPar
Rate of change of leaf turnover rate with temperature (K$^{\text{\sphinxhyphen{}1}}$).

\sphinxAtStartPar
This is 9 in HCTN24 Eq.10.

\end{fulllineitems}

\index{dqcrit\_io (in namelist JULES\_PFTPARM)@\spxentry{dqcrit\_io}\spxextra{in namelist JULES\_PFTPARM}|spxpagem}

\begin{fulllineitems}
\phantomsection\label{\detokenize{namelists/pft_params.nml:JULES_PFTPARM::dqcrit_io}}
\pysigstartsignatures
\pysigline{\sphinxcode{\sphinxupquote{JULES\_PFTPARM::}}\sphinxbfcode{\sphinxupquote{dqcrit\_io}}}
\pysigstopsignatures\begin{quote}\begin{description}
\sphinxlineitem{Type}
\sphinxAtStartPar
real(npft)

\sphinxlineitem{Default}
\sphinxAtStartPar
None

\end{description}\end{quote}

\sphinxAtStartPar
Critical humidity deficit (kg H$_{\text{2}}$O per kg air).

\sphinxAtStartPar
Only used with the Jacobs model of stomatal conductance ({\hyperref[\detokenize{namelists/jules_vegetation.nml:JULES_VEGETATION::stomata_model}]{\sphinxcrossref{\sphinxcode{\sphinxupquote{stomata\_model}}}}} = 1).

\end{fulllineitems}

\index{dz0v\_dh\_io (in namelist JULES\_PFTPARM)@\spxentry{dz0v\_dh\_io}\spxextra{in namelist JULES\_PFTPARM}|spxpagem}

\begin{fulllineitems}
\phantomsection\label{\detokenize{namelists/pft_params.nml:JULES_PFTPARM::dz0v_dh_io}}
\pysigstartsignatures
\pysigline{\sphinxcode{\sphinxupquote{JULES\_PFTPARM::}}\sphinxbfcode{\sphinxupquote{dz0v\_dh\_io}}}
\pysigstopsignatures\begin{quote}\begin{description}
\sphinxlineitem{Type}
\sphinxAtStartPar
real(npft)

\sphinxlineitem{Default}
\sphinxAtStartPar
None

\end{description}\end{quote}

\sphinxAtStartPar
Rate of change of vegetation roughness length for momentum with height.

\sphinxAtStartPar
Roughness length is calculated as \sphinxcode{\sphinxupquote{dz0v\_dh * canht\_ft}}. See HCTN30 p5.

\sphinxAtStartPar
Used if logical {\hyperref[\detokenize{namelists/jules_vegetation.nml:JULES_VEGETATION::l_spec_veg_z0}]{\sphinxcrossref{\sphinxcode{\sphinxupquote{l\_spec\_veg\_z0}}}}} is set to .false.

\end{fulllineitems}

\index{z0v\_io (in namelist JULES\_PFTPARM)@\spxentry{z0v\_io}\spxextra{in namelist JULES\_PFTPARM}|spxpagem}

\begin{fulllineitems}
\phantomsection\label{\detokenize{namelists/pft_params.nml:JULES_PFTPARM::z0v_io}}
\pysigstartsignatures
\pysigline{\sphinxcode{\sphinxupquote{JULES\_PFTPARM::}}\sphinxbfcode{\sphinxupquote{z0v\_io}}}
\pysigstopsignatures\begin{quote}\begin{description}
\sphinxlineitem{Type}
\sphinxAtStartPar
real(npft)

\sphinxlineitem{Default}
\sphinxAtStartPar
None

\end{description}\end{quote}

\sphinxAtStartPar
Specified values for the vegetation roughness length for momentum.

\sphinxAtStartPar
Used if logical {\hyperref[\detokenize{namelists/jules_vegetation.nml:JULES_VEGETATION::l_spec_veg_z0}]{\sphinxcrossref{\sphinxcode{\sphinxupquote{l\_spec\_veg\_z0}}}}} is set to .true.

\end{fulllineitems}

\index{eta\_sl\_io (in namelist JULES\_PFTPARM)@\spxentry{eta\_sl\_io}\spxextra{in namelist JULES\_PFTPARM}|spxpagem}

\begin{fulllineitems}
\phantomsection\label{\detokenize{namelists/pft_params.nml:JULES_PFTPARM::eta_sl_io}}
\pysigstartsignatures
\pysigline{\sphinxcode{\sphinxupquote{JULES\_PFTPARM::}}\sphinxbfcode{\sphinxupquote{eta\_sl\_io}}}
\pysigstopsignatures\begin{quote}\begin{description}
\sphinxlineitem{Type}
\sphinxAtStartPar
real(npft)

\sphinxlineitem{Default}
\sphinxAtStartPar
None

\end{description}\end{quote}

\sphinxAtStartPar
Live stemwood coefficient (kg C/m/(m2 leaf)) (Clark et al., 2011; Table 7).

\end{fulllineitems}

\index{fd\_io (in namelist JULES\_PFTPARM)@\spxentry{fd\_io}\spxextra{in namelist JULES\_PFTPARM}|spxpagem}

\begin{fulllineitems}
\phantomsection\label{\detokenize{namelists/pft_params.nml:JULES_PFTPARM::fd_io}}
\pysigstartsignatures
\pysigline{\sphinxcode{\sphinxupquote{JULES\_PFTPARM::}}\sphinxbfcode{\sphinxupquote{fd\_io}}}
\pysigstopsignatures\begin{quote}\begin{description}
\sphinxlineitem{Type}
\sphinxAtStartPar
real(npft)

\sphinxlineitem{Default}
\sphinxAtStartPar
None

\end{description}\end{quote}

\sphinxAtStartPar
Scale factor for dark respiration. See HCTN 24 Eq. 56.

\end{fulllineitems}

\index{fsmc\_of\_io (in namelist JULES\_PFTPARM)@\spxentry{fsmc\_of\_io}\spxextra{in namelist JULES\_PFTPARM}|spxpagem}

\begin{fulllineitems}
\phantomsection\label{\detokenize{namelists/pft_params.nml:JULES_PFTPARM::fsmc_of_io}}
\pysigstartsignatures
\pysigline{\sphinxcode{\sphinxupquote{JULES\_PFTPARM::}}\sphinxbfcode{\sphinxupquote{fsmc\_of\_io}}}
\pysigstopsignatures\begin{quote}\begin{description}
\sphinxlineitem{Type}
\sphinxAtStartPar
real(npft)

\sphinxlineitem{Default}
\sphinxAtStartPar
None

\end{description}\end{quote}

\sphinxAtStartPar
Moisture availability below which leaves are dropped.

\end{fulllineitems}

\index{f0\_io (in namelist JULES\_PFTPARM)@\spxentry{f0\_io}\spxextra{in namelist JULES\_PFTPARM}|spxpagem}

\begin{fulllineitems}
\phantomsection\label{\detokenize{namelists/pft_params.nml:JULES_PFTPARM::f0_io}}
\pysigstartsignatures
\pysigline{\sphinxcode{\sphinxupquote{JULES\_PFTPARM::}}\sphinxbfcode{\sphinxupquote{f0\_io}}}
\pysigstopsignatures\begin{quote}\begin{description}
\sphinxlineitem{Type}
\sphinxAtStartPar
real(npft)

\sphinxlineitem{Default}
\sphinxAtStartPar
None

\end{description}\end{quote}

\sphinxAtStartPar
\sphinxcode{\sphinxupquote{CI / CA}} for \sphinxcode{\sphinxupquote{DQ = 0}}. See HCTN 24 Eq. 32.

\sphinxAtStartPar
Only used with the Jacobs model of stomatal conductance ({\hyperref[\detokenize{namelists/jules_vegetation.nml:JULES_VEGETATION::stomata_model}]{\sphinxcrossref{\sphinxcode{\sphinxupquote{stomata\_model}}}}} = 1).

\end{fulllineitems}

\index{g1\_stomata\_io (in namelist JULES\_PFTPARM)@\spxentry{g1\_stomata\_io}\spxextra{in namelist JULES\_PFTPARM}|spxpagem}

\begin{fulllineitems}
\phantomsection\label{\detokenize{namelists/pft_params.nml:JULES_PFTPARM::g1_stomata_io}}
\pysigstartsignatures
\pysigline{\sphinxcode{\sphinxupquote{JULES\_PFTPARM::}}\sphinxbfcode{\sphinxupquote{g1\_stomata\_io}}}
\pysigstopsignatures\begin{quote}\begin{description}
\sphinxlineitem{Type}
\sphinxAtStartPar
real(npft)

\sphinxlineitem{Default}
\sphinxAtStartPar
None

\end{description}\end{quote}

\sphinxAtStartPar
Parameter g1 for the Medlyn et al. (2011) model of stomatal conductance (kPa$^{\text{0.5}}$) \sphinxhyphen{} this is the sensitivity of the stomatal conductance to the assimilation rate. See Eqn.11 in Medlyn et al. (2012), \sphinxurl{https://doi.org/10.1111/j.1365-2486.2012.02790.x}.

\sphinxAtStartPar
Only used with the Medlyn model of stomatal conductance ({\hyperref[\detokenize{namelists/jules_vegetation.nml:JULES_VEGETATION::stomata_model}]{\sphinxcrossref{\sphinxcode{\sphinxupquote{stomata\_model}}}}} = 2).

\end{fulllineitems}

\index{g\_leaf\_0\_io (in namelist JULES\_PFTPARM)@\spxentry{g\_leaf\_0\_io}\spxextra{in namelist JULES\_PFTPARM}|spxpagem}

\begin{fulllineitems}
\phantomsection\label{\detokenize{namelists/pft_params.nml:JULES_PFTPARM::g_leaf_0_io}}
\pysigstartsignatures
\pysigline{\sphinxcode{\sphinxupquote{JULES\_PFTPARM::}}\sphinxbfcode{\sphinxupquote{g\_leaf\_0\_io}}}
\pysigstopsignatures\begin{quote}\begin{description}
\sphinxlineitem{Type}
\sphinxAtStartPar
real(npft)

\sphinxlineitem{Default}
\sphinxAtStartPar
None

\end{description}\end{quote}

\sphinxAtStartPar
Minimum turnover rate for leaves (/360days).

\end{fulllineitems}

\index{glmin\_io (in namelist JULES\_PFTPARM)@\spxentry{glmin\_io}\spxextra{in namelist JULES\_PFTPARM}|spxpagem}

\begin{fulllineitems}
\phantomsection\label{\detokenize{namelists/pft_params.nml:JULES_PFTPARM::glmin_io}}
\pysigstartsignatures
\pysigline{\sphinxcode{\sphinxupquote{JULES\_PFTPARM::}}\sphinxbfcode{\sphinxupquote{glmin\_io}}}
\pysigstopsignatures\begin{quote}\begin{description}
\sphinxlineitem{Type}
\sphinxAtStartPar
real(npft)

\sphinxlineitem{Default}
\sphinxAtStartPar
None

\end{description}\end{quote}

\sphinxAtStartPar
Minimum leaf conductance for H$_{\text{2}}$O (m s$^{\text{\sphinxhyphen{}1}}$).

\end{fulllineitems}

\index{infil\_f\_io (in namelist JULES\_PFTPARM)@\spxentry{infil\_f\_io}\spxextra{in namelist JULES\_PFTPARM}|spxpagem}

\begin{fulllineitems}
\phantomsection\label{\detokenize{namelists/pft_params.nml:JULES_PFTPARM::infil_f_io}}
\pysigstartsignatures
\pysigline{\sphinxcode{\sphinxupquote{JULES\_PFTPARM::}}\sphinxbfcode{\sphinxupquote{infil\_f\_io}}}
\pysigstopsignatures\begin{quote}\begin{description}
\sphinxlineitem{Type}
\sphinxAtStartPar
real(npft)

\sphinxlineitem{Default}
\sphinxAtStartPar
None

\end{description}\end{quote}

\sphinxAtStartPar
Infiltration enhancement factor.

\sphinxAtStartPar
The maximum infiltration rate defined by the soil parameters for the whole gridbox may be modified for each PFT to account for PFT\sphinxhyphen{}dependent factors, such as macro\sphinxhyphen{}pores related to vegetation roots.

\sphinxAtStartPar
See HCTN30 p14 for full details.

\end{fulllineitems}

\index{gsoil\_f\_io (in namelist JULES\_PFTPARM)@\spxentry{gsoil\_f\_io}\spxextra{in namelist JULES\_PFTPARM}|spxpagem}

\begin{fulllineitems}
\phantomsection\label{\detokenize{namelists/pft_params.nml:JULES_PFTPARM::gsoil_f_io}}
\pysigstartsignatures
\pysigline{\sphinxcode{\sphinxupquote{JULES\_PFTPARM::}}\sphinxbfcode{\sphinxupquote{gsoil\_f\_io}}}
\pysigstopsignatures\begin{quote}\begin{description}
\sphinxlineitem{Type}
\sphinxAtStartPar
real(npft)

\sphinxlineitem{Default}
\sphinxAtStartPar
None

\end{description}\end{quote}

\sphinxAtStartPar
Soil conductance enhancement factor.

\sphinxAtStartPar
The soil conductance for soil under a PFT canopy may be modified for each PFT (as compared to the bare soil conductance) to account for PFT\sphinxhyphen{}dependent factors.

\end{fulllineitems}

\index{hw\_sw\_io (in namelist JULES\_PFTPARM)@\spxentry{hw\_sw\_io}\spxextra{in namelist JULES\_PFTPARM}|spxpagem}

\begin{fulllineitems}
\phantomsection\label{\detokenize{namelists/pft_params.nml:JULES_PFTPARM::hw_sw_io}}
\pysigstartsignatures
\pysigline{\sphinxcode{\sphinxupquote{JULES\_PFTPARM::}}\sphinxbfcode{\sphinxupquote{hw\_sw\_io}}}
\pysigstopsignatures\begin{quote}\begin{description}
\sphinxlineitem{Type}
\sphinxAtStartPar
real(npft)

\sphinxlineitem{Default}
\sphinxAtStartPar
None

\end{description}\end{quote}

\sphinxAtStartPar
Ratio of N stem to N heartwood (kgN/kgN) from the TRY database.

\sphinxAtStartPar
Only used if {\hyperref[\detokenize{namelists/jules_vegetation.nml:JULES_VEGETATION::l_trait_phys}]{\sphinxcrossref{\sphinxcode{\sphinxupquote{l\_trait\_phys}}}}} = T.

\end{fulllineitems}

\index{kext\_io (in namelist JULES\_PFTPARM)@\spxentry{kext\_io}\spxextra{in namelist JULES\_PFTPARM}|spxpagem}

\begin{fulllineitems}
\phantomsection\label{\detokenize{namelists/pft_params.nml:JULES_PFTPARM::kext_io}}
\pysigstartsignatures
\pysigline{\sphinxcode{\sphinxupquote{JULES\_PFTPARM::}}\sphinxbfcode{\sphinxupquote{kext\_io}}}
\pysigstopsignatures\begin{quote}\begin{description}
\sphinxlineitem{Type}
\sphinxAtStartPar
real(npft)

\sphinxlineitem{Default}
\sphinxAtStartPar
None

\end{description}\end{quote}

\sphinxAtStartPar
Light extinction coefficient \sphinxhyphen{} used with Beer’s Law for light absorption through plant canopies. See HCTN30 Eq.3.

\end{fulllineitems}

\index{kpar\_io (in namelist JULES\_PFTPARM)@\spxentry{kpar\_io}\spxextra{in namelist JULES\_PFTPARM}|spxpagem}

\begin{fulllineitems}
\phantomsection\label{\detokenize{namelists/pft_params.nml:JULES_PFTPARM::kpar_io}}
\pysigstartsignatures
\pysigline{\sphinxcode{\sphinxupquote{JULES\_PFTPARM::}}\sphinxbfcode{\sphinxupquote{kpar\_io}}}
\pysigstopsignatures\begin{quote}\begin{description}
\sphinxlineitem{Type}
\sphinxAtStartPar
real(npft)

\sphinxlineitem{Default}
\sphinxAtStartPar
None

\end{description}\end{quote}

\sphinxAtStartPar
PAR Extinction coefficient (m$^{\text{2}}$ leaf / m$^{\text{2}}$ ground).

\end{fulllineitems}

\index{lai\_alb\_lim\_io (in namelist JULES\_PFTPARM)@\spxentry{lai\_alb\_lim\_io}\spxextra{in namelist JULES\_PFTPARM}|spxpagem}

\begin{fulllineitems}
\phantomsection\label{\detokenize{namelists/pft_params.nml:JULES_PFTPARM::lai_alb_lim_io}}
\pysigstartsignatures
\pysigline{\sphinxcode{\sphinxupquote{JULES\_PFTPARM::}}\sphinxbfcode{\sphinxupquote{lai\_alb\_lim\_io}}}
\pysigstopsignatures\begin{quote}\begin{description}
\sphinxlineitem{Type}
\sphinxAtStartPar
real(npft)

\sphinxlineitem{Default}
\sphinxAtStartPar
None

\end{description}\end{quote}

\sphinxAtStartPar
Minimum LAI permitted in calculation of the albedo in snow\sphinxhyphen{}free conditions.

\end{fulllineitems}

\index{neff\_io (in namelist JULES\_PFTPARM)@\spxentry{neff\_io}\spxextra{in namelist JULES\_PFTPARM}|spxpagem}

\begin{fulllineitems}
\phantomsection\label{\detokenize{namelists/pft_params.nml:JULES_PFTPARM::neff_io}}
\pysigstartsignatures
\pysigline{\sphinxcode{\sphinxupquote{JULES\_PFTPARM::}}\sphinxbfcode{\sphinxupquote{neff\_io}}}
\pysigstopsignatures\begin{quote}\begin{description}
\sphinxlineitem{Type}
\sphinxAtStartPar
real(npft)

\sphinxlineitem{Default}
\sphinxAtStartPar
None

\end{description}\end{quote}

\sphinxAtStartPar
Scale factor relating V$_{\text{cmax}}$ with leaf nitrogen concentration. See HCTN 24 Eq. 51.

\sphinxAtStartPar
Only used if {\hyperref[\detokenize{namelists/jules_vegetation.nml:JULES_VEGETATION::l_trait_phys}]{\sphinxcrossref{\sphinxcode{\sphinxupquote{l\_trait\_phys}}}}} = F.

\end{fulllineitems}

\index{nl0\_io (in namelist JULES\_PFTPARM)@\spxentry{nl0\_io}\spxextra{in namelist JULES\_PFTPARM}|spxpagem}

\begin{fulllineitems}
\phantomsection\label{\detokenize{namelists/pft_params.nml:JULES_PFTPARM::nl0_io}}
\pysigstartsignatures
\pysigline{\sphinxcode{\sphinxupquote{JULES\_PFTPARM::}}\sphinxbfcode{\sphinxupquote{nl0\_io}}}
\pysigstopsignatures\begin{quote}\begin{description}
\sphinxlineitem{Type}
\sphinxAtStartPar
real(npft)

\sphinxlineitem{Default}
\sphinxAtStartPar
None

\end{description}\end{quote}

\sphinxAtStartPar
Top leaf nitrogen concentration (kg N/kg C).

\sphinxAtStartPar
Only used if {\hyperref[\detokenize{namelists/jules_vegetation.nml:JULES_VEGETATION::l_trait_phys}]{\sphinxcrossref{\sphinxcode{\sphinxupquote{l\_trait\_phys}}}}} = F.

\end{fulllineitems}

\index{nr\_nl\_io (in namelist JULES\_PFTPARM)@\spxentry{nr\_nl\_io}\spxextra{in namelist JULES\_PFTPARM}|spxpagem}

\begin{fulllineitems}
\phantomsection\label{\detokenize{namelists/pft_params.nml:JULES_PFTPARM::nr_nl_io}}
\pysigstartsignatures
\pysigline{\sphinxcode{\sphinxupquote{JULES\_PFTPARM::}}\sphinxbfcode{\sphinxupquote{nr\_nl\_io}}}
\pysigstopsignatures\begin{quote}\begin{description}
\sphinxlineitem{Type}
\sphinxAtStartPar
real(npft)

\sphinxlineitem{Default}
\sphinxAtStartPar
None

\end{description}\end{quote}

\sphinxAtStartPar
Ratio of root nitrogen concentration to leaf nitrogen concentration.

\end{fulllineitems}

\index{nr\_io (in namelist JULES\_PFTPARM)@\spxentry{nr\_io}\spxextra{in namelist JULES\_PFTPARM}|spxpagem}

\begin{fulllineitems}
\phantomsection\label{\detokenize{namelists/pft_params.nml:JULES_PFTPARM::nr_io}}
\pysigstartsignatures
\pysigline{\sphinxcode{\sphinxupquote{JULES\_PFTPARM::}}\sphinxbfcode{\sphinxupquote{nr\_io}}}
\pysigstopsignatures\begin{quote}\begin{description}
\sphinxlineitem{Type}
\sphinxAtStartPar
real(npft)

\sphinxlineitem{Default}
\sphinxAtStartPar
None

\end{description}\end{quote}

\sphinxAtStartPar
Root nitrogen concentration  (kgN/kgC).
Only used if {\hyperref[\detokenize{namelists/jules_vegetation.nml:JULES_VEGETATION::l_trait_phys}]{\sphinxcrossref{\sphinxcode{\sphinxupquote{l\_trait\_phys}}}}} = T.

\end{fulllineitems}

\index{ns\_nl\_io (in namelist JULES\_PFTPARM)@\spxentry{ns\_nl\_io}\spxextra{in namelist JULES\_PFTPARM}|spxpagem}

\begin{fulllineitems}
\phantomsection\label{\detokenize{namelists/pft_params.nml:JULES_PFTPARM::ns_nl_io}}
\pysigstartsignatures
\pysigline{\sphinxcode{\sphinxupquote{JULES\_PFTPARM::}}\sphinxbfcode{\sphinxupquote{ns\_nl\_io}}}
\pysigstopsignatures\begin{quote}\begin{description}
\sphinxlineitem{Type}
\sphinxAtStartPar
real(npft)

\sphinxlineitem{Default}
\sphinxAtStartPar
None

\end{description}\end{quote}

\sphinxAtStartPar
Ratio of stem nitrogen concentration to leaf nitrogen concentration.

\end{fulllineitems}

\index{nsw\_io (in namelist JULES\_PFTPARM)@\spxentry{nsw\_io}\spxextra{in namelist JULES\_PFTPARM}|spxpagem}

\begin{fulllineitems}
\phantomsection\label{\detokenize{namelists/pft_params.nml:JULES_PFTPARM::nsw_io}}
\pysigstartsignatures
\pysigline{\sphinxcode{\sphinxupquote{JULES\_PFTPARM::}}\sphinxbfcode{\sphinxupquote{nsw\_io}}}
\pysigstopsignatures\begin{quote}\begin{description}
\sphinxlineitem{Type}
\sphinxAtStartPar
real(npft)

\sphinxlineitem{Default}
\sphinxAtStartPar
None

\end{description}\end{quote}

\sphinxAtStartPar
Stemwood nitrogen concentration (kgN/kgC).
Only used if {\hyperref[\detokenize{namelists/jules_vegetation.nml:JULES_VEGETATION::l_trait_phys}]{\sphinxcrossref{\sphinxcode{\sphinxupquote{l\_trait\_phys}}}}} = T.

\end{fulllineitems}

\index{hw\_sw\_io (in namelist JULES\_PFTPARM)@\spxentry{hw\_sw\_io}\spxextra{in namelist JULES\_PFTPARM}|spxpagem}

\begin{fulllineitems}

\pysigstartsignatures
\pysigline{\sphinxcode{\sphinxupquote{JULES\_PFTPARM::}}\sphinxbfcode{\sphinxupquote{hw\_sw\_io}}}
\pysigstopsignatures\begin{quote}\begin{description}
\sphinxlineitem{Type}
\sphinxAtStartPar
real(npft)

\sphinxlineitem{Default}
\sphinxAtStartPar
None

\end{description}\end{quote}

\sphinxAtStartPar
Ratio of Heartwood to Stemwood Nitrogen Concentration (typically 0.5)
Only used if {\hyperref[\detokenize{namelists/jules_vegetation.nml:JULES_VEGETATION::l_trait_phys}]{\sphinxcrossref{\sphinxcode{\sphinxupquote{l\_trait\_phys}}}}} = T.

\end{fulllineitems}

\index{omega\_io (in namelist JULES\_PFTPARM)@\spxentry{omega\_io}\spxextra{in namelist JULES\_PFTPARM}|spxpagem}

\begin{fulllineitems}
\phantomsection\label{\detokenize{namelists/pft_params.nml:JULES_PFTPARM::omega_io}}
\pysigstartsignatures
\pysigline{\sphinxcode{\sphinxupquote{JULES\_PFTPARM::}}\sphinxbfcode{\sphinxupquote{omega\_io}}}
\pysigstopsignatures\begin{quote}\begin{description}
\sphinxlineitem{Type}
\sphinxAtStartPar
real(npft)

\sphinxlineitem{Default}
\sphinxAtStartPar
None

\end{description}\end{quote}

\sphinxAtStartPar
Leaf scattering coefficient for PAR.

\sphinxAtStartPar
Always used unless {\hyperref[\detokenize{namelists/jules_vegetation.nml:JULES_VEGETATION::can_rad_mod}]{\sphinxcrossref{\sphinxcode{\sphinxupquote{can\_rad\_mod}}}}} = 1 and
{\hyperref[\detokenize{namelists/jules_radiation.nml:JULES_RADIATION::l_spec_albedo}]{\sphinxcrossref{\sphinxcode{\sphinxupquote{l\_spec\_albedo}}}}} = FALSE.

\end{fulllineitems}

\index{omegau\_io (in namelist JULES\_PFTPARM)@\spxentry{omegau\_io}\spxextra{in namelist JULES\_PFTPARM}|spxpagem}

\begin{fulllineitems}
\phantomsection\label{\detokenize{namelists/pft_params.nml:JULES_PFTPARM::omegau_io}}
\pysigstartsignatures
\pysigline{\sphinxcode{\sphinxupquote{JULES\_PFTPARM::}}\sphinxbfcode{\sphinxupquote{omegau\_io}}}
\pysigstopsignatures\begin{quote}\begin{description}
\sphinxlineitem{Type}
\sphinxAtStartPar
real(npft)

\sphinxlineitem{Default}
\sphinxAtStartPar
None

\end{description}\end{quote}

\sphinxAtStartPar
Upper limit for the leaf scattering coefficient for PAR, when
{\hyperref[\detokenize{namelists/jules_radiation.nml:JULES_RADIATION::l_albedo_obs}]{\sphinxcrossref{\sphinxcode{\sphinxupquote{l\_albedo\_obs}}}}} = TRUE and when
{\hyperref[\detokenize{namelists/pft_params.nml:JULES_PFTPARM::omega_io}]{\sphinxcrossref{\sphinxcode{\sphinxupquote{omega\_io}}}}} is used.

\end{fulllineitems}

\index{omegal\_io (in namelist JULES\_PFTPARM)@\spxentry{omegal\_io}\spxextra{in namelist JULES\_PFTPARM}|spxpagem}

\begin{fulllineitems}
\phantomsection\label{\detokenize{namelists/pft_params.nml:JULES_PFTPARM::omegal_io}}
\pysigstartsignatures
\pysigline{\sphinxcode{\sphinxupquote{JULES\_PFTPARM::}}\sphinxbfcode{\sphinxupquote{omegal\_io}}}
\pysigstopsignatures\begin{quote}\begin{description}
\sphinxlineitem{Type}
\sphinxAtStartPar
real(npft)

\sphinxlineitem{Default}
\sphinxAtStartPar
None

\end{description}\end{quote}

\sphinxAtStartPar
Lower limit for the leaf scattering coefficient for PAR, when
{\hyperref[\detokenize{namelists/jules_radiation.nml:JULES_RADIATION::l_albedo_obs}]{\sphinxcrossref{\sphinxcode{\sphinxupquote{l\_albedo\_obs}}}}} = TRUE and when
{\hyperref[\detokenize{namelists/pft_params.nml:JULES_PFTPARM::omega_io}]{\sphinxcrossref{\sphinxcode{\sphinxupquote{omega\_io}}}}} is used.

\end{fulllineitems}

\index{omnir\_io (in namelist JULES\_PFTPARM)@\spxentry{omnir\_io}\spxextra{in namelist JULES\_PFTPARM}|spxpagem}

\begin{fulllineitems}
\phantomsection\label{\detokenize{namelists/pft_params.nml:JULES_PFTPARM::omnir_io}}
\pysigstartsignatures
\pysigline{\sphinxcode{\sphinxupquote{JULES\_PFTPARM::}}\sphinxbfcode{\sphinxupquote{omnir\_io}}}
\pysigstopsignatures\begin{quote}\begin{description}
\sphinxlineitem{Type}
\sphinxAtStartPar
real(npft)

\sphinxlineitem{Default}
\sphinxAtStartPar
None

\end{description}\end{quote}

\sphinxAtStartPar
Leaf scattering coefficient for NIR.

\sphinxAtStartPar
Always used unless {\hyperref[\detokenize{namelists/jules_vegetation.nml:JULES_VEGETATION::can_rad_mod}]{\sphinxcrossref{\sphinxcode{\sphinxupquote{can\_rad\_mod}}}}} = 1 and
{\hyperref[\detokenize{namelists/jules_radiation.nml:JULES_RADIATION::l_spec_albedo}]{\sphinxcrossref{\sphinxcode{\sphinxupquote{l\_spec\_albedo}}}}} = FALSE.

\end{fulllineitems}

\index{omniru\_io (in namelist JULES\_PFTPARM)@\spxentry{omniru\_io}\spxextra{in namelist JULES\_PFTPARM}|spxpagem}

\begin{fulllineitems}
\phantomsection\label{\detokenize{namelists/pft_params.nml:JULES_PFTPARM::omniru_io}}
\pysigstartsignatures
\pysigline{\sphinxcode{\sphinxupquote{JULES\_PFTPARM::}}\sphinxbfcode{\sphinxupquote{omniru\_io}}}
\pysigstopsignatures\begin{quote}\begin{description}
\sphinxlineitem{Type}
\sphinxAtStartPar
real(npft)

\sphinxlineitem{Default}
\sphinxAtStartPar
None

\end{description}\end{quote}

\sphinxAtStartPar
Upper limit for the leaf scattering coefficient for NIR, when
{\hyperref[\detokenize{namelists/jules_radiation.nml:JULES_RADIATION::l_albedo_obs}]{\sphinxcrossref{\sphinxcode{\sphinxupquote{l\_albedo\_obs}}}}} = TRUE and when
{\hyperref[\detokenize{namelists/pft_params.nml:JULES_PFTPARM::omnir_io}]{\sphinxcrossref{\sphinxcode{\sphinxupquote{omnir\_io}}}}} is used.

\end{fulllineitems}

\index{omnirl\_io (in namelist JULES\_PFTPARM)@\spxentry{omnirl\_io}\spxextra{in namelist JULES\_PFTPARM}|spxpagem}

\begin{fulllineitems}
\phantomsection\label{\detokenize{namelists/pft_params.nml:JULES_PFTPARM::omnirl_io}}
\pysigstartsignatures
\pysigline{\sphinxcode{\sphinxupquote{JULES\_PFTPARM::}}\sphinxbfcode{\sphinxupquote{omnirl\_io}}}
\pysigstopsignatures\begin{quote}\begin{description}
\sphinxlineitem{Type}
\sphinxAtStartPar
real(npft)

\sphinxlineitem{Default}
\sphinxAtStartPar
None

\end{description}\end{quote}

\sphinxAtStartPar
Lower limit for the leaf scattering coefficient for NIR, when
{\hyperref[\detokenize{namelists/jules_radiation.nml:JULES_RADIATION::l_albedo_obs}]{\sphinxcrossref{\sphinxcode{\sphinxupquote{l\_albedo\_obs}}}}} = TRUE and when
{\hyperref[\detokenize{namelists/pft_params.nml:JULES_PFTPARM::omnir_io}]{\sphinxcrossref{\sphinxcode{\sphinxupquote{omnir\_io}}}}} is used.

\end{fulllineitems}

\index{r\_grow\_io (in namelist JULES\_PFTPARM)@\spxentry{r\_grow\_io}\spxextra{in namelist JULES\_PFTPARM}|spxpagem}

\begin{fulllineitems}
\phantomsection\label{\detokenize{namelists/pft_params.nml:JULES_PFTPARM::r_grow_io}}
\pysigstartsignatures
\pysigline{\sphinxcode{\sphinxupquote{JULES\_PFTPARM::}}\sphinxbfcode{\sphinxupquote{r\_grow\_io}}}
\pysigstopsignatures\begin{quote}\begin{description}
\sphinxlineitem{Type}
\sphinxAtStartPar
real(npft)

\sphinxlineitem{Default}
\sphinxAtStartPar
None

\end{description}\end{quote}

\sphinxAtStartPar
Growth respiration fraction.

\end{fulllineitems}

\index{fsmc\_mod\_io (in namelist JULES\_PFTPARM)@\spxentry{fsmc\_mod\_io}\spxextra{in namelist JULES\_PFTPARM}|spxpagem}

\begin{fulllineitems}
\phantomsection\label{\detokenize{namelists/pft_params.nml:JULES_PFTPARM::fsmc_mod_io}}
\pysigstartsignatures
\pysigline{\sphinxcode{\sphinxupquote{JULES\_PFTPARM::}}\sphinxbfcode{\sphinxupquote{fsmc\_mod\_io}}}
\pysigstopsignatures\begin{quote}\begin{description}
\sphinxlineitem{Type}
\sphinxAtStartPar
integer(npft)

\sphinxlineitem{Default}
\sphinxAtStartPar
None

\end{description}\end{quote}

\sphinxAtStartPar
Switch for method of weighting the contribution that different soil layers make to the soil moisture availability factor fsmc.
\begin{enumerate}
\sphinxsetlistlabels{\arabic}{enumi}{enumii}{}{.}%
\setcounter{enumi}{-1}
\item {} 
\sphinxAtStartPar
(recommended) Calculate fsmc in each soil layer and take a weighted average, using the fraction of roots in each layer as weights. Root distribution e\sphinxhyphen{}folding depth is given by {\hyperref[\detokenize{namelists/pft_params.nml:JULES_PFTPARM::rootd_ft_io}]{\sphinxcrossref{\sphinxcode{\sphinxupquote{rootd\_ft\_io}}}}}.

\item {} 
\sphinxAtStartPar
Calculate fsmc using average properties for the root zone. Depth of root zone is given by {\hyperref[\detokenize{namelists/pft_params.nml:JULES_PFTPARM::rootd_ft_io}]{\sphinxcrossref{\sphinxcode{\sphinxupquote{rootd\_ft\_io}}}}}. This is not currently allowed if layered soil C ({\hyperref[\detokenize{namelists/jules_soil_biogeochem.nml:JULES_SOIL_BIOGEOCHEM::l_layeredc}]{\sphinxcrossref{\sphinxcode{\sphinxupquote{l\_layeredc}}}}} = TRUE) and the 4\sphinxhyphen{}pool model are selected ({\hyperref[\detokenize{namelists/jules_soil_biogeochem.nml:JULES_SOIL_BIOGEOCHEM::soil_bgc_model}]{\sphinxcrossref{\sphinxcode{\sphinxupquote{soil\_bgc\_model}}}}} = 2)  because of unplanned effects on litter inputs.

\end{enumerate}

\end{fulllineitems}

\index{psi\_open\_io (in namelist JULES\_PFTPARM)@\spxentry{psi\_open\_io}\spxextra{in namelist JULES\_PFTPARM}|spxpagem}

\begin{fulllineitems}
\phantomsection\label{\detokenize{namelists/pft_params.nml:JULES_PFTPARM::psi_open_io}}
\pysigstartsignatures
\pysigline{\sphinxcode{\sphinxupquote{JULES\_PFTPARM::}}\sphinxbfcode{\sphinxupquote{psi\_open\_io}}}
\pysigstopsignatures\begin{quote}\begin{description}
\sphinxlineitem{Type}
\sphinxAtStartPar
real(npft)

\sphinxlineitem{Default}
\sphinxAtStartPar
None

\end{description}\end{quote}

\sphinxAtStartPar
Soil potential above which the soil moisture stress factor on vegetation (fsmc) is one. Unit: Pa. Allowed range: must be negative. Only used if {\hyperref[\detokenize{namelists/jules_vegetation.nml:JULES_VEGETATION::l_use_pft_psi}]{\sphinxcrossref{\sphinxcode{\sphinxupquote{l\_use\_pft\_psi}}}}} = T.

\end{fulllineitems}

\index{psi\_close\_io (in namelist JULES\_PFTPARM)@\spxentry{psi\_close\_io}\spxextra{in namelist JULES\_PFTPARM}|spxpagem}

\begin{fulllineitems}
\phantomsection\label{\detokenize{namelists/pft_params.nml:JULES_PFTPARM::psi_close_io}}
\pysigstartsignatures
\pysigline{\sphinxcode{\sphinxupquote{JULES\_PFTPARM::}}\sphinxbfcode{\sphinxupquote{psi\_close\_io}}}
\pysigstopsignatures\begin{quote}\begin{description}
\sphinxlineitem{Type}
\sphinxAtStartPar
real(npft)

\sphinxlineitem{Default}
\sphinxAtStartPar
None

\end{description}\end{quote}

\sphinxAtStartPar
Soil potential below which the soil moisture stress factor on vegetation (fsmc) is zero. Unit: Pa. Allowed range: must be negative. Only used if {\hyperref[\detokenize{namelists/jules_vegetation.nml:JULES_VEGETATION::l_use_pft_psi}]{\sphinxcrossref{\sphinxcode{\sphinxupquote{l\_use\_pft\_psi}}}}} = T.

\end{fulllineitems}

\index{rootd\_ft\_io (in namelist JULES\_PFTPARM)@\spxentry{rootd\_ft\_io}\spxextra{in namelist JULES\_PFTPARM}|spxpagem}

\begin{fulllineitems}
\phantomsection\label{\detokenize{namelists/pft_params.nml:JULES_PFTPARM::rootd_ft_io}}
\pysigstartsignatures
\pysigline{\sphinxcode{\sphinxupquote{JULES\_PFTPARM::}}\sphinxbfcode{\sphinxupquote{rootd\_ft\_io}}}
\pysigstopsignatures\begin{quote}\begin{description}
\sphinxlineitem{Type}
\sphinxAtStartPar
real(npft)

\sphinxlineitem{Default}
\sphinxAtStartPar
None

\end{description}\end{quote}

\sphinxAtStartPar
Parameter determining the root depth (m).

\sphinxAtStartPar
If {\hyperref[\detokenize{namelists/pft_params.nml:JULES_PFTPARM::fsmc_mod_io}]{\sphinxcrossref{\sphinxcode{\sphinxupquote{fsmc\_mod\_io}}}}} = 0, an exponential root distribution with depth is assumed, with e\sphinxhyphen{}folding depth \sphinxcode{\sphinxupquote{rootd\_ft}} (see HCTN30 Eq.32). Note that this means that generally some of the roots exist at depths greater than \sphinxcode{\sphinxupquote{rootd\_ft}}. If {\hyperref[\detokenize{namelists/pft_params.nml:JULES_PFTPARM::fsmc_mod_io}]{\sphinxcrossref{\sphinxcode{\sphinxupquote{fsmc\_mod\_io}}}}} = 1, \sphinxcode{\sphinxupquote{rootd\_ft}} is the total depth of the root zone.

\end{fulllineitems}

\index{fsmc\_p0\_io (in namelist JULES\_PFTPARM)@\spxentry{fsmc\_p0\_io}\spxextra{in namelist JULES\_PFTPARM}|spxpagem}

\begin{fulllineitems}
\phantomsection\label{\detokenize{namelists/pft_params.nml:JULES_PFTPARM::fsmc_p0_io}}
\pysigstartsignatures
\pysigline{\sphinxcode{\sphinxupquote{JULES\_PFTPARM::}}\sphinxbfcode{\sphinxupquote{fsmc\_p0\_io}}}
\pysigstopsignatures\begin{quote}\begin{description}
\sphinxlineitem{Type}
\sphinxAtStartPar
real(npft)

\sphinxlineitem{Default}
\sphinxAtStartPar
None

\end{description}\end{quote}

\sphinxAtStartPar
Pft\sphinxhyphen{}dependent parameter governing the threshold at which the plant starts to experience water stress due to lack of water in the soil. Only used if {\hyperref[\detokenize{namelists/jules_vegetation.nml:JULES_VEGETATION::l_use_pft_psi}]{\sphinxcrossref{\sphinxcode{\sphinxupquote{l\_use\_pft\_psi}}}}} = F. The volumetric soil moisture content (m$^{\text{3}}$ water per m$^{\text{3}}$ soil) at which the plant starts to become water stressed is \sphinxcode{\sphinxupquote{sm\_wilt+(sm\_crit\sphinxhyphen{}sm\_wilt)*(1\sphinxhyphen{}fsmc\_p0)}} (see {\hyperref[\detokenize{namelists/ancillaries.nml:namelist-JULES_SOIL_PROPS}]{\sphinxcrossref{\sphinxcode{\sphinxupquote{JULES\_SOIL\_PROPS}}}}} for a description of \sphinxcode{\sphinxupquote{sm\_wilt}} and \sphinxcode{\sphinxupquote{sm\_crit}}).

\end{fulllineitems}

\index{sigl\_io (in namelist JULES\_PFTPARM)@\spxentry{sigl\_io}\spxextra{in namelist JULES\_PFTPARM}|spxpagem}

\begin{fulllineitems}
\phantomsection\label{\detokenize{namelists/pft_params.nml:JULES_PFTPARM::sigl_io}}
\pysigstartsignatures
\pysigline{\sphinxcode{\sphinxupquote{JULES\_PFTPARM::}}\sphinxbfcode{\sphinxupquote{sigl\_io}}}
\pysigstopsignatures\begin{quote}\begin{description}
\sphinxlineitem{Type}
\sphinxAtStartPar
real(npft)

\sphinxlineitem{Default}
\sphinxAtStartPar
None

\end{description}\end{quote}

\sphinxAtStartPar
Specific density of leaf carbon (kg C/m$^{\text{2}}$ leaf) (Clark et al., 2011; Table 7).

\sphinxAtStartPar
Only used if {\hyperref[\detokenize{namelists/jules_vegetation.nml:JULES_VEGETATION::l_trait_phys}]{\sphinxcrossref{\sphinxcode{\sphinxupquote{l\_trait\_phys}}}}} = F.

\end{fulllineitems}

\index{tleaf\_of\_io (in namelist JULES\_PFTPARM)@\spxentry{tleaf\_of\_io}\spxextra{in namelist JULES\_PFTPARM}|spxpagem}

\begin{fulllineitems}
\phantomsection\label{\detokenize{namelists/pft_params.nml:JULES_PFTPARM::tleaf_of_io}}
\pysigstartsignatures
\pysigline{\sphinxcode{\sphinxupquote{JULES\_PFTPARM::}}\sphinxbfcode{\sphinxupquote{tleaf\_of\_io}}}
\pysigstopsignatures\begin{quote}\begin{description}
\sphinxlineitem{Type}
\sphinxAtStartPar
real(npft)

\sphinxlineitem{Default}
\sphinxAtStartPar
None

\end{description}\end{quote}

\sphinxAtStartPar
Temperature below which leaves are dropped (K).

\end{fulllineitems}

\index{tlow\_io (in namelist JULES\_PFTPARM)@\spxentry{tlow\_io}\spxextra{in namelist JULES\_PFTPARM}|spxpagem}

\begin{fulllineitems}
\phantomsection\label{\detokenize{namelists/pft_params.nml:JULES_PFTPARM::tlow_io}}
\pysigstartsignatures
\pysigline{\sphinxcode{\sphinxupquote{JULES\_PFTPARM::}}\sphinxbfcode{\sphinxupquote{tlow\_io}}}
\pysigstopsignatures\begin{quote}\begin{description}
\sphinxlineitem{Type}
\sphinxAtStartPar
real(npft)

\sphinxlineitem{Default}
\sphinxAtStartPar
None

\end{description}\end{quote}

\sphinxAtStartPar
Lower temperature parameter for photosynthesis (deg C), for the Collatz model of leaf photosynthesis.

\sphinxAtStartPar
Always used for C$_{\text{4}}$ plants. Only used for C$_{\text{3}}$ plants with the Collatz model of leaf photosynthesis ({\hyperref[\detokenize{namelists/jules_vegetation.nml:JULES_VEGETATION::photo_model}]{\sphinxcrossref{\sphinxcode{\sphinxupquote{photo\_model}}}}} = 1).

\end{fulllineitems}

\index{tupp\_io (in namelist JULES\_PFTPARM)@\spxentry{tupp\_io}\spxextra{in namelist JULES\_PFTPARM}|spxpagem}

\begin{fulllineitems}
\phantomsection\label{\detokenize{namelists/pft_params.nml:JULES_PFTPARM::tupp_io}}
\pysigstartsignatures
\pysigline{\sphinxcode{\sphinxupquote{JULES\_PFTPARM::}}\sphinxbfcode{\sphinxupquote{tupp\_io}}}
\pysigstopsignatures\begin{quote}\begin{description}
\sphinxlineitem{Type}
\sphinxAtStartPar
real(npft)

\sphinxlineitem{Default}
\sphinxAtStartPar
None

\end{description}\end{quote}

\sphinxAtStartPar
Upper temperature  parameter for photosynthesis (deg C), for the Collatz model of leaf photosynthesis.

\sphinxAtStartPar
Always used for C$_{\text{4}}$ plants. Only used for C$_{\text{3}}$ plants with the Collatz model of leaf photosynthesis ({\hyperref[\detokenize{namelists/jules_vegetation.nml:JULES_VEGETATION::photo_model}]{\sphinxcrossref{\sphinxcode{\sphinxupquote{photo\_model}}}}} = 1).

\end{fulllineitems}

\index{emis\_pft\_io (in namelist JULES\_PFTPARM)@\spxentry{emis\_pft\_io}\spxextra{in namelist JULES\_PFTPARM}|spxpagem}

\begin{fulllineitems}
\phantomsection\label{\detokenize{namelists/pft_params.nml:JULES_PFTPARM::emis_pft_io}}
\pysigstartsignatures
\pysigline{\sphinxcode{\sphinxupquote{JULES\_PFTPARM::}}\sphinxbfcode{\sphinxupquote{emis\_pft\_io}}}
\pysigstopsignatures\begin{quote}\begin{description}
\sphinxlineitem{Type}
\sphinxAtStartPar
real(npft)

\sphinxlineitem{Default}
\sphinxAtStartPar
None

\end{description}\end{quote}

\sphinxAtStartPar
Surface emissivity of vegetated surfaces.

\end{fulllineitems}

\index{z0hm\_pft\_io (in namelist JULES\_PFTPARM)@\spxentry{z0hm\_pft\_io}\spxextra{in namelist JULES\_PFTPARM}|spxpagem}

\begin{fulllineitems}
\phantomsection\label{\detokenize{namelists/pft_params.nml:JULES_PFTPARM::z0hm_pft_io}}
\pysigstartsignatures
\pysigline{\sphinxcode{\sphinxupquote{JULES\_PFTPARM::}}\sphinxbfcode{\sphinxupquote{z0hm\_pft\_io}}}
\pysigstopsignatures\begin{quote}\begin{description}
\sphinxlineitem{Type}
\sphinxAtStartPar
real(npft)

\sphinxlineitem{Default}
\sphinxAtStartPar
None

\end{description}\end{quote}

\sphinxAtStartPar
Ratio of the roughness length for heat to the roughness length for momentum.

\sphinxAtStartPar
This is generally assumed to be 0.1. See HCTN30 p6. Note that this is the ratio of the roughness length for heat to that for momentum. It does not alter the roughness length for momentum, which is calculated using {\hyperref[\detokenize{namelists/pft_params.nml:JULES_PFTPARM::canht_ft_io}]{\sphinxcrossref{\sphinxcode{\sphinxupquote{canht\_ft\_io}}}}} and {\hyperref[\detokenize{namelists/pft_params.nml:JULES_PFTPARM::dz0v_dh_io}]{\sphinxcrossref{\sphinxcode{\sphinxupquote{dz0v\_dh\_io}}}}}.

\end{fulllineitems}

\index{z0hm\_classic\_pft\_io (in namelist JULES\_PFTPARM)@\spxentry{z0hm\_classic\_pft\_io}\spxextra{in namelist JULES\_PFTPARM}|spxpagem}

\begin{fulllineitems}
\phantomsection\label{\detokenize{namelists/pft_params.nml:JULES_PFTPARM::z0hm_classic_pft_io}}
\pysigstartsignatures
\pysigline{\sphinxcode{\sphinxupquote{JULES\_PFTPARM::}}\sphinxbfcode{\sphinxupquote{z0hm\_classic\_pft\_io}}}
\pysigstopsignatures\begin{quote}\begin{description}
\sphinxlineitem{Type}
\sphinxAtStartPar
real(npft)

\sphinxlineitem{Default}
\sphinxAtStartPar
None

\end{description}\end{quote}

\sphinxAtStartPar
Ratio of the roughness length for heat to the roughness length for momentum \sphinxstyleemphasis{for the CLASSIC aerosol scheme only}.

\begin{sphinxadmonition}{note}{Note:}
\sphinxAtStartPar
This makes no difference to the model when running standalone, and is only required to keep the standalone and UM interfaces consistent.
\end{sphinxadmonition}

\end{fulllineitems}

\index{fl\_o3\_ct\_io (in namelist JULES\_PFTPARM)@\spxentry{fl\_o3\_ct\_io}\spxextra{in namelist JULES\_PFTPARM}|spxpagem}

\begin{fulllineitems}
\phantomsection\label{\detokenize{namelists/pft_params.nml:JULES_PFTPARM::fl_o3_ct_io}}
\pysigstartsignatures
\pysigline{\sphinxcode{\sphinxupquote{JULES\_PFTPARM::}}\sphinxbfcode{\sphinxupquote{fl\_o3\_ct\_io}}}
\pysigstopsignatures\begin{quote}\begin{description}
\sphinxlineitem{Type}
\sphinxAtStartPar
real(npft)

\sphinxlineitem{Default}
\sphinxAtStartPar
None

\end{description}\end{quote}

\sphinxAtStartPar
Critical flux of O3 to vegetation (nmol m$^{\text{\sphinxhyphen{}2}}$ s$^{\text{\sphinxhyphen{}1}}$).

\end{fulllineitems}

\index{dfp\_dcuo\_io (in namelist JULES\_PFTPARM)@\spxentry{dfp\_dcuo\_io}\spxextra{in namelist JULES\_PFTPARM}|spxpagem}

\begin{fulllineitems}
\phantomsection\label{\detokenize{namelists/pft_params.nml:JULES_PFTPARM::dfp_dcuo_io}}
\pysigstartsignatures
\pysigline{\sphinxcode{\sphinxupquote{JULES\_PFTPARM::}}\sphinxbfcode{\sphinxupquote{dfp\_dcuo\_io}}}
\pysigstopsignatures\begin{quote}\begin{description}
\sphinxlineitem{Type}
\sphinxAtStartPar
real(npft)

\sphinxlineitem{Default}
\sphinxAtStartPar
None

\end{description}\end{quote}

\sphinxAtStartPar
Plant type specific O3 sensitivity parameter (nmol$^{\text{\sphinxhyphen{}1}}$ m$^{\text{2}}$ s).

\end{fulllineitems}

\index{ief\_io (in namelist JULES\_PFTPARM)@\spxentry{ief\_io}\spxextra{in namelist JULES\_PFTPARM}|spxpagem}

\begin{fulllineitems}
\phantomsection\label{\detokenize{namelists/pft_params.nml:JULES_PFTPARM::ief_io}}
\pysigstartsignatures
\pysigline{\sphinxcode{\sphinxupquote{JULES\_PFTPARM::}}\sphinxbfcode{\sphinxupquote{ief\_io}}}
\pysigstopsignatures\begin{quote}\begin{description}
\sphinxlineitem{Type}
\sphinxAtStartPar
real(npft)

\sphinxlineitem{Default}
\sphinxAtStartPar
None

\end{description}\end{quote}

\sphinxAtStartPar
Isoprene Emission Factor (μg g$^{\text{\sphinxhyphen{}1}}$ h$^{\text{\sphinxhyphen{}1}}$).

\end{fulllineitems}

\index{tef\_io (in namelist JULES\_PFTPARM)@\spxentry{tef\_io}\spxextra{in namelist JULES\_PFTPARM}|spxpagem}

\begin{fulllineitems}
\phantomsection\label{\detokenize{namelists/pft_params.nml:JULES_PFTPARM::tef_io}}
\pysigstartsignatures
\pysigline{\sphinxcode{\sphinxupquote{JULES\_PFTPARM::}}\sphinxbfcode{\sphinxupquote{tef\_io}}}
\pysigstopsignatures\begin{quote}\begin{description}
\sphinxlineitem{Type}
\sphinxAtStartPar
real(npft)

\sphinxlineitem{Default}
\sphinxAtStartPar
None

\end{description}\end{quote}

\sphinxAtStartPar
Monoterpene Emission Factor (μg g$^{\text{\sphinxhyphen{}1}}$ h$^{\text{\sphinxhyphen{}1}}$).

\end{fulllineitems}

\index{mef\_io (in namelist JULES\_PFTPARM)@\spxentry{mef\_io}\spxextra{in namelist JULES\_PFTPARM}|spxpagem}

\begin{fulllineitems}
\phantomsection\label{\detokenize{namelists/pft_params.nml:JULES_PFTPARM::mef_io}}
\pysigstartsignatures
\pysigline{\sphinxcode{\sphinxupquote{JULES\_PFTPARM::}}\sphinxbfcode{\sphinxupquote{mef\_io}}}
\pysigstopsignatures\begin{quote}\begin{description}
\sphinxlineitem{Type}
\sphinxAtStartPar
real(npft)

\sphinxlineitem{Default}
\sphinxAtStartPar
None

\end{description}\end{quote}

\sphinxAtStartPar
Methanol Emission Factor (μg g$^{\text{\sphinxhyphen{}1}}$ h$^{\text{\sphinxhyphen{}1}}$).

\end{fulllineitems}

\index{aef\_io (in namelist JULES\_PFTPARM)@\spxentry{aef\_io}\spxextra{in namelist JULES\_PFTPARM}|spxpagem}

\begin{fulllineitems}
\phantomsection\label{\detokenize{namelists/pft_params.nml:JULES_PFTPARM::aef_io}}
\pysigstartsignatures
\pysigline{\sphinxcode{\sphinxupquote{JULES\_PFTPARM::}}\sphinxbfcode{\sphinxupquote{aef\_io}}}
\pysigstopsignatures\begin{quote}\begin{description}
\sphinxlineitem{Type}
\sphinxAtStartPar
real(npft)

\sphinxlineitem{Default}
\sphinxAtStartPar
None

\end{description}\end{quote}

\sphinxAtStartPar
Acetone Emission Factor (μg g$^{\text{\sphinxhyphen{}1}}$ h$^{\text{\sphinxhyphen{}1}}$).

\end{fulllineitems}

\index{ci\_st\_io (in namelist JULES\_PFTPARM)@\spxentry{ci\_st\_io}\spxextra{in namelist JULES\_PFTPARM}|spxpagem}

\begin{fulllineitems}
\phantomsection\label{\detokenize{namelists/pft_params.nml:JULES_PFTPARM::ci_st_io}}
\pysigstartsignatures
\pysigline{\sphinxcode{\sphinxupquote{JULES\_PFTPARM::}}\sphinxbfcode{\sphinxupquote{ci\_st\_io}}}
\pysigstopsignatures\begin{quote}\begin{description}
\sphinxlineitem{Tybe}
\sphinxAtStartPar
real(npft)

\sphinxlineitem{Default}
\sphinxAtStartPar
None

\end{description}\end{quote}

\sphinxAtStartPar
Leaf\sphinxhyphen{}internal CO$_{\text{2}}$concentration at standard conditions (Pa),

\begin{sphinxadmonition}{note}{Note:}
\sphinxAtStartPar
Standard conditions are: T = 303.15K, p = 1013.25 hPa, atmospheric CO$_{\text{2}}$ = 370 ppmv,  PAR = 1000 μmol m$^{\text{\sphinxhyphen{}2}}$ s$^{\text{\sphinxhyphen{}1}}$.
\end{sphinxadmonition}

\end{fulllineitems}

\index{gpp\_st\_io (in namelist JULES\_PFTPARM)@\spxentry{gpp\_st\_io}\spxextra{in namelist JULES\_PFTPARM}|spxpagem}

\begin{fulllineitems}
\phantomsection\label{\detokenize{namelists/pft_params.nml:JULES_PFTPARM::gpp_st_io}}
\pysigstartsignatures
\pysigline{\sphinxcode{\sphinxupquote{JULES\_PFTPARM::}}\sphinxbfcode{\sphinxupquote{gpp\_st\_io}}}
\pysigstopsignatures\begin{quote}\begin{description}
\sphinxlineitem{Tybe}
\sphinxAtStartPar
real(npft)

\sphinxlineitem{Default}
\sphinxAtStartPar
None

\end{description}\end{quote}

\sphinxAtStartPar
Gross primary production (GPP) at standard conditions (kgC m$^{\text{\sphinxhyphen{}2}}$ s$^{\text{\sphinxhyphen{}1}}$),

\begin{sphinxadmonition}{note}{Note:}
\sphinxAtStartPar
Standard conditions are: T = 303.15K, p = 1013.25 hPa, atmospheric CO$_{\text{2}}$ = 370 ppmv, PAR = 1000 μmol m$^{\text{\sphinxhyphen{}2}}$ s$^{\text{\sphinxhyphen{}1}}$.
\end{sphinxadmonition}

\end{fulllineitems}

\index{nmass\_io (in namelist JULES\_PFTPARM)@\spxentry{nmass\_io}\spxextra{in namelist JULES\_PFTPARM}|spxpagem}

\begin{fulllineitems}
\phantomsection\label{\detokenize{namelists/pft_params.nml:JULES_PFTPARM::nmass_io}}
\pysigstartsignatures
\pysigline{\sphinxcode{\sphinxupquote{JULES\_PFTPARM::}}\sphinxbfcode{\sphinxupquote{nmass\_io}}}
\pysigstopsignatures\begin{quote}\begin{description}
\sphinxlineitem{Type}
\sphinxAtStartPar
real(npft)

\sphinxlineitem{Default}
\sphinxAtStartPar
None

\end{description}\end{quote}

\sphinxAtStartPar
Top leaf nitrogen content per unit mass (kgN kgLeaf$^{\text{\sphinxhyphen{}1}}$).

\sphinxAtStartPar
Only used if {\hyperref[\detokenize{namelists/jules_vegetation.nml:JULES_VEGETATION::l_trait_phys}]{\sphinxcrossref{\sphinxcode{\sphinxupquote{l\_trait\_phys}}}}} = T.

\end{fulllineitems}

\index{lma\_io (in namelist JULES\_PFTPARM)@\spxentry{lma\_io}\spxextra{in namelist JULES\_PFTPARM}|spxpagem}

\begin{fulllineitems}
\phantomsection\label{\detokenize{namelists/pft_params.nml:JULES_PFTPARM::lma_io}}
\pysigstartsignatures
\pysigline{\sphinxcode{\sphinxupquote{JULES\_PFTPARM::}}\sphinxbfcode{\sphinxupquote{lma\_io}}}
\pysigstopsignatures\begin{quote}\begin{description}
\sphinxlineitem{Type}
\sphinxAtStartPar
real(npft)

\sphinxlineitem{Default}
\sphinxAtStartPar
None

\end{description}\end{quote}

\sphinxAtStartPar
Leaf mass per unit area (kgLeaf m$^{\text{\sphinxhyphen{}2}}$).

\sphinxAtStartPar
Only used if {\hyperref[\detokenize{namelists/jules_vegetation.nml:JULES_VEGETATION::l_trait_phys}]{\sphinxcrossref{\sphinxcode{\sphinxupquote{l\_trait\_phys}}}}} = T.

\end{fulllineitems}

\index{vint\_io (in namelist JULES\_PFTPARM)@\spxentry{vint\_io}\spxextra{in namelist JULES\_PFTPARM}|spxpagem}

\begin{fulllineitems}
\phantomsection\label{\detokenize{namelists/pft_params.nml:JULES_PFTPARM::vint_io}}
\pysigstartsignatures
\pysigline{\sphinxcode{\sphinxupquote{JULES\_PFTPARM::}}\sphinxbfcode{\sphinxupquote{vint\_io}}}
\pysigstopsignatures\begin{quote}\begin{description}
\sphinxlineitem{Type}
\sphinxAtStartPar
real(npft)

\sphinxlineitem{Default}
\sphinxAtStartPar
None

\end{description}\end{quote}

\sphinxAtStartPar
There is a linear relationship between Vcmax and Narea. Previously Vcmax was calculated as the product of nl0 and neff.

\sphinxAtStartPar
This is now replaced by a linear regression based on data reported in Kattge et al. 2009. Vint is the y\sphinxhyphen{}intercept, vsl is the slope.

\sphinxAtStartPar
Units: μmol CO$_{\text{2}}$ m$^{\text{\sphinxhyphen{}2}}$ s$^{\text{\sphinxhyphen{}1}}$.

\sphinxAtStartPar
Only used if {\hyperref[\detokenize{namelists/jules_vegetation.nml:JULES_VEGETATION::l_trait_phys}]{\sphinxcrossref{\sphinxcode{\sphinxupquote{l\_trait\_phys}}}}} = T.

\end{fulllineitems}

\index{vsl\_io (in namelist JULES\_PFTPARM)@\spxentry{vsl\_io}\spxextra{in namelist JULES\_PFTPARM}|spxpagem}

\begin{fulllineitems}
\phantomsection\label{\detokenize{namelists/pft_params.nml:JULES_PFTPARM::vsl_io}}
\pysigstartsignatures
\pysigline{\sphinxcode{\sphinxupquote{JULES\_PFTPARM::}}\sphinxbfcode{\sphinxupquote{vsl\_io}}}
\pysigstopsignatures\begin{quote}\begin{description}
\sphinxlineitem{Type}
\sphinxAtStartPar
real(npft)

\sphinxlineitem{Default}
\sphinxAtStartPar
None

\end{description}\end{quote}

\sphinxAtStartPar
Slope in the linear regression between Vcmax and Narea.

\sphinxAtStartPar
Units: μmol CO$_{\text{2}}$ gN$^{\text{\sphinxhyphen{}1}}$ s$^{\text{\sphinxhyphen{}1}}$.

\sphinxAtStartPar
Only used if {\hyperref[\detokenize{namelists/jules_vegetation.nml:JULES_VEGETATION::l_trait_phys}]{\sphinxcrossref{\sphinxcode{\sphinxupquote{l\_trait\_phys}}}}} = T.

\end{fulllineitems}

\index{kn\_io (in namelist JULES\_PFTPARM)@\spxentry{kn\_io}\spxextra{in namelist JULES\_PFTPARM}|spxpagem}

\begin{fulllineitems}
\phantomsection\label{\detokenize{namelists/pft_params.nml:JULES_PFTPARM::kn_io}}
\pysigstartsignatures
\pysigline{\sphinxcode{\sphinxupquote{JULES\_PFTPARM::}}\sphinxbfcode{\sphinxupquote{kn\_io}}}
\pysigstopsignatures\begin{quote}\begin{description}
\sphinxlineitem{Type}
\sphinxAtStartPar
real(npft)

\sphinxlineitem{Default}
\sphinxAtStartPar
None.

\end{description}\end{quote}

\sphinxAtStartPar
Parameter for decay of  nitrogen through the canopy, as a function of layers. Only used if {\hyperref[\detokenize{namelists/jules_vegetation.nml:JULES_VEGETATION::can_rad_mod}]{\sphinxcrossref{\sphinxcode{\sphinxupquote{can\_rad\_mod}}}}} = 4 or 5.

\end{fulllineitems}

\index{knl\_io (in namelist JULES\_PFTPARM)@\spxentry{knl\_io}\spxextra{in namelist JULES\_PFTPARM}|spxpagem}

\begin{fulllineitems}
\phantomsection\label{\detokenize{namelists/pft_params.nml:JULES_PFTPARM::knl_io}}
\pysigstartsignatures
\pysigline{\sphinxcode{\sphinxupquote{JULES\_PFTPARM::}}\sphinxbfcode{\sphinxupquote{knl\_io}}}
\pysigstopsignatures\begin{quote}\begin{description}
\sphinxlineitem{Type}
\sphinxAtStartPar
real(npft)

\sphinxlineitem{Default}
\sphinxAtStartPar
None.

\end{description}\end{quote}

\sphinxAtStartPar
Parameter for decay of  nitrogen through the canopy, as a function of LAI. Only used if {\hyperref[\detokenize{namelists/jules_vegetation.nml:JULES_VEGETATION::can_rad_mod}]{\sphinxcrossref{\sphinxcode{\sphinxupquote{can\_rad\_mod}}}}} = 6.

\end{fulllineitems}

\index{q10\_leaf\_io (in namelist JULES\_PFTPARM)@\spxentry{q10\_leaf\_io}\spxextra{in namelist JULES\_PFTPARM}|spxpagem}

\begin{fulllineitems}
\phantomsection\label{\detokenize{namelists/pft_params.nml:JULES_PFTPARM::q10_leaf_io}}
\pysigstartsignatures
\pysigline{\sphinxcode{\sphinxupquote{JULES\_PFTPARM::}}\sphinxbfcode{\sphinxupquote{q10\_leaf\_io}}}
\pysigstopsignatures\begin{quote}\begin{description}
\sphinxlineitem{Type}
\sphinxAtStartPar
real(npft)

\sphinxlineitem{Default}
\sphinxAtStartPar
None.

\end{description}\end{quote}

\sphinxAtStartPar
Q10 factor for plant respiration.

\sphinxAtStartPar
See Cox et al. (1999) Eq. 66.

\begin{sphinxadmonition}{note}{Note:}
\sphinxAtStartPar
Was previously a single parameter but now can have PFT\sphinxhyphen{}dependent values.
\end{sphinxadmonition}

\end{fulllineitems}

\index{fef\_co2\_io (in namelist JULES\_PFTPARM)@\spxentry{fef\_co2\_io}\spxextra{in namelist JULES\_PFTPARM}|spxpagem}

\begin{fulllineitems}
\phantomsection\label{\detokenize{namelists/pft_params.nml:JULES_PFTPARM::fef_co2_io}}
\pysigstartsignatures
\pysigline{\sphinxcode{\sphinxupquote{JULES\_PFTPARM::}}\sphinxbfcode{\sphinxupquote{fef\_co2\_io}}}
\pysigstopsignatures\begin{quote}\begin{description}
\sphinxlineitem{Type}
\sphinxAtStartPar
real(npft)

\sphinxlineitem{Default}
\sphinxAtStartPar
None

\end{description}\end{quote}

\sphinxAtStartPar
Fire CO$_{\text{2}}$ Emission Factor (g kg$^{\text{\sphinxhyphen{}1}}$).

\end{fulllineitems}

\index{fef\_co\_io (in namelist JULES\_PFTPARM)@\spxentry{fef\_co\_io}\spxextra{in namelist JULES\_PFTPARM}|spxpagem}

\begin{fulllineitems}
\phantomsection\label{\detokenize{namelists/pft_params.nml:JULES_PFTPARM::fef_co_io}}
\pysigstartsignatures
\pysigline{\sphinxcode{\sphinxupquote{JULES\_PFTPARM::}}\sphinxbfcode{\sphinxupquote{fef\_co\_io}}}
\pysigstopsignatures\begin{quote}\begin{description}
\sphinxlineitem{Type}
\sphinxAtStartPar
real(npft)

\sphinxlineitem{Default}
\sphinxAtStartPar
None

\end{description}\end{quote}

\sphinxAtStartPar
Fire CO Emission Factor (g kg$^{\text{\sphinxhyphen{}1}}$).

\end{fulllineitems}

\index{fef\_ch4\_io (in namelist JULES\_PFTPARM)@\spxentry{fef\_ch4\_io}\spxextra{in namelist JULES\_PFTPARM}|spxpagem}

\begin{fulllineitems}
\phantomsection\label{\detokenize{namelists/pft_params.nml:JULES_PFTPARM::fef_ch4_io}}
\pysigstartsignatures
\pysigline{\sphinxcode{\sphinxupquote{JULES\_PFTPARM::}}\sphinxbfcode{\sphinxupquote{fef\_ch4\_io}}}
\pysigstopsignatures\begin{quote}\begin{description}
\sphinxlineitem{Type}
\sphinxAtStartPar
real(npft)

\sphinxlineitem{Default}
\sphinxAtStartPar
None

\end{description}\end{quote}

\sphinxAtStartPar
Fire CH$_{\text{4}}$ Emission Factor (g kg$^{\text{\sphinxhyphen{}1}}$).

\end{fulllineitems}

\index{fef\_nox\_io (in namelist JULES\_PFTPARM)@\spxentry{fef\_nox\_io}\spxextra{in namelist JULES\_PFTPARM}|spxpagem}

\begin{fulllineitems}
\phantomsection\label{\detokenize{namelists/pft_params.nml:JULES_PFTPARM::fef_nox_io}}
\pysigstartsignatures
\pysigline{\sphinxcode{\sphinxupquote{JULES\_PFTPARM::}}\sphinxbfcode{\sphinxupquote{fef\_nox\_io}}}
\pysigstopsignatures\begin{quote}\begin{description}
\sphinxlineitem{Type}
\sphinxAtStartPar
real(npft)

\sphinxlineitem{Default}
\sphinxAtStartPar
None

\end{description}\end{quote}

\sphinxAtStartPar
Fire NOx Emission Factor (g kg$^{\text{\sphinxhyphen{}1}}$).

\end{fulllineitems}

\index{fef\_so2\_io (in namelist JULES\_PFTPARM)@\spxentry{fef\_so2\_io}\spxextra{in namelist JULES\_PFTPARM}|spxpagem}

\begin{fulllineitems}
\phantomsection\label{\detokenize{namelists/pft_params.nml:JULES_PFTPARM::fef_so2_io}}
\pysigstartsignatures
\pysigline{\sphinxcode{\sphinxupquote{JULES\_PFTPARM::}}\sphinxbfcode{\sphinxupquote{fef\_so2\_io}}}
\pysigstopsignatures\begin{quote}\begin{description}
\sphinxlineitem{Type}
\sphinxAtStartPar
real(npft)

\sphinxlineitem{Default}
\sphinxAtStartPar
None

\end{description}\end{quote}

\sphinxAtStartPar
Fire SO$_{\text{2}}$ Emission Factor (g kg$^{\text{\sphinxhyphen{}1}}$).

\end{fulllineitems}

\index{fef\_oc\_io (in namelist JULES\_PFTPARM)@\spxentry{fef\_oc\_io}\spxextra{in namelist JULES\_PFTPARM}|spxpagem}

\begin{fulllineitems}
\phantomsection\label{\detokenize{namelists/pft_params.nml:JULES_PFTPARM::fef_oc_io}}
\pysigstartsignatures
\pysigline{\sphinxcode{\sphinxupquote{JULES\_PFTPARM::}}\sphinxbfcode{\sphinxupquote{fef\_oc\_io}}}
\pysigstopsignatures\begin{quote}\begin{description}
\sphinxlineitem{Type}
\sphinxAtStartPar
real(npft)

\sphinxlineitem{Default}
\sphinxAtStartPar
None

\end{description}\end{quote}

\sphinxAtStartPar
Fire OC Emission Factor (g kg$^{\text{\sphinxhyphen{}1}}$).

\end{fulllineitems}

\index{fef\_bc\_io (in namelist JULES\_PFTPARM)@\spxentry{fef\_bc\_io}\spxextra{in namelist JULES\_PFTPARM}|spxpagem}

\begin{fulllineitems}
\phantomsection\label{\detokenize{namelists/pft_params.nml:JULES_PFTPARM::fef_bc_io}}
\pysigstartsignatures
\pysigline{\sphinxcode{\sphinxupquote{JULES\_PFTPARM::}}\sphinxbfcode{\sphinxupquote{fef\_bc\_io}}}
\pysigstopsignatures\begin{quote}\begin{description}
\sphinxlineitem{Type}
\sphinxAtStartPar
real(npft)

\sphinxlineitem{Default}
\sphinxAtStartPar
None

\end{description}\end{quote}

\sphinxAtStartPar
Fire BC Emission Factor (g kg$^{\text{\sphinxhyphen{}1}}$).

\end{fulllineitems}

\index{ccleaf\_min\_io (in namelist JULES\_PFTPARM)@\spxentry{ccleaf\_min\_io}\spxextra{in namelist JULES\_PFTPARM}|spxpagem}

\begin{fulllineitems}
\phantomsection\label{\detokenize{namelists/pft_params.nml:JULES_PFTPARM::ccleaf_min_io}}
\pysigstartsignatures
\pysigline{\sphinxcode{\sphinxupquote{JULES\_PFTPARM::}}\sphinxbfcode{\sphinxupquote{ccleaf\_min\_io}}}
\pysigstopsignatures\begin{quote}\begin{description}
\sphinxlineitem{Type}
\sphinxAtStartPar
real(npft)

\sphinxlineitem{Default}
\sphinxAtStartPar
None

\end{description}\end{quote}

\sphinxAtStartPar
Leaf minimum combustion completeness.

\end{fulllineitems}

\index{ccleaf\_max\_io (in namelist JULES\_PFTPARM)@\spxentry{ccleaf\_max\_io}\spxextra{in namelist JULES\_PFTPARM}|spxpagem}

\begin{fulllineitems}
\phantomsection\label{\detokenize{namelists/pft_params.nml:JULES_PFTPARM::ccleaf_max_io}}
\pysigstartsignatures
\pysigline{\sphinxcode{\sphinxupquote{JULES\_PFTPARM::}}\sphinxbfcode{\sphinxupquote{ccleaf\_max\_io}}}
\pysigstopsignatures\begin{quote}\begin{description}
\sphinxlineitem{Type}
\sphinxAtStartPar
real(npft)

\sphinxlineitem{Default}
\sphinxAtStartPar
None

\end{description}\end{quote}

\sphinxAtStartPar
Leaf maximum combustion completeness.

\end{fulllineitems}

\index{ccwood\_min\_io (in namelist JULES\_PFTPARM)@\spxentry{ccwood\_min\_io}\spxextra{in namelist JULES\_PFTPARM}|spxpagem}

\begin{fulllineitems}
\phantomsection\label{\detokenize{namelists/pft_params.nml:JULES_PFTPARM::ccwood_min_io}}
\pysigstartsignatures
\pysigline{\sphinxcode{\sphinxupquote{JULES\_PFTPARM::}}\sphinxbfcode{\sphinxupquote{ccwood\_min\_io}}}
\pysigstopsignatures\begin{quote}\begin{description}
\sphinxlineitem{Type}
\sphinxAtStartPar
real(npft)

\sphinxlineitem{Default}
\sphinxAtStartPar
None

\end{description}\end{quote}

\sphinxAtStartPar
Wood minimum combustion completeness.

\end{fulllineitems}

\index{ccwood\_max\_io (in namelist JULES\_PFTPARM)@\spxentry{ccwood\_max\_io}\spxextra{in namelist JULES\_PFTPARM}|spxpagem}

\begin{fulllineitems}
\phantomsection\label{\detokenize{namelists/pft_params.nml:JULES_PFTPARM::ccwood_max_io}}
\pysigstartsignatures
\pysigline{\sphinxcode{\sphinxupquote{JULES\_PFTPARM::}}\sphinxbfcode{\sphinxupquote{ccwood\_max\_io}}}
\pysigstopsignatures\begin{quote}\begin{description}
\sphinxlineitem{Type}
\sphinxAtStartPar
real(npft)

\sphinxlineitem{Default}
\sphinxAtStartPar
None

\end{description}\end{quote}

\sphinxAtStartPar
Wood maximum combustion completeness.

\end{fulllineitems}

\index{avg\_ba\_io (in namelist JULES\_PFTPARM)@\spxentry{avg\_ba\_io}\spxextra{in namelist JULES\_PFTPARM}|spxpagem}

\begin{fulllineitems}
\phantomsection\label{\detokenize{namelists/pft_params.nml:JULES_PFTPARM::avg_ba_io}}
\pysigstartsignatures
\pysigline{\sphinxcode{\sphinxupquote{JULES\_PFTPARM::}}\sphinxbfcode{\sphinxupquote{avg\_ba\_io}}}
\pysigstopsignatures\begin{quote}\begin{description}
\sphinxlineitem{Type}
\sphinxAtStartPar
real(npft)

\sphinxlineitem{Default}
\sphinxAtStartPar
None

\end{description}\end{quote}

\sphinxAtStartPar
Average PFT Burnt Area per fire (m$^{\text{2}}$).

\end{fulllineitems}

\index{fire\_mort\_io (in namelist JULES\_PFTPARM)@\spxentry{fire\_mort\_io}\spxextra{in namelist JULES\_PFTPARM}|spxpagem}

\begin{fulllineitems}
\phantomsection\label{\detokenize{namelists/pft_params.nml:JULES_PFTPARM::fire_mort_io}}
\pysigstartsignatures
\pysigline{\sphinxcode{\sphinxupquote{JULES\_PFTPARM::}}\sphinxbfcode{\sphinxupquote{fire\_mort\_io}}}
\pysigstopsignatures\begin{quote}\begin{description}
\sphinxlineitem{Type}
\sphinxAtStartPar
real(npft)

\sphinxlineitem{Default}
\sphinxAtStartPar
None

\end{description}\end{quote}

\sphinxAtStartPar
Scaling factor for vegetation mortality caused by fire (from INFERNO burned area). Can be varied between 0.0 (no morality) and 1.0 (100\% mortality) for each PFT.


\sphinxstrong{See also:}
\nopagebreak


\sphinxAtStartPar
References:
\begin{itemize}
\item {} 
\sphinxAtStartPar
Clark et al., 2011, The Joint UK Land Environment Simulator
(JULES), model description \textendash{} Part 2: Carbon fluxes and
vegetation dynamics, Geosci. Model Dev., 4, 701\sphinxhyphen{}722,
\sphinxurl{https://doi.org/10.5194/gmd-4-701-2011}

\item {} 
\sphinxAtStartPar
Pinty, B., T. Lavergne, R. E. Dickinson,
J.\sphinxhyphen{}L. Widlowski, N. Gobron, and M. M. Verstraete (2006),
Simplifying the interaction of land surfaces with radiation
for relating remote sensing products to climate
models, J. Geophys. Res., 111, D02116,
\sphinxurl{https://doi.org/10.1029/2005JD005952}.

\end{itemize}



\end{fulllineitems}


\begin{sphinxadmonition}{note}{Only used with the Farquhar model of leaf photosynthesis (\sphinxstyleliteralintitle{\sphinxupquote{photo\_model}} = 2). A value is required for each PFT, but only those for C$_{\text{3}}$ plants are used (since only C$_{\text{3}}$ plants use the Farquhar model). Below, J$_{\text{max}}$ is the potential rate of electron transport, and V$_{\text{cmax}}$ is the maximum rate of carboxylation of Rubisco.}
\index{act\_jmax\_io (in namelist JULES\_PFTPARM)@\spxentry{act\_jmax\_io}\spxextra{in namelist JULES\_PFTPARM}|spxpagem}

\begin{fulllineitems}
\phantomsection\label{\detokenize{namelists/pft_params.nml:JULES_PFTPARM::act_jmax_io}}
\pysigstartsignatures
\pysigline{\sphinxcode{\sphinxupquote{JULES\_PFTPARM::}}\sphinxbfcode{\sphinxupquote{act\_jmax\_io}}}
\pysigstopsignatures\begin{quote}\begin{description}
\sphinxlineitem{Type}
\sphinxAtStartPar
real(npft)

\sphinxlineitem{Default}
\sphinxAtStartPar
None

\end{description}\end{quote}

\sphinxAtStartPar
Activation energy for temperature response of J$_{\text{max}}$ (J mol$^{\text{\sphinxhyphen{}1}}$).

\end{fulllineitems}

\index{act\_vcmax\_io (in namelist JULES\_PFTPARM)@\spxentry{act\_vcmax\_io}\spxextra{in namelist JULES\_PFTPARM}|spxpagem}

\begin{fulllineitems}
\phantomsection\label{\detokenize{namelists/pft_params.nml:JULES_PFTPARM::act_vcmax_io}}
\pysigstartsignatures
\pysigline{\sphinxcode{\sphinxupquote{JULES\_PFTPARM::}}\sphinxbfcode{\sphinxupquote{act\_vcmax\_io}}}
\pysigstopsignatures\begin{quote}\begin{description}
\sphinxlineitem{Type}
\sphinxAtStartPar
real(npft)

\sphinxlineitem{Default}
\sphinxAtStartPar
None

\end{description}\end{quote}

\sphinxAtStartPar
Activation energy for temperature response of V$_{\text{cmax}}$ (J mol$^{\text{\sphinxhyphen{}1}}$).

\begin{sphinxadmonition}{note}{Note:}
\sphinxAtStartPar
{\hyperref[\detokenize{namelists/pft_params.nml:JULES_PFTPARM::act_jmax_io}]{\sphinxcrossref{\sphinxcode{\sphinxupquote{act\_jmax\_io}}}}} and {\hyperref[\detokenize{namelists/pft_params.nml:JULES_PFTPARM::act_vcmax_io}]{\sphinxcrossref{\sphinxcode{\sphinxupquote{act\_vcmax\_io}}}}} are NOT required if thermal adaptation or acclimation of photosynthesis is selected ({\hyperref[\detokenize{namelists/jules_vegetation.nml:JULES_VEGETATION::photo_acclim_model}]{\sphinxcrossref{\sphinxcode{\sphinxupquote{photo\_acclim\_model}}}}} = 1, 2 or 3) together with {\hyperref[\detokenize{namelists/jules_vegetation.nml:JULES_VEGETATION::photo_act_model}]{\sphinxcrossref{\sphinxcode{\sphinxupquote{photo\_act\_model}}}}} = 2.
\end{sphinxadmonition}

\end{fulllineitems}

\index{alpha\_elec\_io (in namelist JULES\_PFTPARM)@\spxentry{alpha\_elec\_io}\spxextra{in namelist JULES\_PFTPARM}|spxpagem}

\begin{fulllineitems}
\phantomsection\label{\detokenize{namelists/pft_params.nml:JULES_PFTPARM::alpha_elec_io}}
\pysigstartsignatures
\pysigline{\sphinxcode{\sphinxupquote{JULES\_PFTPARM::}}\sphinxbfcode{\sphinxupquote{alpha\_elec\_io}}}
\pysigstopsignatures\begin{quote}\begin{description}
\sphinxlineitem{Type}
\sphinxAtStartPar
real(npft)

\sphinxlineitem{Default}
\sphinxAtStartPar
None

\end{description}\end{quote}

\sphinxAtStartPar
Quantum yield of electron transport (mol electrons {[}mol$^{\text{\sphinxhyphen{}1}}$ PAR photons{]}).

\end{fulllineitems}

\index{deact\_jmax\_io (in namelist JULES\_PFTPARM)@\spxentry{deact\_jmax\_io}\spxextra{in namelist JULES\_PFTPARM}|spxpagem}

\begin{fulllineitems}
\phantomsection\label{\detokenize{namelists/pft_params.nml:JULES_PFTPARM::deact_jmax_io}}
\pysigstartsignatures
\pysigline{\sphinxcode{\sphinxupquote{JULES\_PFTPARM::}}\sphinxbfcode{\sphinxupquote{deact\_jmax\_io}}}
\pysigstopsignatures\begin{quote}\begin{description}
\sphinxlineitem{Type}
\sphinxAtStartPar
real(npft)

\sphinxlineitem{Default}
\sphinxAtStartPar
None

\end{description}\end{quote}

\sphinxAtStartPar
Deactivation energy for temperature response of J$_{\text{max}}$ (J mol$^{\text{\sphinxhyphen{}1}}$). This describes the rate of decrease above the optimum temperature.

\end{fulllineitems}

\index{deact\_vcmax\_io (in namelist JULES\_PFTPARM)@\spxentry{deact\_vcmax\_io}\spxextra{in namelist JULES\_PFTPARM}|spxpagem}

\begin{fulllineitems}
\phantomsection\label{\detokenize{namelists/pft_params.nml:JULES_PFTPARM::deact_vcmax_io}}
\pysigstartsignatures
\pysigline{\sphinxcode{\sphinxupquote{JULES\_PFTPARM::}}\sphinxbfcode{\sphinxupquote{deact\_vcmax\_io}}}
\pysigstopsignatures\begin{quote}\begin{description}
\sphinxlineitem{Type}
\sphinxAtStartPar
real(npft)

\sphinxlineitem{Default}
\sphinxAtStartPar
None

\end{description}\end{quote}

\sphinxAtStartPar
Deactivation energy for temperature response of V$_{\text{cmax}}$ (J mol$^{\text{\sphinxhyphen{}1}}$). This describes the rate of decrease above the optimum temperature.

\end{fulllineitems}

\index{jv25\_ratio\_io (in namelist JULES\_PFTPARM)@\spxentry{jv25\_ratio\_io}\spxextra{in namelist JULES\_PFTPARM}|spxpagem}

\begin{fulllineitems}
\phantomsection\label{\detokenize{namelists/pft_params.nml:JULES_PFTPARM::jv25_ratio_io}}
\pysigstartsignatures
\pysigline{\sphinxcode{\sphinxupquote{JULES\_PFTPARM::}}\sphinxbfcode{\sphinxupquote{jv25\_ratio\_io}}}
\pysigstopsignatures\begin{quote}\begin{description}
\sphinxlineitem{Type}
\sphinxAtStartPar
real(npft)

\sphinxlineitem{Default}
\sphinxAtStartPar
None

\end{description}\end{quote}

\sphinxAtStartPar
Ratio of J$_{\text{max}}$ to V$_{\text{cmax}}$ at 25 deg C (mol electrons {[}mol$^{\text{\sphinxhyphen{}1}}$ CO$_{\text{2}}${]}).

\begin{sphinxadmonition}{note}{Note:}
\sphinxAtStartPar
If thermal adaptation or acclimation of photosynthesis is selected ({\hyperref[\detokenize{namelists/jules_vegetation.nml:JULES_VEGETATION::photo_acclim_model}]{\sphinxcrossref{\sphinxcode{\sphinxupquote{photo\_acclim\_model}}}}} = 1 or 2) together with {\hyperref[\detokenize{namelists/jules_vegetation.nml:JULES_VEGETATION::photo_jv_model}]{\sphinxcrossref{\sphinxcode{\sphinxupquote{photo\_jv\_model}}}}} =2 (J$_{\text{max}}$/V$_{\text{cmax}}$ calculated assuming constant total nitrogen allocation)), this value is used along with parameters {\hyperref[\detokenize{namelists/jules_vegetation.nml:JULES_VEGETATION::n_alloc_jmax}]{\sphinxcrossref{\sphinxcode{\sphinxupquote{n\_alloc\_jmax}}}}} and {\hyperref[\detokenize{namelists/jules_vegetation.nml:JULES_VEGETATION::n_alloc_vcmax}]{\sphinxcrossref{\sphinxcode{\sphinxupquote{n\_alloc\_vcmax}}}}} to calculate the final value of J$_{\text{max}}$/V$_{\text{cmax}}$.
\end{sphinxadmonition}

\end{fulllineitems}

\end{sphinxadmonition}

\begin{sphinxadmonition}{note}{Only used if thermal adaptation or acclimation of photosynthetic capacity is NOT modelled (\sphinxstyleliteralintitle{\sphinxupquote{photo\_acclim\_model}} = 0). A value is required for each PFT, but only those for C$_{\text{3}}$ plants are used (since only C$_{\text{3}}$ plants use the Farquhar model).}
\index{ds\_jmax\_io (in namelist JULES\_PFTPARM)@\spxentry{ds\_jmax\_io}\spxextra{in namelist JULES\_PFTPARM}|spxpagem}

\begin{fulllineitems}
\phantomsection\label{\detokenize{namelists/pft_params.nml:JULES_PFTPARM::ds_jmax_io}}
\pysigstartsignatures
\pysigline{\sphinxcode{\sphinxupquote{JULES\_PFTPARM::}}\sphinxbfcode{\sphinxupquote{ds\_jmax\_io}}}
\pysigstopsignatures\begin{quote}\begin{description}
\sphinxlineitem{Type}
\sphinxAtStartPar
real(npft)

\sphinxlineitem{Default}
\sphinxAtStartPar
None

\end{description}\end{quote}

\sphinxAtStartPar
Entropy factor for temperature reponse of  J$_{\text{max}}$ (J mol$^{\text{\sphinxhyphen{}1}}$ K$^{\text{\sphinxhyphen{}1}}$).

\end{fulllineitems}

\index{ds\_vcmax\_io (in namelist JULES\_PFTPARM)@\spxentry{ds\_vcmax\_io}\spxextra{in namelist JULES\_PFTPARM}|spxpagem}

\begin{fulllineitems}
\phantomsection\label{\detokenize{namelists/pft_params.nml:JULES_PFTPARM::ds_vcmax_io}}
\pysigstartsignatures
\pysigline{\sphinxcode{\sphinxupquote{JULES\_PFTPARM::}}\sphinxbfcode{\sphinxupquote{ds\_vcmax\_io}}}
\pysigstopsignatures\begin{quote}\begin{description}
\sphinxlineitem{Type}
\sphinxAtStartPar
real(npft)

\sphinxlineitem{Default}
\sphinxAtStartPar
None

\end{description}\end{quote}

\sphinxAtStartPar
Entropy factor for temperature reponse of  V$_{\text{cmax}}$ (J mol$^{\text{\sphinxhyphen{}1}}$ K$^{\text{\sphinxhyphen{}1}}$).

\end{fulllineitems}

\end{sphinxadmonition}

\sphinxstepscope


\section{\sphinxstyleliteralintitle{\sphinxupquote{cable\_pftparm.nml}}}
\label{\detokenize{namelists/cable_pftparm.nml:cable-pftparm-nml}}\label{\detokenize{namelists/cable_pftparm.nml::doc}}
\sphinxAtStartPar
This file sets the time and space\sphinxhyphen{}invariant parameters for plant functional types for the CABLE land surface model. It contains one namelist called {\hyperref[\detokenize{namelists/cable_pftparm.nml:namelist-CABLE_PFTPARM}]{\sphinxcrossref{\sphinxcode{\sphinxupquote{CABLE\_PFTPARM}}}}}.


\subsection{\sphinxstyleliteralintitle{\sphinxupquote{CABLE\_PFTPARM}} namelist members}
\label{\detokenize{namelists/cable_pftparm.nml:namelist-CABLE_PFTPARM}}\label{\detokenize{namelists/cable_pftparm.nml:cable-pftparm-namelist-members}}\index{CABLE\_PFTPARM (namelist)@\spxentry{CABLE\_PFTPARM}\spxextra{namelist}|spxpagem}
\sphinxAtStartPar
This namelist reads the values of parameters for each of the plant functional types (PFTs) if the CABLE land surface model is being used. These parameters are a function of PFT only. Every member must be given a value for every run.
CABLE uses the same parameters for veg and non\sphinxhyphen{}veg surface types, unlike JULES, and therefore its arrays are of dimension (npft + nnvg).
\index{a1gs\_io (in namelist CABLE\_PFTPARM)@\spxentry{a1gs\_io}\spxextra{in namelist CABLE\_PFTPARM}|spxpagem}

\begin{fulllineitems}
\phantomsection\label{\detokenize{namelists/cable_pftparm.nml:CABLE_PFTPARM::a1gs_io}}
\pysigstartsignatures
\pysigline{\sphinxcode{\sphinxupquote{CABLE\_PFTPARM::}}\sphinxbfcode{\sphinxupquote{a1gs\_io}}}
\pysigstopsignatures\begin{quote}\begin{description}
\sphinxlineitem{Type}
\sphinxAtStartPar
real(npft + nnvg)

\sphinxlineitem{Default}
\sphinxAtStartPar
MDI

\end{description}\end{quote}

\sphinxAtStartPar
Represents the sensitivity of stomatal conductance to the assimilation rate (unitless).

\end{fulllineitems}

\index{alpha\_io (in namelist CABLE\_PFTPARM)@\spxentry{alpha\_io}\spxextra{in namelist CABLE\_PFTPARM}|spxpagem}

\begin{fulllineitems}
\phantomsection\label{\detokenize{namelists/cable_pftparm.nml:CABLE_PFTPARM::alpha_io}}
\pysigstartsignatures
\pysigline{\sphinxcode{\sphinxupquote{CABLE\_PFTPARM::}}\sphinxbfcode{\sphinxupquote{alpha\_io}}}
\pysigstopsignatures\begin{quote}\begin{description}
\sphinxlineitem{Type}
\sphinxAtStartPar
real(npft + nnvg)

\sphinxlineitem{Default}
\sphinxAtStartPar
MDI

\end{description}\end{quote}

\sphinxAtStartPar
Initial slope of J\sphinxhyphen{}Q response curve. Units: mol (electrons) mol$^{\text{\sphinxhyphen{}1}}$ (photons) (C3)
mol (CO$_{\text{2}}$) mol$^{\text{\sphinxhyphen{}1}}$ (photons) (C4)

\end{fulllineitems}

\index{canst1\_io (in namelist CABLE\_PFTPARM)@\spxentry{canst1\_io}\spxextra{in namelist CABLE\_PFTPARM}|spxpagem}

\begin{fulllineitems}
\phantomsection\label{\detokenize{namelists/cable_pftparm.nml:CABLE_PFTPARM::canst1_io}}
\pysigstartsignatures
\pysigline{\sphinxcode{\sphinxupquote{CABLE\_PFTPARM::}}\sphinxbfcode{\sphinxupquote{canst1\_io}}}
\pysigstopsignatures\begin{quote}\begin{description}
\sphinxlineitem{Type}
\sphinxAtStartPar
real(npft + nnvg)

\sphinxlineitem{Default}
\sphinxAtStartPar
MDI

\end{description}\end{quote}

\sphinxAtStartPar
Maximum intercepted water by canopy. (mm LAI$^{\text{\sphinxhyphen{}1}}$)

\end{fulllineitems}

\index{cfrd\_io (in namelist CABLE\_PFTPARM)@\spxentry{cfrd\_io}\spxextra{in namelist CABLE\_PFTPARM}|spxpagem}

\begin{fulllineitems}
\phantomsection\label{\detokenize{namelists/cable_pftparm.nml:CABLE_PFTPARM::cfrd_io}}
\pysigstartsignatures
\pysigline{\sphinxcode{\sphinxupquote{CABLE\_PFTPARM::}}\sphinxbfcode{\sphinxupquote{cfrd\_io}}}
\pysigstopsignatures\begin{quote}\begin{description}
\sphinxlineitem{Type}
\sphinxAtStartPar
real(npft + nnvg)

\sphinxlineitem{Default}
\sphinxAtStartPar
MDI

\end{description}\end{quote}

\sphinxAtStartPar
Ratio of day respiration to vcmax

\end{fulllineitems}

\index{clitt\_io (in namelist CABLE\_PFTPARM)@\spxentry{clitt\_io}\spxextra{in namelist CABLE\_PFTPARM}|spxpagem}

\begin{fulllineitems}
\phantomsection\label{\detokenize{namelists/cable_pftparm.nml:CABLE_PFTPARM::clitt_io}}
\pysigstartsignatures
\pysigline{\sphinxcode{\sphinxupquote{CABLE\_PFTPARM::}}\sphinxbfcode{\sphinxupquote{clitt\_io}}}
\pysigstopsignatures\begin{quote}\begin{description}
\sphinxlineitem{Type}
\sphinxAtStartPar
real(npft + nnvg)

\sphinxlineitem{Default}
\sphinxAtStartPar
MDI

\end{description}\end{quote}

\sphinxAtStartPar
Leaf litter (alters resistance to soil evaporation) (tC ha$^{\text{\sphinxhyphen{}1}}$)

\end{fulllineitems}

\index{conkc0\_io (in namelist CABLE\_PFTPARM)@\spxentry{conkc0\_io}\spxextra{in namelist CABLE\_PFTPARM}|spxpagem}

\begin{fulllineitems}
\phantomsection\label{\detokenize{namelists/cable_pftparm.nml:CABLE_PFTPARM::conkc0_io}}
\pysigstartsignatures
\pysigline{\sphinxcode{\sphinxupquote{CABLE\_PFTPARM::}}\sphinxbfcode{\sphinxupquote{conkc0\_io}}}
\pysigstopsignatures\begin{quote}\begin{description}
\sphinxlineitem{Type}
\sphinxAtStartPar
real(npft + nnvg)

\sphinxlineitem{Default}
\sphinxAtStartPar
MDI

\end{description}\end{quote}

\sphinxAtStartPar
Michaelis\sphinxhyphen{}menton constant for carboxylase (bar)

\end{fulllineitems}

\index{conko0\_io (in namelist CABLE\_PFTPARM)@\spxentry{conko0\_io}\spxextra{in namelist CABLE\_PFTPARM}|spxpagem}

\begin{fulllineitems}
\phantomsection\label{\detokenize{namelists/cable_pftparm.nml:CABLE_PFTPARM::conko0_io}}
\pysigstartsignatures
\pysigline{\sphinxcode{\sphinxupquote{CABLE\_PFTPARM::}}\sphinxbfcode{\sphinxupquote{conko0\_io}}}
\pysigstopsignatures\begin{quote}\begin{description}
\sphinxlineitem{Type}
\sphinxAtStartPar
real(npft + nnvg)

\sphinxlineitem{Default}
\sphinxAtStartPar
MDI

\end{description}\end{quote}

\sphinxAtStartPar
Michaelis\sphinxhyphen{}menton constant for oxygenase (bar)

\end{fulllineitems}

\index{convex\_io (in namelist CABLE\_PFTPARM)@\spxentry{convex\_io}\spxextra{in namelist CABLE\_PFTPARM}|spxpagem}

\begin{fulllineitems}
\phantomsection\label{\detokenize{namelists/cable_pftparm.nml:CABLE_PFTPARM::convex_io}}
\pysigstartsignatures
\pysigline{\sphinxcode{\sphinxupquote{CABLE\_PFTPARM::}}\sphinxbfcode{\sphinxupquote{convex\_io}}}
\pysigstopsignatures\begin{quote}\begin{description}
\sphinxlineitem{Type}
\sphinxAtStartPar
real(npft + nnvg)

\sphinxlineitem{Default}
\sphinxAtStartPar
MDI

\end{description}\end{quote}

\sphinxAtStartPar
Convexity of J\sphinxhyphen{}Q response curve (unitless).

\end{fulllineitems}

\index{cplant1\_io (in namelist CABLE\_PFTPARM)@\spxentry{cplant1\_io}\spxextra{in namelist CABLE\_PFTPARM}|spxpagem}

\begin{fulllineitems}
\phantomsection\label{\detokenize{namelists/cable_pftparm.nml:CABLE_PFTPARM::cplant1_io}}
\pysigstartsignatures
\pysigline{\sphinxcode{\sphinxupquote{CABLE\_PFTPARM::}}\sphinxbfcode{\sphinxupquote{cplant1\_io}}}
\pysigstopsignatures\begin{quote}\begin{description}
\sphinxlineitem{Type}
\sphinxAtStartPar
real(npft + nnvg)

\sphinxlineitem{Default}
\sphinxAtStartPar
MDI

\end{description}\end{quote}

\sphinxAtStartPar
Plant carbon in 1st vegetation carbon store (g C m$^{\text{\sphinxhyphen{}2}}$)

\end{fulllineitems}

\index{cplant2\_io (in namelist CABLE\_PFTPARM)@\spxentry{cplant2\_io}\spxextra{in namelist CABLE\_PFTPARM}|spxpagem}

\begin{fulllineitems}
\phantomsection\label{\detokenize{namelists/cable_pftparm.nml:CABLE_PFTPARM::cplant2_io}}
\pysigstartsignatures
\pysigline{\sphinxcode{\sphinxupquote{CABLE\_PFTPARM::}}\sphinxbfcode{\sphinxupquote{cplant2\_io}}}
\pysigstopsignatures\begin{quote}\begin{description}
\sphinxlineitem{Type}
\sphinxAtStartPar
real(npft + nnvg)

\sphinxlineitem{Default}
\sphinxAtStartPar
MDI

\end{description}\end{quote}

\sphinxAtStartPar
Plant carbon in 2nd vegetation carbon store (g C m$^{\text{\sphinxhyphen{}2}}$)

\end{fulllineitems}

\index{cplant3\_io (in namelist CABLE\_PFTPARM)@\spxentry{cplant3\_io}\spxextra{in namelist CABLE\_PFTPARM}|spxpagem}

\begin{fulllineitems}
\phantomsection\label{\detokenize{namelists/cable_pftparm.nml:CABLE_PFTPARM::cplant3_io}}
\pysigstartsignatures
\pysigline{\sphinxcode{\sphinxupquote{CABLE\_PFTPARM::}}\sphinxbfcode{\sphinxupquote{cplant3\_io}}}
\pysigstopsignatures\begin{quote}\begin{description}
\sphinxlineitem{Type}
\sphinxAtStartPar
real(npft + nnvg)

\sphinxlineitem{Default}
\sphinxAtStartPar
MDI

\end{description}\end{quote}

\sphinxAtStartPar
Plant carbon in 3rd vegetation carbon store (g C m$^{\text{\sphinxhyphen{}2}}$)

\end{fulllineitems}

\index{csoil1\_io (in namelist CABLE\_PFTPARM)@\spxentry{csoil1\_io}\spxextra{in namelist CABLE\_PFTPARM}|spxpagem}

\begin{fulllineitems}
\phantomsection\label{\detokenize{namelists/cable_pftparm.nml:CABLE_PFTPARM::csoil1_io}}
\pysigstartsignatures
\pysigline{\sphinxcode{\sphinxupquote{CABLE\_PFTPARM::}}\sphinxbfcode{\sphinxupquote{csoil1\_io}}}
\pysigstopsignatures\begin{quote}\begin{description}
\sphinxlineitem{Type}
\sphinxAtStartPar
real(npft + nnvg)

\sphinxlineitem{Default}
\sphinxAtStartPar
MDI

\end{description}\end{quote}

\sphinxAtStartPar
Soil carbon in 1st soil carbon store (g C m$^{\text{\sphinxhyphen{}2}}$)

\end{fulllineitems}

\index{csoil2\_io (in namelist CABLE\_PFTPARM)@\spxentry{csoil2\_io}\spxextra{in namelist CABLE\_PFTPARM}|spxpagem}

\begin{fulllineitems}
\phantomsection\label{\detokenize{namelists/cable_pftparm.nml:CABLE_PFTPARM::csoil2_io}}
\pysigstartsignatures
\pysigline{\sphinxcode{\sphinxupquote{CABLE\_PFTPARM::}}\sphinxbfcode{\sphinxupquote{csoil2\_io}}}
\pysigstopsignatures\begin{quote}\begin{description}
\sphinxlineitem{Type}
\sphinxAtStartPar
real(npft + nnvg)

\sphinxlineitem{Default}
\sphinxAtStartPar
MDI

\end{description}\end{quote}

\sphinxAtStartPar
Soil carbon in 2nd soil carbon store (g C m$^{\text{\sphinxhyphen{}2}}$)

\end{fulllineitems}

\index{d0gs\_io (in namelist CABLE\_PFTPARM)@\spxentry{d0gs\_io}\spxextra{in namelist CABLE\_PFTPARM}|spxpagem}

\begin{fulllineitems}
\phantomsection\label{\detokenize{namelists/cable_pftparm.nml:CABLE_PFTPARM::d0gs_io}}
\pysigstartsignatures
\pysigline{\sphinxcode{\sphinxupquote{CABLE\_PFTPARM::}}\sphinxbfcode{\sphinxupquote{d0gs\_io}}}
\pysigstopsignatures\begin{quote}\begin{description}
\sphinxlineitem{Type}
\sphinxAtStartPar
real(npft + nnvg)

\sphinxlineitem{Default}
\sphinxAtStartPar
MDI

\end{description}\end{quote}

\sphinxAtStartPar
d0 in stomatal conductance model (kPa)

\end{fulllineitems}

\index{ejmax\_io (in namelist CABLE\_PFTPARM)@\spxentry{ejmax\_io}\spxextra{in namelist CABLE\_PFTPARM}|spxpagem}

\begin{fulllineitems}
\phantomsection\label{\detokenize{namelists/cable_pftparm.nml:CABLE_PFTPARM::ejmax_io}}
\pysigstartsignatures
\pysigline{\sphinxcode{\sphinxupquote{CABLE\_PFTPARM::}}\sphinxbfcode{\sphinxupquote{ejmax\_io}}}
\pysigstopsignatures\begin{quote}\begin{description}
\sphinxlineitem{Type}
\sphinxAtStartPar
real(npft + nnvg)

\sphinxlineitem{Default}
\sphinxAtStartPar
MDI

\end{description}\end{quote}

\sphinxAtStartPar
Maximum potential electron transport rate top leaf, currently double the assigned value of vcmax. (mol m$^{\text{\sphinxhyphen{}2}}$ s$^{\text{\sphinxhyphen{}1}}$)

\end{fulllineitems}

\index{ekc\_io (in namelist CABLE\_PFTPARM)@\spxentry{ekc\_io}\spxextra{in namelist CABLE\_PFTPARM}|spxpagem}

\begin{fulllineitems}
\phantomsection\label{\detokenize{namelists/cable_pftparm.nml:CABLE_PFTPARM::ekc_io}}
\pysigstartsignatures
\pysigline{\sphinxcode{\sphinxupquote{CABLE\_PFTPARM::}}\sphinxbfcode{\sphinxupquote{ekc\_io}}}
\pysigstopsignatures\begin{quote}\begin{description}
\sphinxlineitem{Type}
\sphinxAtStartPar
real(npft + nnvg)

\sphinxlineitem{Default}
\sphinxAtStartPar
MDI

\end{description}\end{quote}

\sphinxAtStartPar
Activation energy for carboxylase (J mol$^{\text{\sphinxhyphen{}1}}$)

\end{fulllineitems}

\index{eko\_io (in namelist CABLE\_PFTPARM)@\spxentry{eko\_io}\spxextra{in namelist CABLE\_PFTPARM}|spxpagem}

\begin{fulllineitems}
\phantomsection\label{\detokenize{namelists/cable_pftparm.nml:CABLE_PFTPARM::eko_io}}
\pysigstartsignatures
\pysigline{\sphinxcode{\sphinxupquote{CABLE\_PFTPARM::}}\sphinxbfcode{\sphinxupquote{eko\_io}}}
\pysigstopsignatures\begin{quote}\begin{description}
\sphinxlineitem{Type}
\sphinxAtStartPar
real(npft + nnvg)

\sphinxlineitem{Default}
\sphinxAtStartPar
MDI

\end{description}\end{quote}

\sphinxAtStartPar
Activation energy for oxygenase (J mol$^{\text{\sphinxhyphen{}1}}$)

\end{fulllineitems}

\index{extkn\_io (in namelist CABLE\_PFTPARM)@\spxentry{extkn\_io}\spxextra{in namelist CABLE\_PFTPARM}|spxpagem}

\begin{fulllineitems}
\phantomsection\label{\detokenize{namelists/cable_pftparm.nml:CABLE_PFTPARM::extkn_io}}
\pysigstartsignatures
\pysigline{\sphinxcode{\sphinxupquote{CABLE\_PFTPARM::}}\sphinxbfcode{\sphinxupquote{extkn\_io}}}
\pysigstopsignatures\begin{quote}\begin{description}
\sphinxlineitem{Type}
\sphinxAtStartPar
real(npft + nnvg)

\sphinxlineitem{Default}
\sphinxAtStartPar
MDI

\end{description}\end{quote}

\sphinxAtStartPar
Extinction coefficient for vertical profile of N.

\end{fulllineitems}

\index{frac4\_io (in namelist CABLE\_PFTPARM)@\spxentry{frac4\_io}\spxextra{in namelist CABLE\_PFTPARM}|spxpagem}

\begin{fulllineitems}
\phantomsection\label{\detokenize{namelists/cable_pftparm.nml:CABLE_PFTPARM::frac4_io}}
\pysigstartsignatures
\pysigline{\sphinxcode{\sphinxupquote{CABLE\_PFTPARM::}}\sphinxbfcode{\sphinxupquote{frac4\_io}}}
\pysigstopsignatures\begin{quote}\begin{description}
\sphinxlineitem{Type}
\sphinxAtStartPar
real(npft + nnvg)

\sphinxlineitem{Default}
\sphinxAtStartPar
MDI

\end{description}\end{quote}

\sphinxAtStartPar
Fraction of c4 plants

\end{fulllineitems}

\index{froot1\_io (in namelist CABLE\_PFTPARM)@\spxentry{froot1\_io}\spxextra{in namelist CABLE\_PFTPARM}|spxpagem}

\begin{fulllineitems}
\phantomsection\label{\detokenize{namelists/cable_pftparm.nml:CABLE_PFTPARM::froot1_io}}
\pysigstartsignatures
\pysigline{\sphinxcode{\sphinxupquote{CABLE\_PFTPARM::}}\sphinxbfcode{\sphinxupquote{froot1\_io}}}
\pysigstopsignatures\begin{quote}\begin{description}
\sphinxlineitem{Type}
\sphinxAtStartPar
real(npft + nnvg)

\sphinxlineitem{Default}
\sphinxAtStartPar
MDI

\end{description}\end{quote}

\sphinxAtStartPar
Fraction of root in 1st soil layer.

\end{fulllineitems}

\index{froot2\_io (in namelist CABLE\_PFTPARM)@\spxentry{froot2\_io}\spxextra{in namelist CABLE\_PFTPARM}|spxpagem}

\begin{fulllineitems}
\phantomsection\label{\detokenize{namelists/cable_pftparm.nml:CABLE_PFTPARM::froot2_io}}
\pysigstartsignatures
\pysigline{\sphinxcode{\sphinxupquote{CABLE\_PFTPARM::}}\sphinxbfcode{\sphinxupquote{froot2\_io}}}
\pysigstopsignatures\begin{quote}\begin{description}
\sphinxlineitem{Type}
\sphinxAtStartPar
real(npft + nnvg)

\sphinxlineitem{Default}
\sphinxAtStartPar
MDI

\end{description}\end{quote}

\sphinxAtStartPar
Fraction of root in 2nd soil layer.

\end{fulllineitems}

\index{froot3\_io (in namelist CABLE\_PFTPARM)@\spxentry{froot3\_io}\spxextra{in namelist CABLE\_PFTPARM}|spxpagem}

\begin{fulllineitems}
\phantomsection\label{\detokenize{namelists/cable_pftparm.nml:CABLE_PFTPARM::froot3_io}}
\pysigstartsignatures
\pysigline{\sphinxcode{\sphinxupquote{CABLE\_PFTPARM::}}\sphinxbfcode{\sphinxupquote{froot3\_io}}}
\pysigstopsignatures\begin{quote}\begin{description}
\sphinxlineitem{Type}
\sphinxAtStartPar
real(npft + nnvg)

\sphinxlineitem{Default}
\sphinxAtStartPar
MDI

\end{description}\end{quote}

\sphinxAtStartPar
Fraction of root in 3rd soil layer.

\end{fulllineitems}

\index{froot4\_io (in namelist CABLE\_PFTPARM)@\spxentry{froot4\_io}\spxextra{in namelist CABLE\_PFTPARM}|spxpagem}

\begin{fulllineitems}
\phantomsection\label{\detokenize{namelists/cable_pftparm.nml:CABLE_PFTPARM::froot4_io}}
\pysigstartsignatures
\pysigline{\sphinxcode{\sphinxupquote{CABLE\_PFTPARM::}}\sphinxbfcode{\sphinxupquote{froot4\_io}}}
\pysigstopsignatures\begin{quote}\begin{description}
\sphinxlineitem{Type}
\sphinxAtStartPar
real(npft + nnvg)

\sphinxlineitem{Default}
\sphinxAtStartPar
MDI

\end{description}\end{quote}

\sphinxAtStartPar
Fraction of root in 4th soil layer.

\end{fulllineitems}

\index{froot5\_io (in namelist CABLE\_PFTPARM)@\spxentry{froot5\_io}\spxextra{in namelist CABLE\_PFTPARM}|spxpagem}

\begin{fulllineitems}
\phantomsection\label{\detokenize{namelists/cable_pftparm.nml:CABLE_PFTPARM::froot5_io}}
\pysigstartsignatures
\pysigline{\sphinxcode{\sphinxupquote{CABLE\_PFTPARM::}}\sphinxbfcode{\sphinxupquote{froot5\_io}}}
\pysigstopsignatures\begin{quote}\begin{description}
\sphinxlineitem{Type}
\sphinxAtStartPar
real(npft + nnvg)

\sphinxlineitem{Default}
\sphinxAtStartPar
MDI

\end{description}\end{quote}

\sphinxAtStartPar
Fraction of root in 5th soil layer.

\end{fulllineitems}

\index{froot6\_io (in namelist CABLE\_PFTPARM)@\spxentry{froot6\_io}\spxextra{in namelist CABLE\_PFTPARM}|spxpagem}

\begin{fulllineitems}
\phantomsection\label{\detokenize{namelists/cable_pftparm.nml:CABLE_PFTPARM::froot6_io}}
\pysigstartsignatures
\pysigline{\sphinxcode{\sphinxupquote{CABLE\_PFTPARM::}}\sphinxbfcode{\sphinxupquote{froot6\_io}}}
\pysigstopsignatures\begin{quote}\begin{description}
\sphinxlineitem{Type}
\sphinxAtStartPar
real(npft + nnvg)

\sphinxlineitem{Default}
\sphinxAtStartPar
MDI

\end{description}\end{quote}

\sphinxAtStartPar
Fraction of root in 6th soil layer.

\end{fulllineitems}

\index{g0\_io (in namelist CABLE\_PFTPARM)@\spxentry{g0\_io}\spxextra{in namelist CABLE\_PFTPARM}|spxpagem}

\begin{fulllineitems}
\phantomsection\label{\detokenize{namelists/cable_pftparm.nml:CABLE_PFTPARM::g0_io}}
\pysigstartsignatures
\pysigline{\sphinxcode{\sphinxupquote{CABLE\_PFTPARM::}}\sphinxbfcode{\sphinxupquote{g0\_io}}}
\pysigstopsignatures\begin{quote}\begin{description}
\sphinxlineitem{Type}
\sphinxAtStartPar
real(npft + nnvg)

\sphinxlineitem{Default}
\sphinxAtStartPar
MDI

\end{description}\end{quote}

\sphinxAtStartPar
Residual stomatal conductance as net assimilation rate reaches zero (mol m$^{\text{\sphinxhyphen{}2}}$ s$^{\text{\sphinxhyphen{}1}}$)

\end{fulllineitems}

\index{g1\_io (in namelist CABLE\_PFTPARM)@\spxentry{g1\_io}\spxextra{in namelist CABLE\_PFTPARM}|spxpagem}

\begin{fulllineitems}
\phantomsection\label{\detokenize{namelists/cable_pftparm.nml:CABLE_PFTPARM::g1_io}}
\pysigstartsignatures
\pysigline{\sphinxcode{\sphinxupquote{CABLE\_PFTPARM::}}\sphinxbfcode{\sphinxupquote{g1\_io}}}
\pysigstopsignatures\begin{quote}\begin{description}
\sphinxlineitem{Type}
\sphinxAtStartPar
real(npft + nnvg)

\sphinxlineitem{Default}
\sphinxAtStartPar
MDI

\end{description}\end{quote}

\sphinxAtStartPar
Sensitivity of stomatal conductance to the assimilation rate (kPa).

\end{fulllineitems}

\index{gswmin\_io (in namelist CABLE\_PFTPARM)@\spxentry{gswmin\_io}\spxextra{in namelist CABLE\_PFTPARM}|spxpagem}

\begin{fulllineitems}
\phantomsection\label{\detokenize{namelists/cable_pftparm.nml:CABLE_PFTPARM::gswmin_io}}
\pysigstartsignatures
\pysigline{\sphinxcode{\sphinxupquote{CABLE\_PFTPARM::}}\sphinxbfcode{\sphinxupquote{gswmin\_io}}}
\pysigstopsignatures\begin{quote}\begin{description}
\sphinxlineitem{Type}
\sphinxAtStartPar
real(npft + nnvg)

\sphinxlineitem{Default}
\sphinxAtStartPar
MDI

\end{description}\end{quote}

\sphinxAtStartPar
Minimal stomatal conductance (mol m$^{\text{\sphinxhyphen{}2}}$ s$^{\text{\sphinxhyphen{}1}}$)

\end{fulllineitems}

\index{hc\_io (in namelist CABLE\_PFTPARM)@\spxentry{hc\_io}\spxextra{in namelist CABLE\_PFTPARM}|spxpagem}

\begin{fulllineitems}
\phantomsection\label{\detokenize{namelists/cable_pftparm.nml:CABLE_PFTPARM::hc_io}}
\pysigstartsignatures
\pysigline{\sphinxcode{\sphinxupquote{CABLE\_PFTPARM::}}\sphinxbfcode{\sphinxupquote{hc\_io}}}
\pysigstopsignatures\begin{quote}\begin{description}
\sphinxlineitem{Type}
\sphinxAtStartPar
real(npft + nnvg)

\sphinxlineitem{Default}
\sphinxAtStartPar
MDI

\end{description}\end{quote}

\sphinxAtStartPar
Height of canopy (m)

\end{fulllineitems}

\index{lai\_io (in namelist CABLE\_PFTPARM)@\spxentry{lai\_io}\spxextra{in namelist CABLE\_PFTPARM}|spxpagem}

\begin{fulllineitems}
\phantomsection\label{\detokenize{namelists/cable_pftparm.nml:CABLE_PFTPARM::lai_io}}
\pysigstartsignatures
\pysigline{\sphinxcode{\sphinxupquote{CABLE\_PFTPARM::}}\sphinxbfcode{\sphinxupquote{lai\_io}}}
\pysigstopsignatures\begin{quote}\begin{description}
\sphinxlineitem{Type}
\sphinxAtStartPar
real(npft + nnvg)

\sphinxlineitem{Default}
\sphinxAtStartPar
None

\end{description}\end{quote}

\sphinxAtStartPar
The leaf area index (LAI) of each PFT.

\end{fulllineitems}

\index{length\_io (in namelist CABLE\_PFTPARM)@\spxentry{length\_io}\spxextra{in namelist CABLE\_PFTPARM}|spxpagem}

\begin{fulllineitems}
\phantomsection\label{\detokenize{namelists/cable_pftparm.nml:CABLE_PFTPARM::length_io}}
\pysigstartsignatures
\pysigline{\sphinxcode{\sphinxupquote{CABLE\_PFTPARM::}}\sphinxbfcode{\sphinxupquote{length\_io}}}
\pysigstopsignatures\begin{quote}\begin{description}
\sphinxlineitem{Type}
\sphinxAtStartPar
real(npft + nnvg)

\sphinxlineitem{Default}
\sphinxAtStartPar
MDI

\end{description}\end{quote}

\sphinxAtStartPar
Leaf length (m)

\end{fulllineitems}

\index{ratecp1\_io (in namelist CABLE\_PFTPARM)@\spxentry{ratecp1\_io}\spxextra{in namelist CABLE\_PFTPARM}|spxpagem}

\begin{fulllineitems}
\phantomsection\label{\detokenize{namelists/cable_pftparm.nml:CABLE_PFTPARM::ratecp1_io}}
\pysigstartsignatures
\pysigline{\sphinxcode{\sphinxupquote{CABLE\_PFTPARM::}}\sphinxbfcode{\sphinxupquote{ratecp1\_io}}}
\pysigstopsignatures\begin{quote}\begin{description}
\sphinxlineitem{Type}
\sphinxAtStartPar
real(npft + nnvg)

\sphinxlineitem{Default}
\sphinxAtStartPar
MDI

\end{description}\end{quote}

\sphinxAtStartPar
Plant carbon pool rate constant in 1st vegetation carbon store (year$^{\text{\sphinxhyphen{}1}}$).

\end{fulllineitems}

\index{ratecp2\_io (in namelist CABLE\_PFTPARM)@\spxentry{ratecp2\_io}\spxextra{in namelist CABLE\_PFTPARM}|spxpagem}

\begin{fulllineitems}
\phantomsection\label{\detokenize{namelists/cable_pftparm.nml:CABLE_PFTPARM::ratecp2_io}}
\pysigstartsignatures
\pysigline{\sphinxcode{\sphinxupquote{CABLE\_PFTPARM::}}\sphinxbfcode{\sphinxupquote{ratecp2\_io}}}
\pysigstopsignatures\begin{quote}\begin{description}
\sphinxlineitem{Type}
\sphinxAtStartPar
real(npft + nnvg)

\sphinxlineitem{Default}
\sphinxAtStartPar
MDI

\end{description}\end{quote}

\sphinxAtStartPar
Plant carbon pool rate constant in 2nd vegetation carbon store (year$^{\text{\sphinxhyphen{}1}}$).

\end{fulllineitems}

\index{ratecp3\_io (in namelist CABLE\_PFTPARM)@\spxentry{ratecp3\_io}\spxextra{in namelist CABLE\_PFTPARM}|spxpagem}

\begin{fulllineitems}
\phantomsection\label{\detokenize{namelists/cable_pftparm.nml:CABLE_PFTPARM::ratecp3_io}}
\pysigstartsignatures
\pysigline{\sphinxcode{\sphinxupquote{CABLE\_PFTPARM::}}\sphinxbfcode{\sphinxupquote{ratecp3\_io}}}
\pysigstopsignatures\begin{quote}\begin{description}
\sphinxlineitem{Type}
\sphinxAtStartPar
real(npft + nnvg)

\sphinxlineitem{Default}
\sphinxAtStartPar
MDI

\end{description}\end{quote}

\sphinxAtStartPar
Plant carbon pool rate constant in 3rd vegetation carbon store (year$^{\text{\sphinxhyphen{}1}}$).

\end{fulllineitems}

\index{ratecs1\_io (in namelist CABLE\_PFTPARM)@\spxentry{ratecs1\_io}\spxextra{in namelist CABLE\_PFTPARM}|spxpagem}

\begin{fulllineitems}
\phantomsection\label{\detokenize{namelists/cable_pftparm.nml:CABLE_PFTPARM::ratecs1_io}}
\pysigstartsignatures
\pysigline{\sphinxcode{\sphinxupquote{CABLE\_PFTPARM::}}\sphinxbfcode{\sphinxupquote{ratecs1\_io}}}
\pysigstopsignatures\begin{quote}\begin{description}
\sphinxlineitem{Type}
\sphinxAtStartPar
real(npft + nnvg)

\sphinxlineitem{Default}
\sphinxAtStartPar
MDI

\end{description}\end{quote}

\sphinxAtStartPar
Soil carbon pool rate constant in 1st soil carbon store (year$^{\text{\sphinxhyphen{}1}}$).

\end{fulllineitems}

\index{ratecs2\_io (in namelist CABLE\_PFTPARM)@\spxentry{ratecs2\_io}\spxextra{in namelist CABLE\_PFTPARM}|spxpagem}

\begin{fulllineitems}
\phantomsection\label{\detokenize{namelists/cable_pftparm.nml:CABLE_PFTPARM::ratecs2_io}}
\pysigstartsignatures
\pysigline{\sphinxcode{\sphinxupquote{CABLE\_PFTPARM::}}\sphinxbfcode{\sphinxupquote{ratecs2\_io}}}
\pysigstopsignatures\begin{quote}\begin{description}
\sphinxlineitem{Type}
\sphinxAtStartPar
real(npft + nnvg)

\sphinxlineitem{Default}
\sphinxAtStartPar
MDI

\end{description}\end{quote}

\sphinxAtStartPar
Soil carbon pool rate constant in 2nd soil carbon store (year$^{\text{\sphinxhyphen{}1}}$).

\end{fulllineitems}

\index{refl1\_io (in namelist CABLE\_PFTPARM)@\spxentry{refl1\_io}\spxextra{in namelist CABLE\_PFTPARM}|spxpagem}

\begin{fulllineitems}
\phantomsection\label{\detokenize{namelists/cable_pftparm.nml:CABLE_PFTPARM::refl1_io}}
\pysigstartsignatures
\pysigline{\sphinxcode{\sphinxupquote{CABLE\_PFTPARM::}}\sphinxbfcode{\sphinxupquote{refl1\_io}}}
\pysigstopsignatures\begin{quote}\begin{description}
\sphinxlineitem{Type}
\sphinxAtStartPar
real(npft + nnvg)

\sphinxlineitem{Default}
\sphinxAtStartPar
MDI

\end{description}\end{quote}

\sphinxAtStartPar
Leaf reflectance in 1st radiation band.

\end{fulllineitems}

\index{refl2\_io (in namelist CABLE\_PFTPARM)@\spxentry{refl2\_io}\spxextra{in namelist CABLE\_PFTPARM}|spxpagem}

\begin{fulllineitems}
\phantomsection\label{\detokenize{namelists/cable_pftparm.nml:CABLE_PFTPARM::refl2_io}}
\pysigstartsignatures
\pysigline{\sphinxcode{\sphinxupquote{CABLE\_PFTPARM::}}\sphinxbfcode{\sphinxupquote{refl2\_io}}}
\pysigstopsignatures\begin{quote}\begin{description}
\sphinxlineitem{Type}
\sphinxAtStartPar
real(npft + nnvg)

\sphinxlineitem{Default}
\sphinxAtStartPar
MDI

\end{description}\end{quote}

\sphinxAtStartPar
Leaf reflectance in 2nd radiation band.

\end{fulllineitems}

\index{refl3\_io (in namelist CABLE\_PFTPARM)@\spxentry{refl3\_io}\spxextra{in namelist CABLE\_PFTPARM}|spxpagem}

\begin{fulllineitems}
\phantomsection\label{\detokenize{namelists/cable_pftparm.nml:CABLE_PFTPARM::refl3_io}}
\pysigstartsignatures
\pysigline{\sphinxcode{\sphinxupquote{CABLE\_PFTPARM::}}\sphinxbfcode{\sphinxupquote{refl3\_io}}}
\pysigstopsignatures\begin{quote}\begin{description}
\sphinxlineitem{Type}
\sphinxAtStartPar
real(npft + nnvg)

\sphinxlineitem{Default}
\sphinxAtStartPar
MDI

\end{description}\end{quote}

\sphinxAtStartPar
Leaf reflectance in 3rd radiation band.

\end{fulllineitems}

\index{rp20\_io (in namelist CABLE\_PFTPARM)@\spxentry{rp20\_io}\spxextra{in namelist CABLE\_PFTPARM}|spxpagem}

\begin{fulllineitems}
\phantomsection\label{\detokenize{namelists/cable_pftparm.nml:CABLE_PFTPARM::rp20_io}}
\pysigstartsignatures
\pysigline{\sphinxcode{\sphinxupquote{CABLE\_PFTPARM::}}\sphinxbfcode{\sphinxupquote{rp20\_io}}}
\pysigstopsignatures\begin{quote}\begin{description}
\sphinxlineitem{Type}
\sphinxAtStartPar
real(npft + nnvg)

\sphinxlineitem{Default}
\sphinxAtStartPar
MDI

\end{description}\end{quote}

\sphinxAtStartPar
Plant respiration scaler

\end{fulllineitems}

\index{rpcoef\_io (in namelist CABLE\_PFTPARM)@\spxentry{rpcoef\_io}\spxextra{in namelist CABLE\_PFTPARM}|spxpagem}

\begin{fulllineitems}
\phantomsection\label{\detokenize{namelists/cable_pftparm.nml:CABLE_PFTPARM::rpcoef_io}}
\pysigstartsignatures
\pysigline{\sphinxcode{\sphinxupquote{CABLE\_PFTPARM::}}\sphinxbfcode{\sphinxupquote{rpcoef\_io}}}
\pysigstopsignatures\begin{quote}\begin{description}
\sphinxlineitem{Type}
\sphinxAtStartPar
real(npft + nnvg)

\sphinxlineitem{Default}
\sphinxAtStartPar
MDI

\end{description}\end{quote}

\sphinxAtStartPar
Temperature coefficient for non\sphinxhyphen{}leaf plant respiration (C$^{\text{\sphinxhyphen{}1}}$)

\end{fulllineitems}

\index{rs20\_io (in namelist CABLE\_PFTPARM)@\spxentry{rs20\_io}\spxextra{in namelist CABLE\_PFTPARM}|spxpagem}

\begin{fulllineitems}
\phantomsection\label{\detokenize{namelists/cable_pftparm.nml:CABLE_PFTPARM::rs20_io}}
\pysigstartsignatures
\pysigline{\sphinxcode{\sphinxupquote{CABLE\_PFTPARM::}}\sphinxbfcode{\sphinxupquote{rs20\_io}}}
\pysigstopsignatures\begin{quote}\begin{description}
\sphinxlineitem{Type}
\sphinxAtStartPar
real(npft + nnvg)

\sphinxlineitem{Default}
\sphinxAtStartPar
MDI

\end{description}\end{quote}

\sphinxAtStartPar
Soil respiration at 20 deg C

\end{fulllineitems}

\index{shelrb\_io (in namelist CABLE\_PFTPARM)@\spxentry{shelrb\_io}\spxextra{in namelist CABLE\_PFTPARM}|spxpagem}

\begin{fulllineitems}
\phantomsection\label{\detokenize{namelists/cable_pftparm.nml:CABLE_PFTPARM::shelrb_io}}
\pysigstartsignatures
\pysigline{\sphinxcode{\sphinxupquote{CABLE\_PFTPARM::}}\sphinxbfcode{\sphinxupquote{shelrb\_io}}}
\pysigstopsignatures\begin{quote}\begin{description}
\sphinxlineitem{Type}
\sphinxAtStartPar
real(npft + nnvg)

\sphinxlineitem{Default}
\sphinxAtStartPar
MDI

\end{description}\end{quote}

\sphinxAtStartPar
Sheltering factor

\end{fulllineitems}

\index{taul1\_io (in namelist CABLE\_PFTPARM)@\spxentry{taul1\_io}\spxextra{in namelist CABLE\_PFTPARM}|spxpagem}

\begin{fulllineitems}
\phantomsection\label{\detokenize{namelists/cable_pftparm.nml:CABLE_PFTPARM::taul1_io}}
\pysigstartsignatures
\pysigline{\sphinxcode{\sphinxupquote{CABLE\_PFTPARM::}}\sphinxbfcode{\sphinxupquote{taul1\_io}}}
\pysigstopsignatures\begin{quote}\begin{description}
\sphinxlineitem{Type}
\sphinxAtStartPar
real(npft + nnvg)

\sphinxlineitem{Default}
\sphinxAtStartPar
MDI

\end{description}\end{quote}

\sphinxAtStartPar
Leaf transmittance in 1st radiation band

\end{fulllineitems}

\index{taul2\_io (in namelist CABLE\_PFTPARM)@\spxentry{taul2\_io}\spxextra{in namelist CABLE\_PFTPARM}|spxpagem}

\begin{fulllineitems}
\phantomsection\label{\detokenize{namelists/cable_pftparm.nml:CABLE_PFTPARM::taul2_io}}
\pysigstartsignatures
\pysigline{\sphinxcode{\sphinxupquote{CABLE\_PFTPARM::}}\sphinxbfcode{\sphinxupquote{taul2\_io}}}
\pysigstopsignatures\begin{quote}\begin{description}
\sphinxlineitem{Type}
\sphinxAtStartPar
real(npft + nnvg)

\sphinxlineitem{Default}
\sphinxAtStartPar
MDI

\end{description}\end{quote}

\sphinxAtStartPar
Leaf transmittance in 2nd radiation band

\end{fulllineitems}

\index{taul3\_io (in namelist CABLE\_PFTPARM)@\spxentry{taul3\_io}\spxextra{in namelist CABLE\_PFTPARM}|spxpagem}

\begin{fulllineitems}
\phantomsection\label{\detokenize{namelists/cable_pftparm.nml:CABLE_PFTPARM::taul3_io}}
\pysigstartsignatures
\pysigline{\sphinxcode{\sphinxupquote{CABLE\_PFTPARM::}}\sphinxbfcode{\sphinxupquote{taul3\_io}}}
\pysigstopsignatures\begin{quote}\begin{description}
\sphinxlineitem{Type}
\sphinxAtStartPar
real(npft + nnvg)

\sphinxlineitem{Default}
\sphinxAtStartPar
MDI

\end{description}\end{quote}

\sphinxAtStartPar
Leaf transmittance in 3rd radiation band

\end{fulllineitems}

\index{tmaxvj\_io (in namelist CABLE\_PFTPARM)@\spxentry{tmaxvj\_io}\spxextra{in namelist CABLE\_PFTPARM}|spxpagem}

\begin{fulllineitems}
\phantomsection\label{\detokenize{namelists/cable_pftparm.nml:CABLE_PFTPARM::tmaxvj_io}}
\pysigstartsignatures
\pysigline{\sphinxcode{\sphinxupquote{CABLE\_PFTPARM::}}\sphinxbfcode{\sphinxupquote{tmaxvj\_io}}}
\pysigstopsignatures\begin{quote}\begin{description}
\sphinxlineitem{Type}
\sphinxAtStartPar
real(npft + nnvg)

\sphinxlineitem{Default}
\sphinxAtStartPar
MDI

\end{description}\end{quote}

\sphinxAtStartPar
Maximum temperature of the start of photosynthesis (deg C)

\end{fulllineitems}

\index{tminvj\_io (in namelist CABLE\_PFTPARM)@\spxentry{tminvj\_io}\spxextra{in namelist CABLE\_PFTPARM}|spxpagem}

\begin{fulllineitems}
\phantomsection\label{\detokenize{namelists/cable_pftparm.nml:CABLE_PFTPARM::tminvj_io}}
\pysigstartsignatures
\pysigline{\sphinxcode{\sphinxupquote{CABLE\_PFTPARM::}}\sphinxbfcode{\sphinxupquote{tminvj\_io}}}
\pysigstopsignatures\begin{quote}\begin{description}
\sphinxlineitem{Type}
\sphinxAtStartPar
real(npft + nnvg)

\sphinxlineitem{Default}
\sphinxAtStartPar
MDI

\end{description}\end{quote}

\sphinxAtStartPar
Minimum temperature of the start of photosynthesis (deg C)

\end{fulllineitems}

\index{vbeta\_io (in namelist CABLE\_PFTPARM)@\spxentry{vbeta\_io}\spxextra{in namelist CABLE\_PFTPARM}|spxpagem}

\begin{fulllineitems}
\phantomsection\label{\detokenize{namelists/cable_pftparm.nml:CABLE_PFTPARM::vbeta_io}}
\pysigstartsignatures
\pysigline{\sphinxcode{\sphinxupquote{CABLE\_PFTPARM::}}\sphinxbfcode{\sphinxupquote{vbeta\_io}}}
\pysigstopsignatures\begin{quote}\begin{description}
\sphinxlineitem{Type}
\sphinxAtStartPar
real(npft + nnvg)

\sphinxlineitem{Default}
\sphinxAtStartPar
MDI

\end{description}\end{quote}

\sphinxAtStartPar
Stomatal sensitivity to soil water.

\end{fulllineitems}

\index{vcmax\_io (in namelist CABLE\_PFTPARM)@\spxentry{vcmax\_io}\spxextra{in namelist CABLE\_PFTPARM}|spxpagem}

\begin{fulllineitems}
\phantomsection\label{\detokenize{namelists/cable_pftparm.nml:CABLE_PFTPARM::vcmax_io}}
\pysigstartsignatures
\pysigline{\sphinxcode{\sphinxupquote{CABLE\_PFTPARM::}}\sphinxbfcode{\sphinxupquote{vcmax\_io}}}
\pysigstopsignatures\begin{quote}\begin{description}
\sphinxlineitem{Type}
\sphinxAtStartPar
real(npft + nnvg)

\sphinxlineitem{Default}
\sphinxAtStartPar
MDI

\end{description}\end{quote}

\sphinxAtStartPar
Maximum RuBP carboxylation rate top leaf. (mol m$^{\text{\sphinxhyphen{}2}}$ s$^{\text{\sphinxhyphen{}1}}$)

\end{fulllineitems}

\index{vegcf\_io (in namelist CABLE\_PFTPARM)@\spxentry{vegcf\_io}\spxextra{in namelist CABLE\_PFTPARM}|spxpagem}

\begin{fulllineitems}
\phantomsection\label{\detokenize{namelists/cable_pftparm.nml:CABLE_PFTPARM::vegcf_io}}
\pysigstartsignatures
\pysigline{\sphinxcode{\sphinxupquote{CABLE\_PFTPARM::}}\sphinxbfcode{\sphinxupquote{vegcf\_io}}}
\pysigstopsignatures\begin{quote}\begin{description}
\sphinxlineitem{Type}
\sphinxAtStartPar
real(npft + nnvg)

\sphinxlineitem{Default}
\sphinxAtStartPar
MDI

\end{description}\end{quote}

\sphinxAtStartPar
Scalar on soil respiration (place\sphinxhyphen{}holder scheme)

\end{fulllineitems}

\index{wai\_io (in namelist CABLE\_PFTPARM)@\spxentry{wai\_io}\spxextra{in namelist CABLE\_PFTPARM}|spxpagem}

\begin{fulllineitems}
\phantomsection\label{\detokenize{namelists/cable_pftparm.nml:CABLE_PFTPARM::wai_io}}
\pysigstartsignatures
\pysigline{\sphinxcode{\sphinxupquote{CABLE\_PFTPARM::}}\sphinxbfcode{\sphinxupquote{wai\_io}}}
\pysigstopsignatures\begin{quote}\begin{description}
\sphinxlineitem{Type}
\sphinxAtStartPar
real(npft + nnvg)

\sphinxlineitem{Default}
\sphinxAtStartPar
MDI

\end{description}\end{quote}

\sphinxAtStartPar
Wood area index (stem + branches + twigs) (not currently used in any calculations)

\end{fulllineitems}

\index{width\_io (in namelist CABLE\_PFTPARM)@\spxentry{width\_io}\spxextra{in namelist CABLE\_PFTPARM}|spxpagem}

\begin{fulllineitems}
\phantomsection\label{\detokenize{namelists/cable_pftparm.nml:CABLE_PFTPARM::width_io}}
\pysigstartsignatures
\pysigline{\sphinxcode{\sphinxupquote{CABLE\_PFTPARM::}}\sphinxbfcode{\sphinxupquote{width\_io}}}
\pysigstopsignatures\begin{quote}\begin{description}
\sphinxlineitem{Type}
\sphinxAtStartPar
real(npft + nnvg)

\sphinxlineitem{Default}
\sphinxAtStartPar
MDI

\end{description}\end{quote}

\sphinxAtStartPar
Leaf width (m)

\end{fulllineitems}

\index{xalbnir\_io (in namelist CABLE\_PFTPARM)@\spxentry{xalbnir\_io}\spxextra{in namelist CABLE\_PFTPARM}|spxpagem}

\begin{fulllineitems}
\phantomsection\label{\detokenize{namelists/cable_pftparm.nml:CABLE_PFTPARM::xalbnir_io}}
\pysigstartsignatures
\pysigline{\sphinxcode{\sphinxupquote{CABLE\_PFTPARM::}}\sphinxbfcode{\sphinxupquote{xalbnir\_io}}}
\pysigstopsignatures\begin{quote}\begin{description}
\sphinxlineitem{Type}
\sphinxAtStartPar
real(npft + nnvg)

\sphinxlineitem{Default}
\sphinxAtStartPar
MDI

\end{description}\end{quote}

\sphinxAtStartPar
Not currently used in any calculations.

\end{fulllineitems}

\index{xfang\_io (in namelist CABLE\_PFTPARM)@\spxentry{xfang\_io}\spxextra{in namelist CABLE\_PFTPARM}|spxpagem}

\begin{fulllineitems}
\phantomsection\label{\detokenize{namelists/cable_pftparm.nml:CABLE_PFTPARM::xfang_io}}
\pysigstartsignatures
\pysigline{\sphinxcode{\sphinxupquote{CABLE\_PFTPARM::}}\sphinxbfcode{\sphinxupquote{xfang\_io}}}
\pysigstopsignatures\begin{quote}\begin{description}
\sphinxlineitem{Type}
\sphinxAtStartPar
real(npft + nnvg)

\sphinxlineitem{Default}
\sphinxAtStartPar
MDI

\end{description}\end{quote}

\sphinxAtStartPar
Leaf angle parameter

\end{fulllineitems}

\index{zr\_io (in namelist CABLE\_PFTPARM)@\spxentry{zr\_io}\spxextra{in namelist CABLE\_PFTPARM}|spxpagem}

\begin{fulllineitems}
\phantomsection\label{\detokenize{namelists/cable_pftparm.nml:CABLE_PFTPARM::zr_io}}
\pysigstartsignatures
\pysigline{\sphinxcode{\sphinxupquote{CABLE\_PFTPARM::}}\sphinxbfcode{\sphinxupquote{zr\_io}}}
\pysigstopsignatures\begin{quote}\begin{description}
\sphinxlineitem{Type}
\sphinxAtStartPar
real(npft + nnvg)

\sphinxlineitem{Default}
\sphinxAtStartPar
MDI

\end{description}\end{quote}

\sphinxAtStartPar
Maximum rooting depth (cm)

\end{fulllineitems}


\sphinxstepscope


\section{\sphinxstyleliteralintitle{\sphinxupquote{nveg\_params.nml}}}
\label{\detokenize{namelists/nveg_params.nml:nveg-params-nml}}\label{\detokenize{namelists/nveg_params.nml::doc}}
\sphinxAtStartPar
This file contains a namelist called {\hyperref[\detokenize{namelists/nveg_params.nml:namelist-JULES_NVEGPARM}]{\sphinxcrossref{\sphinxcode{\sphinxupquote{JULES\_NVEGPARM}}}}} that sets time\sphinxhyphen{}invariant parameters for non\sphinxhyphen{}vegetation surface types for the JULES land surface model.


\subsection{\sphinxstyleliteralintitle{\sphinxupquote{JULES\_NVEGPARM}} namelist members}
\label{\detokenize{namelists/nveg_params.nml:namelist-JULES_NVEGPARM}}\label{\detokenize{namelists/nveg_params.nml:jules-nvegparm-namelist-members}}\index{JULES\_NVEGPARM (namelist)@\spxentry{JULES\_NVEGPARM}\spxextra{namelist}|spxpagem}
\sphinxAtStartPar
This namelist reads the values of parameters for each of the non\sphinxhyphen{}vegetation surface types if the JULES land surface model is being used. These parameters are a function of surface type only. All parameters must be defined for any configuration.

\sphinxAtStartPar
HCTN30 refers to Hadley Centre technical note 30, available from \sphinxhref{http://www.metoffice.gov.uk/learning/library/publications/science/climate-science-technical-notes}{the Met Office Library}.
\index{albsnc\_nvg\_io (in namelist JULES\_NVEGPARM)@\spxentry{albsnc\_nvg\_io}\spxextra{in namelist JULES\_NVEGPARM}|spxpagem}

\begin{fulllineitems}
\phantomsection\label{\detokenize{namelists/nveg_params.nml:JULES_NVEGPARM::albsnc_nvg_io}}
\pysigstartsignatures
\pysigline{\sphinxcode{\sphinxupquote{JULES\_NVEGPARM::}}\sphinxbfcode{\sphinxupquote{albsnc\_nvg\_io}}}
\pysigstopsignatures\begin{quote}\begin{description}
\sphinxlineitem{Type}
\sphinxAtStartPar
real(nnvg)

\sphinxlineitem{Default}
\sphinxAtStartPar
None

\end{description}\end{quote}

\sphinxAtStartPar
Snow\sphinxhyphen{}covered albedo.

\sphinxAtStartPar
Only used if {\hyperref[\detokenize{namelists/jules_radiation.nml:JULES_RADIATION::l_snow_albedo}]{\sphinxcrossref{\sphinxcode{\sphinxupquote{l\_snow\_albedo}}}}} = FALSE. See HCTN30 Table 1.

\end{fulllineitems}

\index{albsnf\_nvg\_io (in namelist JULES\_NVEGPARM)@\spxentry{albsnf\_nvg\_io}\spxextra{in namelist JULES\_NVEGPARM}|spxpagem}

\begin{fulllineitems}
\phantomsection\label{\detokenize{namelists/nveg_params.nml:JULES_NVEGPARM::albsnf_nvg_io}}
\pysigstartsignatures
\pysigline{\sphinxcode{\sphinxupquote{JULES\_NVEGPARM::}}\sphinxbfcode{\sphinxupquote{albsnf\_nvg\_io}}}
\pysigstopsignatures\begin{quote}\begin{description}
\sphinxlineitem{Type}
\sphinxAtStartPar
real(nnvg)

\sphinxlineitem{Default}
\sphinxAtStartPar
None

\end{description}\end{quote}

\sphinxAtStartPar
Snow\sphinxhyphen{}free albedo.

\sphinxAtStartPar
See HCTN30 Table 1.

\sphinxAtStartPar
A  bare soil snow\sphinxhyphen{}free albedo of \sphinxhyphen{}1.0 indicates that it is supplied by an ancillary field.

\end{fulllineitems}

\index{albsnf\_nvgu\_io (in namelist JULES\_NVEGPARM)@\spxentry{albsnf\_nvgu\_io}\spxextra{in namelist JULES\_NVEGPARM}|spxpagem}

\begin{fulllineitems}
\phantomsection\label{\detokenize{namelists/nveg_params.nml:JULES_NVEGPARM::albsnf_nvgu_io}}
\pysigstartsignatures
\pysigline{\sphinxcode{\sphinxupquote{JULES\_NVEGPARM::}}\sphinxbfcode{\sphinxupquote{albsnf\_nvgu\_io}}}
\pysigstopsignatures\begin{quote}\begin{description}
\sphinxlineitem{Type}
\sphinxAtStartPar
real(nnvg)

\sphinxlineitem{Default}
\sphinxAtStartPar
None

\end{description}\end{quote}

\sphinxAtStartPar
Upper limit on snow\sphinxhyphen{}free albedo, when {\hyperref[\detokenize{namelists/jules_radiation.nml:JULES_RADIATION::l_albedo_obs}]{\sphinxcrossref{\sphinxcode{\sphinxupquote{l\_albedo\_obs}}}}} = TRUE.

\end{fulllineitems}

\index{albsnf\_nvgl\_io (in namelist JULES\_NVEGPARM)@\spxentry{albsnf\_nvgl\_io}\spxextra{in namelist JULES\_NVEGPARM}|spxpagem}

\begin{fulllineitems}
\phantomsection\label{\detokenize{namelists/nveg_params.nml:JULES_NVEGPARM::albsnf_nvgl_io}}
\pysigstartsignatures
\pysigline{\sphinxcode{\sphinxupquote{JULES\_NVEGPARM::}}\sphinxbfcode{\sphinxupquote{albsnf\_nvgl\_io}}}
\pysigstopsignatures\begin{quote}\begin{description}
\sphinxlineitem{Type}
\sphinxAtStartPar
real(nnvg)

\sphinxlineitem{Default}
\sphinxAtStartPar
None

\end{description}\end{quote}

\sphinxAtStartPar
Lower limit on snow\sphinxhyphen{}free albedo, when {\hyperref[\detokenize{namelists/jules_radiation.nml:JULES_RADIATION::l_albedo_obs}]{\sphinxcrossref{\sphinxcode{\sphinxupquote{l\_albedo\_obs}}}}} = TRUE.

\end{fulllineitems}

\index{catch\_nvg\_io (in namelist JULES\_NVEGPARM)@\spxentry{catch\_nvg\_io}\spxextra{in namelist JULES\_NVEGPARM}|spxpagem}

\begin{fulllineitems}
\phantomsection\label{\detokenize{namelists/nveg_params.nml:JULES_NVEGPARM::catch_nvg_io}}
\pysigstartsignatures
\pysigline{\sphinxcode{\sphinxupquote{JULES\_NVEGPARM::}}\sphinxbfcode{\sphinxupquote{catch\_nvg\_io}}}
\pysigstopsignatures\begin{quote}\begin{description}
\sphinxlineitem{Type}
\sphinxAtStartPar
real(nnvg)

\sphinxlineitem{Default}
\sphinxAtStartPar
None

\end{description}\end{quote}

\sphinxAtStartPar
Capacity for water (kg m$^{\text{\sphinxhyphen{}2}}$).

\sphinxAtStartPar
See HCTN30 p7.

\end{fulllineitems}

\index{gs\_nvg\_io (in namelist JULES\_NVEGPARM)@\spxentry{gs\_nvg\_io}\spxextra{in namelist JULES\_NVEGPARM}|spxpagem}

\begin{fulllineitems}
\phantomsection\label{\detokenize{namelists/nveg_params.nml:JULES_NVEGPARM::gs_nvg_io}}
\pysigstartsignatures
\pysigline{\sphinxcode{\sphinxupquote{JULES\_NVEGPARM::}}\sphinxbfcode{\sphinxupquote{gs\_nvg\_io}}}
\pysigstopsignatures\begin{quote}\begin{description}
\sphinxlineitem{Type}
\sphinxAtStartPar
real(nnvg)

\sphinxlineitem{Default}
\sphinxAtStartPar
None

\end{description}\end{quote}

\sphinxAtStartPar
Surface conductance (m s$^{\text{\sphinxhyphen{}1}}$).

\sphinxAtStartPar
See HCTN30 p7. Soil conductance is modified by soil moisture according to HCTN30 Eq35.

\end{fulllineitems}

\index{infil\_nvg\_io (in namelist JULES\_NVEGPARM)@\spxentry{infil\_nvg\_io}\spxextra{in namelist JULES\_NVEGPARM}|spxpagem}

\begin{fulllineitems}
\phantomsection\label{\detokenize{namelists/nveg_params.nml:JULES_NVEGPARM::infil_nvg_io}}
\pysigstartsignatures
\pysigline{\sphinxcode{\sphinxupquote{JULES\_NVEGPARM::}}\sphinxbfcode{\sphinxupquote{infil\_nvg\_io}}}
\pysigstopsignatures\begin{quote}\begin{description}
\sphinxlineitem{Type}
\sphinxAtStartPar
real(nnvg)

\sphinxlineitem{Default}
\sphinxAtStartPar
None

\end{description}\end{quote}

\sphinxAtStartPar
Infiltration enhancement factor.

\sphinxAtStartPar
The maximum infiltration rate defined by the soil parameters for the whole gridbox may be modified for each surface tile to account for tile\sphinxhyphen{}dependent factors.

\sphinxAtStartPar
See HCTN30 p14 for full details.

\end{fulllineitems}

\index{z0\_nvg\_io (in namelist JULES\_NVEGPARM)@\spxentry{z0\_nvg\_io}\spxextra{in namelist JULES\_NVEGPARM}|spxpagem}

\begin{fulllineitems}
\phantomsection\label{\detokenize{namelists/nveg_params.nml:JULES_NVEGPARM::z0_nvg_io}}
\pysigstartsignatures
\pysigline{\sphinxcode{\sphinxupquote{JULES\_NVEGPARM::}}\sphinxbfcode{\sphinxupquote{z0\_nvg\_io}}}
\pysigstopsignatures\begin{quote}\begin{description}
\sphinxlineitem{Type}
\sphinxAtStartPar
real(nnvg)

\sphinxlineitem{Default}
\sphinxAtStartPar
None

\end{description}\end{quote}

\sphinxAtStartPar
Roughness length for momentum (m).

\sphinxAtStartPar
See HCTN30 Table 4.

\end{fulllineitems}

\index{ch\_nvg\_io (in namelist JULES\_NVEGPARM)@\spxentry{ch\_nvg\_io}\spxextra{in namelist JULES\_NVEGPARM}|spxpagem}

\begin{fulllineitems}
\phantomsection\label{\detokenize{namelists/nveg_params.nml:JULES_NVEGPARM::ch_nvg_io}}
\pysigstartsignatures
\pysigline{\sphinxcode{\sphinxupquote{JULES\_NVEGPARM::}}\sphinxbfcode{\sphinxupquote{ch\_nvg\_io}}}
\pysigstopsignatures\begin{quote}\begin{description}
\sphinxlineitem{Type}
\sphinxAtStartPar
real(nnvg)

\sphinxlineitem{Default}
\sphinxAtStartPar
None

\end{description}\end{quote}

\sphinxAtStartPar
Heat capacity of this surface type (J K$^{\text{\sphinxhyphen{}1}}$ m$^{\text{\sphinxhyphen{}2}}$).

\sphinxAtStartPar
Used only if {\hyperref[\detokenize{namelists/jules_vegetation.nml:JULES_VEGETATION::can_model}]{\sphinxcrossref{\sphinxcode{\sphinxupquote{can\_model}}}}} is 3 or 4.

\end{fulllineitems}

\index{vf\_nvg\_io (in namelist JULES\_NVEGPARM)@\spxentry{vf\_nvg\_io}\spxextra{in namelist JULES\_NVEGPARM}|spxpagem}

\begin{fulllineitems}
\phantomsection\label{\detokenize{namelists/nveg_params.nml:JULES_NVEGPARM::vf_nvg_io}}
\pysigstartsignatures
\pysigline{\sphinxcode{\sphinxupquote{JULES\_NVEGPARM::}}\sphinxbfcode{\sphinxupquote{vf\_nvg\_io}}}
\pysigstopsignatures\begin{quote}\begin{description}
\sphinxlineitem{Type}
\sphinxAtStartPar
real(nnvg)

\sphinxlineitem{Default}
\sphinxAtStartPar
None

\end{description}\end{quote}

\sphinxAtStartPar
Fractional coverage of non\sphinxhyphen{}vegetation “canopy”.

\sphinxAtStartPar
Typically set to 0.0 (conductively coupled), but value of 1.0 (radiatively coupled) used if surface tile should have a heat capacity in conjunction with {\hyperref[\detokenize{namelists/jules_vegetation.nml:JULES_VEGETATION::can_model}]{\sphinxcrossref{\sphinxcode{\sphinxupquote{can\_model}}}}} options 3 or 4.

\begin{sphinxadmonition}{note}{Note:}
\sphinxAtStartPar
If {\hyperref[\detokenize{namelists/urban.nml:JULES_URBAN::l_moruses_storage}]{\sphinxcrossref{\sphinxcode{\sphinxupquote{l\_moruses\_storage}}}}} = T, then for the roof coupling: 0 = \sphinxstylestrong{uncoupled}
\end{sphinxadmonition}

\end{fulllineitems}

\index{emis\_nvg\_io (in namelist JULES\_NVEGPARM)@\spxentry{emis\_nvg\_io}\spxextra{in namelist JULES\_NVEGPARM}|spxpagem}

\begin{fulllineitems}
\phantomsection\label{\detokenize{namelists/nveg_params.nml:JULES_NVEGPARM::emis_nvg_io}}
\pysigstartsignatures
\pysigline{\sphinxcode{\sphinxupquote{JULES\_NVEGPARM::}}\sphinxbfcode{\sphinxupquote{emis\_nvg\_io}}}
\pysigstopsignatures\begin{quote}\begin{description}
\sphinxlineitem{Type}
\sphinxAtStartPar
real(nnvg)

\sphinxlineitem{Default}
\sphinxAtStartPar
None

\end{description}\end{quote}

\sphinxAtStartPar
Surface emissivity of non\sphinxhyphen{}vegetated surfaces.

\end{fulllineitems}

\index{z0hm\_nvg\_io (in namelist JULES\_NVEGPARM)@\spxentry{z0hm\_nvg\_io}\spxextra{in namelist JULES\_NVEGPARM}|spxpagem}

\begin{fulllineitems}
\phantomsection\label{\detokenize{namelists/nveg_params.nml:JULES_NVEGPARM::z0hm_nvg_io}}
\pysigstartsignatures
\pysigline{\sphinxcode{\sphinxupquote{JULES\_NVEGPARM::}}\sphinxbfcode{\sphinxupquote{z0hm\_nvg\_io}}}
\pysigstopsignatures\begin{quote}\begin{description}
\sphinxlineitem{Type}
\sphinxAtStartPar
real(nnvg)

\sphinxlineitem{Default}
\sphinxAtStartPar
None

\end{description}\end{quote}

\sphinxAtStartPar
Ratio of the roughness length for heat to the roughness length for momentum.

\sphinxAtStartPar
This is generally assumed to be 0.1. See HCTN30 p6. Note that this is the ratio of the roughness length for heat to that for momentum. It does not alter the roughness length for momentum, which is given by {\hyperref[\detokenize{namelists/nveg_params.nml:JULES_NVEGPARM::z0_nvg_io}]{\sphinxcrossref{\sphinxcode{\sphinxupquote{z0\_nvg\_io}}}}}.

\end{fulllineitems}

\index{z0hm\_classic\_nvg\_io (in namelist JULES\_NVEGPARM)@\spxentry{z0hm\_classic\_nvg\_io}\spxextra{in namelist JULES\_NVEGPARM}|spxpagem}

\begin{fulllineitems}
\phantomsection\label{\detokenize{namelists/nveg_params.nml:JULES_NVEGPARM::z0hm_classic_nvg_io}}
\pysigstartsignatures
\pysigline{\sphinxcode{\sphinxupquote{JULES\_NVEGPARM::}}\sphinxbfcode{\sphinxupquote{z0hm\_classic\_nvg\_io}}}
\pysigstopsignatures\begin{quote}\begin{description}
\sphinxlineitem{Type}
\sphinxAtStartPar
real(nnvg)

\sphinxlineitem{Default}
\sphinxAtStartPar
None

\end{description}\end{quote}

\sphinxAtStartPar
Ratio of the roughness length for heat to the roughness length for momentum \sphinxstyleemphasis{for the CLASSIC aerosol scheme only}.

\begin{sphinxadmonition}{note}{Note:}
\sphinxAtStartPar
This makes no difference to the model when running standalone, and is only required to keep the standalone and UM interfaces consistent.
\end{sphinxadmonition}

\end{fulllineitems}


\sphinxstepscope


\section{\sphinxstyleliteralintitle{\sphinxupquote{cable\_soilparm.nml}}}
\label{\detokenize{namelists/cable_soilparm.nml:cable-soilparm-nml}}\label{\detokenize{namelists/cable_soilparm.nml::doc}}
\sphinxAtStartPar
This file contains a namelist called {\hyperref[\detokenize{namelists/cable_soilparm.nml:namelist-CABLE_SOILPARM}]{\sphinxcrossref{\sphinxcode{\sphinxupquote{CABLE\_SOILPARM}}}}} that sets time\sphinxhyphen{}invariant parameters for different soil types for the CABLE land surface model.


\subsection{\sphinxstyleliteralintitle{\sphinxupquote{CABLE\_SOILPARM}} namelist members}
\label{\detokenize{namelists/cable_soilparm.nml:namelist-CABLE_SOILPARM}}\label{\detokenize{namelists/cable_soilparm.nml:cable-soilparm-namelist-members}}\index{CABLE\_SOILPARM (namelist)@\spxentry{CABLE\_SOILPARM}\spxextra{namelist}|spxpagem}
\sphinxAtStartPar
This namelist reads the values of parameters for each of the soil types if the CABLE land surface model is being used. These parameters are a function of surface type only. All parameters must be defined for any configuration.
The number of soil types is stored in the \sphinxtitleref{n\_soiltypes} parameter and for the current version of CABLE is set to \sphinxtitleref{9}.
\index{silt\_io (in namelist CABLE\_SOILPARM)@\spxentry{silt\_io}\spxextra{in namelist CABLE\_SOILPARM}|spxpagem}

\begin{fulllineitems}
\phantomsection\label{\detokenize{namelists/cable_soilparm.nml:CABLE_SOILPARM::silt_io}}
\pysigstartsignatures
\pysigline{\sphinxcode{\sphinxupquote{CABLE\_SOILPARM::}}\sphinxbfcode{\sphinxupquote{silt\_io}}}
\pysigstopsignatures\begin{quote}\begin{description}
\sphinxlineitem{Type}
\sphinxAtStartPar
real(n\_soiltypes)

\sphinxlineitem{Default}
\sphinxAtStartPar
MDI

\end{description}\end{quote}

\sphinxAtStartPar
Fraction of soil which is silt.

\end{fulllineitems}

\index{clay\_io (in namelist CABLE\_SOILPARM)@\spxentry{clay\_io}\spxextra{in namelist CABLE\_SOILPARM}|spxpagem}

\begin{fulllineitems}
\phantomsection\label{\detokenize{namelists/cable_soilparm.nml:CABLE_SOILPARM::clay_io}}
\pysigstartsignatures
\pysigline{\sphinxcode{\sphinxupquote{CABLE\_SOILPARM::}}\sphinxbfcode{\sphinxupquote{clay\_io}}}
\pysigstopsignatures\begin{quote}\begin{description}
\sphinxlineitem{Type}
\sphinxAtStartPar
real(n\_soiltypes)

\sphinxlineitem{Default}
\sphinxAtStartPar
MDI

\end{description}\end{quote}

\sphinxAtStartPar
Fraction of soil which is clay.

\end{fulllineitems}

\index{sand\_io (in namelist CABLE\_SOILPARM)@\spxentry{sand\_io}\spxextra{in namelist CABLE\_SOILPARM}|spxpagem}

\begin{fulllineitems}
\phantomsection\label{\detokenize{namelists/cable_soilparm.nml:CABLE_SOILPARM::sand_io}}
\pysigstartsignatures
\pysigline{\sphinxcode{\sphinxupquote{CABLE\_SOILPARM::}}\sphinxbfcode{\sphinxupquote{sand\_io}}}
\pysigstopsignatures\begin{quote}\begin{description}
\sphinxlineitem{Type}
\sphinxAtStartPar
real(n\_soiltypes)

\sphinxlineitem{Default}
\sphinxAtStartPar
MDI

\end{description}\end{quote}

\sphinxAtStartPar
Fraction of soil which is sand.

\end{fulllineitems}

\index{swilt\_io (in namelist CABLE\_SOILPARM)@\spxentry{swilt\_io}\spxextra{in namelist CABLE\_SOILPARM}|spxpagem}

\begin{fulllineitems}
\phantomsection\label{\detokenize{namelists/cable_soilparm.nml:CABLE_SOILPARM::swilt_io}}
\pysigstartsignatures
\pysigline{\sphinxcode{\sphinxupquote{CABLE\_SOILPARM::}}\sphinxbfcode{\sphinxupquote{swilt\_io}}}
\pysigstopsignatures\begin{quote}\begin{description}
\sphinxlineitem{Type}
\sphinxAtStartPar
real(n\_soiltypes)

\sphinxlineitem{Default}
\sphinxAtStartPar
MDI

\end{description}\end{quote}

\sphinxAtStartPar
Volume of H$_{\text{2}}$O at wilting (m$^{\text{3}}$ m$^{\text{\sphinxhyphen{}3}}$)

\end{fulllineitems}

\index{sfc\_io (in namelist CABLE\_SOILPARM)@\spxentry{sfc\_io}\spxextra{in namelist CABLE\_SOILPARM}|spxpagem}

\begin{fulllineitems}
\phantomsection\label{\detokenize{namelists/cable_soilparm.nml:CABLE_SOILPARM::sfc_io}}
\pysigstartsignatures
\pysigline{\sphinxcode{\sphinxupquote{CABLE\_SOILPARM::}}\sphinxbfcode{\sphinxupquote{sfc\_io}}}
\pysigstopsignatures\begin{quote}\begin{description}
\sphinxlineitem{Type}
\sphinxAtStartPar
real(n\_soiltypes)

\sphinxlineitem{Default}
\sphinxAtStartPar
MDI

\end{description}\end{quote}

\sphinxAtStartPar
Volume of H$_{\text{2}}$O at field capacity (m$^{\text{3}}$ m$^{\text{\sphinxhyphen{}3}}$)

\end{fulllineitems}

\index{ssat\_io (in namelist CABLE\_SOILPARM)@\spxentry{ssat\_io}\spxextra{in namelist CABLE\_SOILPARM}|spxpagem}

\begin{fulllineitems}
\phantomsection\label{\detokenize{namelists/cable_soilparm.nml:CABLE_SOILPARM::ssat_io}}
\pysigstartsignatures
\pysigline{\sphinxcode{\sphinxupquote{CABLE\_SOILPARM::}}\sphinxbfcode{\sphinxupquote{ssat\_io}}}
\pysigstopsignatures\begin{quote}\begin{description}
\sphinxlineitem{Type}
\sphinxAtStartPar
real(n\_soiltypes)

\sphinxlineitem{Default}
\sphinxAtStartPar
MDI

\end{description}\end{quote}

\sphinxAtStartPar
Volume of H$_{\text{2}}$O at saturation (m$^{\text{3}}$ m$^{\text{\sphinxhyphen{}3}}$)

\end{fulllineitems}

\index{bch\_io (in namelist CABLE\_SOILPARM)@\spxentry{bch\_io}\spxextra{in namelist CABLE\_SOILPARM}|spxpagem}

\begin{fulllineitems}
\phantomsection\label{\detokenize{namelists/cable_soilparm.nml:CABLE_SOILPARM::bch_io}}
\pysigstartsignatures
\pysigline{\sphinxcode{\sphinxupquote{CABLE\_SOILPARM::}}\sphinxbfcode{\sphinxupquote{bch\_io}}}
\pysigstopsignatures\begin{quote}\begin{description}
\sphinxlineitem{Type}
\sphinxAtStartPar
real(n\_soiltypes)

\sphinxlineitem{Default}
\sphinxAtStartPar
MDI

\end{description}\end{quote}

\sphinxAtStartPar
Parameter b in Campbell equation.

\end{fulllineitems}

\index{hyds\_io (in namelist CABLE\_SOILPARM)@\spxentry{hyds\_io}\spxextra{in namelist CABLE\_SOILPARM}|spxpagem}

\begin{fulllineitems}
\phantomsection\label{\detokenize{namelists/cable_soilparm.nml:CABLE_SOILPARM::hyds_io}}
\pysigstartsignatures
\pysigline{\sphinxcode{\sphinxupquote{CABLE\_SOILPARM::}}\sphinxbfcode{\sphinxupquote{hyds\_io}}}
\pysigstopsignatures\begin{quote}\begin{description}
\sphinxlineitem{Type}
\sphinxAtStartPar
real(n\_soiltypes)

\sphinxlineitem{Default}
\sphinxAtStartPar
MDI

\end{description}\end{quote}

\sphinxAtStartPar
Hydraulic conductivity at saturation (m$^{\text{\sphinxhyphen{}1}}$).

\end{fulllineitems}

\index{sucs\_io (in namelist CABLE\_SOILPARM)@\spxentry{sucs\_io}\spxextra{in namelist CABLE\_SOILPARM}|spxpagem}

\begin{fulllineitems}
\phantomsection\label{\detokenize{namelists/cable_soilparm.nml:CABLE_SOILPARM::sucs_io}}
\pysigstartsignatures
\pysigline{\sphinxcode{\sphinxupquote{CABLE\_SOILPARM::}}\sphinxbfcode{\sphinxupquote{sucs\_io}}}
\pysigstopsignatures\begin{quote}\begin{description}
\sphinxlineitem{Type}
\sphinxAtStartPar
real(n\_soiltypes)

\sphinxlineitem{Default}
\sphinxAtStartPar
MDI

\end{description}\end{quote}

\sphinxAtStartPar
Suction at saturation (m).

\end{fulllineitems}

\index{rhosoil\_io (in namelist CABLE\_SOILPARM)@\spxentry{rhosoil\_io}\spxextra{in namelist CABLE\_SOILPARM}|spxpagem}

\begin{fulllineitems}
\phantomsection\label{\detokenize{namelists/cable_soilparm.nml:CABLE_SOILPARM::rhosoil_io}}
\pysigstartsignatures
\pysigline{\sphinxcode{\sphinxupquote{CABLE\_SOILPARM::}}\sphinxbfcode{\sphinxupquote{rhosoil\_io}}}
\pysigstopsignatures\begin{quote}\begin{description}
\sphinxlineitem{Type}
\sphinxAtStartPar
real(n\_soiltypes)

\sphinxlineitem{Default}
\sphinxAtStartPar
MDI

\end{description}\end{quote}

\sphinxAtStartPar
Soil bulk density (kg m$^{\text{\sphinxhyphen{}3}}$)

\end{fulllineitems}

\index{css\_io (in namelist CABLE\_SOILPARM)@\spxentry{css\_io}\spxextra{in namelist CABLE\_SOILPARM}|spxpagem}

\begin{fulllineitems}
\phantomsection\label{\detokenize{namelists/cable_soilparm.nml:CABLE_SOILPARM::css_io}}
\pysigstartsignatures
\pysigline{\sphinxcode{\sphinxupquote{CABLE\_SOILPARM::}}\sphinxbfcode{\sphinxupquote{css\_io}}}
\pysigstopsignatures\begin{quote}\begin{description}
\sphinxlineitem{Type}
\sphinxAtStartPar
real(n\_soiltypes)

\sphinxlineitem{Default}
\sphinxAtStartPar
MDI

\end{description}\end{quote}

\sphinxAtStartPar
Soil specific heat capacity (J kg$^{\text{\sphinxhyphen{}1}}$ K$^{\text{\sphinxhyphen{}1}}$).

\end{fulllineitems}


\sphinxstepscope


\section{\sphinxstyleliteralintitle{\sphinxupquote{crop\_params.nml}}}
\label{\detokenize{namelists/crop_params.nml:crop-params-nml}}\label{\detokenize{namelists/crop_params.nml::doc}}
\sphinxAtStartPar
This file contains a single namelist called {\hyperref[\detokenize{namelists/crop_params.nml:namelist-JULES_CROPPARM}]{\sphinxcrossref{\sphinxcode{\sphinxupquote{JULES\_CROPPARM}}}}} that sets time\sphinxhyphen{} and space\sphinxhyphen{}invariant parameters for each crop type.


\subsection{\sphinxstyleliteralintitle{\sphinxupquote{JULES\_CROPPARM}} namelist members}
\label{\detokenize{namelists/crop_params.nml:namelist-JULES_CROPPARM}}\label{\detokenize{namelists/crop_params.nml:jules-cropparm-namelist-members}}\index{JULES\_CROPPARM (namelist)@\spxentry{JULES\_CROPPARM}\spxextra{namelist}|spxpagem}
\sphinxAtStartPar
This namelist reads the values of parameters for each of the crop functional types. These parameters are a function of crop pft only.  These parameters are only required if {\hyperref[\detokenize{namelists/jules_surface_types.nml:JULES_SURFACE_TYPES::ncpft}]{\sphinxcrossref{\sphinxcode{\sphinxupquote{ncpft}}}}} \textgreater{} 0.  The crop pfts should be in the same order as in {\hyperref[\detokenize{namelists/pft_params.nml::doc}]{\sphinxcrossref{\DUrole{doc}{pft\_params.nml}}}}.


\sphinxstrong{See also:}
\nopagebreak


\sphinxAtStartPar
References:
\begin{itemize}
\item {} 
\sphinxAtStartPar
Osborne et al, \sphinxhref{http://www.geosci-model-dev.net/8/1139/2015/gmd-8-1139-2015.html}{JULES\sphinxhyphen{}crop: a parametrisation of crops in the Joint UK Land Environment Simulator}, Geosci. Model Dev., 8, 1139\sphinxhyphen{}1155, 2015.

\end{itemize}

\sphinxAtStartPar
Parameters introduced after the Osborne et al 2015 paper are described in the appendix of
\begin{itemize}
\item {} 
\sphinxAtStartPar
Williams et al, \sphinxhref{https://www.geosci-model-dev.net/10/1291/2017/gmd-10-1291-2017.html}{Evaluation of JULES\sphinxhyphen{}crop performance against site observations of irrigated maize from Mead, Nebraska}, Geosci. Model Dev., 10, 1291\sphinxhyphen{}1320, 2017.

\end{itemize}


\index{t\_bse\_io (in namelist JULES\_CROPPARM)@\spxentry{t\_bse\_io}\spxextra{in namelist JULES\_CROPPARM}|spxpagem}

\begin{fulllineitems}
\phantomsection\label{\detokenize{namelists/crop_params.nml:JULES_CROPPARM::t_bse_io}}
\pysigstartsignatures
\pysigline{\sphinxcode{\sphinxupquote{JULES\_CROPPARM::}}\sphinxbfcode{\sphinxupquote{t\_bse\_io}}}
\pysigstopsignatures\begin{quote}\begin{description}
\sphinxlineitem{Type}
\sphinxAtStartPar
real(ncpft)

\sphinxlineitem{Default}
\sphinxAtStartPar
None

\end{description}\end{quote}

\sphinxAtStartPar
Base temperature (K).

\end{fulllineitems}

\index{t\_opt\_io (in namelist JULES\_CROPPARM)@\spxentry{t\_opt\_io}\spxextra{in namelist JULES\_CROPPARM}|spxpagem}

\begin{fulllineitems}
\phantomsection\label{\detokenize{namelists/crop_params.nml:JULES_CROPPARM::t_opt_io}}
\pysigstartsignatures
\pysigline{\sphinxcode{\sphinxupquote{JULES\_CROPPARM::}}\sphinxbfcode{\sphinxupquote{t\_opt\_io}}}
\pysigstopsignatures\begin{quote}\begin{description}
\sphinxlineitem{Type}
\sphinxAtStartPar
real(ncpft)

\sphinxlineitem{Default}
\sphinxAtStartPar
None

\end{description}\end{quote}

\sphinxAtStartPar
Optimum temperature (K).

\end{fulllineitems}

\index{tmax\_io (in namelist JULES\_CROPPARM)@\spxentry{tmax\_io}\spxextra{in namelist JULES\_CROPPARM}|spxpagem}

\begin{fulllineitems}
\phantomsection\label{\detokenize{namelists/crop_params.nml:JULES_CROPPARM::tmax_io}}
\pysigstartsignatures
\pysigline{\sphinxcode{\sphinxupquote{JULES\_CROPPARM::}}\sphinxbfcode{\sphinxupquote{tmax\_io}}}
\pysigstopsignatures\begin{quote}\begin{description}
\sphinxlineitem{Type}
\sphinxAtStartPar
real(ncpft)

\sphinxlineitem{Default}
\sphinxAtStartPar
None

\end{description}\end{quote}

\sphinxAtStartPar
Maximum temperature (K).

\end{fulllineitems}

\index{tt\_emr\_io (in namelist JULES\_CROPPARM)@\spxentry{tt\_emr\_io}\spxextra{in namelist JULES\_CROPPARM}|spxpagem}

\begin{fulllineitems}
\phantomsection\label{\detokenize{namelists/crop_params.nml:JULES_CROPPARM::tt_emr_io}}
\pysigstartsignatures
\pysigline{\sphinxcode{\sphinxupquote{JULES\_CROPPARM::}}\sphinxbfcode{\sphinxupquote{tt\_emr\_io}}}
\pysigstopsignatures\begin{quote}\begin{description}
\sphinxlineitem{Type}
\sphinxAtStartPar
real(ncpft)

\sphinxlineitem{Default}
\sphinxAtStartPar
None

\end{description}\end{quote}

\sphinxAtStartPar
Thermal time between sowing and emergence (deg Cd).

\end{fulllineitems}

\index{crit\_pp\_io (in namelist JULES\_CROPPARM)@\spxentry{crit\_pp\_io}\spxextra{in namelist JULES\_CROPPARM}|spxpagem}

\begin{fulllineitems}
\phantomsection\label{\detokenize{namelists/crop_params.nml:JULES_CROPPARM::crit_pp_io}}
\pysigstartsignatures
\pysigline{\sphinxcode{\sphinxupquote{JULES\_CROPPARM::}}\sphinxbfcode{\sphinxupquote{crit\_pp\_io}}}
\pysigstopsignatures\begin{quote}\begin{description}
\sphinxlineitem{Type}
\sphinxAtStartPar
real(ncpft)

\sphinxlineitem{Default}
\sphinxAtStartPar
None

\end{description}\end{quote}

\sphinxAtStartPar
Critical photoperiod (hours).

\end{fulllineitems}

\index{pp\_sens\_io (in namelist JULES\_CROPPARM)@\spxentry{pp\_sens\_io}\spxextra{in namelist JULES\_CROPPARM}|spxpagem}

\begin{fulllineitems}
\phantomsection\label{\detokenize{namelists/crop_params.nml:JULES_CROPPARM::pp_sens_io}}
\pysigstartsignatures
\pysigline{\sphinxcode{\sphinxupquote{JULES\_CROPPARM::}}\sphinxbfcode{\sphinxupquote{pp\_sens\_io}}}
\pysigstopsignatures\begin{quote}\begin{description}
\sphinxlineitem{Type}
\sphinxAtStartPar
real(ncpft)

\sphinxlineitem{Default}
\sphinxAtStartPar
None

\end{description}\end{quote}

\sphinxAtStartPar
Sensitivity of development rate to photoperiod (hours$^{\text{\sphinxhyphen{}1}}$).

\end{fulllineitems}

\index{rt\_dir\_io (in namelist JULES\_CROPPARM)@\spxentry{rt\_dir\_io}\spxextra{in namelist JULES\_CROPPARM}|spxpagem}

\begin{fulllineitems}
\phantomsection\label{\detokenize{namelists/crop_params.nml:JULES_CROPPARM::rt_dir_io}}
\pysigstartsignatures
\pysigline{\sphinxcode{\sphinxupquote{JULES\_CROPPARM::}}\sphinxbfcode{\sphinxupquote{rt\_dir\_io}}}
\pysigstopsignatures\begin{quote}\begin{description}
\sphinxlineitem{Type}
\sphinxAtStartPar
real(ncpft)

\sphinxlineitem{Default}
\sphinxAtStartPar
None

\end{description}\end{quote}

\sphinxAtStartPar
Coefficient determining relative growth of roots vertically and horizontally.

\end{fulllineitems}

\index{alpha1\_io (in namelist JULES\_CROPPARM)@\spxentry{alpha1\_io}\spxextra{in namelist JULES\_CROPPARM}|spxpagem}

\begin{fulllineitems}
\phantomsection\label{\detokenize{namelists/crop_params.nml:JULES_CROPPARM::alpha1_io}}
\pysigstartsignatures
\pysigline{\sphinxcode{\sphinxupquote{JULES\_CROPPARM::}}\sphinxbfcode{\sphinxupquote{alpha1\_io}}}
\pysigstopsignatures\begin{quote}\begin{description}
\sphinxlineitem{Type}
\sphinxAtStartPar
real(ncpft)

\sphinxlineitem{Default}
\sphinxAtStartPar
None

\end{description}\end{quote}

\sphinxAtStartPar
Coefficient for determining partitioning.

\end{fulllineitems}

\index{alpha2\_io (in namelist JULES\_CROPPARM)@\spxentry{alpha2\_io}\spxextra{in namelist JULES\_CROPPARM}|spxpagem}

\begin{fulllineitems}
\phantomsection\label{\detokenize{namelists/crop_params.nml:JULES_CROPPARM::alpha2_io}}
\pysigstartsignatures
\pysigline{\sphinxcode{\sphinxupquote{JULES\_CROPPARM::}}\sphinxbfcode{\sphinxupquote{alpha2\_io}}}
\pysigstopsignatures\begin{quote}\begin{description}
\sphinxlineitem{Type}
\sphinxAtStartPar
real(ncpft)

\sphinxlineitem{Default}
\sphinxAtStartPar
None

\end{description}\end{quote}

\sphinxAtStartPar
Coefficient for determining partitioning.

\end{fulllineitems}

\index{alpha3\_io (in namelist JULES\_CROPPARM)@\spxentry{alpha3\_io}\spxextra{in namelist JULES\_CROPPARM}|spxpagem}

\begin{fulllineitems}
\phantomsection\label{\detokenize{namelists/crop_params.nml:JULES_CROPPARM::alpha3_io}}
\pysigstartsignatures
\pysigline{\sphinxcode{\sphinxupquote{JULES\_CROPPARM::}}\sphinxbfcode{\sphinxupquote{alpha3\_io}}}
\pysigstopsignatures\begin{quote}\begin{description}
\sphinxlineitem{Type}
\sphinxAtStartPar
real(ncpft)

\sphinxlineitem{Default}
\sphinxAtStartPar
None

\end{description}\end{quote}

\sphinxAtStartPar
Coefficient for determining partitioning.

\end{fulllineitems}

\index{beta1\_io (in namelist JULES\_CROPPARM)@\spxentry{beta1\_io}\spxextra{in namelist JULES\_CROPPARM}|spxpagem}

\begin{fulllineitems}
\phantomsection\label{\detokenize{namelists/crop_params.nml:JULES_CROPPARM::beta1_io}}
\pysigstartsignatures
\pysigline{\sphinxcode{\sphinxupquote{JULES\_CROPPARM::}}\sphinxbfcode{\sphinxupquote{beta1\_io}}}
\pysigstopsignatures\begin{quote}\begin{description}
\sphinxlineitem{Type}
\sphinxAtStartPar
real(ncpft)

\sphinxlineitem{Default}
\sphinxAtStartPar
None

\end{description}\end{quote}

\sphinxAtStartPar
Coefficient for determining partitioning.

\end{fulllineitems}

\index{beta2\_io (in namelist JULES\_CROPPARM)@\spxentry{beta2\_io}\spxextra{in namelist JULES\_CROPPARM}|spxpagem}

\begin{fulllineitems}
\phantomsection\label{\detokenize{namelists/crop_params.nml:JULES_CROPPARM::beta2_io}}
\pysigstartsignatures
\pysigline{\sphinxcode{\sphinxupquote{JULES\_CROPPARM::}}\sphinxbfcode{\sphinxupquote{beta2\_io}}}
\pysigstopsignatures\begin{quote}\begin{description}
\sphinxlineitem{Type}
\sphinxAtStartPar
real(ncpft)

\sphinxlineitem{Default}
\sphinxAtStartPar
None

\end{description}\end{quote}

\sphinxAtStartPar
Coefficient for determining partitioning.

\end{fulllineitems}

\index{beta3\_io (in namelist JULES\_CROPPARM)@\spxentry{beta3\_io}\spxextra{in namelist JULES\_CROPPARM}|spxpagem}

\begin{fulllineitems}
\phantomsection\label{\detokenize{namelists/crop_params.nml:JULES_CROPPARM::beta3_io}}
\pysigstartsignatures
\pysigline{\sphinxcode{\sphinxupquote{JULES\_CROPPARM::}}\sphinxbfcode{\sphinxupquote{beta3\_io}}}
\pysigstopsignatures\begin{quote}\begin{description}
\sphinxlineitem{Type}
\sphinxAtStartPar
real(ncpft)

\sphinxlineitem{Default}
\sphinxAtStartPar
None

\end{description}\end{quote}

\sphinxAtStartPar
Coefficient for determining partitioning.

\end{fulllineitems}

\index{gamma\_io (in namelist JULES\_CROPPARM)@\spxentry{gamma\_io}\spxextra{in namelist JULES\_CROPPARM}|spxpagem}

\begin{fulllineitems}
\phantomsection\label{\detokenize{namelists/crop_params.nml:JULES_CROPPARM::gamma_io}}
\pysigstartsignatures
\pysigline{\sphinxcode{\sphinxupquote{JULES\_CROPPARM::}}\sphinxbfcode{\sphinxupquote{gamma\_io}}}
\pysigstopsignatures\begin{quote}\begin{description}
\sphinxlineitem{Type}
\sphinxAtStartPar
real(ncpft)

\sphinxlineitem{Default}
\sphinxAtStartPar
None

\end{description}\end{quote}

\sphinxAtStartPar
Coefficient for determining specific leaf area (m$^{\text{2}}$ kg$^{\text{\sphinxhyphen{}1}}$).

\end{fulllineitems}

\index{delta\_io (in namelist JULES\_CROPPARM)@\spxentry{delta\_io}\spxextra{in namelist JULES\_CROPPARM}|spxpagem}

\begin{fulllineitems}
\phantomsection\label{\detokenize{namelists/crop_params.nml:JULES_CROPPARM::delta_io}}
\pysigstartsignatures
\pysigline{\sphinxcode{\sphinxupquote{JULES\_CROPPARM::}}\sphinxbfcode{\sphinxupquote{delta\_io}}}
\pysigstopsignatures\begin{quote}\begin{description}
\sphinxlineitem{Type}
\sphinxAtStartPar
real(ncpft)

\sphinxlineitem{Default}
\sphinxAtStartPar
None

\end{description}\end{quote}

\sphinxAtStartPar
Coefficient for determining specific leaf area (m$^{\text{2}}$ kg$^{\text{\sphinxhyphen{}1}}$).

\end{fulllineitems}

\index{remob\_io (in namelist JULES\_CROPPARM)@\spxentry{remob\_io}\spxextra{in namelist JULES\_CROPPARM}|spxpagem}

\begin{fulllineitems}
\phantomsection\label{\detokenize{namelists/crop_params.nml:JULES_CROPPARM::remob_io}}
\pysigstartsignatures
\pysigline{\sphinxcode{\sphinxupquote{JULES\_CROPPARM::}}\sphinxbfcode{\sphinxupquote{remob\_io}}}
\pysigstopsignatures\begin{quote}\begin{description}
\sphinxlineitem{Type}
\sphinxAtStartPar
real(ncpft)

\sphinxlineitem{Default}
\sphinxAtStartPar
None

\end{description}\end{quote}

\sphinxAtStartPar
Remobilisation factor. Fraction of stem growth partitioned to RESERVEC.

\end{fulllineitems}

\index{cfrac\_s\_io (in namelist JULES\_CROPPARM)@\spxentry{cfrac\_s\_io}\spxextra{in namelist JULES\_CROPPARM}|spxpagem}

\begin{fulllineitems}
\phantomsection\label{\detokenize{namelists/crop_params.nml:JULES_CROPPARM::cfrac_s_io}}
\pysigstartsignatures
\pysigline{\sphinxcode{\sphinxupquote{JULES\_CROPPARM::}}\sphinxbfcode{\sphinxupquote{cfrac\_s\_io}}}
\pysigstopsignatures\begin{quote}\begin{description}
\sphinxlineitem{Type}
\sphinxAtStartPar
real(npft)

\sphinxlineitem{Default}
\sphinxAtStartPar
None

\end{description}\end{quote}

\sphinxAtStartPar
Carbon fraction of dry matter for stems.

\end{fulllineitems}

\index{cfrac\_r\_io (in namelist JULES\_CROPPARM)@\spxentry{cfrac\_r\_io}\spxextra{in namelist JULES\_CROPPARM}|spxpagem}

\begin{fulllineitems}
\phantomsection\label{\detokenize{namelists/crop_params.nml:JULES_CROPPARM::cfrac_r_io}}
\pysigstartsignatures
\pysigline{\sphinxcode{\sphinxupquote{JULES\_CROPPARM::}}\sphinxbfcode{\sphinxupquote{cfrac\_r\_io}}}
\pysigstopsignatures\begin{quote}\begin{description}
\sphinxlineitem{Type}
\sphinxAtStartPar
real(ncpft)

\sphinxlineitem{Default}
\sphinxAtStartPar
None

\end{description}\end{quote}

\sphinxAtStartPar
Carbon fraction of dry matter for roots.

\end{fulllineitems}

\index{cfrac\_l\_io (in namelist JULES\_CROPPARM)@\spxentry{cfrac\_l\_io}\spxextra{in namelist JULES\_CROPPARM}|spxpagem}

\begin{fulllineitems}
\phantomsection\label{\detokenize{namelists/crop_params.nml:JULES_CROPPARM::cfrac_l_io}}
\pysigstartsignatures
\pysigline{\sphinxcode{\sphinxupquote{JULES\_CROPPARM::}}\sphinxbfcode{\sphinxupquote{cfrac\_l\_io}}}
\pysigstopsignatures\begin{quote}\begin{description}
\sphinxlineitem{Type}
\sphinxAtStartPar
real(ncpft)

\sphinxlineitem{Default}
\sphinxAtStartPar
None

\end{description}\end{quote}

\sphinxAtStartPar
Carbon fraction of dry matter for leaves.

\end{fulllineitems}

\index{allo1\_io (in namelist JULES\_CROPPARM)@\spxentry{allo1\_io}\spxextra{in namelist JULES\_CROPPARM}|spxpagem}

\begin{fulllineitems}
\phantomsection\label{\detokenize{namelists/crop_params.nml:JULES_CROPPARM::allo1_io}}
\pysigstartsignatures
\pysigline{\sphinxcode{\sphinxupquote{JULES\_CROPPARM::}}\sphinxbfcode{\sphinxupquote{allo1\_io}}}
\pysigstopsignatures\begin{quote}\begin{description}
\sphinxlineitem{Type}
\sphinxAtStartPar
real(ncpft)

\sphinxlineitem{Default}
\sphinxAtStartPar
None

\end{description}\end{quote}

\sphinxAtStartPar
Allometric coefficient relating STEMC to CANHT.

\end{fulllineitems}

\index{allo2\_io (in namelist JULES\_CROPPARM)@\spxentry{allo2\_io}\spxextra{in namelist JULES\_CROPPARM}|spxpagem}

\begin{fulllineitems}
\phantomsection\label{\detokenize{namelists/crop_params.nml:JULES_CROPPARM::allo2_io}}
\pysigstartsignatures
\pysigline{\sphinxcode{\sphinxupquote{JULES\_CROPPARM::}}\sphinxbfcode{\sphinxupquote{allo2\_io}}}
\pysigstopsignatures\begin{quote}\begin{description}
\sphinxlineitem{Type}
\sphinxAtStartPar
real(ncpft)

\sphinxlineitem{Default}
\sphinxAtStartPar
None

\end{description}\end{quote}

\sphinxAtStartPar
Allometric coefficient relating STEMC to CANHT.

\end{fulllineitems}

\index{mu\_io (in namelist JULES\_CROPPARM)@\spxentry{mu\_io}\spxextra{in namelist JULES\_CROPPARM}|spxpagem}

\begin{fulllineitems}
\phantomsection\label{\detokenize{namelists/crop_params.nml:JULES_CROPPARM::mu_io}}
\pysigstartsignatures
\pysigline{\sphinxcode{\sphinxupquote{JULES\_CROPPARM::}}\sphinxbfcode{\sphinxupquote{mu\_io}}}
\pysigstopsignatures\begin{quote}\begin{description}
\sphinxlineitem{Type}
\sphinxAtStartPar
real(ncpft)

\sphinxlineitem{Default}
\sphinxAtStartPar
None

\end{description}\end{quote}

\sphinxAtStartPar
Allometric coefficient for calculation of senescence. MIN(mu\_io * (dvi \sphinxhyphen{} sen\_dvi\_io) ** nu\_io, 1.0) is the fraction of leaf carbon that is moved to the harvest pool per day once senescence has started.

\end{fulllineitems}

\index{nu\_io (in namelist JULES\_CROPPARM)@\spxentry{nu\_io}\spxextra{in namelist JULES\_CROPPARM}|spxpagem}

\begin{fulllineitems}
\phantomsection\label{\detokenize{namelists/crop_params.nml:JULES_CROPPARM::nu_io}}
\pysigstartsignatures
\pysigline{\sphinxcode{\sphinxupquote{JULES\_CROPPARM::}}\sphinxbfcode{\sphinxupquote{nu\_io}}}
\pysigstopsignatures\begin{quote}\begin{description}
\sphinxlineitem{Type}
\sphinxAtStartPar
real(ncpft)

\sphinxlineitem{Default}
\sphinxAtStartPar
None

\end{description}\end{quote}

\sphinxAtStartPar
Allometric coefficient for calculation of senescence. See description for {\hyperref[\detokenize{namelists/crop_params.nml:JULES_CROPPARM::mu_io}]{\sphinxcrossref{\sphinxcode{\sphinxupquote{mu\_io}}}}}

\end{fulllineitems}

\index{yield\_frac\_io (in namelist JULES\_CROPPARM)@\spxentry{yield\_frac\_io}\spxextra{in namelist JULES\_CROPPARM}|spxpagem}

\begin{fulllineitems}
\phantomsection\label{\detokenize{namelists/crop_params.nml:JULES_CROPPARM::yield_frac_io}}
\pysigstartsignatures
\pysigline{\sphinxcode{\sphinxupquote{JULES\_CROPPARM::}}\sphinxbfcode{\sphinxupquote{yield\_frac\_io}}}
\pysigstopsignatures\begin{quote}\begin{description}
\sphinxlineitem{Type}
\sphinxAtStartPar
real(ncpft)

\sphinxlineitem{Default}
\sphinxAtStartPar
None

\end{description}\end{quote}

\sphinxAtStartPar
Fraction of the harvest carbon pool converted to yield carbon (yield is the economically valuable component of the harvest pool e.g. kernel).

\end{fulllineitems}

\index{initial\_carbon\_io (in namelist JULES\_CROPPARM)@\spxentry{initial\_carbon\_io}\spxextra{in namelist JULES\_CROPPARM}|spxpagem}

\begin{fulllineitems}
\phantomsection\label{\detokenize{namelists/crop_params.nml:JULES_CROPPARM::initial_carbon_io}}
\pysigstartsignatures
\pysigline{\sphinxcode{\sphinxupquote{JULES\_CROPPARM::}}\sphinxbfcode{\sphinxupquote{initial\_carbon\_io}}}
\pysigstopsignatures\begin{quote}\begin{description}
\sphinxlineitem{Type}
\sphinxAtStartPar
real(ncpft)

\sphinxlineitem{Default}
\sphinxAtStartPar
None

\end{description}\end{quote}

\sphinxAtStartPar
Carbon in crop at emergence in kgC/m2.

\end{fulllineitems}

\index{initial\_c\_dvi\_io (in namelist JULES\_CROPPARM)@\spxentry{initial\_c\_dvi\_io}\spxextra{in namelist JULES\_CROPPARM}|spxpagem}

\begin{fulllineitems}
\phantomsection\label{\detokenize{namelists/crop_params.nml:JULES_CROPPARM::initial_c_dvi_io}}
\pysigstartsignatures
\pysigline{\sphinxcode{\sphinxupquote{JULES\_CROPPARM::}}\sphinxbfcode{\sphinxupquote{initial\_c\_dvi\_io}}}
\pysigstopsignatures\begin{quote}\begin{description}
\sphinxlineitem{Type}
\sphinxAtStartPar
real(ncpft)

\sphinxlineitem{Default}
\sphinxAtStartPar
None

\end{description}\end{quote}

\sphinxAtStartPar
DVI at which the crop carbon is set to {\hyperref[\detokenize{namelists/crop_params.nml:JULES_CROPPARM::initial_carbon_io}]{\sphinxcrossref{\sphinxcode{\sphinxupquote{initial\_carbon\_io}}}}}. Should be at emergence (0.0) or shortly after.

\end{fulllineitems}

\index{sen\_dvi\_io (in namelist JULES\_CROPPARM)@\spxentry{sen\_dvi\_io}\spxextra{in namelist JULES\_CROPPARM}|spxpagem}

\begin{fulllineitems}
\phantomsection\label{\detokenize{namelists/crop_params.nml:JULES_CROPPARM::sen_dvi_io}}
\pysigstartsignatures
\pysigline{\sphinxcode{\sphinxupquote{JULES\_CROPPARM::}}\sphinxbfcode{\sphinxupquote{sen\_dvi\_io}}}
\pysigstopsignatures\begin{quote}\begin{description}
\sphinxlineitem{Type}
\sphinxAtStartPar
real(ncpft)

\sphinxlineitem{Default}
\sphinxAtStartPar
None

\end{description}\end{quote}

\sphinxAtStartPar
DVI at which leaf senescence begins.

\end{fulllineitems}

\index{t\_mort\_io (in namelist JULES\_CROPPARM)@\spxentry{t\_mort\_io}\spxextra{in namelist JULES\_CROPPARM}|spxpagem}

\begin{fulllineitems}
\phantomsection\label{\detokenize{namelists/crop_params.nml:JULES_CROPPARM::t_mort_io}}
\pysigstartsignatures
\pysigline{\sphinxcode{\sphinxupquote{JULES\_CROPPARM::}}\sphinxbfcode{\sphinxupquote{t\_mort\_io}}}
\pysigstopsignatures\begin{quote}\begin{description}
\sphinxlineitem{Type}
\sphinxAtStartPar
real(ncpft)

\sphinxlineitem{Default}
\sphinxAtStartPar
None

\end{description}\end{quote}

\sphinxAtStartPar
Soil temperature (second level) at which to kill crop if DVI\textgreater{}1.

\end{fulllineitems}


\sphinxstepscope


\section{\sphinxstyleliteralintitle{\sphinxupquote{triffid\_params.nml}}}
\label{\detokenize{namelists/triffid_params.nml:triffid-params-nml}}\label{\detokenize{namelists/triffid_params.nml::doc}}
\sphinxAtStartPar
This file contains a single namelist called {\hyperref[\detokenize{namelists/triffid_params.nml:namelist-JULES_TRIFFID}]{\sphinxcrossref{\sphinxcode{\sphinxupquote{JULES\_TRIFFID}}}}} that sets parameters relevant to the TRIFFID submodel.


\subsection{\sphinxstyleliteralintitle{\sphinxupquote{JULES\_TRIFFID}} namelist members}
\label{\detokenize{namelists/triffid_params.nml:namelist-JULES_TRIFFID}}\label{\detokenize{namelists/triffid_params.nml:jules-triffid-namelist-members}}\index{JULES\_TRIFFID (namelist)@\spxentry{JULES\_TRIFFID}\spxextra{namelist}|spxpagem}
\sphinxAtStartPar
This namelist is used to read PFT parameters that are only needed by the dynamic vegetation model (TRIFFID). Values are not used if TRIFFID is not selected.

\begin{sphinxadmonition}{note}{Note:}
\sphinxAtStartPar
Where a quantity is said to have units of “/360days”, this means that it is an amount per 360 days.
\end{sphinxadmonition}

\begin{sphinxadmonition}{note}{Note:}
\sphinxAtStartPar
If the crop model is on (i.e. {\hyperref[\detokenize{namelists/jules_surface_types.nml:JULES_SURFACE_TYPES::ncpft}]{\sphinxcrossref{\sphinxcode{\sphinxupquote{ncpft}}}}} \textgreater{} 0), only \sphinxcode{\sphinxupquote{nnpft = npft \sphinxhyphen{} ncpft}} values will be used for each variable.
\end{sphinxadmonition}
\index{crop\_io (in namelist JULES\_TRIFFID)@\spxentry{crop\_io}\spxextra{in namelist JULES\_TRIFFID}|spxpagem}

\begin{fulllineitems}
\phantomsection\label{\detokenize{namelists/triffid_params.nml:JULES_TRIFFID::crop_io}}
\pysigstartsignatures
\pysigline{\sphinxcode{\sphinxupquote{JULES\_TRIFFID::}}\sphinxbfcode{\sphinxupquote{crop\_io}}}
\pysigstopsignatures\begin{quote}\begin{description}
\sphinxlineitem{Type}
\sphinxAtStartPar
integer(npft)

\sphinxlineitem{Permitted}
\sphinxAtStartPar
0,1,2,3

\sphinxlineitem{Default}
\sphinxAtStartPar
None

\end{description}\end{quote}

\sphinxAtStartPar
Flag indicating whether the PFT is natural, crop, or pasture. Only
crop / pasture PFTs are allowed to grow in the agricultural
area. See {\hyperref[\detokenize{namelists/jules_vegetation.nml:JULES_VEGETATION::l_trif_crop}]{\sphinxcrossref{\sphinxcode{\sphinxupquote{l\_trif\_crop}}}}} for more
details.

\begin{DUlineblock}{0em}
\item[] If {\hyperref[\detokenize{namelists/jules_vegetation.nml:JULES_VEGETATION::l_trif_crop}]{\sphinxcrossref{\sphinxcode{\sphinxupquote{l\_trif\_crop}}}}} is FALSE permitted values of \sphinxcode{\sphinxupquote{crop\_io}} are 0 and 1.
\end{DUlineblock}
\begin{enumerate}
\sphinxsetlistlabels{\arabic}{enumi}{enumii}{}{.}%
\setcounter{enumi}{-1}
\item {} 
\sphinxAtStartPar
Natural vegetation (not a crop).

\item {} 
\sphinxAtStartPar
A crop.

\end{enumerate}

\begin{DUlineblock}{0em}
\item[] If {\hyperref[\detokenize{namelists/jules_vegetation.nml:JULES_VEGETATION::l_trif_crop}]{\sphinxcrossref{\sphinxcode{\sphinxupquote{l\_trif\_crop}}}}} is TRUE permitted values of \sphinxcode{\sphinxupquote{crop\_io}} are 0, 1 and 2.
\end{DUlineblock}
\begin{enumerate}
\sphinxsetlistlabels{\arabic}{enumi}{enumii}{}{.}%
\setcounter{enumi}{-1}
\item {} 
\sphinxAtStartPar
Natural vegetation (neither crop nor pasture).

\item {} 
\sphinxAtStartPar
Crop.

\item {} 
\sphinxAtStartPar
Pasture.

\end{enumerate}

\begin{DUlineblock}{0em}
\item[] If {\hyperref[\detokenize{namelists/jules_vegetation.nml:JULES_VEGETATION::l_trif_biocrop}]{\sphinxcrossref{\sphinxcode{\sphinxupquote{l\_trif\_biocrop}}}}} is TRUE permitted values are 0, 1, 2 and 3. Flag indicating whether the PFT is natural, crop, pasture, or bioenergy. See {\hyperref[\detokenize{namelists/jules_vegetation.nml:JULES_VEGETATION::l_trif_biocrop}]{\sphinxcrossref{\sphinxcode{\sphinxupquote{l\_trif\_biocrop}}}}} for more details.
\end{DUlineblock}
\begin{enumerate}
\sphinxsetlistlabels{\arabic}{enumi}{enumii}{}{.}%
\setcounter{enumi}{-1}
\item {} 
\sphinxAtStartPar
Natural vegetation.

\item {} 
\sphinxAtStartPar
Crop.

\item {} 
\sphinxAtStartPar
Pasture.

\item {} 
\sphinxAtStartPar
Bioenergy crops or trees.

\end{enumerate}

\end{fulllineitems}

\index{g\_area\_io (in namelist JULES\_TRIFFID)@\spxentry{g\_area\_io}\spxextra{in namelist JULES\_TRIFFID}|spxpagem}

\begin{fulllineitems}
\phantomsection\label{\detokenize{namelists/triffid_params.nml:JULES_TRIFFID::g_area_io}}
\pysigstartsignatures
\pysigline{\sphinxcode{\sphinxupquote{JULES\_TRIFFID::}}\sphinxbfcode{\sphinxupquote{g\_area\_io}}}
\pysigstopsignatures\begin{quote}\begin{description}
\sphinxlineitem{Type}
\sphinxAtStartPar
real(npft)

\sphinxlineitem{Default}
\sphinxAtStartPar
None

\end{description}\end{quote}

\sphinxAtStartPar
Disturbance rate (/360days).

\end{fulllineitems}

\index{g\_grow\_io (in namelist JULES\_TRIFFID)@\spxentry{g\_grow\_io}\spxextra{in namelist JULES\_TRIFFID}|spxpagem}

\begin{fulllineitems}
\phantomsection\label{\detokenize{namelists/triffid_params.nml:JULES_TRIFFID::g_grow_io}}
\pysigstartsignatures
\pysigline{\sphinxcode{\sphinxupquote{JULES\_TRIFFID::}}\sphinxbfcode{\sphinxupquote{g\_grow\_io}}}
\pysigstopsignatures\begin{quote}\begin{description}
\sphinxlineitem{Type}
\sphinxAtStartPar
real(npft)

\sphinxlineitem{Default}
\sphinxAtStartPar
None

\end{description}\end{quote}

\sphinxAtStartPar
Rate of leaf growth (/360days).

\end{fulllineitems}

\index{g\_root\_io (in namelist JULES\_TRIFFID)@\spxentry{g\_root\_io}\spxextra{in namelist JULES\_TRIFFID}|spxpagem}

\begin{fulllineitems}
\phantomsection\label{\detokenize{namelists/triffid_params.nml:JULES_TRIFFID::g_root_io}}
\pysigstartsignatures
\pysigline{\sphinxcode{\sphinxupquote{JULES\_TRIFFID::}}\sphinxbfcode{\sphinxupquote{g\_root\_io}}}
\pysigstopsignatures\begin{quote}\begin{description}
\sphinxlineitem{Type}
\sphinxAtStartPar
real(npft)

\sphinxlineitem{Default}
\sphinxAtStartPar
None

\end{description}\end{quote}

\sphinxAtStartPar
Turnover rate for root biomass (/360days).

\end{fulllineitems}

\index{g\_wood\_io (in namelist JULES\_TRIFFID)@\spxentry{g\_wood\_io}\spxextra{in namelist JULES\_TRIFFID}|spxpagem}

\begin{fulllineitems}
\phantomsection\label{\detokenize{namelists/triffid_params.nml:JULES_TRIFFID::g_wood_io}}
\pysigstartsignatures
\pysigline{\sphinxcode{\sphinxupquote{JULES\_TRIFFID::}}\sphinxbfcode{\sphinxupquote{g\_wood\_io}}}
\pysigstopsignatures\begin{quote}\begin{description}
\sphinxlineitem{Type}
\sphinxAtStartPar
real(npft)

\sphinxlineitem{Default}
\sphinxAtStartPar
None

\end{description}\end{quote}

\sphinxAtStartPar
Turnover rate for woody biomass (/360days).

\end{fulllineitems}

\index{lai\_max\_io (in namelist JULES\_TRIFFID)@\spxentry{lai\_max\_io}\spxextra{in namelist JULES\_TRIFFID}|spxpagem}

\begin{fulllineitems}
\phantomsection\label{\detokenize{namelists/triffid_params.nml:JULES_TRIFFID::lai_max_io}}
\pysigstartsignatures
\pysigline{\sphinxcode{\sphinxupquote{JULES\_TRIFFID::}}\sphinxbfcode{\sphinxupquote{lai\_max\_io}}}
\pysigstopsignatures\begin{quote}\begin{description}
\sphinxlineitem{Type}
\sphinxAtStartPar
real(npft)

\sphinxlineitem{Default}
\sphinxAtStartPar
None

\end{description}\end{quote}

\sphinxAtStartPar
Maximum LAI.

\end{fulllineitems}

\index{lai\_min\_io (in namelist JULES\_TRIFFID)@\spxentry{lai\_min\_io}\spxextra{in namelist JULES\_TRIFFID}|spxpagem}

\begin{fulllineitems}
\phantomsection\label{\detokenize{namelists/triffid_params.nml:JULES_TRIFFID::lai_min_io}}
\pysigstartsignatures
\pysigline{\sphinxcode{\sphinxupquote{JULES\_TRIFFID::}}\sphinxbfcode{\sphinxupquote{lai\_min\_io}}}
\pysigstopsignatures\begin{quote}\begin{description}
\sphinxlineitem{Type}
\sphinxAtStartPar
real(npft)

\sphinxlineitem{Default}
\sphinxAtStartPar
None

\end{description}\end{quote}

\sphinxAtStartPar
Minimum LAI.

\end{fulllineitems}

\index{alloc\_fast\_io (in namelist JULES\_TRIFFID)@\spxentry{alloc\_fast\_io}\spxextra{in namelist JULES\_TRIFFID}|spxpagem}

\begin{fulllineitems}
\phantomsection\label{\detokenize{namelists/triffid_params.nml:JULES_TRIFFID::alloc_fast_io}}
\pysigstartsignatures
\pysigline{\sphinxcode{\sphinxupquote{JULES\_TRIFFID::}}\sphinxbfcode{\sphinxupquote{alloc\_fast\_io}}}
\pysigstopsignatures\begin{quote}\begin{description}
\sphinxlineitem{Type}
\sphinxAtStartPar
real(npft)

\sphinxlineitem{Default}
\sphinxAtStartPar
None

\end{description}\end{quote}

\sphinxAtStartPar
Fraction of the carbon flux from vegetation to wood products to add to the rapidly decaying wood products pool (wood\_prod\_fast).

\end{fulllineitems}

\index{alloc\_med\_io (in namelist JULES\_TRIFFID)@\spxentry{alloc\_med\_io}\spxextra{in namelist JULES\_TRIFFID}|spxpagem}

\begin{fulllineitems}
\phantomsection\label{\detokenize{namelists/triffid_params.nml:JULES_TRIFFID::alloc_med_io}}
\pysigstartsignatures
\pysigline{\sphinxcode{\sphinxupquote{JULES\_TRIFFID::}}\sphinxbfcode{\sphinxupquote{alloc\_med\_io}}}
\pysigstopsignatures\begin{quote}\begin{description}
\sphinxlineitem{Type}
\sphinxAtStartPar
real(npft)

\sphinxlineitem{Default}
\sphinxAtStartPar
None

\end{description}\end{quote}

\sphinxAtStartPar
Fraction of the carbon flux from vegetation to wood products to add to the wood products pool with a moderate decay rate (wood\_prod\_med).

\end{fulllineitems}

\index{alloc\_slow\_io (in namelist JULES\_TRIFFID)@\spxentry{alloc\_slow\_io}\spxextra{in namelist JULES\_TRIFFID}|spxpagem}

\begin{fulllineitems}
\phantomsection\label{\detokenize{namelists/triffid_params.nml:JULES_TRIFFID::alloc_slow_io}}
\pysigstartsignatures
\pysigline{\sphinxcode{\sphinxupquote{JULES\_TRIFFID::}}\sphinxbfcode{\sphinxupquote{alloc\_slow\_io}}}
\pysigstopsignatures\begin{quote}\begin{description}
\sphinxlineitem{Type}
\sphinxAtStartPar
real(npft)

\sphinxlineitem{Default}
\sphinxAtStartPar
None

\end{description}\end{quote}

\sphinxAtStartPar
Fraction of the carbon flux from vegetation to wood products to add to the slowly decaying wood products pool (wood\_prod\_slow).

\end{fulllineitems}

\index{retran\_l\_io (in namelist JULES\_TRIFFID)@\spxentry{retran\_l\_io}\spxextra{in namelist JULES\_TRIFFID}|spxpagem}

\begin{fulllineitems}
\phantomsection\label{\detokenize{namelists/triffid_params.nml:JULES_TRIFFID::retran_l_io}}
\pysigstartsignatures
\pysigline{\sphinxcode{\sphinxupquote{JULES\_TRIFFID::}}\sphinxbfcode{\sphinxupquote{retran\_l\_io}}}
\pysigstopsignatures\begin{quote}\begin{description}
\sphinxlineitem{Type}
\sphinxAtStartPar
real(npft)

\sphinxlineitem{Default}
\sphinxAtStartPar
0.5

\end{description}\end{quote}

\sphinxAtStartPar
Fraction of retranslocated leaf N.

\end{fulllineitems}

\index{retran\_r\_io (in namelist JULES\_TRIFFID)@\spxentry{retran\_r\_io}\spxextra{in namelist JULES\_TRIFFID}|spxpagem}

\begin{fulllineitems}
\phantomsection\label{\detokenize{namelists/triffid_params.nml:JULES_TRIFFID::retran_r_io}}
\pysigstartsignatures
\pysigline{\sphinxcode{\sphinxupquote{JULES\_TRIFFID::}}\sphinxbfcode{\sphinxupquote{retran\_r\_io}}}
\pysigstopsignatures\begin{quote}\begin{description}
\sphinxlineitem{Type}
\sphinxAtStartPar
real(npft)

\sphinxlineitem{Default}
\sphinxAtStartPar
0.2

\end{description}\end{quote}

\sphinxAtStartPar
Fraction of retranslocated root N.

\end{fulllineitems}

\index{ag\_expand\_io (in namelist JULES\_TRIFFID)@\spxentry{ag\_expand\_io}\spxextra{in namelist JULES\_TRIFFID}|spxpagem}

\begin{fulllineitems}
\phantomsection\label{\detokenize{namelists/triffid_params.nml:JULES_TRIFFID::ag_expand_io}}
\pysigstartsignatures
\pysigline{\sphinxcode{\sphinxupquote{JULES\_TRIFFID::}}\sphinxbfcode{\sphinxupquote{ag\_expand\_io}}}
\pysigstopsignatures\begin{quote}\begin{description}
\sphinxlineitem{Type}
\sphinxAtStartPar
integer(npft)

\sphinxlineitem{Permitted}
\sphinxAtStartPar
0,1

\sphinxlineitem{Default}
\sphinxAtStartPar
0

\end{description}\end{quote}

\sphinxAtStartPar
Only used if {\hyperref[\detokenize{namelists/jules_vegetation.nml:JULES_VEGETATION::l_ag_expand}]{\sphinxcrossref{\sphinxcode{\sphinxupquote{l\_ag\_expand}}}}} = TRUE.
\begin{enumerate}
\sphinxsetlistlabels{\arabic}{enumi}{enumii}{}{.}%
\setcounter{enumi}{-1}
\item {} 
\sphinxAtStartPar
No automatic expansion of PFT area when the agricultural area increases.

\item {} 
\sphinxAtStartPar
Automatically plant out new crop areas with the selected PFT.

\end{enumerate}

\end{fulllineitems}


\begin{sphinxadmonition}{note}{Only used when \sphinxstyleliteralintitle{\sphinxupquote{l\_trif\_biocrop}} = TRUE}
\index{harvest\_type\_io (in namelist JULES\_TRIFFID)@\spxentry{harvest\_type\_io}\spxextra{in namelist JULES\_TRIFFID}|spxpagem}

\begin{fulllineitems}
\phantomsection\label{\detokenize{namelists/triffid_params.nml:JULES_TRIFFID::harvest_type_io}}
\pysigstartsignatures
\pysigline{\sphinxcode{\sphinxupquote{JULES\_TRIFFID::}}\sphinxbfcode{\sphinxupquote{harvest\_type\_io}}}
\pysigstopsignatures\begin{quote}\begin{description}
\sphinxlineitem{Type}
\sphinxAtStartPar
integer(npft)

\sphinxlineitem{Permitted}
\sphinxAtStartPar
0,1,2

\sphinxlineitem{Default}
\sphinxAtStartPar
0

\end{description}\end{quote}
\begin{enumerate}
\sphinxsetlistlabels{\arabic}{enumi}{enumii}{}{.}%
\setcounter{enumi}{-1}
\item {} 
\sphinxAtStartPar
No harvest (default).

\item {} 
\sphinxAtStartPar
Continuous harvest from litter, as per {\hyperref[\detokenize{namelists/jules_vegetation.nml:JULES_VEGETATION::l_trif_crop}]{\sphinxcrossref{\sphinxcode{\sphinxupquote{l\_trif\_crop}}}}}.

\item {} 
\sphinxAtStartPar
Periodic harvesting.

\end{enumerate}

\begin{sphinxadmonition}{note}{Note:}
\sphinxAtStartPar
For “natural” PFTs ({\hyperref[\detokenize{namelists/triffid_params.nml:JULES_TRIFFID::crop_io}]{\sphinxcrossref{\sphinxcode{\sphinxupquote{crop\_io}}}}} = 0), this must be set to 0. For agricultural PFTs, this can be set to 0, 1 or 2.
\end{sphinxadmonition}

\end{fulllineitems}


\begin{sphinxadmonition}{note}{Only used when \sphinxstyleliteralintitle{\sphinxupquote{harvest\_type\_io}} = 2}

\sphinxAtStartPar
A placeholder value must be used for all other PFTs.
\index{harvest\_freq\_io (in namelist JULES\_TRIFFID)@\spxentry{harvest\_freq\_io}\spxextra{in namelist JULES\_TRIFFID}|spxpagem}

\begin{fulllineitems}
\phantomsection\label{\detokenize{namelists/triffid_params.nml:JULES_TRIFFID::harvest_freq_io}}
\pysigstartsignatures
\pysigline{\sphinxcode{\sphinxupquote{JULES\_TRIFFID::}}\sphinxbfcode{\sphinxupquote{harvest\_freq\_io}}}
\pysigstopsignatures\begin{quote}\begin{description}
\sphinxlineitem{Type}
\sphinxAtStartPar
integer(npft)

\sphinxlineitem{Permitted}
\sphinxAtStartPar
\textgreater{}= 0

\sphinxlineitem{Default}
\sphinxAtStartPar
none

\end{description}\end{quote}

\sphinxAtStartPar
Harvest freqency in years.

\end{fulllineitems}

\index{harvest\_ht\_io (in namelist JULES\_TRIFFID)@\spxentry{harvest\_ht\_io}\spxextra{in namelist JULES\_TRIFFID}|spxpagem}

\begin{fulllineitems}
\phantomsection\label{\detokenize{namelists/triffid_params.nml:JULES_TRIFFID::harvest_ht_io}}
\pysigstartsignatures
\pysigline{\sphinxcode{\sphinxupquote{JULES\_TRIFFID::}}\sphinxbfcode{\sphinxupquote{harvest\_ht\_io}}}
\pysigstopsignatures\begin{quote}\begin{description}
\sphinxlineitem{Type}
\sphinxAtStartPar
real(npft)

\sphinxlineitem{Permitted}
\sphinxAtStartPar
\textgreater{} 0

\sphinxlineitem{Default}
\sphinxAtStartPar
none

\end{description}\end{quote}

\sphinxAtStartPar
Height {[}m{]} to which the PFT is reduced at each harvest cycle.

\end{fulllineitems}


\begin{sphinxadmonition}{note}{Note:}
\sphinxAtStartPar
{\hyperref[\detokenize{namelists/triffid_params.nml:JULES_TRIFFID::lai_min_io}]{\sphinxcrossref{\sphinxcode{\sphinxupquote{lai\_min\_io}}}}} must be set such that PFT height at {\hyperref[\detokenize{namelists/triffid_params.nml:JULES_TRIFFID::lai_min_io}]{\sphinxcrossref{\sphinxcode{\sphinxupquote{lai\_min\_io}}}}} \textless{}= {\hyperref[\detokenize{namelists/triffid_params.nml:JULES_TRIFFID::harvest_ht_io}]{\sphinxcrossref{\sphinxcode{\sphinxupquote{harvest\_ht\_io}}}}}, otherwise JULES will not start (the required value will be shown in error output).
\end{sphinxadmonition}
\end{sphinxadmonition}
\end{sphinxadmonition}

\sphinxstepscope


\section{\sphinxstyleliteralintitle{\sphinxupquote{urban.nml}}}
\label{\detokenize{namelists/urban.nml:urban-nml}}\label{\detokenize{namelists/urban.nml::doc}}
\sphinxAtStartPar
This file contains one namelist called {\hyperref[\detokenize{namelists/urban.nml:namelist-JULES_URBAN}]{\sphinxcrossref{\sphinxcode{\sphinxupquote{JULES\_URBAN}}}}}.

\sphinxAtStartPar
This section predominantly sets the options available for the two\sphinxhyphen{}tile urban scheme MORUSES. These namelists are only read if {\hyperref[\detokenize{namelists/jules_surface.nml:JULES_SURFACE::l_urban2t}]{\sphinxcrossref{\sphinxcode{\sphinxupquote{l\_urban2t}}}}}, which requires both the {\hyperref[\detokenize{namelists/jules_surface_types.nml:JULES_SURFACE_TYPES::urban_canyon}]{\sphinxcrossref{\sphinxcode{\sphinxupquote{urban\_canyon}}}}} and {\hyperref[\detokenize{namelists/jules_surface_types.nml:JULES_SURFACE_TYPES::urban_roof}]{\sphinxcrossref{\sphinxcode{\sphinxupquote{urban\_roof}}}}} surface types to be used. MORUSES provides parameters for: snow free canyon albedo ({\hyperref[\detokenize{namelists/urban.nml:JULES_URBAN::l_moruses_albedo}]{\sphinxcrossref{\sphinxcode{\sphinxupquote{l\_moruses\_albedo}}}}}), canyon emissivity ({\hyperref[\detokenize{namelists/urban.nml:JULES_URBAN::l_moruses_emissivity}]{\sphinxcrossref{\sphinxcode{\sphinxupquote{l\_moruses\_emissivity}}}}}), roughness length for heat ({\hyperref[\detokenize{namelists/urban.nml:JULES_URBAN::l_moruses_rough}]{\sphinxcrossref{\sphinxcode{\sphinxupquote{l\_moruses\_rough}}}}}), roughness length for momentum ({\hyperref[\detokenize{namelists/urban.nml:JULES_URBAN::l_moruses_macdonald}]{\sphinxcrossref{\sphinxcode{\sphinxupquote{l\_moruses\_macdonald}}}}}) and thermal inertia ({\hyperref[\detokenize{namelists/urban.nml:JULES_URBAN::l_moruses_storage}]{\sphinxcrossref{\sphinxcode{\sphinxupquote{l\_moruses\_storage}}}}}). Ancillary data, predominantly required for MORUSES, is read in via {\hyperref[\detokenize{namelists/ancillaries.nml:namelist-URBAN_PROPERTIES}]{\sphinxcrossref{\sphinxcode{\sphinxupquote{URBAN\_PROPERTIES}}}}}.

\sphinxAtStartPar
For all other parameters that MORUSES does not provide, and for any MORUSES parametrisations that are turned off, values from {\hyperref[\detokenize{namelists/nveg_params.nml::doc}]{\sphinxcrossref{\DUrole{doc}{nveg\_params.nml}}}} will be used instead. See the switches below for more information.
\begin{quote}


\sphinxstrong{See also:}
\nopagebreak


\sphinxAtStartPar
References:
\begin{itemize}
\item {} 
\sphinxAtStartPar
Porson, A., et al. (2010), Implementation of a new urban energy budget scheme in the MetUM. Part I: Description and idealized simulations, Quarterly Journal of the Royal Meteorological Society, 136: 1514\sphinxhyphen{}1529. doi: 10.1002/qj.668

\item {} 
\sphinxAtStartPar
Porson, A., et al. (2010), Implementation of a new urban energy budget scheme into MetUM. Part II: Validation against observations and model Intercomparison, Quarterly Journal of the Royal Meteorological Society, 136: 1530\sphinxhyphen{}1542. doi: 10.1002/qj.572

\end{itemize}


\end{quote}


\subsection{\sphinxstyleliteralintitle{\sphinxupquote{JULES\_URBAN}} namelist members}
\label{\detokenize{namelists/urban.nml:namelist-JULES_URBAN}}\label{\detokenize{namelists/urban.nml:jules-urban-namelist-members}}\index{JULES\_URBAN (namelist)@\spxentry{JULES\_URBAN}\spxextra{namelist}|spxpagem}\index{anthrop\_heat\_scale (in namelist JULES\_URBAN)@\spxentry{anthrop\_heat\_scale}\spxextra{in namelist JULES\_URBAN}|spxpagem}

\begin{fulllineitems}
\phantomsection\label{\detokenize{namelists/urban.nml:JULES_URBAN::anthrop_heat_scale}}
\pysigstartsignatures
\pysigline{\sphinxcode{\sphinxupquote{JULES\_URBAN::}}\sphinxbfcode{\sphinxupquote{anthrop\_heat\_scale}}}
\pysigstopsignatures\begin{quote}\begin{description}
\sphinxlineitem{Type}
\sphinxAtStartPar
real

\sphinxlineitem{Default}
\sphinxAtStartPar
1.0

\end{description}\end{quote}

\sphinxAtStartPar
Distribution scaling factor, which allows the anthropogenic heat flux to be spread between the {\hyperref[\detokenize{namelists/jules_surface_types.nml:JULES_SURFACE_TYPES::urban_canyon}]{\sphinxcrossref{\sphinxcode{\sphinxupquote{urban\_canyon}}}}} and {\hyperref[\detokenize{namelists/jules_surface_types.nml:JULES_SURFACE_TYPES::urban_roof}]{\sphinxcrossref{\sphinxcode{\sphinxupquote{urban\_roof}}}}} surface tiles such that:
\begin{itemize}
\item {} 
\sphinxAtStartPar
\sphinxcode{\sphinxupquote{H\_roof = anthrop\_heat\_scale x H\_canyon}}

\item {} 
\sphinxAtStartPar
\sphinxcode{\sphinxupquote{H\_canyon x (W/R) + H\_roof x ( 1.0 \sphinxhyphen{} W/R ) = anthrop\_heat}}

\end{itemize}

\sphinxAtStartPar
Has a value between 0.0 and 1.0 where the extremes correspond to:
\begin{itemize}
\item {} 
\sphinxAtStartPar
0.0 = all released within the canyon.

\item {} 
\sphinxAtStartPar
1.0 = evenly spread between canyon and roof.

\end{itemize}

\sphinxAtStartPar
Only used if {\hyperref[\detokenize{namelists/jules_surface.nml:JULES_SURFACE::l_anthrop_heat_src}]{\sphinxcrossref{\sphinxcode{\sphinxupquote{l\_anthrop\_heat\_src}}}}} = TRUE.

\end{fulllineitems}

\index{l\_moruses\_albedo (in namelist JULES\_URBAN)@\spxentry{l\_moruses\_albedo}\spxextra{in namelist JULES\_URBAN}|spxpagem}

\begin{fulllineitems}
\phantomsection\label{\detokenize{namelists/urban.nml:JULES_URBAN::l_moruses_albedo}}
\pysigstartsignatures
\pysigline{\sphinxcode{\sphinxupquote{JULES\_URBAN::}}\sphinxbfcode{\sphinxupquote{l\_moruses\_albedo}}}
\pysigstopsignatures\begin{quote}\begin{description}
\sphinxlineitem{Type}
\sphinxAtStartPar
logical

\sphinxlineitem{Default}
\sphinxAtStartPar
F

\end{description}\end{quote}

\sphinxAtStartPar
MORUSES switch for effective canyon albedo parameterisation (snow free).

\sphinxAtStartPar
Shortwave radiative exchange in the form of an effective canyon albedo, including shading and multiple reflections, which depends on building materials, geometry and zenith angle.
\begin{description}
\sphinxlineitem{TRUE}
\sphinxAtStartPar
Use MORUSES parameterisation. Requires that {\hyperref[\detokenize{namelists/jules_radiation.nml:JULES_RADIATION::l_cosz}]{\sphinxcrossref{\sphinxcode{\sphinxupquote{l\_cosz}}}}} = TRUE. Also, check whether the data are provided in UTC or local solar time. To assume local solar time set {\hyperref[\detokenize{namelists/timesteps.nml:JULES_TIME::l_local_solar_time}]{\sphinxcrossref{\sphinxcode{\sphinxupquote{l\_local\_solar\_time}}}}} = TRUE.

\sphinxlineitem{FALSE}
\sphinxAtStartPar
The snow free canyon albedo is taken from {\hyperref[\detokenize{namelists/nveg_params.nml:JULES_NVEGPARM::albsnf_nvg_io}]{\sphinxcrossref{\sphinxcode{\sphinxupquote{albsnf\_nvg\_io}}}}}.

\end{description}

\sphinxAtStartPar
In all cases the snow covered albedo is {\hyperref[\detokenize{namelists/nveg_params.nml:JULES_NVEGPARM::albsnc_nvg_io}]{\sphinxcrossref{\sphinxcode{\sphinxupquote{albsnc\_nvg\_io}}}}}. MORUSES does not parameterise the roof albedo, so this is also taken from {\hyperref[\detokenize{namelists/nveg_params.nml:JULES_NVEGPARM::albsnf_nvg_io}]{\sphinxcrossref{\sphinxcode{\sphinxupquote{albsnf\_nvg\_io}}}}}.

\end{fulllineitems}

\index{l\_moruses\_emissivity (in namelist JULES\_URBAN)@\spxentry{l\_moruses\_emissivity}\spxextra{in namelist JULES\_URBAN}|spxpagem}

\begin{fulllineitems}
\phantomsection\label{\detokenize{namelists/urban.nml:JULES_URBAN::l_moruses_emissivity}}
\pysigstartsignatures
\pysigline{\sphinxcode{\sphinxupquote{JULES\_URBAN::}}\sphinxbfcode{\sphinxupquote{l\_moruses\_emissivity}}}
\pysigstopsignatures\begin{quote}\begin{description}
\sphinxlineitem{Type}
\sphinxAtStartPar
logical

\sphinxlineitem{Default}
\sphinxAtStartPar
F

\end{description}\end{quote}

\sphinxAtStartPar
MORUSES switch for effective canyon emissivity parameterisation.

\sphinxAtStartPar
Long\sphinxhyphen{}wave radiative exchange in the form of an effective canyon emissivity, including multiple reflections, which depends on building materials and geometry.
\begin{description}
\sphinxlineitem{TRUE}
\sphinxAtStartPar
Use MORUSES parameterisation.

\sphinxlineitem{FALSE}
\sphinxAtStartPar
The canyon emissivity is taken from {\hyperref[\detokenize{namelists/nveg_params.nml:JULES_NVEGPARM::emis_nvg_io}]{\sphinxcrossref{\sphinxcode{\sphinxupquote{emis\_nvg\_io}}}}}.

\end{description}

\sphinxAtStartPar
In either case, the roof emissivity is taken from {\hyperref[\detokenize{namelists/nveg_params.nml:JULES_NVEGPARM::emis_nvg_io}]{\sphinxcrossref{\sphinxcode{\sphinxupquote{emis\_nvg\_io}}}}}.

\end{fulllineitems}

\index{l\_moruses\_rough (in namelist JULES\_URBAN)@\spxentry{l\_moruses\_rough}\spxextra{in namelist JULES\_URBAN}|spxpagem}

\begin{fulllineitems}
\phantomsection\label{\detokenize{namelists/urban.nml:JULES_URBAN::l_moruses_rough}}
\pysigstartsignatures
\pysigline{\sphinxcode{\sphinxupquote{JULES\_URBAN::}}\sphinxbfcode{\sphinxupquote{l\_moruses\_rough}}}
\pysigstopsignatures\begin{quote}\begin{description}
\sphinxlineitem{Type}
\sphinxAtStartPar
logical

\sphinxlineitem{Default}
\sphinxAtStartPar
F

\end{description}\end{quote}

\sphinxAtStartPar
MORUSES switch for effective roughness length for heat parameterisation.

\sphinxAtStartPar
The effective roughness length for heat has a physical basis and is not calculated as a fraction of momentum. It depends on the geometry of the canyon, which affects the recirculation of the jet within the canyon. Flow within the canyon can be broken down into two regions; the recirculation and ventilation regions, where the recirculation region forms in the wake of each building. Three different flow regimes are represented:
\begin{enumerate}
\sphinxsetlistlabels{\arabic}{enumi}{enumii}{}{.}%
\item {} 
\sphinxAtStartPar
Isolated roughness \sphinxhyphen{} Canyon has separate recirculation and ventilation regions

\item {} 
\sphinxAtStartPar
Wake interference \sphinxhyphen{} Recirculation region begins to impinge on the downstream building

\item {} 
\sphinxAtStartPar
Skimming flow \sphinxhyphen{} Recirculation region fills the entire canyon

\end{enumerate}

\begin{DUlineblock}{0em}
\item[] The effective roughness length for heat is calculated using a resistance network within these regions.
\end{DUlineblock}
\begin{description}
\sphinxlineitem{TRUE}
\sphinxAtStartPar
Use MORUSES parameterisation for canyon and roof.

\sphinxlineitem{FALSE}
\sphinxAtStartPar
Values for canyon and roof are taken from {\hyperref[\detokenize{namelists/nveg_params.nml:JULES_NVEGPARM::z0_nvg_io}]{\sphinxcrossref{\sphinxcode{\sphinxupquote{z0\_nvg\_io}}}}} and {\hyperref[\detokenize{namelists/nveg_params.nml:JULES_NVEGPARM::z0hm_nvg_io}]{\sphinxcrossref{\sphinxcode{\sphinxupquote{z0hm\_nvg\_io}}}}}.

\end{description}

\end{fulllineitems}

\index{l\_moruses\_storage (in namelist JULES\_URBAN)@\spxentry{l\_moruses\_storage}\spxextra{in namelist JULES\_URBAN}|spxpagem}

\begin{fulllineitems}
\phantomsection\label{\detokenize{namelists/urban.nml:JULES_URBAN::l_moruses_storage}}
\pysigstartsignatures
\pysigline{\sphinxcode{\sphinxupquote{JULES\_URBAN::}}\sphinxbfcode{\sphinxupquote{l\_moruses\_storage}}}
\pysigstopsignatures\begin{quote}\begin{description}
\sphinxlineitem{Type}
\sphinxAtStartPar
logical

\sphinxlineitem{Default}
\sphinxAtStartPar
F

\end{description}\end{quote}

\sphinxAtStartPar
MORUSES switch for thermal inertia and coupling with the underlying soil for canyon and roof.

\sphinxAtStartPar
MORUSES consists of two surfaces; a canyon ({\hyperref[\detokenize{namelists/jules_surface_types.nml:JULES_SURFACE_TYPES::urban_canyon}]{\sphinxcrossref{\sphinxcode{\sphinxupquote{urban\_canyon}}}}}) and a roof ({\hyperref[\detokenize{namelists/jules_surface_types.nml:JULES_SURFACE_TYPES::urban_roof}]{\sphinxcrossref{\sphinxcode{\sphinxupquote{urban\_roof}}}}}). This MORUSES parametrisation calculates the heat capacity of each of these surface types and also modifies how they are coupled with the underlying soil. The heat capacities of the canyon and roof are calculated using the properties of the urban fabric and the geometry of the canyon. The roof has a lower thermal inertia and can respond more rapidly to changes in forcing. The nature of the coupling (radiative, conductive or none) is controlled via {\hyperref[\detokenize{namelists/nveg_params.nml:JULES_NVEGPARM::vf_nvg_io}]{\sphinxcrossref{\sphinxcode{\sphinxupquote{vf\_nvg\_io}}}}} as descibed below.

\sphinxAtStartPar
\sphinxstylestrong{The canyon}:
Consists of two walls and a road where the road only is coupled to the underlying soil. The walls are uncoupled and have a zero\sphinxhyphen{}flux boundary condition. The coupling of the road is therefore parametrised using a canyon scaling factor. The nature of the canyon (or road surface) coupling is specified as follows:
\begin{description}
\sphinxlineitem{{\hyperref[\detokenize{namelists/nveg_params.nml:JULES_NVEGPARM::vf_nvg_io}]{\sphinxcrossref{\sphinxcode{\sphinxupquote{vf\_nvg\_io}}}}} ({\hyperref[\detokenize{namelists/jules_surface_types.nml:JULES_SURFACE_TYPES::urban_canyon}]{\sphinxcrossref{\sphinxcode{\sphinxupquote{urban\_canyon}}}}}):}

\begin{savenotes}\sphinxattablestart
\centering
\begin{tabulary}{\linewidth}[t]{|T|T|}
\hline

\sphinxAtStartPar
0
&
\sphinxAtStartPar
conductively coupled
\\
\hline
\sphinxAtStartPar
1
&
\sphinxAtStartPar
radiatively coupled
\\
\hline
\end{tabulary}
\par
\sphinxattableend\end{savenotes}

\end{description}

\sphinxAtStartPar
\sphinxstylestrong{The roof}:
As the roof is not in direct contact with the soil, it physically cannot be conductively coupled. It can either be radiatively coupled or uncoupled. To allow for no coupling, MORUSES modifies the code to change the meaning of conductively coupled to \sphinxstylestrong{NOT} coupled. The nature of the coupling is therefore specified as follows:
\begin{description}
\sphinxlineitem{{\hyperref[\detokenize{namelists/nveg_params.nml:JULES_NVEGPARM::vf_nvg_io}]{\sphinxcrossref{\sphinxcode{\sphinxupquote{vf\_nvg\_io}}}}} ({\hyperref[\detokenize{namelists/jules_surface_types.nml:JULES_SURFACE_TYPES::urban_roof}]{\sphinxcrossref{\sphinxcode{\sphinxupquote{urban\_roof}}}}}):}

\begin{savenotes}\sphinxattablestart
\centering
\begin{tabulary}{\linewidth}[t]{|T|T|}
\hline

\sphinxAtStartPar
0
&
\sphinxAtStartPar
\sphinxstylestrong{NOT} coupled
\\
\hline
\sphinxAtStartPar
1
&
\sphinxAtStartPar
radiatively coupled
\\
\hline
\end{tabulary}
\par
\sphinxattableend\end{savenotes}

\sphinxlineitem{TRUE}
\sphinxAtStartPar
Use MORUSES parameterisation as described above.

\sphinxlineitem{FALSE}
\sphinxAtStartPar
Values for canyon and roof are taken from {\hyperref[\detokenize{namelists/nveg_params.nml:JULES_NVEGPARM::ch_nvg_io}]{\sphinxcrossref{\sphinxcode{\sphinxupquote{ch\_nvg\_io}}}}} and {\hyperref[\detokenize{namelists/nveg_params.nml:JULES_NVEGPARM::vf_nvg_io}]{\sphinxcrossref{\sphinxcode{\sphinxupquote{vf\_nvg\_io}}}}} (with no modifications to coupling).

\end{description}

\end{fulllineitems}

\index{l\_moruses\_storage\_thin (in namelist JULES\_URBAN)@\spxentry{l\_moruses\_storage\_thin}\spxextra{in namelist JULES\_URBAN}|spxpagem}

\begin{fulllineitems}
\phantomsection\label{\detokenize{namelists/urban.nml:JULES_URBAN::l_moruses_storage_thin}}
\pysigstartsignatures
\pysigline{\sphinxcode{\sphinxupquote{JULES\_URBAN::}}\sphinxbfcode{\sphinxupquote{l\_moruses\_storage\_thin}}}
\pysigstopsignatures\begin{quote}\begin{description}
\sphinxlineitem{Type}
\sphinxAtStartPar
logical

\sphinxlineitem{Default}
\sphinxAtStartPar
F

\end{description}\end{quote}

\sphinxAtStartPar
MORUSES switch to use a thin roof to simulate the effects of insulation.

\sphinxAtStartPar
Only used if {\hyperref[\detokenize{namelists/urban.nml:JULES_URBAN::l_moruses_storage}]{\sphinxcrossref{\sphinxcode{\sphinxupquote{l\_moruses\_storage}}}}} = TRUE.
\begin{description}
\sphinxlineitem{TRUE}
\sphinxAtStartPar
Use thin, insulated roof.

\sphinxlineitem{FALSE}
\sphinxAtStartPar
Use damping depth diffusivity of roofing materials.

\end{description}

\end{fulllineitems}

\index{l\_moruses\_macdonald (in namelist JULES\_URBAN)@\spxentry{l\_moruses\_macdonald}\spxextra{in namelist JULES\_URBAN}|spxpagem}

\begin{fulllineitems}
\phantomsection\label{\detokenize{namelists/urban.nml:JULES_URBAN::l_moruses_macdonald}}
\pysigstartsignatures
\pysigline{\sphinxcode{\sphinxupquote{JULES\_URBAN::}}\sphinxbfcode{\sphinxupquote{l\_moruses\_macdonald}}}
\pysigstopsignatures\begin{quote}\begin{description}
\sphinxlineitem{Type}
\sphinxAtStartPar
logical

\sphinxlineitem{Default}
\sphinxAtStartPar
F

\end{description}\end{quote}

\sphinxAtStartPar
MORUSES switch for using MacDonald et al. (1998) to calculate effective roughness length of urban areas and displacement height from urban geometry.
\begin{description}
\sphinxlineitem{TRUE}
\sphinxAtStartPar
Use MacDonald et al. (1998) formulations.

\sphinxlineitem{FALSE}
\sphinxAtStartPar
Appropriate data needs to be supplied instead.

\end{description}

\begin{sphinxadmonition}{note}{Note:}
\sphinxAtStartPar
If {\hyperref[\detokenize{namelists/urban.nml:JULES_URBAN::l_urban_empirical}]{\sphinxcrossref{\sphinxcode{\sphinxupquote{l\_urban\_empirical}}}}} = TRUE then {\hyperref[\detokenize{namelists/urban.nml:JULES_URBAN::l_moruses_macdonald}]{\sphinxcrossref{\sphinxcode{\sphinxupquote{l\_moruses\_macdonald}}}}} should also be TRUE, to keep the roughness length and displacement height consistent with the morphology.
\end{sphinxadmonition}


\sphinxstrong{See also:}
\nopagebreak


\sphinxAtStartPar
References:
\begin{itemize}
\item {} 
\sphinxAtStartPar
Macdonald RW, Griffiths RF, Hall D. 1998. An improved method for the estimation of surface roughness of obstacle arrays. Atmos. Env. 32: 1857\sphinxhyphen{}1864

\end{itemize}



\end{fulllineitems}

\index{l\_urban\_empirical (in namelist JULES\_URBAN)@\spxentry{l\_urban\_empirical}\spxextra{in namelist JULES\_URBAN}|spxpagem}

\begin{fulllineitems}
\phantomsection\label{\detokenize{namelists/urban.nml:JULES_URBAN::l_urban_empirical}}
\pysigstartsignatures
\pysigline{\sphinxcode{\sphinxupquote{JULES\_URBAN::}}\sphinxbfcode{\sphinxupquote{l\_urban\_empirical}}}
\pysigstopsignatures\begin{quote}\begin{description}
\sphinxlineitem{Type}
\sphinxAtStartPar
logical

\sphinxlineitem{Default}
\sphinxAtStartPar
F

\end{description}\end{quote}

\sphinxAtStartPar
Switch to use empirical relationships for urban geometry, based on total urban fraction. Dimensions calculated are W/R, H/W and H.

\sphinxAtStartPar
If no MORUSES parametrisations are used, i.e. the basic URBAN\sphinxhyphen{}2T, then only W/R is required.

\sphinxAtStartPar
If the roof fraction is not supplied i.e. canyon fraction = total urban fraction, then W/R will be used to calculate the canyon and roof fractions. W/R is also used to distribute anthropogenic heat between the roof and the canyon if {\hyperref[\detokenize{namelists/jules_surface.nml:JULES_SURFACE::l_anthrop_heat_src}]{\sphinxcrossref{\sphinxcode{\sphinxupquote{l\_anthrop\_heat\_src}}}}} = TRUE.
\begin{description}
\sphinxlineitem{TRUE}
\sphinxAtStartPar
Use empirical relationships for urban geometry.

\sphinxlineitem{FALSE}
\sphinxAtStartPar
Appropriate data needs to be supplied instead.

\end{description}

\begin{sphinxadmonition}{warning}{Warning:}
\sphinxAtStartPar
These are only valid for high resolutions (\textasciitilde{}1 km).
\end{sphinxadmonition}


\sphinxstrong{See also:}
\nopagebreak


\sphinxAtStartPar
References:
\begin{itemize}
\item {} 
\sphinxAtStartPar
Bohnenstengel SI, Evans S, Clark P, Belcher SE (2010). Simulations of the London urban heat island, Quarterly Journal of the Royal Meteorological Society (submitted)

\end{itemize}



\end{fulllineitems}


\sphinxstepscope


\section{\sphinxstyleliteralintitle{\sphinxupquote{fire.nml}}}
\label{\detokenize{namelists/fire.nml:fire-nml}}\label{\detokenize{namelists/fire.nml::doc}}
\sphinxAtStartPar
This file contains a single namelist called {\hyperref[\detokenize{namelists/fire.nml:namelist-FIRE_SWITCHES}]{\sphinxcrossref{\sphinxcode{\sphinxupquote{FIRE\_SWITCHES}}}}} that sets time\sphinxhyphen{}invariant parameters for performing wildfire\sphinxhyphen{}related calculations.


\subsection{\sphinxstyleliteralintitle{\sphinxupquote{FIRE\_SWITCHES}} namelist members}
\label{\detokenize{namelists/fire.nml:namelist-FIRE_SWITCHES}}\label{\detokenize{namelists/fire.nml:fire-switches-namelist-members}}\index{FIRE\_SWITCHES (namelist)@\spxentry{FIRE\_SWITCHES}\spxextra{namelist}|spxpagem}\index{l\_fire (in namelist FIRE\_SWITCHES)@\spxentry{l\_fire}\spxextra{in namelist FIRE\_SWITCHES}|spxpagem}

\begin{fulllineitems}
\phantomsection\label{\detokenize{namelists/fire.nml:FIRE_SWITCHES::l_fire}}
\pysigstartsignatures
\pysigline{\sphinxcode{\sphinxupquote{FIRE\_SWITCHES::}}\sphinxbfcode{\sphinxupquote{l\_fire}}}
\pysigstopsignatures\begin{quote}\begin{description}
\sphinxlineitem{Type}
\sphinxAtStartPar
logical

\sphinxlineitem{Default}
\sphinxAtStartPar
F

\end{description}\end{quote}

\sphinxAtStartPar
Switch to enable the fire module.
\begin{description}
\sphinxlineitem{TRUE}
\sphinxAtStartPar
The fire module will be executed according to the settings of subsequent namelist members.

\sphinxlineitem{FALSE}
\sphinxAtStartPar
The fire module will not be executed and subsequent members of the namelist will have no effect.

\end{description}

\end{fulllineitems}

\index{mcarthur\_flag (in namelist FIRE\_SWITCHES)@\spxentry{mcarthur\_flag}\spxextra{in namelist FIRE\_SWITCHES}|spxpagem}

\begin{fulllineitems}
\phantomsection\label{\detokenize{namelists/fire.nml:FIRE_SWITCHES::mcarthur_flag}}
\pysigstartsignatures
\pysigline{\sphinxcode{\sphinxupquote{FIRE\_SWITCHES::}}\sphinxbfcode{\sphinxupquote{mcarthur\_flag}}}
\pysigstopsignatures\begin{quote}\begin{description}
\sphinxlineitem{Type}
\sphinxAtStartPar
boolean

\sphinxlineitem{Default}
\sphinxAtStartPar
F

\end{description}\end{quote}

\sphinxAtStartPar
Switch for calculating the McArthur Forest Fire Danger Index (FFDI).

\end{fulllineitems}

\index{mcarthur\_opt (in namelist FIRE\_SWITCHES)@\spxentry{mcarthur\_opt}\spxextra{in namelist FIRE\_SWITCHES}|spxpagem}

\begin{fulllineitems}
\phantomsection\label{\detokenize{namelists/fire.nml:FIRE_SWITCHES::mcarthur_opt}}
\pysigstartsignatures
\pysigline{\sphinxcode{\sphinxupquote{FIRE\_SWITCHES::}}\sphinxbfcode{\sphinxupquote{mcarthur\_opt}}}
\pysigstopsignatures\begin{quote}\begin{description}
\sphinxlineitem{Type}
\sphinxAtStartPar
real

\sphinxlineitem{Default}
\sphinxAtStartPar
MDI

\end{description}\end{quote}

\sphinxAtStartPar
Switch for choosing which method of calculating the soil moisture deficit required for the McArthur Forest Fire Danger Index (FFDI). 1 uses the model soil moisture, 2 uses a fixed value of 120 mm.

\end{fulllineitems}

\index{canadian\_flag (in namelist FIRE\_SWITCHES)@\spxentry{canadian\_flag}\spxextra{in namelist FIRE\_SWITCHES}|spxpagem}

\begin{fulllineitems}
\phantomsection\label{\detokenize{namelists/fire.nml:FIRE_SWITCHES::canadian_flag}}
\pysigstartsignatures
\pysigline{\sphinxcode{\sphinxupquote{FIRE\_SWITCHES::}}\sphinxbfcode{\sphinxupquote{canadian\_flag}}}
\pysigstopsignatures\begin{quote}\begin{description}
\sphinxlineitem{Type}
\sphinxAtStartPar
boolean

\sphinxlineitem{Default}
\sphinxAtStartPar
F

\end{description}\end{quote}

\sphinxAtStartPar
Switch for calculating the Canadian Fire Weather Index (FWI).

\end{fulllineitems}

\index{canadian\_hemi\_opt (in namelist FIRE\_SWITCHES)@\spxentry{canadian\_hemi\_opt}\spxextra{in namelist FIRE\_SWITCHES}|spxpagem}

\begin{fulllineitems}
\phantomsection\label{\detokenize{namelists/fire.nml:FIRE_SWITCHES::canadian_hemi_opt}}
\pysigstartsignatures
\pysigline{\sphinxcode{\sphinxupquote{FIRE\_SWITCHES::}}\sphinxbfcode{\sphinxupquote{canadian\_hemi\_opt}}}
\pysigstopsignatures\begin{quote}\begin{description}
\sphinxlineitem{Type}
\sphinxAtStartPar
boolean

\sphinxlineitem{Default}
\sphinxAtStartPar
F

\end{description}\end{quote}

\sphinxAtStartPar
If TRUE, then the month\sphinxhyphen{}dependent parameters used in the calculation will be offset by 6 months for the southern hemisphere. This will cause a discontinuity in results when crossing the equator.

\end{fulllineitems}

\index{nesterov\_flag (in namelist FIRE\_SWITCHES)@\spxentry{nesterov\_flag}\spxextra{in namelist FIRE\_SWITCHES}|spxpagem}

\begin{fulllineitems}
\phantomsection\label{\detokenize{namelists/fire.nml:FIRE_SWITCHES::nesterov_flag}}
\pysigstartsignatures
\pysigline{\sphinxcode{\sphinxupquote{FIRE\_SWITCHES::}}\sphinxbfcode{\sphinxupquote{nesterov\_flag}}}
\pysigstopsignatures\begin{quote}\begin{description}
\sphinxlineitem{Type}
\sphinxAtStartPar
boolean

\sphinxlineitem{Default}
\sphinxAtStartPar
F

\end{description}\end{quote}

\sphinxAtStartPar
Switch for calculating the Nesterov Index.

\end{fulllineitems}


\sphinxstepscope


\section{\sphinxstyleliteralintitle{\sphinxupquote{drive.nml}}}
\label{\detokenize{namelists/drive.nml:drive-nml}}\label{\detokenize{namelists/drive.nml::doc}}
\sphinxAtStartPar
This file contains a single namelist called {\hyperref[\detokenize{namelists/drive.nml:namelist-JULES_DRIVE}]{\sphinxcrossref{\sphinxcode{\sphinxupquote{JULES\_DRIVE}}}}} that indicates how meteorological driving data is input.


\subsection{\sphinxstyleliteralintitle{\sphinxupquote{JULES\_DRIVE}} namelist members}
\label{\detokenize{namelists/drive.nml:namelist-JULES_DRIVE}}\label{\detokenize{namelists/drive.nml:jules-drive-namelist-members}}\index{JULES\_DRIVE (namelist)@\spxentry{JULES\_DRIVE}\spxextra{namelist}|spxpagem}\index{t\_for\_snow (in namelist JULES\_DRIVE)@\spxentry{t\_for\_snow}\spxextra{in namelist JULES\_DRIVE}|spxpagem}

\begin{fulllineitems}
\phantomsection\label{\detokenize{namelists/drive.nml:JULES_DRIVE::t_for_snow}}
\pysigstartsignatures
\pysigline{\sphinxcode{\sphinxupquote{JULES\_DRIVE::}}\sphinxbfcode{\sphinxupquote{t\_for\_snow}}}
\pysigstopsignatures\begin{quote}\begin{description}
\sphinxlineitem{Type}
\sphinxAtStartPar
real

\sphinxlineitem{Default}
\sphinxAtStartPar
None

\end{description}\end{quote}

\sphinxAtStartPar
If total precipitation is given as a forcing variable, then {\hyperref[\detokenize{namelists/drive.nml:JULES_DRIVE::t_for_snow}]{\sphinxcrossref{\sphinxcode{\sphinxupquote{t\_for\_snow}}}}} is the near\sphinxhyphen{}surface air temperature (K) at or below which the precipitation is assumed to be snowfall. At higher temperatures, all the precipitation is assumed to be liquid. The default value used to be 274.0 K.

\end{fulllineitems}

\index{t\_for\_con\_rain (in namelist JULES\_DRIVE)@\spxentry{t\_for\_con\_rain}\spxextra{in namelist JULES\_DRIVE}|spxpagem}

\begin{fulllineitems}
\phantomsection\label{\detokenize{namelists/drive.nml:JULES_DRIVE::t_for_con_rain}}
\pysigstartsignatures
\pysigline{\sphinxcode{\sphinxupquote{JULES\_DRIVE::}}\sphinxbfcode{\sphinxupquote{t\_for\_con\_rain}}}
\pysigstopsignatures\begin{quote}\begin{description}
\sphinxlineitem{Type}
\sphinxAtStartPar
real

\sphinxlineitem{Default}
\sphinxAtStartPar
None

\end{description}\end{quote}

\sphinxAtStartPar
If total preciption or total rainfall are given, then {\hyperref[\detokenize{namelists/drive.nml:JULES_DRIVE::t_for_con_rain}]{\sphinxcrossref{\sphinxcode{\sphinxupquote{t\_for\_con\_rain}}}}} is the near\sphinxhyphen{}surface air temperature (K) at or above which rainfall is assumed to be convective in origin. At lower temperatures, all the rainfall is assumed to be large\sphinxhyphen{}scale in origin. In this configuration all snow is assumed to be large\sphinxhyphen{}scale in origin. The default value used to be 373.15 K but in general this is not recommended as it effectively means all precipitation is large\sphinxhyphen{}scale; a value of 293.15 K might be more appropriate.

\sphinxAtStartPar
Also see {\hyperref[\detokenize{namelists/jules_soil.nml:JULES_SOIL::confrac}]{\sphinxcrossref{\sphinxcode{\sphinxupquote{confrac}}}}}.

\sphinxAtStartPar
{\hyperref[\detokenize{namelists/drive.nml:JULES_DRIVE::t_for_con_rain}]{\sphinxcrossref{\sphinxcode{\sphinxupquote{t\_for\_con\_rain}}}}} is not used if {\hyperref[\detokenize{namelists/jules_surface.nml:JULES_SURFACE::l_point_data}]{\sphinxcrossref{\sphinxcode{\sphinxupquote{l\_point\_data}}}}} = TRUE, since then there is no convective precipitation.

\end{fulllineitems}

\index{diff\_frac\_const (in namelist JULES\_DRIVE)@\spxentry{diff\_frac\_const}\spxextra{in namelist JULES\_DRIVE}|spxpagem}

\begin{fulllineitems}
\phantomsection\label{\detokenize{namelists/drive.nml:JULES_DRIVE::diff_frac_const}}
\pysigstartsignatures
\pysigline{\sphinxcode{\sphinxupquote{JULES\_DRIVE::}}\sphinxbfcode{\sphinxupquote{diff\_frac\_const}}}
\pysigstopsignatures\begin{quote}\begin{description}
\sphinxlineitem{Type}
\sphinxAtStartPar
real

\sphinxlineitem{Default}
\sphinxAtStartPar
None

\end{description}\end{quote}

\sphinxAtStartPar
A constant value used to calculate diffuse radiation from the total downward shortwave radiation.

\sphinxAtStartPar
Only used if diffuse radiation is not given as a forcing variable (see {\hyperref[\detokenize{namelists/drive.nml:list-of-jules-forcing-variables}]{\sphinxcrossref{\DUrole{std,std-ref}{List of JULES forcing variables}}}}).

\end{fulllineitems}


\begin{sphinxadmonition}{note}{Members used to control the daily disaggregator}

\sphinxAtStartPar
HCTN96 refer to Hadley Centre technical note 96, available from \sphinxhref{http://www.metoffice.gov.uk/learning/library/publications/science/climate-science-technical-notes}{the Met Office Library}.
\index{l\_daily\_disagg (in namelist JULES\_DRIVE)@\spxentry{l\_daily\_disagg}\spxextra{in namelist JULES\_DRIVE}|spxpagem}

\begin{fulllineitems}
\phantomsection\label{\detokenize{namelists/drive.nml:JULES_DRIVE::l_daily_disagg}}
\pysigstartsignatures
\pysigline{\sphinxcode{\sphinxupquote{JULES\_DRIVE::}}\sphinxbfcode{\sphinxupquote{l\_daily\_disagg}}}
\pysigstopsignatures\begin{quote}\begin{description}
\sphinxlineitem{Type}
\sphinxAtStartPar
logical

\sphinxlineitem{Default}
\sphinxAtStartPar
F

\end{description}\end{quote}

\sphinxAtStartPar
Switch controlling whether the disaggregator is used to convert daily data driving data to driving data at the model timestep. See HCTN96 for a description of the disaggregation methods used.
\begin{description}
\sphinxlineitem{TRUE}
\sphinxAtStartPar
Disaggregator is used.

\begin{sphinxadmonition}{warning}{Warning:}
\sphinxAtStartPar
The disaggregator requires:
\begin{enumerate}
\sphinxsetlistlabels{\arabic}{enumi}{enumii}{}{.}%
\item {} 
\sphinxAtStartPar
Daily forcing data, i.e. {\hyperref[\detokenize{namelists/drive.nml:JULES_DRIVE::data_period}]{\sphinxcrossref{\sphinxcode{\sphinxupquote{data\_period}}}}} = 86400

\item {} 
\sphinxAtStartPar
{\hyperref[\detokenize{namelists/timesteps.nml:JULES_TIME::main_run_start}]{\sphinxcrossref{\sphinxcode{\sphinxupquote{main\_run\_start}}}}}, {\hyperref[\detokenize{namelists/timesteps.nml:JULES_SPINUP::spinup_start}]{\sphinxcrossref{\sphinxcode{\sphinxupquote{spinup\_start}}}}} and {\hyperref[\detokenize{namelists/drive.nml:JULES_DRIVE::data_start}]{\sphinxcrossref{\sphinxcode{\sphinxupquote{data\_start}}}}} to be 00:00:00 for some day.

\end{enumerate}
\end{sphinxadmonition}

\sphinxlineitem{FALSE}
\sphinxAtStartPar
Disaggregator is not used.

\end{description}

\end{fulllineitems}

\index{l\_disagg\_const\_rh (in namelist JULES\_DRIVE)@\spxentry{l\_disagg\_const\_rh}\spxextra{in namelist JULES\_DRIVE}|spxpagem}

\begin{fulllineitems}
\phantomsection\label{\detokenize{namelists/drive.nml:JULES_DRIVE::l_disagg_const_rh}}
\pysigstartsignatures
\pysigline{\sphinxcode{\sphinxupquote{JULES\_DRIVE::}}\sphinxbfcode{\sphinxupquote{l\_disagg\_const\_rh}}}
\pysigstopsignatures\begin{quote}\begin{description}
\sphinxlineitem{Type}
\sphinxAtStartPar
logical

\sphinxlineitem{Default}
\sphinxAtStartPar
F

\end{description}\end{quote}

\sphinxAtStartPar
Switch controlling sub\sphinxhyphen{}daily disaggregation of humidity.

\sphinxAtStartPar
Only used if {\hyperref[\detokenize{namelists/drive.nml:JULES_DRIVE::l_daily_disagg}]{\sphinxcrossref{\sphinxcode{\sphinxupquote{l\_daily\_disagg}}}}} = TRUE.
\begin{description}
\sphinxlineitem{TRUE}
\sphinxAtStartPar
Relative humidity is kept constant over day.

\sphinxlineitem{FALSE}
\sphinxAtStartPar
Specific humidity is kept constant over day (apart from when limited by specific humidity at saturation).

\end{description}

\end{fulllineitems}

\index{dur\_conv\_rain (in namelist JULES\_DRIVE)@\spxentry{dur\_conv\_rain}\spxextra{in namelist JULES\_DRIVE}|spxpagem}

\begin{fulllineitems}
\phantomsection\label{\detokenize{namelists/drive.nml:JULES_DRIVE::dur_conv_rain}}
\pysigstartsignatures
\pysigline{\sphinxcode{\sphinxupquote{JULES\_DRIVE::}}\sphinxbfcode{\sphinxupquote{dur\_conv\_rain}}}
\pysigstopsignatures\begin{quote}\begin{description}
\sphinxlineitem{Type}
\sphinxAtStartPar
real

\sphinxlineitem{Default}
\sphinxAtStartPar
None

\end{description}\end{quote}

\sphinxAtStartPar
Duration of a convective rainfall event in seconds for use in the disaggregator. See HCTN96 section 2.4. A value of 21600s (6 hours) used to be the default.

\sphinxAtStartPar
Only used if {\hyperref[\detokenize{namelists/drive.nml:JULES_DRIVE::l_daily_disagg}]{\sphinxcrossref{\sphinxcode{\sphinxupquote{l\_daily\_disagg}}}}} = TRUE.

\end{fulllineitems}

\index{dur\_ls\_rain (in namelist JULES\_DRIVE)@\spxentry{dur\_ls\_rain}\spxextra{in namelist JULES\_DRIVE}|spxpagem}

\begin{fulllineitems}
\phantomsection\label{\detokenize{namelists/drive.nml:JULES_DRIVE::dur_ls_rain}}
\pysigstartsignatures
\pysigline{\sphinxcode{\sphinxupquote{JULES\_DRIVE::}}\sphinxbfcode{\sphinxupquote{dur\_ls\_rain}}}
\pysigstopsignatures\begin{quote}\begin{description}
\sphinxlineitem{Type}
\sphinxAtStartPar
real

\sphinxlineitem{Default}
\sphinxAtStartPar
None

\end{description}\end{quote}

\sphinxAtStartPar
Duration of a large\sphinxhyphen{}scale rainfall event in seconds for use in the disaggregator. See HCTN96 section 2.4. A value of 3600s (1 hour) used to be the default.

\sphinxAtStartPar
Only used if {\hyperref[\detokenize{namelists/drive.nml:JULES_DRIVE::l_daily_disagg}]{\sphinxcrossref{\sphinxcode{\sphinxupquote{l\_daily\_disagg}}}}} = TRUE.

\end{fulllineitems}

\index{dur\_conv\_snow (in namelist JULES\_DRIVE)@\spxentry{dur\_conv\_snow}\spxextra{in namelist JULES\_DRIVE}|spxpagem}

\begin{fulllineitems}
\phantomsection\label{\detokenize{namelists/drive.nml:JULES_DRIVE::dur_conv_snow}}
\pysigstartsignatures
\pysigline{\sphinxcode{\sphinxupquote{JULES\_DRIVE::}}\sphinxbfcode{\sphinxupquote{dur\_conv\_snow}}}
\pysigstopsignatures\begin{quote}\begin{description}
\sphinxlineitem{Type}
\sphinxAtStartPar
real

\sphinxlineitem{Default}
\sphinxAtStartPar
None

\end{description}\end{quote}

\sphinxAtStartPar
Duration of a convective snowfall event in seconds for use in the disaggregator. See HCTN96 section 2.4. A value of 3600s (1 hour) used to be the default.

\sphinxAtStartPar
Only used if {\hyperref[\detokenize{namelists/drive.nml:JULES_DRIVE::l_daily_disagg}]{\sphinxcrossref{\sphinxcode{\sphinxupquote{l\_daily\_disagg}}}}} = TRUE.

\end{fulllineitems}

\index{dur\_ls\_snow (in namelist JULES\_DRIVE)@\spxentry{dur\_ls\_snow}\spxextra{in namelist JULES\_DRIVE}|spxpagem}

\begin{fulllineitems}
\phantomsection\label{\detokenize{namelists/drive.nml:JULES_DRIVE::dur_ls_snow}}
\pysigstartsignatures
\pysigline{\sphinxcode{\sphinxupquote{JULES\_DRIVE::}}\sphinxbfcode{\sphinxupquote{dur\_ls\_snow}}}
\pysigstopsignatures\begin{quote}\begin{description}
\sphinxlineitem{Type}
\sphinxAtStartPar
real

\sphinxlineitem{Default}
\sphinxAtStartPar
None

\end{description}\end{quote}

\sphinxAtStartPar
Duration of a large\sphinxhyphen{}scale snowfall event in seconds for use in the disaggregator. See HCTN96 section 2.4. A value of 3600s (1 hour) used to be the default.

\sphinxAtStartPar
Only used if {\hyperref[\detokenize{namelists/drive.nml:JULES_DRIVE::l_daily_disagg}]{\sphinxcrossref{\sphinxcode{\sphinxupquote{l\_daily\_disagg}}}}} = TRUE.

\end{fulllineitems}

\index{precip\_disagg\_method (in namelist JULES\_DRIVE)@\spxentry{precip\_disagg\_method}\spxextra{in namelist JULES\_DRIVE}|spxpagem}

\begin{fulllineitems}
\phantomsection\label{\detokenize{namelists/drive.nml:JULES_DRIVE::precip_disagg_method}}
\pysigstartsignatures
\pysigline{\sphinxcode{\sphinxupquote{JULES\_DRIVE::}}\sphinxbfcode{\sphinxupquote{precip\_disagg\_method}}}
\pysigstopsignatures\begin{quote}\begin{description}
\sphinxlineitem{Type}
\sphinxAtStartPar
integer

\sphinxlineitem{Permitted}
\sphinxAtStartPar
1, 2, 3 or 4

\sphinxlineitem{Default}
\sphinxAtStartPar
None

\end{description}\end{quote}

\sphinxAtStartPar
Switch controlling the disaggregation method for precipitation. See HCTN96 section 2.4. The default value used to be 2.

\sphinxAtStartPar
Only used if {\hyperref[\detokenize{namelists/drive.nml:JULES_DRIVE::l_daily_disagg}]{\sphinxcrossref{\sphinxcode{\sphinxupquote{l\_daily\_disagg}}}}} = TRUE.
\begin{enumerate}
\sphinxsetlistlabels{\arabic}{enumi}{enumii}{}{.}%
\item {} 
\sphinxAtStartPar
Do not disaggregate precipitation.

\item {} 
\sphinxAtStartPar
Disaggregate precipitation using the method implemented in IMOGEN, which allocates the daily precipitation each type into one event of duration {\hyperref[\detokenize{namelists/drive.nml:JULES_DRIVE::dur_conv_rain}]{\sphinxcrossref{\sphinxcode{\sphinxupquote{dur\_conv\_rain}}}}}, {\hyperref[\detokenize{namelists/drive.nml:JULES_DRIVE::dur_ls_rain}]{\sphinxcrossref{\sphinxcode{\sphinxupquote{dur\_ls\_rain}}}}}, {\hyperref[\detokenize{namelists/drive.nml:JULES_DRIVE::dur_conv_snow}]{\sphinxcrossref{\sphinxcode{\sphinxupquote{dur\_conv\_snow}}}}} and {\hyperref[\detokenize{namelists/drive.nml:JULES_DRIVE::dur_ls_snow}]{\sphinxcrossref{\sphinxcode{\sphinxupquote{dur\_ls\_snow}}}}} for convective rain, large\sphinxhyphen{}scale rain, convective snow and large\sphinxhyphen{}scale snow respectively.
The start time of this event is randomly distributed from the beginning of the day to the end of the day minus the event duration.
If the rate of precipitation in any timestep of any type is greater than a hard\sphinxhyphen{}coded maximum (currently 350 mm/day), the precipitation is redistributed by the \sphinxcode{\sphinxupquote{redis}} routine in IMOGEN.

\item {} 
\sphinxAtStartPar
As for 2, except no upper limit on the precipitation in a timestep.

\item {} 
\sphinxAtStartPar
The event duration variable is used to determine the fraction of wet and dry timesteps, which are then distributed randomly throughout the day.

\end{enumerate}

\end{fulllineitems}

\end{sphinxadmonition}

\begin{sphinxadmonition}{note}{Members used to specify perturbations to the driving data}
\index{l\_perturb\_driving (in namelist JULES\_DRIVE)@\spxentry{l\_perturb\_driving}\spxextra{in namelist JULES\_DRIVE}|spxpagem}

\begin{fulllineitems}
\phantomsection\label{\detokenize{namelists/drive.nml:JULES_DRIVE::l_perturb_driving}}
\pysigstartsignatures
\pysigline{\sphinxcode{\sphinxupquote{JULES\_DRIVE::}}\sphinxbfcode{\sphinxupquote{l\_perturb\_driving}}}
\pysigstopsignatures\begin{quote}\begin{description}
\sphinxlineitem{Type}
\sphinxAtStartPar
logical

\sphinxlineitem{Default}
\sphinxAtStartPar
F

\end{description}\end{quote}

\sphinxAtStartPar
Apply perturbation to driving data.

\end{fulllineitems}

\index{temperature\_abs\_perturbation (in namelist JULES\_DRIVE)@\spxentry{temperature\_abs\_perturbation}\spxextra{in namelist JULES\_DRIVE}|spxpagem}

\begin{fulllineitems}
\phantomsection\label{\detokenize{namelists/drive.nml:JULES_DRIVE::temperature_abs_perturbation}}
\pysigstartsignatures
\pysigline{\sphinxcode{\sphinxupquote{JULES\_DRIVE::}}\sphinxbfcode{\sphinxupquote{temperature\_abs\_perturbation}}}
\pysigstopsignatures\begin{quote}\begin{description}
\sphinxlineitem{Type}
\sphinxAtStartPar
real

\sphinxlineitem{Default}
\sphinxAtStartPar
None

\end{description}\end{quote}

\sphinxAtStartPar
Absolute perturbation amount to add to temperature. Can be positive or negative. Only used if {\hyperref[\detokenize{namelists/drive.nml:JULES_DRIVE::l_perturb_driving}]{\sphinxcrossref{\sphinxcode{\sphinxupquote{l\_perturb\_driving}}}}} = TRUE.

\end{fulllineitems}

\index{precip\_rel\_perturbation (in namelist JULES\_DRIVE)@\spxentry{precip\_rel\_perturbation}\spxextra{in namelist JULES\_DRIVE}|spxpagem}

\begin{fulllineitems}
\phantomsection\label{\detokenize{namelists/drive.nml:JULES_DRIVE::precip_rel_perturbation}}
\pysigstartsignatures
\pysigline{\sphinxcode{\sphinxupquote{JULES\_DRIVE::}}\sphinxbfcode{\sphinxupquote{precip\_rel\_perturbation}}}
\pysigstopsignatures\begin{quote}\begin{description}
\sphinxlineitem{Type}
\sphinxAtStartPar
real

\sphinxlineitem{Permitted}
\sphinxAtStartPar
\textgreater{}= 0.0

\sphinxlineitem{Default}
\sphinxAtStartPar
None

\end{description}\end{quote}

\sphinxAtStartPar
Relative perturbation for precipitation variables (a multiplicative factor). Only used if {\hyperref[\detokenize{namelists/drive.nml:JULES_DRIVE::l_perturb_driving}]{\sphinxcrossref{\sphinxcode{\sphinxupquote{l\_perturb\_driving}}}}} = TRUE.

\end{fulllineitems}

\end{sphinxadmonition}

\begin{sphinxadmonition}{note}{Members used to specify \sphinxstyleliteralintitle{\sphinxupquote{z1\_tq}} and \sphinxstyleliteralintitle{\sphinxupquote{z1\_uv}}}
\index{z1\_uv\_in (in namelist JULES\_DRIVE)@\spxentry{z1\_uv\_in}\spxextra{in namelist JULES\_DRIVE}|spxpagem}

\begin{fulllineitems}
\phantomsection\label{\detokenize{namelists/drive.nml:JULES_DRIVE::z1_uv_in}}
\pysigstartsignatures
\pysigline{\sphinxcode{\sphinxupquote{JULES\_DRIVE::}}\sphinxbfcode{\sphinxupquote{z1\_uv\_in}}}
\pysigstopsignatures\begin{quote}\begin{description}
\sphinxlineitem{Type}
\sphinxAtStartPar
real

\sphinxlineitem{Permitted}
\sphinxAtStartPar
\textgreater{} 0.0

\sphinxlineitem{Default}
\sphinxAtStartPar
None

\end{description}\end{quote}

\sphinxAtStartPar
Constant value for the height (m) at which the wind data are valid for every point. This height is relative to the zero\sphinxhyphen{}plane, not the ground.

\end{fulllineitems}

\index{z1\_tq\_vary (in namelist JULES\_DRIVE)@\spxentry{z1\_tq\_vary}\spxextra{in namelist JULES\_DRIVE}|spxpagem}

\begin{fulllineitems}
\phantomsection\label{\detokenize{namelists/drive.nml:JULES_DRIVE::z1_tq_vary}}
\pysigstartsignatures
\pysigline{\sphinxcode{\sphinxupquote{JULES\_DRIVE::}}\sphinxbfcode{\sphinxupquote{z1\_tq\_vary}}}
\pysigstopsignatures\begin{quote}\begin{description}
\sphinxlineitem{Type}
\sphinxAtStartPar
logical

\sphinxlineitem{Default}
\sphinxAtStartPar
F

\end{description}\end{quote}

\sphinxAtStartPar
Switch to indicate whether \sphinxcode{\sphinxupquote{z1\_tq}} (the height (m) at which the temperature and humidity data are valid) should be constant for all points or spatially varying. The height is relative to the zero\sphinxhyphen{}plane, not the ground.
\begin{description}
\sphinxlineitem{TRUE}
\sphinxAtStartPar
Spatially varying \sphinxcode{\sphinxupquote{z1\_tq}} will be read from the file specified in {\hyperref[\detokenize{namelists/drive.nml:JULES_DRIVE::z1_tq_file}]{\sphinxcrossref{\sphinxcode{\sphinxupquote{z1\_tq\_file}}}}}.

\sphinxlineitem{FALSE}
\sphinxAtStartPar
\sphinxcode{\sphinxupquote{z1\_tq}} will be set to a constant value, specified in {\hyperref[\detokenize{namelists/drive.nml:JULES_DRIVE::z1_tq_in}]{\sphinxcrossref{\sphinxcode{\sphinxupquote{z1\_tq\_in}}}}}, at all points.

\end{description}

\end{fulllineitems}

\index{z1\_tq\_in (in namelist JULES\_DRIVE)@\spxentry{z1\_tq\_in}\spxextra{in namelist JULES\_DRIVE}|spxpagem}

\begin{fulllineitems}
\phantomsection\label{\detokenize{namelists/drive.nml:JULES_DRIVE::z1_tq_in}}
\pysigstartsignatures
\pysigline{\sphinxcode{\sphinxupquote{JULES\_DRIVE::}}\sphinxbfcode{\sphinxupquote{z1\_tq\_in}}}
\pysigstopsignatures\begin{quote}\begin{description}
\sphinxlineitem{Type}
\sphinxAtStartPar
real

\sphinxlineitem{Permitted}
\sphinxAtStartPar
\textgreater{} 0.0

\sphinxlineitem{Default}
\sphinxAtStartPar
None

\end{description}\end{quote}

\sphinxAtStartPar
Constant value for \sphinxcode{\sphinxupquote{z1\_tq}} to be used for every point.

\sphinxAtStartPar
Only required if {\hyperref[\detokenize{namelists/drive.nml:JULES_DRIVE::z1_tq_vary}]{\sphinxcrossref{\sphinxcode{\sphinxupquote{z1\_tq\_vary}}}}} = F.

\end{fulllineitems}

\index{z1\_tq\_file (in namelist JULES\_DRIVE)@\spxentry{z1\_tq\_file}\spxextra{in namelist JULES\_DRIVE}|spxpagem}

\begin{fulllineitems}
\phantomsection\label{\detokenize{namelists/drive.nml:JULES_DRIVE::z1_tq_file}}
\pysigstartsignatures
\pysigline{\sphinxcode{\sphinxupquote{JULES\_DRIVE::}}\sphinxbfcode{\sphinxupquote{z1\_tq\_file}}}
\pysigstopsignatures\begin{quote}\begin{description}
\sphinxlineitem{Type}
\sphinxAtStartPar
character

\sphinxlineitem{Default}
\sphinxAtStartPar
None

\end{description}\end{quote}

\sphinxAtStartPar
File to read spatially varying \sphinxcode{\sphinxupquote{z1\_tq}} from.

\sphinxAtStartPar
Only required if {\hyperref[\detokenize{namelists/drive.nml:JULES_DRIVE::z1_tq_vary}]{\sphinxcrossref{\sphinxcode{\sphinxupquote{z1\_tq\_vary}}}}} = T.

\end{fulllineitems}

\index{z1\_tq\_var\_name (in namelist JULES\_DRIVE)@\spxentry{z1\_tq\_var\_name}\spxextra{in namelist JULES\_DRIVE}|spxpagem}

\begin{fulllineitems}
\phantomsection\label{\detokenize{namelists/drive.nml:JULES_DRIVE::z1_tq_var_name}}
\pysigstartsignatures
\pysigline{\sphinxcode{\sphinxupquote{JULES\_DRIVE::}}\sphinxbfcode{\sphinxupquote{z1\_tq\_var\_name}}}
\pysigstopsignatures\begin{quote}\begin{description}
\sphinxlineitem{Type}
\sphinxAtStartPar
character

\sphinxlineitem{Default}
\sphinxAtStartPar
‘z1\_tq\_in’

\end{description}\end{quote}

\sphinxAtStartPar
The name of the variable in {\hyperref[\detokenize{namelists/drive.nml:JULES_DRIVE::z1_tq_file}]{\sphinxcrossref{\sphinxcode{\sphinxupquote{z1\_tq\_file}}}}} containing the data for \sphinxcode{\sphinxupquote{z1\_tq}}.

\sphinxAtStartPar
The variable should have no levels dimensions and no time dimension.

\begin{sphinxadmonition}{note}{Note:}
\sphinxAtStartPar
This is not used for ASCII files.

\sphinxAtStartPar
However, since ASCII files can only be used for single\sphinxhyphen{}point runs, it is recommended to set {\hyperref[\detokenize{namelists/drive.nml:JULES_DRIVE::z1_tq_vary}]{\sphinxcrossref{\sphinxcode{\sphinxupquote{z1\_tq\_vary}}}}} = F and use {\hyperref[\detokenize{namelists/drive.nml:JULES_DRIVE::z1_tq_in}]{\sphinxcrossref{\sphinxcode{\sphinxupquote{z1\_tq\_in}}}}} anyway.
\end{sphinxadmonition}

\end{fulllineitems}

\end{sphinxadmonition}

\begin{sphinxadmonition}{note}{Members used to specify boundary layer height}
\index{bl\_height (in namelist JULES\_DRIVE)@\spxentry{bl\_height}\spxextra{in namelist JULES\_DRIVE}|spxpagem}

\begin{fulllineitems}
\phantomsection\label{\detokenize{namelists/drive.nml:JULES_DRIVE::bl_height}}
\pysigstartsignatures
\pysigline{\sphinxcode{\sphinxupquote{JULES\_DRIVE::}}\sphinxbfcode{\sphinxupquote{bl\_height}}}
\pysigstopsignatures\begin{quote}\begin{description}
\sphinxlineitem{Type}
\sphinxAtStartPar
real

\sphinxlineitem{Permitted}
\sphinxAtStartPar
\textgreater{} 0.0

\sphinxlineitem{Default}
\sphinxAtStartPar
1000.0

\end{description}\end{quote}

\sphinxAtStartPar
Height above ground to top of the atmospheric boundary layer (m). This value is disregarded if \sphinxcode{\sphinxupquote{bl\_height}} is provided as prescribed data (see {\hyperref[\detokenize{namelists/prescribed_data.nml:supported-prescribed-variables}]{\sphinxcrossref{\DUrole{std,std-ref}{List of supported variables}}}}).

\end{fulllineitems}

\end{sphinxadmonition}

\begin{sphinxadmonition}{note}{Members used to specify the start, end and period of the data}
\index{data\_start (in namelist JULES\_DRIVE)@\spxentry{data\_start}\spxextra{in namelist JULES\_DRIVE}|spxpagem}

\begin{fulllineitems}
\phantomsection\label{\detokenize{namelists/drive.nml:JULES_DRIVE::data_start}}
\pysigstartsignatures
\pysigline{\sphinxcode{\sphinxupquote{JULES\_DRIVE::}}\sphinxbfcode{\sphinxupquote{data\_start}}}
\pysigstopsignatures
\end{fulllineitems}

\index{data\_end (in namelist JULES\_DRIVE)@\spxentry{data\_end}\spxextra{in namelist JULES\_DRIVE}|spxpagem}

\begin{fulllineitems}
\phantomsection\label{\detokenize{namelists/drive.nml:JULES_DRIVE::data_end}}
\pysigstartsignatures
\pysigline{\sphinxcode{\sphinxupquote{JULES\_DRIVE::}}\sphinxbfcode{\sphinxupquote{data\_end}}}
\pysigstopsignatures\begin{quote}\begin{description}
\sphinxlineitem{Type}
\sphinxAtStartPar
character

\sphinxlineitem{Default}
\sphinxAtStartPar
None

\end{description}\end{quote}

\sphinxAtStartPar
The times of the start of the first timestep of data and the end of the last timestep of data.

\sphinxAtStartPar
Each run of JULES (configured in {\hyperref[\detokenize{namelists/timesteps.nml::doc}]{\sphinxcrossref{\DUrole{doc}{timesteps.nml}}}}) can use part or all of the specified data. However, there must be data for all times between run start and run end (determined by {\hyperref[\detokenize{namelists/timesteps.nml:JULES_TIME::main_run_start}]{\sphinxcrossref{\sphinxcode{\sphinxupquote{main\_run\_start}}}}}, {\hyperref[\detokenize{namelists/timesteps.nml:JULES_TIME::main_run_end}]{\sphinxcrossref{\sphinxcode{\sphinxupquote{main\_run\_end}}}}}, {\hyperref[\detokenize{namelists/timesteps.nml:JULES_SPINUP::spinup_start}]{\sphinxcrossref{\sphinxcode{\sphinxupquote{spinup\_start}}}}} and {\hyperref[\detokenize{namelists/timesteps.nml:JULES_SPINUP::spinup_end}]{\sphinxcrossref{\sphinxcode{\sphinxupquote{spinup\_end}}}}}).

\sphinxAtStartPar
The times must be given in the format:

\begin{sphinxVerbatim}[commandchars=\\\{\}]
\PYG{l+s+s2}{\PYGZdq{}yyyy\PYGZhy{}mm\PYGZhy{}dd hh:mm:ss\PYGZdq{}}
\end{sphinxVerbatim}

\end{fulllineitems}

\index{data\_period (in namelist JULES\_DRIVE)@\spxentry{data\_period}\spxextra{in namelist JULES\_DRIVE}|spxpagem}

\begin{fulllineitems}
\phantomsection\label{\detokenize{namelists/drive.nml:JULES_DRIVE::data_period}}
\pysigstartsignatures
\pysigline{\sphinxcode{\sphinxupquote{JULES\_DRIVE::}}\sphinxbfcode{\sphinxupquote{data\_period}}}
\pysigstopsignatures\begin{quote}\begin{description}
\sphinxlineitem{Type}
\sphinxAtStartPar
integer

\sphinxlineitem{Permitted}
\sphinxAtStartPar
\sphinxhyphen{}2, \sphinxhyphen{}1 or \textgreater{} 0

\sphinxlineitem{Default}
\sphinxAtStartPar
None

\end{description}\end{quote}

\sphinxAtStartPar
The period, in seconds, of the data.

\sphinxAtStartPar
Special cases:

\begin{DUlineblock}{0em}
\item[] \sphinxstylestrong{\sphinxhyphen{}1:} Monthly data
\item[] \sphinxstylestrong{\sphinxhyphen{}2:} Yearly data
\end{DUlineblock}

\end{fulllineitems}

\end{sphinxadmonition}

\begin{sphinxadmonition}{note}{Members used to specify the files containing the data}
\index{read\_list (in namelist JULES\_DRIVE)@\spxentry{read\_list}\spxextra{in namelist JULES\_DRIVE}|spxpagem}

\begin{fulllineitems}
\phantomsection\label{\detokenize{namelists/drive.nml:JULES_DRIVE::read_list}}
\pysigstartsignatures
\pysigline{\sphinxcode{\sphinxupquote{JULES\_DRIVE::}}\sphinxbfcode{\sphinxupquote{read\_list}}}
\pysigstopsignatures\begin{quote}\begin{description}
\sphinxlineitem{Type}
\sphinxAtStartPar
logical

\sphinxlineitem{Default}
\sphinxAtStartPar
F

\end{description}\end{quote}

\sphinxAtStartPar
Switch controlling how data file names are determined for a given time.
\begin{description}
\sphinxlineitem{TRUE}
\sphinxAtStartPar
Use a list of data file names with times of first data.

\sphinxlineitem{FALSE}
\sphinxAtStartPar
Use a single data file for all times or a template describing the names of the data files.

\end{description}

\end{fulllineitems}

\index{nfiles (in namelist JULES\_DRIVE)@\spxentry{nfiles}\spxextra{in namelist JULES\_DRIVE}|spxpagem}

\begin{fulllineitems}
\phantomsection\label{\detokenize{namelists/drive.nml:JULES_DRIVE::nfiles}}
\pysigstartsignatures
\pysigline{\sphinxcode{\sphinxupquote{JULES\_DRIVE::}}\sphinxbfcode{\sphinxupquote{nfiles}}}
\pysigstopsignatures\begin{quote}\begin{description}
\sphinxlineitem{Type}
\sphinxAtStartPar
integer

\sphinxlineitem{Permitted}
\sphinxAtStartPar
\textgreater{}= 0

\sphinxlineitem{Default}
\sphinxAtStartPar
0

\end{description}\end{quote}

\sphinxAtStartPar
Only used if {\hyperref[\detokenize{namelists/drive.nml:JULES_DRIVE::read_list}]{\sphinxcrossref{\sphinxcode{\sphinxupquote{read\_list}}}}} = TRUE.

\sphinxAtStartPar
The number of data files to read name and time of first data for.

\end{fulllineitems}

\index{file (in namelist JULES\_DRIVE)@\spxentry{file}\spxextra{in namelist JULES\_DRIVE}|spxpagem}

\begin{fulllineitems}
\phantomsection\label{\detokenize{namelists/drive.nml:JULES_DRIVE::file}}
\pysigstartsignatures
\pysigline{\sphinxcode{\sphinxupquote{JULES\_DRIVE::}}\sphinxbfcode{\sphinxupquote{file}}}
\pysigstopsignatures\begin{quote}\begin{description}
\sphinxlineitem{Type}
\sphinxAtStartPar
character

\sphinxlineitem{Default}
\sphinxAtStartPar
None

\end{description}\end{quote}

\sphinxAtStartPar
If {\hyperref[\detokenize{namelists/drive.nml:JULES_DRIVE::read_list}]{\sphinxcrossref{\sphinxcode{\sphinxupquote{read\_list}}}}} = TRUE, this is the file to read the list of data file names and times from. Each line should be of the form:

\begin{sphinxVerbatim}[commandchars=\\\{\}]
\PYG{l+s+s1}{\PYGZsq{}/data/file\PYGZsq{}}\PYG{p}{,} \PYG{l+s+s1}{\PYGZsq{}yyyy\PYGZhy{}mm\PYGZhy{}dd hh:mm:ss\PYGZsq{}}
\end{sphinxVerbatim}

\sphinxAtStartPar
In this case data file names may contain variable name templating only, with the proviso that either no file names use variable name templating or all file names do. The files must appear in chronological order.

\sphinxAtStartPar
If {\hyperref[\detokenize{namelists/drive.nml:JULES_DRIVE::read_list}]{\sphinxcrossref{\sphinxcode{\sphinxupquote{read\_list}}}}} = FALSE, this is either the single data file (if no templating is used) or a template for data file names. Both {\hyperref[\detokenize{input/file-name-templating::doc}]{\sphinxcrossref{\DUrole{doc}{time and variable name templating}}}} may be used.

\end{fulllineitems}

\end{sphinxadmonition}

\begin{sphinxadmonition}{note}{Members used to specify the provided variables}
\index{nvars (in namelist JULES\_DRIVE)@\spxentry{nvars}\spxextra{in namelist JULES\_DRIVE}|spxpagem}

\begin{fulllineitems}
\phantomsection\label{\detokenize{namelists/drive.nml:JULES_DRIVE::nvars}}
\pysigstartsignatures
\pysigline{\sphinxcode{\sphinxupquote{JULES\_DRIVE::}}\sphinxbfcode{\sphinxupquote{nvars}}}
\pysigstopsignatures\begin{quote}\begin{description}
\sphinxlineitem{Type}
\sphinxAtStartPar
integer

\sphinxlineitem{Permitted}
\sphinxAtStartPar
\textgreater{}= 0

\sphinxlineitem{Default}
\sphinxAtStartPar
0

\end{description}\end{quote}

\sphinxAtStartPar
The number of forcing variables that will be provided.

\sphinxAtStartPar
See {\hyperref[\detokenize{namelists/drive.nml:list-of-jules-forcing-variables}]{\sphinxcrossref{\DUrole{std,std-ref}{List of JULES forcing variables}}}} for the available forcing variables and their possible configurations.

\end{fulllineitems}

\index{var (in namelist JULES\_DRIVE)@\spxentry{var}\spxextra{in namelist JULES\_DRIVE}|spxpagem}

\begin{fulllineitems}
\phantomsection\label{\detokenize{namelists/drive.nml:JULES_DRIVE::var}}
\pysigstartsignatures
\pysigline{\sphinxcode{\sphinxupquote{JULES\_DRIVE::}}\sphinxbfcode{\sphinxupquote{var}}}
\pysigstopsignatures\begin{quote}\begin{description}
\sphinxlineitem{Type}
\sphinxAtStartPar
character(nvars)

\sphinxlineitem{Default}
\sphinxAtStartPar
None

\end{description}\end{quote}

\sphinxAtStartPar
List of forcing variable names as recognised by JULES (see {\hyperref[\detokenize{namelists/drive.nml:list-of-jules-forcing-variables}]{\sphinxcrossref{\DUrole{std,std-ref}{List of JULES forcing variables}}}}). Names are case sensitive.

\begin{sphinxadmonition}{note}{Note:}
\sphinxAtStartPar
For ASCII files, variable names must be in the order they appear in the file.
\end{sphinxadmonition}

\end{fulllineitems}

\index{var\_name (in namelist JULES\_DRIVE)@\spxentry{var\_name}\spxextra{in namelist JULES\_DRIVE}|spxpagem}

\begin{fulllineitems}
\phantomsection\label{\detokenize{namelists/drive.nml:JULES_DRIVE::var_name}}
\pysigstartsignatures
\pysigline{\sphinxcode{\sphinxupquote{JULES\_DRIVE::}}\sphinxbfcode{\sphinxupquote{var\_name}}}
\pysigstopsignatures\begin{quote}\begin{description}
\sphinxlineitem{Type}
\sphinxAtStartPar
character(nvars)

\sphinxlineitem{Default}
\sphinxAtStartPar
‘’ (empty string)

\end{description}\end{quote}

\sphinxAtStartPar
For each JULES variable specified in {\hyperref[\detokenize{namelists/drive.nml:JULES_DRIVE::var}]{\sphinxcrossref{\sphinxcode{\sphinxupquote{var}}}}}, this is the name of the variable in the file(s) containing the data.

\sphinxAtStartPar
If the empty string (the default) is given for any variable, then the corresponding value from {\hyperref[\detokenize{namelists/drive.nml:JULES_DRIVE::var}]{\sphinxcrossref{\sphinxcode{\sphinxupquote{var}}}}} is used instead.

\begin{sphinxadmonition}{note}{Note:}
\sphinxAtStartPar
For ASCII files, this is not used \sphinxhyphen{} only the order in the file matters, as described above.
\end{sphinxadmonition}

\end{fulllineitems}

\index{tpl\_name (in namelist JULES\_DRIVE)@\spxentry{tpl\_name}\spxextra{in namelist JULES\_DRIVE}|spxpagem}

\begin{fulllineitems}
\phantomsection\label{\detokenize{namelists/drive.nml:JULES_DRIVE::tpl_name}}
\pysigstartsignatures
\pysigline{\sphinxcode{\sphinxupquote{JULES\_DRIVE::}}\sphinxbfcode{\sphinxupquote{tpl\_name}}}
\pysigstopsignatures\begin{quote}\begin{description}
\sphinxlineitem{Type}
\sphinxAtStartPar
character(nvars)

\sphinxlineitem{Default}
\sphinxAtStartPar
None

\end{description}\end{quote}

\sphinxAtStartPar
For each JULES variable specified in {\hyperref[\detokenize{namelists/drive.nml:JULES_DRIVE::var}]{\sphinxcrossref{\sphinxcode{\sphinxupquote{var}}}}}, this is the string to substitute into the file name(s) in place of the variable name substitution string.

\sphinxAtStartPar
If the file name(s) do not use variable name templating, this is not used.

\end{fulllineitems}

\index{interp (in namelist JULES\_DRIVE)@\spxentry{interp}\spxextra{in namelist JULES\_DRIVE}|spxpagem}

\begin{fulllineitems}
\phantomsection\label{\detokenize{namelists/drive.nml:JULES_DRIVE::interp}}
\pysigstartsignatures
\pysigline{\sphinxcode{\sphinxupquote{JULES\_DRIVE::}}\sphinxbfcode{\sphinxupquote{interp}}}
\pysigstopsignatures\begin{quote}\begin{description}
\sphinxlineitem{Type}
\sphinxAtStartPar
character(nvars)

\sphinxlineitem{Default}
\sphinxAtStartPar
None

\end{description}\end{quote}

\sphinxAtStartPar
For each JULES variable specified in {\hyperref[\detokenize{namelists/drive.nml:JULES_DRIVE::var}]{\sphinxcrossref{\sphinxcode{\sphinxupquote{var}}}}}, this indicates how the variable is to be interpolated in time (see {\hyperref[\detokenize{input/temporal-interpolation::doc}]{\sphinxcrossref{\DUrole{doc}{Temporal interpolation}}}}).

\end{fulllineitems}

\end{sphinxadmonition}


\subsubsection{List of JULES forcing variables}
\label{\detokenize{namelists/drive.nml:list-of-jules-forcing-variables}}\label{\detokenize{namelists/drive.nml:id1}}
\sphinxAtStartPar
All of the available forcing variables listed in the sections below, are expected to have no levels dimensions, but must have a time dimension called {\hyperref[\detokenize{namelists/model_grid.nml:JULES_INPUT_GRID::time_dim_name}]{\sphinxcrossref{\sphinxcode{\sphinxupquote{time\_dim\_name}}}}}.


\paragraph{Pressure, Humidity and Temperature}
\label{\detokenize{namelists/drive.nml:pressure-humidity-and-temperature}}

\begin{savenotes}\sphinxattablestart
\centering
\begin{tabulary}{\linewidth}[t]{|T|T|}
\hline
\sphinxstyletheadfamily 
\sphinxAtStartPar
Name
&\sphinxstyletheadfamily 
\sphinxAtStartPar
Description
\\
\hline
\sphinxAtStartPar
\sphinxcode{\sphinxupquote{pstar}}
&
\sphinxAtStartPar
Air pressure (Pa).
\\
\hline
\sphinxAtStartPar
\sphinxcode{\sphinxupquote{q}}
&
\sphinxAtStartPar
Specific humidity (kg kg$^{\text{\sphinxhyphen{}1}}$).
\\
\hline
\sphinxAtStartPar
\sphinxcode{\sphinxupquote{t}}
&
\sphinxAtStartPar
Air temperature (K).
\\
\hline
\end{tabulary}
\par
\sphinxattableend\end{savenotes}


\paragraph{Radiation variables}
\label{\detokenize{namelists/drive.nml:radiation-variables}}
\sphinxAtStartPar
The radiation forcing variables can be given in one of four ways:
\begin{description}
\sphinxlineitem{\sphinxcode{\sphinxupquote{sw\_down}} and \sphinxcode{\sphinxupquote{lw\_down}}}
\sphinxAtStartPar
Downward fluxes of short\sphinxhyphen{} and longwave radiation are input. \sphinxstyleemphasis{This is the preferred option.}

\sphinxlineitem{\sphinxcode{\sphinxupquote{rad\_net}} and \sphinxcode{\sphinxupquote{sw\_down}}}
\sphinxAtStartPar
Downward shortwave and net all wavelength (downward is positive) radiation are input. The modelled albedo and surface temperature are used to calculate the downward longwave flux.

\sphinxlineitem{\sphinxcode{\sphinxupquote{lw\_net}} and \sphinxcode{\sphinxupquote{sw\_net}}}
\sphinxAtStartPar
Net downward fluxes of short\sphinxhyphen{} and longwave radiation are input. The modelled albedo and surface temperature are used to calculate the downward fluxes of shortwave and longwave radiation.

\sphinxlineitem{\sphinxcode{\sphinxupquote{lw\_down}} and \sphinxcode{\sphinxupquote{sw\_net}}}
\sphinxAtStartPar
Downward flux of longwave radiation and net downward flux of shortwave radiation are input. The modelled albedo is used to calculate the downward flux of shortwave radiation.

\end{description}

\sphinxAtStartPar
If any of the four combinations of radiation variables listed above are provided, then these are used to drive JULES. There is no default option. JULES will give a fatal error and stop if there are too many, too few or invalid forcing variables provided in the variable list.

\begin{sphinxadmonition}{warning}{Warning:}
\sphinxAtStartPar
If {\hyperref[\detokenize{namelists/drive.nml:JULES_DRIVE::l_daily_disagg}]{\sphinxcrossref{\sphinxcode{\sphinxupquote{l\_daily\_disagg}}}}} = TRUE, then the first method must be used.
\end{sphinxadmonition}

\sphinxAtStartPar
\sphinxcode{\sphinxupquote{diff\_rad}} can be used with any of the four methods. If it is given, diffuse radiation is input from file. If it is not given, {\hyperref[\detokenize{namelists/drive.nml:JULES_DRIVE::diff_frac_const}]{\sphinxcrossref{\sphinxcode{\sphinxupquote{diff\_frac\_const}}}}} is used instead to partition the downward shortwave radiation into diffuse and direct.


\begin{savenotes}\sphinxattablestart
\centering
\begin{tabulary}{\linewidth}[t]{|T|T|}
\hline
\sphinxstyletheadfamily 
\sphinxAtStartPar
Name
&\sphinxstyletheadfamily 
\sphinxAtStartPar
Description
\\
\hline
\sphinxAtStartPar
\sphinxcode{\sphinxupquote{rad\_net}}
&
\sphinxAtStartPar
Net (all wavelength) downward radiation (W m$^{\text{\sphinxhyphen{}2}}$).
\\
\hline
\sphinxAtStartPar
\sphinxcode{\sphinxupquote{lw\_net}}
&
\sphinxAtStartPar
Net downward longwave radiation (W m$^{\text{\sphinxhyphen{}2}}$).
\\
\hline
\sphinxAtStartPar
\sphinxcode{\sphinxupquote{sw\_net}}
&
\sphinxAtStartPar
Net downward shortwave radiation (W m$^{\text{\sphinxhyphen{}2}}$).
\\
\hline
\sphinxAtStartPar
\sphinxcode{\sphinxupquote{lw\_down}}
&
\sphinxAtStartPar
Downward longwave radiation (W m$^{\text{\sphinxhyphen{}2}}$).
\\
\hline
\sphinxAtStartPar
\sphinxcode{\sphinxupquote{sw\_down}}
&
\sphinxAtStartPar
Downward shortwave radiation (W m$^{\text{\sphinxhyphen{}2}}$).
\\
\hline
\sphinxAtStartPar
\sphinxcode{\sphinxupquote{diff\_rad}}
&
\sphinxAtStartPar
Diffuse radiation (W m$^{\text{\sphinxhyphen{}2}}$).
\\
\hline
\end{tabulary}
\par
\sphinxattableend\end{savenotes}


\paragraph{Precipitation variables}
\label{\detokenize{namelists/drive.nml:precipitation-variables}}
\sphinxAtStartPar
The precipitation variables can be specified in one of four ways:
\begin{description}
\sphinxlineitem{\sphinxcode{\sphinxupquote{precip}}}
\sphinxAtStartPar
A single precipitation field is input. This represents the total precipitation (rainfall and snowfall). The total is partitioned between snowfall and rainfall using {\hyperref[\detokenize{namelists/drive.nml:JULES_DRIVE::t_for_snow}]{\sphinxcrossref{\sphinxcode{\sphinxupquote{t\_for\_snow}}}}}, and rainfall is then further partitioned into large\sphinxhyphen{}scale and convective components using {\hyperref[\detokenize{namelists/drive.nml:JULES_DRIVE::t_for_con_rain}]{\sphinxcrossref{\sphinxcode{\sphinxupquote{t\_for\_con\_rain}}}}}. Convective snowfall is assumed to be zero.

\sphinxlineitem{\sphinxcode{\sphinxupquote{tot\_rain}} and \sphinxcode{\sphinxupquote{tot\_snow}}}
\sphinxAtStartPar
Two precipitation fields are input: total rainfall and total snowfall. The rainfall is partitioned between large\sphinxhyphen{}scale and convective, using {\hyperref[\detokenize{namelists/drive.nml:JULES_DRIVE::t_for_con_rain}]{\sphinxcrossref{\sphinxcode{\sphinxupquote{t\_for\_con\_rain}}}}}. Convective snowfall is assumed to be zero.

\sphinxlineitem{\sphinxcode{\sphinxupquote{ls\_rain}}, \sphinxcode{\sphinxupquote{con\_rain}} and \sphinxcode{\sphinxupquote{tot\_snow}}}
\sphinxAtStartPar
Three precipitation fields are input: large\sphinxhyphen{}scale rainfall, convective rainfall and total snowfall. This cannot be used with {\hyperref[\detokenize{namelists/jules_surface.nml:JULES_SURFACE::l_point_data}]{\sphinxcrossref{\sphinxcode{\sphinxupquote{l\_point\_data}}}}} = TRUE. Convective snowfall is assumed to be zero.

\sphinxlineitem{\sphinxcode{\sphinxupquote{ls\_rain}}, \sphinxcode{\sphinxupquote{con\_rain}}, \sphinxcode{\sphinxupquote{ls\_snow}} and \sphinxcode{\sphinxupquote{con\_snow}}}
\sphinxAtStartPar
Four precipitation fields are input: large\sphinxhyphen{}scale rainfall, convective rainfall, large\sphinxhyphen{}scale snowfall and convective snowfall. This cannot be used with {\hyperref[\detokenize{namelists/jules_surface.nml:JULES_SURFACE::l_point_data}]{\sphinxcrossref{\sphinxcode{\sphinxupquote{l\_point\_data}}}}} = TRUE. Note that this is the only option that considers convective snowfall.

\end{description}

\sphinxAtStartPar
If \sphinxcode{\sphinxupquote{precip}} is given, the first method is used. If \sphinxcode{\sphinxupquote{precip}} is \sphinxstyleemphasis{not} given but \sphinxcode{\sphinxupquote{tot\_rain}} is, the second method is used. If \sphinxstyleemphasis{neither} \sphinxcode{\sphinxupquote{precip}} \sphinxstyleemphasis{nor} \sphinxcode{\sphinxupquote{tot\_rain}} are given but \sphinxcode{\sphinxupquote{tot\_snow}} is, the third method is used. The fourth method is used in all other cases.

\sphinxAtStartPar
The concept of convective and large\sphinxhyphen{}scale (or dynamical) components of precipitation comes from atmospheric models, in which the precipitation from small\sphinxhyphen{}scale (convective) and large\sphinxhyphen{}scale motions is often calculated separately. If JULES is to be driven by the output from such a model, the driving data might include these components.

\begin{sphinxadmonition}{warning}{Warning:}
\sphinxAtStartPar
If {\hyperref[\detokenize{namelists/drive.nml:JULES_DRIVE::l_daily_disagg}]{\sphinxcrossref{\sphinxcode{\sphinxupquote{l\_daily\_disagg}}}}} = TRUE, then {\hyperref[\detokenize{namelists/drive.nml:JULES_DRIVE::interp}]{\sphinxcrossref{\sphinxcode{\sphinxupquote{interp}}}}} for each precipitation variable should be \sphinxcode{\sphinxupquote{f}} or \sphinxcode{\sphinxupquote{nf}}.
\end{sphinxadmonition}


\begin{savenotes}\sphinxattablestart
\centering
\begin{tabulary}{\linewidth}[t]{|T|T|}
\hline
\sphinxstyletheadfamily 
\sphinxAtStartPar
Name
&\sphinxstyletheadfamily 
\sphinxAtStartPar
Description
\\
\hline
\sphinxAtStartPar
\sphinxcode{\sphinxupquote{precip}}
&
\sphinxAtStartPar
Precipitation rate (kg m$^{\text{\sphinxhyphen{}2}}$ s$^{\text{\sphinxhyphen{}1}}$).
\\
\hline
\sphinxAtStartPar
\sphinxcode{\sphinxupquote{tot\_rain}}
&
\sphinxAtStartPar
Rainfall rate (kg m$^{\text{\sphinxhyphen{}2}}$ s$^{\text{\sphinxhyphen{}1}}$).
\\
\hline
\sphinxAtStartPar
\sphinxcode{\sphinxupquote{tot\_snow}}
&
\sphinxAtStartPar
Snowfall rate (kg m$^{\text{\sphinxhyphen{}2}}$ s$^{\text{\sphinxhyphen{}1}}$).
\\
\hline
\sphinxAtStartPar
\sphinxcode{\sphinxupquote{ls\_rain}}
&
\sphinxAtStartPar
Large\sphinxhyphen{}scale rainfall rate (kg m$^{\text{\sphinxhyphen{}2}}$ s$^{\text{\sphinxhyphen{}1}}$).
\\
\hline
\sphinxAtStartPar
\sphinxcode{\sphinxupquote{con\_rain}}
&
\sphinxAtStartPar
Convective rainfall rate (kg m$^{\text{\sphinxhyphen{}2}}$ s$^{\text{\sphinxhyphen{}1}}$).
\\
\hline
\sphinxAtStartPar
\sphinxcode{\sphinxupquote{ls\_snow}}
&
\sphinxAtStartPar
Large\sphinxhyphen{}scale snowfall rate (kg m$^{\text{\sphinxhyphen{}2}}$ s$^{\text{\sphinxhyphen{}1}}$).
\\
\hline
\sphinxAtStartPar
\sphinxcode{\sphinxupquote{con\_snow}}
&
\sphinxAtStartPar
Convective snowfall rate (kg m$^{\text{\sphinxhyphen{}2}}$ s$^{\text{\sphinxhyphen{}1}}$).
\\
\hline
\end{tabulary}
\par
\sphinxattableend\end{savenotes}


\paragraph{Wind variables}
\label{\detokenize{namelists/drive.nml:wind-variables}}
\sphinxAtStartPar
The wind variables can be given in one of two ways:
\begin{description}
\sphinxlineitem{\sphinxcode{\sphinxupquote{wind}}}
\sphinxAtStartPar
The wind speed is input.

\sphinxlineitem{\sphinxcode{\sphinxupquote{u}} and \sphinxcode{\sphinxupquote{v}}}
\sphinxAtStartPar
The two components of the horizontal wind (e.g. the southerly and westerly components) are input.

\end{description}

\sphinxAtStartPar
If \sphinxcode{\sphinxupquote{wind}} is given, then the first method is used. The second method is used in all other cases.


\begin{savenotes}\sphinxattablestart
\centering
\begin{tabulary}{\linewidth}[t]{|T|T|}
\hline
\sphinxstyletheadfamily 
\sphinxAtStartPar
Name
&\sphinxstyletheadfamily 
\sphinxAtStartPar
Description
\\
\hline
\sphinxAtStartPar
\sphinxcode{\sphinxupquote{wind}}
&
\sphinxAtStartPar
Total wind speed (m s$^{\text{\sphinxhyphen{}1}}$).
\\
\hline
\sphinxAtStartPar
\sphinxcode{\sphinxupquote{u}}
&
\sphinxAtStartPar
Zonal component of the wind (m s$^{\text{\sphinxhyphen{}1}}$).
\\
\hline
\sphinxAtStartPar
\sphinxcode{\sphinxupquote{v}}
&
\sphinxAtStartPar
Meridional component of the wind (m s$^{\text{\sphinxhyphen{}1}}$).
\\
\hline
\end{tabulary}
\par
\sphinxattableend\end{savenotes}


\paragraph{Daily disaggregator variables}
\label{\detokenize{namelists/drive.nml:daily-disaggregator-variables}}
\sphinxAtStartPar
If {\hyperref[\detokenize{namelists/drive.nml:JULES_DRIVE::l_daily_disagg}]{\sphinxcrossref{\sphinxcode{\sphinxupquote{l\_daily\_disagg}}}}} = TRUE, then the diurnal temperature range is also required:


\begin{savenotes}\sphinxattablestart
\centering
\begin{tabulary}{\linewidth}[t]{|T|T|}
\hline
\sphinxstyletheadfamily 
\sphinxAtStartPar
Name
&\sphinxstyletheadfamily 
\sphinxAtStartPar
Description
\\
\hline
\sphinxAtStartPar
\sphinxcode{\sphinxupquote{dt\_range}}
&
\sphinxAtStartPar
Diurnal temperature range (K).
\\
\hline
\end{tabulary}
\par
\sphinxattableend\end{savenotes}


\subsection{Examples of specifying driving data}
\label{\detokenize{namelists/drive.nml:examples-of-specifying-driving-data}}
\sphinxAtStartPar
The examples below illustrate the use of some of the key settings in the namelist; other settings are omitted for clarity.


\subsubsection{Single point ASCII driving data for one year}
\label{\detokenize{namelists/drive.nml:single-point-ascii-driving-data-for-one-year}}
\begin{sphinxVerbatim}[commandchars=\\\{\}]
\PYG{n+nn}{\PYGZam{}JULES\PYGZus{}DRIVE}

  \PYG{n+nv}{data\PYGZus{}start}  \PYG{o}{=} \PYG{l+s+s1}{\PYGZsq{}1997\PYGZhy{}01\PYGZhy{}01 00:00:00\PYGZsq{}}\PYG{p}{,}
  \PYG{n+nv}{data\PYGZus{}end}    \PYG{o}{=} \PYG{l+s+s1}{\PYGZsq{}1998\PYGZhy{}01\PYGZhy{}01 00:00:00\PYGZsq{}}\PYG{p}{,}
  \PYG{n+nv}{data\PYGZus{}period} \PYG{o}{=} \PYG{l+m+mi}{1800}\PYG{p}{,}

  \PYG{n+nv}{file} \PYG{o}{=} \PYG{l+s+s2}{\PYGZdq{}met\PYGZus{}data.dat\PYGZdq{}}\PYG{p}{,}

  \PYG{n+nv}{nvars}  \PYG{o}{=} \PYG{l+m+mi}{8}\PYG{p}{,}
  \PYG{n+nv}{var}    \PYG{o}{=} \PYG{l+s+s1}{\PYGZsq{}sw\PYGZus{}down\PYGZsq{}}  \PYG{l+s+s1}{\PYGZsq{}lw\PYGZus{}down\PYGZsq{}}  \PYG{l+s+s1}{\PYGZsq{}tot\PYGZus{}rain\PYGZsq{}}  \PYG{l+s+s1}{\PYGZsq{}tot\PYGZus{}snow\PYGZsq{}}   \PYG{l+s+s1}{\PYGZsq{}t\PYGZsq{}}  \PYG{l+s+s1}{\PYGZsq{}wind\PYGZsq{}}  \PYG{l+s+s1}{\PYGZsq{}pstar\PYGZsq{}}   \PYG{l+s+s1}{\PYGZsq{}q\PYGZsq{}}\PYG{p}{,}
  \PYG{n+nv}{interp} \PYG{o}{=}      \PYG{l+s+s1}{\PYGZsq{}nf\PYGZsq{}}       \PYG{l+s+s1}{\PYGZsq{}nf\PYGZsq{}}        \PYG{l+s+s1}{\PYGZsq{}nf\PYGZsq{}}        \PYG{l+s+s1}{\PYGZsq{}nf\PYGZsq{}}  \PYG{l+s+s1}{\PYGZsq{}nf\PYGZsq{}}    \PYG{l+s+s1}{\PYGZsq{}nf\PYGZsq{}}     \PYG{l+s+s1}{\PYGZsq{}nf\PYGZsq{}}  \PYG{l+s+s1}{\PYGZsq{}nf\PYGZsq{}}\PYG{p}{,}

  \PYG{n+nv}{diff\PYGZus{}frac\PYGZus{}const} \PYG{o}{=} \PYG{l+m+mf}{0.1}\PYG{p}{,}
  \PYG{n+nv}{t\PYGZus{}for\PYGZus{}con\PYGZus{}rain}  \PYG{o}{=} \PYG{l+m+mf}{293.15}
\PYG{n+nn}{/}
\end{sphinxVerbatim}

\sphinxAtStartPar
{\hyperref[\detokenize{namelists/drive.nml:JULES_DRIVE::data_start}]{\sphinxcrossref{\sphinxcode{\sphinxupquote{data\_start}}}}}, {\hyperref[\detokenize{namelists/drive.nml:JULES_DRIVE::data_end}]{\sphinxcrossref{\sphinxcode{\sphinxupquote{data\_end}}}}} and {\hyperref[\detokenize{namelists/drive.nml:JULES_DRIVE::data_period}]{\sphinxcrossref{\sphinxcode{\sphinxupquote{data\_period}}}}} specify that the driving dataset provides one year (1997) of half\sphinxhyphen{}hourly data.

\sphinxAtStartPar
{\hyperref[\detokenize{namelists/drive.nml:JULES_DRIVE::read_list}]{\sphinxcrossref{\sphinxcode{\sphinxupquote{read\_list}}}}} is not given, so takes its default value of FALSE. This means that {\hyperref[\detokenize{namelists/drive.nml:JULES_DRIVE::file}]{\sphinxcrossref{\sphinxcode{\sphinxupquote{file}}}}} is used as either the single data file or a file name template. In this case there is no templating, so JULES treats the given file as the single data file for all data times.

\sphinxAtStartPar
\sphinxcode{\sphinxupquote{sw\_down}} and \sphinxcode{\sphinxupquote{lw\_down}} are given, so the first radiation scheme (above) is used.

\sphinxAtStartPar
\sphinxcode{\sphinxupquote{precip}} is not given but \sphinxcode{\sphinxupquote{tot\_rain}} is, so the second precipitation scheme (above) is used. {\hyperref[\detokenize{namelists/drive.nml:JULES_DRIVE::t_for_con_rain}]{\sphinxcrossref{\sphinxcode{\sphinxupquote{t\_for\_con\_rain}}}}} = 293.15K means that rainfall is treated as convective in nature for temperatures at or above that threshold (not used if {\hyperref[\detokenize{namelists/jules_surface.nml:JULES_SURFACE::l_point_data}]{\sphinxcrossref{\sphinxcode{\sphinxupquote{l\_point\_data}}}}} = TRUE).

\sphinxAtStartPar
\sphinxcode{\sphinxupquote{wind}} is given, so total wind speed is used (first scheme above).

\sphinxAtStartPar
\sphinxcode{\sphinxupquote{diff\_rad}} is not given, so the diffuse radiation is calculated as 0.1 (the value of {\hyperref[\detokenize{namelists/drive.nml:JULES_DRIVE::diff_frac_const}]{\sphinxcrossref{\sphinxcode{\sphinxupquote{diff\_frac\_const}}}}}) times the total shortwave radiation.

\sphinxAtStartPar
The driving data file (\sphinxcode{\sphinxupquote{met\_data.dat}}) should look similar to:

\begin{sphinxVerbatim}[commandchars=\\\{\}]
\PYG{c+c1}{\PYGZsh{} solar   long  rain  snow    temp   wind     press      humid}
    \PYG{l+m+mf}{3.3}  \PYG{l+m+mf}{187.8}   \PYG{l+m+mf}{0.0}   \PYG{l+m+mf}{0.0}  \PYG{l+m+mf}{259.10}  \PYG{l+m+mf}{3.610}  \PYG{l+m+mf}{102400.5}  \PYG{l+m+mf}{1.351E\PYGZhy{}03}
   \PYG{l+m+mf}{89.5}  \PYG{l+m+mf}{185.8}   \PYG{l+m+mf}{0.0}   \PYG{l+m+mf}{0.0}  \PYG{l+m+mf}{259.45}  \PYG{l+m+mf}{3.140}  \PYG{l+m+mf}{102401.9}  \PYG{l+m+mf}{1.357E\PYGZhy{}03}
  \PYG{l+m+mf}{142.3}  \PYG{l+m+mf}{186.4}   \PYG{l+m+mf}{0.0}   \PYG{l+m+mf}{0.0}  \PYG{l+m+mf}{259.85}  \PYG{l+m+mf}{2.890}  \PYG{l+m+mf}{102401.0}  \PYG{l+m+mf}{1.369E\PYGZhy{}03}
\PYG{c+c1}{\PYGZsh{} \PYGZhy{}\PYGZhy{}\PYGZhy{}\PYGZhy{}\PYGZhy{} data for later times \PYGZhy{}\PYGZhy{}\PYGZhy{}\PYGZhy{}}
\end{sphinxVerbatim}


\subsubsection{Driving data from NetCDF files with one variable per file}
\label{\detokenize{namelists/drive.nml:driving-data-from-netcdf-files-with-one-variable-per-file}}
\begin{sphinxVerbatim}[commandchars=\\\{\}]
\PYG{n+nn}{\PYGZam{}JULES\PYGZus{}DRIVE}

  \PYG{n+nv}{data\PYGZus{}start}  \PYG{o}{=} \PYG{l+s+s1}{\PYGZsq{}1982\PYGZhy{}07\PYGZhy{}01 03:00:00\PYGZsq{}}\PYG{p}{,}
  \PYG{n+nv}{data\PYGZus{}end}    \PYG{o}{=} \PYG{l+s+s1}{\PYGZsq{}1996\PYGZhy{}01\PYGZhy{}01 00:00:00\PYGZsq{}}\PYG{p}{,}
  \PYG{n+nv}{data\PYGZus{}period} \PYG{o}{=} \PYG{l+m+mi}{10800}\PYG{p}{,}

  \PYG{n+nv}{read\PYGZus{}list} \PYG{o}{=} \PYG{l+s+ss}{T}\PYG{p}{,}
  \PYG{n+nv}{nfiles}    \PYG{o}{=} \PYG{l+m+mi}{162}\PYG{p}{,}

  \PYG{n+nv}{file} \PYG{o}{=} \PYG{l+s+s2}{\PYGZdq{}./file\PYGZus{}list.txt\PYGZdq{}}\PYG{p}{,}

  \PYG{n+nv}{nvars}    \PYG{o}{=} \PYG{l+m+mi}{8}\PYG{p}{,}
  \PYG{n+nv}{var}      \PYG{o}{=} \PYG{l+s+s1}{\PYGZsq{}sw\PYGZus{}down\PYGZsq{}}  \PYG{l+s+s1}{\PYGZsq{}lw\PYGZus{}down\PYGZsq{}}  \PYG{l+s+s1}{\PYGZsq{}tot\PYGZus{}rain\PYGZsq{}}  \PYG{l+s+s1}{\PYGZsq{}tot\PYGZus{}snow\PYGZsq{}}     \PYG{l+s+s1}{\PYGZsq{}t\PYGZsq{}}  \PYG{l+s+s1}{\PYGZsq{}wind\PYGZsq{}}  \PYG{l+s+s1}{\PYGZsq{}pstar\PYGZsq{}}     \PYG{l+s+s1}{\PYGZsq{}q\PYGZsq{}}\PYG{p}{,}
  \PYG{n+nv}{var\PYGZus{}name} \PYG{o}{=}  \PYG{l+s+s1}{\PYGZsq{}SWdown\PYGZsq{}}   \PYG{l+s+s1}{\PYGZsq{}LWdown\PYGZsq{}}     \PYG{l+s+s1}{\PYGZsq{}Rainf\PYGZsq{}}     \PYG{l+s+s1}{\PYGZsq{}Snowf\PYGZsq{}}  \PYG{l+s+s1}{\PYGZsq{}Tair\PYGZsq{}}  \PYG{l+s+s1}{\PYGZsq{}Wind\PYGZsq{}}  \PYG{l+s+s1}{\PYGZsq{}PSurf\PYGZsq{}}  \PYG{l+s+s1}{\PYGZsq{}Qair\PYGZsq{}}\PYG{p}{,}
  \PYG{n+nv}{tpl\PYGZus{}name} \PYG{o}{=}  \PYG{l+s+s1}{\PYGZsq{}SWdown\PYGZsq{}}   \PYG{l+s+s1}{\PYGZsq{}LWdown\PYGZsq{}}     \PYG{l+s+s1}{\PYGZsq{}Rainf\PYGZsq{}}     \PYG{l+s+s1}{\PYGZsq{}Snowf\PYGZsq{}}  \PYG{l+s+s1}{\PYGZsq{}Tair\PYGZsq{}}  \PYG{l+s+s1}{\PYGZsq{}Wind\PYGZsq{}}  \PYG{l+s+s1}{\PYGZsq{}PSurf\PYGZsq{}}  \PYG{l+s+s1}{\PYGZsq{}Qair\PYGZsq{}}\PYG{p}{,}
  \PYG{n+nv}{interp}   \PYG{o}{=}      \PYG{l+s+s1}{\PYGZsq{}nb\PYGZsq{}}       \PYG{l+s+s1}{\PYGZsq{}nb\PYGZsq{}}        \PYG{l+s+s1}{\PYGZsq{}nb\PYGZsq{}}        \PYG{l+s+s1}{\PYGZsq{}nb\PYGZsq{}}     \PYG{l+s+s1}{\PYGZsq{}i\PYGZsq{}}     \PYG{l+s+s1}{\PYGZsq{}i\PYGZsq{}}      \PYG{l+s+s1}{\PYGZsq{}i\PYGZsq{}}     \PYG{l+s+s1}{\PYGZsq{}i\PYGZsq{}}\PYG{p}{,}

  \PYG{n+nv}{diff\PYGZus{}frac\PYGZus{}const} \PYG{o}{=} \PYG{l+m+mf}{0.1}\PYG{p}{,}
  \PYG{n+nv}{t\PYGZus{}for\PYGZus{}con\PYGZus{}rain}  \PYG{o}{=} \PYG{l+m+mf}{293.15}
\PYG{n+nn}{/}
\end{sphinxVerbatim}

\sphinxAtStartPar
In this example, the driving dataset provides 13.5 years of driving data on a 3 hourly timestep.

\sphinxAtStartPar
{\hyperref[\detokenize{namelists/drive.nml:JULES_DRIVE::read_list}]{\sphinxcrossref{\sphinxcode{\sphinxupquote{read\_list}}}}} = TRUE indicates that the names and start times of the data files should be read from \sphinxcode{\sphinxupquote{file\_list.txt}}. The first few lines of this file are:

\begin{sphinxVerbatim}[commandchars=\\\{\}]
\PYG{l+s+s1}{\PYGZsq{}met\PYGZus{}data/\PYGZpc{}vv\PYGZus{}data/\PYGZpc{}vv198207.nc\PYGZsq{}}\PYG{p}{,} \PYG{l+s+s1}{\PYGZsq{}1982\PYGZhy{}07\PYGZhy{}01 03:00:00\PYGZsq{}}
\PYG{l+s+s1}{\PYGZsq{}met\PYGZus{}data/\PYGZpc{}vv\PYGZus{}data/\PYGZpc{}vv198208.nc\PYGZsq{}}\PYG{p}{,} \PYG{l+s+s1}{\PYGZsq{}1982\PYGZhy{}08\PYGZhy{}01 03:00:00\PYGZsq{}}
\PYG{l+s+s1}{\PYGZsq{}met\PYGZus{}data/\PYGZpc{}vv\PYGZus{}data/\PYGZpc{}vv198209.nc\PYGZsq{}}\PYG{p}{,} \PYG{l+s+s1}{\PYGZsq{}1982\PYGZhy{}09\PYGZhy{}01 03:00:00\PYGZsq{}}
\PYG{c+c1}{\PYGZsh{} \PYGZhy{}\PYGZhy{}\PYGZhy{}\PYGZhy{}\PYGZhy{}\PYGZhy{} rest of file not shown \PYGZhy{}\PYGZhy{}\PYGZhy{}\PYGZhy{}\PYGZhy{}}
\end{sphinxVerbatim}

\sphinxAtStartPar
The presence of the variable name templating string in each file name shows that we are using {\hyperref[\detokenize{input/file-name-templating::doc}]{\sphinxcrossref{\DUrole{doc}{variable name templating}}}}. The dates show that we do in fact have monthly files, but we cannot use time templating for these files because the start time of 03H does not conform to the requirements.

\sphinxAtStartPar
Furthermore, files for each variable are stored in separate directories. The values from {\hyperref[\detokenize{namelists/drive.nml:JULES_DRIVE::tpl_name}]{\sphinxcrossref{\sphinxcode{\sphinxupquote{tpl\_name}}}}} will be substituted into the file name templates in place of the substitution string (\sphinxcode{\sphinxupquote{\%vv}}). For example, pressure is held in files with names like \sphinxcode{\sphinxupquote{met\_data/PSurf\_data/PSurf198207.nc}}, and temperature in files like \sphinxcode{\sphinxupquote{met\_data/Tair\_data/Tair198207.nc}}.

\sphinxAtStartPar
The driving variable setup is as the previous example.

\sphinxstepscope


\section{\sphinxstyleliteralintitle{\sphinxupquote{imogen.nml}}}
\label{\detokenize{namelists/imogen.nml:imogen-nml}}\label{\detokenize{namelists/imogen.nml::doc}}
\sphinxAtStartPar
This file contains three namelists called {\hyperref[\detokenize{namelists/imogen.nml:namelist-IMOGEN_ONOFF_SWITCH}]{\sphinxcrossref{\sphinxcode{\sphinxupquote{IMOGEN\_ONOFF\_SWITCH}}}}}, {\hyperref[\detokenize{namelists/imogen.nml:namelist-IMOGEN_RUN_LIST}]{\sphinxcrossref{\sphinxcode{\sphinxupquote{IMOGEN\_RUN\_LIST}}}}} and {\hyperref[\detokenize{namelists/imogen.nml:namelist-IMOGEN_ANLG_VALS_LIST}]{\sphinxcrossref{\sphinxcode{\sphinxupquote{IMOGEN\_ANLG\_VALS\_LIST}}}}}. Values from this section are only used if IMOGEN is enabled. This is done via the following switch: {\hyperref[\detokenize{namelists/imogen.nml:IMOGEN_ONOFF_SWITCH::l_imogen}]{\sphinxcrossref{\sphinxcode{\sphinxupquote{l\_imogen}}}}} = TRUE.

\sphinxAtStartPar
Since IMOGEN calculates the forcing for an entire year at once, an IMOGEN run must have a start time of 00:00:00 on the 1st of January for some year.

\sphinxAtStartPar
IMOGEN is currently restricted to run only on the HadCM3LC grid, i.e. there are 96 x 56 grid cells where each cell has size 3.75 degrees longitude by 2.5 degrees latitude with no Antarctica. This means that:
\begin{itemize}
\item {} 
\sphinxAtStartPar
{\hyperref[\detokenize{namelists/model_grid.nml:JULES_INPUT_GRID::nx}]{\sphinxcrossref{\sphinxcode{\sphinxupquote{nx}}}}} = 96 and {\hyperref[\detokenize{namelists/model_grid.nml:JULES_INPUT_GRID::ny}]{\sphinxcrossref{\sphinxcode{\sphinxupquote{ny}}}}} = 56.

\end{itemize}

\sphinxAtStartPar
IMOGEN also uses its own I/O, so it expects IMOGEN specific files in a different format to JULES \sphinxhyphen{} this may change in the future. Examples of IMOGEN input files can be found in the data provided to run the ‘rose stem’ test suites on supported platforms (e.g. JASMIN).


\sphinxstrong{See also:}
\nopagebreak


\sphinxAtStartPar
References:
\begin{itemize}
\item {} 
\sphinxAtStartPar
Huntingford, C. and P. M. Cox (2000),
An analogue model to derive additional climate change scenarios
from existing GCM simulations, Climate Dynamics 16(8): 575\sphinxhyphen{}586.
\sphinxurl{https://doi.org/10.1007/s003820000067}

\item {} 
\sphinxAtStartPar
Huntingford, C., et al. (2010),
IMOGEN: an intermediate complexity model to evaluate terrestrial
impacts of a changing climate,
Geoscientific Model Development 3(2): 679\sphinxhyphen{}687.
\sphinxurl{https://doi.org/10.5194/gmd-3-679-2010}

\item {} 
\sphinxAtStartPar
Comyn\sphinxhyphen{}Platt, E., et al. (2018),
Carbon budgets for 1.5 and 2 C targets lowered by natural
wetland and permafrost feedbacks,
Nature Geoscience 11(8): 568\sphinxhyphen{}573.
\sphinxurl{https://doi.org/10.1038/s41561-018-0174-9}

\item {} 
\sphinxAtStartPar
Zelazowski, P., et al. (2018),
Climate pattern\sphinxhyphen{}scaling set for an ensemble of 22
GCMs\textendash{}adding uncertainty to the IMOGEN version 2.0 impact system,
Geoscientific Model Development 11.2: 541\sphinxhyphen{}560.
\sphinxurl{https://doi.org/10.5194/gmd-11-541-2018}

\end{itemize}




\subsection{\sphinxstyleliteralintitle{\sphinxupquote{IMOGEN\_ONOFF\_SWITCH}} namelist members}
\label{\detokenize{namelists/imogen.nml:namelist-IMOGEN_ONOFF_SWITCH}}\label{\detokenize{namelists/imogen.nml:imogen-onoff-switch-namelist-members}}\index{IMOGEN\_ONOFF\_SWITCH (namelist)@\spxentry{IMOGEN\_ONOFF\_SWITCH}\spxextra{namelist}|spxpagem}\index{l\_imogen (in namelist IMOGEN\_ONOFF\_SWITCH)@\spxentry{l\_imogen}\spxextra{in namelist IMOGEN\_ONOFF\_SWITCH}|spxpagem}

\begin{fulllineitems}
\phantomsection\label{\detokenize{namelists/imogen.nml:IMOGEN_ONOFF_SWITCH::l_imogen}}
\pysigstartsignatures
\pysigline{\sphinxcode{\sphinxupquote{IMOGEN\_ONOFF\_SWITCH::}}\sphinxbfcode{\sphinxupquote{l\_imogen}}}
\pysigstopsignatures\begin{quote}\begin{description}
\sphinxlineitem{Type}
\sphinxAtStartPar
logical

\sphinxlineitem{Default}
\sphinxAtStartPar
F

\end{description}\end{quote}

\sphinxAtStartPar
Switch for IMOGEN.
\begin{description}
\sphinxlineitem{TRUE}
\sphinxAtStartPar
IMOGEN is used to generate meteorological forcing data.

\sphinxlineitem{FALSE}
\sphinxAtStartPar
No effect.

\end{description}

\begin{sphinxadmonition}{note}{Note:}
\sphinxAtStartPar
If IMOGEN is enabled, at most only {\hyperref[\detokenize{namelists/drive.nml:JULES_DRIVE::z1_tq_vary}]{\sphinxcrossref{\sphinxcode{\sphinxupquote{z1\_tq\_vary}}}}}, {\hyperref[\detokenize{namelists/drive.nml:JULES_DRIVE::z1_tq_in}]{\sphinxcrossref{\sphinxcode{\sphinxupquote{z1\_tq\_in}}}}}, {\hyperref[\detokenize{namelists/drive.nml:JULES_DRIVE::z1_uv_in}]{\sphinxcrossref{\sphinxcode{\sphinxupquote{z1\_uv\_in}}}}}, {\hyperref[\detokenize{namelists/drive.nml:JULES_DRIVE::z1_tq_file}]{\sphinxcrossref{\sphinxcode{\sphinxupquote{z1\_tq\_file}}}}} and {\hyperref[\detokenize{namelists/drive.nml:JULES_DRIVE::z1_tq_var_name}]{\sphinxcrossref{\sphinxcode{\sphinxupquote{z1\_tq\_var\_name}}}}} are used from the {\hyperref[\detokenize{namelists/drive.nml:namelist-JULES_DRIVE}]{\sphinxcrossref{\sphinxcode{\sphinxupquote{JULES\_DRIVE}}}}} namelist.
\end{sphinxadmonition}

\end{fulllineitems}



\subsection{\sphinxstyleliteralintitle{\sphinxupquote{IMOGEN\_RUN\_LIST}} namelist members}
\label{\detokenize{namelists/imogen.nml:namelist-IMOGEN_RUN_LIST}}\label{\detokenize{namelists/imogen.nml:imogen-run-list-namelist-members}}\index{IMOGEN\_RUN\_LIST (namelist)@\spxentry{IMOGEN\_RUN\_LIST}\spxextra{namelist}|spxpagem}\index{co2\_init\_ppmv (in namelist IMOGEN\_RUN\_LIST)@\spxentry{co2\_init\_ppmv}\spxextra{in namelist IMOGEN\_RUN\_LIST}|spxpagem}

\begin{fulllineitems}
\phantomsection\label{\detokenize{namelists/imogen.nml:IMOGEN_RUN_LIST::co2_init_ppmv}}
\pysigstartsignatures
\pysigline{\sphinxcode{\sphinxupquote{IMOGEN\_RUN\_LIST::}}\sphinxbfcode{\sphinxupquote{co2\_init\_ppmv}}}
\pysigstopsignatures\begin{quote}\begin{description}
\sphinxlineitem{Type}
\sphinxAtStartPar
real

\sphinxlineitem{Default}
\sphinxAtStartPar
286.085

\end{description}\end{quote}

\sphinxAtStartPar
Initial CO2 concentration (ppmv).

\end{fulllineitems}

\index{file\_scen\_emits (in namelist IMOGEN\_RUN\_LIST)@\spxentry{file\_scen\_emits}\spxextra{in namelist IMOGEN\_RUN\_LIST}|spxpagem}

\begin{fulllineitems}
\phantomsection\label{\detokenize{namelists/imogen.nml:IMOGEN_RUN_LIST::file_scen_emits}}
\pysigstartsignatures
\pysigline{\sphinxcode{\sphinxupquote{IMOGEN\_RUN\_LIST::}}\sphinxbfcode{\sphinxupquote{file\_scen\_emits}}}
\pysigstopsignatures\begin{quote}\begin{description}
\sphinxlineitem{Type}
\sphinxAtStartPar
character

\sphinxlineitem{Default}
\sphinxAtStartPar
None

\end{description}\end{quote}

\sphinxAtStartPar
If used, file containing CO2 emissions.

\sphinxAtStartPar
This file is expected to be in a specific format \sphinxhyphen{} see the IMOGEN example.

\end{fulllineitems}

\index{file\_non\_co2\_radf (in namelist IMOGEN\_RUN\_LIST)@\spxentry{file\_non\_co2\_radf}\spxextra{in namelist IMOGEN\_RUN\_LIST}|spxpagem}

\begin{fulllineitems}
\phantomsection\label{\detokenize{namelists/imogen.nml:IMOGEN_RUN_LIST::file_non_co2_radf}}
\pysigstartsignatures
\pysigline{\sphinxcode{\sphinxupquote{IMOGEN\_RUN\_LIST::}}\sphinxbfcode{\sphinxupquote{file\_non\_co2\_radf}}}
\pysigstopsignatures\begin{quote}\begin{description}
\sphinxlineitem{Type}
\sphinxAtStartPar
character

\sphinxlineitem{Default}
\sphinxAtStartPar
None

\end{description}\end{quote}

\sphinxAtStartPar
If used, file containing non\sphinxhyphen{}CO2 radiative forcing values.

\sphinxAtStartPar
This file is expected to be in a specific format \sphinxhyphen{} see the IMOGEN example.

\end{fulllineitems}

\index{nyr\_non\_co2 (in namelist IMOGEN\_RUN\_LIST)@\spxentry{nyr\_non\_co2}\spxextra{in namelist IMOGEN\_RUN\_LIST}|spxpagem}

\begin{fulllineitems}
\phantomsection\label{\detokenize{namelists/imogen.nml:IMOGEN_RUN_LIST::nyr_non_co2}}
\pysigstartsignatures
\pysigline{\sphinxcode{\sphinxupquote{IMOGEN\_RUN\_LIST::}}\sphinxbfcode{\sphinxupquote{nyr\_non\_co2}}}
\pysigstopsignatures\begin{quote}\begin{description}
\sphinxlineitem{Type}
\sphinxAtStartPar
integer

\sphinxlineitem{Default}
\sphinxAtStartPar
21

\end{description}\end{quote}

\sphinxAtStartPar
Number of years for which non\sphinxhyphen{}co2 forcing is prescribed.

\end{fulllineitems}

\index{file\_scen\_co2\_ppmv (in namelist IMOGEN\_RUN\_LIST)@\spxentry{file\_scen\_co2\_ppmv}\spxextra{in namelist IMOGEN\_RUN\_LIST}|spxpagem}

\begin{fulllineitems}
\phantomsection\label{\detokenize{namelists/imogen.nml:IMOGEN_RUN_LIST::file_scen_co2_ppmv}}
\pysigstartsignatures
\pysigline{\sphinxcode{\sphinxupquote{IMOGEN\_RUN\_LIST::}}\sphinxbfcode{\sphinxupquote{file\_scen\_co2\_ppmv}}}
\pysigstopsignatures\begin{quote}\begin{description}
\sphinxlineitem{Type}
\sphinxAtStartPar
character

\sphinxlineitem{Default}
\sphinxAtStartPar
None

\end{description}\end{quote}

\sphinxAtStartPar
If used, file containing CO2 concentration (ppmv).

\sphinxAtStartPar
This file is expected to be in a specific format \sphinxhyphen{} see the IMOGEN example.

\end{fulllineitems}

\index{ch4\_init\_ppbv (in namelist IMOGEN\_RUN\_LIST)@\spxentry{ch4\_init\_ppbv}\spxextra{in namelist IMOGEN\_RUN\_LIST}|spxpagem}

\begin{fulllineitems}
\phantomsection\label{\detokenize{namelists/imogen.nml:IMOGEN_RUN_LIST::ch4_init_ppbv}}
\pysigstartsignatures
\pysigline{\sphinxcode{\sphinxupquote{IMOGEN\_RUN\_LIST::}}\sphinxbfcode{\sphinxupquote{ch4\_init\_ppbv}}}
\pysigstopsignatures\begin{quote}\begin{description}
\sphinxlineitem{Type}
\sphinxAtStartPar
real

\sphinxlineitem{Default}
\sphinxAtStartPar
774.1

\end{description}\end{quote}

\sphinxAtStartPar
Initial CH4 concentration (ppbv).

\sphinxAtStartPar
Only if {\hyperref[\detokenize{namelists/imogen.nml:IMOGEN_RUN_LIST::land_feed_ch4}]{\sphinxcrossref{\sphinxcode{\sphinxupquote{land\_feed\_ch4}}}}} = TRUE.

\end{fulllineitems}

\index{yr\_fch4\_ref (in namelist IMOGEN\_RUN\_LIST)@\spxentry{yr\_fch4\_ref}\spxextra{in namelist IMOGEN\_RUN\_LIST}|spxpagem}

\begin{fulllineitems}
\phantomsection\label{\detokenize{namelists/imogen.nml:IMOGEN_RUN_LIST::yr_fch4_ref}}
\pysigstartsignatures
\pysigline{\sphinxcode{\sphinxupquote{IMOGEN\_RUN\_LIST::}}\sphinxbfcode{\sphinxupquote{yr\_fch4\_ref}}}
\pysigstopsignatures\begin{quote}\begin{description}
\sphinxlineitem{Type}
\sphinxAtStartPar
real

\sphinxlineitem{Default}
\sphinxAtStartPar
2000

\end{description}\end{quote}

\sphinxAtStartPar
Year for reference wetland CH4 emissions and atmospheric CH4 decay rate, i.e. {\hyperref[\detokenize{namelists/imogen.nml:IMOGEN_RUN_LIST::fch4_ref}]{\sphinxcrossref{\sphinxcode{\sphinxupquote{fch4\_ref}}}}}, {\hyperref[\detokenize{namelists/imogen.nml:IMOGEN_RUN_LIST::tau_ch4_ref}]{\sphinxcrossref{\sphinxcode{\sphinxupquote{tau\_ch4\_ref}}}}} \& {\hyperref[\detokenize{namelists/imogen.nml:IMOGEN_RUN_LIST::ch4_ppbv_ref}]{\sphinxcrossref{\sphinxcode{\sphinxupquote{ch4\_ppbv\_ref}}}}}.

\sphinxAtStartPar
Only if {\hyperref[\detokenize{namelists/imogen.nml:IMOGEN_RUN_LIST::land_feed_ch4}]{\sphinxcrossref{\sphinxcode{\sphinxupquote{land\_feed\_ch4}}}}} = TRUE.

\end{fulllineitems}

\index{ch4\_ppbv\_ref (in namelist IMOGEN\_RUN\_LIST)@\spxentry{ch4\_ppbv\_ref}\spxextra{in namelist IMOGEN\_RUN\_LIST}|spxpagem}

\begin{fulllineitems}
\phantomsection\label{\detokenize{namelists/imogen.nml:IMOGEN_RUN_LIST::ch4_ppbv_ref}}
\pysigstartsignatures
\pysigline{\sphinxcode{\sphinxupquote{IMOGEN\_RUN\_LIST::}}\sphinxbfcode{\sphinxupquote{ch4\_ppbv\_ref}}}
\pysigstopsignatures\begin{quote}\begin{description}
\sphinxlineitem{Type}
\sphinxAtStartPar
real

\sphinxlineitem{Default}
\sphinxAtStartPar
1751.02

\end{description}\end{quote}

\sphinxAtStartPar
Reference atmosphere CH4 concentration at {\hyperref[\detokenize{namelists/imogen.nml:IMOGEN_RUN_LIST::yr_fch4_ref}]{\sphinxcrossref{\sphinxcode{\sphinxupquote{yr\_fch4\_ref}}}}} (ppbv).

\sphinxAtStartPar
Only if {\hyperref[\detokenize{namelists/imogen.nml:IMOGEN_RUN_LIST::land_feed_ch4}]{\sphinxcrossref{\sphinxcode{\sphinxupquote{land\_feed\_ch4}}}}} = TRUE.

\end{fulllineitems}

\index{tau\_ch4\_ref (in namelist IMOGEN\_RUN\_LIST)@\spxentry{tau\_ch4\_ref}\spxextra{in namelist IMOGEN\_RUN\_LIST}|spxpagem}

\begin{fulllineitems}
\phantomsection\label{\detokenize{namelists/imogen.nml:IMOGEN_RUN_LIST::tau_ch4_ref}}
\pysigstartsignatures
\pysigline{\sphinxcode{\sphinxupquote{IMOGEN\_RUN\_LIST::}}\sphinxbfcode{\sphinxupquote{tau\_ch4\_ref}}}
\pysigstopsignatures\begin{quote}\begin{description}
\sphinxlineitem{Type}
\sphinxAtStartPar
real

\sphinxlineitem{Default}
\sphinxAtStartPar
8.4

\end{description}\end{quote}

\sphinxAtStartPar
Lifetime of CH4 in atmosphere at {\hyperref[\detokenize{namelists/imogen.nml:IMOGEN_RUN_LIST::yr_fch4_ref}]{\sphinxcrossref{\sphinxcode{\sphinxupquote{yr\_fch4\_ref}}}}} (years). Value used in Gedney et al. (2004) S3 (Table 1) from TAR, Table 4.3 (subscript d).

\sphinxAtStartPar
Only if {\hyperref[\detokenize{namelists/imogen.nml:IMOGEN_RUN_LIST::land_feed_ch4}]{\sphinxcrossref{\sphinxcode{\sphinxupquote{land\_feed\_ch4}}}}} = TRUE.

\end{fulllineitems}

\index{fch4\_ref (in namelist IMOGEN\_RUN\_LIST)@\spxentry{fch4\_ref}\spxextra{in namelist IMOGEN\_RUN\_LIST}|spxpagem}

\begin{fulllineitems}
\phantomsection\label{\detokenize{namelists/imogen.nml:IMOGEN_RUN_LIST::fch4_ref}}
\pysigstartsignatures
\pysigline{\sphinxcode{\sphinxupquote{IMOGEN\_RUN\_LIST::}}\sphinxbfcode{\sphinxupquote{fch4\_ref}}}
\pysigstopsignatures\begin{quote}\begin{description}
\sphinxlineitem{Type}
\sphinxAtStartPar
real

\sphinxlineitem{Default}
\sphinxAtStartPar
180.0

\end{description}\end{quote}

\sphinxAtStartPar
Reference wetland CH4 emissions for reference year {\hyperref[\detokenize{namelists/imogen.nml:IMOGEN_RUN_LIST::yr_fch4_ref}]{\sphinxcrossref{\sphinxcode{\sphinxupquote{yr\_fch4\_ref}}}}} (Tg CH4/yr).

\sphinxAtStartPar
Only if {\hyperref[\detokenize{namelists/imogen.nml:IMOGEN_RUN_LIST::land_feed_ch4}]{\sphinxcrossref{\sphinxcode{\sphinxupquote{land\_feed\_ch4}}}}} = TRUE.

\end{fulllineitems}

\index{file\_ch4\_n2o (in namelist IMOGEN\_RUN\_LIST)@\spxentry{file\_ch4\_n2o}\spxextra{in namelist IMOGEN\_RUN\_LIST}|spxpagem}

\begin{fulllineitems}
\phantomsection\label{\detokenize{namelists/imogen.nml:IMOGEN_RUN_LIST::file_ch4_n2o}}
\pysigstartsignatures
\pysigline{\sphinxcode{\sphinxupquote{IMOGEN\_RUN\_LIST::}}\sphinxbfcode{\sphinxupquote{file\_ch4\_n2o}}}
\pysigstopsignatures\begin{quote}\begin{description}
\sphinxlineitem{Type}
\sphinxAtStartPar
character

\sphinxlineitem{Default}
\sphinxAtStartPar
None

\end{description}\end{quote}

\sphinxAtStartPar
File containing the CH4 and N2O atmos concs. The number of years in this file is defined by {\hyperref[\detokenize{namelists/imogen.nml:IMOGEN_RUN_LIST::nyr_ch4_n2o}]{\sphinxcrossref{\sphinxcode{\sphinxupquote{nyr\_ch4\_n2o}}}}}. This file is expected to be an ascii file with three columns: the first column is the year, the second column is the CH4 concentration (ppbv) and the third column is the N2O concentration (ppbv). There is one row for each year and no header.

\sphinxAtStartPar
Only if {\hyperref[\detokenize{namelists/imogen.nml:IMOGEN_RUN_LIST::land_feed_ch4}]{\sphinxcrossref{\sphinxcode{\sphinxupquote{land\_feed\_ch4}}}}} = TRUE.

\end{fulllineitems}

\index{nyr\_ch4\_n2o (in namelist IMOGEN\_RUN\_LIST)@\spxentry{nyr\_ch4\_n2o}\spxextra{in namelist IMOGEN\_RUN\_LIST}|spxpagem}

\begin{fulllineitems}
\phantomsection\label{\detokenize{namelists/imogen.nml:IMOGEN_RUN_LIST::nyr_ch4_n2o}}
\pysigstartsignatures
\pysigline{\sphinxcode{\sphinxupquote{IMOGEN\_RUN\_LIST::}}\sphinxbfcode{\sphinxupquote{nyr\_ch4\_n2o}}}
\pysigstopsignatures\begin{quote}\begin{description}
\sphinxlineitem{Type}
\sphinxAtStartPar
integer

\sphinxlineitem{Default}
\sphinxAtStartPar
241

\end{description}\end{quote}

\sphinxAtStartPar
Number of years of CH4 and N2O data in {\hyperref[\detokenize{namelists/imogen.nml:IMOGEN_RUN_LIST::file_ch4_n2o}]{\sphinxcrossref{\sphinxcode{\sphinxupquote{file\_ch4\_n2o}}}}}.

\sphinxAtStartPar
Only if {\hyperref[\detokenize{namelists/imogen.nml:IMOGEN_RUN_LIST::land_feed_ch4}]{\sphinxcrossref{\sphinxcode{\sphinxupquote{land\_feed\_ch4}}}}} = TRUE.

\end{fulllineitems}

\index{anlg (in namelist IMOGEN\_RUN\_LIST)@\spxentry{anlg}\spxextra{in namelist IMOGEN\_RUN\_LIST}|spxpagem}

\begin{fulllineitems}
\phantomsection\label{\detokenize{namelists/imogen.nml:IMOGEN_RUN_LIST::anlg}}
\pysigstartsignatures
\pysigline{\sphinxcode{\sphinxupquote{IMOGEN\_RUN\_LIST::}}\sphinxbfcode{\sphinxupquote{anlg}}}
\pysigstopsignatures\begin{quote}\begin{description}
\sphinxlineitem{Type}
\sphinxAtStartPar
logical

\sphinxlineitem{Default}
\sphinxAtStartPar
T

\end{description}\end{quote}

\sphinxAtStartPar
If TRUE, then use the GCM analogue model.

\end{fulllineitems}

\index{anom (in namelist IMOGEN\_RUN\_LIST)@\spxentry{anom}\spxextra{in namelist IMOGEN\_RUN\_LIST}|spxpagem}

\begin{fulllineitems}
\phantomsection\label{\detokenize{namelists/imogen.nml:IMOGEN_RUN_LIST::anom}}
\pysigstartsignatures
\pysigline{\sphinxcode{\sphinxupquote{IMOGEN\_RUN\_LIST::}}\sphinxbfcode{\sphinxupquote{anom}}}
\pysigstopsignatures\begin{quote}\begin{description}
\sphinxlineitem{Type}
\sphinxAtStartPar
logical

\sphinxlineitem{Default}
\sphinxAtStartPar
T

\end{description}\end{quote}

\sphinxAtStartPar
If TRUE, then incorporate anomalies.

\end{fulllineitems}

\index{c\_emissions (in namelist IMOGEN\_RUN\_LIST)@\spxentry{c\_emissions}\spxextra{in namelist IMOGEN\_RUN\_LIST}|spxpagem}

\begin{fulllineitems}
\phantomsection\label{\detokenize{namelists/imogen.nml:IMOGEN_RUN_LIST::c_emissions}}
\pysigstartsignatures
\pysigline{\sphinxcode{\sphinxupquote{IMOGEN\_RUN\_LIST::}}\sphinxbfcode{\sphinxupquote{c\_emissions}}}
\pysigstopsignatures\begin{quote}\begin{description}
\sphinxlineitem{Type}
\sphinxAtStartPar
logical

\sphinxlineitem{Default}
\sphinxAtStartPar
T

\end{description}\end{quote}

\sphinxAtStartPar
If TRUE, CO2 concentration is calculated.

\end{fulllineitems}

\index{include\_co2 (in namelist IMOGEN\_RUN\_LIST)@\spxentry{include\_co2}\spxextra{in namelist IMOGEN\_RUN\_LIST}|spxpagem}

\begin{fulllineitems}
\phantomsection\label{\detokenize{namelists/imogen.nml:IMOGEN_RUN_LIST::include_co2}}
\pysigstartsignatures
\pysigline{\sphinxcode{\sphinxupquote{IMOGEN\_RUN\_LIST::}}\sphinxbfcode{\sphinxupquote{include\_co2}}}
\pysigstopsignatures\begin{quote}\begin{description}
\sphinxlineitem{Type}
\sphinxAtStartPar
logical

\sphinxlineitem{Default}
\sphinxAtStartPar
T

\end{description}\end{quote}

\sphinxAtStartPar
If TRUE, include adjustments to CO2 values.

\end{fulllineitems}

\index{include\_non\_co2\_radf (in namelist IMOGEN\_RUN\_LIST)@\spxentry{include\_non\_co2\_radf}\spxextra{in namelist IMOGEN\_RUN\_LIST}|spxpagem}

\begin{fulllineitems}
\phantomsection\label{\detokenize{namelists/imogen.nml:IMOGEN_RUN_LIST::include_non_co2_radf}}
\pysigstartsignatures
\pysigline{\sphinxcode{\sphinxupquote{IMOGEN\_RUN\_LIST::}}\sphinxbfcode{\sphinxupquote{include\_non\_co2\_radf}}}
\pysigstopsignatures\begin{quote}\begin{description}
\sphinxlineitem{Type}
\sphinxAtStartPar
logical

\sphinxlineitem{Default}
\sphinxAtStartPar
T

\end{description}\end{quote}

\sphinxAtStartPar
If TRUE, include adjustments to non\sphinxhyphen{}CO2 radiative forcing.

\end{fulllineitems}

\index{l\_drive\_with\_global\_temps (in namelist IMOGEN\_RUN\_LIST)@\spxentry{l\_drive\_with\_global\_temps}\spxextra{in namelist IMOGEN\_RUN\_LIST}|spxpagem}

\begin{fulllineitems}
\phantomsection\label{\detokenize{namelists/imogen.nml:IMOGEN_RUN_LIST::l_drive_with_global_temps}}
\pysigstartsignatures
\pysigline{\sphinxcode{\sphinxupquote{IMOGEN\_RUN\_LIST::}}\sphinxbfcode{\sphinxupquote{l\_drive\_with\_global\_temps}}}
\pysigstopsignatures\begin{quote}\begin{description}
\sphinxlineitem{Type}
\sphinxAtStartPar
logical

\sphinxlineitem{Default}
\sphinxAtStartPar
F

\end{description}\end{quote}

\sphinxAtStartPar
If TRUE, use imogen to provide jules forcing based on the global mean temperature change and the climate patterns.

\end{fulllineitems}

\index{land\_feed\_co2 (in namelist IMOGEN\_RUN\_LIST)@\spxentry{land\_feed\_co2}\spxextra{in namelist IMOGEN\_RUN\_LIST}|spxpagem}

\begin{fulllineitems}
\phantomsection\label{\detokenize{namelists/imogen.nml:IMOGEN_RUN_LIST::land_feed_co2}}
\pysigstartsignatures
\pysigline{\sphinxcode{\sphinxupquote{IMOGEN\_RUN\_LIST::}}\sphinxbfcode{\sphinxupquote{land\_feed\_co2}}}
\pysigstopsignatures\begin{quote}\begin{description}
\sphinxlineitem{Type}
\sphinxAtStartPar
logical

\sphinxlineitem{Default}
\sphinxAtStartPar
F

\end{description}\end{quote}

\sphinxAtStartPar
If TRUE, include land CO2 feedbacks on atmospheric CO2.

\end{fulllineitems}

\index{land\_feed\_ch4 (in namelist IMOGEN\_RUN\_LIST)@\spxentry{land\_feed\_ch4}\spxextra{in namelist IMOGEN\_RUN\_LIST}|spxpagem}

\begin{fulllineitems}
\phantomsection\label{\detokenize{namelists/imogen.nml:IMOGEN_RUN_LIST::land_feed_ch4}}
\pysigstartsignatures
\pysigline{\sphinxcode{\sphinxupquote{IMOGEN\_RUN\_LIST::}}\sphinxbfcode{\sphinxupquote{land\_feed\_ch4}}}
\pysigstopsignatures\begin{quote}\begin{description}
\sphinxlineitem{Type}
\sphinxAtStartPar
logical

\sphinxlineitem{Default}
\sphinxAtStartPar
F

\end{description}\end{quote}

\sphinxAtStartPar
If TRUE, include wetland CH4 feedbacks on atmospheric CH4. Prescribed CH4 concentrations assume a non\sphinxhyphen{}varying natural wetland CH4 component. However, when {\hyperref[\detokenize{namelists/imogen.nml:IMOGEN_RUN_LIST::land_feed_ch4}]{\sphinxcrossref{\sphinxcode{\sphinxupquote{land\_feed\_ch4}}}}} = TRUE the constant wetland CH4 emissions are perturbed using the anomaly in modelled natural wetland CH4 emission. The methane emissions are calculated for the diagnosed wetland area when {\hyperref[\detokenize{namelists/jules_hydrology.nml:JULES_HYDROLOGY::l_top}]{\sphinxcrossref{\sphinxcode{\sphinxupquote{l\_top}}}}} = TRUE. These are accumulated and passed to IMOGEN.

\sphinxAtStartPar
To ensure consistency with the observed atmospheric CH4 growth rate the model needs to be calibrated to produce {\hyperref[\detokenize{namelists/imogen.nml:IMOGEN_RUN_LIST::fch4_ref}]{\sphinxcrossref{\sphinxcode{\sphinxupquote{fch4\_ref}}}}} TgCh4 per year (default 180) for the year  {\hyperref[\detokenize{namelists/imogen.nml:IMOGEN_RUN_LIST::yr_fch4_ref}]{\sphinxcrossref{\sphinxcode{\sphinxupquote{yr\_fch4\_ref}}}}} (default 2000). This is done by calibrating q10\_ch4 (either {\hyperref[\detokenize{namelists/jules_soil_biogeochem.nml:JULES_SOIL_BIOGEOCHEM::q10_ch4_cs}]{\sphinxcrossref{\sphinxcode{\sphinxupquote{q10\_ch4\_cs}}}}}, {\hyperref[\detokenize{namelists/jules_soil_biogeochem.nml:JULES_SOIL_BIOGEOCHEM::q10_ch4_npp}]{\sphinxcrossref{\sphinxcode{\sphinxupquote{q10\_ch4\_npp}}}}}, {\hyperref[\detokenize{namelists/jules_soil_biogeochem.nml:JULES_SOIL_BIOGEOCHEM::q10_ch4_resps}]{\sphinxcrossref{\sphinxcode{\sphinxupquote{q10\_ch4\_resps}}}}}, depending on whether cs, npp or resps is defined as the substrate by {\hyperref[\detokenize{namelists/jules_soil_biogeochem.nml:JULES_SOIL_BIOGEOCHEM::ch4_substrate}]{\sphinxcrossref{\sphinxcode{\sphinxupquote{ch4\_substrate}}}}}) and const\_ch4 (either {\hyperref[\detokenize{namelists/jules_soil_biogeochem.nml:JULES_SOIL_BIOGEOCHEM::const_ch4_cs}]{\sphinxcrossref{\sphinxcode{\sphinxupquote{const\_ch4\_cs}}}}}, {\hyperref[\detokenize{namelists/jules_soil_biogeochem.nml:JULES_SOIL_BIOGEOCHEM::const_ch4_npp}]{\sphinxcrossref{\sphinxcode{\sphinxupquote{const\_ch4\_npp}}}}}, {\hyperref[\detokenize{namelists/jules_soil_biogeochem.nml:JULES_SOIL_BIOGEOCHEM::const_ch4_resps}]{\sphinxcrossref{\sphinxcode{\sphinxupquote{const\_ch4\_resps}}}}}, again depending on whether cs, npp or resps is defined as the substrate  by {\hyperref[\detokenize{namelists/jules_soil_biogeochem.nml:JULES_SOIL_BIOGEOCHEM::ch4_substrate}]{\sphinxcrossref{\sphinxcode{\sphinxupquote{ch4\_substrate}}}}}). The calibration can be carried out as discussed in Comyn\sphinxhyphen{}Platt et al. (2018) and needs to be checked before proceeding because the model won’t necessarily produce the correct values by default.

\sphinxAtStartPar
For wetland CH4 feedbacks values for the following: {\hyperref[\detokenize{namelists/imogen.nml:IMOGEN_RUN_LIST::fch4_ref}]{\sphinxcrossref{\sphinxcode{\sphinxupquote{fch4\_ref}}}}}, {\hyperref[\detokenize{namelists/imogen.nml:IMOGEN_RUN_LIST::tau_ch4_ref}]{\sphinxcrossref{\sphinxcode{\sphinxupquote{tau\_ch4\_ref}}}}}, {\hyperref[\detokenize{namelists/imogen.nml:IMOGEN_RUN_LIST::ch4_ppbv_ref}]{\sphinxcrossref{\sphinxcode{\sphinxupquote{ch4\_ppbv\_ref}}}}}, {\hyperref[\detokenize{namelists/imogen.nml:IMOGEN_RUN_LIST::yr_fch4_ref}]{\sphinxcrossref{\sphinxcode{\sphinxupquote{yr\_fch4\_ref}}}}}, {\hyperref[\detokenize{namelists/imogen.nml:IMOGEN_RUN_LIST::ch4_init_ppbv}]{\sphinxcrossref{\sphinxcode{\sphinxupquote{ch4\_init\_ppbv}}}}}, {\hyperref[\detokenize{namelists/imogen.nml:IMOGEN_RUN_LIST::file_ch4_n2o}]{\sphinxcrossref{\sphinxcode{\sphinxupquote{file\_ch4\_n2o}}}}}, and {\hyperref[\detokenize{namelists/imogen.nml:IMOGEN_RUN_LIST::nyr_ch4_n2o}]{\sphinxcrossref{\sphinxcode{\sphinxupquote{nyr\_ch4\_n2o}}}}} are also required.


\sphinxstrong{See also:}
\nopagebreak


\sphinxAtStartPar
References:
\begin{itemize}
\item {} 
\sphinxAtStartPar
Gedney, N., Cox, P. M. \& Huntingford, C. Climate feedback from wetland methane emissions. Geophys. Res. Lett. 31, L20503 (2004). \sphinxurl{https://doi.org/10.1029/2004GL020919}

\item {} 
\sphinxAtStartPar
Comyn\sphinxhyphen{}Platt, E., et al. (2018),
Carbon budgets for 1.5 and 2 C targets lowered by natural
wetland and permafrost feedbacks,
Nature Geoscience 11(8): 568\sphinxhyphen{}573.
\sphinxurl{https://doi.org/10.1038/s41561-018-0174-9}

\end{itemize}



\end{fulllineitems}

\index{ocean\_feed (in namelist IMOGEN\_RUN\_LIST)@\spxentry{ocean\_feed}\spxextra{in namelist IMOGEN\_RUN\_LIST}|spxpagem}

\begin{fulllineitems}
\phantomsection\label{\detokenize{namelists/imogen.nml:IMOGEN_RUN_LIST::ocean_feed}}
\pysigstartsignatures
\pysigline{\sphinxcode{\sphinxupquote{IMOGEN\_RUN\_LIST::}}\sphinxbfcode{\sphinxupquote{ocean\_feed}}}
\pysigstopsignatures\begin{quote}\begin{description}
\sphinxlineitem{Type}
\sphinxAtStartPar
logical

\sphinxlineitem{Default}
\sphinxAtStartPar
F

\end{description}\end{quote}

\sphinxAtStartPar
If TRUE, include ocean feedbacks on atmospheric CO2.

\end{fulllineitems}

\index{nyr\_emiss (in namelist IMOGEN\_RUN\_LIST)@\spxentry{nyr\_emiss}\spxextra{in namelist IMOGEN\_RUN\_LIST}|spxpagem}

\begin{fulllineitems}
\phantomsection\label{\detokenize{namelists/imogen.nml:IMOGEN_RUN_LIST::nyr_emiss}}
\pysigstartsignatures
\pysigline{\sphinxcode{\sphinxupquote{IMOGEN\_RUN\_LIST::}}\sphinxbfcode{\sphinxupquote{nyr\_emiss}}}
\pysigstopsignatures\begin{quote}\begin{description}
\sphinxlineitem{Type}
\sphinxAtStartPar
integer

\sphinxlineitem{Default}
\sphinxAtStartPar
241

\end{description}\end{quote}

\sphinxAtStartPar
Number of years of emission data in file.

\end{fulllineitems}

\index{file\_points\_order (in namelist IMOGEN\_RUN\_LIST)@\spxentry{file\_points\_order}\spxextra{in namelist IMOGEN\_RUN\_LIST}|spxpagem}

\begin{fulllineitems}
\phantomsection\label{\detokenize{namelists/imogen.nml:IMOGEN_RUN_LIST::file_points_order}}
\pysigstartsignatures
\pysigline{\sphinxcode{\sphinxupquote{IMOGEN\_RUN\_LIST::}}\sphinxbfcode{\sphinxupquote{file\_points\_order}}}
\pysigstopsignatures\begin{quote}\begin{description}
\sphinxlineitem{Type}
\sphinxAtStartPar
character

\sphinxlineitem{Default}
\sphinxAtStartPar
None

\end{description}\end{quote}

\sphinxAtStartPar
File containing the mapping of IMOGEN global grid points onto IMOGEN land points (different from the JULES land points).

\end{fulllineitems}

\index{initialise\_from\_dump (in namelist IMOGEN\_RUN\_LIST)@\spxentry{initialise\_from\_dump}\spxextra{in namelist IMOGEN\_RUN\_LIST}|spxpagem}

\begin{fulllineitems}
\phantomsection\label{\detokenize{namelists/imogen.nml:IMOGEN_RUN_LIST::initialise_from_dump}}
\pysigstartsignatures
\pysigline{\sphinxcode{\sphinxupquote{IMOGEN\_RUN\_LIST::}}\sphinxbfcode{\sphinxupquote{initialise\_from\_dump}}}
\pysigstopsignatures\begin{quote}\begin{description}
\sphinxlineitem{Type}
\sphinxAtStartPar
logical

\sphinxlineitem{Default}
\sphinxAtStartPar
F

\end{description}\end{quote}

\sphinxAtStartPar
Indicates how the IMOGEN prognostic variables will be initialised.
\begin{description}
\sphinxlineitem{TRUE}
\sphinxAtStartPar
Use a dump file (specified in {\hyperref[\detokenize{namelists/imogen.nml:IMOGEN_RUN_LIST::dump_file}]{\sphinxcrossref{\sphinxcode{\sphinxupquote{dump\_file}}}}} below) from a previous run with IMOGEN to initialise the IMOGEN prognostics.

\sphinxlineitem{FALSE}
\sphinxAtStartPar
IMOGEN will handle the initialisation of its prognostics internally.

\end{description}

\end{fulllineitems}

\index{dump\_file (in namelist IMOGEN\_RUN\_LIST)@\spxentry{dump\_file}\spxextra{in namelist IMOGEN\_RUN\_LIST}|spxpagem}

\begin{fulllineitems}
\phantomsection\label{\detokenize{namelists/imogen.nml:IMOGEN_RUN_LIST::dump_file}}
\pysigstartsignatures
\pysigline{\sphinxcode{\sphinxupquote{IMOGEN\_RUN\_LIST::}}\sphinxbfcode{\sphinxupquote{dump\_file}}}
\pysigstopsignatures\begin{quote}\begin{description}
\sphinxlineitem{Type}
\sphinxAtStartPar
character

\sphinxlineitem{Default}
\sphinxAtStartPar
None

\end{description}\end{quote}

\sphinxAtStartPar
The name of the dump file to initialise from.

\sphinxAtStartPar
Only used if {\hyperref[\detokenize{namelists/imogen.nml:IMOGEN_RUN_LIST::initialise_from_dump}]{\sphinxcrossref{\sphinxcode{\sphinxupquote{initialise\_from\_dump}}}}} = TRUE.

\end{fulllineitems}



\subsection{\sphinxstyleliteralintitle{\sphinxupquote{IMOGEN\_ANLG\_VALS\_LIST}} namelist members}
\label{\detokenize{namelists/imogen.nml:namelist-IMOGEN_ANLG_VALS_LIST}}\label{\detokenize{namelists/imogen.nml:imogen-anlg-vals-list-namelist-members}}\index{IMOGEN\_ANLG\_VALS\_LIST (namelist)@\spxentry{IMOGEN\_ANLG\_VALS\_LIST}\spxextra{namelist}|spxpagem}\index{diff\_frac\_const\_imogen (in namelist IMOGEN\_ANLG\_VALS\_LIST)@\spxentry{diff\_frac\_const\_imogen}\spxextra{in namelist IMOGEN\_ANLG\_VALS\_LIST}|spxpagem}

\begin{fulllineitems}
\phantomsection\label{\detokenize{namelists/imogen.nml:IMOGEN_ANLG_VALS_LIST::diff_frac_const_imogen}}
\pysigstartsignatures
\pysigline{\sphinxcode{\sphinxupquote{IMOGEN\_ANLG\_VALS\_LIST::}}\sphinxbfcode{\sphinxupquote{diff\_frac\_const\_imogen}}}
\pysigstopsignatures\begin{quote}\begin{description}
\sphinxlineitem{Type}
\sphinxAtStartPar
real

\sphinxlineitem{Default}
\sphinxAtStartPar
0.4

\end{description}\end{quote}

\sphinxAtStartPar
IMOGEN uses this instead of {\hyperref[\detokenize{namelists/drive.nml:JULES_DRIVE::diff_frac_const}]{\sphinxcrossref{\sphinxcode{\sphinxupquote{diff\_frac\_const}}}}}

\end{fulllineitems}

\index{q2co2 (in namelist IMOGEN\_ANLG\_VALS\_LIST)@\spxentry{q2co2}\spxextra{in namelist IMOGEN\_ANLG\_VALS\_LIST}|spxpagem}

\begin{fulllineitems}
\phantomsection\label{\detokenize{namelists/imogen.nml:IMOGEN_ANLG_VALS_LIST::q2co2}}
\pysigstartsignatures
\pysigline{\sphinxcode{\sphinxupquote{IMOGEN\_ANLG\_VALS\_LIST::}}\sphinxbfcode{\sphinxupquote{q2co2}}}
\pysigstopsignatures\begin{quote}\begin{description}
\sphinxlineitem{Type}
\sphinxAtStartPar
real

\sphinxlineitem{Default}
\sphinxAtStartPar
3.74

\end{description}\end{quote}

\sphinxAtStartPar
Radiative forcing due to doubling CO2 (W m$^{\text{\sphinxhyphen{}2}}$).

\end{fulllineitems}

\index{f\_ocean (in namelist IMOGEN\_ANLG\_VALS\_LIST)@\spxentry{f\_ocean}\spxextra{in namelist IMOGEN\_ANLG\_VALS\_LIST}|spxpagem}

\begin{fulllineitems}
\phantomsection\label{\detokenize{namelists/imogen.nml:IMOGEN_ANLG_VALS_LIST::f_ocean}}
\pysigstartsignatures
\pysigline{\sphinxcode{\sphinxupquote{IMOGEN\_ANLG\_VALS\_LIST::}}\sphinxbfcode{\sphinxupquote{f\_ocean}}}
\pysigstopsignatures\begin{quote}\begin{description}
\sphinxlineitem{Type}
\sphinxAtStartPar
real

\sphinxlineitem{Default}
\sphinxAtStartPar
0.711

\end{description}\end{quote}

\sphinxAtStartPar
Fractional coverage of the ocean.

\end{fulllineitems}

\index{kappa\_o (in namelist IMOGEN\_ANLG\_VALS\_LIST)@\spxentry{kappa\_o}\spxextra{in namelist IMOGEN\_ANLG\_VALS\_LIST}|spxpagem}

\begin{fulllineitems}
\phantomsection\label{\detokenize{namelists/imogen.nml:IMOGEN_ANLG_VALS_LIST::kappa_o}}
\pysigstartsignatures
\pysigline{\sphinxcode{\sphinxupquote{IMOGEN\_ANLG\_VALS\_LIST::}}\sphinxbfcode{\sphinxupquote{kappa\_o}}}
\pysigstopsignatures\begin{quote}\begin{description}
\sphinxlineitem{Type}
\sphinxAtStartPar
real

\sphinxlineitem{Default}
\sphinxAtStartPar
383.8

\end{description}\end{quote}

\sphinxAtStartPar
Ocean eddy diffusivity (W m$^{\text{\sphinxhyphen{}1}}$ K$^{\text{\sphinxhyphen{}1}}$).

\end{fulllineitems}

\index{lambda\_l (in namelist IMOGEN\_ANLG\_VALS\_LIST)@\spxentry{lambda\_l}\spxextra{in namelist IMOGEN\_ANLG\_VALS\_LIST}|spxpagem}

\begin{fulllineitems}
\phantomsection\label{\detokenize{namelists/imogen.nml:IMOGEN_ANLG_VALS_LIST::lambda_l}}
\pysigstartsignatures
\pysigline{\sphinxcode{\sphinxupquote{IMOGEN\_ANLG\_VALS\_LIST::}}\sphinxbfcode{\sphinxupquote{lambda\_l}}}
\pysigstopsignatures\begin{quote}\begin{description}
\sphinxlineitem{Type}
\sphinxAtStartPar
real

\sphinxlineitem{Default}
\sphinxAtStartPar
0.52

\end{description}\end{quote}

\sphinxAtStartPar
Inverse of climate sensitivity over land (W m$^{\text{\sphinxhyphen{}2}}$ K$^{\text{\sphinxhyphen{}1}}$).

\end{fulllineitems}

\index{lambda\_o (in namelist IMOGEN\_ANLG\_VALS\_LIST)@\spxentry{lambda\_o}\spxextra{in namelist IMOGEN\_ANLG\_VALS\_LIST}|spxpagem}

\begin{fulllineitems}
\phantomsection\label{\detokenize{namelists/imogen.nml:IMOGEN_ANLG_VALS_LIST::lambda_o}}
\pysigstartsignatures
\pysigline{\sphinxcode{\sphinxupquote{IMOGEN\_ANLG\_VALS\_LIST::}}\sphinxbfcode{\sphinxupquote{lambda\_o}}}
\pysigstopsignatures\begin{quote}\begin{description}
\sphinxlineitem{Type}
\sphinxAtStartPar
real

\sphinxlineitem{Default}
\sphinxAtStartPar
1.75

\end{description}\end{quote}

\sphinxAtStartPar
Inverse of climate sensitivity over ocean (W m$^{\text{\sphinxhyphen{}2}}$ K$^{\text{\sphinxhyphen{}1}}$).

\end{fulllineitems}

\index{mu (in namelist IMOGEN\_ANLG\_VALS\_LIST)@\spxentry{mu}\spxextra{in namelist IMOGEN\_ANLG\_VALS\_LIST}|spxpagem}

\begin{fulllineitems}
\phantomsection\label{\detokenize{namelists/imogen.nml:IMOGEN_ANLG_VALS_LIST::mu}}
\pysigstartsignatures
\pysigline{\sphinxcode{\sphinxupquote{IMOGEN\_ANLG\_VALS\_LIST::}}\sphinxbfcode{\sphinxupquote{mu}}}
\pysigstopsignatures\begin{quote}\begin{description}
\sphinxlineitem{Type}
\sphinxAtStartPar
real

\sphinxlineitem{Default}
\sphinxAtStartPar
1.87

\end{description}\end{quote}

\sphinxAtStartPar
Ratio of land to ocean temperature anomalies.

\end{fulllineitems}

\index{t\_ocean\_init (in namelist IMOGEN\_ANLG\_VALS\_LIST)@\spxentry{t\_ocean\_init}\spxextra{in namelist IMOGEN\_ANLG\_VALS\_LIST}|spxpagem}

\begin{fulllineitems}
\phantomsection\label{\detokenize{namelists/imogen.nml:IMOGEN_ANLG_VALS_LIST::t_ocean_init}}
\pysigstartsignatures
\pysigline{\sphinxcode{\sphinxupquote{IMOGEN\_ANLG\_VALS\_LIST::}}\sphinxbfcode{\sphinxupquote{t\_ocean\_init}}}
\pysigstopsignatures\begin{quote}\begin{description}
\sphinxlineitem{Type}
\sphinxAtStartPar
real

\sphinxlineitem{Default}
\sphinxAtStartPar
289.28

\end{description}\end{quote}

\sphinxAtStartPar
Initial ocean temperature (K).

\end{fulllineitems}

\index{dir\_patt (in namelist IMOGEN\_ANLG\_VALS\_LIST)@\spxentry{dir\_patt}\spxextra{in namelist IMOGEN\_ANLG\_VALS\_LIST}|spxpagem}

\begin{fulllineitems}
\phantomsection\label{\detokenize{namelists/imogen.nml:IMOGEN_ANLG_VALS_LIST::dir_patt}}
\pysigstartsignatures
\pysigline{\sphinxcode{\sphinxupquote{IMOGEN\_ANLG\_VALS\_LIST::}}\sphinxbfcode{\sphinxupquote{dir\_patt}}}
\pysigstopsignatures\begin{quote}\begin{description}
\sphinxlineitem{Type}
\sphinxAtStartPar
character

\sphinxlineitem{Default}
\sphinxAtStartPar
None

\end{description}\end{quote}

\sphinxAtStartPar
Directory containing the patterns.

\sphinxAtStartPar
Files in this directory are expected to be in a specific format \sphinxhyphen{} see the IMOGEN example.

\end{fulllineitems}

\index{dir\_clim (in namelist IMOGEN\_ANLG\_VALS\_LIST)@\spxentry{dir\_clim}\spxextra{in namelist IMOGEN\_ANLG\_VALS\_LIST}|spxpagem}

\begin{fulllineitems}
\phantomsection\label{\detokenize{namelists/imogen.nml:IMOGEN_ANLG_VALS_LIST::dir_clim}}
\pysigstartsignatures
\pysigline{\sphinxcode{\sphinxupquote{IMOGEN\_ANLG\_VALS\_LIST::}}\sphinxbfcode{\sphinxupquote{dir\_clim}}}
\pysigstopsignatures\begin{quote}\begin{description}
\sphinxlineitem{Type}
\sphinxAtStartPar
character

\sphinxlineitem{Default}
\sphinxAtStartPar
None

\end{description}\end{quote}

\sphinxAtStartPar
Directory containing initialising climatology.

\sphinxAtStartPar
Files in this directory are expected to be in a specific format \sphinxhyphen{} see the IMOGEN example.

\end{fulllineitems}

\index{dir\_anom (in namelist IMOGEN\_ANLG\_VALS\_LIST)@\spxentry{dir\_anom}\spxextra{in namelist IMOGEN\_ANLG\_VALS\_LIST}|spxpagem}

\begin{fulllineitems}
\phantomsection\label{\detokenize{namelists/imogen.nml:IMOGEN_ANLG_VALS_LIST::dir_anom}}
\pysigstartsignatures
\pysigline{\sphinxcode{\sphinxupquote{IMOGEN\_ANLG\_VALS\_LIST::}}\sphinxbfcode{\sphinxupquote{dir\_anom}}}
\pysigstopsignatures\begin{quote}\begin{description}
\sphinxlineitem{Type}
\sphinxAtStartPar
character

\sphinxlineitem{Default}
\sphinxAtStartPar
None

\end{description}\end{quote}

\sphinxAtStartPar
Directory containing prescribed anomalies.

\sphinxAtStartPar
Files in this directory are expected to be in a specific format \sphinxhyphen{} see the IMOGEN example.

\end{fulllineitems}


\sphinxstepscope


\section{\sphinxstyleliteralintitle{\sphinxupquote{prescribed\_data.nml}}}
\label{\detokenize{namelists/prescribed_data.nml:prescribed-data-nml}}\label{\detokenize{namelists/prescribed_data.nml::doc}}
\sphinxAtStartPar
This file contains a variable number of namelists that are used to prescribe time\sphinxhyphen{}varying input data that is not meteorological forcing. The namelist {\hyperref[\detokenize{namelists/prescribed_data.nml:namelist-JULES_PRESCRIBED}]{\sphinxcrossref{\sphinxcode{\sphinxupquote{JULES\_PRESCRIBED}}}}} should occur only once at the top of the file. The value of {\hyperref[\detokenize{namelists/prescribed_data.nml:JULES_PRESCRIBED::n_datasets}]{\sphinxcrossref{\sphinxcode{\sphinxupquote{n\_datasets}}}}} in {\hyperref[\detokenize{namelists/prescribed_data.nml:namelist-JULES_PRESCRIBED}]{\sphinxcrossref{\sphinxcode{\sphinxupquote{JULES\_PRESCRIBED}}}}} then determines how many times the namelist {\hyperref[\detokenize{namelists/prescribed_data.nml:namelist-JULES_PRESCRIBED_DATASET}]{\sphinxcrossref{\sphinxcode{\sphinxupquote{JULES\_PRESCRIBED\_DATASET}}}}} should occur.


\subsection{\sphinxstyleliteralintitle{\sphinxupquote{JULES\_PRESCRIBED}} namelist members}
\label{\detokenize{namelists/prescribed_data.nml:namelist-JULES_PRESCRIBED}}\label{\detokenize{namelists/prescribed_data.nml:jules-prescribed-namelist-members}}\index{JULES\_PRESCRIBED (namelist)@\spxentry{JULES\_PRESCRIBED}\spxextra{namelist}|spxpagem}\index{n\_datasets (in namelist JULES\_PRESCRIBED)@\spxentry{n\_datasets}\spxextra{in namelist JULES\_PRESCRIBED}|spxpagem}

\begin{fulllineitems}
\phantomsection\label{\detokenize{namelists/prescribed_data.nml:JULES_PRESCRIBED::n_datasets}}
\pysigstartsignatures
\pysigline{\sphinxcode{\sphinxupquote{JULES\_PRESCRIBED::}}\sphinxbfcode{\sphinxupquote{n\_datasets}}}
\pysigstopsignatures\begin{quote}\begin{description}
\sphinxlineitem{Type}
\sphinxAtStartPar
integer

\sphinxlineitem{Permitted}
\sphinxAtStartPar
\textgreater{}= 0

\sphinxlineitem{Default}
\sphinxAtStartPar
0

\end{description}\end{quote}

\sphinxAtStartPar
The number of datasets that will be specified using instances of the {\hyperref[\detokenize{namelists/prescribed_data.nml:namelist-JULES_PRESCRIBED_DATASET}]{\sphinxcrossref{\sphinxcode{\sphinxupquote{JULES\_PRESCRIBED\_DATASET}}}}} namelist.

\end{fulllineitems}



\subsection{\sphinxstyleliteralintitle{\sphinxupquote{JULES\_PRESCRIBED\_DATASET}} namelist members}
\label{\detokenize{namelists/prescribed_data.nml:namelist-JULES_PRESCRIBED_DATASET}}\label{\detokenize{namelists/prescribed_data.nml:jules-prescribed-dataset-namelist-members}}\index{JULES\_PRESCRIBED\_DATASET (namelist)@\spxentry{JULES\_PRESCRIBED\_DATASET}\spxextra{namelist}|spxpagem}
\sphinxAtStartPar
This namelist should occur {\hyperref[\detokenize{namelists/prescribed_data.nml:JULES_PRESCRIBED::n_datasets}]{\sphinxcrossref{\sphinxcode{\sphinxupquote{n\_datasets}}}}} times. Each occurrence of this namelist contains information about a single dataset (i.e. set of related files).

\begin{sphinxadmonition}{note}{Members used to specify the start, end and period of the data}
\index{data\_start (in namelist JULES\_PRESCRIBED\_DATASET)@\spxentry{data\_start}\spxextra{in namelist JULES\_PRESCRIBED\_DATASET}|spxpagem}

\begin{fulllineitems}
\phantomsection\label{\detokenize{namelists/prescribed_data.nml:JULES_PRESCRIBED_DATASET::data_start}}
\pysigstartsignatures
\pysigline{\sphinxcode{\sphinxupquote{JULES\_PRESCRIBED\_DATASET::}}\sphinxbfcode{\sphinxupquote{data\_start}}}
\pysigstopsignatures
\end{fulllineitems}

\index{data\_end (in namelist JULES\_PRESCRIBED\_DATASET)@\spxentry{data\_end}\spxextra{in namelist JULES\_PRESCRIBED\_DATASET}|spxpagem}

\begin{fulllineitems}
\phantomsection\label{\detokenize{namelists/prescribed_data.nml:JULES_PRESCRIBED_DATASET::data_end}}
\pysigstartsignatures
\pysigline{\sphinxcode{\sphinxupquote{JULES\_PRESCRIBED\_DATASET::}}\sphinxbfcode{\sphinxupquote{data\_end}}}
\pysigstopsignatures\begin{quote}\begin{description}
\sphinxlineitem{Type}
\sphinxAtStartPar
character

\sphinxlineitem{Default}
\sphinxAtStartPar
None

\end{description}\end{quote}

\sphinxAtStartPar
The times of the start of the first timestep of data and the end of the last timestep of data.

\sphinxAtStartPar
Each run of JULES (configured in {\hyperref[\detokenize{namelists/timesteps.nml::doc}]{\sphinxcrossref{\DUrole{doc}{timesteps.nml}}}}) can use part or all of the specified data. However, there must be data for all times between run start and run end (determined by {\hyperref[\detokenize{namelists/timesteps.nml:JULES_TIME::main_run_start}]{\sphinxcrossref{\sphinxcode{\sphinxupquote{main\_run\_start}}}}}, {\hyperref[\detokenize{namelists/timesteps.nml:JULES_TIME::main_run_end}]{\sphinxcrossref{\sphinxcode{\sphinxupquote{main\_run\_end}}}}}, {\hyperref[\detokenize{namelists/timesteps.nml:JULES_SPINUP::spinup_start}]{\sphinxcrossref{\sphinxcode{\sphinxupquote{spinup\_start}}}}} and {\hyperref[\detokenize{namelists/timesteps.nml:JULES_SPINUP::spinup_end}]{\sphinxcrossref{\sphinxcode{\sphinxupquote{spinup\_end}}}}}).

\sphinxAtStartPar
The times must be given in the format:

\begin{sphinxVerbatim}[commandchars=\\\{\}]
\PYG{l+s+s2}{\PYGZdq{}yyyy\PYGZhy{}mm\PYGZhy{}dd hh:mm:ss\PYGZdq{}}
\end{sphinxVerbatim}

\end{fulllineitems}

\index{data\_period (in namelist JULES\_PRESCRIBED\_DATASET)@\spxentry{data\_period}\spxextra{in namelist JULES\_PRESCRIBED\_DATASET}|spxpagem}

\begin{fulllineitems}
\phantomsection\label{\detokenize{namelists/prescribed_data.nml:JULES_PRESCRIBED_DATASET::data_period}}
\pysigstartsignatures
\pysigline{\sphinxcode{\sphinxupquote{JULES\_PRESCRIBED\_DATASET::}}\sphinxbfcode{\sphinxupquote{data\_period}}}
\pysigstopsignatures\begin{quote}\begin{description}
\sphinxlineitem{Type}
\sphinxAtStartPar
integer

\sphinxlineitem{Permitted}
\sphinxAtStartPar
\sphinxhyphen{}2, \sphinxhyphen{}1 or \textgreater{} 0

\sphinxlineitem{Default}
\sphinxAtStartPar
None

\end{description}\end{quote}

\sphinxAtStartPar
The period, in seconds, of the data.

\sphinxAtStartPar
Special cases:

\begin{DUlineblock}{0em}
\item[] \sphinxstylestrong{\sphinxhyphen{}1:} Monthly data
\item[] \sphinxstylestrong{\sphinxhyphen{}2:} Yearly data
\end{DUlineblock}

\end{fulllineitems}

\index{is\_climatology (in namelist JULES\_PRESCRIBED\_DATASET)@\spxentry{is\_climatology}\spxextra{in namelist JULES\_PRESCRIBED\_DATASET}|spxpagem}

\begin{fulllineitems}
\phantomsection\label{\detokenize{namelists/prescribed_data.nml:JULES_PRESCRIBED_DATASET::is_climatology}}
\pysigstartsignatures
\pysigline{\sphinxcode{\sphinxupquote{JULES\_PRESCRIBED\_DATASET::}}\sphinxbfcode{\sphinxupquote{is\_climatology}}}
\pysigstopsignatures\begin{quote}\begin{description}
\sphinxlineitem{Type}
\sphinxAtStartPar
logical

\sphinxlineitem{Default}
\sphinxAtStartPar
F

\end{description}\end{quote}

\sphinxAtStartPar
Indicates whether the data is to be used as a climatology (use the same data for every year).
\begin{description}
\sphinxlineitem{TRUE}
\sphinxAtStartPar
Interpret the data as a climatology. {\hyperref[\detokenize{namelists/prescribed_data.nml:JULES_PRESCRIBED_DATASET::data_start}]{\sphinxcrossref{\sphinxcode{\sphinxupquote{data\_start}}}}} and {\hyperref[\detokenize{namelists/prescribed_data.nml:JULES_PRESCRIBED_DATASET::data_end}]{\sphinxcrossref{\sphinxcode{\sphinxupquote{data\_end}}}}} must be such that exactly one year of data is specified.

\sphinxlineitem{FALSE}
\sphinxAtStartPar
Do not interpret the data as a climatology.

\end{description}

\end{fulllineitems}

\end{sphinxadmonition}

\begin{sphinxadmonition}{note}{Members used to specify the files containing the data}
\index{read\_list (in namelist JULES\_PRESCRIBED\_DATASET)@\spxentry{read\_list}\spxextra{in namelist JULES\_PRESCRIBED\_DATASET}|spxpagem}

\begin{fulllineitems}
\phantomsection\label{\detokenize{namelists/prescribed_data.nml:JULES_PRESCRIBED_DATASET::read_list}}
\pysigstartsignatures
\pysigline{\sphinxcode{\sphinxupquote{JULES\_PRESCRIBED\_DATASET::}}\sphinxbfcode{\sphinxupquote{read\_list}}}
\pysigstopsignatures\begin{quote}\begin{description}
\sphinxlineitem{Type}
\sphinxAtStartPar
logical

\sphinxlineitem{Default}
\sphinxAtStartPar
F

\end{description}\end{quote}

\sphinxAtStartPar
Switch controlling how data file names are determined for a given time.
\begin{description}
\sphinxlineitem{TRUE}
\sphinxAtStartPar
Use a list of data file names with times of first data.

\sphinxlineitem{FALSE}
\sphinxAtStartPar
Use a single data file for all times or a template describing the names of the data files.

\end{description}

\end{fulllineitems}

\index{nfiles (in namelist JULES\_PRESCRIBED\_DATASET)@\spxentry{nfiles}\spxextra{in namelist JULES\_PRESCRIBED\_DATASET}|spxpagem}

\begin{fulllineitems}
\phantomsection\label{\detokenize{namelists/prescribed_data.nml:JULES_PRESCRIBED_DATASET::nfiles}}
\pysigstartsignatures
\pysigline{\sphinxcode{\sphinxupquote{JULES\_PRESCRIBED\_DATASET::}}\sphinxbfcode{\sphinxupquote{nfiles}}}
\pysigstopsignatures\begin{quote}\begin{description}
\sphinxlineitem{Type}
\sphinxAtStartPar
integer

\sphinxlineitem{Permitted}
\sphinxAtStartPar
\textgreater{}= 0

\sphinxlineitem{Default}
\sphinxAtStartPar
0

\end{description}\end{quote}

\sphinxAtStartPar
Only used if {\hyperref[\detokenize{namelists/prescribed_data.nml:JULES_PRESCRIBED_DATASET::read_list}]{\sphinxcrossref{\sphinxcode{\sphinxupquote{read\_list}}}}} = TRUE.

\sphinxAtStartPar
The number of data files to read name and time of first data for.

\end{fulllineitems}

\index{file (in namelist JULES\_PRESCRIBED\_DATASET)@\spxentry{file}\spxextra{in namelist JULES\_PRESCRIBED\_DATASET}|spxpagem}

\begin{fulllineitems}
\phantomsection\label{\detokenize{namelists/prescribed_data.nml:JULES_PRESCRIBED_DATASET::file}}
\pysigstartsignatures
\pysigline{\sphinxcode{\sphinxupquote{JULES\_PRESCRIBED\_DATASET::}}\sphinxbfcode{\sphinxupquote{file}}}
\pysigstopsignatures\begin{quote}\begin{description}
\sphinxlineitem{Type}
\sphinxAtStartPar
character

\sphinxlineitem{Default}
\sphinxAtStartPar
None

\end{description}\end{quote}

\sphinxAtStartPar
If {\hyperref[\detokenize{namelists/prescribed_data.nml:JULES_PRESCRIBED_DATASET::read_list}]{\sphinxcrossref{\sphinxcode{\sphinxupquote{read\_list}}}}} = TRUE, this is the file to read the list of data file names and times from. Each line should be of the form:

\begin{sphinxVerbatim}[commandchars=\\\{\}]
\PYG{l+s+s1}{\PYGZsq{}/data/file\PYGZsq{}}\PYG{p}{,} \PYG{l+s+s1}{\PYGZsq{}yyyy\PYGZhy{}mm\PYGZhy{}dd hh:mm:ss\PYGZsq{}}
\end{sphinxVerbatim}

\sphinxAtStartPar
In this case data file names may contain variable name templating only, with the proviso that either no file names use variable name templating or all file names do. The files must appear in chronological order.

\sphinxAtStartPar
If {\hyperref[\detokenize{namelists/prescribed_data.nml:JULES_PRESCRIBED_DATASET::read_list}]{\sphinxcrossref{\sphinxcode{\sphinxupquote{read\_list}}}}} = FALSE, this is either the single data file (if no templating is used) or a template for data file names. Both {\hyperref[\detokenize{input/file-name-templating::doc}]{\sphinxcrossref{\DUrole{doc}{time and variable name templating}}}} may be used.

\end{fulllineitems}

\end{sphinxadmonition}

\begin{sphinxadmonition}{note}{Members used to specify the provided variables}
\index{nvars (in namelist JULES\_PRESCRIBED\_DATASET)@\spxentry{nvars}\spxextra{in namelist JULES\_PRESCRIBED\_DATASET}|spxpagem}

\begin{fulllineitems}
\phantomsection\label{\detokenize{namelists/prescribed_data.nml:JULES_PRESCRIBED_DATASET::nvars}}
\pysigstartsignatures
\pysigline{\sphinxcode{\sphinxupquote{JULES\_PRESCRIBED\_DATASET::}}\sphinxbfcode{\sphinxupquote{nvars}}}
\pysigstopsignatures\begin{quote}\begin{description}
\sphinxlineitem{Type}
\sphinxAtStartPar
integer

\sphinxlineitem{Permitted}
\sphinxAtStartPar
\textgreater{}= 0

\sphinxlineitem{Default}
\sphinxAtStartPar
0

\end{description}\end{quote}

\sphinxAtStartPar
The number of variables that the dataset will provide.

\sphinxAtStartPar
See {\hyperref[\detokenize{namelists/prescribed_data.nml:supported-prescribed-variables}]{\sphinxcrossref{\DUrole{std,std-ref}{List of supported variables}}}} for the supported variables.

\end{fulllineitems}

\index{var (in namelist JULES\_PRESCRIBED\_DATASET)@\spxentry{var}\spxextra{in namelist JULES\_PRESCRIBED\_DATASET}|spxpagem}

\begin{fulllineitems}
\phantomsection\label{\detokenize{namelists/prescribed_data.nml:JULES_PRESCRIBED_DATASET::var}}
\pysigstartsignatures
\pysigline{\sphinxcode{\sphinxupquote{JULES\_PRESCRIBED\_DATASET::}}\sphinxbfcode{\sphinxupquote{var}}}
\pysigstopsignatures\begin{quote}\begin{description}
\sphinxlineitem{Type}
\sphinxAtStartPar
character(nvars)

\sphinxlineitem{Default}
\sphinxAtStartPar
None

\end{description}\end{quote}

\sphinxAtStartPar
List of variable names as recognised by JULES (see {\hyperref[\detokenize{namelists/prescribed_data.nml:supported-prescribed-variables}]{\sphinxcrossref{\DUrole{std,std-ref}{List of supported variables}}}}). Names are case sensitive.

\begin{sphinxadmonition}{note}{Note:}
\sphinxAtStartPar
For ASCII files, variable names must be in the order they appear in the file.
\end{sphinxadmonition}

\end{fulllineitems}

\index{var\_name (in namelist JULES\_PRESCRIBED\_DATASET)@\spxentry{var\_name}\spxextra{in namelist JULES\_PRESCRIBED\_DATASET}|spxpagem}

\begin{fulllineitems}
\phantomsection\label{\detokenize{namelists/prescribed_data.nml:JULES_PRESCRIBED_DATASET::var_name}}
\pysigstartsignatures
\pysigline{\sphinxcode{\sphinxupquote{JULES\_PRESCRIBED\_DATASET::}}\sphinxbfcode{\sphinxupquote{var\_name}}}
\pysigstopsignatures\begin{quote}\begin{description}
\sphinxlineitem{Type}
\sphinxAtStartPar
character(nvars)

\sphinxlineitem{Default}
\sphinxAtStartPar
‘’ (empty string)

\end{description}\end{quote}

\sphinxAtStartPar
For each JULES variable specified in {\hyperref[\detokenize{namelists/prescribed_data.nml:JULES_PRESCRIBED_DATASET::var}]{\sphinxcrossref{\sphinxcode{\sphinxupquote{var}}}}}, this is the name of the variable in the file(s) containing the data.

\sphinxAtStartPar
If the empty string (the default) is given for any variable, then the corresponding value from {\hyperref[\detokenize{namelists/prescribed_data.nml:JULES_PRESCRIBED_DATASET::var}]{\sphinxcrossref{\sphinxcode{\sphinxupquote{var}}}}} is used instead.

\begin{sphinxadmonition}{note}{Note:}
\sphinxAtStartPar
For ASCII files, this is not used \sphinxhyphen{} only the order in the file matters, as described above.
\end{sphinxadmonition}

\end{fulllineitems}

\index{tpl\_name (in namelist JULES\_PRESCRIBED\_DATASET)@\spxentry{tpl\_name}\spxextra{in namelist JULES\_PRESCRIBED\_DATASET}|spxpagem}

\begin{fulllineitems}
\phantomsection\label{\detokenize{namelists/prescribed_data.nml:JULES_PRESCRIBED_DATASET::tpl_name}}
\pysigstartsignatures
\pysigline{\sphinxcode{\sphinxupquote{JULES\_PRESCRIBED\_DATASET::}}\sphinxbfcode{\sphinxupquote{tpl\_name}}}
\pysigstopsignatures\begin{quote}\begin{description}
\sphinxlineitem{Type}
\sphinxAtStartPar
character(nvars)

\sphinxlineitem{Default}
\sphinxAtStartPar
None

\end{description}\end{quote}

\sphinxAtStartPar
For each JULES variable specified in {\hyperref[\detokenize{namelists/prescribed_data.nml:JULES_PRESCRIBED_DATASET::var}]{\sphinxcrossref{\sphinxcode{\sphinxupquote{var}}}}}, this is the string to substitute into the file name(s) in place of the variable name substitution string.

\sphinxAtStartPar
If the file name(s) do not use variable name templating, this is not used.

\end{fulllineitems}

\index{interp (in namelist JULES\_PRESCRIBED\_DATASET)@\spxentry{interp}\spxextra{in namelist JULES\_PRESCRIBED\_DATASET}|spxpagem}

\begin{fulllineitems}
\phantomsection\label{\detokenize{namelists/prescribed_data.nml:JULES_PRESCRIBED_DATASET::interp}}
\pysigstartsignatures
\pysigline{\sphinxcode{\sphinxupquote{JULES\_PRESCRIBED\_DATASET::}}\sphinxbfcode{\sphinxupquote{interp}}}
\pysigstopsignatures\begin{quote}\begin{description}
\sphinxlineitem{Type}
\sphinxAtStartPar
character(nvars)

\sphinxlineitem{Default}
\sphinxAtStartPar
None

\end{description}\end{quote}

\sphinxAtStartPar
For each JULES variable specified in {\hyperref[\detokenize{namelists/prescribed_data.nml:JULES_PRESCRIBED_DATASET::var}]{\sphinxcrossref{\sphinxcode{\sphinxupquote{var}}}}}, this indicates how the variable is to be interpolated in time (see {\hyperref[\detokenize{input/temporal-interpolation::doc}]{\sphinxcrossref{\DUrole{doc}{Temporal interpolation}}}}).

\end{fulllineitems}

\index{prescribed\_levels (in namelist JULES\_PRESCRIBED\_DATASET)@\spxentry{prescribed\_levels}\spxextra{in namelist JULES\_PRESCRIBED\_DATASET}|spxpagem}

\begin{fulllineitems}
\phantomsection\label{\detokenize{namelists/prescribed_data.nml:JULES_PRESCRIBED_DATASET::prescribed_levels}}
\pysigstartsignatures
\pysigline{\sphinxcode{\sphinxupquote{JULES\_PRESCRIBED\_DATASET::}}\sphinxbfcode{\sphinxupquote{prescribed\_levels}}}
\pysigstopsignatures\begin{quote}\begin{description}
\sphinxlineitem{Type}
\sphinxAtStartPar
integer(n) where n ranges from 1 (one level prescribed) to {\hyperref[\detokenize{namelists/jules_soil.nml:JULES_SOIL::sm_levels}]{\sphinxcrossref{\sphinxcode{\sphinxupquote{sm\_levels}}}}} (all levels prescribed)

\sphinxlineitem{Default}
\sphinxAtStartPar
1, …, {\hyperref[\detokenize{namelists/jules_soil.nml:JULES_SOIL::sm_levels}]{\sphinxcrossref{\sphinxcode{\sphinxupquote{sm\_levels}}}}} i.e. all levels prescribed

\end{description}\end{quote}

\sphinxAtStartPar
Indices of the subset of levels to be prescribed. Currently only implemented for {\hyperref[\detokenize{namelists/prescribed_data.nml:JULES_PRESCRIBED_DATASET::var}]{\sphinxcrossref{\sphinxcode{\sphinxupquote{var}}}}} = \sphinxcode{\sphinxupquote{sthuf}} and {\hyperref[\detokenize{namelists/prescribed_data.nml:JULES_PRESCRIBED_DATASET::nvars}]{\sphinxcrossref{\sphinxcode{\sphinxupquote{nvars}}}}} = 1. The numbering of the soil level indices starts at 1 (corresponding to the layer touching the surface). Note that \sphinxcode{\sphinxupquote{sthuf}} data must be provided for all soil levels, but can be set to dummy values for the levels that are not prescribed.

\end{fulllineitems}

\end{sphinxadmonition}


\subsubsection{List of supported variables}
\label{\detokenize{namelists/prescribed_data.nml:list-of-supported-variables}}\label{\detokenize{namelists/prescribed_data.nml:supported-prescribed-variables}}
\sphinxAtStartPar
All variables input using {\hyperref[\detokenize{namelists/prescribed_data.nml::doc}]{\sphinxcrossref{\DUrole{doc}{prescribed\_data.nml}}}} must have a time dimension using {\hyperref[\detokenize{namelists/model_grid.nml:JULES_INPUT_GRID::time_dim_name}]{\sphinxcrossref{\sphinxcode{\sphinxupquote{time\_dim\_name}}}}}.

\sphinxAtStartPar
In theory, any variable with an entry in the subroutine \sphinxcode{\sphinxupquote{populate\_var}} in \sphinxcode{\sphinxupquote{model\_interface\_mod}} (see {\hyperref[\detokenize{code/io::doc}]{\sphinxcrossref{\DUrole{doc}{I/O framework}}}}) can be updated via this mechanism, and the use of any of these variables is not explicitly prevented. However, it is up to the user to assess whether using this mechanism to update any particular variable is appropriate or desirable.

\sphinxAtStartPar
The use of the following variables is explicitly supported:


\begin{savenotes}\sphinxattablestart
\centering
\begin{tabulary}{\linewidth}[t]{|p{2.5cm}|p{8cm}|p{4cm}|}
\hline
\sphinxstyletheadfamily 
\sphinxAtStartPar
Name
&\sphinxstyletheadfamily 
\sphinxAtStartPar
Description
&\sphinxstyletheadfamily 
\sphinxAtStartPar
Levels dimension(s) required in files
\\
\hline
\sphinxAtStartPar
\sphinxcode{\sphinxupquote{ozone}}
&
\sphinxAtStartPar
Surface ozone concentration (ppb).

\begin{sphinxadmonition}{note}{Note:}
\sphinxAtStartPar
Required if {\hyperref[\detokenize{namelists/jules_vegetation.nml:JULES_VEGETATION::l_o3_damage}]{\sphinxcrossref{\sphinxcode{\sphinxupquote{l\_o3\_damage}}}}} = TRUE.
\end{sphinxadmonition}
&
\sphinxAtStartPar
None
\\
\hline
\sphinxAtStartPar
\sphinxcode{\sphinxupquote{canht}}
&
\sphinxAtStartPar
PFT canopy height (m).

\begin{sphinxadmonition}{note}{Note:}
\sphinxAtStartPar
Not possible if {\hyperref[\detokenize{namelists/jules_vegetation.nml:JULES_VEGETATION::l_triffid}]{\sphinxcrossref{\sphinxcode{\sphinxupquote{l\_triffid}}}}} = TRUE
\end{sphinxadmonition}
&
\sphinxAtStartPar
Single levels dimension of size
{\hyperref[\detokenize{namelists/jules_surface_types.nml:JULES_SURFACE_TYPES::npft}]{\sphinxcrossref{\sphinxcode{\sphinxupquote{npft}}}}} using
{\hyperref[\detokenize{namelists/model_grid.nml:JULES_INPUT_GRID::pft_dim_name}]{\sphinxcrossref{\sphinxcode{\sphinxupquote{pft\_dim\_name}}}}}.
\\
\hline
\sphinxAtStartPar
\sphinxcode{\sphinxupquote{lai}}
&
\sphinxAtStartPar
PFT leaf area index.

\begin{sphinxadmonition}{note}{Note:}
\sphinxAtStartPar
Not possible if {\hyperref[\detokenize{namelists/jules_vegetation.nml:JULES_VEGETATION::l_triffid}]{\sphinxcrossref{\sphinxcode{\sphinxupquote{l\_triffid}}}}} = TRUE
or {\hyperref[\detokenize{namelists/jules_vegetation.nml:JULES_VEGETATION::l_phenol}]{\sphinxcrossref{\sphinxcode{\sphinxupquote{l\_phenol}}}}} = TRUE
\end{sphinxadmonition}
&
\sphinxAtStartPar
Single levels dimension of size
{\hyperref[\detokenize{namelists/jules_surface_types.nml:JULES_SURFACE_TYPES::npft}]{\sphinxcrossref{\sphinxcode{\sphinxupquote{npft}}}}} using
{\hyperref[\detokenize{namelists/model_grid.nml:JULES_INPUT_GRID::pft_dim_name}]{\sphinxcrossref{\sphinxcode{\sphinxupquote{pft\_dim\_name}}}}}.
\\
\hline
\sphinxAtStartPar
\sphinxcode{\sphinxupquote{albobs\_sw}}
&
\sphinxAtStartPar
Observed SW diffuse albedo.

\begin{sphinxadmonition}{note}{Note:}
\sphinxAtStartPar
Required if {\hyperref[\detokenize{namelists/jules_radiation.nml:JULES_RADIATION::l_albedo_obs}]{\sphinxcrossref{\sphinxcode{\sphinxupquote{l\_albedo\_obs}}}}} = TRUE and
{\hyperref[\detokenize{namelists/jules_radiation.nml:JULES_RADIATION::l_spec_albedo}]{\sphinxcrossref{\sphinxcode{\sphinxupquote{l\_spec\_albedo}}}}} = FALSE.
\end{sphinxadmonition}
&
\sphinxAtStartPar
None
\\
\hline
\sphinxAtStartPar
\sphinxcode{\sphinxupquote{albobs\_vis}}
&
\sphinxAtStartPar
Observed VIS diffuse albedo.

\begin{sphinxadmonition}{note}{Note:}
\sphinxAtStartPar
Required if {\hyperref[\detokenize{namelists/jules_radiation.nml:JULES_RADIATION::l_albedo_obs}]{\sphinxcrossref{\sphinxcode{\sphinxupquote{l\_albedo\_obs}}}}} = TRUE and
{\hyperref[\detokenize{namelists/jules_radiation.nml:JULES_RADIATION::l_spec_albedo}]{\sphinxcrossref{\sphinxcode{\sphinxupquote{l\_spec\_albedo}}}}} = TRUE.
\end{sphinxadmonition}
&
\sphinxAtStartPar
None
\\
\hline
\sphinxAtStartPar
\sphinxcode{\sphinxupquote{albobs\_nir}}
&
\sphinxAtStartPar
Observed NIR diffuse albedo.

\begin{sphinxadmonition}{note}{Note:}
\sphinxAtStartPar
Required if {\hyperref[\detokenize{namelists/jules_radiation.nml:JULES_RADIATION::l_albedo_obs}]{\sphinxcrossref{\sphinxcode{\sphinxupquote{l\_albedo\_obs}}}}} = TRUE and
{\hyperref[\detokenize{namelists/jules_radiation.nml:JULES_RADIATION::l_spec_albedo}]{\sphinxcrossref{\sphinxcode{\sphinxupquote{l\_spec\_albedo}}}}} = TRUE.
\end{sphinxadmonition}
&
\sphinxAtStartPar
None
\\
\hline
\sphinxAtStartPar
\sphinxcode{\sphinxupquote{co2\_mmr}}
&
\sphinxAtStartPar
Concentration of atmospheric CO2, expressed as a mass mixing
ratio.

\begin{sphinxadmonition}{note}{Note:}
\sphinxAtStartPar
A single value of co2\_mmr is applied globally. Data must be
supplied for each gridpoint, but only the value of the first
grid\sphinxhyphen{}point is used.
\end{sphinxadmonition}
&
\sphinxAtStartPar
None
\\
\hline
\sphinxAtStartPar
\sphinxcode{\sphinxupquote{sthuf}}
&
\sphinxAtStartPar
Soil wetness for each soil layer. This is the mass of soil water
(liquid and frozen), expressed as a fraction of the water
content at saturation.

\begin{sphinxadmonition}{note}{Note:}
\sphinxAtStartPar
Soil wetness will be set to the prescribed value at the
beginning of each timestep but will be incremented during
that timestep. Also, it is recommended that the prescribed
\sphinxcode{\sphinxupquote{sthuf}} does not exceed one.
\end{sphinxadmonition}
&
\sphinxAtStartPar
Single levels dimension of size
{\hyperref[\detokenize{namelists/jules_soil.nml:JULES_SOIL::sm_levels}]{\sphinxcrossref{\sphinxcode{\sphinxupquote{sm\_levels}}}}} using
{\hyperref[\detokenize{namelists/model_grid.nml:JULES_INPUT_GRID::soil_dim_name}]{\sphinxcrossref{\sphinxcode{\sphinxupquote{soil\_dim\_name}}}}}.
\\
\hline
\sphinxAtStartPar
\sphinxcode{\sphinxupquote{frac\_agr}}
&
\sphinxAtStartPar
Fractional area of agricultural land in each gridbox.
&
\sphinxAtStartPar
None
\\
\hline
\sphinxAtStartPar
\sphinxcode{\sphinxupquote{frac\_past}}
&
\sphinxAtStartPar
Fractional area of pasture land in each gridbox.
&
\sphinxAtStartPar
None
\\
\hline
\sphinxAtStartPar
\sphinxcode{\sphinxupquote{frac\_biocrop}}
&
\sphinxAtStartPar
Fractional area of bioenergy cropland in each gridbox.
&
\sphinxAtStartPar
None
\\
\hline
\sphinxAtStartPar
\sphinxcode{\sphinxupquote{tracer\_field}}
&
\sphinxAtStartPar
Surface concentration of atmospheric chemical tracers in the
atmosphere, for calculation of deposition, as mass mixing ratio
(kg kg $^{\text{\sphinxhyphen{}1}}$).
&
\sphinxAtStartPar
Single levels dimension of size
{\hyperref[\detokenize{namelists/jules_deposition.nml:JULES_DEPOSITION::ndry_dep_species}]{\sphinxcrossref{\sphinxcode{\sphinxupquote{ndry\_dep\_species}}}}}
using
{\hyperref[\detokenize{namelists/model_grid.nml:JULES_INPUT_GRID::tracer_dim_name}]{\sphinxcrossref{\sphinxcode{\sphinxupquote{tracer\_dim\_name}}}}}.
\\
\hline
\sphinxAtStartPar
\sphinxcode{\sphinxupquote{bl\_height}}
&
\sphinxAtStartPar
Height above surface of top of atmospheric boundary layer (m).
&
\sphinxAtStartPar
None
\\
\hline
\sphinxAtStartPar
\sphinxcode{\sphinxupquote{level\_separation}}
&
\sphinxAtStartPar
Separation of boundary layer levels (m).
The levels are listed starting at the surface and working up.
&
\sphinxAtStartPar
Single levels dimension of size
{\hyperref[\detokenize{namelists/model_grid.nml:JULES_NLSIZES::bl_levels}]{\sphinxcrossref{\sphinxcode{\sphinxupquote{bl\_levels}}}}} using
{\hyperref[\detokenize{namelists/model_grid.nml:JULES_INPUT_GRID::bl_level_dim_name}]{\sphinxcrossref{\sphinxcode{\sphinxupquote{bl\_level\_dim\_name}}}}}.
\\
\hline
\sphinxAtStartPar
\sphinxcode{\sphinxupquote{demand\_rate\_domestic}}
&
\sphinxAtStartPar
Demand for water for domestic use (kg kg $^{\text{\sphinxhyphen{}1}}$)

\begin{sphinxadmonition}{note}{Note:}
\sphinxAtStartPar
Required if {\hyperref[\detokenize{namelists/jules_water_resources.nml:JULES_WATER_RESOURCES::l_water_domestic}]{\sphinxcrossref{\sphinxcode{\sphinxupquote{l\_water\_domestic}}}}}
= TRUE.
\end{sphinxadmonition}
&
\sphinxAtStartPar
None
\\
\hline
\sphinxAtStartPar
\sphinxcode{\sphinxupquote{demand\_rate\_industry}}
&
\sphinxAtStartPar
Demand for water for industrial use (kg kg $^{\text{\sphinxhyphen{}1}}$)

\begin{sphinxadmonition}{note}{Note:}
\sphinxAtStartPar
Required if {\hyperref[\detokenize{namelists/jules_water_resources.nml:JULES_WATER_RESOURCES::l_water_industry}]{\sphinxcrossref{\sphinxcode{\sphinxupquote{l\_water\_industry}}}}}
= TRUE.
\end{sphinxadmonition}
&
\sphinxAtStartPar
None
\\
\hline
\sphinxAtStartPar
\sphinxcode{\sphinxupquote{demand\_rate\_livestock}}
&
\sphinxAtStartPar
Demand for water for livestock (kg kg $^{\text{\sphinxhyphen{}1}}$)

\begin{sphinxadmonition}{note}{Note:}
\sphinxAtStartPar
Required if {\hyperref[\detokenize{namelists/jules_water_resources.nml:JULES_WATER_RESOURCES::l_water_livestock}]{\sphinxcrossref{\sphinxcode{\sphinxupquote{l\_water\_livestock}}}}}
= TRUE.
\end{sphinxadmonition}
&
\sphinxAtStartPar
None
\\
\hline
\sphinxAtStartPar
\sphinxcode{\sphinxupquote{demand\_rate\_trasnfers}}
&
\sphinxAtStartPar
Demand for water for transfers (kg kg $^{\text{\sphinxhyphen{}1}}$)

\begin{sphinxadmonition}{note}{Note:}
\sphinxAtStartPar
Required if {\hyperref[\detokenize{namelists/jules_water_resources.nml:JULES_WATER_RESOURCES::l_water_transfers}]{\sphinxcrossref{\sphinxcode{\sphinxupquote{l\_water\_transfers}}}}}
= TRUE.
\end{sphinxadmonition}
&
\sphinxAtStartPar
None
\\
\hline
\end{tabulary}
\par
\sphinxattableend\end{savenotes}

\sphinxstepscope


\section{\sphinxstyleliteralintitle{\sphinxupquote{initial\_conditions.nml}}}
\label{\detokenize{namelists/initial_conditions.nml:initial-conditions-nml}}\label{\detokenize{namelists/initial_conditions.nml::doc}}
\sphinxAtStartPar
This file contains a single namelist called {\hyperref[\detokenize{namelists/initial_conditions.nml:namelist-JULES_INITIAL}]{\sphinxcrossref{\sphinxcode{\sphinxupquote{JULES\_INITIAL}}}}} that is used to set up the initial state of prognostic variables.


\subsection{\sphinxstyleliteralintitle{\sphinxupquote{JULES\_INITIAL}} namelist members}
\label{\detokenize{namelists/initial_conditions.nml:namelist-JULES_INITIAL}}\label{\detokenize{namelists/initial_conditions.nml:jules-initial-namelist-members}}\index{JULES\_INITIAL (namelist)@\spxentry{JULES\_INITIAL}\spxextra{namelist}|spxpagem}
\sphinxAtStartPar
The values of all prognostic variables must be set at the start of a run. This initial state, or initial condition, can be read from a “dump” from an earlier run of the model, or may be read from a different file. Another option is to prescribe a simple or idealised initial state by giving constant values for the prognostic variables directly in the namelist. It is also possible to set some fields using values from a file (e.g. a dump) but to set others to constants given in the namelist.
\index{dump\_file (in namelist JULES\_INITIAL)@\spxentry{dump\_file}\spxextra{in namelist JULES\_INITIAL}|spxpagem}

\begin{fulllineitems}
\phantomsection\label{\detokenize{namelists/initial_conditions.nml:JULES_INITIAL::dump_file}}
\pysigstartsignatures
\pysigline{\sphinxcode{\sphinxupquote{JULES\_INITIAL::}}\sphinxbfcode{\sphinxupquote{dump\_file}}}
\pysigstopsignatures\begin{quote}\begin{description}
\sphinxlineitem{Type}
\sphinxAtStartPar
logical

\sphinxlineitem{Default}
\sphinxAtStartPar
F

\end{description}\end{quote}

\sphinxAtStartPar
Indicates whether the given {\hyperref[\detokenize{namelists/initial_conditions.nml:JULES_INITIAL::file}]{\sphinxcrossref{\sphinxcode{\sphinxupquote{file}}}}} is a dump from a previous run of JULES.
\begin{description}
\sphinxlineitem{TRUE}
\sphinxAtStartPar
The file is a JULES dump file.

\sphinxlineitem{FALSE}
\sphinxAtStartPar
The file is not a JULES dump file.

\end{description}

\end{fulllineitems}

\index{total\_snow (in namelist JULES\_INITIAL)@\spxentry{total\_snow}\spxextra{in namelist JULES\_INITIAL}|spxpagem}

\begin{fulllineitems}
\phantomsection\label{\detokenize{namelists/initial_conditions.nml:JULES_INITIAL::total_snow}}
\pysigstartsignatures
\pysigline{\sphinxcode{\sphinxupquote{JULES\_INITIAL::}}\sphinxbfcode{\sphinxupquote{total\_snow}}}
\pysigstopsignatures\begin{quote}\begin{description}
\sphinxlineitem{Type}
\sphinxAtStartPar
logical

\sphinxlineitem{Default}
\sphinxAtStartPar
F

\end{description}\end{quote}

\sphinxAtStartPar
Switch controlling simplified initialisation of snow variables.
\begin{description}
\sphinxlineitem{TRUE}
\sphinxAtStartPar
Only the total mass of snow on each surface tile (see \sphinxcode{\sphinxupquote{snow\_tile}} in {\hyperref[\detokenize{namelists/initial_conditions.nml:list-of-initial-condition-variables}]{\sphinxcrossref{\DUrole{std,std-ref}{List of initial condition variables}}}}) is required to be input, and all related variables will be calculated from this or simple assumptions made. All the snow is assumed to be on the ground (not in the canopy).

\sphinxlineitem{FALSE}
\sphinxAtStartPar
All snow variables required for the current configuration must be input separately (see {\hyperref[\detokenize{namelists/initial_conditions.nml:list-of-initial-condition-variables}]{\sphinxcrossref{\DUrole{std,std-ref}{List of initial condition variables}}}}).

\end{description}

\end{fulllineitems}


\begin{sphinxadmonition}{note}{Members used to set up spatially varying properties}
\index{file (in namelist JULES\_INITIAL)@\spxentry{file}\spxextra{in namelist JULES\_INITIAL}|spxpagem}

\begin{fulllineitems}
\phantomsection\label{\detokenize{namelists/initial_conditions.nml:JULES_INITIAL::file}}
\pysigstartsignatures
\pysigline{\sphinxcode{\sphinxupquote{JULES\_INITIAL::}}\sphinxbfcode{\sphinxupquote{file}}}
\pysigstopsignatures\begin{quote}\begin{description}
\sphinxlineitem{Type}
\sphinxAtStartPar
character

\sphinxlineitem{Default}
\sphinxAtStartPar
None

\end{description}\end{quote}

\sphinxAtStartPar
The file to read initial conditions from.

\sphinxAtStartPar
If {\hyperref[\detokenize{namelists/initial_conditions.nml:JULES_INITIAL::use_file}]{\sphinxcrossref{\sphinxcode{\sphinxupquote{use\_file}}}}} (see below) is FALSE for every variable, this will not be used.

\sphinxAtStartPar
If {\hyperref[\detokenize{namelists/initial_conditions.nml:JULES_INITIAL::dump_file}]{\sphinxcrossref{\sphinxcode{\sphinxupquote{dump\_file}}}}} = TRUE, this should be a JULES dump file.

\sphinxAtStartPar
If {\hyperref[\detokenize{namelists/initial_conditions.nml:JULES_INITIAL::dump_file}]{\sphinxcrossref{\sphinxcode{\sphinxupquote{dump\_file}}}}} = FALSE, this should be a file conforming to the {\hyperref[\detokenize{input/overview::doc}]{\sphinxcrossref{\DUrole{doc}{JULES input requirements}}}}. This file name may use {\hyperref[\detokenize{input/file-name-templating::doc}]{\sphinxcrossref{\DUrole{doc}{variable name templating}}}}.

\end{fulllineitems}

\index{nvars (in namelist JULES\_INITIAL)@\spxentry{nvars}\spxextra{in namelist JULES\_INITIAL}|spxpagem}

\begin{fulllineitems}
\phantomsection\label{\detokenize{namelists/initial_conditions.nml:JULES_INITIAL::nvars}}
\pysigstartsignatures
\pysigline{\sphinxcode{\sphinxupquote{JULES\_INITIAL::}}\sphinxbfcode{\sphinxupquote{nvars}}}
\pysigstopsignatures\begin{quote}\begin{description}
\sphinxlineitem{Type}
\sphinxAtStartPar
integer

\sphinxlineitem{Permitted}
\sphinxAtStartPar
\textgreater{}= 0

\sphinxlineitem{Default}
\sphinxAtStartPar
0

\end{description}\end{quote}

\sphinxAtStartPar
The number of initial condition variables that will be provided.

\sphinxAtStartPar
See {\hyperref[\detokenize{namelists/initial_conditions.nml:list-of-initial-condition-variables}]{\sphinxcrossref{\DUrole{std,std-ref}{List of initial condition variables}}}} for those required for a particular configuration.

\begin{sphinxadmonition}{note}{Note:}
\sphinxAtStartPar
If {\hyperref[\detokenize{namelists/initial_conditions.nml:JULES_INITIAL::dump_file}]{\sphinxcrossref{\sphinxcode{\sphinxupquote{dump\_file}}}}} = TRUE and {\hyperref[\detokenize{namelists/initial_conditions.nml:JULES_INITIAL::nvars}]{\sphinxcrossref{\sphinxcode{\sphinxupquote{nvars}}}}} = 0, then the model will attempt to initialise all required variables from the given dump file.
\end{sphinxadmonition}

\end{fulllineitems}

\index{var (in namelist JULES\_INITIAL)@\spxentry{var}\spxextra{in namelist JULES\_INITIAL}|spxpagem}

\begin{fulllineitems}
\phantomsection\label{\detokenize{namelists/initial_conditions.nml:JULES_INITIAL::var}}
\pysigstartsignatures
\pysigline{\sphinxcode{\sphinxupquote{JULES\_INITIAL::}}\sphinxbfcode{\sphinxupquote{var}}}
\pysigstopsignatures\begin{quote}\begin{description}
\sphinxlineitem{Type}
\sphinxAtStartPar
character(nvars)

\sphinxlineitem{Default}
\sphinxAtStartPar
None

\end{description}\end{quote}

\sphinxAtStartPar
List of initial condition variable names as recognised by JULES (see {\hyperref[\detokenize{namelists/initial_conditions.nml:list-of-initial-condition-variables}]{\sphinxcrossref{\DUrole{std,std-ref}{List of initial condition variables}}}}). Names are case sensitive.

\begin{sphinxadmonition}{note}{Note:}
\sphinxAtStartPar
For ASCII files, variable names must be in the order they appear in the file.
\end{sphinxadmonition}

\end{fulllineitems}

\index{use\_file (in namelist JULES\_INITIAL)@\spxentry{use\_file}\spxextra{in namelist JULES\_INITIAL}|spxpagem}

\begin{fulllineitems}
\phantomsection\label{\detokenize{namelists/initial_conditions.nml:JULES_INITIAL::use_file}}
\pysigstartsignatures
\pysigline{\sphinxcode{\sphinxupquote{JULES\_INITIAL::}}\sphinxbfcode{\sphinxupquote{use\_file}}}
\pysigstopsignatures\begin{quote}\begin{description}
\sphinxlineitem{Type}
\sphinxAtStartPar
logical(nvars)

\sphinxlineitem{Default}
\sphinxAtStartPar
T

\end{description}\end{quote}

\sphinxAtStartPar
For each JULES variable specified in {\hyperref[\detokenize{namelists/initial_conditions.nml:JULES_INITIAL::var}]{\sphinxcrossref{\sphinxcode{\sphinxupquote{var}}}}}, this indicates if it should be read from the specified file or whether a constant value is to be used.
\begin{description}
\sphinxlineitem{TRUE}
\sphinxAtStartPar
The variable will be read from the file.

\sphinxlineitem{FALSE}
\sphinxAtStartPar
The variable will be set to a constant value everywhere using {\hyperref[\detokenize{namelists/initial_conditions.nml:JULES_INITIAL::const_val}]{\sphinxcrossref{\sphinxcode{\sphinxupquote{const\_val}}}}} below.

\end{description}

\end{fulllineitems}

\index{var\_name (in namelist JULES\_INITIAL)@\spxentry{var\_name}\spxextra{in namelist JULES\_INITIAL}|spxpagem}

\begin{fulllineitems}
\phantomsection\label{\detokenize{namelists/initial_conditions.nml:JULES_INITIAL::var_name}}
\pysigstartsignatures
\pysigline{\sphinxcode{\sphinxupquote{JULES\_INITIAL::}}\sphinxbfcode{\sphinxupquote{var\_name}}}
\pysigstopsignatures\begin{quote}\begin{description}
\sphinxlineitem{Type}
\sphinxAtStartPar
character(nvars)

\sphinxlineitem{Default}
\sphinxAtStartPar
‘’ (empty string)

\end{description}\end{quote}

\sphinxAtStartPar
For each JULES variable specified in {\hyperref[\detokenize{namelists/initial_conditions.nml:JULES_INITIAL::var}]{\sphinxcrossref{\sphinxcode{\sphinxupquote{var}}}}} where {\hyperref[\detokenize{namelists/initial_conditions.nml:JULES_INITIAL::use_file}]{\sphinxcrossref{\sphinxcode{\sphinxupquote{use\_file}}}}} = TRUE, this is the name of the variable in the file containing the data.

\sphinxAtStartPar
If the empty string (the default) is given for any variable, then the corresponding value from {\hyperref[\detokenize{namelists/initial_conditions.nml:JULES_INITIAL::var}]{\sphinxcrossref{\sphinxcode{\sphinxupquote{var}}}}} is used instead.

\sphinxAtStartPar
This is not used for variables where {\hyperref[\detokenize{namelists/initial_conditions.nml:JULES_INITIAL::use_file}]{\sphinxcrossref{\sphinxcode{\sphinxupquote{use\_file}}}}} = FALSE, but a placeholder must still be given in that case.

\begin{sphinxadmonition}{note}{Note:}
\sphinxAtStartPar
For ASCII files, this is not used \sphinxhyphen{} only the order in the file matters, as described above.
\end{sphinxadmonition}

\end{fulllineitems}

\index{tpl\_name (in namelist JULES\_INITIAL)@\spxentry{tpl\_name}\spxextra{in namelist JULES\_INITIAL}|spxpagem}

\begin{fulllineitems}
\phantomsection\label{\detokenize{namelists/initial_conditions.nml:JULES_INITIAL::tpl_name}}
\pysigstartsignatures
\pysigline{\sphinxcode{\sphinxupquote{JULES\_INITIAL::}}\sphinxbfcode{\sphinxupquote{tpl\_name}}}
\pysigstopsignatures\begin{quote}\begin{description}
\sphinxlineitem{Type}
\sphinxAtStartPar
character(nvars)

\sphinxlineitem{Default}
\sphinxAtStartPar
None

\end{description}\end{quote}

\sphinxAtStartPar
For each JULES variable specified in {\hyperref[\detokenize{namelists/initial_conditions.nml:JULES_INITIAL::var}]{\sphinxcrossref{\sphinxcode{\sphinxupquote{var}}}}}, this is the string to substitute into the file name in place of the variable name substitution string.

\sphinxAtStartPar
If the file name does not use variable name templating, this is not used.

\end{fulllineitems}

\index{const\_val (in namelist JULES\_INITIAL)@\spxentry{const\_val}\spxextra{in namelist JULES\_INITIAL}|spxpagem}

\begin{fulllineitems}
\phantomsection\label{\detokenize{namelists/initial_conditions.nml:JULES_INITIAL::const_val}}
\pysigstartsignatures
\pysigline{\sphinxcode{\sphinxupquote{JULES\_INITIAL::}}\sphinxbfcode{\sphinxupquote{const\_val}}}
\pysigstopsignatures\begin{quote}\begin{description}
\sphinxlineitem{Type}
\sphinxAtStartPar
real(nvars)

\sphinxlineitem{Default}
\sphinxAtStartPar
None

\end{description}\end{quote}

\sphinxAtStartPar
For each JULES variable specified in {\hyperref[\detokenize{namelists/initial_conditions.nml:JULES_INITIAL::var}]{\sphinxcrossref{\sphinxcode{\sphinxupquote{var}}}}} where {\hyperref[\detokenize{namelists/initial_conditions.nml:JULES_INITIAL::use_file}]{\sphinxcrossref{\sphinxcode{\sphinxupquote{use\_file}}}}} = FALSE, this is a constant value that the variable will be set to at every point in every layer.

\sphinxAtStartPar
This is not used for variables where {\hyperref[\detokenize{namelists/initial_conditions.nml:JULES_INITIAL::use_file}]{\sphinxcrossref{\sphinxcode{\sphinxupquote{use\_file}}}}} = TRUE, but a placeholder must still be given.

\end{fulllineitems}

\index{l\_broadcast\_soilt (in namelist JULES\_INITIAL)@\spxentry{l\_broadcast\_soilt}\spxextra{in namelist JULES\_INITIAL}|spxpagem}

\begin{fulllineitems}
\phantomsection\label{\detokenize{namelists/initial_conditions.nml:JULES_INITIAL::l_broadcast_soilt}}
\pysigstartsignatures
\pysigline{\sphinxcode{\sphinxupquote{JULES\_INITIAL::}}\sphinxbfcode{\sphinxupquote{l\_broadcast\_soilt}}}
\pysigstopsignatures\begin{quote}\begin{description}
\sphinxlineitem{Type}
\sphinxAtStartPar
logical

\sphinxlineitem{Default}
\sphinxAtStartPar
False

\end{description}\end{quote}

\sphinxAtStartPar
Switch to allow non\sphinxhyphen{}soil tiled initial condition data to be broadcast to all soil tiles. This is only used when {\hyperref[\detokenize{namelists/jules_soil.nml:JULES_SOIL::l_tile_soil}]{\sphinxcrossref{\sphinxcode{\sphinxupquote{l\_tile\_soil}}}}} is enabled. This helps distribute the model state, for example from a non\sphinxhyphen{}soil tiled run into a new run with soil tiling. Spin up of the model state should be considered when using this option.

\sphinxAtStartPar
Note that if {\hyperref[\detokenize{namelists/jules_soil.nml:JULES_SOIL::l_tile_soil}]{\sphinxcrossref{\sphinxcode{\sphinxupquote{l\_tile\_soil}}}}} = TRUE and values on soil tiles are available to define the initial state (e.g. from a previous run with soil tiling), {\hyperref[\detokenize{namelists/initial_conditions.nml:JULES_INITIAL::l_broadcast_soilt}]{\sphinxcrossref{\sphinxcode{\sphinxupquote{l\_broadcast\_soilt}}}}} should be set to FALSE. Setting it to TRUE will result in the run failing because it will attempt to read a non\sphinxhyphen{}tiled variable.

\end{fulllineitems}

\end{sphinxadmonition}


\subsubsection{List of initial condition variables}
\label{\detokenize{namelists/initial_conditions.nml:list-of-initial-condition-variables}}\label{\detokenize{namelists/initial_conditions.nml:id1}}
\sphinxAtStartPar
All input to the model must be on the same grid (see {\hyperref[\detokenize{input/overview::doc}]{\sphinxcrossref{\DUrole{doc}{Input files for JULES}}}}), and initial conditions are no different. Even when the variable is only required for land points, values must be provided for the full input grid. Variables read as initial conditions must have no time dimension.

\sphinxAtStartPar
The variables it is possible to specify as initial conditions can be grouped into ‘types’ depending on the number and size of the levels dimensions they are required to have. For NetCDF files, the dimension names are those specified in the {\hyperref[\detokenize{namelists/model_grid.nml:namelist-JULES_INPUT_GRID}]{\sphinxcrossref{\sphinxcode{\sphinxupquote{JULES\_INPUT\_GRID}}}}} namelist. For variables with no type specified, no levels dimensions should be used.

\sphinxAtStartPar
The required levels dimensions for each initial condition ‘type’ are given in the following table:


\begin{savenotes}\sphinxattablestart
\centering
\begin{tabulary}{\linewidth}[t]{|p{2.5cm}|p{2cm}|p{4cm}|p{6.5cm}|}
\hline
\sphinxstyletheadfamily 
\sphinxAtStartPar
Type
&\sphinxstyletheadfamily 
\sphinxAtStartPar
Number of levels
dimensions
&\sphinxstyletheadfamily 
\sphinxAtStartPar
Levels dimension name(s)
&\sphinxstyletheadfamily 
\sphinxAtStartPar
Levels dimension size(s)
\\
\hline
\sphinxAtStartPar
soil
&
\sphinxAtStartPar
1
&
\sphinxAtStartPar
{\hyperref[\detokenize{namelists/model_grid.nml:JULES_INPUT_GRID::soil_dim_name}]{\sphinxcrossref{\sphinxcode{\sphinxupquote{soil\_dim\_name}}}}}
&
\sphinxAtStartPar
{\hyperref[\detokenize{namelists/jules_soil.nml:JULES_SOIL::sm_levels}]{\sphinxcrossref{\sphinxcode{\sphinxupquote{sm\_levels}}}}}
\\
\hline
\sphinxAtStartPar
pft
&
\sphinxAtStartPar
1
&
\sphinxAtStartPar
{\hyperref[\detokenize{namelists/model_grid.nml:JULES_INPUT_GRID::pft_dim_name}]{\sphinxcrossref{\sphinxcode{\sphinxupquote{pft\_dim\_name}}}}}
&
\sphinxAtStartPar
{\hyperref[\detokenize{namelists/jules_surface_types.nml:JULES_SURFACE_TYPES::npft}]{\sphinxcrossref{\sphinxcode{\sphinxupquote{npft}}}}}
\\
\hline
\sphinxAtStartPar
cpft
&
\sphinxAtStartPar
1
&
\sphinxAtStartPar
{\hyperref[\detokenize{namelists/model_grid.nml:JULES_INPUT_GRID::cpft_dim_name}]{\sphinxcrossref{\sphinxcode{\sphinxupquote{cpft\_dim\_name}}}}}
&
\sphinxAtStartPar
{\hyperref[\detokenize{namelists/jules_surface_types.nml:JULES_SURFACE_TYPES::ncpft}]{\sphinxcrossref{\sphinxcode{\sphinxupquote{ncpft}}}}}
\\
\hline
\sphinxAtStartPar
type
&
\sphinxAtStartPar
1
&
\sphinxAtStartPar
{\hyperref[\detokenize{namelists/model_grid.nml:JULES_INPUT_GRID::type_dim_name}]{\sphinxcrossref{\sphinxcode{\sphinxupquote{type\_dim\_name}}}}}
&
\sphinxAtStartPar
\sphinxcode{\sphinxupquote{ntype}} ({\hyperref[\detokenize{namelists/jules_surface_types.nml:JULES_SURFACE_TYPES::npft}]{\sphinxcrossref{\sphinxcode{\sphinxupquote{npft}}}}} +
{\hyperref[\detokenize{namelists/jules_surface_types.nml:JULES_SURFACE_TYPES::nnvg}]{\sphinxcrossref{\sphinxcode{\sphinxupquote{nnvg}}}}})
\\
\hline
\sphinxAtStartPar
surft
&
\sphinxAtStartPar
1
&
\sphinxAtStartPar
{\hyperref[\detokenize{namelists/model_grid.nml:JULES_INPUT_GRID::tile_dim_name}]{\sphinxcrossref{\sphinxcode{\sphinxupquote{tile\_dim\_name}}}}}
&
\sphinxAtStartPar
\sphinxcode{\sphinxupquote{nsurft}} (1 if {\hyperref[\detokenize{namelists/jules_surface.nml:JULES_SURFACE::l_aggregate}]{\sphinxcrossref{\sphinxcode{\sphinxupquote{l\_aggregate}}}}}
= TRUE, \sphinxcode{\sphinxupquote{ntype}} otherwise)
\\
\hline
\sphinxAtStartPar
sclayer
&
\sphinxAtStartPar
1
&
\sphinxAtStartPar
{\hyperref[\detokenize{namelists/model_grid.nml:JULES_INPUT_GRID::sclayer_dim_name}]{\sphinxcrossref{\sphinxcode{\sphinxupquote{sclayer\_dim\_name}}}}}
&
\sphinxAtStartPar
Number of soil biogeochemistry layers.

\sphinxAtStartPar
If using the single\sphinxhyphen{}pool moodel
({\hyperref[\detokenize{namelists/jules_soil_biogeochem.nml:JULES_SOIL_BIOGEOCHEM::soil_bgc_model}]{\sphinxcrossref{\sphinxcode{\sphinxupquote{soil\_bgc\_model}}}}} = 1 )
this is 1.

\sphinxAtStartPar
If using the 4\sphinxhyphen{}pool model
({\hyperref[\detokenize{namelists/jules_soil_biogeochem.nml:JULES_SOIL_BIOGEOCHEM::soil_bgc_model}]{\sphinxcrossref{\sphinxcode{\sphinxupquote{soil\_bgc\_model}}}}} = 2)
with
{\hyperref[\detokenize{namelists/jules_soil_biogeochem.nml:JULES_SOIL_BIOGEOCHEM::l_layeredc}]{\sphinxcrossref{\sphinxcode{\sphinxupquote{l\_layeredc}}}}} = FALSE
this is 1, else with
{\hyperref[\detokenize{namelists/jules_soil_biogeochem.nml:JULES_SOIL_BIOGEOCHEM::l_layeredc}]{\sphinxcrossref{\sphinxcode{\sphinxupquote{l\_layeredc}}}}} = TRUE
this is equal to {\hyperref[\detokenize{namelists/jules_soil.nml:JULES_SOIL::sm_levels}]{\sphinxcrossref{\sphinxcode{\sphinxupquote{sm\_levels}}}}}.

\sphinxAtStartPar
If using the ECOSSE model
({\hyperref[\detokenize{namelists/jules_soil_biogeochem.nml:JULES_SOIL_BIOGEOCHEM::soil_bgc_model}]{\sphinxcrossref{\sphinxcode{\sphinxupquote{soil\_bgc\_model}}}}} = 3)
this is equal to
{\hyperref[\detokenize{namelists/jules_soil_ecosse.nml:JULES_SOIL_ECOSSE::dim_cslayer}]{\sphinxcrossref{\sphinxcode{\sphinxupquote{dim\_cslayer}}}}}.
\\
\hline
\sphinxAtStartPar
scpool
&
\sphinxAtStartPar
2
&
\sphinxAtStartPar
{\hyperref[\detokenize{namelists/model_grid.nml:JULES_INPUT_GRID::scpool_dim_name}]{\sphinxcrossref{\sphinxcode{\sphinxupquote{scpool\_dim\_name}}}}},
{\hyperref[\detokenize{namelists/model_grid.nml:JULES_INPUT_GRID::sclayer_dim_name}]{\sphinxcrossref{\sphinxcode{\sphinxupquote{sclayer\_dim\_name}}}}}
&
\sphinxAtStartPar
number of soil carbon pools (1 if
{\hyperref[\detokenize{namelists/jules_soil_biogeochem.nml:JULES_SOIL_BIOGEOCHEM::soil_bgc_model}]{\sphinxcrossref{\sphinxcode{\sphinxupquote{soil\_bgc\_model}}}}} = 1,
4 otherwise) and number of soil biogeochemistry layers
(see \sphinxcode{\sphinxupquote{sclayer}} above)
\\
\hline
\sphinxAtStartPar
bedrock
&
\sphinxAtStartPar
1
&
\sphinxAtStartPar
{\hyperref[\detokenize{namelists/model_grid.nml:JULES_INPUT_GRID::bedrock_dim_name}]{\sphinxcrossref{\sphinxcode{\sphinxupquote{bedrock\_dim\_name}}}}}
&
\sphinxAtStartPar
{\hyperref[\detokenize{namelists/jules_soil.nml:JULES_SOIL::ns_deep}]{\sphinxcrossref{\sphinxcode{\sphinxupquote{ns\_deep}}}}}

\sphinxAtStartPar
Only applicable if {\hyperref[\detokenize{namelists/jules_soil.nml:JULES_SOIL::l_bedrock}]{\sphinxcrossref{\sphinxcode{\sphinxupquote{l\_bedrock}}}}} =
TRUE
\\
\hline
\sphinxAtStartPar
snow
&
\sphinxAtStartPar
2
&
\sphinxAtStartPar
{\hyperref[\detokenize{namelists/model_grid.nml:JULES_INPUT_GRID::tile_dim_name}]{\sphinxcrossref{\sphinxcode{\sphinxupquote{tile\_dim\_name}}}}},
{\hyperref[\detokenize{namelists/model_grid.nml:JULES_INPUT_GRID::snow_dim_name}]{\sphinxcrossref{\sphinxcode{\sphinxupquote{snow\_dim\_name}}}}}
&
\sphinxAtStartPar
\sphinxcode{\sphinxupquote{nsurft}} (see above), {\hyperref[\detokenize{namelists/jules_snow.nml:JULES_SNOW::nsmax}]{\sphinxcrossref{\sphinxcode{\sphinxupquote{nsmax}}}}}

\sphinxAtStartPar
Only applicable if \sphinxcode{\sphinxupquote{nsmax \textgreater{} 0}}
\\
\hline
\end{tabulary}
\par
\sphinxattableend\end{savenotes}

\sphinxAtStartPar
The required variables for a particular configuration, along with their ‘type’ as specified above, are given in the following table.


\begin{savenotes}\sphinxatlongtablestart\begin{longtable}[c]{|p{3cm}|p{9cm}|p{3.5cm}|}
\hline
\sphinxstyletheadfamily 
\sphinxAtStartPar
Name
&\sphinxstyletheadfamily 
\sphinxAtStartPar
Description
&\sphinxstyletheadfamily 
\sphinxAtStartPar
Type
\\
\hline
\endfirsthead

\multicolumn{3}{c}%
{\makebox[0pt]{\sphinxtablecontinued{\tablename\ \thetable{} \textendash{} continued from previous page}}}\\
\hline
\sphinxstyletheadfamily 
\sphinxAtStartPar
Name
&\sphinxstyletheadfamily 
\sphinxAtStartPar
Description
&\sphinxstyletheadfamily 
\sphinxAtStartPar
Type
\\
\hline
\endhead

\hline
\multicolumn{3}{r}{\makebox[0pt][r]{\sphinxtablecontinued{continues on next page}}}\\
\endfoot

\endlastfoot
\sphinxstartmulticolumn{3}%
\begin{varwidth}[t]{\sphinxcolwidth{3}{3}}
\sphinxAtStartPar
Always required
\par
\vskip-\baselineskip\vbox{\hbox{\strut}}\end{varwidth}%
\sphinxstopmulticolumn
\\
\hline
\sphinxAtStartPar
\sphinxcode{\sphinxupquote{canopy}}
&
\sphinxAtStartPar
Amount of intercepted water that is held on each surface tile (kg m$^{\text{\sphinxhyphen{}2}}$).
&
\sphinxAtStartPar
surft
\\
\hline
\sphinxAtStartPar
\sphinxcode{\sphinxupquote{cs}}
&
\sphinxAtStartPar
Soil carbon (kg m$^{\text{\sphinxhyphen{}2}}$).

\sphinxAtStartPar
If using the single\sphinxhyphen{}pool model ({\hyperref[\detokenize{namelists/jules_soil_biogeochem.nml:JULES_SOIL_BIOGEOCHEM::soil_bgc_model}]{\sphinxcrossref{\sphinxcode{\sphinxupquote{soil\_bgc\_model}}}}} = 1),
this is the total soil carbon.

\sphinxAtStartPar
Otherwise, this is the carbon in each of the 4 pools of the 4\sphinxhyphen{}pool or ECOSSE models.
&
\sphinxAtStartPar
scpool
\\
\hline
\sphinxAtStartPar
\sphinxcode{\sphinxupquote{snow\_tile}}
&
\sphinxAtStartPar
If {\hyperref[\detokenize{namelists/jules_vegetation.nml:JULES_VEGETATION::can_model}]{\sphinxcrossref{\sphinxcode{\sphinxupquote{can\_model}}}}} /= 4, this is the total snow on the surface
tile (since there is a single store which doesn’t distinguish between snow on canopy
and under canopy).

\sphinxAtStartPar
If {\hyperref[\detokenize{namelists/jules_vegetation.nml:JULES_VEGETATION::can_model}]{\sphinxcrossref{\sphinxcode{\sphinxupquote{can\_model}}}}} = 4 (and then only at surface tiles where
{\hyperref[\detokenize{namelists/jules_snow.nml:JULES_SNOW::cansnowpft}]{\sphinxcrossref{\sphinxcode{\sphinxupquote{cansnowpft}}}}} = TRUE), \sphinxcode{\sphinxupquote{snow\_tile}} is interpreted as the
snow on the canopy, except when overridden by {\hyperref[\detokenize{namelists/initial_conditions.nml:JULES_INITIAL::total_snow}]{\sphinxcrossref{\sphinxcode{\sphinxupquote{total\_snow}}}}}
= TRUE.

\sphinxAtStartPar
If {\hyperref[\detokenize{namelists/initial_conditions.nml:JULES_INITIAL::total_snow}]{\sphinxcrossref{\sphinxcode{\sphinxupquote{total\_snow}}}}} = TRUE, \sphinxcode{\sphinxupquote{snow\_tile}} is used to hold the
total snow on the surface tile (and is subsequently put onto the ground at tiles that
distinguish between ground and canopy stores).

\sphinxAtStartPar
Further details of snow initialisation are given below.
&
\sphinxAtStartPar
surft
\\
\hline
\sphinxAtStartPar
\sphinxcode{\sphinxupquote{t\_soil}}
&
\sphinxAtStartPar
Temperature of each soil layer (K).
&
\sphinxAtStartPar
soil
\\
\hline
\sphinxAtStartPar
\sphinxcode{\sphinxupquote{tstar\_tile}}
&
\sphinxAtStartPar
Temperature of each surface tile (K). This is the surface or skin temperature.
&
\sphinxAtStartPar
surft
\\
\hline\sphinxstartmulticolumn{3}%
\begin{varwidth}[t]{\sphinxcolwidth{3}{3}}
\sphinxAtStartPar
Required if {\hyperref[\detokenize{namelists/jules_vegetation.nml:JULES_VEGETATION::can_rad_mod}]{\sphinxcrossref{\sphinxcode{\sphinxupquote{can\_rad\_mod}}}}} = 1
\par
\vskip-\baselineskip\vbox{\hbox{\strut}}\end{varwidth}%
\sphinxstopmulticolumn
\\
\hline
\sphinxAtStartPar
\sphinxcode{\sphinxupquote{gs}}
&
\sphinxAtStartPar
Surface conductance for water vapour (m s$^{\text{\sphinxhyphen{}1}}$).

\sphinxAtStartPar
This is used to start the iterative calculation of gs for the first timestep only.
&
\sphinxAtStartPar
None
\\
\hline\sphinxstartmulticolumn{3}%
\begin{varwidth}[t]{\sphinxcolwidth{3}{3}}
\sphinxAtStartPar
Required if \sphinxcode{\sphinxupquote{sthuf}} is not prescribed for all levels in {\hyperref[\detokenize{namelists/prescribed_data.nml:namelist-JULES_PRESCRIBED}]{\sphinxcrossref{\sphinxcode{\sphinxupquote{JULES\_PRESCRIBED}}}}}
\par
\vskip-\baselineskip\vbox{\hbox{\strut}}\end{varwidth}%
\sphinxstopmulticolumn
\\
\hline
\sphinxAtStartPar
\sphinxcode{\sphinxupquote{sthuf}}
&
\sphinxAtStartPar
Soil wetness for each soil layer. This is the mass of soil water (liquid and frozen),
expressed as a fraction of the water content at saturation.
&
\sphinxAtStartPar
soil
\\
\hline\sphinxstartmulticolumn{3}%
\begin{varwidth}[t]{\sphinxcolwidth{3}{3}}
\sphinxAtStartPar
Required if {\hyperref[\detokenize{namelists/jules_vegetation.nml:JULES_VEGETATION::l_phenol}]{\sphinxcrossref{\sphinxcode{\sphinxupquote{l\_phenol}}}}} = TRUE
\par
\vskip-\baselineskip\vbox{\hbox{\strut}}\end{varwidth}%
\sphinxstopmulticolumn
\\
\hline
\sphinxAtStartPar
\sphinxcode{\sphinxupquote{lai}}
&
\sphinxAtStartPar
Leaf area index of each PFT.
&
\sphinxAtStartPar
pft
\\
\hline\sphinxstartmulticolumn{3}%
\begin{varwidth}[t]{\sphinxcolwidth{3}{3}}
\sphinxAtStartPar
Required if {\hyperref[\detokenize{namelists/jules_vegetation.nml:JULES_VEGETATION::l_triffid}]{\sphinxcrossref{\sphinxcode{\sphinxupquote{l\_triffid}}}}} = TRUE
\par
\vskip-\baselineskip\vbox{\hbox{\strut}}\end{varwidth}%
\sphinxstopmulticolumn
\\
\hline
\sphinxAtStartPar
\sphinxcode{\sphinxupquote{canht}}
&
\sphinxAtStartPar
Height (m) of each PFT.
&
\sphinxAtStartPar
pft
\\
\hline\sphinxstartmulticolumn{3}%
\begin{varwidth}[t]{\sphinxcolwidth{3}{3}}
\sphinxAtStartPar
Required if {\hyperref[\detokenize{namelists/jules_vegetation.nml:JULES_VEGETATION::l_trif_biocrop}]{\sphinxcrossref{\sphinxcode{\sphinxupquote{l\_trif\_biocrop}}}}} = TRUE
\par
\vskip-\baselineskip\vbox{\hbox{\strut}}\end{varwidth}%
\sphinxstopmulticolumn
\\
\hline
\sphinxAtStartPar
\sphinxcode{\sphinxupquote{years\_since\_harvest}}
&
\sphinxAtStartPar
Number of years since the previous harvest.
&
\sphinxAtStartPar
pft
\\
\hline\sphinxstartmulticolumn{3}%
\begin{varwidth}[t]{\sphinxcolwidth{3}{3}}
\sphinxAtStartPar
Required if {\hyperref[\detokenize{namelists/jules_vegetation.nml:JULES_VEGETATION::l_veg_compete}]{\sphinxcrossref{\sphinxcode{\sphinxupquote{l\_veg\_compete}}}}} = TRUE
\par
\vskip-\baselineskip\vbox{\hbox{\strut}}\end{varwidth}%
\sphinxstopmulticolumn
\\
\hline
\sphinxAtStartPar
\sphinxcode{\sphinxupquote{frac}}
&
\sphinxAtStartPar
The fraction of land area of each gridbox that is covered by each surface type.
N.B. values specified here will override those at {\hyperref[\detokenize{namelists/ancillaries.nml:namelist-JULES_FRAC}]{\sphinxcrossref{\sphinxcode{\sphinxupquote{JULES\_FRAC}}}}}
&
\sphinxAtStartPar
type
\\
\hline\sphinxstartmulticolumn{3}%
\begin{varwidth}[t]{\sphinxcolwidth{3}{3}}
\sphinxAtStartPar
Required if {\hyperref[\detokenize{namelists/jules_irrig.nml:JULES_IRRIG::l_irrig_dmd}]{\sphinxcrossref{\sphinxcode{\sphinxupquote{l\_irrig\_dmd}}}}} = TRUE
\par
\vskip-\baselineskip\vbox{\hbox{\strut}}\end{varwidth}%
\sphinxstopmulticolumn
\\
\hline
\sphinxAtStartPar
\sphinxcode{\sphinxupquote{sthu\_irr}}
&
\sphinxAtStartPar
Unfrozen soil wetness of each layer as a fraction of saturation in irrigated fraction.
&
\sphinxAtStartPar
soil
\\
\hline\sphinxstartmulticolumn{3}%
\begin{varwidth}[t]{\sphinxcolwidth{3}{3}}
\sphinxAtStartPar
Required if {\hyperref[\detokenize{namelists/jules_surface_types.nml:JULES_SURFACE_TYPES::ncpft}]{\sphinxcrossref{\sphinxcode{\sphinxupquote{ncpft}}}}} \textgreater{} 0
\par
\vskip-\baselineskip\vbox{\hbox{\strut}}\end{varwidth}%
\sphinxstopmulticolumn
\\
\hline
\sphinxAtStartPar
\sphinxcode{\sphinxupquote{cropdvi}}
&
\sphinxAtStartPar
Development index for each crop pft.
&
\sphinxAtStartPar
cpft
\\
\hline
\sphinxAtStartPar
\sphinxcode{\sphinxupquote{croprootc}}
&
\sphinxAtStartPar
Root carbon pool for each crop pft (kg m$^{\text{\sphinxhyphen{}2}}$).
&
\sphinxAtStartPar
cpft
\\
\hline
\sphinxAtStartPar
\sphinxcode{\sphinxupquote{cropharvc}}
&
\sphinxAtStartPar
Carbon in ‘harvest parts’ pool for each crop pft (kg m$^{\text{\sphinxhyphen{}2}}$) .
&
\sphinxAtStartPar
cpft
\\
\hline
\sphinxAtStartPar
\sphinxcode{\sphinxupquote{cropreservec}}
&
\sphinxAtStartPar
Carbon in stem reserves pool for each crop pft (kg m$^{\text{\sphinxhyphen{}2}}$).
&
\sphinxAtStartPar
cpft
\\
\hline
\sphinxAtStartPar
\sphinxcode{\sphinxupquote{croplai}}
&
\sphinxAtStartPar
Leaf area index of each crop pft.
&
\sphinxAtStartPar
cpft
\\
\hline
\sphinxAtStartPar
\sphinxcode{\sphinxupquote{cropcanht}}
&
\sphinxAtStartPar
Height (m) of each crop pft.
&
\sphinxAtStartPar
cpft
\\
\hline\sphinxstartmulticolumn{3}%
\begin{varwidth}[t]{\sphinxcolwidth{3}{3}}
\sphinxAtStartPar
Required if {\hyperref[\detokenize{namelists/jules_hydrology.nml:JULES_HYDROLOGY::l_top}]{\sphinxcrossref{\sphinxcode{\sphinxupquote{l\_top}}}}} = TRUE
\par
\vskip-\baselineskip\vbox{\hbox{\strut}}\end{varwidth}%
\sphinxstopmulticolumn
\\
\hline
\sphinxAtStartPar
\sphinxcode{\sphinxupquote{sthzw}}
&
\sphinxAtStartPar
Soil wetness in the deep LSH/TOPMODEL layer beneath the standard soil column.

\sphinxAtStartPar
This is the mass of soil water (liquid and frozen), expressed as a fraction of the
water content at saturation.
&
\sphinxAtStartPar
None
\\
\hline
\sphinxAtStartPar
\sphinxcode{\sphinxupquote{zw}}
&
\sphinxAtStartPar
Depth from the surface to the water table (m).
&
\sphinxAtStartPar
None
\\
\hline\sphinxstartmulticolumn{3}%
\begin{varwidth}[t]{\sphinxcolwidth{3}{3}}
\sphinxAtStartPar
Required if {\hyperref[\detokenize{namelists/jules_soil.nml:JULES_SOIL::l_bedrock}]{\sphinxcrossref{\sphinxcode{\sphinxupquote{l\_bedrock}}}}} = TRUE
\par
\vskip-\baselineskip\vbox{\hbox{\strut}}\end{varwidth}%
\sphinxstopmulticolumn
\\
\hline
\sphinxAtStartPar
\sphinxcode{\sphinxupquote{tsoil\_deep}}
&
\sphinxAtStartPar
Temperature of each bedrock layer (K)
&
\sphinxAtStartPar
bedrock
\\
\hline\sphinxstartmulticolumn{3}%
\begin{varwidth}[t]{\sphinxcolwidth{3}{3}}
\sphinxAtStartPar
Required if {\hyperref[\detokenize{namelists/jules_radiation.nml:JULES_RADIATION::l_snow_albedo}]{\sphinxcrossref{\sphinxcode{\sphinxupquote{l\_snow\_albedo}}}}} = TRUE
\par
\vskip-\baselineskip\vbox{\hbox{\strut}}\end{varwidth}%
\sphinxstopmulticolumn
\\
\hline
\sphinxAtStartPar
\sphinxcode{\sphinxupquote{rgrain}}
&
\sphinxAtStartPar
Snow surface grain size (μm) on each surface tile.
&
\sphinxAtStartPar
None
\\
\hline\sphinxstartmulticolumn{3}%
\begin{varwidth}[t]{\sphinxcolwidth{3}{3}}
\sphinxAtStartPar
Required if {\hyperref[\detokenize{namelists/initial_conditions.nml:JULES_INITIAL::total_snow}]{\sphinxcrossref{\sphinxcode{\sphinxupquote{total\_snow}}}}} = FALSE
\par
\vskip-\baselineskip\vbox{\hbox{\strut}}\end{varwidth}%
\sphinxstopmulticolumn
\\
\hline
\sphinxAtStartPar
\sphinxcode{\sphinxupquote{rho\_snow}}
&
\sphinxAtStartPar
Bulk density of lying snow (kg m$^{\text{\sphinxhyphen{}3}}$).

\sphinxAtStartPar
If {\hyperref[\detokenize{namelists/initial_conditions.nml:JULES_INITIAL::total_snow}]{\sphinxcrossref{\sphinxcode{\sphinxupquote{total\_snow}}}}} = TRUE then this is set as follows:
\begin{itemize}
\item {} 
\sphinxAtStartPar
If {\hyperref[\detokenize{namelists/jules_snow.nml:JULES_SNOW::nsmax}]{\sphinxcrossref{\sphinxcode{\sphinxupquote{nsmax}}}}} = 0, it is set to
{\hyperref[\detokenize{namelists/jules_snow.nml:JULES_SNOW::rho_snow_const}]{\sphinxcrossref{\sphinxcode{\sphinxupquote{rho\_snow\_const}}}}}.

\item {} 
\sphinxAtStartPar
If {\hyperref[\detokenize{namelists/jules_snow.nml:JULES_SNOW::nsmax}]{\sphinxcrossref{\sphinxcode{\sphinxupquote{nsmax}}}}} \textgreater{} 0 and there is an existing snow pack, it is set to
{\hyperref[\detokenize{namelists/jules_snow.nml:JULES_SNOW::rho_snow_const}]{\sphinxcrossref{\sphinxcode{\sphinxupquote{rho\_snow\_const}}}}}.

\item {} 
\sphinxAtStartPar
If {\hyperref[\detokenize{namelists/jules_snow.nml:JULES_SNOW::nsmax}]{\sphinxcrossref{\sphinxcode{\sphinxupquote{nsmax}}}}} \textgreater{} 0 and there is no snow pack, it is set to
{\hyperref[\detokenize{namelists/jules_snow.nml:JULES_SNOW::rho_snow_fresh}]{\sphinxcrossref{\sphinxcode{\sphinxupquote{rho\_snow\_fresh}}}}}.

\end{itemize}
&
\sphinxAtStartPar
surft
\\
\hline
\sphinxAtStartPar
\sphinxcode{\sphinxupquote{snow\_depth}}
&
\sphinxAtStartPar
Depth of snow (kg m).

\sphinxAtStartPar
If {\hyperref[\detokenize{namelists/initial_conditions.nml:JULES_INITIAL::total_snow}]{\sphinxcrossref{\sphinxcode{\sphinxupquote{total\_snow}}}}} = TRUE, this is calculated from mass and
density of snow.
&
\sphinxAtStartPar
surft
\\
\hline\sphinxstartmulticolumn{3}%
\begin{varwidth}[t]{\sphinxcolwidth{3}{3}}
\sphinxAtStartPar
Required if {\hyperref[\detokenize{namelists/initial_conditions.nml:JULES_INITIAL::total_snow}]{\sphinxcrossref{\sphinxcode{\sphinxupquote{total\_snow}}}}} = FALSE and {\hyperref[\detokenize{namelists/jules_vegetation.nml:JULES_VEGETATION::can_model}]{\sphinxcrossref{\sphinxcode{\sphinxupquote{can\_model}}}}} = 4
\par
\vskip-\baselineskip\vbox{\hbox{\strut}}\end{varwidth}%
\sphinxstopmulticolumn
\\
\hline
\sphinxAtStartPar
\sphinxcode{\sphinxupquote{snow\_grnd}}
&
\sphinxAtStartPar
Amount of snow on the ground, beneath the canopy (kg m$^{\text{\sphinxhyphen{}2}}$), on each surface
tile.

\sphinxAtStartPar
If {\hyperref[\detokenize{namelists/initial_conditions.nml:JULES_INITIAL::total_snow}]{\sphinxcrossref{\sphinxcode{\sphinxupquote{total\_snow}}}}} = TRUE this is set to \sphinxcode{\sphinxupquote{snow\_tile}} at tiles
where {\hyperref[\detokenize{namelists/jules_vegetation.nml:JULES_VEGETATION::can_model}]{\sphinxcrossref{\sphinxcode{\sphinxupquote{can\_model}}}}} = 4 is active, and to zero at all other
tiles.
&
\sphinxAtStartPar
surft
\\
\hline\sphinxstartmulticolumn{3}%
\begin{varwidth}[t]{\sphinxcolwidth{3}{3}}
\sphinxAtStartPar
Required if {\hyperref[\detokenize{namelists/initial_conditions.nml:JULES_INITIAL::total_snow}]{\sphinxcrossref{\sphinxcode{\sphinxupquote{total\_snow}}}}} = FALSE and {\hyperref[\detokenize{namelists/jules_snow.nml:JULES_SNOW::nsmax}]{\sphinxcrossref{\sphinxcode{\sphinxupquote{nsmax}}}}} \textgreater{} 0
\par
\vskip-\baselineskip\vbox{\hbox{\strut}}\end{varwidth}%
\sphinxstopmulticolumn
\\
\hline
\sphinxAtStartPar
\sphinxcode{\sphinxupquote{nsnow}}
&
\sphinxAtStartPar
The number of snow layers on each surface tile.

\sphinxAtStartPar
If {\hyperref[\detokenize{namelists/initial_conditions.nml:JULES_INITIAL::total_snow}]{\sphinxcrossref{\sphinxcode{\sphinxupquote{total\_snow}}}}} = TRUE this is calculated from the snow depth.
&
\sphinxAtStartPar
surft
\\
\hline
\sphinxAtStartPar
\sphinxcode{\sphinxupquote{snow\_ds}}
&
\sphinxAtStartPar
Depth of snow in each layer (kg m).

\sphinxAtStartPar
If {\hyperref[\detokenize{namelists/initial_conditions.nml:JULES_INITIAL::total_snow}]{\sphinxcrossref{\sphinxcode{\sphinxupquote{total\_snow}}}}} = TRUE this is calculated from the snow depth
and the number of snow layers.
&
\sphinxAtStartPar
snow
\\
\hline
\sphinxAtStartPar
\sphinxcode{\sphinxupquote{snow\_ice}}
&
\sphinxAtStartPar
Mass of frozen water in each snow layer (kg m$^{\text{\sphinxhyphen{}2}}$).

\sphinxAtStartPar
If {\hyperref[\detokenize{namelists/initial_conditions.nml:JULES_INITIAL::total_snow}]{\sphinxcrossref{\sphinxcode{\sphinxupquote{total\_snow}}}}} = TRUE all snow is assumed to be ice.
&
\sphinxAtStartPar
snow
\\
\hline
\sphinxAtStartPar
\sphinxcode{\sphinxupquote{snow\_liq}}
&
\sphinxAtStartPar
Mass of liquid water in each snow layer (kg m$^{\text{\sphinxhyphen{}2}}$).

\sphinxAtStartPar
If {\hyperref[\detokenize{namelists/initial_conditions.nml:JULES_INITIAL::total_snow}]{\sphinxcrossref{\sphinxcode{\sphinxupquote{total\_snow}}}}} = TRUE this is set to zero.
&
\sphinxAtStartPar
snow
\\
\hline
\sphinxAtStartPar
\sphinxcode{\sphinxupquote{tsnow}}
&
\sphinxAtStartPar
Temperature of each snow layer (K).

\sphinxAtStartPar
If {\hyperref[\detokenize{namelists/initial_conditions.nml:JULES_INITIAL::total_snow}]{\sphinxcrossref{\sphinxcode{\sphinxupquote{total\_snow}}}}} = TRUE this is set to the temperature of the
top soil layer.
&
\sphinxAtStartPar
snow
\\
\hline\sphinxstartmulticolumn{3}%
\begin{varwidth}[t]{\sphinxcolwidth{3}{3}}
\sphinxAtStartPar
Required if {\hyperref[\detokenize{namelists/initial_conditions.nml:JULES_INITIAL::total_snow}]{\sphinxcrossref{\sphinxcode{\sphinxupquote{total\_snow}}}}} = FALSE, {\hyperref[\detokenize{namelists/jules_snow.nml:JULES_SNOW::nsmax}]{\sphinxcrossref{\sphinxcode{\sphinxupquote{nsmax}}}}} \textgreater{} 0 and {\hyperref[\detokenize{namelists/jules_radiation.nml:JULES_RADIATION::l_snow_albedo}]{\sphinxcrossref{\sphinxcode{\sphinxupquote{l\_snow\_albedo}}}}} = TRUE
\par
\vskip-\baselineskip\vbox{\hbox{\strut}}\end{varwidth}%
\sphinxstopmulticolumn
\\
\hline
\sphinxAtStartPar
\sphinxcode{\sphinxupquote{rgrainl}}
&
\sphinxAtStartPar
Snow grain size (μm) on each surface tile in each snow layer.

\sphinxAtStartPar
If {\hyperref[\detokenize{namelists/initial_conditions.nml:JULES_INITIAL::total_snow}]{\sphinxcrossref{\sphinxcode{\sphinxupquote{total\_snow}}}}} = TRUE this is set to \sphinxcode{\sphinxupquote{rgrain}}.
&
\sphinxAtStartPar
snow
\\
\hline\sphinxstartmulticolumn{3}%
\begin{varwidth}[t]{\sphinxcolwidth{3}{3}}
\sphinxAtStartPar
Required if {\hyperref[\detokenize{namelists/jules_vegetation.nml:JULES_VEGETATION::l_triffid}]{\sphinxcrossref{\sphinxcode{\sphinxupquote{l\_triffid}}}}} = TRUE and {\hyperref[\detokenize{namelists/jules_vegetation.nml:JULES_VEGETATION::l_landuse}]{\sphinxcrossref{\sphinxcode{\sphinxupquote{l\_landuse}}}}} = TRUE
\par
\vskip-\baselineskip\vbox{\hbox{\strut}}\end{varwidth}%
\sphinxstopmulticolumn
\\
\hline
\sphinxAtStartPar
\sphinxcode{\sphinxupquote{frac\_agr\_prev}}
&
\sphinxAtStartPar
Gridbox agricultural/crop fraction from previous TRIFFID timestep.
&
\sphinxAtStartPar
none
\\
\hline
\sphinxAtStartPar
\sphinxcode{\sphinxupquote{wood\_prod\_fast}}
&
\sphinxAtStartPar
Carbon content of the wood products pool with a fast decay rate.
&
\sphinxAtStartPar
none
\\
\hline
\sphinxAtStartPar
\sphinxcode{\sphinxupquote{wood\_prod\_med}}
&
\sphinxAtStartPar
Carbon content of the wood products pool with a medium decay rate.
&
\sphinxAtStartPar
none
\\
\hline
\sphinxAtStartPar
\sphinxcode{\sphinxupquote{wood\_prod\_slow}}
&
\sphinxAtStartPar
Carbon content of the wood products pool with a slow decay rate.
&
\sphinxAtStartPar
none
\\
\hline\sphinxstartmulticolumn{3}%
\begin{varwidth}[t]{\sphinxcolwidth{3}{3}}
\sphinxAtStartPar
Required if {\hyperref[\detokenize{namelists/jules_vegetation.nml:JULES_VEGETATION::l_triffid}]{\sphinxcrossref{\sphinxcode{\sphinxupquote{l\_triffid}}}}} = TRUE and {\hyperref[\detokenize{namelists/jules_vegetation.nml:JULES_VEGETATION::l_landuse}]{\sphinxcrossref{\sphinxcode{\sphinxupquote{l\_landuse}}}}} = TRUE and
{\hyperref[\detokenize{namelists/jules_vegetation.nml:JULES_VEGETATION::l_trif_crop}]{\sphinxcrossref{\sphinxcode{\sphinxupquote{l\_trif\_crop}}}}} = TRUE
\par
\vskip-\baselineskip\vbox{\hbox{\strut}}\end{varwidth}%
\sphinxstopmulticolumn
\\
\hline
\sphinxAtStartPar
\sphinxcode{\sphinxupquote{frac\_past\_prev}}
&
\sphinxAtStartPar
Gridbox pasture fraction from previous TRIFFID timestep.
&
\sphinxAtStartPar
none
\\
\hline\sphinxstartmulticolumn{3}%
\begin{varwidth}[t]{\sphinxcolwidth{3}{3}}
\sphinxAtStartPar
Required if {\hyperref[\detokenize{namelists/jules_vegetation.nml:JULES_VEGETATION::l_triffid}]{\sphinxcrossref{\sphinxcode{\sphinxupquote{l\_triffid}}}}} = TRUE and {\hyperref[\detokenize{namelists/jules_vegetation.nml:JULES_VEGETATION::l_landuse}]{\sphinxcrossref{\sphinxcode{\sphinxupquote{l\_landuse}}}}} = TRUE and
{\hyperref[\detokenize{namelists/jules_vegetation.nml:JULES_VEGETATION::l_trif_biocrop}]{\sphinxcrossref{\sphinxcode{\sphinxupquote{l\_trif\_biocrop}}}}} = TRUE
\par
\vskip-\baselineskip\vbox{\hbox{\strut}}\end{varwidth}%
\sphinxstopmulticolumn
\\
\hline
\sphinxAtStartPar
\sphinxcode{\sphinxupquote{frac\_biocrop\_prev}}
&
\sphinxAtStartPar
Gridbox bioenergy fraction from previous TRIFFID timestep.
&
\sphinxAtStartPar
none
\\
\hline\sphinxstartmulticolumn{2}%
\begin{varwidth}[t]{\sphinxcolwidth{2}{3}}
\sphinxAtStartPar
Required if using 4\sphinxhyphen{}pool model ({\hyperref[\detokenize{namelists/jules_soil_biogeochem.nml:JULES_SOIL_BIOGEOCHEM::soil_bgc_model}]{\sphinxcrossref{\sphinxcode{\sphinxupquote{soil\_bgc\_model}}}}} = 2) with
{\hyperref[\detokenize{namelists/jules_vegetation.nml:JULES_VEGETATION::l_nitrogen}]{\sphinxcrossref{\sphinxcode{\sphinxupquote{l\_nitrogen}}}}} = TRUE., or if using ECOSSE
({\hyperref[\detokenize{namelists/jules_soil_biogeochem.nml:JULES_SOIL_BIOGEOCHEM::soil_bgc_model}]{\sphinxcrossref{\sphinxcode{\sphinxupquote{soil\_bgc\_model}}}}} = 3) with {\hyperref[\detokenize{namelists/jules_soil_ecosse.nml:JULES_SOIL_ECOSSE::l_soil_n}]{\sphinxcrossref{\sphinxcode{\sphinxupquote{l\_soil\_n}}}}} = TRUE.
\par
\vskip-\baselineskip\vbox{\hbox{\strut}}\end{varwidth}%
\sphinxstopmulticolumn
&\\
\hline
\sphinxAtStartPar
\sphinxcode{\sphinxupquote{ns}}
&
\sphinxAtStartPar
Soil nitrogen (kg m$^{\text{\sphinxhyphen{}2}}$).
&
\sphinxAtStartPar
scpool
\\
\hline\sphinxstartmulticolumn{2}%
\begin{varwidth}[t]{\sphinxcolwidth{2}{3}}
\sphinxAtStartPar
Required if using 4\sphinxhyphen{}pool model ({\hyperref[\detokenize{namelists/jules_soil_biogeochem.nml:JULES_SOIL_BIOGEOCHEM::soil_bgc_model}]{\sphinxcrossref{\sphinxcode{\sphinxupquote{soil\_bgc\_model}}}}} = 2) and
{\hyperref[\detokenize{namelists/jules_vegetation.nml:JULES_VEGETATION::l_nitrogen}]{\sphinxcrossref{\sphinxcode{\sphinxupquote{l\_nitrogen}}}}} = TRUE.
\par
\vskip-\baselineskip\vbox{\hbox{\strut}}\end{varwidth}%
\sphinxstopmulticolumn
&\\
\hline
\sphinxAtStartPar
\sphinxcode{\sphinxupquote{n\_inorg}}
&
\sphinxAtStartPar
Soil inorganic nitrogen  (kg m$^{\text{\sphinxhyphen{}2}}$).
&
\sphinxAtStartPar
sclayer
\\
\hline\sphinxstartmulticolumn{2}%
\begin{varwidth}[t]{\sphinxcolwidth{2}{3}}
\sphinxAtStartPar
Required if using ECOSSE ({\hyperref[\detokenize{namelists/jules_soil_biogeochem.nml:JULES_SOIL_BIOGEOCHEM::soil_bgc_model}]{\sphinxcrossref{\sphinxcode{\sphinxupquote{soil\_bgc\_model}}}}} = 3) and
{\hyperref[\detokenize{namelists/jules_soil_ecosse.nml:JULES_SOIL_ECOSSE::l_soil_n}]{\sphinxcrossref{\sphinxcode{\sphinxupquote{l\_soil\_n}}}}} = TRUE.
\par
\vskip-\baselineskip\vbox{\hbox{\strut}}\end{varwidth}%
\sphinxstopmulticolumn
&\\
\hline
\sphinxAtStartPar
\sphinxcode{\sphinxupquote{n\_amm}}
&
\sphinxAtStartPar
Soil ammonium  (kg m$^{\text{\sphinxhyphen{}2}}$).
&
\sphinxAtStartPar
sclayer
\\
\hline
\sphinxAtStartPar
\sphinxcode{\sphinxupquote{n\_nit}}
&
\sphinxAtStartPar
Soil nitrate  (kg m$^{\text{\sphinxhyphen{}2}}$).
&
\sphinxAtStartPar
sclayer
\\
\hline\sphinxstartmulticolumn{3}%
\begin{varwidth}[t]{\sphinxcolwidth{3}{3}}
\sphinxAtStartPar
Required if {\hyperref[\detokenize{namelists/jules_rivers.nml:JULES_RIVERS::l_rivers}]{\sphinxcrossref{\sphinxcode{\sphinxupquote{l\_rivers}}}}} = TRUE, {\hyperref[\detokenize{namelists/jules_rivers.nml:JULES_RIVERS::i_river_vn}]{\sphinxcrossref{\sphinxcode{\sphinxupquote{i\_river\_vn}}}}} = ‘2’ and
{\hyperref[\detokenize{namelists/initial_conditions.nml:JULES_INITIAL::dump_file}]{\sphinxcrossref{\sphinxcode{\sphinxupquote{dump\_file}}}}} = TRUE
\par
\vskip-\baselineskip\vbox{\hbox{\strut}}\end{varwidth}%
\sphinxstopmulticolumn
\\
\hline
\sphinxAtStartPar
\sphinxcode{\sphinxupquote{rfm\_surfstore\_rp}}
&
\sphinxAtStartPar
Surface water storage on river routing points (m3)
&
\sphinxAtStartPar
none
\\
\hline
\sphinxAtStartPar
\sphinxcode{\sphinxupquote{rfm\_substore\_rp}}
&
\sphinxAtStartPar
Sub\sphinxhyphen{}surface water storage on river routing points (m3)
&
\sphinxAtStartPar
none
\\
\hline
\sphinxAtStartPar
\sphinxcode{\sphinxupquote{rfm\_flowin\_rp}}
&
\sphinxAtStartPar
Surface flow into a grid box on river routing points (m3)
&
\sphinxAtStartPar
none
\\
\hline
\sphinxAtStartPar
\sphinxcode{\sphinxupquote{rfm\_bflowin\_rp}}
&
\sphinxAtStartPar
Sub\sphinxhyphen{}surface flow into a grid box on river routing points (m3)
&
\sphinxAtStartPar
none
\\
\hline\sphinxstartmulticolumn{3}%
\begin{varwidth}[t]{\sphinxcolwidth{3}{3}}
\sphinxAtStartPar
Required if {\hyperref[\detokenize{namelists/jules_rivers.nml:JULES_RIVERS::l_rivers}]{\sphinxcrossref{\sphinxcode{\sphinxupquote{l\_rivers}}}}} = TRUE, {\hyperref[\detokenize{namelists/jules_rivers.nml:JULES_RIVERS::i_river_vn}]{\sphinxcrossref{\sphinxcode{\sphinxupquote{i\_river\_vn}}}}} = ‘1,3’ and
{\hyperref[\detokenize{namelists/initial_conditions.nml:JULES_INITIAL::dump_file}]{\sphinxcrossref{\sphinxcode{\sphinxupquote{dump\_file}}}}} = TRUE
\par
\vskip-\baselineskip\vbox{\hbox{\strut}}\end{varwidth}%
\sphinxstopmulticolumn
\\
\hline
\sphinxAtStartPar
\sphinxcode{\sphinxupquote{rivers\_sto\_rp}}
&
\sphinxAtStartPar
Water storage (kg)
&
\sphinxAtStartPar
none
\\
\hline\sphinxstartmulticolumn{3}%
\begin{varwidth}[t]{\sphinxcolwidth{3}{3}}
\sphinxAtStartPar
Required if {\hyperref[\detokenize{namelists/jules_vegetation.nml:JULES_VEGETATION::photo_acclim_model}]{\sphinxcrossref{\sphinxcode{\sphinxupquote{photo\_acclim\_model}}}}} = 2 or 3
\par
\vskip-\baselineskip\vbox{\hbox{\strut}}\end{varwidth}%
\sphinxstopmulticolumn
\\
\hline
\sphinxAtStartPar
\sphinxcode{\sphinxupquote{t\_growth\_gb}}
&
\sphinxAtStartPar
Running mean air temperature (K)
&
\sphinxAtStartPar
none
\\
\hline
\end{longtable}\sphinxatlongtableend\end{savenotes}

\begin{sphinxadmonition}{warning}{Warning:}\begin{description}
\sphinxlineitem{if {\hyperref[\detokenize{namelists/jules_rivers.nml:JULES_RIVERS::l_rivers}]{\sphinxcrossref{\sphinxcode{\sphinxupquote{l\_rivers}}}}} = TRUE, {\hyperref[\detokenize{namelists/jules_rivers.nml:JULES_RIVERS::i_river_vn}]{\sphinxcrossref{\sphinxcode{\sphinxupquote{i\_river\_vn}}}}} = ‘2’ and {\hyperref[\detokenize{namelists/initial_conditions.nml:JULES_INITIAL::dump_file}]{\sphinxcrossref{\sphinxcode{\sphinxupquote{dump\_file}}}}} = FALSE,}
\sphinxAtStartPar
rfm\_surfstore\_rp, rfm\_substore\_rp, rfm\_flowin\_rp and rfm\_bflowin are initialised to zero.

\end{description}
\end{sphinxadmonition}

\begin{sphinxadmonition}{warning}{Warning:}
\sphinxAtStartPar
if {\hyperref[\detokenize{namelists/jules_rivers.nml:JULES_RIVERS::l_rivers}]{\sphinxcrossref{\sphinxcode{\sphinxupquote{l\_rivers}}}}} = TRUE, {\hyperref[\detokenize{namelists/jules_rivers.nml:JULES_RIVERS::i_river_vn}]{\sphinxcrossref{\sphinxcode{\sphinxupquote{i\_river\_vn}}}}} = ‘1,3’ and {\hyperref[\detokenize{namelists/initial_conditions.nml:JULES_INITIAL::dump_file}]{\sphinxcrossref{\sphinxcode{\sphinxupquote{dump\_file}}}}} = FALSE,
rivers\_sto\_rp is initialised to zero.
\end{sphinxadmonition}


\subsection{Examples of specification of initial state}
\label{\detokenize{namelists/initial_conditions.nml:examples-of-specification-of-initial-state}}

\subsubsection{Specification of initial state at a single point}
\label{\detokenize{namelists/initial_conditions.nml:specification-of-initial-state-at-a-single-point}}
\sphinxAtStartPar
This assumes that {\hyperref[\detokenize{namelists/jules_vegetation.nml:JULES_VEGETATION::l_phenol}]{\sphinxcrossref{\sphinxcode{\sphinxupquote{l\_phenol}}}}} = FALSE, {\hyperref[\detokenize{namelists/jules_vegetation.nml:JULES_VEGETATION::l_triffid}]{\sphinxcrossref{\sphinxcode{\sphinxupquote{l\_triffid}}}}} = FALSE, {\hyperref[\detokenize{namelists/jules_soil_biogeochem.nml:JULES_SOIL_BIOGEOCHEM::soil_bgc_model}]{\sphinxcrossref{\sphinxcode{\sphinxupquote{soil\_bgc\_model}}}}} = 1 and {\hyperref[\detokenize{namelists/jules_snow.nml:JULES_SNOW::nsmax}]{\sphinxcrossref{\sphinxcode{\sphinxupquote{nsmax}}}}} = 0.

\begin{sphinxVerbatim}[commandchars=\\\{\}]
\PYG{n+nn}{\PYGZam{}JULES\PYGZus{}INITIAL}
  \PYG{n+nv}{file} \PYG{o}{=} \PYG{l+s+s2}{\PYGZdq{}initial\PYGZus{}conditions.dat\PYGZdq{}}\PYG{p}{,}

  \PYG{n+nv}{nvars} \PYG{o}{=} \PYG{l+m+mi}{8}\PYG{p}{,}
  \PYG{n+nv}{var}       \PYG{o}{=} \PYG{l+s+s1}{\PYGZsq{}canopy\PYGZsq{}}  \PYG{l+s+s1}{\PYGZsq{}tstar\PYGZus{}tile\PYGZsq{}}   \PYG{l+s+s1}{\PYGZsq{}cs\PYGZsq{}}  \PYG{l+s+s1}{\PYGZsq{}gs\PYGZsq{}}  \PYG{l+s+s1}{\PYGZsq{}rgrain\PYGZsq{}}  \PYG{l+s+s1}{\PYGZsq{}snow\PYGZus{}tile\PYGZsq{}}  \PYG{l+s+s1}{\PYGZsq{}sthuf\PYGZsq{}}  \PYG{l+s+s1}{\PYGZsq{}t\PYGZus{}soil\PYGZsq{}}\PYG{p}{,}
  \PYG{n+nv}{use\PYGZus{}file}  \PYG{o}{=}       \PYG{l+s+ss}{F}             \PYG{l+s+ss}{F}      \PYG{l+s+ss}{F}     \PYG{l+s+ss}{F}         \PYG{l+s+ss}{F}            \PYG{l+s+ss}{F}        \PYG{l+s+ss}{T}         \PYG{l+s+ss}{T} \PYG{p}{,}
  \PYG{n+nv}{const\PYGZus{}val} \PYG{o}{=}     \PYG{l+m+mf}{0.0}        \PYG{l+m+mf}{276.78}   \PYG{l+m+mf}{12.1}   \PYG{l+m+mf}{0.0}      \PYG{l+m+mf}{50.0}          \PYG{l+m+mf}{0.0}
\PYG{n+nn}{/}
\end{sphinxVerbatim}

\sphinxAtStartPar
Or using the alternative list syntax (see {\hyperref[\detokenize{namelists/intro::doc}]{\sphinxcrossref{\DUrole{doc}{Introduction to Fortran namelists}}}}):

\begin{sphinxVerbatim}[commandchars=\\\{\}]
\PYG{n+nn}{\PYGZam{}JULES\PYGZus{}INITIAL}
  \PYG{n+nv}{file} \PYG{o}{=} \PYG{l+s+s2}{\PYGZdq{}initial\PYGZus{}conditions.dat\PYGZdq{}}\PYG{p}{,}

  \PYG{n+nv}{nvars} \PYG{o}{=} \PYG{l+m+mi}{8}\PYG{p}{,}
  \PYG{n+nv}{var}\PYG{p}{(}\PYG{l+m+mi}{1}\PYG{p}{)} \PYG{o}{=} \PYG{l+s+s1}{\PYGZsq{}canopy\PYGZsq{}}\PYG{p}{,}      \PYG{n+nv}{use\PYGZus{}file}\PYG{p}{(}\PYG{l+m+mi}{1}\PYG{p}{)} \PYG{o}{=} \PYG{l+s+ss}{F}\PYG{p}{,}  \PYG{n+nv}{const\PYGZus{}val}\PYG{p}{(}\PYG{l+m+mi}{1}\PYG{p}{)} \PYG{o}{=} \PYG{l+m+mf}{0.0} \PYG{p}{,}
  \PYG{n+nv}{var}\PYG{p}{(}\PYG{l+m+mi}{2}\PYG{p}{)} \PYG{o}{=} \PYG{l+s+s1}{\PYGZsq{}tstar\PYGZus{}tile\PYGZsq{}}\PYG{p}{,}  \PYG{n+nv}{use\PYGZus{}file}\PYG{p}{(}\PYG{l+m+mi}{2}\PYG{p}{)} \PYG{o}{=} \PYG{l+s+ss}{F}\PYG{p}{,}  \PYG{n+nv}{const\PYGZus{}val}\PYG{p}{(}\PYG{l+m+mi}{2}\PYG{p}{)} \PYG{o}{=} \PYG{l+m+mf}{276.78}\PYG{p}{,}
  \PYG{n+nv}{var}\PYG{p}{(}\PYG{l+m+mi}{3}\PYG{p}{)} \PYG{o}{=} \PYG{l+s+s1}{\PYGZsq{}cs\PYGZsq{}}\PYG{p}{,}          \PYG{n+nv}{use\PYGZus{}file}\PYG{p}{(}\PYG{l+m+mi}{3}\PYG{p}{)} \PYG{o}{=} \PYG{l+s+ss}{F}\PYG{p}{,}  \PYG{n+nv}{const\PYGZus{}val}\PYG{p}{(}\PYG{l+m+mi}{3}\PYG{p}{)} \PYG{o}{=} \PYG{l+m+mf}{12.1}\PYG{p}{,}
  \PYG{n+nv}{var}\PYG{p}{(}\PYG{l+m+mi}{4}\PYG{p}{)} \PYG{o}{=} \PYG{l+s+s1}{\PYGZsq{}gs\PYGZsq{}}\PYG{p}{,}          \PYG{n+nv}{use\PYGZus{}file}\PYG{p}{(}\PYG{l+m+mi}{4}\PYG{p}{)} \PYG{o}{=} \PYG{l+s+ss}{F}\PYG{p}{,}  \PYG{n+nv}{const\PYGZus{}val}\PYG{p}{(}\PYG{l+m+mi}{4}\PYG{p}{)} \PYG{o}{=} \PYG{l+m+mf}{0.0}\PYG{p}{,}
  \PYG{n+nv}{var}\PYG{p}{(}\PYG{l+m+mi}{5}\PYG{p}{)} \PYG{o}{=} \PYG{l+s+s1}{\PYGZsq{}rgrain\PYGZsq{}}\PYG{p}{,}      \PYG{n+nv}{use\PYGZus{}file}\PYG{p}{(}\PYG{l+m+mi}{5}\PYG{p}{)} \PYG{o}{=} \PYG{l+s+ss}{F}\PYG{p}{,}  \PYG{n+nv}{const\PYGZus{}val}\PYG{p}{(}\PYG{l+m+mi}{5}\PYG{p}{)} \PYG{o}{=} \PYG{l+m+mf}{50.0}\PYG{p}{,}
  \PYG{n+nv}{var}\PYG{p}{(}\PYG{l+m+mi}{6}\PYG{p}{)} \PYG{o}{=} \PYG{l+s+s1}{\PYGZsq{}snow\PYGZus{}tile\PYGZsq{}}\PYG{p}{,}   \PYG{n+nv}{use\PYGZus{}file}\PYG{p}{(}\PYG{l+m+mi}{6}\PYG{p}{)} \PYG{o}{=} \PYG{l+s+ss}{F}\PYG{p}{,}  \PYG{n+nv}{const\PYGZus{}val}\PYG{p}{(}\PYG{l+m+mi}{6}\PYG{p}{)} \PYG{o}{=} \PYG{l+m+mf}{0.0}\PYG{p}{,}
  \PYG{n+nv}{var}\PYG{p}{(}\PYG{l+m+mi}{7}\PYG{p}{)} \PYG{o}{=} \PYG{l+s+s1}{\PYGZsq{}sthuf\PYGZsq{}}\PYG{p}{,}       \PYG{n+nv}{use\PYGZus{}file}\PYG{p}{(}\PYG{l+m+mi}{7}\PYG{p}{)} \PYG{o}{=} \PYG{l+s+ss}{T}\PYG{p}{,}
  \PYG{n+nv}{var}\PYG{p}{(}\PYG{l+m+mi}{8}\PYG{p}{)} \PYG{o}{=} \PYG{l+s+s1}{\PYGZsq{}t\PYGZus{}soil\PYGZsq{}}\PYG{p}{,}      \PYG{n+nv}{use\PYGZus{}file}\PYG{p}{(}\PYG{l+m+mi}{8}\PYG{p}{)} \PYG{o}{=} \PYG{l+s+ss}{T}\PYG{p}{,}
\PYG{n+nn}{/}
\end{sphinxVerbatim}

\sphinxAtStartPar
This shows how a mixture of constant values and initial state from a file can be used. In this case, the first 6 variables will be set to constant values everywhere ({\hyperref[\detokenize{namelists/initial_conditions.nml:JULES_INITIAL::use_file}]{\sphinxcrossref{\sphinxcode{\sphinxupquote{use\_file}}}}} = FALSE) with the last 2 read from the specified file ({\hyperref[\detokenize{namelists/initial_conditions.nml:JULES_INITIAL::use_file}]{\sphinxcrossref{\sphinxcode{\sphinxupquote{use\_file}}}}} = TRUE).

\sphinxAtStartPar
{\hyperref[\detokenize{namelists/initial_conditions.nml:JULES_INITIAL::file}]{\sphinxcrossref{\sphinxcode{\sphinxupquote{file}}}}} specifies an ASCII file to read the variables for which {\hyperref[\detokenize{namelists/initial_conditions.nml:JULES_INITIAL::use_file}]{\sphinxcrossref{\sphinxcode{\sphinxupquote{use\_file}}}}} = TRUE from.

\sphinxAtStartPar
Since the variables are arranged such that all those with {\hyperref[\detokenize{namelists/initial_conditions.nml:JULES_INITIAL::use_file}]{\sphinxcrossref{\sphinxcode{\sphinxupquote{use\_file}}}}} = FALSE are first, we need only supply constant values for those variables that require them.

\sphinxAtStartPar
The contents of \sphinxcode{\sphinxupquote{initial\_conditions.dat}} should look similar to:

\begin{sphinxVerbatim}[commandchars=\\\{\}]
\PYG{c+c1}{\PYGZsh{} sthuf(1:4)                    t\PYGZus{}soil(1:4)}
  \PYG{l+m+mf}{0.749}  \PYG{l+m+mf}{0.743}  \PYG{l+m+mf}{0.754}  \PYG{l+m+mf}{0.759}    \PYG{l+m+mf}{276.78}  \PYG{l+m+mf}{277.46}  \PYG{l+m+mf}{278.99}  \PYG{l+m+mf}{282.48}
\end{sphinxVerbatim}

\sphinxAtStartPar
The data for each soil layer is given in consecutive columns. A comment line is used to indicate which columns comprise which variable (see {\hyperref[\detokenize{input/overview::doc}]{\sphinxcrossref{\DUrole{doc}{Input files for JULES}}}} for more details).

\sphinxAtStartPar
Specifying initial state for gridded data using NetCDF files is similar, except that:
\begin{itemize}
\item {} 
\sphinxAtStartPar
{\hyperref[\detokenize{namelists/initial_conditions.nml:JULES_INITIAL::var_name}]{\sphinxcrossref{\sphinxcode{\sphinxupquote{var\_name}}}}} is required for each variable read from file.

\item {} 
\sphinxAtStartPar
If variable name templating is used, {\hyperref[\detokenize{namelists/initial_conditions.nml:JULES_INITIAL::tpl_name}]{\sphinxcrossref{\sphinxcode{\sphinxupquote{tpl\_name}}}}} is required for each variable read from file.

\end{itemize}


\subsubsection{Specification of initial state from an existing dump file}
\label{\detokenize{namelists/initial_conditions.nml:specification-of-initial-state-from-an-existing-dump-file}}
\sphinxAtStartPar
In this example, we use an existing dump file (from a previous run) to set the initial values of all required variables.

\begin{sphinxVerbatim}[commandchars=\\\{\}]
\PYG{n+nn}{\PYGZam{}JULES\PYGZus{}INITIAL}
  \PYG{n+nv}{dump\PYGZus{}file} \PYG{o}{=} \PYG{l+s+ss}{T}\PYG{p}{,}
  \PYG{n+nv}{file} \PYG{o}{=} \PYG{l+s+s2}{\PYGZdq{}jules\PYGZus{}dump.nc\PYGZdq{}}
\PYG{n+nn}{/}
\end{sphinxVerbatim}

\sphinxAtStartPar
{\hyperref[\detokenize{namelists/initial_conditions.nml:JULES_INITIAL::dump_file}]{\sphinxcrossref{\sphinxcode{\sphinxupquote{dump\_file}}}}} = TRUE indicates that the given file should be interpreted as a JULES dump file.

\sphinxAtStartPar
{\hyperref[\detokenize{namelists/initial_conditions.nml:JULES_INITIAL::file}]{\sphinxcrossref{\sphinxcode{\sphinxupquote{file}}}}} specifies the dump file to read (in this case a NetCDF dump file).

\sphinxAtStartPar
Since it is not specified, {\hyperref[\detokenize{namelists/initial_conditions.nml:JULES_INITIAL::nvars}]{\sphinxcrossref{\sphinxcode{\sphinxupquote{nvars}}}}} takes its default value of 0, which indicates that JULES should attempt to read all required variables from the given dump file.

\sphinxstepscope


\section{\sphinxstyleliteralintitle{\sphinxupquote{output.nml}}}
\label{\detokenize{namelists/output.nml:output-nml}}\label{\detokenize{namelists/output.nml::doc}}
\sphinxAtStartPar
This file contains a variable number of namelists that are used to specify the output required by the user. The namelist {\hyperref[\detokenize{namelists/output.nml:namelist-JULES_OUTPUT}]{\sphinxcrossref{\sphinxcode{\sphinxupquote{JULES\_OUTPUT}}}}} should occur only once at the top of the file. The value of {\hyperref[\detokenize{namelists/output.nml:JULES_OUTPUT::nprofiles}]{\sphinxcrossref{\sphinxcode{\sphinxupquote{nprofiles}}}}} in {\hyperref[\detokenize{namelists/output.nml:namelist-JULES_OUTPUT}]{\sphinxcrossref{\sphinxcode{\sphinxupquote{JULES\_OUTPUT}}}}} then determines how many times the namelist {\hyperref[\detokenize{namelists/output.nml:namelist-JULES_OUTPUT_PROFILE}]{\sphinxcrossref{\sphinxcode{\sphinxupquote{JULES\_OUTPUT\_PROFILE}}}}} should appear.


\subsection{\sphinxstyleliteralintitle{\sphinxupquote{JULES\_OUTPUT}} namelist members}
\label{\detokenize{namelists/output.nml:namelist-JULES_OUTPUT}}\label{\detokenize{namelists/output.nml:jules-output-namelist-members}}\index{JULES\_OUTPUT (namelist)@\spxentry{JULES\_OUTPUT}\spxextra{namelist}|spxpagem}\index{output\_dir (in namelist JULES\_OUTPUT)@\spxentry{output\_dir}\spxextra{in namelist JULES\_OUTPUT}|spxpagem}

\begin{fulllineitems}
\phantomsection\label{\detokenize{namelists/output.nml:JULES_OUTPUT::output_dir}}
\pysigstartsignatures
\pysigline{\sphinxcode{\sphinxupquote{JULES\_OUTPUT::}}\sphinxbfcode{\sphinxupquote{output\_dir}}}
\pysigstopsignatures\begin{quote}\begin{description}
\sphinxlineitem{Type}
\sphinxAtStartPar
character

\sphinxlineitem{Default}
\sphinxAtStartPar
None

\end{description}\end{quote}

\sphinxAtStartPar
The directory used for output files. This can be an absolute or relative path.

\end{fulllineitems}

\index{run\_id (in namelist JULES\_OUTPUT)@\spxentry{run\_id}\spxextra{in namelist JULES\_OUTPUT}|spxpagem}

\begin{fulllineitems}
\phantomsection\label{\detokenize{namelists/output.nml:JULES_OUTPUT::run_id}}
\pysigstartsignatures
\pysigline{\sphinxcode{\sphinxupquote{JULES\_OUTPUT::}}\sphinxbfcode{\sphinxupquote{run\_id}}}
\pysigstopsignatures\begin{quote}\begin{description}
\sphinxlineitem{Type}
\sphinxAtStartPar
character

\sphinxlineitem{Default}
\sphinxAtStartPar
None

\end{description}\end{quote}

\sphinxAtStartPar
A name or identifier for the run. This is used to name output files and model dumps.

\end{fulllineitems}

\index{nprofiles (in namelist JULES\_OUTPUT)@\spxentry{nprofiles}\spxextra{in namelist JULES\_OUTPUT}|spxpagem}

\begin{fulllineitems}
\phantomsection\label{\detokenize{namelists/output.nml:JULES_OUTPUT::nprofiles}}
\pysigstartsignatures
\pysigline{\sphinxcode{\sphinxupquote{JULES\_OUTPUT::}}\sphinxbfcode{\sphinxupquote{nprofiles}}}
\pysigstopsignatures\begin{quote}\begin{description}
\sphinxlineitem{Type}
\sphinxAtStartPar
integer

\sphinxlineitem{Permitted}
\sphinxAtStartPar
\textgreater{}= 0

\sphinxlineitem{Default}
\sphinxAtStartPar
0

\end{description}\end{quote}

\sphinxAtStartPar
The number of output profiles that will be specified using instances of the {\hyperref[\detokenize{namelists/output.nml:namelist-JULES_OUTPUT_PROFILE}]{\sphinxcrossref{\sphinxcode{\sphinxupquote{JULES\_OUTPUT\_PROFILE}}}}} namelist.

\end{fulllineitems}

\index{dump\_period (in namelist JULES\_OUTPUT)@\spxentry{dump\_period}\spxextra{in namelist JULES\_OUTPUT}|spxpagem}

\begin{fulllineitems}
\phantomsection\label{\detokenize{namelists/output.nml:JULES_OUTPUT::dump_period}}
\pysigstartsignatures
\pysigline{\sphinxcode{\sphinxupquote{JULES\_OUTPUT::}}\sphinxbfcode{\sphinxupquote{dump\_period}}}
\pysigstopsignatures\begin{quote}\begin{description}
\sphinxlineitem{Type}
\sphinxAtStartPar
integer

\sphinxlineitem{Permitted}
\sphinxAtStartPar
\textgreater{}= 1

\sphinxlineitem{Default}
\sphinxAtStartPar
1

\end{description}\end{quote}

\sphinxAtStartPar
The period between model dumps, unit depends on {\hyperref[\detokenize{namelists/output.nml:JULES_OUTPUT::dump_period_unit}]{\sphinxcrossref{\sphinxcode{\sphinxupquote{dump\_period\_unit}}}}}.

\sphinxAtStartPar
In calendar year mode, the number of years between model dumps. Note that the calendar year (date) is used to determine whether a dump is written, not the number of years simulated so far. For example, a run that starts in the year 2012 and with {\hyperref[\detokenize{namelists/output.nml:JULES_OUTPUT::dump_period}]{\sphinxcrossref{\sphinxcode{\sphinxupquote{dump\_period}}}}} = 5 will write dumps at the start of years 2015, 2020, 2025,…

\sphinxAtStartPar
In second of calendar day mode, the number of seconds between model dumps. Note that this is calculated using the number of seconds into the day, not the number seconds simulated so far. For example, with {\hyperref[\detokenize{namelists/output.nml:JULES_OUTPUT::dump_period_unit}]{\sphinxcrossref{\sphinxcode{\sphinxupquote{dump\_period\_unit}}}}} = ‘T’ and {\hyperref[\detokenize{namelists/output.nml:JULES_OUTPUT::dump_period}]{\sphinxcrossref{\sphinxcode{\sphinxupquote{dump\_period}}}}} = 10800, a run will write dumps at T00:00, T03:00, T06:00, T09:00, T12:00, T15:00, T18:00 and T21:00 of a calendar day.

\sphinxAtStartPar
Dumps are also written at the start and end of the main run, and at the start of each cycle of any spin up.

\end{fulllineitems}

\index{dump\_period\_unit (in namelist JULES\_OUTPUT)@\spxentry{dump\_period\_unit}\spxextra{in namelist JULES\_OUTPUT}|spxpagem}

\begin{fulllineitems}
\phantomsection\label{\detokenize{namelists/output.nml:JULES_OUTPUT::dump_period_unit}}
\pysigstartsignatures
\pysigline{\sphinxcode{\sphinxupquote{JULES\_OUTPUT::}}\sphinxbfcode{\sphinxupquote{dump\_period\_unit}}}
\pysigstopsignatures\begin{quote}\begin{description}
\sphinxlineitem{Type}
\sphinxAtStartPar
character(1)

\sphinxlineitem{Permitted}
\sphinxAtStartPar
‘Y’, ‘T’

\sphinxlineitem{Default}
\sphinxAtStartPar
‘Y’

\end{description}\end{quote}

\sphinxAtStartPar
The unit/mode for the model dump period setting {\hyperref[\detokenize{namelists/output.nml:JULES_OUTPUT::dump_period}]{\sphinxcrossref{\sphinxcode{\sphinxupquote{dump\_period}}}}}. If {\hyperref[\detokenize{namelists/output.nml:JULES_OUTPUT::dump_period_unit}]{\sphinxcrossref{\sphinxcode{\sphinxupquote{dump\_period\_unit}}}}} = ‘Y’, use calendar year mode (default) and {\hyperref[\detokenize{namelists/output.nml:JULES_OUTPUT::dump_period}]{\sphinxcrossref{\sphinxcode{\sphinxupquote{dump\_period}}}}} is in number of years. If {\hyperref[\detokenize{namelists/output.nml:JULES_OUTPUT::dump_period_unit}]{\sphinxcrossref{\sphinxcode{\sphinxupquote{dump\_period\_unit}}}}} = ‘T’, use second of calendar day mode and {\hyperref[\detokenize{namelists/output.nml:JULES_OUTPUT::dump_period}]{\sphinxcrossref{\sphinxcode{\sphinxupquote{dump\_period}}}}} is in number of seconds.

\end{fulllineitems}



\subsection{\sphinxstyleliteralintitle{\sphinxupquote{JULES\_OUTPUT\_PROFILE}} namelist members}
\label{\detokenize{namelists/output.nml:namelist-JULES_OUTPUT_PROFILE}}\label{\detokenize{namelists/output.nml:jules-output-profile-namelist-members}}\index{JULES\_OUTPUT\_PROFILE (namelist)@\spxentry{JULES\_OUTPUT\_PROFILE}\spxextra{namelist}|spxpagem}
\sphinxAtStartPar
This namelist should occur {\hyperref[\detokenize{namelists/output.nml:JULES_OUTPUT::nprofiles}]{\sphinxcrossref{\sphinxcode{\sphinxupquote{nprofiles}}}}} times. Each occurrence of this namelist contains information about a single output profile, as described in {\hyperref[\detokenize{output::doc}]{\sphinxcrossref{\DUrole{doc}{JULES output}}}}.
\index{profile\_name (in namelist JULES\_OUTPUT\_PROFILE)@\spxentry{profile\_name}\spxextra{in namelist JULES\_OUTPUT\_PROFILE}|spxpagem}

\begin{fulllineitems}
\phantomsection\label{\detokenize{namelists/output.nml:JULES_OUTPUT_PROFILE::profile_name}}
\pysigstartsignatures
\pysigline{\sphinxcode{\sphinxupquote{JULES\_OUTPUT\_PROFILE::}}\sphinxbfcode{\sphinxupquote{profile\_name}}}
\pysigstopsignatures\begin{quote}\begin{description}
\sphinxlineitem{Type}
\sphinxAtStartPar
character

\sphinxlineitem{Default}
\sphinxAtStartPar
None

\end{description}\end{quote}

\sphinxAtStartPar
The name of the output profile.

\sphinxAtStartPar
This is used in file names and should be specified even if there is only one profile. The name for each profile should be unique to avoid overwriting data unintentionally.

\sphinxAtStartPar
Although any name can be used for a profile, the user may wish to choose a name that reflects the variables in the file (e.g. ‘carbon’, ‘water’) or the data frequency (e.g. ‘daily’, ‘monthly’).

\end{fulllineitems}

\index{l\_land\_frac (in namelist JULES\_OUTPUT\_PROFILE)@\spxentry{l\_land\_frac}\spxextra{in namelist JULES\_OUTPUT\_PROFILE}|spxpagem}

\begin{fulllineitems}
\phantomsection\label{\detokenize{namelists/output.nml:JULES_OUTPUT_PROFILE::l_land_frac}}
\pysigstartsignatures
\pysigline{\sphinxcode{\sphinxupquote{JULES\_OUTPUT\_PROFILE::}}\sphinxbfcode{\sphinxupquote{l\_land\_frac}}}
\pysigstopsignatures\begin{quote}\begin{description}
\sphinxlineitem{Type}
\sphinxAtStartPar
logical

\sphinxlineitem{Default}
\sphinxAtStartPar
F

\end{description}\end{quote}

\sphinxAtStartPar
Output gridbox land fraction to output profile.

\sphinxAtStartPar
Required to add gribox land fraction to profile for example to
allow JULES output to drive another JULES model.

\end{fulllineitems}

\index{file\_period (in namelist JULES\_OUTPUT\_PROFILE)@\spxentry{file\_period}\spxextra{in namelist JULES\_OUTPUT\_PROFILE}|spxpagem}

\begin{fulllineitems}
\phantomsection\label{\detokenize{namelists/output.nml:JULES_OUTPUT_PROFILE::file_period}}
\pysigstartsignatures
\pysigline{\sphinxcode{\sphinxupquote{JULES\_OUTPUT\_PROFILE::}}\sphinxbfcode{\sphinxupquote{file\_period}}}
\pysigstopsignatures\begin{quote}\begin{description}
\sphinxlineitem{Type}
\sphinxAtStartPar
integer

\sphinxlineitem{Permitted}
\sphinxAtStartPar
\sphinxhyphen{}3, \sphinxhyphen{}2, \sphinxhyphen{}1 or 0

\sphinxlineitem{Default}
\sphinxAtStartPar
0

\end{description}\end{quote}

\sphinxAtStartPar
The period for output files, i.e. the time interval during which all output goes to the same file.

\begin{sphinxadmonition}{note}{Note:}
\sphinxAtStartPar
In all cases, output during spin\sphinxhyphen{}up goes into a separate file for each spin\sphinxhyphen{}up cycle and output during the main run goes into its own file(s).
\end{sphinxadmonition}

\sphinxAtStartPar
This can be one of three values:
\begin{description}
\sphinxlineitem{0}
\sphinxAtStartPar
All output goes into the same file.

\sphinxlineitem{\sphinxhyphen{}1}
\sphinxAtStartPar
Monthly files are produced (i.e. all output for a month goes into the same file).

\sphinxlineitem{\sphinxhyphen{}2}
\sphinxAtStartPar
Annual files (calendar years) are produced.

\sphinxlineitem{\sphinxhyphen{}3}
\sphinxAtStartPar
Daily files are produced.

\end{description}

\end{fulllineitems}


\begin{sphinxadmonition}{note}{Members used to specify the times that the profile will generate output}
\index{output\_spinup (in namelist JULES\_OUTPUT\_PROFILE)@\spxentry{output\_spinup}\spxextra{in namelist JULES\_OUTPUT\_PROFILE}|spxpagem}

\begin{fulllineitems}
\phantomsection\label{\detokenize{namelists/output.nml:JULES_OUTPUT_PROFILE::output_spinup}}
\pysigstartsignatures
\pysigline{\sphinxcode{\sphinxupquote{JULES\_OUTPUT\_PROFILE::}}\sphinxbfcode{\sphinxupquote{output\_spinup}}}
\pysigstopsignatures\begin{quote}\begin{description}
\sphinxlineitem{Type}
\sphinxAtStartPar
logical

\sphinxlineitem{Default}
\sphinxAtStartPar
F

\end{description}\end{quote}

\sphinxAtStartPar
Determines whether the profile will provide output during model spin\sphinxhyphen{}up.
\begin{description}
\sphinxlineitem{TRUE}
\sphinxAtStartPar
Provide output during spin\sphinxhyphen{}up. Output is provided for the whole of the model spin\sphinxhyphen{}up. Output from each spin\sphinxhyphen{}up cycle goes into separate files.

\sphinxlineitem{FALSE}
\sphinxAtStartPar
Do not provide any output during spin\sphinxhyphen{}up.

\end{description}

\end{fulllineitems}

\index{output\_main\_run (in namelist JULES\_OUTPUT\_PROFILE)@\spxentry{output\_main\_run}\spxextra{in namelist JULES\_OUTPUT\_PROFILE}|spxpagem}

\begin{fulllineitems}
\phantomsection\label{\detokenize{namelists/output.nml:JULES_OUTPUT_PROFILE::output_main_run}}
\pysigstartsignatures
\pysigline{\sphinxcode{\sphinxupquote{JULES\_OUTPUT\_PROFILE::}}\sphinxbfcode{\sphinxupquote{output\_main\_run}}}
\pysigstopsignatures\begin{quote}\begin{description}
\sphinxlineitem{Type}
\sphinxAtStartPar
logical

\sphinxlineitem{Default}
\sphinxAtStartPar
F

\end{description}\end{quote}

\sphinxAtStartPar
Determines whether the profile will provide output during the main model run (i.e. any part of the run after spin\sphinxhyphen{}up).
\begin{description}
\sphinxlineitem{TRUE}
\sphinxAtStartPar
Provide output during the main model run. Output will be provided for all times between {\hyperref[\detokenize{namelists/output.nml:JULES_OUTPUT_PROFILE::output_start}]{\sphinxcrossref{\sphinxcode{\sphinxupquote{output\_start}}}}} and {\hyperref[\detokenize{namelists/output.nml:JULES_OUTPUT_PROFILE::output_end}]{\sphinxcrossref{\sphinxcode{\sphinxupquote{output\_end}}}}} below.

\sphinxlineitem{FALSE}
\sphinxAtStartPar
Do not provide any output during the main model run.

\end{description}

\end{fulllineitems}

\index{output\_initial (in namelist JULES\_OUTPUT\_PROFILE)@\spxentry{output\_initial}\spxextra{in namelist JULES\_OUTPUT\_PROFILE}|spxpagem}

\begin{fulllineitems}
\phantomsection\label{\detokenize{namelists/output.nml:JULES_OUTPUT_PROFILE::output_initial}}
\pysigstartsignatures
\pysigline{\sphinxcode{\sphinxupquote{JULES\_OUTPUT\_PROFILE::}}\sphinxbfcode{\sphinxupquote{output\_initial}}}
\pysigstopsignatures\begin{quote}\begin{description}
\sphinxlineitem{Type}
\sphinxAtStartPar
logical

\sphinxlineitem{Default}
\sphinxAtStartPar
F

\end{description}\end{quote}

\sphinxAtStartPar
Determines whether the profile will output initial data for the sections for which it is outputting.

\sphinxAtStartPar
See {\hyperref[\detokenize{output:initial-data}]{\sphinxcrossref{\DUrole{std,std-ref}{Initial data}}}} for caveats on the initial data file(s) produced.
\begin{description}
\sphinxlineitem{TRUE}
\sphinxAtStartPar
Output initial data for the profile.

\sphinxAtStartPar
If {\hyperref[\detokenize{namelists/output.nml:JULES_OUTPUT_PROFILE::output_spinup}]{\sphinxcrossref{\sphinxcode{\sphinxupquote{output\_spinup}}}}} = T, an initial data file will be output at the start of each spinup cycle.

\sphinxAtStartPar
If {\hyperref[\detokenize{namelists/output.nml:JULES_OUTPUT_PROFILE::output_main_run}]{\sphinxcrossref{\sphinxcode{\sphinxupquote{output\_main\_run}}}}} = T and {\hyperref[\detokenize{namelists/output.nml:JULES_OUTPUT_PROFILE::output_start}]{\sphinxcrossref{\sphinxcode{\sphinxupquote{output\_start}}}}} = {\hyperref[\detokenize{namelists/timesteps.nml:JULES_TIME::main_run_start}]{\sphinxcrossref{\sphinxcode{\sphinxupquote{main\_run\_start}}}}}, an initial data file will be output at the start of the main run.

\sphinxlineitem{FALSE}
\sphinxAtStartPar
Do not output any initial data.

\end{description}

\end{fulllineitems}

\index{output\_start (in namelist JULES\_OUTPUT\_PROFILE)@\spxentry{output\_start}\spxextra{in namelist JULES\_OUTPUT\_PROFILE}|spxpagem}

\begin{fulllineitems}
\phantomsection\label{\detokenize{namelists/output.nml:JULES_OUTPUT_PROFILE::output_start}}
\pysigstartsignatures
\pysigline{\sphinxcode{\sphinxupquote{JULES\_OUTPUT\_PROFILE::}}\sphinxbfcode{\sphinxupquote{output\_start}}}
\pysigstopsignatures
\end{fulllineitems}

\index{output\_end (in namelist JULES\_OUTPUT\_PROFILE)@\spxentry{output\_end}\spxextra{in namelist JULES\_OUTPUT\_PROFILE}|spxpagem}

\begin{fulllineitems}
\phantomsection\label{\detokenize{namelists/output.nml:JULES_OUTPUT_PROFILE::output_end}}
\pysigstartsignatures
\pysigline{\sphinxcode{\sphinxupquote{JULES\_OUTPUT\_PROFILE::}}\sphinxbfcode{\sphinxupquote{output\_end}}}
\pysigstopsignatures\begin{quote}\begin{description}
\sphinxlineitem{Type}
\sphinxAtStartPar
character

\sphinxlineitem{Default}
\sphinxAtStartPar
{\hyperref[\detokenize{namelists/timesteps.nml:JULES_TIME::main_run_start}]{\sphinxcrossref{\sphinxcode{\sphinxupquote{main\_run\_start}}}}}, {\hyperref[\detokenize{namelists/timesteps.nml:JULES_TIME::main_run_end}]{\sphinxcrossref{\sphinxcode{\sphinxupquote{main\_run\_end}}}}}

\end{description}\end{quote}

\sphinxAtStartPar
The time to start and stop collecting data for output. The times that output is actually produced are determined by {\hyperref[\detokenize{namelists/output.nml:JULES_OUTPUT_PROFILE::output_period}]{\sphinxcrossref{\sphinxcode{\sphinxupquote{output\_period}}}}} below.

\sphinxAtStartPar
If {\hyperref[\detokenize{namelists/output.nml:JULES_OUTPUT_PROFILE::output_period}]{\sphinxcrossref{\sphinxcode{\sphinxupquote{output\_period}}}}} is monthly, then {\hyperref[\detokenize{namelists/output.nml:JULES_OUTPUT_PROFILE::output_start}]{\sphinxcrossref{\sphinxcode{\sphinxupquote{output\_start}}}}} must be 00:00:00 on the 1st of some month.

\sphinxAtStartPar
If {\hyperref[\detokenize{namelists/output.nml:JULES_OUTPUT_PROFILE::output_period}]{\sphinxcrossref{\sphinxcode{\sphinxupquote{output\_period}}}}} is yearly, then {\hyperref[\detokenize{namelists/output.nml:JULES_OUTPUT_PROFILE::output_start}]{\sphinxcrossref{\sphinxcode{\sphinxupquote{output\_start}}}}} must be 00:00:00 on the 1st of January for some year.

\sphinxAtStartPar
If {\hyperref[\detokenize{namelists/output.nml:JULES_OUTPUT_PROFILE::output_end}]{\sphinxcrossref{\sphinxcode{\sphinxupquote{output\_end}}}}} is given such that an output period is not complete when the run reaches {\hyperref[\detokenize{namelists/output.nml:JULES_OUTPUT_PROFILE::output_end}]{\sphinxcrossref{\sphinxcode{\sphinxupquote{output\_end}}}}}, output will not be generated for that final period (e.g. if values are being output monthly but {\hyperref[\detokenize{namelists/output.nml:JULES_OUTPUT_PROFILE::output_end}]{\sphinxcrossref{\sphinxcode{\sphinxupquote{output\_end}}}}} is midday on the 31st December, then output will not be generated for December, even though most of December has been run).

\sphinxAtStartPar
The times must be given in the format:

\begin{sphinxVerbatim}[commandchars=\\\{\}]
\PYG{l+s+s2}{\PYGZdq{}yyyy\PYGZhy{}mm\PYGZhy{}dd hh:mm:ss\PYGZdq{}}
\end{sphinxVerbatim}

\end{fulllineitems}

\index{output\_period (in namelist JULES\_OUTPUT\_PROFILE)@\spxentry{output\_period}\spxextra{in namelist JULES\_OUTPUT\_PROFILE}|spxpagem}

\begin{fulllineitems}
\phantomsection\label{\detokenize{namelists/output.nml:JULES_OUTPUT_PROFILE::output_period}}
\pysigstartsignatures
\pysigline{\sphinxcode{\sphinxupquote{JULES\_OUTPUT\_PROFILE::}}\sphinxbfcode{\sphinxupquote{output\_period}}}
\pysigstopsignatures\begin{quote}\begin{description}
\sphinxlineitem{Type}
\sphinxAtStartPar
integer

\sphinxlineitem{Permitted}
\sphinxAtStartPar
\sphinxhyphen{}2, \sphinxhyphen{}1 or \textgreater{} 1

\sphinxlineitem{Default}
\sphinxAtStartPar
{\hyperref[\detokenize{namelists/timesteps.nml:JULES_TIME::timestep_len}]{\sphinxcrossref{\sphinxcode{\sphinxupquote{timestep\_len}}}}}

\end{description}\end{quote}

\sphinxAtStartPar
The period for output, in seconds. This controls the frequency with which output values are calculated; for time averages this is the length of time over which the average is calculated.

\sphinxAtStartPar
This must be a multiple of the timestep length, except for the special cases:

\begin{DUlineblock}{0em}
\item[] \sphinxstylestrong{\sphinxhyphen{}1:} Monthly period
\item[] \sphinxstylestrong{\sphinxhyphen{}2:} Annual period
\end{DUlineblock}

\end{fulllineitems}

\index{sample\_period (in namelist JULES\_OUTPUT\_PROFILE)@\spxentry{sample\_period}\spxextra{in namelist JULES\_OUTPUT\_PROFILE}|spxpagem}

\begin{fulllineitems}
\phantomsection\label{\detokenize{namelists/output.nml:JULES_OUTPUT_PROFILE::sample_period}}
\pysigstartsignatures
\pysigline{\sphinxcode{\sphinxupquote{JULES\_OUTPUT\_PROFILE::}}\sphinxbfcode{\sphinxupquote{sample\_period}}}
\pysigstopsignatures\begin{quote}\begin{description}
\sphinxlineitem{Type}
\sphinxAtStartPar
integer

\sphinxlineitem{Permitted}
\sphinxAtStartPar
\textgreater{} 1

\sphinxlineitem{Default}
\sphinxAtStartPar
{\hyperref[\detokenize{namelists/timesteps.nml:JULES_TIME::timestep_len}]{\sphinxcrossref{\sphinxcode{\sphinxupquote{timestep\_len}}}}}

\end{description}\end{quote}

\sphinxAtStartPar
The sampling period, in seconds, for time\sphinxhyphen{}averages, minima, maxima and accumulations.

\sphinxAtStartPar
This must be a factor of {\hyperref[\detokenize{namelists/output.nml:JULES_OUTPUT_PROFILE::output_period}]{\sphinxcrossref{\sphinxcode{\sphinxupquote{output\_period}}}}} and a multiple of {\hyperref[\detokenize{namelists/timesteps.nml:JULES_TIME::timestep_len}]{\sphinxcrossref{\sphinxcode{\sphinxupquote{timestep\_len}}}}}. For the ‘special’ cases of {\hyperref[\detokenize{namelists/output.nml:JULES_OUTPUT_PROFILE::output_period}]{\sphinxcrossref{\sphinxcode{\sphinxupquote{output\_period}}}}} (e.g. monthly and annual outputs), one day must be a mutiple of {\hyperref[\detokenize{namelists/output.nml:JULES_OUTPUT_PROFILE::sample_period}]{\sphinxcrossref{\sphinxcode{\sphinxupquote{sample\_period}}}}}.

\sphinxAtStartPar
{\hyperref[\detokenize{namelists/output.nml:JULES_OUTPUT_PROFILE::output_period}]{\sphinxcrossref{\sphinxcode{\sphinxupquote{output\_period}}}}} controls the length of time over which any statistic (e.g. an average) is calculated; {\hyperref[\detokenize{namelists/output.nml:JULES_OUTPUT_PROFILE::sample_period}]{\sphinxcrossref{\sphinxcode{\sphinxupquote{sample\_period}}}}} controls how often values are sampled within that time to construct the statistic.

\begin{sphinxadmonition}{note}{Note:}
\sphinxAtStartPar
It is strongly recommended that {\hyperref[\detokenize{namelists/output.nml:JULES_OUTPUT_PROFILE::sample_period}]{\sphinxcrossref{\sphinxcode{\sphinxupquote{sample\_period}}}}} be left at the default value for most applications.

\sphinxAtStartPar
An example of when intermittent sampling can be useful is to construct a time average of values from particular timesteps, e.g. those on which a particular sub\sphinxhyphen{}model is called, for a variable that is reset to zero on other timesteps. Sampling every timestep will average over both the ‘good’ values and the zeros, whereas intermittent sampling can be used to pick up just the ‘good’ values. However, this is very rarely required with JULES.

\sphinxAtStartPar
Another use for intermittent sampling is to save computational cost (at the expense of losing accuracy), though in practice this is rarely required. In exceptional cases, sampling every timestep can be relatively expensive and acceptable output can be achieved by sampling less frequently. For example, with a large domain, many output diagnostics and a timestep of 30 minutes, a monthly average would be calculated from several hundred values if every timestep was used. For variables that evolve relatively slowly, an acceptable monthly average might be obtained by sampling only every 12 hours.

\sphinxAtStartPar
If fields are not sampled every timestep, the averages/minima/maxima/accumulations will only be approximations.
\end{sphinxadmonition}

\end{fulllineitems}

\end{sphinxadmonition}

\begin{sphinxadmonition}{note}{Members used to specify the variables that the profile will output}
\index{nvars (in namelist JULES\_OUTPUT\_PROFILE)@\spxentry{nvars}\spxextra{in namelist JULES\_OUTPUT\_PROFILE}|spxpagem}

\begin{fulllineitems}
\phantomsection\label{\detokenize{namelists/output.nml:JULES_OUTPUT_PROFILE::nvars}}
\pysigstartsignatures
\pysigline{\sphinxcode{\sphinxupquote{JULES\_OUTPUT\_PROFILE::}}\sphinxbfcode{\sphinxupquote{nvars}}}
\pysigstopsignatures\begin{quote}\begin{description}
\sphinxlineitem{Type}
\sphinxAtStartPar
integer

\sphinxlineitem{Permitted}
\sphinxAtStartPar
\textgreater{}= 0

\sphinxlineitem{Default}
\sphinxAtStartPar
0

\end{description}\end{quote}

\sphinxAtStartPar
The number of variables that the profile will provide output for.

\sphinxAtStartPar
The variables available for output are given in {\hyperref[\detokenize{output-variables::doc}]{\sphinxcrossref{\DUrole{doc}{JULES Output variables}}}}.

\end{fulllineitems}

\index{var (in namelist JULES\_OUTPUT\_PROFILE)@\spxentry{var}\spxextra{in namelist JULES\_OUTPUT\_PROFILE}|spxpagem}

\begin{fulllineitems}
\phantomsection\label{\detokenize{namelists/output.nml:JULES_OUTPUT_PROFILE::var}}
\pysigstartsignatures
\pysigline{\sphinxcode{\sphinxupquote{JULES\_OUTPUT\_PROFILE::}}\sphinxbfcode{\sphinxupquote{var}}}
\pysigstopsignatures\begin{quote}\begin{description}
\sphinxlineitem{Type}
\sphinxAtStartPar
character(nvars)

\sphinxlineitem{Default}
\sphinxAtStartPar
None

\end{description}\end{quote}

\sphinxAtStartPar
List of variable names to output, as recognised by JULES (see {\hyperref[\detokenize{output-variables::doc}]{\sphinxcrossref{\DUrole{doc}{JULES Output variables}}}}). Names are case sensitive.

\end{fulllineitems}

\index{var\_name (in namelist JULES\_OUTPUT\_PROFILE)@\spxentry{var\_name}\spxextra{in namelist JULES\_OUTPUT\_PROFILE}|spxpagem}

\begin{fulllineitems}
\phantomsection\label{\detokenize{namelists/output.nml:JULES_OUTPUT_PROFILE::var_name}}
\pysigstartsignatures
\pysigline{\sphinxcode{\sphinxupquote{JULES\_OUTPUT\_PROFILE::}}\sphinxbfcode{\sphinxupquote{var\_name}}}
\pysigstopsignatures\begin{quote}\begin{description}
\sphinxlineitem{Type}
\sphinxAtStartPar
character(nvars)

\sphinxlineitem{Default}
\sphinxAtStartPar
‘’ (empty string)

\end{description}\end{quote}

\sphinxAtStartPar
For each variable specified in {\hyperref[\detokenize{namelists/output.nml:JULES_OUTPUT_PROFILE::var}]{\sphinxcrossref{\sphinxcode{\sphinxupquote{var}}}}}, this is the name to give the variable in output files.

\sphinxAtStartPar
If the empty string (the default) is given for any variable, then the corresponding value from {\hyperref[\detokenize{namelists/output.nml:JULES_OUTPUT_PROFILE::var}]{\sphinxcrossref{\sphinxcode{\sphinxupquote{var}}}}} is used instead.

\end{fulllineitems}

\index{output\_type (in namelist JULES\_OUTPUT\_PROFILE)@\spxentry{output\_type}\spxextra{in namelist JULES\_OUTPUT\_PROFILE}|spxpagem}

\begin{fulllineitems}
\phantomsection\label{\detokenize{namelists/output.nml:JULES_OUTPUT_PROFILE::output_type}}
\pysigstartsignatures
\pysigline{\sphinxcode{\sphinxupquote{JULES\_OUTPUT\_PROFILE::}}\sphinxbfcode{\sphinxupquote{output\_type}}}
\pysigstopsignatures\begin{quote}\begin{description}
\sphinxlineitem{Type}
\sphinxAtStartPar
character(nvars)

\sphinxlineitem{Default}
\sphinxAtStartPar
‘S’

\end{description}\end{quote}

\sphinxAtStartPar
For each variable specified in {\hyperref[\detokenize{namelists/output.nml:JULES_OUTPUT_PROFILE::var}]{\sphinxcrossref{\sphinxcode{\sphinxupquote{var}}}}}, this indicates the type of processing required.

\sphinxAtStartPar
Recognised values are:
\begin{description}
\sphinxlineitem{S}
\sphinxAtStartPar
Instantaneous or snapshot value.

\sphinxlineitem{M}
\sphinxAtStartPar
Time mean value.

\sphinxlineitem{N}
\sphinxAtStartPar
Time minimum value.

\sphinxlineitem{X}
\sphinxAtStartPar
Time maximum value.

\sphinxlineitem{A}
\sphinxAtStartPar
Accumulation over time.

\end{description}

\sphinxAtStartPar
For time average/minimum/maximum variables, the period over which each output value is calculated is given by {\hyperref[\detokenize{namelists/output.nml:JULES_OUTPUT_PROFILE::output_period}]{\sphinxcrossref{\sphinxcode{\sphinxupquote{output\_period}}}}}. For time\sphinxhyphen{}accumulation variables, {\hyperref[\detokenize{namelists/output.nml:JULES_OUTPUT_PROFILE::output_period}]{\sphinxcrossref{\sphinxcode{\sphinxupquote{output\_period}}}}} gives the period for output of an updated accumulation (i.e. how often the value is reported). For time averages, minima, maxima and accumulations, the sampling frequency is given by {\hyperref[\detokenize{namelists/output.nml:JULES_OUTPUT_PROFILE::sample_period}]{\sphinxcrossref{\sphinxcode{\sphinxupquote{sample\_period}}}}}.

\begin{sphinxadmonition}{note}{Note:}
\sphinxAtStartPar
A time\sphinxhyphen{}accumulation is initialised at the start of a run and thereafter accumulates until the end of the run. This may mean that accuracy is lost, particularly towards the end of long runs, if small increments are added to an already large sum. Furthermore, before being output the accumulation is multiplied by the number of model timesteps in a sample period. This adjusts the accumulation for any intermittent sampling and is designed so that the time accumulation of a flux (e.g. kg s$^{\text{\sphinxhyphen{}1}}$) can easily be converted to a total (e.g. kg) by subsequently multiplying by the model timestep length during post processing.
\end{sphinxadmonition}

\end{fulllineitems}

\end{sphinxadmonition}


\subsection{Example of requesting output}
\label{\detokenize{namelists/output.nml:example-of-requesting-output}}
\sphinxAtStartPar
In this example, the user has requested two output profiles. One provides gridbox monthly means for the whole of the main run, the other provides snapshot values of per\sphinxhyphen{}tile variables every timestep for a single year. We assume that {\hyperref[\detokenize{namelists/timesteps.nml:JULES_TIME::timestep_len}]{\sphinxcrossref{\sphinxcode{\sphinxupquote{timestep\_len}}}}} = 1800, {\hyperref[\detokenize{namelists/timesteps.nml:JULES_TIME::main_run_start}]{\sphinxcrossref{\sphinxcode{\sphinxupquote{main\_run\_start}}}}} = ‘1995\sphinxhyphen{}01\sphinxhyphen{}01 00:00:00’ and {\hyperref[\detokenize{namelists/timesteps.nml:JULES_TIME::main_run_end}]{\sphinxcrossref{\sphinxcode{\sphinxupquote{main\_run\_end}}}}} = ‘2005\sphinxhyphen{}01\sphinxhyphen{}01 00:00:00’, so that exactly 10 years will be run. There is no spin\sphinxhyphen{}up.

\begin{sphinxVerbatim}[commandchars=\\\{\}]
\PYG{n+nn}{\PYGZam{}JULES\PYGZus{}OUTPUT}
  \PYG{n+nv}{run\PYGZus{}id} \PYG{o}{=} \PYG{l+s+s2}{\PYGZdq{}jules\PYGZus{}run001\PYGZdq{}}\PYG{p}{,}

  \PYG{n+nv}{output\PYGZus{}dir} \PYG{o}{=} \PYG{l+s+s2}{\PYGZdq{}./output\PYGZdq{}}\PYG{p}{,}

  \PYG{n+nv}{nprofiles} \PYG{o}{=} \PYG{l+m+mi}{2}
\PYG{n+nn}{/}

\PYG{n+nn}{\PYGZam{}JULES\PYGZus{}OUTPUT\PYGZus{}PROFILE}
  \PYG{n+nv}{profile\PYGZus{}name} \PYG{o}{=} \PYG{l+s+s2}{\PYGZdq{}month\PYGZdq{}}\PYG{p}{,}

  \PYG{n+nv}{output\PYGZus{}main\PYGZus{}run} \PYG{o}{=} \PYG{l+s+ss}{T}\PYG{p}{,}

  \PYG{n+nv}{output\PYGZus{}period} \PYG{o}{=} \PYG{l+m+mi}{\PYGZhy{}1}\PYG{p}{,}

  \PYG{n+nv}{nvars} \PYG{o}{=} \PYG{l+m+mi}{4}\PYG{p}{,}
  \PYG{n+nv}{var}         \PYG{o}{=}    \PYG{l+s+s2}{\PYGZdq{}emis\PYGZus{}gb\PYGZdq{}}        \PYG{l+s+s2}{\PYGZdq{}ftl\PYGZus{}gb\PYGZdq{}}  \PYG{l+s+s2}{\PYGZdq{}snow\PYGZus{}mass\PYGZus{}gb\PYGZdq{}}  \PYG{l+s+s2}{\PYGZdq{}tstar\PYGZus{}gb\PYGZdq{}}\PYG{p}{,}
  \PYG{n+nv}{var\PYGZus{}name}    \PYG{o}{=} \PYG{l+s+s2}{\PYGZdq{}Emissivity\PYGZdq{}}  \PYG{l+s+s2}{\PYGZdq{}SensibleHeat\PYGZdq{}}      \PYG{l+s+s2}{\PYGZdq{}SnowMass\PYGZdq{}}      \PYG{l+s+s2}{\PYGZdq{}Temp\PYGZdq{}}\PYG{p}{,}
  \PYG{n+nv}{output\PYGZus{}type} \PYG{o}{=}          \PYG{l+s+s1}{\PYGZsq{}M\PYGZsq{}}             \PYG{l+s+s1}{\PYGZsq{}M\PYGZsq{}}             \PYG{l+s+s1}{\PYGZsq{}M\PYGZsq{}}         \PYG{l+s+s1}{\PYGZsq{}M\PYGZsq{}}
\PYG{n+nn}{/}

\PYG{n+nn}{\PYGZam{}JULES\PYGZus{}OUTPUT\PYGZus{}PROFILE}
  \PYG{n+nv}{profile\PYGZus{}name} \PYG{o}{=} \PYG{l+s+s2}{\PYGZdq{}tstep\PYGZdq{}}\PYG{p}{,}

  \PYG{n+nv}{output\PYGZus{}main\PYGZus{}run} \PYG{o}{=} \PYG{l+s+ss}{T}\PYG{p}{,}
  \PYG{n+nv}{output\PYGZus{}start}    \PYG{o}{=} \PYG{l+s+s2}{\PYGZdq{}2000\PYGZhy{}01\PYGZhy{}01 00:00:00\PYGZdq{}}\PYG{p}{,}
  \PYG{n+nv}{output\PYGZus{}end}      \PYG{o}{=} \PYG{l+s+s2}{\PYGZdq{}2001\PYGZhy{}01\PYGZhy{}01 00:00:00\PYGZdq{}}\PYG{p}{,}

  \PYG{n+nv}{nvars} \PYG{o}{=} \PYG{l+m+mi}{4}\PYG{p}{,}
  \PYG{n+nv}{var}         \PYG{o}{=} \PYG{l+s+s2}{\PYGZdq{}emis\PYGZdq{}}  \PYG{l+s+s2}{\PYGZdq{}ftl\PYGZdq{}}  \PYG{l+s+s2}{\PYGZdq{}snow\PYGZus{}mass\PYGZdq{}}  \PYG{l+s+s2}{\PYGZdq{}tstar\PYGZdq{}}\PYG{p}{,}
  \PYG{n+nv}{output\PYGZus{}type} \PYG{o}{=}    \PYG{l+s+s1}{\PYGZsq{}S\PYGZsq{}}    \PYG{l+s+s1}{\PYGZsq{}S\PYGZsq{}}          \PYG{l+s+s1}{\PYGZsq{}S\PYGZsq{}}      \PYG{l+s+s1}{\PYGZsq{}S\PYGZsq{}}
\PYG{n+nn}{/}
\end{sphinxVerbatim}

\sphinxAtStartPar
Or using the alternative list syntax (see {\hyperref[\detokenize{namelists/intro::doc}]{\sphinxcrossref{\DUrole{doc}{Introduction to Fortran namelists}}}}):

\begin{sphinxVerbatim}[commandchars=\\\{\}]
\PYG{n+nn}{\PYGZam{}JULES\PYGZus{}OUTPUT}
  \PYG{c+c1}{\PYGZsh{} ... as above ...}
\PYG{n+nn}{/}

\PYG{n+nn}{\PYGZam{}JULES\PYGZus{}OUTPUT\PYGZus{}PROFILE}
  \PYG{c+c1}{\PYGZsh{} ... as above ...}

  \PYG{n+nv}{nvars} \PYG{o}{=} \PYG{l+m+mi}{4}\PYG{p}{,}
  \PYG{n+nv}{var}\PYG{p}{(}\PYG{l+m+mi}{1}\PYG{p}{)} \PYG{o}{=} \PYG{l+s+s2}{\PYGZdq{}emis\PYGZus{}gb\PYGZdq{}}\PYG{p}{,}       \PYG{n+nv}{var\PYGZus{}name}\PYG{p}{(}\PYG{l+m+mi}{1}\PYG{p}{)} \PYG{o}{=} \PYG{l+s+s2}{\PYGZdq{}Emissivity\PYGZdq{}}\PYG{p}{,}    \PYG{n+nv}{output\PYGZus{}type}\PYG{p}{(}\PYG{l+m+mi}{1}\PYG{p}{)} \PYG{o}{=} \PYG{l+s+s1}{\PYGZsq{}M\PYGZsq{}}\PYG{p}{,}
  \PYG{n+nv}{var}\PYG{p}{(}\PYG{l+m+mi}{2}\PYG{p}{)} \PYG{o}{=} \PYG{l+s+s2}{\PYGZdq{}ftl\PYGZus{}gb\PYGZdq{}}\PYG{p}{,}        \PYG{n+nv}{var\PYGZus{}name}\PYG{p}{(}\PYG{l+m+mi}{2}\PYG{p}{)} \PYG{o}{=} \PYG{l+s+s2}{\PYGZdq{}SensibleHeat\PYGZdq{}}\PYG{p}{,}  \PYG{n+nv}{output\PYGZus{}type}\PYG{p}{(}\PYG{l+m+mi}{2}\PYG{p}{)} \PYG{o}{=} \PYG{l+s+s1}{\PYGZsq{}M\PYGZsq{}}\PYG{p}{,}
  \PYG{n+nv}{var}\PYG{p}{(}\PYG{l+m+mi}{3}\PYG{p}{)} \PYG{o}{=} \PYG{l+s+s2}{\PYGZdq{}snow\PYGZus{}mass\PYGZus{}gb\PYGZdq{}}\PYG{p}{,}  \PYG{n+nv}{var\PYGZus{}name}\PYG{p}{(}\PYG{l+m+mi}{3}\PYG{p}{)} \PYG{o}{=} \PYG{l+s+s2}{\PYGZdq{}SnowMass\PYGZdq{}}\PYG{p}{,}      \PYG{n+nv}{output\PYGZus{}type}\PYG{p}{(}\PYG{l+m+mi}{3}\PYG{p}{)} \PYG{o}{=} \PYG{l+s+s1}{\PYGZsq{}M\PYGZsq{}}\PYG{p}{,}
  \PYG{n+nv}{var}\PYG{p}{(}\PYG{l+m+mi}{4}\PYG{p}{)} \PYG{o}{=} \PYG{l+s+s2}{\PYGZdq{}tstar\PYGZus{}gb\PYGZdq{}}\PYG{p}{,}      \PYG{n+nv}{var\PYGZus{}name}\PYG{p}{(}\PYG{l+m+mi}{4}\PYG{p}{)} \PYG{o}{=} \PYG{l+s+s2}{\PYGZdq{}Temp\PYGZdq{}}\PYG{p}{,}          \PYG{n+nv}{output\PYGZus{}type}\PYG{p}{(}\PYG{l+m+mi}{4}\PYG{p}{)} \PYG{o}{=} \PYG{l+s+s1}{\PYGZsq{}M\PYGZsq{}}
\PYG{n+nn}{/}

\PYG{n+nn}{\PYGZam{}JULES\PYGZus{}OUTPUT\PYGZus{}PROFILE}
  \PYG{c+c1}{\PYGZsh{} ... as above ...}

  \PYG{n+nv}{nvars} \PYG{o}{=} \PYG{l+m+mi}{4}\PYG{p}{,}
  \PYG{n+nv}{var}\PYG{p}{(}\PYG{l+m+mi}{1}\PYG{p}{)} \PYG{o}{=} \PYG{l+s+s2}{\PYGZdq{}emis\PYGZdq{}}\PYG{p}{,}       \PYG{n+nv}{output\PYGZus{}type}\PYG{p}{(}\PYG{l+m+mi}{1}\PYG{p}{)} \PYG{o}{=} \PYG{l+s+s1}{\PYGZsq{}S\PYGZsq{}}\PYG{p}{,}
  \PYG{n+nv}{var}\PYG{p}{(}\PYG{l+m+mi}{2}\PYG{p}{)} \PYG{o}{=} \PYG{l+s+s2}{\PYGZdq{}ftl\PYGZdq{}}\PYG{p}{,}        \PYG{n+nv}{output\PYGZus{}type}\PYG{p}{(}\PYG{l+m+mi}{2}\PYG{p}{)} \PYG{o}{=} \PYG{l+s+s1}{\PYGZsq{}S\PYGZsq{}}\PYG{p}{,}
  \PYG{n+nv}{var}\PYG{p}{(}\PYG{l+m+mi}{3}\PYG{p}{)} \PYG{o}{=} \PYG{l+s+s2}{\PYGZdq{}snow\PYGZus{}mass\PYGZdq{}}\PYG{p}{,}  \PYG{n+nv}{output\PYGZus{}type}\PYG{p}{(}\PYG{l+m+mi}{3}\PYG{p}{)} \PYG{o}{=} \PYG{l+s+s1}{\PYGZsq{}S\PYGZsq{}}\PYG{p}{,}
  \PYG{n+nv}{var}\PYG{p}{(}\PYG{l+m+mi}{4}\PYG{p}{)} \PYG{o}{=} \PYG{l+s+s2}{\PYGZdq{}tstar\PYGZdq{}}\PYG{p}{,}      \PYG{n+nv}{output\PYGZus{}type}\PYG{p}{(}\PYG{l+m+mi}{4}\PYG{p}{)} \PYG{o}{=} \PYG{l+s+s1}{\PYGZsq{}S\PYGZsq{}}
\PYG{n+nn}{/}
\end{sphinxVerbatim}

\sphinxAtStartPar
The {\hyperref[\detokenize{namelists/output.nml:namelist-JULES_OUTPUT}]{\sphinxcrossref{\sphinxcode{\sphinxupquote{JULES\_OUTPUT}}}}} namelist is simple \sphinxhyphen{} it gives an id for the run, specifies the directory to put output in and specifies the number of profile definitions that follow.

\sphinxAtStartPar
Each profile is given a unique name. Both profiles want to output part of the main run, so must set {\hyperref[\detokenize{namelists/output.nml:JULES_OUTPUT_PROFILE::output_main_run}]{\sphinxcrossref{\sphinxcode{\sphinxupquote{output\_main\_run}}}}} to TRUE. Since {\hyperref[\detokenize{namelists/output.nml:JULES_OUTPUT_PROFILE::sample_period}]{\sphinxcrossref{\sphinxcode{\sphinxupquote{sample\_period}}}}} is not given for either profile, both will use the default (sample every timestep). The same is true for {\hyperref[\detokenize{namelists/output.nml:JULES_OUTPUT_PROFILE::file_period}]{\sphinxcrossref{\sphinxcode{\sphinxupquote{file\_period}}}}} \sphinxhyphen{} since it is not given for either profile, it takes its default value and all output for each profile will go into a single file.

\sphinxAtStartPar
The first output profile wants to output monthly averages, so {\hyperref[\detokenize{namelists/output.nml:JULES_OUTPUT_PROFILE::output_period}]{\sphinxcrossref{\sphinxcode{\sphinxupquote{output\_period}}}}} is set to \sphinxhyphen{}1 (the special value indicating that calendar months should be used for the output period) and {\hyperref[\detokenize{namelists/output.nml:JULES_OUTPUT_PROFILE::output_type}]{\sphinxcrossref{\sphinxcode{\sphinxupquote{output\_type}}}}} is set to ‘M’ (for mean) for each variable. The user wants this profile to output for the whole of the main run, so does not need to specify {\hyperref[\detokenize{namelists/output.nml:JULES_OUTPUT_PROFILE::output_start}]{\sphinxcrossref{\sphinxcode{\sphinxupquote{output\_start}}}}} or {\hyperref[\detokenize{namelists/output.nml:JULES_OUTPUT_PROFILE::output_end}]{\sphinxcrossref{\sphinxcode{\sphinxupquote{output\_end}}}}} (note that this is only possible because the main run starts at 00:00:00 on the 1st of a month \sphinxhyphen{} if this was not the case, the user would have to specify a different time for {\hyperref[\detokenize{namelists/output.nml:JULES_OUTPUT_PROFILE::output_start}]{\sphinxcrossref{\sphinxcode{\sphinxupquote{output\_start}}}}}). The user has also chosen to supply the names that each variable will use in output files using {\hyperref[\detokenize{namelists/output.nml:JULES_OUTPUT_PROFILE::var_name}]{\sphinxcrossref{\sphinxcode{\sphinxupquote{var\_name}}}}}.

\sphinxAtStartPar
The second output profile has specified a section of the main run that it will provide output for using {\hyperref[\detokenize{namelists/output.nml:JULES_OUTPUT_PROFILE::output_start}]{\sphinxcrossref{\sphinxcode{\sphinxupquote{output\_start}}}}} and {\hyperref[\detokenize{namelists/output.nml:JULES_OUTPUT_PROFILE::output_end}]{\sphinxcrossref{\sphinxcode{\sphinxupquote{output\_end}}}}} such that exactly a year of data will be output. Since {\hyperref[\detokenize{namelists/output.nml:JULES_OUTPUT_PROFILE::output_period}]{\sphinxcrossref{\sphinxcode{\sphinxupquote{output\_period}}}}} is not specified, it takes its default value, and output will be produced every timestep. The user has not specified the names to use in output files for this profile, so the values from {\hyperref[\detokenize{namelists/output.nml:JULES_OUTPUT_PROFILE::var}]{\sphinxcrossref{\sphinxcode{\sphinxupquote{var}}}}} will be used.

\sphinxstepscope


\section{\sphinxstyleliteralintitle{\sphinxupquote{oasis\_rivers.nml}}}
\label{\detokenize{namelists/oasis_rivers.nml:oasis-rivers-nml}}\label{\detokenize{namelists/oasis_rivers.nml::doc}}
\sphinxAtStartPar
This file contains a single namelists called {\hyperref[\detokenize{namelists/oasis_rivers.nml:namelist-OASIS_RIVERS}]{\sphinxcrossref{\sphinxcode{\sphinxupquote{OASIS\_RIVERS}}}}}, which indicates how the Rivers\sphinxhyphen{}only executable couples via OASIS to other models (currently LFRIC and NEMO). This namelist is only used when running the river standalone program in coupled mode, i.e., compiling with the parameters RIVERS\_ONLY and RIVER\_CPL

\begin{sphinxadmonition}{note}{Note:}
\sphinxAtStartPar
This namelist is only actually used when running the Rivers\sphinxhyphen{}only executable (compilation flag \sphinxtitleref{RIVERS\_ONLY}) in coupled mode (compilation flag \sphinxtitleref{RIVER\_CPL})
\end{sphinxadmonition}


\subsection{\sphinxstyleliteralintitle{\sphinxupquote{OASIS\_RIVERS}} namelist members}
\label{\detokenize{namelists/oasis_rivers.nml:namelist-OASIS_RIVERS}}\label{\detokenize{namelists/oasis_rivers.nml:oasis-rivers-namelist-members}}\index{OASIS\_RIVERS (namelist)@\spxentry{OASIS\_RIVERS}\spxextra{namelist}|spxpagem}\index{np\_receive (in namelist OASIS\_RIVERS)@\spxentry{np\_receive}\spxextra{in namelist OASIS\_RIVERS}|spxpagem}

\begin{fulllineitems}
\phantomsection\label{\detokenize{namelists/oasis_rivers.nml:OASIS_RIVERS::np_receive}}
\pysigstartsignatures
\pysigline{\sphinxcode{\sphinxupquote{OASIS\_RIVERS::}}\sphinxbfcode{\sphinxupquote{np\_receive}}}
\pysigstopsignatures\begin{quote}\begin{description}
\sphinxlineitem{Type}
\sphinxAtStartPar
integer

\sphinxlineitem{Permitted}
\sphinxAtStartPar
2

\sphinxlineitem{Default}
\sphinxAtStartPar
imdi

\end{description}\end{quote}

\sphinxAtStartPar
The number of fields that are received from other models via OASIS coupling.

\end{fulllineitems}

\index{np\_send (in namelist OASIS\_RIVERS)@\spxentry{np\_send}\spxextra{in namelist OASIS\_RIVERS}|spxpagem}

\begin{fulllineitems}
\phantomsection\label{\detokenize{namelists/oasis_rivers.nml:OASIS_RIVERS::np_send}}
\pysigstartsignatures
\pysigline{\sphinxcode{\sphinxupquote{OASIS\_RIVERS::}}\sphinxbfcode{\sphinxupquote{np\_send}}}
\pysigstopsignatures\begin{quote}\begin{description}
\sphinxlineitem{Type}
\sphinxAtStartPar
integer

\sphinxlineitem{Permitted}
\sphinxAtStartPar
0,1

\sphinxlineitem{Default}
\sphinxAtStartPar
imdi

\end{description}\end{quote}

\sphinxAtStartPar
The number of fields that are sent to other models via OASIS coupling.

\end{fulllineitems}

\index{cpl\_freq (in namelist OASIS\_RIVERS)@\spxentry{cpl\_freq}\spxextra{in namelist OASIS\_RIVERS}|spxpagem}

\begin{fulllineitems}
\phantomsection\label{\detokenize{namelists/oasis_rivers.nml:OASIS_RIVERS::cpl_freq}}
\pysigstartsignatures
\pysigline{\sphinxcode{\sphinxupquote{OASIS\_RIVERS::}}\sphinxbfcode{\sphinxupquote{cpl\_freq}}}
\pysigstopsignatures\begin{quote}\begin{description}
\sphinxlineitem{Type}
\sphinxAtStartPar
integer

\sphinxlineitem{Permitted}
\sphinxAtStartPar
1:

\sphinxlineitem{Default}
\sphinxAtStartPar
imdi

\end{description}\end{quote}

\sphinxAtStartPar
The river coupling frequency in seconds.

\end{fulllineitems}


\begin{sphinxadmonition}{note}{Note:}
\sphinxAtStartPar
The river coupling frequency must be a multiple of the river executable time step, and of the time steps of the models to which it is coupled.
\end{sphinxadmonition}
\index{send\_fields (in namelist OASIS\_RIVERS)@\spxentry{send\_fields}\spxextra{in namelist OASIS\_RIVERS}|spxpagem}

\begin{fulllineitems}
\phantomsection\label{\detokenize{namelists/oasis_rivers.nml:OASIS_RIVERS::send_fields}}
\pysigstartsignatures
\pysigline{\sphinxcode{\sphinxupquote{OASIS\_RIVERS::}}\sphinxbfcode{\sphinxupquote{send\_fields}}}
\pysigstopsignatures\begin{quote}\begin{description}
\sphinxlineitem{Type}
\sphinxAtStartPar
character(:)

\sphinxlineitem{Permitted}
\sphinxAtStartPar
‘rflow\_outflow’

\sphinxlineitem{Default}
\sphinxAtStartPar
‘’

\end{description}\end{quote}

\sphinxAtStartPar
List of fields to be sent via coupling from the river executable to other models. Names are case sensitive

\end{fulllineitems}


\begin{sphinxadmonition}{note}{Note:}
\sphinxAtStartPar
The only field that can be sent via coupling is the total river runoff (\sphinxtitleref{rflow\_outflow}).
\end{sphinxadmonition}
\index{receive\_fields (in namelist OASIS\_RIVERS)@\spxentry{receive\_fields}\spxextra{in namelist OASIS\_RIVERS}|spxpagem}

\begin{fulllineitems}
\phantomsection\label{\detokenize{namelists/oasis_rivers.nml:OASIS_RIVERS::receive_fields}}
\pysigstartsignatures
\pysigline{\sphinxcode{\sphinxupquote{OASIS\_RIVERS::}}\sphinxbfcode{\sphinxupquote{receive\_fields}}}
\pysigstopsignatures\begin{quote}\begin{description}
\sphinxlineitem{Type}
\sphinxAtStartPar
character(:)

\sphinxlineitem{Permitted}
\sphinxAtStartPar
‘sub\_surf\_roff\_rp’, ‘surf\_roff\_rp’

\sphinxlineitem{Default}
\sphinxAtStartPar
‘’

\end{description}\end{quote}

\sphinxAtStartPar
List of fields to be received by the river executable via coupling from other models. Names are case sensitive

\end{fulllineitems}


\begin{sphinxadmonition}{note}{Note:}
\sphinxAtStartPar
Coupled receive fields are used to substitute driving data read from file using the namelist {\hyperref[\detokenize{namelists/drive.nml:namelist-JULES_DRIVE}]{\sphinxcrossref{\sphinxcode{\sphinxupquote{JULES\_DRIVE}}}}} by the same fields generated by a driving model running in parallel to the river executable. The only fields that can be received via coupling are the surface runoff (\sphinxtitleref{surf\_roff\_rp}) and the sub\sphinxhyphen{}surface runoff (\sphinxtitleref{sub\_surf\_roff\_rp}).
\end{sphinxadmonition}
\index{riv\_number\_file (in namelist OASIS\_RIVERS)@\spxentry{riv\_number\_file}\spxextra{in namelist OASIS\_RIVERS}|spxpagem}

\begin{fulllineitems}
\phantomsection\label{\detokenize{namelists/oasis_rivers.nml:OASIS_RIVERS::riv_number_file}}
\pysigstartsignatures
\pysigline{\sphinxcode{\sphinxupquote{OASIS\_RIVERS::}}\sphinxbfcode{\sphinxupquote{riv\_number\_file}}}
\pysigstopsignatures\begin{quote}\begin{description}
\sphinxlineitem{Type}
\sphinxAtStartPar
character

\sphinxlineitem{Default}
\sphinxAtStartPar
‘’

\end{description}\end{quote}

\sphinxAtStartPar
Ancillary file containing the river number. This information is necessary when sending via coupling the total runoff ({\hyperref[\detokenize{namelists/oasis_rivers.nml:OASIS_RIVERS::send_fields}]{\sphinxcrossref{\sphinxcode{\sphinxupquote{send\_fields}}}}} = ‘rflow\_outflow’), so that rivers discharge in the right ocean grid point. The ancillary file identifies the river in which the river outflow on each grid point will discharge, so the total runoff for each river is calculated as the sum of the river outflow corresponding to that river.

\end{fulllineitems}



\subsection{Example of coupling request}
\label{\detokenize{namelists/oasis_rivers.nml:example-of-coupling-request}}
\sphinxAtStartPar
In this example, the user has requested receiving the surface and sub\sphinxhyphen{}surface runoffs, and sending the total river runoff via coupling. The coupling exchanges take place every hour.

\begin{sphinxVerbatim}[commandchars=\\\{\}]
\PYG{n+nn}{\PYGZam{}OASIS\PYGZus{}RIVERS}
  \PYG{n+nv}{cpl\PYGZus{}freq} \PYG{o}{=} \PYG{l+m+mi}{3600}\PYG{p}{,}

  \PYG{n+nv}{np\PYGZus{}receive} \PYG{o}{=} \PYG{l+m+mi}{2}\PYG{p}{,}

  \PYG{n+nv}{np\PYGZus{}send} \PYG{o}{=} \PYG{l+m+mi}{1}\PYG{p}{,}

  \PYG{n+nv}{receive\PYGZus{}fields} \PYG{o}{=} \PYG{l+s+s1}{\PYGZsq{}sub\PYGZus{}surf\PYGZus{}roff\PYGZus{}rp\PYGZsq{}}\PYG{p}{,}\PYG{l+s+s1}{\PYGZsq{}surf\PYGZus{}roff\PYGZus{}rp\PYGZsq{}}\PYG{p}{,}

  \PYG{n+nv}{riv\PYGZus{}number\PYGZus{}file} \PYG{o}{=} \PYG{l+s+s1}{\PYGZsq{}\PYGZdl{}RIV\PYGZus{}NUMBER\PYGZus{}ANCILLARY/river\PYGZus{}number\PYGZus{}um.nc\PYGZsq{}}

  \PYG{n+nv}{send\PYGZus{}fields} \PYG{o}{=} \PYG{l+s+s1}{\PYGZsq{}rflow\PYGZus{}outflow\PYGZsq{}}\PYG{p}{,}

\PYG{n+nn}{/}
\end{sphinxVerbatim}

\sphinxstepscope


\chapter{JULES examples}
\label{\detokenize{examples:jules-examples}}\label{\detokenize{examples::doc}}
\sphinxAtStartPar
If you are using a version of JULES since vn5.0, please see the list of standard jobs in the table given \sphinxhref{https://code.metoffice.gov.uk/trac/jules\#ReleaseSchedule}{here} or the
‘Example configurations’ row of the table \sphinxhref{http://jules.jchmr.org/content/configurations}{here}.

\sphinxAtStartPar
For earlier versions, please see chapter 8 of the appropriate \sphinxhref{http://jules-lsm.github.io}{JULES User Guide}.

\sphinxstepscope


\chapter{Aspects of the code}
\label{\detokenize{code/contents:aspects-of-the-code}}\label{\detokenize{code/contents::doc}}
\sphinxstepscope


\section{I/O framework}
\label{\detokenize{code/io:i-o-framework}}\label{\detokenize{code/io::doc}}
\sphinxAtStartPar
JULES version 3.1 saw a complete rewrite of the I/O code to use a more modular and flexible structure. This section attempts to give a brief description of the low\sphinxhyphen{}level I/O framework, and explains how to make some commonly required changes.

\begin{sphinxadmonition}{warning}{Warning:}
\sphinxAtStartPar
This section requires a good knowledge of Fortran.
\end{sphinxadmonition}


\subsection{Overview}
\label{\detokenize{code/io:overview}}
\sphinxAtStartPar
The JULES I/O code is comprised of several ‘layers’ with clearly defined responsibilities that communicate with each other, as shown in the figure {\hyperref[\detokenize{code/io:figure-modular-structure}]{\sphinxcrossref{\DUrole{std,std-ref}{Modular structure of the JULES I/O code}}}} (the relevant Fortran modules for each layer are also given). The blocks in orange are the JULES specific pieces of code \sphinxhyphen{} in theory, the rest of the code could be used with other models if different implementations of these modules were provided.

\begin{figure}[htbp]
\centering
\capstart

\noindent\sphinxincludegraphics{{io_modular_structure}.png}
\caption{Modular structure of the JULES I/O code}\label{\detokenize{code/io:id3}}\label{\detokenize{code/io:figure-modular-structure}}\end{figure}

\sphinxAtStartPar
The core component in the I/O framework is the common file handling API. This layer provides a common interface for different file formats that is then used by the rest of the code. The drivers for ASCII and NetCDF files implement this interface. The interface is based around the concepts of dimensions and variables, much like NetCDF (except that nothing is inferred from metadata \sphinxhyphen{} all information about variables and dimensions must be prescribed), but adds the concept of a “record” to that:
\begin{description}
\sphinxlineitem{Dimension}
\sphinxAtStartPar
A file has one or more dimensions. Each regular dimension has a name and a size.

\sphinxAtStartPar
One dimension is special, and is referred to as the record dimension. It has a name but has no defined size. A typical use of the record dimension is to represent time.

\sphinxlineitem{Variable}
\sphinxAtStartPar
A file has one or more variables. The size of each variable is defined using the dimensions previously defined in the file. Each variable can opt to use the record dimension or not \sphinxhyphen{} if a variable uses the record dimension it must be the last dimension that the variable has.

\sphinxlineitem{Record}
\sphinxAtStartPar
A record is the collection of all variables at a certain value of the record dimension. The figure {\hyperref[\detokenize{code/io:figure-records-in-file}]{\sphinxcrossref{\DUrole{std,std-ref}{Records in a file}}}} gives an example of this:

\begin{figure}[htbp]
\centering
\capstart

\noindent\sphinxincludegraphics{{records_in_file}.png}
\caption{Records in a file}\label{\detokenize{code/io:id4}}\label{\detokenize{code/io:figure-records-in-file}}\end{figure}

\sphinxAtStartPar
In the figure, each variable has dimensions x, y and n, where n is the record dimension. Each green box represents the (2D plane of) values of a variable for a certain value of n. A record is then the collection of all variables at a certain value of n.

\sphinxAtStartPar
A good analogy is the lines in an ASCII file, where each column represents a variable and each line is a record (in fact, this is a generalisation to multiple dimensions of that exact concept).

\end{description}

\sphinxAtStartPar
Files keep track of the record they are currently pointing at (it is the responsibility of the file\sphinxhyphen{}type drivers to do this in the way that best suits the file format they implement). When a file receives a read or write request for a particular variable, the values are read from or written to the current record.

\sphinxAtStartPar
The record abstraction also allows two useful operations \sphinxhyphen{} seek and advance. When a file receives an instruction to seek to a particular record, it sets its internal pointer so that read/write requests access the given record (a use of this within JULES is looping the input files round spin\sphinxhyphen{}up cycles). An advance instruction just moves the internal pointer on to the next record.

\sphinxAtStartPar
The routines in \sphinxcode{\sphinxupquote{file\_mod}} define the interface that each file\sphinxhyphen{}type driver must implement, and are responsible for deciding which driver to defer to. Support for a file format is provided by implementing this interface and declaring the implementation in \sphinxcode{\sphinxupquote{file\_mod}}. This is discussed in further detail in {\hyperref[\detokenize{code/io:implementing-a-new-file-format}]{\sphinxcrossref{\DUrole{std,std-ref}{Implementing a new file format}}}}.

\sphinxAtStartPar
The gridded file API then imposes the concept of reading and writing cubes of gridded data (i.e. x and y dimensions for the grid, plus zero or more ‘levels’ dimensions) on top of the common file handling API. The underlying files may have a 1D or 2D grid (see {\hyperref[\detokenize{input/overview::doc}]{\sphinxcrossref{\DUrole{doc}{Input files for JULES}}}}), and this layer handles the grid dimensions transparently. It is this layer that handles the extraction of a subgrid from a larger grid (see \sphinxcode{\sphinxupquote{file\_gridded\_read\_var}} and \sphinxcode{\sphinxupquote{file\_gridded\_write\_var}}).

\sphinxAtStartPar
The time series file API builds on the gridded file API by explicitly presenting the record dimension as a time dimension. It provides an interface that allows users to treat multiple files (e.g. monthly files, yearly files) as if they were a single file (i.e. seek and advance will automatically open and close files if required).

\sphinxAtStartPar
The input and output layers interact with the model via an interface provided by \sphinxcode{\sphinxupquote{model\_interface\_mod}}. \sphinxcode{\sphinxupquote{model\_interface\_mod}} allows the input and output layers to read values from and write values to the internal model variables. This is discussed in more detail in {\hyperref[\detokenize{code/io:implementing-new-variables-for-input-and-output}]{\sphinxcrossref{\DUrole{std,std-ref}{Implementing new variables for input and output}}}}. The input and output layers use the time series file API to read from and write to file.

\sphinxAtStartPar
This should provide a reasonable introduction to the JULES I/O framework, but looking at the code is the best way to learn about it.


\subsection{Implementing new variables for input and output}
\label{\detokenize{code/io:implementing-new-variables-for-input-and-output}}\label{\detokenize{code/io:id1}}
\sphinxAtStartPar
The only I/O code that needs to be modified to add new variables for input and output is in \sphinxcode{\sphinxupquote{model\_interface\_mod}} (the routines in \sphinxcode{\sphinxupquote{src/io/model\_interface}}). All interaction between the I/O code and the model happens in this module (apart from reading and writing dump files).

\sphinxAtStartPar
Before adding any code to \sphinxcode{\sphinxupquote{model\_interface\_mod}}, the variable the user wishes to make available for input and/or output must be accessible to \sphinxcode{\sphinxupquote{model\_interface\_mod}}. This is usually accomplished by placing the variable in a module and importing the module into \sphinxcode{\sphinxupquote{model\_interface\_mod}} where required, e.g.:

\begin{sphinxVerbatim}[commandchars=\\\{\}]
\PYG{c}{! Declare the variable in a module}
\PYG{k}{MODULE }\PYG{n}{my\PYGZus{}module}

\PYG{+w}{  }\PYG{k+kt}{REAL}\PYG{p}{,}\PYG{+w}{ }\PYG{k}{ALLOCATABLE}\PYG{+w}{ }\PYG{k+kd}{::}\PYG{+w}{ }\PYG{n}{my\PYGZus{}var}\PYG{p}{(}\PYG{p}{:}\PYG{p}{)}

\PYG{+w}{  }\PYG{c}{! ...}
\PYG{k}{END }\PYG{k}{MODULE }\PYG{n}{my\PYGZus{}module}


\PYG{c}{! ... Later, in model\PYGZus{}interface\PYGZus{}mod}
\PYG{k}{USE }\PYG{n}{my\PYGZus{}module}\PYG{p}{,}\PYG{+w}{ }\PYG{k}{ONLY}\PYG{+w}{ }\PYG{p}{:}\PYG{+w}{ }\PYG{n}{my\PYGZus{}var}
\end{sphinxVerbatim}

\sphinxAtStartPar
\sphinxcode{\sphinxupquote{model\_interface\_mod}} contains several routines:
\begin{itemize}
\item {} 
\sphinxAtStartPar
Two routines that populate and extract data from the relevant model variables. These are \sphinxcode{\sphinxupquote{populate\_var}} and \sphinxcode{\sphinxupquote{extract\_var}} respectively.

\item {} 
\sphinxAtStartPar
Routines that provide various pieces of information (e.g. string identifiers, number and size of ‘levels’ dimensions) about the available variables to the input and output layers. Internally, a metadata array that contains information about the available variables is used to implement these ‘information providing’ routines.

\end{itemize}

\sphinxAtStartPar
In most cases, the following edits will be sufficient to add a variable for input and/or output:
\begin{description}
\sphinxlineitem{\sphinxcode{\sphinxupquote{model\_interface\_mod.F90}}}
\begin{sphinxadmonition}{note}{Note:}
\sphinxAtStartPar
Required for both input and output variables.
\end{sphinxadmonition}

\sphinxAtStartPar
Increment the constant \sphinxcode{\sphinxupquote{N\_VARS}}. This \sphinxcode{\sphinxupquote{PARAMETER}} indicates how many elements are in the metadata array. If you forget to do this, the module will fail to compile.

\sphinxlineitem{\sphinxcode{\sphinxupquote{populate\_var.inc}}}
\begin{sphinxadmonition}{note}{Note:}
\sphinxAtStartPar
Required for input variables only.
\end{sphinxadmonition}

\sphinxAtStartPar
\sphinxcode{\sphinxupquote{populate\_var}} takes a variable identifier and a cube of data on the full model grid, and populates the associated model variable using that data. This is done using a \sphinxcode{\sphinxupquote{SELECT}} statement, to which a case must be added for the new variable.

\sphinxlineitem{\sphinxcode{\sphinxupquote{extract\_var.inc}}}
\begin{sphinxadmonition}{note}{Note:}
\sphinxAtStartPar
Required for output variables only.
\end{sphinxadmonition}

\sphinxAtStartPar
\sphinxcode{\sphinxupquote{extract\_var}} takes a variable identifier, extracts the values from the associated model variable, and returns those values as a cube of data on the full model grid. This is done using a \sphinxcode{\sphinxupquote{SELECT}} statement, to which a case must be added for the new variable.

\sphinxlineitem{\sphinxcode{\sphinxupquote{variable\_metadata.inc}}}
\begin{sphinxadmonition}{note}{Note:}
\sphinxAtStartPar
Required for both input and output variables.
\end{sphinxadmonition}

\sphinxAtStartPar
This file contains the \sphinxcode{\sphinxupquote{DATA}} definition for the variable metadata array. The metadata array contains objects of the derived type \sphinxcode{\sphinxupquote{var\_metadata}}, which is defined in \sphinxcode{\sphinxupquote{model\_interface\_mod.F90}}. A typical entry in this array will look something like:

\begin{sphinxVerbatim}[commandchars=\\\{\}]
\PYG{c}{!\PYGZhy{}\PYGZhy{}\PYGZhy{}\PYGZhy{}\PYGZhy{}\PYGZhy{}\PYGZhy{}\PYGZhy{}\PYGZhy{}\PYGZhy{}\PYGZhy{}\PYGZhy{}\PYGZhy{}\PYGZhy{}\PYGZhy{}\PYGZhy{}\PYGZhy{}\PYGZhy{}\PYGZhy{}\PYGZhy{}\PYGZhy{}\PYGZhy{}\PYGZhy{}\PYGZhy{}\PYGZhy{}\PYGZhy{}\PYGZhy{}\PYGZhy{}\PYGZhy{}\PYGZhy{}\PYGZhy{}\PYGZhy{}\PYGZhy{}\PYGZhy{}\PYGZhy{}\PYGZhy{}\PYGZhy{}\PYGZhy{}\PYGZhy{}\PYGZhy{}\PYGZhy{}\PYGZhy{}\PYGZhy{}\PYGZhy{}\PYGZhy{}\PYGZhy{}\PYGZhy{}\PYGZhy{}\PYGZhy{}\PYGZhy{}\PYGZhy{}\PYGZhy{}\PYGZhy{}\PYGZhy{}\PYGZhy{}\PYGZhy{}\PYGZhy{}\PYGZhy{}\PYGZhy{}\PYGZhy{}\PYGZhy{}\PYGZhy{}\PYGZhy{}\PYGZhy{}\PYGZhy{}\PYGZhy{}\PYGZhy{}\PYGZhy{}\PYGZhy{}\PYGZhy{}\PYGZhy{}\PYGZhy{}\PYGZhy{}\PYGZhy{}\PYGZhy{}\PYGZhy{}\PYGZhy{}}
\PYG{c}{! Metadata for latitude}
\PYG{c}{!\PYGZhy{}\PYGZhy{}\PYGZhy{}\PYGZhy{}\PYGZhy{}\PYGZhy{}\PYGZhy{}\PYGZhy{}\PYGZhy{}\PYGZhy{}\PYGZhy{}\PYGZhy{}\PYGZhy{}\PYGZhy{}\PYGZhy{}\PYGZhy{}\PYGZhy{}\PYGZhy{}\PYGZhy{}\PYGZhy{}\PYGZhy{}\PYGZhy{}\PYGZhy{}\PYGZhy{}\PYGZhy{}\PYGZhy{}\PYGZhy{}\PYGZhy{}\PYGZhy{}\PYGZhy{}\PYGZhy{}\PYGZhy{}\PYGZhy{}\PYGZhy{}\PYGZhy{}\PYGZhy{}\PYGZhy{}\PYGZhy{}\PYGZhy{}\PYGZhy{}\PYGZhy{}\PYGZhy{}\PYGZhy{}\PYGZhy{}\PYGZhy{}\PYGZhy{}\PYGZhy{}\PYGZhy{}\PYGZhy{}\PYGZhy{}\PYGZhy{}\PYGZhy{}\PYGZhy{}\PYGZhy{}\PYGZhy{}\PYGZhy{}\PYGZhy{}\PYGZhy{}\PYGZhy{}\PYGZhy{}\PYGZhy{}\PYGZhy{}\PYGZhy{}\PYGZhy{}\PYGZhy{}\PYGZhy{}\PYGZhy{}\PYGZhy{}\PYGZhy{}\PYGZhy{}\PYGZhy{}\PYGZhy{}\PYGZhy{}\PYGZhy{}\PYGZhy{}\PYGZhy{}\PYGZhy{}}
\PYG{k}{DATA }\PYG{n}{metadata}\PYG{p}{(}\PYG{l+m+mi}{1}\PYG{p}{)}\PYG{+w}{ }\PYG{o}{/}\PYG{+w}{ }\PYG{n}{var\PYGZus{}metadata}\PYG{p}{(}\PYG{+w}{                                              }\PYG{p}{\PYGZam{}}
\PYG{c}{! String identifier}
\PYG{+w}{    }\PYG{l+s+s1}{\PYGZsq{}latitude\PYGZsq{}}\PYG{p}{,}\PYG{+w}{                                                               }\PYG{p}{\PYGZam{}}
\PYG{c}{! Variable type}
\PYG{+w}{    }\PYG{n}{VAR\PYGZus{}TYPE\PYGZus{}SURFACE}\PYG{p}{,}\PYG{+w}{                                                         }\PYG{p}{\PYGZam{}}
\PYG{c}{! Long name}
\PYG{+w}{    }\PYG{l+s+s2}{\PYGZdq{}Gridbox latitude\PYGZdq{}}\PYG{p}{,}\PYG{+w}{                                                       }\PYG{p}{\PYGZam{}}
\PYG{c}{! Units}
\PYG{+w}{    }\PYG{l+s+s2}{\PYGZdq{}degrees\PYGZdq{}}\PYG{+w}{                                                                 }\PYG{p}{\PYGZam{}}
\PYG{+w}{  }\PYG{p}{)}\PYG{+w}{ }\PYG{o}{/}
\end{sphinxVerbatim}

\sphinxAtStartPar
This allows us to define all the static information about a variable in one place:
\begin{description}
\sphinxlineitem{String identifier}
\sphinxAtStartPar
This is the name used to identify the variable in namelists (as seen elsewhere in the User Guide)

\sphinxlineitem{Variable type}
\sphinxAtStartPar
This indicates the number and size of the ‘levels’ dimensions for the variable. For a full list of types see the file \sphinxcode{\sphinxupquote{get\_var\_levs\_dims.inc}}; some of the available types are:


\begin{savenotes}\sphinxattablestart
\centering
\begin{tabulary}{\linewidth}[t]{|T|T|}
\hline
\sphinxstyletheadfamily 
\sphinxAtStartPar
Type
&\sphinxstyletheadfamily 
\sphinxAtStartPar
Number and size of ‘levels’ dimension(s)
\\
\hline
\sphinxAtStartPar
\sphinxcode{\sphinxupquote{VAR\_TYPE\_SURFACE}}
&
\sphinxAtStartPar
No levels dimension
\\
\hline
\sphinxAtStartPar
\sphinxcode{\sphinxupquote{VAR\_TYPE\_PFT}}
&
\sphinxAtStartPar
Single levels dimension of size {\hyperref[\detokenize{namelists/jules_surface_types.nml:JULES_SURFACE_TYPES::npft}]{\sphinxcrossref{\sphinxcode{\sphinxupquote{npft}}}}}
\\
\hline
\sphinxAtStartPar
\sphinxcode{\sphinxupquote{VAR\_TYPE\_NVG}}
&
\sphinxAtStartPar
Single levels dimension of size {\hyperref[\detokenize{namelists/jules_surface_types.nml:JULES_SURFACE_TYPES::nnvg}]{\sphinxcrossref{\sphinxcode{\sphinxupquote{nnvg}}}}}
\\
\hline
\sphinxAtStartPar
\sphinxcode{\sphinxupquote{VAR\_TYPE\_TYPE}}
&
\sphinxAtStartPar
Single levels dimension of size \sphinxcode{\sphinxupquote{ntype}} ({\hyperref[\detokenize{namelists/jules_surface_types.nml:JULES_SURFACE_TYPES::npft}]{\sphinxcrossref{\sphinxcode{\sphinxupquote{npft}}}}} +
{\hyperref[\detokenize{namelists/jules_surface_types.nml:JULES_SURFACE_TYPES::nnvg}]{\sphinxcrossref{\sphinxcode{\sphinxupquote{nnvg}}}}})
\\
\hline
\sphinxAtStartPar
\sphinxcode{\sphinxupquote{VAR\_TYPE\_TILE}}
&
\sphinxAtStartPar
Single levels dimension of size \sphinxcode{\sphinxupquote{nsurft}} (1 if {\hyperref[\detokenize{namelists/jules_surface.nml:JULES_SURFACE::l_aggregate}]{\sphinxcrossref{\sphinxcode{\sphinxupquote{l\_aggregate}}}}} = TRUE,
\sphinxcode{\sphinxupquote{ntype}} otherwise)
\\
\hline
\sphinxAtStartPar
\sphinxcode{\sphinxupquote{VAR\_TYPE\_SOIL}}
&
\sphinxAtStartPar
Single levels dimension of size {\hyperref[\detokenize{namelists/jules_soil.nml:JULES_SOIL::sm_levels}]{\sphinxcrossref{\sphinxcode{\sphinxupquote{sm\_levels}}}}}
\\
\hline
\sphinxAtStartPar
\sphinxcode{\sphinxupquote{VAR\_TYPE\_SCPOOL}}
&
\sphinxAtStartPar
Single levels dimension of size \sphinxcode{\sphinxupquote{dim\_cs1}} (number of soil carbon pools, i.e. 4 if
{\hyperref[\detokenize{namelists/jules_vegetation.nml:JULES_VEGETATION::l_triffid}]{\sphinxcrossref{\sphinxcode{\sphinxupquote{l\_triffid}}}}} = TRUE, 1 otherwise)
\\
\hline
\sphinxAtStartPar
\sphinxcode{\sphinxupquote{VAR\_TYPE\_SNOW}}
&
\sphinxAtStartPar
Two levels dimensions: the first of size \sphinxcode{\sphinxupquote{nsurft}} and the second of size
{\hyperref[\detokenize{namelists/jules_snow.nml:JULES_SNOW::nsmax}]{\sphinxcrossref{\sphinxcode{\sphinxupquote{nsmax}}}}}
\\
\hline
\end{tabulary}
\par
\sphinxattableend\end{savenotes}

\sphinxAtStartPar
Adding a new type is a relatively simple procedure:
\begin{enumerate}
\sphinxsetlistlabels{\arabic}{enumi}{enumii}{}{.}%
\item {} 
\sphinxAtStartPar
A new \sphinxcode{\sphinxupquote{PARAMETER}} must be added for the type in \sphinxcode{\sphinxupquote{model\_interface\_mod.F90}}

\item {} 
\sphinxAtStartPar
A new \sphinxcode{\sphinxupquote{CASE}} must be added to the \sphinxcode{\sphinxupquote{SELECT}} statement in \sphinxcode{\sphinxupquote{get\_var\_levs\_dims.inc}} that correctly returns the number, names and sizes of the levels dimensions.

\end{enumerate}

\sphinxlineitem{Long name}
\sphinxAtStartPar
This is the name used in the \sphinxcode{\sphinxupquote{long\_name}} attribute for the variable in output files.

\sphinxlineitem{Units}
\sphinxAtStartPar
This is the units given in the \sphinxcode{\sphinxupquote{units}} attribute for the variable in output files.

\end{description}

\end{description}

\sphinxAtStartPar
\sphinxcode{\sphinxupquote{map\_from\_land}} and \sphinxcode{\sphinxupquote{map\_to\_land}} are provided as utilities for use with variables that are defined on land points only. \sphinxcode{\sphinxupquote{tiles\_to\_gbm}} is used to provide gridbox mean diagnostics for model variables that have one value per surface tile.

\sphinxAtStartPar
As always, the best way to go about implementing new variables for input and output is to follow the examples that are already there.


\subsection{Implementing a new file format}
\label{\detokenize{code/io:implementing-a-new-file-format}}\label{\detokenize{code/io:id2}}
\sphinxAtStartPar
To understand how to implement a new file format, it first helps to understand how the common file handling layer works under the hood.

\sphinxAtStartPar
Each of the routines in \sphinxcode{\sphinxupquote{file\_mod}} (see files in \sphinxcode{\sphinxupquote{src/io/file\_handling/core}}) takes a \sphinxcode{\sphinxupquote{file\_handle}} as its first argument. The \sphinxcode{\sphinxupquote{file\_handle}} is a Fortran derived type that contains a flag indicating the format of the file it represents, and each of the routines in \sphinxcode{\sphinxupquote{file\_mod}} contains a \sphinxcode{\sphinxupquote{SELECT}} statement that defers to the correct implementation of the routine based on that flag.

\sphinxAtStartPar
\sphinxcode{\sphinxupquote{file\_handles}} are created in \sphinxcode{\sphinxupquote{file\_open}}. Each file format implementation defines a list of recognised file extensions, and the appropriate file opening routine is deferred to by comparing the extension of the given file name to the recognised extensions for each file format.

\sphinxAtStartPar
To implement a new file format, an implementation of each of the routines in \sphinxcode{\sphinxupquote{file\_mod}} must first be provided (the implementations for ASCII and NetCDF formats should be used as a reference). A new \sphinxcode{\sphinxupquote{CASE}} deferring to the new implementation should then be added to the \sphinxcode{\sphinxupquote{SELECT}} statement in each of the routines in \sphinxcode{\sphinxupquote{file\_mod}}. The recognised file extensions for the new format should also be added to the checks in \sphinxcode{\sphinxupquote{file\_open}} to allow the new the file opening routine to be called.

\sphinxAtStartPar
Implementations of these routines for ASCII and NetCDF file formats are given in \sphinxcode{\sphinxupquote{driver\_ascii}} (see \sphinxcode{\sphinxupquote{src/io/file\_handling/core/drivers/ascii}}) and \sphinxcode{\sphinxupquote{driver\_ncdf}} (see \sphinxcode{\sphinxupquote{src/io/file\_handling/core/drivers/ncdf}}) respectively. These should be used as examples of how to implement a file format.

\sphinxAtStartPar
These two file formats suffer from opposite problems when implementing the concepts of dimensions, variables and records. For NetCDF, the concepts of dimensions and variables already exist, but the idea of a record has to be imposed. For ASCII, the concept of a record is a natural fit (think lines in a file), but the concepts of dimensions and variables have to be imposed. Between them, these implementations should provide sufficient examples of how to implement a new file format.

\sphinxstepscope


\section{Known limitations of the code}
\label{\detokenize{code/known-limitations:known-limitations-of-the-code}}\label{\detokenize{code/known-limitations::doc}}

\subsection{Limit to longest possible run}
\label{\detokenize{code/known-limitations:limit-to-longest-possible-run}}
\sphinxAtStartPar
The longest possible run that can be attempted with JULES is approximately 100 years. A longer run should be split into smaller sections, with each later section starting from the final dump of the previous section. This restriction on run length arises because some of the time variables can become too large for the declared type of variable meaning that calculations return incorrect results and the program will probably crash. The size of each variable is in part affected by the compiler used, but a maximum run length of \textasciitilde{}100 years appears to be a common case for 32\sphinxhyphen{}bit machines. Note that JULES uses the compiler’s default KIND for each type of variable. Changes to the KIND of any variable would have to be propagated through the code.


\subsection{Spin\sphinxhyphen{}up over short periods}
\label{\detokenize{code/known-limitations:spin-up-over-short-periods}}
\sphinxAtStartPar
The current code has not been tested with a spin\sphinxhyphen{}up cycle that is short in comparison to the period of any input data (e.g. a spin\sphinxhyphen{}up cycle of 1 day using prescribed vegetation data with a period of 10 days). The code will likely run but the evolution of the vegetation data may not be what was intended. However, it is unlikely that a user would want to try such a run.


\subsection{Upgrade macros for the \sphinxstyleliteralintitle{\sphinxupquote{JULES\_VEGETATION\_PROPS}} namelist}
\label{\detokenize{code/known-limitations:upgrade-macros-for-the-jules-vegetation-props-namelist}}
\sphinxAtStartPar
The {\hyperref[\detokenize{namelists/ancillaries.nml:namelist-JULES_VEGETATION_PROPS}]{\sphinxcrossref{\sphinxcode{\sphinxupquote{JULES\_VEGETATION\_PROPS}}}}} namelist was added to the JULES source at vn5.7, but the upgrade macro to add this namelist to JULES Rose apps was not added until vn6.1.  This means that when \sphinxcode{\sphinxupquote{rose app\sphinxhyphen{}upgrade}} is used to upgrade a JULES app to versions vn5.7 through vn6.0, the {\hyperref[\detokenize{namelists/ancillaries.nml:namelist-JULES_VEGETATION_PROPS}]{\sphinxcrossref{\sphinxcode{\sphinxupquote{JULES\_VEGETATION\_PROPS}}}}} namelist will neither be added to the app and nor be described by the corresponding \sphinxcode{\sphinxupquote{rose\sphinxhyphen{}meta}}.  This namelist is needed only if {\hyperref[\detokenize{namelists/jules_vegetation.nml:JULES_VEGETATION::photo_acclim_model}]{\sphinxcrossref{\sphinxcode{\sphinxupquote{photo\_acclim\_model}}}}} is set to \sphinxcode{\sphinxupquote{1}}, in which case the user must manually edit their JULES app and \sphinxcode{\sphinxupquote{rose\sphinxhyphen{}meta}} to include the relevant information.  For this reason, we recommend using this science option only with JULES vn6.1 or later.

\sphinxstepscope


\chapter{JULES Output variables}
\label{\detokenize{output-variables:jules-output-variables}}\label{\detokenize{output-variables:output-variables-section}}\label{\detokenize{output-variables::doc}}
\sphinxAtStartPar
Variables that are available for output from JULES are listed in this section, separated according to the broad area of science. If a variable cannot be found, users should also check in related sections.

\begin{sphinxadmonition}{note}{Note:}
\sphinxAtStartPar
Most variables are output on the full model grid (even for variables defined on land points only).

\sphinxAtStartPar
Most river variables are output on the river routing model grid, which is a 1D grid defined on the valid river routing points only. These are indicated by names ending in \sphinxcode{\sphinxupquote{\_rp}} and have a single spatial dimension of size \sphinxcode{\sphinxupquote{np\_rivers}}. A regridded version of some river variables can also be output on the full model grid.”

\sphinxAtStartPar
Any points on the grid for which a variable is not defined with be filled with a missing data value.

\sphinxAtStartPar
Any variables also available on soil tiles are indicated next to their non\sphinxhyphen{}soil\sphinxhyphen{}tiled analogues, e.g. \sphinxcode{\sphinxupquote{var{[}\_soilt{]}}}, in which case they have an additional dimension of size \sphinxcode{\sphinxupquote{nsoilt}}.

\sphinxAtStartPar
All variables include a land point dimension, unless specified otherwise.

\sphinxAtStartPar
The sizes of dimensions are indicated in the tables below using links to other sections of this documentation, wherever possible. Other sizes are discussed in the table below.
\end{sphinxadmonition}


\begin{savenotes}\sphinxattablestart
\centering
\begin{tabulary}{\linewidth}[t]{|p{2.5cm}|p{12.0cm}|}
\hline
\sphinxstyletheadfamily 
\sphinxAtStartPar
Name
&\sphinxstyletheadfamily 
\sphinxAtStartPar
Description
\\
\hline
\sphinxAtStartPar
\sphinxcode{\sphinxupquote{ch4layer}}
&
\sphinxAtStartPar
Number of soil methane layers.

\sphinxAtStartPar
Equals {\hyperref[\detokenize{namelists/jules_soil.nml:JULES_SOIL::sm_levels}]{\sphinxcrossref{\sphinxcode{\sphinxupquote{sm\_levels}}}}} if {\hyperref[\detokenize{namelists/jules_soil_biogeochem.nml:JULES_SOIL_BIOGEOCHEM::l_ch4_tlayered}]{\sphinxcrossref{\sphinxcode{\sphinxupquote{l\_ch4\_tlayered}}}}} = TRUE,
otherwise = 1.
\\
\hline
\sphinxAtStartPar
\sphinxcode{\sphinxupquote{cslayer}}
&
\sphinxAtStartPar
Number of layers for soil carbon and nitrogen.

\sphinxAtStartPar
With the single\sphinxhyphen{}pool soil model ({\hyperref[\detokenize{namelists/jules_soil_biogeochem.nml:JULES_SOIL_BIOGEOCHEM::soil_bgc_model}]{\sphinxcrossref{\sphinxcode{\sphinxupquote{soil\_bgc\_model}}}}} = 1),
\sphinxcode{\sphinxupquote{cslayer}} = 1.

\sphinxAtStartPar
With 4\sphinxhyphen{}pool model ({\hyperref[\detokenize{namelists/jules_soil_biogeochem.nml:JULES_SOIL_BIOGEOCHEM::soil_bgc_model}]{\sphinxcrossref{\sphinxcode{\sphinxupquote{soil\_bgc\_model}}}}} = 2),
if {\hyperref[\detokenize{namelists/jules_soil_biogeochem.nml:JULES_SOIL_BIOGEOCHEM::l_layeredc}]{\sphinxcrossref{\sphinxcode{\sphinxupquote{l\_layeredc}}}}} = TRUE \sphinxcode{\sphinxupquote{cslayer}} = {\hyperref[\detokenize{namelists/jules_soil.nml:JULES_SOIL::sm_levels}]{\sphinxcrossref{\sphinxcode{\sphinxupquote{sm\_levels}}}}},
otherwise \sphinxcode{\sphinxupquote{cslayer}} = 1.

\sphinxAtStartPar
With the ECOSSE model ({\hyperref[\detokenize{namelists/jules_soil_biogeochem.nml:JULES_SOIL_BIOGEOCHEM::soil_bgc_model}]{\sphinxcrossref{\sphinxcode{\sphinxupquote{soil\_bgc\_model}}}}} = 3),
\sphinxcode{\sphinxupquote{cslayer}} = {\hyperref[\detokenize{namelists/jules_soil_ecosse.nml:JULES_SOIL_ECOSSE::dim_cslayer}]{\sphinxcrossref{\sphinxcode{\sphinxupquote{dim\_cslayer}}}}}.
\\
\hline
\sphinxAtStartPar
\sphinxcode{\sphinxupquote{cspool}}
&
\sphinxAtStartPar
Number of soil carbon pools.

\sphinxAtStartPar
=1 with the single\sphinxhyphen{}pool soil model ({\hyperref[\detokenize{namelists/jules_soil_biogeochem.nml:JULES_SOIL_BIOGEOCHEM::soil_bgc_model}]{\sphinxcrossref{\sphinxcode{\sphinxupquote{soil\_bgc\_model}}}}} = 1).

\sphinxAtStartPar
=4 with the 4\sphinxhyphen{}pool or ECOSSE models ({\hyperref[\detokenize{namelists/jules_soil_biogeochem.nml:JULES_SOIL_BIOGEOCHEM::soil_bgc_model}]{\sphinxcrossref{\sphinxcode{\sphinxupquote{soil\_bgc\_model}}}}} = 2 or 3).
\\
\hline
\sphinxAtStartPar
\sphinxcode{\sphinxupquote{land+sea}}
&
\sphinxAtStartPar
Variable is available on all points (land and sea).
\\
\hline
\sphinxAtStartPar
\sphinxcode{\sphinxupquote{ncpft}}
&
\sphinxAtStartPar
Number of crop plant functional types \sphinxhyphen{} see {\hyperref[\detokenize{namelists/jules_surface_types.nml:JULES_SURFACE_TYPES::ncpft}]{\sphinxcrossref{\sphinxcode{\sphinxupquote{ncpft}}}}}.
\\
\hline
\sphinxAtStartPar
\sphinxcode{\sphinxupquote{npft}}
&
\sphinxAtStartPar
Number of plant functional types \sphinxhyphen{} see {\hyperref[\detokenize{namelists/jules_surface_types.nml:JULES_SURFACE_TYPES::npft}]{\sphinxcrossref{\sphinxcode{\sphinxupquote{npft}}}}}.
\\
\hline
\sphinxAtStartPar
\sphinxcode{\sphinxupquote{ns\_deep}}
&
\sphinxAtStartPar
The number of levels in the thermal\sphinxhyphen{}only bedrock \sphinxhyphen{} see {\hyperref[\detokenize{namelists/jules_soil.nml:JULES_SOIL::ns_deep}]{\sphinxcrossref{\sphinxcode{\sphinxupquote{ns\_deep}}}}}.
\\
\hline
\sphinxAtStartPar
\sphinxcode{\sphinxupquote{nsmax}}
&
\sphinxAtStartPar
Maximum\sphinxhyphen{}allowed number of snow layers \sphinxhyphen{} see {\hyperref[\detokenize{namelists/jules_snow.nml:JULES_SNOW::nsmax}]{\sphinxcrossref{\sphinxcode{\sphinxupquote{nsmax}}}}}.
\\
\hline
\sphinxAtStartPar
\sphinxcode{\sphinxupquote{nsoilt}}
&
\sphinxAtStartPar
Number of soil tiles.
\sphinxcode{\sphinxupquote{nsurft}} if {\hyperref[\detokenize{namelists/jules_soil.nml:JULES_SOIL::l_tile_soil}]{\sphinxcrossref{\sphinxcode{\sphinxupquote{l\_tile\_soil}}}}} = TRUE, otherwise 1.
\\
\hline
\sphinxAtStartPar
\sphinxcode{\sphinxupquote{nsurft}}
&
\sphinxAtStartPar
Number of surface tiles.
1 if {\hyperref[\detokenize{namelists/jules_surface.nml:JULES_SURFACE::l_aggregate}]{\sphinxcrossref{\sphinxcode{\sphinxupquote{l\_aggregate}}}}} = TRUE, otherwise \sphinxcode{\sphinxupquote{ntype}}.
\\
\hline
\sphinxAtStartPar
\sphinxcode{\sphinxupquote{ntype}}
&
\sphinxAtStartPar
Number of surface types,
= {\hyperref[\detokenize{namelists/jules_surface_types.nml:JULES_SURFACE_TYPES::npft}]{\sphinxcrossref{\sphinxcode{\sphinxupquote{npft}}}}} + {\hyperref[\detokenize{namelists/jules_surface_types.nml:JULES_SURFACE_TYPES::nnvg}]{\sphinxcrossref{\sphinxcode{\sphinxupquote{nnvg}}}}}
\\
\hline
\sphinxAtStartPar
\sphinxcode{\sphinxupquote{sm\_levels}}
&
\sphinxAtStartPar
Number of soil layers (for soil moisture) \sphinxhyphen{} see {\hyperref[\detokenize{namelists/jules_soil.nml:JULES_SOIL::sm_levels}]{\sphinxcrossref{\sphinxcode{\sphinxupquote{sm\_levels}}}}}.
\\
\hline
\end{tabulary}
\par
\sphinxattableend\end{savenotes}


\section{Meteorology}
\label{\detokenize{output-variables:meteorology}}
\sphinxAtStartPar
Unlesss stated otherwise these variables have values at both land and sea points.


\begin{savenotes}\sphinxattablestart
\centering
\begin{tabulary}{\linewidth}[t]{|p{2.5cm}|p{10.8cm}|p{2.2cm}|}
\hline
\sphinxstyletheadfamily 
\sphinxAtStartPar
Name
&\sphinxstyletheadfamily 
\sphinxAtStartPar
Description
&\sphinxstyletheadfamily 
\sphinxAtStartPar
Dimensions
\\
\hline
\sphinxAtStartPar
\sphinxcode{\sphinxupquote{precip}}
&
\sphinxAtStartPar
Gridbox precipitation rate (kg m$^{\text{\sphinxhyphen{}2}}$ s$^{\text{\sphinxhyphen{}1}}$).
&\\
\hline
\sphinxAtStartPar
\sphinxcode{\sphinxupquote{rainfall}}
&
\sphinxAtStartPar
Gridbox rainfall rate (kg m$^{\text{\sphinxhyphen{}2}}$ s$^{\text{\sphinxhyphen{}1}}$).
&\\
\hline
\sphinxAtStartPar
\sphinxcode{\sphinxupquote{snowfall}}
&
\sphinxAtStartPar
Gridbox snowfall rate (kg m$^{\text{\sphinxhyphen{}2}}$ s$^{\text{\sphinxhyphen{}1}}$).
&\\
\hline
\sphinxAtStartPar
\sphinxcode{\sphinxupquote{con\_rain}}
&
\sphinxAtStartPar
Gridbox convective rainfall (kg m$^{\text{\sphinxhyphen{}2}}$ s$^{\text{\sphinxhyphen{}1}}$).
&\\
\hline
\sphinxAtStartPar
\sphinxcode{\sphinxupquote{con\_snow}}
&
\sphinxAtStartPar
Gridbox convective snowfall (kg m$^{\text{\sphinxhyphen{}2}}$ s$^{\text{\sphinxhyphen{}1}}$).
&\\
\hline
\sphinxAtStartPar
\sphinxcode{\sphinxupquote{ls\_rain}}
&
\sphinxAtStartPar
Gridbox large\sphinxhyphen{}scale rainfall (kg m$^{\text{\sphinxhyphen{}2}}$ s$^{\text{\sphinxhyphen{}1}}$).
&\\
\hline
\sphinxAtStartPar
\sphinxcode{\sphinxupquote{ls\_snow}}
&
\sphinxAtStartPar
Gridbox large\sphinxhyphen{}scale snowfall (kg m$^{\text{\sphinxhyphen{}2}}$ s$^{\text{\sphinxhyphen{}1}}$).
&\\
\hline
\sphinxAtStartPar
\sphinxcode{\sphinxupquote{pstar}}
&
\sphinxAtStartPar
Gridbox surface pressure (Pa).
&\\
\hline
\sphinxAtStartPar
\sphinxcode{\sphinxupquote{q1p5m\_gb}}
&
\sphinxAtStartPar
Gridbox specific humidity at 1.5m height (kg kg$^{\text{\sphinxhyphen{}1}}$).
&\\
\hline
\sphinxAtStartPar
\sphinxcode{\sphinxupquote{qw1}}
&
\sphinxAtStartPar
Gridbox specific humidity (total water content) (kg kg$^{\text{\sphinxhyphen{}1}}$).
&\\
\hline
\sphinxAtStartPar
\sphinxcode{\sphinxupquote{q1p5m}}
&
\sphinxAtStartPar
Tile specific humidity at 1.5m over land tiles (kg kg$^{\text{\sphinxhyphen{}1}}$).
&
\sphinxAtStartPar
land,nsurft
\\
\hline
\sphinxAtStartPar
\sphinxcode{\sphinxupquote{lw\_down}}
&
\sphinxAtStartPar
Gridbox surface downward LW radiation (W m$^{\text{\sphinxhyphen{}2}}$).
&\\
\hline
\sphinxAtStartPar
\sphinxcode{\sphinxupquote{sw\_down}}
&
\sphinxAtStartPar
Gridbox surface downward SW radiation (W m$^{\text{\sphinxhyphen{}2}}$).
&\\
\hline
\sphinxAtStartPar
\sphinxcode{\sphinxupquote{t1p5m\_gb}}
&
\sphinxAtStartPar
Gridbox temperature at 1.5m height (K).
&\\
\hline
\sphinxAtStartPar
\sphinxcode{\sphinxupquote{t1p5m}}
&
\sphinxAtStartPar
Tile temperature at 1.5m over land tiles (K).
&
\sphinxAtStartPar
land,nsurft
\\
\hline
\sphinxAtStartPar
\sphinxcode{\sphinxupquote{tl1}}
&
\sphinxAtStartPar
Gridbox ice/liquid water temperature (K).
&\\
\hline
\sphinxAtStartPar
\sphinxcode{\sphinxupquote{u1}}
&
\sphinxAtStartPar
Gridbox westerly wind component (m s$^{\text{\sphinxhyphen{}1}}$).
&\\
\hline
\sphinxAtStartPar
\sphinxcode{\sphinxupquote{u10m}}
&
\sphinxAtStartPar
Gridbox westerly wind component at 10 m height (m s$^{\text{\sphinxhyphen{}1}}$).
&\\
\hline
\sphinxAtStartPar
\sphinxcode{\sphinxupquote{v1}}
&
\sphinxAtStartPar
Gridbox southerly wind component (m s$^{\text{\sphinxhyphen{}1}}$).
&\\
\hline
\sphinxAtStartPar
\sphinxcode{\sphinxupquote{v10m}}
&
\sphinxAtStartPar
Gridbox southerly wind component at 10m height (m s$^{\text{\sphinxhyphen{}1}}$).
&\\
\hline
\sphinxAtStartPar
\sphinxcode{\sphinxupquote{wind}}
&
\sphinxAtStartPar
Gridbox wind speed (m s$^{\text{\sphinxhyphen{}1}}$).
&\\
\hline
\end{tabulary}
\par
\sphinxattableend\end{savenotes}


\section{Radiation}
\label{\detokenize{output-variables:radiation}}

\begin{savenotes}\sphinxattablestart
\centering
\begin{tabulary}{\linewidth}[t]{|p{3.0cm}|p{10.3cm}|p{2.2cm}|}
\hline
\sphinxstyletheadfamily 
\sphinxAtStartPar
Name
&\sphinxstyletheadfamily 
\sphinxAtStartPar
Description
&\sphinxstyletheadfamily 
\sphinxAtStartPar
Dimensions
\\
\hline\sphinxstartmulticolumn{3}%
\begin{varwidth}[t]{\sphinxcolwidth{3}{3}}
\sphinxAtStartPar
Albedos and emissivities
\par
\vskip-\baselineskip\vbox{\hbox{\strut}}\end{varwidth}%
\sphinxstopmulticolumn
\\
\hline
\sphinxAtStartPar
\sphinxcode{\sphinxupquote{albedo\_land}}
&
\sphinxAtStartPar
Gridbox albedo (as used to calculate net shortwave radiation) (\sphinxhyphen{}).
&\\
\hline
\sphinxAtStartPar
\sphinxcode{\sphinxupquote{alb\_tile\_1}}
&
\sphinxAtStartPar
Tile land albedo, waveband 1 (direct beam visible).
&
\sphinxAtStartPar
nsurft
\\
\hline
\sphinxAtStartPar
\sphinxcode{\sphinxupquote{alb\_tile\_2}}
&
\sphinxAtStartPar
Tile land albedo, waveband 2 (diffuse visible).
&
\sphinxAtStartPar
nsurft
\\
\hline
\sphinxAtStartPar
\sphinxcode{\sphinxupquote{alb\_tile\_3}}
&
\sphinxAtStartPar
Tile land albedo, waveband 3 (direct beam NIR).
&
\sphinxAtStartPar
nsurft
\\
\hline
\sphinxAtStartPar
\sphinxcode{\sphinxupquote{alb\_tile\_4}}
&
\sphinxAtStartPar
Tile land albedo, waveband 4 (diffuse NIR).
&
\sphinxAtStartPar
nsurft
\\
\hline
\sphinxAtStartPar
\sphinxcode{\sphinxupquote{land\_albedo\_1}}
&
\sphinxAtStartPar
Gridbox band 1 albedo (direct beam visible).
&
\sphinxAtStartPar
land+sea
\\
\hline
\sphinxAtStartPar
\sphinxcode{\sphinxupquote{land\_albedo\_2}}
&
\sphinxAtStartPar
Gridbox band 2 albedo (diffuse visible).
&
\sphinxAtStartPar
land+sea
\\
\hline
\sphinxAtStartPar
\sphinxcode{\sphinxupquote{land\_albedo\_3}}
&
\sphinxAtStartPar
Gridbox band 3 albedo (direct beam NIR).
&
\sphinxAtStartPar
land+sea
\\
\hline
\sphinxAtStartPar
\sphinxcode{\sphinxupquote{land\_albedo\_4}}
&
\sphinxAtStartPar
Gridbox band 4 albedo (diffuse NIR).
&
\sphinxAtStartPar
land+sea
\\
\hline
\sphinxAtStartPar
\sphinxcode{\sphinxupquote{emis\_gb}}
&
\sphinxAtStartPar
Gridbox emissivity.
&\\
\hline
\sphinxAtStartPar
\sphinxcode{\sphinxupquote{emis}}
&
\sphinxAtStartPar
Tile emissivity.
&
\sphinxAtStartPar
nsurft
\\
\hline\sphinxstartmulticolumn{3}%
\begin{varwidth}[t]{\sphinxcolwidth{3}{3}}
\sphinxAtStartPar
Radiation fluxes
\par
\vskip-\baselineskip\vbox{\hbox{\strut}}\end{varwidth}%
\sphinxstopmulticolumn
\\
\hline
\sphinxAtStartPar
\sphinxcode{\sphinxupquote{apar}}
&
\sphinxAtStartPar
PFT absorbed photosynthetically active radiation (W m$^{\text{\sphinxhyphen{}2}}$).
&
\sphinxAtStartPar
npft
\\
\hline
\sphinxAtStartPar
\sphinxcode{\sphinxupquote{apar\_gb}}
&
\sphinxAtStartPar
Gridbox absorbed photosynthetically active radiation (W m$^{\text{\sphinxhyphen{}2}}$).
&\\
\hline
\sphinxAtStartPar
\sphinxcode{\sphinxupquote{lw\_down\_surft}}
&
\sphinxAtStartPar
Tile downwelling longwave radiation  (W m$^{\text{\sphinxhyphen{}2}}$).
&
\sphinxAtStartPar
nsurft
\\
\hline
\sphinxAtStartPar
\sphinxcode{\sphinxupquote{lw\_up\_surft}}
&
\sphinxAtStartPar
Tile upwelling longwave radiation  (W m$^{\text{\sphinxhyphen{}2}}$).
&
\sphinxAtStartPar
nsurft
\\
\hline
\sphinxAtStartPar
\sphinxcode{\sphinxupquote{lw\_net}}
&
\sphinxAtStartPar
Gridbox surface net LW radiation (W m$^{\text{\sphinxhyphen{}2}}$).
&\\
\hline
\sphinxAtStartPar
\sphinxcode{\sphinxupquote{lw\_up}}
&
\sphinxAtStartPar
Gridbox surface upward LW radiation (W m$^{\text{\sphinxhyphen{}2}}$).
&\\
\hline
\sphinxAtStartPar
\sphinxcode{\sphinxupquote{rad\_net}}
&
\sphinxAtStartPar
Surface net radiation (W m$^{\text{\sphinxhyphen{}2}}$).
&\\
\hline
\sphinxAtStartPar
\sphinxcode{\sphinxupquote{rad\_net\_tile}}
&
\sphinxAtStartPar
Tile surface net radiation (W m$^{\text{\sphinxhyphen{}2}}$).
&
\sphinxAtStartPar
nsurft
\\
\hline
\sphinxAtStartPar
\sphinxcode{\sphinxupquote{sw\_net}}
&
\sphinxAtStartPar
Gribox net shortwave radiation at the surface (W m$^{\text{\sphinxhyphen{}2}}$).
&\\
\hline
\sphinxAtStartPar
\sphinxcode{\sphinxupquote{sw\_net\_surft}}
&
\sphinxAtStartPar
Tile net shortwave radiation  (W m$^{\text{\sphinxhyphen{}2}}$).
&
\sphinxAtStartPar
nsurft
\\
\hline\sphinxstartmulticolumn{3}%
\begin{varwidth}[t]{\sphinxcolwidth{3}{3}}
\sphinxAtStartPar
Other radiation variables
\par
\vskip-\baselineskip\vbox{\hbox{\strut}}\end{varwidth}%
\sphinxstopmulticolumn
\\
\hline
\sphinxAtStartPar
\sphinxcode{\sphinxupquote{cosz}}
&
\sphinxAtStartPar
Cosine of the zenith angle (\sphinxhyphen{}).
&
\sphinxAtStartPar
land+sea
\\
\hline
\sphinxAtStartPar
\sphinxcode{\sphinxupquote{diff\_frac}}
&
\sphinxAtStartPar
Gridbox fraction of radiation that is diffuse (\sphinxhyphen{}).
&
\sphinxAtStartPar
land+sea
\\
\hline
\sphinxAtStartPar
\sphinxcode{\sphinxupquote{fapar}}
&
\sphinxAtStartPar
PFT fraction of absorbed photosynthetically active radiation (\sphinxhyphen{}).
&
\sphinxAtStartPar
npft
\\
\hline
\sphinxAtStartPar
\sphinxcode{\sphinxupquote{NDVI\_land}}
&
\sphinxAtStartPar
Gridbox NDVI (using sum of direct and diffuse for (NIR\sphinxhyphen{}VIS)/(NIR+VIS)).
&\\
\hline
\sphinxAtStartPar
\sphinxcode{\sphinxupquote{trad}}
&
\sphinxAtStartPar
Gridbox effective radiative temperature (K).
&\\
\hline
\end{tabulary}
\par
\sphinxattableend\end{savenotes}


\section{Energy and momentum fluxes, and surface temperatures}
\label{\detokenize{output-variables:energy-and-momentum-fluxes-and-surface-temperatures}}

\begin{savenotes}\sphinxattablestart
\centering
\begin{tabulary}{\linewidth}[t]{|p{3.5cm}|p{9.8cm}|p{2.2cm}|}
\hline
\sphinxstyletheadfamily 
\sphinxAtStartPar
Name
&\sphinxstyletheadfamily 
\sphinxAtStartPar
Description
&\sphinxstyletheadfamily 
\sphinxAtStartPar
Dimensions
\\
\hline
\sphinxAtStartPar
\sphinxcode{\sphinxupquote{ftl}}
&
\sphinxAtStartPar
Tile surface sensible heat flux for land tiles (W m$^{\text{\sphinxhyphen{}2}}$).
&
\sphinxAtStartPar
nsurft
\\
\hline
\sphinxAtStartPar
\sphinxcode{\sphinxupquote{ftl\_gb}}
&
\sphinxAtStartPar
Gridbox surface sensible heat flux (W m$^{\text{\sphinxhyphen{}2}}$).
&
\sphinxAtStartPar
land+sea
\\
\hline
\sphinxAtStartPar
\sphinxcode{\sphinxupquote{le}}
&
\sphinxAtStartPar
Tile surface latent heat flux for land tiles (W m$^{\text{\sphinxhyphen{}2}}$).
&
\sphinxAtStartPar
nsurft
\\
\hline
\sphinxAtStartPar
\sphinxcode{\sphinxupquote{latent\_heat}}
&
\sphinxAtStartPar
Gridbox surface latent heat flux (W m$^{\text{\sphinxhyphen{}2}}$).
&
\sphinxAtStartPar
land+sea
\\
\hline
\sphinxAtStartPar
\sphinxcode{\sphinxupquote{surf\_ht\_flux}}
&
\sphinxAtStartPar
Downward heat flux for each tile (W m$^{\text{\sphinxhyphen{}2}}$).
&
\sphinxAtStartPar
nsurft
\\
\hline
\sphinxAtStartPar
\sphinxcode{\sphinxupquote{surf\_ht\_store}}
&
\sphinxAtStartPar
C*(dT/dt) for each tile (W m$^{\text{\sphinxhyphen{}2}}$).
&
\sphinxAtStartPar
nsurft
\\
\hline
\sphinxAtStartPar
\sphinxcode{\sphinxupquote{surf\_ht\_flux\_gb}}
&
\sphinxAtStartPar
Gridbox net downward heat flux at surface over land and sea\sphinxhyphen{}ice fraction of gridbox
(W m$^{\text{\sphinxhyphen{}2}}$).
&
\sphinxAtStartPar
land+sea
\\
\hline
\sphinxAtStartPar
\sphinxcode{\sphinxupquote{anthrop\_heat}}
&
\sphinxAtStartPar
Anthropogenic heat flux for each tile (W m$^{\text{\sphinxhyphen{}2}}$).
&
\sphinxAtStartPar
nsurft
\\
\hline
\sphinxAtStartPar
\sphinxcode{\sphinxupquote{hf\_snow\_melt}}
&
\sphinxAtStartPar
Gridbox snowmelt heat flux (W m$^{\text{\sphinxhyphen{}2}}$).
&\\
\hline
\sphinxAtStartPar
\sphinxcode{\sphinxupquote{snomlt\_surf\_htf}}
&
\sphinxAtStartPar
Gridbox heat flux used for surface melting of snow (W m$^{\text{\sphinxhyphen{}2}}$).
&
\sphinxAtStartPar
land+sea
\\
\hline
\sphinxAtStartPar
\sphinxcode{\sphinxupquote{snomlt\_sub\_htf}}
&
\sphinxAtStartPar
Gridbox sub\sphinxhyphen{}canopy snowmelt heat flux (W m$^{\text{\sphinxhyphen{}2}}$).
&\\
\hline
\sphinxAtStartPar
\sphinxcode{\sphinxupquote{tstar\_gb}}
&
\sphinxAtStartPar
Gridbox surface temperature (K).
&
\sphinxAtStartPar
land+sea
\\
\hline
\sphinxAtStartPar
\sphinxcode{\sphinxupquote{tstar}}
&
\sphinxAtStartPar
Tile surface temperature (K).
&
\sphinxAtStartPar
nsurft
\\
\hline
\sphinxAtStartPar
\sphinxcode{\sphinxupquote{tsurf\_elev\_surft}}
&
\sphinxAtStartPar
Tile temperature of elevated subsurface tiles (K).
&
\sphinxAtStartPar
nsurft
\\
\hline
\sphinxAtStartPar
\sphinxcode{\sphinxupquote{tau}}
&
\sphinxAtStartPar
Tile surface wind stress for land tiles (N m$^{\text{\sphinxhyphen{}2}}$).
&
\sphinxAtStartPar
nsurft
\\
\hline
\sphinxAtStartPar
\sphinxcode{\sphinxupquote{taux1}}
&
\sphinxAtStartPar
Gridbox westerly component of surface wind stress (N m$^{\text{\sphinxhyphen{}2}}$).
&
\sphinxAtStartPar
land+sea
\\
\hline
\sphinxAtStartPar
\sphinxcode{\sphinxupquote{tauy1}}
&
\sphinxAtStartPar
Gridbox southerly component of surface wind stress (N m$^{\text{\sphinxhyphen{}2}}$).
&
\sphinxAtStartPar
land+sea
\\
\hline
\sphinxAtStartPar
\sphinxcode{\sphinxupquote{tauy\_gb}}
&
\sphinxAtStartPar
Gridbox scalar magnitude of surface wind stress (N m$^{\text{\sphinxhyphen{}2}}$).
&
\sphinxAtStartPar
land+sea
\\
\hline
\sphinxAtStartPar
\sphinxcode{\sphinxupquote{z0}}
&
\sphinxAtStartPar
Tile surface roughness (m).
&
\sphinxAtStartPar
nsurft
\\
\hline
\end{tabulary}
\par
\sphinxattableend\end{savenotes}


\section{Soil moisture and temperature, and soil characteristics}
\label{\detokenize{output-variables:soil-moisture-and-temperature-and-soil-characteristics}}

\begin{savenotes}\sphinxattablestart
\centering
\begin{tabulary}{\linewidth}[t]{|p{4.0cm}|p{9.3cm}|p{2.2cm}|}
\hline
\sphinxstyletheadfamily 
\sphinxAtStartPar
Name
&\sphinxstyletheadfamily 
\sphinxAtStartPar
Description
&\sphinxstyletheadfamily 
\sphinxAtStartPar
Dimensions
\\
\hline\sphinxstartmulticolumn{3}%
\begin{varwidth}[t]{\sphinxcolwidth{3}{3}}
\sphinxAtStartPar
Soil moisture
\par
\vskip-\baselineskip\vbox{\hbox{\strut}}\end{varwidth}%
\sphinxstopmulticolumn
\\
\hline
\sphinxAtStartPar
\sphinxcode{\sphinxupquote{smcl{[}\_soilt{]}}}
&
\sphinxAtStartPar
Moisture content of each soil layer (kg m$^{\text{\sphinxhyphen{}2}}$).
&
\sphinxAtStartPar
sm\_levels
\\
\hline
\sphinxAtStartPar
\sphinxcode{\sphinxupquote{soil\_wet}}
&
\sphinxAtStartPar
Total moisture content of each soil layer, as fraction of saturation (\sphinxhyphen{}).
&
\sphinxAtStartPar
sm\_levels
\\
\hline
\sphinxAtStartPar
\sphinxcode{\sphinxupquote{sthu{[}\_soilt{]}}}
&
\sphinxAtStartPar
Unfrozen moisture content of each soil layer as a fraction of saturation (\sphinxhyphen{}).
&
\sphinxAtStartPar
sm\_levels
\\
\hline
\sphinxAtStartPar
\sphinxcode{\sphinxupquote{sthu\_irr{[}\_soilt{]}}}
&
\sphinxAtStartPar
Unfrozen moisture content of each soil layer as a fraction of saturation in
irrigated fraction (\sphinxhyphen{}) (only available if l\_irrig\_dmd = T).
&
\sphinxAtStartPar
sm\_levels
\\
\hline
\sphinxAtStartPar
\sphinxcode{\sphinxupquote{sthf{[}\_soilt{]}}}
&
\sphinxAtStartPar
Frozen moisture content of each soil layer as a fraction of saturation (\sphinxhyphen{}).
&
\sphinxAtStartPar
sm\_levels
\\
\hline
\sphinxAtStartPar
\sphinxcode{\sphinxupquote{smc\_tot}}
&
\sphinxAtStartPar
Gridbox total soil moisture in column (kg m$^{\text{\sphinxhyphen{}2}}$).
&\\
\hline
\sphinxAtStartPar
\sphinxcode{\sphinxupquote{swet\_liq\_tot}}
&
\sphinxAtStartPar
Gridbox unfrozen soil moisture as fraction of saturation (\sphinxhyphen{}).
&\\
\hline
\sphinxAtStartPar
\sphinxcode{\sphinxupquote{swet\_tot}}
&
\sphinxAtStartPar
Gridbox soil moisture as fraction of saturation (\sphinxhyphen{}).
&\\
\hline
\sphinxAtStartPar
\sphinxcode{\sphinxupquote{sthzw{[}\_soilt{]}}}
&
\sphinxAtStartPar
Soil wetness in the deep LSH/TOPMODEL layer (\sphinxhyphen{}).
&\\
\hline
\sphinxAtStartPar
\sphinxcode{\sphinxupquote{zw{[}\_soilt{]}}}
&
\sphinxAtStartPar
Gridbox mean depth to water table (m).
&\\
\hline\sphinxstartmulticolumn{3}%
\begin{varwidth}[t]{\sphinxcolwidth{3}{3}}
\sphinxAtStartPar
Soil temperature
\par
\vskip-\baselineskip\vbox{\hbox{\strut}}\end{varwidth}%
\sphinxstopmulticolumn
\\
\hline
\sphinxAtStartPar
\sphinxcode{\sphinxupquote{t\_soil{[}\_soilt{]}}}
&
\sphinxAtStartPar
Sub\sphinxhyphen{}surface temperature of each layer (K).
&
\sphinxAtStartPar
sm\_levels
\\
\hline
\sphinxAtStartPar
\sphinxcode{\sphinxupquote{tsoil\_deep}}
&
\sphinxAtStartPar
Temperature of each bedrock layer (K).
Only available when {\hyperref[\detokenize{namelists/jules_soil.nml:JULES_SOIL::l_bedrock}]{\sphinxcrossref{\sphinxcode{\sphinxupquote{l\_bedrock}}}}} = TRUE.
&
\sphinxAtStartPar
ns\_deep
\\
\hline
\sphinxAtStartPar
\sphinxcode{\sphinxupquote{depth\_frozen}}
&
\sphinxAtStartPar
Gridbox depth of frozen ground at surface defined from soil temperature (m).
&\\
\hline
\sphinxAtStartPar
\sphinxcode{\sphinxupquote{depth\_frozen\_sthf}}
&
\sphinxAtStartPar
Gridbox depth of frozen ground at surface defined from soil moisture (m).
Recommended over \sphinxcode{\sphinxupquote{depth\_frozen}} except where the soil is very dry.
&\\
\hline
\sphinxAtStartPar
\sphinxcode{\sphinxupquote{depth\_unfrozen}}
&
\sphinxAtStartPar
Gridbox depth of unfrozen ground at surface defined from soil temperature (m).
&\\
\hline
\sphinxAtStartPar
\sphinxcode{\sphinxupquote{depth\_unfrozen\_sthf}}
&
\sphinxAtStartPar
Gridbox depth of unfrozen ground at surface defined from soil moisture (m).
Recommended over \sphinxcode{\sphinxupquote{depth\_unfrozen}} except where the soil is very dry.
&\\
\hline\sphinxstartmulticolumn{3}%
\begin{varwidth}[t]{\sphinxcolwidth{3}{3}}
\sphinxAtStartPar
Soil characteristics
\par
\vskip-\baselineskip\vbox{\hbox{\strut}}\end{varwidth}%
\sphinxstopmulticolumn
\\
\hline
\sphinxAtStartPar
\sphinxcode{\sphinxupquote{b{[}\_soilt{]}}}
&
\sphinxAtStartPar
Brooks\sphinxhyphen{}Corey exponent for each soil layer (\sphinxhyphen{}).
&
\sphinxAtStartPar
sm\_levels
\\
\hline
\sphinxAtStartPar
\sphinxcode{\sphinxupquote{hcap{[}\_soilt{]}}}
&
\sphinxAtStartPar
Dry soil heat capacity (J K$^{\text{\sphinxhyphen{}1}}$ m$^{\text{\sphinxhyphen{}3}}$) for each soil layer.
&
\sphinxAtStartPar
sm\_levels
\\
\hline
\sphinxAtStartPar
\sphinxcode{\sphinxupquote{hcon{[}\_soilt{]}}}
&
\sphinxAtStartPar
Dry soil thermal conductivity (W m$^{\text{\sphinxhyphen{}1}}$ K$^{\text{\sphinxhyphen{}1}}$) for each soil
layer.
&
\sphinxAtStartPar
sm\_levels
\\
\hline
\sphinxAtStartPar
\sphinxcode{\sphinxupquote{satcon{[}\_soilt{]}}}
&
\sphinxAtStartPar
Saturated hydraulic conductivity (kg m$^{\text{\sphinxhyphen{}2}}$ s$^{\text{\sphinxhyphen{}1}}$) for each soil
layer.
&
\sphinxAtStartPar
sm\_levels
\\
\hline
\sphinxAtStartPar
\sphinxcode{\sphinxupquote{sathh{[}\_soilt{]}}}
&
\sphinxAtStartPar
Saturated soil water pressure (m) for each soil layer.
&
\sphinxAtStartPar
sm\_levels
\\
\hline
\sphinxAtStartPar
\sphinxcode{\sphinxupquote{sm\_crit{[}\_soilt{]}}}
&
\sphinxAtStartPar
Volumetric moisture content at critical point for each soil layer (\sphinxhyphen{}),
as given in {\hyperref[\detokenize{namelists/ancillaries.nml:namelist-JULES_SOIL_PROPS}]{\sphinxcrossref{\sphinxcode{\sphinxupquote{JULES\_SOIL\_PROPS}}}}}.
&
\sphinxAtStartPar
sm\_levels
\\
\hline
\sphinxAtStartPar
\sphinxcode{\sphinxupquote{sm\_sat{[}\_soilt{]}}}
&
\sphinxAtStartPar
Volumetric moisture content at saturation for each soil layer (\sphinxhyphen{}).
&
\sphinxAtStartPar
sm\_levels
\\
\hline
\sphinxAtStartPar
\sphinxcode{\sphinxupquote{sm\_wilt{[}\_soilt{]}}}
&
\sphinxAtStartPar
Volumetric moisture content at wilting point for each soil layer (\sphinxhyphen{}),
as given in {\hyperref[\detokenize{namelists/ancillaries.nml:namelist-JULES_SOIL_PROPS}]{\sphinxcrossref{\sphinxcode{\sphinxupquote{JULES\_SOIL\_PROPS}}}}}.
&
\sphinxAtStartPar
sm\_levels
\\
\hline
\end{tabulary}
\par
\sphinxattableend\end{savenotes}


\section{Hydrology}
\label{\detokenize{output-variables:hydrology}}

\begin{savenotes}\sphinxatlongtablestart\begin{longtable}[c]{|p{4.6cm}|p{8.7cm}|p{2.2cm}|}
\hline
\sphinxstyletheadfamily 
\sphinxAtStartPar
Name
&\sphinxstyletheadfamily 
\sphinxAtStartPar
Description
&\sphinxstyletheadfamily 
\sphinxAtStartPar
Dimensions
\\
\hline
\endfirsthead

\multicolumn{3}{c}%
{\makebox[0pt]{\sphinxtablecontinued{\tablename\ \thetable{} \textendash{} continued from previous page}}}\\
\hline
\sphinxstyletheadfamily 
\sphinxAtStartPar
Name
&\sphinxstyletheadfamily 
\sphinxAtStartPar
Description
&\sphinxstyletheadfamily 
\sphinxAtStartPar
Dimensions
\\
\hline
\endhead

\hline
\multicolumn{3}{r}{\makebox[0pt][r]{\sphinxtablecontinued{continues on next page}}}\\
\endfoot

\endlastfoot
\sphinxstartmulticolumn{3}%
\begin{varwidth}[t]{\sphinxcolwidth{3}{3}}
\sphinxAtStartPar
Canopy hydrology
\par
\vskip-\baselineskip\vbox{\hbox{\strut}}\end{varwidth}%
\sphinxstopmulticolumn
\\
\hline
\sphinxAtStartPar
\sphinxcode{\sphinxupquote{canopy\_gb}}
&
\sphinxAtStartPar
Gridbox canopy water content (kg m$^{\text{\sphinxhyphen{}2}}$).
&\\
\hline
\sphinxAtStartPar
\sphinxcode{\sphinxupquote{canopy}}
&
\sphinxAtStartPar
Tile surface/canopy water for snow\sphinxhyphen{}free land tiles (kg m$^{\text{\sphinxhyphen{}2}}$).
&
\sphinxAtStartPar
nsurft
\\
\hline
\sphinxAtStartPar
\sphinxcode{\sphinxupquote{catch}}
&
\sphinxAtStartPar
Tile surface/canopy water capacity of snow\sphinxhyphen{}free land tiles (kg m$^{\text{\sphinxhyphen{}2}}$).
&
\sphinxAtStartPar
nsurft
\\
\hline
\sphinxAtStartPar
\sphinxcode{\sphinxupquote{tfall}}
&
\sphinxAtStartPar
Gridbox throughfall (kg m$^{\text{\sphinxhyphen{}2}}$ s$^{\text{\sphinxhyphen{}1}}$).
&\\
\hline\sphinxstartmulticolumn{3}%
\begin{varwidth}[t]{\sphinxcolwidth{3}{3}}
\sphinxAtStartPar
Evaporation and sublimation
\par
\vskip-\baselineskip\vbox{\hbox{\strut}}\end{varwidth}%
\sphinxstopmulticolumn
\\
\hline
\sphinxAtStartPar
\sphinxcode{\sphinxupquote{fqw}}
&
\sphinxAtStartPar
Tile surface moisture flux for land tiles (kg m$^{\text{\sphinxhyphen{}2}}$ s$^{\text{\sphinxhyphen{}1}}$).
&
\sphinxAtStartPar
nsurft
\\
\hline
\sphinxAtStartPar
\sphinxcode{\sphinxupquote{fqw\_gb}}
&
\sphinxAtStartPar
Gridbox moisture flux from surface (kg m$^{\text{\sphinxhyphen{}2}}$ s$^{\text{\sphinxhyphen{}1}}$).
&
\sphinxAtStartPar
land+sea
\\
\hline
\sphinxAtStartPar
\sphinxcode{\sphinxupquote{ecan}}
&
\sphinxAtStartPar
Tile evaporation from canopy/surface store for snow\sphinxhyphen{}free
land tiles (kg m$^{\text{\sphinxhyphen{}2}}$ s$^{\text{\sphinxhyphen{}1}}$).
&
\sphinxAtStartPar
nsurft
\\
\hline
\sphinxAtStartPar
\sphinxcode{\sphinxupquote{ecan\_gb}}
&
\sphinxAtStartPar
Gridbox mean evaporation from canopy/surface store (kg m$^{\text{\sphinxhyphen{}2}}$ s$^{\text{\sphinxhyphen{}1}}$).
&
\sphinxAtStartPar
land+sea
\\
\hline
\sphinxAtStartPar
\sphinxcode{\sphinxupquote{ei}}
&
\sphinxAtStartPar
Tile sublimation from lying snow for land tiles (kg m$^{\text{\sphinxhyphen{}2}}$ s$^{\text{\sphinxhyphen{}1}}$).
&
\sphinxAtStartPar
nsurft
\\
\hline
\sphinxAtStartPar
\sphinxcode{\sphinxupquote{ei\_gb}}
&
\sphinxAtStartPar
Gridbox sublimation from lying snow or sea\sphinxhyphen{}ice (kg m$^{\text{\sphinxhyphen{}2}}$ s$^{\text{\sphinxhyphen{}1}}$).
&
\sphinxAtStartPar
land+sea
\\
\hline
\sphinxAtStartPar
\sphinxcode{\sphinxupquote{elake}}
&
\sphinxAtStartPar
Gridbox mean evaporation from lakes (kg m$^{\text{\sphinxhyphen{}2}}$ s$^{\text{\sphinxhyphen{}1}}$).
&\\
\hline
\sphinxAtStartPar
\sphinxcode{\sphinxupquote{esoil}}
&
\sphinxAtStartPar
Tile surface evapotranspiration from soil moisture store for
snow\sphinxhyphen{}free land tile (kg m$^{\text{\sphinxhyphen{}2}}$ s$^{\text{\sphinxhyphen{}1}}$).
&
\sphinxAtStartPar
nsurft
\\
\hline
\sphinxAtStartPar
\sphinxcode{\sphinxupquote{esoil\_gb}}
&
\sphinxAtStartPar
Gridbox surface evapotranspiration from soil moisture store (kg m$^{\text{\sphinxhyphen{}2}}$ s$^{\text{\sphinxhyphen{}1}}$).
&
\sphinxAtStartPar
land+sea
\\
\hline
\sphinxAtStartPar
\sphinxcode{\sphinxupquote{et\_stom}}
&
\sphinxAtStartPar
Tile transpiration (kg m$^{\text{\sphinxhyphen{}2}}$ s$^{\text{\sphinxhyphen{}1}}$).
&
\sphinxAtStartPar
nsurft
\\
\hline
\sphinxAtStartPar
\sphinxcode{\sphinxupquote{et\_stom\_gb}}
&
\sphinxAtStartPar
Gridbox transpiration (kg m$^{\text{\sphinxhyphen{}2}}$ s$^{\text{\sphinxhyphen{}1}}$).
&
\sphinxAtStartPar
land+sea
\\
\hline
\sphinxAtStartPar
\sphinxcode{\sphinxupquote{fao\_et0}}
&
\sphinxAtStartPar
FAO Penman\sphinxhyphen{}Monteith evapotranspiration for reference crop (kg m$^{\text{\sphinxhyphen{}2}}$ s$^{\text{\sphinxhyphen{}1}}$)
&\\
\hline
\sphinxAtStartPar
\sphinxcode{\sphinxupquote{gc}}
&
\sphinxAtStartPar
Tile surface conductance to evaporation for land tiles (m s$^{\text{\sphinxhyphen{}1}}$).
&
\sphinxAtStartPar
nsurft
\\
\hline
\sphinxAtStartPar
\sphinxcode{\sphinxupquote{gs}}
&
\sphinxAtStartPar
Gridbox surface conductance to evaporation (m s$^{\text{\sphinxhyphen{}1}}$).
&\\
\hline
\sphinxAtStartPar
\sphinxcode{\sphinxupquote{ext{[}\_soilt{]}}}
&
\sphinxAtStartPar
Extraction of water from each soil layer (kg m$^{\text{\sphinxhyphen{}2}}$ s$^{\text{\sphinxhyphen{}1}}$).
&
\sphinxAtStartPar
sm\_levels
\\
\hline
\sphinxAtStartPar
\sphinxcode{\sphinxupquote{fsmc\_gb}}
&
\sphinxAtStartPar
Gridbox soil moisture availability factor (beta) (\sphinxhyphen{}).
&\\
\hline
\sphinxAtStartPar
\sphinxcode{\sphinxupquote{fsmc}}
&
\sphinxAtStartPar
PFT soil moisture availability factor (\sphinxhyphen{}).
&
\sphinxAtStartPar
npft
\\
\hline
\sphinxAtStartPar
\sphinxcode{\sphinxupquote{smc\_avail\_top}}
&
\sphinxAtStartPar
Gridbox available moisture in surface layer of depth given by {\hyperref[\detokenize{namelists/jules_soil.nml:JULES_SOIL::zsmc}]{\sphinxcrossref{\sphinxcode{\sphinxupquote{zsmc}}}}}
(kg m$^{\text{\sphinxhyphen{}2}}$). Calculated using \sphinxcode{\sphinxupquote{sm\_wilt}} from {\hyperref[\detokenize{namelists/ancillaries.nml:namelist-JULES_SOIL_PROPS}]{\sphinxcrossref{\sphinxcode{\sphinxupquote{JULES\_SOIL\_PROPS}}}}}.
&\\
\hline
\sphinxAtStartPar
\sphinxcode{\sphinxupquote{smc\_avail\_tot}}
&
\sphinxAtStartPar
Gridbox available moisture in soil column (kg m$^{\text{\sphinxhyphen{}2}}$).
Calculated using \sphinxcode{\sphinxupquote{sm\_wilt}} from {\hyperref[\detokenize{namelists/ancillaries.nml:namelist-JULES_SOIL_PROPS}]{\sphinxcrossref{\sphinxcode{\sphinxupquote{JULES\_SOIL\_PROPS}}}}}.
&\\
\hline\sphinxstartmulticolumn{3}%
\begin{varwidth}[t]{\sphinxcolwidth{3}{3}}
\sphinxAtStartPar
Runoff
\par
\vskip-\baselineskip\vbox{\hbox{\strut}}\end{varwidth}%
\sphinxstopmulticolumn
\\
\hline
\sphinxAtStartPar
\sphinxcode{\sphinxupquote{runoff}}
&
\sphinxAtStartPar
Gridbox runoff rate (kg m$^{\text{\sphinxhyphen{}2}}$ s$^{\text{\sphinxhyphen{}1}}$).
&\\
\hline
\sphinxAtStartPar
\sphinxcode{\sphinxupquote{sub\_surf\_roff}}
&
\sphinxAtStartPar
Gridbox sub\sphinxhyphen{}surface runoff (kg m$^{\text{\sphinxhyphen{}2}}$ s$^{\text{\sphinxhyphen{}1}}$).
&\\
\hline
\sphinxAtStartPar
\sphinxcode{\sphinxupquote{surf\_roff}}
&
\sphinxAtStartPar
Gridbox surface runoff (kg m$^{\text{\sphinxhyphen{}2}}$ s$^{\text{\sphinxhyphen{}1}}$).
&\\
\hline
\sphinxAtStartPar
\sphinxcode{\sphinxupquote{sat\_excess\_roff{[}\_soilt{]}}}
&
\sphinxAtStartPar
Gridbox saturation excess runoff rate (kg m$^{\text{\sphinxhyphen{}2}}$ s$^{\text{\sphinxhyphen{}1}}$).
&\\
\hline
\sphinxAtStartPar
\sphinxcode{\sphinxupquote{drain{[}\_soilt{]}}}
&
\sphinxAtStartPar
Gridbox drainage at bottom of soil column (kg m$^{\text{\sphinxhyphen{}2}}$ s$^{\text{\sphinxhyphen{}1}}$).
&\\
\hline
\sphinxAtStartPar
\sphinxcode{\sphinxupquote{qbase{[}\_soilt{]}}}
&
\sphinxAtStartPar
Gridbox baseflow (lateral subsurface runoff) (kg m$^{\text{\sphinxhyphen{}2}}$ s$^{\text{\sphinxhyphen{}1}}$), i.e. the sum of
surface and subsurface lateral flows from all soil layers (inc. deep LSH/TOPMODEL layer).
Only available if {\hyperref[\detokenize{namelists/jules_hydrology.nml:JULES_HYDROLOGY::l_top}]{\sphinxcrossref{\sphinxcode{\sphinxupquote{l\_top}}}}} = TRUE.
&\\
\hline
\sphinxAtStartPar
\sphinxcode{\sphinxupquote{qbase\_zw{[}\_soilt{]}}}
&
\sphinxAtStartPar
Gridbox baseflow (lateral subsurface runoff) from deep LSH/TOPMODEL layer
(kg m$^{\text{\sphinxhyphen{}2}}$ s$^{\text{\sphinxhyphen{}1}}$).
Only available if {\hyperref[\detokenize{namelists/jules_hydrology.nml:JULES_HYDROLOGY::l_top}]{\sphinxcrossref{\sphinxcode{\sphinxupquote{l\_top}}}}} = TRUE.
&\\
\hline\sphinxstartmulticolumn{3}%
\begin{varwidth}[t]{\sphinxcolwidth{3}{3}}
\sphinxAtStartPar
Other hydrological variables
\par
\vskip-\baselineskip\vbox{\hbox{\strut}}\end{varwidth}%
\sphinxstopmulticolumn
\\
\hline
\sphinxAtStartPar
\sphinxcode{\sphinxupquote{fsat{[}\_soilt{]}}}
&
\sphinxAtStartPar
Gridbox surface saturated fraction (\sphinxhyphen{}). The fraction of grid cell where the water table is
above the land surface. Only available if {\hyperref[\detokenize{namelists/jules_hydrology.nml:JULES_HYDROLOGY::l_top}]{\sphinxcrossref{\sphinxcode{\sphinxupquote{l\_top}}}}} = TRUE.
&\\
\hline
\sphinxAtStartPar
\sphinxcode{\sphinxupquote{fwetl{[}\_soilt{]}}}
&
\sphinxAtStartPar
Gridbox wetland fraction at end of model timestep (\sphinxhyphen{}). The fraction of grid cell where the
water table is above the land surface, but water is not flowing (stagnant) (fwetl\textless{}=fsat).
Only available if {\hyperref[\detokenize{namelists/jules_hydrology.nml:JULES_HYDROLOGY::l_top}]{\sphinxcrossref{\sphinxcode{\sphinxupquote{l\_top}}}}} = TRUE.
&\\
\hline
\end{longtable}\sphinxatlongtableend\end{savenotes}


\section{Rivers}
\label{\detokenize{output-variables:rivers}}

\begin{savenotes}\sphinxattablestart
\centering
\begin{tabulary}{\linewidth}[t]{|p{4.6cm}|p{8.7cm}|p{2.2cm}|}
\hline
\sphinxstyletheadfamily 
\sphinxAtStartPar
Name
&\sphinxstyletheadfamily 
\sphinxAtStartPar
Description
&\sphinxstyletheadfamily 
\sphinxAtStartPar
Dimensions
\\
\hline\sphinxstartmulticolumn{3}%
\begin{varwidth}[t]{\sphinxcolwidth{3}{3}}
\sphinxAtStartPar
Output on the river routing model grid
\par
\vskip-\baselineskip\vbox{\hbox{\strut}}\end{varwidth}%
\sphinxstopmulticolumn
\\
\hline
\sphinxAtStartPar
\sphinxcode{\sphinxupquote{rflow\_rp}}
&
\sphinxAtStartPar
River routing gridbox river flow rate (kg m$^{\text{\sphinxhyphen{}2}}$ s$^{\text{\sphinxhyphen{}1}}$).
Only available if {\hyperref[\detokenize{namelists/jules_rivers.nml:JULES_RIVERS::l_rivers}]{\sphinxcrossref{\sphinxcode{\sphinxupquote{l\_rivers}}}}} = TRUE.
&
\sphinxAtStartPar
np\_rivers
\\
\hline
\sphinxAtStartPar
\sphinxcode{\sphinxupquote{rrun\_rp}}
&
\sphinxAtStartPar
River routing gridbox runoff rate received by river routing routine
(kg m$^{\text{\sphinxhyphen{}2}}$ s$^{\text{\sphinxhyphen{}1}}$).
Only available if {\hyperref[\detokenize{namelists/jules_rivers.nml:JULES_RIVERS::l_rivers}]{\sphinxcrossref{\sphinxcode{\sphinxupquote{l\_rivers}}}}} = TRUE.
&
\sphinxAtStartPar
np\_rivers
\\
\hline
\sphinxAtStartPar
\sphinxcode{\sphinxupquote{rrun\_surf\_rp}}
&
\sphinxAtStartPar
River routing gridbox surface runoff rate received by river routing routine
(kg m$^{\text{\sphinxhyphen{}2}}$ s$^{\text{\sphinxhyphen{}1}}$).
Only available if {\hyperref[\detokenize{namelists/jules_rivers.nml:JULES_RIVERS::l_rivers}]{\sphinxcrossref{\sphinxcode{\sphinxupquote{l\_rivers}}}}} = TRUE.
&
\sphinxAtStartPar
np\_rivers
\\
\hline
\sphinxAtStartPar
\sphinxcode{\sphinxupquote{rrun\_sub\_surf\_rp}}
&
\sphinxAtStartPar
River routing gridbox sub\sphinxhyphen{}surface runoff rate received by river routing routine
(kg m$^{\text{\sphinxhyphen{}2}}$ s$^{\text{\sphinxhyphen{}1}}$).
Only available if {\hyperref[\detokenize{namelists/jules_rivers.nml:JULES_RIVERS::l_rivers}]{\sphinxcrossref{\sphinxcode{\sphinxupquote{l\_rivers}}}}} = TRUE.
&
\sphinxAtStartPar
np\_rivers
\\
\hline
\sphinxAtStartPar
\sphinxcode{\sphinxupquote{rfm\_surfstore\_rp}}
&
\sphinxAtStartPar
Surface storage on river points (m$^{\text{3}}$).
Only available if {\hyperref[\detokenize{namelists/jules_rivers.nml:JULES_RIVERS::l_rivers}]{\sphinxcrossref{\sphinxcode{\sphinxupquote{l\_rivers}}}}} = TRUE and
{\hyperref[\detokenize{namelists/jules_rivers.nml:JULES_RIVERS::i_river_vn}]{\sphinxcrossref{\sphinxcode{\sphinxupquote{i\_river\_vn}}}}} = 2.
&
\sphinxAtStartPar
np\_rivers
\\
\hline
\sphinxAtStartPar
\sphinxcode{\sphinxupquote{rfm\_substore\_rp}}
&
\sphinxAtStartPar
Sub\sphinxhyphen{}surface storage on river points (m$^{\text{3}}$).
Only available if {\hyperref[\detokenize{namelists/jules_rivers.nml:JULES_RIVERS::l_rivers}]{\sphinxcrossref{\sphinxcode{\sphinxupquote{l\_rivers}}}}} = TRUE and
{\hyperref[\detokenize{namelists/jules_rivers.nml:JULES_RIVERS::i_river_vn}]{\sphinxcrossref{\sphinxcode{\sphinxupquote{i\_river\_vn}}}}} = 2.
&
\sphinxAtStartPar
np\_rivers
\\
\hline
\sphinxAtStartPar
\sphinxcode{\sphinxupquote{rfm\_flowin\_rp}}
&
\sphinxAtStartPar
Surface inflow on river points (m$^{\text{3}}$ s$^{\text{\sphinxhyphen{}1}}$).
Only available if {\hyperref[\detokenize{namelists/jules_rivers.nml:JULES_RIVERS::l_rivers}]{\sphinxcrossref{\sphinxcode{\sphinxupquote{l\_rivers}}}}} = TRUE and
{\hyperref[\detokenize{namelists/jules_rivers.nml:JULES_RIVERS::i_river_vn}]{\sphinxcrossref{\sphinxcode{\sphinxupquote{i\_river\_vn}}}}} = 2.
&
\sphinxAtStartPar
np\_rivers
\\
\hline
\sphinxAtStartPar
\sphinxcode{\sphinxupquote{rfm\_bflowin\_rp}}
&
\sphinxAtStartPar
Sub\sphinxhyphen{}surface inflow on river points (m$^{\text{3}}$ s$^{\text{\sphinxhyphen{}1}}$).
Only available if {\hyperref[\detokenize{namelists/jules_rivers.nml:JULES_RIVERS::l_rivers}]{\sphinxcrossref{\sphinxcode{\sphinxupquote{l\_rivers}}}}} = TRUE and
{\hyperref[\detokenize{namelists/jules_rivers.nml:JULES_RIVERS::i_river_vn}]{\sphinxcrossref{\sphinxcode{\sphinxupquote{i\_river\_vn}}}}} = 2.
&
\sphinxAtStartPar
np\_rivers
\\
\hline
\sphinxAtStartPar
\sphinxcode{\sphinxupquote{rivers\_sto\_rp}}
&
\sphinxAtStartPar
River routing gridbox river storage (kg)
Only available if {\hyperref[\detokenize{namelists/jules_rivers.nml:JULES_RIVERS::l_rivers}]{\sphinxcrossref{\sphinxcode{\sphinxupquote{l\_rivers}}}}} = TRUE and
{\hyperref[\detokenize{namelists/jules_rivers.nml:JULES_RIVERS::i_river_vn}]{\sphinxcrossref{\sphinxcode{\sphinxupquote{i\_river\_vn}}}}} = 3.
&
\sphinxAtStartPar
np\_rivers
\\
\hline
\sphinxAtStartPar
\sphinxcode{\sphinxupquote{frac\_fplain\_rp}}
&
\sphinxAtStartPar
Overbank inundation area as a fraction of river routing gridcell area.
Only available if {\hyperref[\detokenize{namelists/jules_rivers.nml:JULES_OVERBANK::l_riv_overbank}]{\sphinxcrossref{\sphinxcode{\sphinxupquote{l\_riv\_overbank}}}}} = TRUE.
&
\sphinxAtStartPar
np\_rivers
\\
\hline\sphinxstartmulticolumn{3}%
\begin{varwidth}[t]{\sphinxcolwidth{3}{3}}
\sphinxAtStartPar
Output regridded to the JULES model grid
\par
\vskip-\baselineskip\vbox{\hbox{\strut}}\end{varwidth}%
\sphinxstopmulticolumn
\\
\hline
\sphinxAtStartPar
\sphinxcode{\sphinxupquote{rflow}}
&
\sphinxAtStartPar
Gridbox river flow rate (kg m$^{\text{\sphinxhyphen{}2}}$ s$^{\text{\sphinxhyphen{}1}}$).
Only available if {\hyperref[\detokenize{namelists/jules_rivers.nml:JULES_RIVERS::l_rivers}]{\sphinxcrossref{\sphinxcode{\sphinxupquote{l\_rivers}}}}} = TRUE.
&\\
\hline
\sphinxAtStartPar
\sphinxcode{\sphinxupquote{rrun}}
&
\sphinxAtStartPar
Gridbox runoff rate received by river routing routine (kg m$^{\text{\sphinxhyphen{}2}}$ s$^{\text{\sphinxhyphen{}1}}$).
Only available if {\hyperref[\detokenize{namelists/jules_rivers.nml:JULES_RIVERS::l_rivers}]{\sphinxcrossref{\sphinxcode{\sphinxupquote{l\_rivers}}}}} = TRUE
&\\
\hline
\sphinxAtStartPar
\sphinxcode{\sphinxupquote{frac\_fplain\_lp}}
&
\sphinxAtStartPar
Overbank inundation area as a fraction of gridcell area.
Only available if {\hyperref[\detokenize{namelists/jules_rivers.nml:JULES_OVERBANK::l_riv_overbank}]{\sphinxcrossref{\sphinxcode{\sphinxupquote{l\_riv\_overbank}}}}} = TRUE.
&\\
\hline
\end{tabulary}
\par
\sphinxattableend\end{savenotes}


\section{Snow}
\label{\detokenize{output-variables:snow}}

\begin{savenotes}\sphinxatlongtablestart\begin{longtable}[c]{|p{4.5cm}|p{8.8cm}|p{2.2cm}|}
\hline
\sphinxstyletheadfamily 
\sphinxAtStartPar
Name
&\sphinxstyletheadfamily 
\sphinxAtStartPar
Description
&\sphinxstyletheadfamily 
\sphinxAtStartPar
Dimensions
\\
\hline
\endfirsthead

\multicolumn{3}{c}%
{\makebox[0pt]{\sphinxtablecontinued{\tablename\ \thetable{} \textendash{} continued from previous page}}}\\
\hline
\sphinxstyletheadfamily 
\sphinxAtStartPar
Name
&\sphinxstyletheadfamily 
\sphinxAtStartPar
Description
&\sphinxstyletheadfamily 
\sphinxAtStartPar
Dimensions
\\
\hline
\endhead

\hline
\multicolumn{3}{r}{\makebox[0pt][r]{\sphinxtablecontinued{continues on next page}}}\\
\endfoot

\endlastfoot
\sphinxstartmulticolumn{3}%
\begin{varwidth}[t]{\sphinxcolwidth{3}{3}}
\sphinxAtStartPar
Snow state
\par
\vskip-\baselineskip\vbox{\hbox{\strut}}\end{varwidth}%
\sphinxstopmulticolumn
\\
\hline
\sphinxAtStartPar
\sphinxcode{\sphinxupquote{snow\_mass}}
&
\sphinxAtStartPar
Tile lying snow (total) (kg m$^{\text{\sphinxhyphen{}2}}$).
&
\sphinxAtStartPar
nsurft
\\
\hline
\sphinxAtStartPar
\sphinxcode{\sphinxupquote{snow\_mass\_gb}}
&
\sphinxAtStartPar
Gridbox snowmass (kg m$^{\text{\sphinxhyphen{}2}}$).
&
\sphinxAtStartPar
land+sea
\\
\hline
\sphinxAtStartPar
\sphinxcode{\sphinxupquote{snow\_depth}}
&
\sphinxAtStartPar
Tile snow depth (m).
&
\sphinxAtStartPar
nsurft
\\
\hline
\sphinxAtStartPar
\sphinxcode{\sphinxupquote{snow\_depth\_gb}}
&
\sphinxAtStartPar
Gridbox depth of snow (m).
&\\
\hline
\sphinxAtStartPar
\sphinxcode{\sphinxupquote{snow\_can}}
&
\sphinxAtStartPar
Tile snow on canopy (kg m$^{\text{\sphinxhyphen{}2}}$).
&
\sphinxAtStartPar
nsurft
\\
\hline
\sphinxAtStartPar
\sphinxcode{\sphinxupquote{snow\_can\_gb}}
&
\sphinxAtStartPar
Gridbox snow on canopy (kg m$^{\text{\sphinxhyphen{}2}}$).
&\\
\hline
\sphinxAtStartPar
\sphinxcode{\sphinxupquote{snow\_ground}}
&
\sphinxAtStartPar
Tile snow on ground (\sphinxcode{\sphinxupquote{snow\_tile}} or \sphinxcode{\sphinxupquote{snow\_grnd}} depending
on configuration) (kg m$^{\text{\sphinxhyphen{}2}}$).
&
\sphinxAtStartPar
nsurft
\\
\hline
\sphinxAtStartPar
\sphinxcode{\sphinxupquote{snow\_grnd\_gb}}
&
\sphinxAtStartPar
Gridbox average snow beneath canopy (snow\_grnd) (kg m$^{\text{\sphinxhyphen{}2}}$).
&\\
\hline
\sphinxAtStartPar
\sphinxcode{\sphinxupquote{snow\_grnd}}
&
\sphinxAtStartPar
Tile snow on ground below canopy (kg m$^{\text{\sphinxhyphen{}2}}$).
&
\sphinxAtStartPar
nsurft
\\
\hline
\sphinxAtStartPar
\sphinxcode{\sphinxupquote{snow\_grnd\_rho}}
&
\sphinxAtStartPar
Tile bulk density of snow on ground (kg m$^{\text{\sphinxhyphen{}3}}$).
&
\sphinxAtStartPar
nsurft
\\
\hline
\sphinxAtStartPar
\sphinxcode{\sphinxupquote{snow\_frac}}
&
\sphinxAtStartPar
Gridbox snow\sphinxhyphen{}covered fraction of land points (\sphinxhyphen{}).
&\\
\hline
\sphinxAtStartPar
\sphinxcode{\sphinxupquote{snow\_ice\_tile}}
&
\sphinxAtStartPar
Tile total frozen mass in snow on ground (kg m$^{\text{\sphinxhyphen{}2}}$).
Only available if {\hyperref[\detokenize{namelists/jules_snow.nml:JULES_SNOW::nsmax}]{\sphinxcrossref{\sphinxcode{\sphinxupquote{nsmax}}}}} \textgreater{} 0.
&
\sphinxAtStartPar
nsurft
\\
\hline
\sphinxAtStartPar
\sphinxcode{\sphinxupquote{snow\_ice\_gb}}
&
\sphinxAtStartPar
Gridbox frozen water in snowpack (kg m$^{\text{\sphinxhyphen{}2}}$).
Only available if {\hyperref[\detokenize{namelists/jules_snow.nml:JULES_SNOW::nsmax}]{\sphinxcrossref{\sphinxcode{\sphinxupquote{nsmax}}}}} \textgreater{} 0.
&\\
\hline
\sphinxAtStartPar
\sphinxcode{\sphinxupquote{snow\_liq\_tile}}
&
\sphinxAtStartPar
Tile total liquid mass in snow on ground (kg m$^{\text{\sphinxhyphen{}2}}$).
Only available if {\hyperref[\detokenize{namelists/jules_snow.nml:JULES_SNOW::nsmax}]{\sphinxcrossref{\sphinxcode{\sphinxupquote{nsmax}}}}} \textgreater{} 0.
&
\sphinxAtStartPar
nsurft
\\
\hline
\sphinxAtStartPar
\sphinxcode{\sphinxupquote{snow\_liq\_gb}}
&
\sphinxAtStartPar
Gridbox liquid water in snowpack (kg m$^{\text{\sphinxhyphen{}2}}$).
Only available if {\hyperref[\detokenize{namelists/jules_snow.nml:JULES_SNOW::nsmax}]{\sphinxcrossref{\sphinxcode{\sphinxupquote{nsmax}}}}} \textgreater{} 0.
&\\
\hline
\sphinxAtStartPar
\sphinxcode{\sphinxupquote{nsnow}}
&
\sphinxAtStartPar
Tile number of snow layers (\sphinxhyphen{}).
&
\sphinxAtStartPar
nsurft
\\
\hline
\sphinxAtStartPar
\sphinxcode{\sphinxupquote{rgrain}}
&
\sphinxAtStartPar
Tile snow surface grain size (μm).
&
\sphinxAtStartPar
nsurft
\\
\hline\sphinxstartmulticolumn{3}%
\begin{varwidth}[t]{\sphinxcolwidth{3}{3}}
\sphinxAtStartPar
Snow layer variables
\par
\vskip-\baselineskip\vbox{\hbox{\strut}}\end{varwidth}%
\sphinxstopmulticolumn
\\
\hline
\sphinxAtStartPar
\sphinxcode{\sphinxupquote{snow\_ds}}
&
\sphinxAtStartPar
Depth of each snow layer for each tile (m).
Only available if {\hyperref[\detokenize{namelists/jules_snow.nml:JULES_SNOW::nsmax}]{\sphinxcrossref{\sphinxcode{\sphinxupquote{nsmax}}}}} \textgreater{} 0.
&
\sphinxAtStartPar
nsurft,nsmax
\\
\hline
\sphinxAtStartPar
\sphinxcode{\sphinxupquote{snow\_ice}}
&
\sphinxAtStartPar
Mass of ice in each snow layer for each tile (kg m$^{\text{\sphinxhyphen{}2}}$).
Only available if {\hyperref[\detokenize{namelists/jules_snow.nml:JULES_SNOW::nsmax}]{\sphinxcrossref{\sphinxcode{\sphinxupquote{nsmax}}}}} \textgreater{} 0.
&
\sphinxAtStartPar
nsurft,nsmax
\\
\hline
\sphinxAtStartPar
\sphinxcode{\sphinxupquote{snow\_liq}}
&
\sphinxAtStartPar
Mass of liquid water in each snow layer for each tile (kg m$^{\text{\sphinxhyphen{}2}}$).
Only available if {\hyperref[\detokenize{namelists/jules_snow.nml:JULES_SNOW::nsmax}]{\sphinxcrossref{\sphinxcode{\sphinxupquote{nsmax}}}}} \textgreater{} 0.
&
\sphinxAtStartPar
nsurft,nsmax
\\
\hline
\sphinxAtStartPar
\sphinxcode{\sphinxupquote{tsnow}}
&
\sphinxAtStartPar
Temperature of each snow layer (K).
Only available if {\hyperref[\detokenize{namelists/jules_snow.nml:JULES_SNOW::nsmax}]{\sphinxcrossref{\sphinxcode{\sphinxupquote{nsmax}}}}} \textgreater{} 0.
&
\sphinxAtStartPar
nsurft,nsmax
\\
\hline
\sphinxAtStartPar
\sphinxcode{\sphinxupquote{rgrainl}}
&
\sphinxAtStartPar
Grain size in snow layers for each tile (μm).
Only available if {\hyperref[\detokenize{namelists/jules_snow.nml:JULES_SNOW::nsmax}]{\sphinxcrossref{\sphinxcode{\sphinxupquote{nsmax}}}}} \textgreater{} 0.
&
\sphinxAtStartPar
nsurft,nsmax
\\
\hline\sphinxstartmulticolumn{3}%
\begin{varwidth}[t]{\sphinxcolwidth{3}{3}}
\sphinxAtStartPar
Snow fluxes and rates of change
\par
\vskip-\baselineskip\vbox{\hbox{\strut}}\end{varwidth}%
\sphinxstopmulticolumn
\\
\hline
\sphinxAtStartPar
\sphinxcode{\sphinxupquote{snow\_melt}}
&
\sphinxAtStartPar
Tile snow melt rate (melt\_tile) (kg m$^{\text{\sphinxhyphen{}2}}$ s$^{\text{\sphinxhyphen{}1}}$).
&
\sphinxAtStartPar
nsurft
\\
\hline
\sphinxAtStartPar
\sphinxcode{\sphinxupquote{snow\_melt\_gb}}
&
\sphinxAtStartPar
Gridbox rate of snowmelt (kg m$^{\text{\sphinxhyphen{}2}}$ s$^{\text{\sphinxhyphen{}1}}$).
&\\
\hline
\sphinxAtStartPar
\sphinxcode{\sphinxupquote{snow\_can\_melt}}
&
\sphinxAtStartPar
Tile melt of snow on canopy (kg m$^{\text{\sphinxhyphen{}2}}$ s$^{\text{\sphinxhyphen{}1}}$).
&
\sphinxAtStartPar
nsurft
\\
\hline
\sphinxAtStartPar
\sphinxcode{\sphinxupquote{snice\_freez\_surft}}
&
\sphinxAtStartPar
Tile internal refreezing rate in snowpack (kg m$^{\text{\sphinxhyphen{}2}}$ s$^{\text{\sphinxhyphen{}1}}$).
&
\sphinxAtStartPar
nsurft
\\
\hline
\sphinxAtStartPar
\sphinxcode{\sphinxupquote{snice\_m\_surft}}
&
\sphinxAtStartPar
Tile total internal melt rate of snowpack (kg m$^{\text{\sphinxhyphen{}2}}$ s$^{\text{\sphinxhyphen{}1}}$).
&
\sphinxAtStartPar
nsurft
\\
\hline
\sphinxAtStartPar
\sphinxcode{\sphinxupquote{snice\_runoff\_surft}}
&
\sphinxAtStartPar
Tile net rate of liquid leaving snowpack on tiles (kg m$^{\text{\sphinxhyphen{}2}}$ s$^{\text{\sphinxhyphen{}1}}$).
&
\sphinxAtStartPar
nsurft
\\
\hline
\sphinxAtStartPar
\sphinxcode{\sphinxupquote{snice\_sicerate\_surft}}
&
\sphinxAtStartPar
Tile rate of change of solid mass in snowpack (kg m$^{\text{\sphinxhyphen{}2}}$ s$^{\text{\sphinxhyphen{}1}}$).
&
\sphinxAtStartPar
nsurft
\\
\hline
\sphinxAtStartPar
\sphinxcode{\sphinxupquote{snice\_sliqrate\_surft}}
&
\sphinxAtStartPar
Tile rate of change of liquid in snowpack (kg m$^{\text{\sphinxhyphen{}2}}$ s$^{\text{\sphinxhyphen{}1}}$).
&
\sphinxAtStartPar
nsurft
\\
\hline
\sphinxAtStartPar
\sphinxcode{\sphinxupquote{snice\_smb\_surft}}
&
\sphinxAtStartPar
Tile rate of change of snowpack mass (kg m$^{\text{\sphinxhyphen{}2}}$ s$^{\text{\sphinxhyphen{}1}}$).
&
\sphinxAtStartPar
nsurft
\\
\hline
\sphinxAtStartPar
\sphinxcode{\sphinxupquote{snow\_soil\_htf}}
&
\sphinxAtStartPar
Tile downward heat flux after snowpack to subsurface” (W m$^{\text{\sphinxhyphen{}2}}$).
&
\sphinxAtStartPar
nsurft
\\
\hline
\end{longtable}\sphinxatlongtableend\end{savenotes}


\section{Vegetation carbon and related fluxes}
\label{\detokenize{output-variables:vegetation-carbon-and-related-fluxes}}

\begin{savenotes}\sphinxatlongtablestart\begin{longtable}[c]{|p{3.5cm}|p{9.8cm}|p{2.2cm}|}
\hline
\sphinxstyletheadfamily 
\sphinxAtStartPar
Name
&\sphinxstyletheadfamily 
\sphinxAtStartPar
Description
&\sphinxstyletheadfamily 
\sphinxAtStartPar
Dimensions
\\
\hline
\endfirsthead

\multicolumn{3}{c}%
{\makebox[0pt]{\sphinxtablecontinued{\tablename\ \thetable{} \textendash{} continued from previous page}}}\\
\hline
\sphinxstyletheadfamily 
\sphinxAtStartPar
Name
&\sphinxstyletheadfamily 
\sphinxAtStartPar
Description
&\sphinxstyletheadfamily 
\sphinxAtStartPar
Dimensions
\\
\hline
\endhead

\hline
\multicolumn{3}{r}{\makebox[0pt][r]{\sphinxtablecontinued{continues on next page}}}\\
\endfoot

\endlastfoot

\sphinxAtStartPar
\sphinxcode{\sphinxupquote{c\_veg}}
&\begin{description}
\sphinxlineitem{PFT total carbon content of the vegetation at end of model timestep (kg C m$^{\text{\sphinxhyphen{}2}}$).}
\sphinxAtStartPar
(including leaf, wood and root carbon, both above and below ground)

\end{description}
&
\sphinxAtStartPar
npft
\\
\hline
\sphinxAtStartPar
\sphinxcode{\sphinxupquote{cv}}
&
\sphinxAtStartPar
Gridbox mean vegetation carbon at end of model timestep (kg m$^{\text{\sphinxhyphen{}2}}$).
&\\
\hline
\sphinxAtStartPar
\sphinxcode{\sphinxupquote{leafC}}
&
\sphinxAtStartPar
PFT carbon in leaf biomass (kg m$^{\text{\sphinxhyphen{}2}}$ ).
&
\sphinxAtStartPar
npft
\\
\hline
\sphinxAtStartPar
\sphinxcode{\sphinxupquote{rootC}}
&
\sphinxAtStartPar
PFT carbon in root biomass (kg m$^{\text{\sphinxhyphen{}2}}$ ).
&
\sphinxAtStartPar
npft
\\
\hline
\sphinxAtStartPar
\sphinxcode{\sphinxupquote{woodC}}
&
\sphinxAtStartPar
PFT carbon in woody biomass (kg m$^{\text{\sphinxhyphen{}2}}$ ).
&
\sphinxAtStartPar
npft
\\
\hline
\sphinxAtStartPar
\sphinxcode{\sphinxupquote{frac\_agr}}
&
\sphinxAtStartPar
Fractional area of agricultural land in each gridbox. If {\hyperref[\detokenize{namelists/jules_vegetation.nml:JULES_VEGETATION::l_trif_crop}]{\sphinxcrossref{\sphinxcode{\sphinxupquote{l\_trif\_crop}}}}}
is TRUE, frac\_agr is the fractional area of crop land in each gridbox.
&\\
\hline
\sphinxAtStartPar
\sphinxcode{\sphinxupquote{frac\_agr\_prev}}
&
\sphinxAtStartPar
Fractional area of agricultural land at the previous timestep.
&\\
\hline
\sphinxAtStartPar
\sphinxcode{\sphinxupquote{frac\_past}}
&
\sphinxAtStartPar
Fractional area of pasture land in each gridbox.
&\\
\hline
\sphinxAtStartPar
\sphinxcode{\sphinxupquote{plantNumDensity}}
&
\sphinxAtStartPar
Number density of plants  ( m$^{\text{\sphinxhyphen{}2}}$ ).
&
\sphinxAtStartPar
npft,nmasst
\\
\hline\sphinxstartmulticolumn{3}%
\begin{varwidth}[t]{\sphinxcolwidth{3}{3}}
\sphinxAtStartPar
Fractional cover, leaf area and turnover, and canopy height
\par
\vskip-\baselineskip\vbox{\hbox{\strut}}\end{varwidth}%
\sphinxstopmulticolumn
\\
\hline
\sphinxAtStartPar
\sphinxcode{\sphinxupquote{frac}}
&
\sphinxAtStartPar
Fractional cover of each surface type.
&
\sphinxAtStartPar
ntype
\\
\hline
\sphinxAtStartPar
\sphinxcode{\sphinxupquote{lai}}
&
\sphinxAtStartPar
PFT leaf area index (\sphinxhyphen{}).
&
\sphinxAtStartPar
npft
\\
\hline
\sphinxAtStartPar
\sphinxcode{\sphinxupquote{lai\_gb}}
&
\sphinxAtStartPar
Gridbox leaf area index (\sphinxhyphen{}).
&\\
\hline
\sphinxAtStartPar
\sphinxcode{\sphinxupquote{lai\_bal}}
&
\sphinxAtStartPar
PFT balanced leaf area index in sf\_stom (\sphinxhyphen{}).
&
\sphinxAtStartPar
npft
\\
\hline
\sphinxAtStartPar
\sphinxcode{\sphinxupquote{lai\_phen}}
&
\sphinxAtStartPar
PFT leaf area index after phenology (\sphinxhyphen{}).
&
\sphinxAtStartPar
npft
\\
\hline
\sphinxAtStartPar
\sphinxcode{\sphinxupquote{canht}}
&
\sphinxAtStartPar
PFT canopy height (m).
&
\sphinxAtStartPar
npft
\\
\hline
\sphinxAtStartPar
\sphinxcode{\sphinxupquote{g\_leaf}}
&
\sphinxAtStartPar
PFT leaf turnover rate ({[}360days{]}$^{\text{\sphinxhyphen{}1}}$).
&
\sphinxAtStartPar
npft
\\
\hline
\sphinxAtStartPar
\sphinxcode{\sphinxupquote{g\_leaf\_day}}
&
\sphinxAtStartPar
PFT mean leaf turnover rate for input to PHENOL ({[}360days{]}$^{\text{\sphinxhyphen{}1}}$).
&
\sphinxAtStartPar
npft
\\
\hline
\sphinxAtStartPar
\sphinxcode{\sphinxupquote{g\_leaf\_dr\_out}}
&
\sphinxAtStartPar
PFT mean leaf turnover rate for driving TRIFFID ({[}360days{]}$^{\text{\sphinxhyphen{}1}}$).
&
\sphinxAtStartPar
npft
\\
\hline
\sphinxAtStartPar
\sphinxcode{\sphinxupquote{g\_leaf\_phen}}
&
\sphinxAtStartPar
PFT mean leaf turnover rate over phenology period({[}360days{]}$^{\text{\sphinxhyphen{}1}}$).
&
\sphinxAtStartPar
npft
\\
\hline\sphinxstartmulticolumn{3}%
\begin{varwidth}[t]{\sphinxcolwidth{3}{3}}
\sphinxAtStartPar
GPP, NPP, respiration
\par
\vskip-\baselineskip\vbox{\hbox{\strut}}\end{varwidth}%
\sphinxstopmulticolumn
\\
\hline
\sphinxAtStartPar
\sphinxcode{\sphinxupquote{gpp}}
&
\sphinxAtStartPar
PFT gross primary productivity of biomass expressed as carbon
(kg C m$^{\text{\sphinxhyphen{}2}}$ s$^{\text{\sphinxhyphen{}1}}$).
&
\sphinxAtStartPar
npft
\\
\hline
\sphinxAtStartPar
\sphinxcode{\sphinxupquote{gpp\_gb}}
&
\sphinxAtStartPar
Gridbox gross primary productivity of biomass expressed as carbon
(kg C m$^{\text{\sphinxhyphen{}2}}$ s$^{\text{\sphinxhyphen{}1}}$).
&\\
\hline
\sphinxAtStartPar
\sphinxcode{\sphinxupquote{npp}}
&
\sphinxAtStartPar
PFT net primary productivity of biomass expressed as carbon prior to nitrogen limitation
(kg C m$^{\text{\sphinxhyphen{}2}}$ s$^{\text{\sphinxhyphen{}1}}$).
&
\sphinxAtStartPar
npft
\\
\hline
\sphinxAtStartPar
\sphinxcode{\sphinxupquote{npp\_n\_gb}}
&
\sphinxAtStartPar
Gridbox net primary productivity of biomass expressed as carbon after nitrogen limitation
(kg C m$^{\text{\sphinxhyphen{}2}}$ (360days)$^{\text{\sphinxhyphen{}1}}$).
&\\
\hline
\sphinxAtStartPar
\sphinxcode{\sphinxupquote{npp\_n}}
&
\sphinxAtStartPar
PFT net primary productivity of biomass expressed as carbon after nitrogen limitation
(kg C m$^{\text{\sphinxhyphen{}2}}$ (360days)$^{\text{\sphinxhyphen{}1}}$).
&
\sphinxAtStartPar
npft
\\
\hline
\sphinxAtStartPar
\sphinxcode{\sphinxupquote{npp\_dr\_out}}
&
\sphinxAtStartPar
PFT mean NPP of biomass expressed as carbon for driving TRIFFID
(kg C m$^{\text{\sphinxhyphen{}2}}$ (360days)$^{\text{\sphinxhyphen{}1}}$).
&
\sphinxAtStartPar
npft
\\
\hline
\sphinxAtStartPar
\sphinxcode{\sphinxupquote{resp\_p}}
&
\sphinxAtStartPar
PFT plant respiration carbon flux (kg m$^{\text{\sphinxhyphen{}2}}$ s$^{\text{\sphinxhyphen{}1}}$).
&
\sphinxAtStartPar
npft
\\
\hline
\sphinxAtStartPar
\sphinxcode{\sphinxupquote{resp\_p\_gb}}
&
\sphinxAtStartPar
Gridbox plant respiration carbon flux (kg m$^{\text{\sphinxhyphen{}2}}$ s$^{\text{\sphinxhyphen{}1}}$).
&\\
\hline
\sphinxAtStartPar
\sphinxcode{\sphinxupquote{resp\_w\_dr\_out}}
&
\sphinxAtStartPar
PFT mean wood respiration carbon flux for driving TRIFFID
(kg m$^{\text{\sphinxhyphen{}2}}$ (360days)$^{\text{\sphinxhyphen{}1}}$).
&
\sphinxAtStartPar
npft
\\
\hline
\sphinxAtStartPar
\sphinxcode{\sphinxupquote{resp\_l}}
&
\sphinxAtStartPar
PFT leaf respiration carbon flux (kg m$^{\text{\sphinxhyphen{}2}}$ s$^{\text{\sphinxhyphen{}1}}$).
&
\sphinxAtStartPar
npft
\\
\hline
\sphinxAtStartPar
\sphinxcode{\sphinxupquote{resp\_r}}
&
\sphinxAtStartPar
PFT root respiration carbon flux (kg m$^{\text{\sphinxhyphen{}2}}$ s$^{\text{\sphinxhyphen{}1}}$).
&
\sphinxAtStartPar
npft
\\
\hline
\sphinxAtStartPar
\sphinxcode{\sphinxupquote{resp\_w}}
&
\sphinxAtStartPar
PFT wood respiration carbon flux (kg m$^{\text{\sphinxhyphen{}2}}$ s$^{\text{\sphinxhyphen{}1}}$).
&
\sphinxAtStartPar
npft
\\
\hline\sphinxstartmulticolumn{3}%
\begin{varwidth}[t]{\sphinxcolwidth{3}{3}}
\sphinxAtStartPar
Litter carbon fluxes
\par
\vskip-\baselineskip\vbox{\hbox{\strut}}\end{varwidth}%
\sphinxstopmulticolumn
\\
\hline
\sphinxAtStartPar
\sphinxcode{\sphinxupquote{lit\_c}}
&
\sphinxAtStartPar
PFT carbon litter (kg m$^{\text{\sphinxhyphen{}2}}$ (360days)$^{\text{\sphinxhyphen{}1}}$).
&
\sphinxAtStartPar
npft
\\
\hline
\sphinxAtStartPar
\sphinxcode{\sphinxupquote{lit\_c\_mean}}
&
\sphinxAtStartPar
Gridbox mean carbon litter (kg m$^{\text{\sphinxhyphen{}2}}$ (360days)$^{\text{\sphinxhyphen{}1}}$).
&\\
\hline
\sphinxAtStartPar
\sphinxcode{\sphinxupquote{lit\_c\_ag}}
&
\sphinxAtStartPar
PFT carbon litter from LU/agriculture (kg C m$^{\text{\sphinxhyphen{}2}}$ (360days)$^{\text{\sphinxhyphen{}1}}$).
&
\sphinxAtStartPar
npft
\\
\hline
\sphinxAtStartPar
\sphinxcode{\sphinxupquote{lit\_c\_orig}}
&
\sphinxAtStartPar
PFT carbon litter including LU (kg C m$^{\text{\sphinxhyphen{}2}}$ (360days)$^{\text{\sphinxhyphen{}1}}$).
&
\sphinxAtStartPar
npft
\\
\hline
\sphinxAtStartPar
\sphinxcode{\sphinxupquote{leaf\_litC}}
&
\sphinxAtStartPar
PFT litter carbon due to leaf turnover (kg m$^{\text{\sphinxhyphen{}2}}$ )(360days)$^{\text{\sphinxhyphen{}1}}$).
&
\sphinxAtStartPar
npft
\\
\hline
\sphinxAtStartPar
\sphinxcode{\sphinxupquote{root\_litC}}
&
\sphinxAtStartPar
PFT litter carbon due to root turnover (kg m$^{\text{\sphinxhyphen{}2}}$ )(360days)$^{\text{\sphinxhyphen{}1}}$).
&
\sphinxAtStartPar
npft
\\
\hline
\sphinxAtStartPar
\sphinxcode{\sphinxupquote{wood\_litC}}
&
\sphinxAtStartPar
PFT litter carbon due to wood turnover (kg m$^{\text{\sphinxhyphen{}2}}$ )(360days)$^{\text{\sphinxhyphen{}1}}$).
&
\sphinxAtStartPar
npft
\\
\hline
\sphinxAtStartPar
\sphinxcode{\sphinxupquote{plant\_input\_c\_gb}}
&
\sphinxAtStartPar
Gridbox input of C to the soil by plant litterfall (kg m$^{\text{\sphinxhyphen{}2}}$ s$^{\text{\sphinxhyphen{}1}}$).
Only available with the ECOSSE soil model ({\hyperref[\detokenize{namelists/jules_soil_biogeochem.nml:JULES_SOIL_BIOGEOCHEM::soil_bgc_model}]{\sphinxcrossref{\sphinxcode{\sphinxupquote{soil\_bgc\_model}}}}} = 3).
&\\
\hline\sphinxstartmulticolumn{3}%
\begin{varwidth}[t]{\sphinxcolwidth{3}{3}}
\sphinxAtStartPar
Other carbon fluxes
\par
\vskip-\baselineskip\vbox{\hbox{\strut}}\end{varwidth}%
\sphinxstopmulticolumn
\\
\hline
\sphinxAtStartPar
\sphinxcode{\sphinxupquote{exudates}}
&
\sphinxAtStartPar
PFT exudates \sphinxhyphen{} excess carbon not assimilable into plant due lack of nitrogen
availability (kg C m$^{\text{\sphinxhyphen{}2}}$ (360days)$^{\text{\sphinxhyphen{}1}}$).
&
\sphinxAtStartPar
npft
\\
\hline
\sphinxAtStartPar
\sphinxcode{\sphinxupquote{exudates\_gb}}
&
\sphinxAtStartPar
Gridbox exudates: excess carbon not assimilable into plant due lack of nitrogen
(kg m$^{\text{\sphinxhyphen{}2}}$ (360days)$^{\text{\sphinxhyphen{}1}}$).
&\\
\hline
\sphinxAtStartPar
\sphinxcode{\sphinxupquote{pc\_s}}
&
\sphinxAtStartPar
PFT net carbon available for spreading in TRIFFID
(kg m$^{\text{\sphinxhyphen{}2}}$ (360 days)$^{\text{\sphinxhyphen{}1}}$).
&
\sphinxAtStartPar
npft
\\
\hline\sphinxstartmulticolumn{3}%
\begin{varwidth}[t]{\sphinxcolwidth{3}{3}}
\sphinxAtStartPar
Harvest, wood products and land use
\par
\vskip-\baselineskip\vbox{\hbox{\strut}}\end{varwidth}%
\sphinxstopmulticolumn
\\
\hline
\sphinxAtStartPar
\sphinxcode{\sphinxupquote{root\_abandon}}
&
\sphinxAtStartPar
PFT carbon flux from roots abandoned during landuse change to soil
(kg C m$^{\text{\sphinxhyphen{}2}}$ (360days)$^{\text{\sphinxhyphen{}1}}$).
&
\sphinxAtStartPar
npft
\\
\hline
\sphinxAtStartPar
\sphinxcode{\sphinxupquote{root\_abandon\_gb}}
&
\sphinxAtStartPar
Carbon from roots abandoned during landuse change to soil kg C m$^{\text{\sphinxhyphen{}2}}$ (360days)$^{\text{\sphinxhyphen{}1}}$).
&\\
\hline
\sphinxAtStartPar
\sphinxcode{\sphinxupquote{harvest}}
&
\sphinxAtStartPar
Flux of carbon to product pools due to harvest (kg C m$^{\text{\sphinxhyphen{}2}}$ (360days)$^{\text{\sphinxhyphen{}1}}$).
&
\sphinxAtStartPar
npft
\\
\hline
\sphinxAtStartPar
\sphinxcode{\sphinxupquote{harvest\_gb}}
&
\sphinxAtStartPar
Gridbox flux of carbon to product pools due to harvest (kg C m$^{\text{\sphinxhyphen{}2}}$ (360days)$^{\text{\sphinxhyphen{}1}}$).
&\\
\hline
\sphinxAtStartPar
\sphinxcode{\sphinxupquote{harvest\_biocrop}}
&
\sphinxAtStartPar
Flux of carbon to product pools due to biocrop harvest (kg C m$^{\text{\sphinxhyphen{}2}}$ (360days)$^{\text{\sphinxhyphen{}1}}$).
&
\sphinxAtStartPar
npft
\\
\hline
\sphinxAtStartPar
\sphinxcode{\sphinxupquote{harvest\_biocrop\_gb}}
&
\sphinxAtStartPar
Gridbox flux of carbon to product pools due to biocrop harvest (kg C m$^{\text{\sphinxhyphen{}2}}$
(360days)$^{\text{\sphinxhyphen{}1}}$).
&\\
\hline
\sphinxAtStartPar
\sphinxcode{\sphinxupquote{wood\_prod\_fast}}
&
\sphinxAtStartPar
Carbon content of the fast decay\sphinxhyphen{}rate wood product pool (kg m$^{\text{\sphinxhyphen{}2}}$).
&\\
\hline
\sphinxAtStartPar
\sphinxcode{\sphinxupquote{wood\_prod\_med}}
&
\sphinxAtStartPar
Carbon content of the medium decay\sphinxhyphen{}rate wood product pool (kg m$^{\text{\sphinxhyphen{}2}}$).
&\\
\hline
\sphinxAtStartPar
\sphinxcode{\sphinxupquote{wood\_prod\_slow}}
&
\sphinxAtStartPar
Carbon content of the slow decay\sphinxhyphen{}rate wood product pool (kg m$^{\text{\sphinxhyphen{}2}}$).
&\\
\hline
\sphinxAtStartPar
\sphinxcode{\sphinxupquote{WP\_fast\_in}}
&
\sphinxAtStartPar
Carbon flux from vegetation to the fast decay\sphinxhyphen{}rate wood product pool
(kg m$^{\text{\sphinxhyphen{}2}}$ {[}360days{]}$^{\text{\sphinxhyphen{}1}}$).
&\\
\hline
\sphinxAtStartPar
\sphinxcode{\sphinxupquote{WP\_med\_in}}
&
\sphinxAtStartPar
Carbon flux from vegetation to the medium decay\sphinxhyphen{}rate wood product pool
(kg m$^{\text{\sphinxhyphen{}2}}$ {[}360days{]}$^{\text{\sphinxhyphen{}1}}$).
&\\
\hline
\sphinxAtStartPar
\sphinxcode{\sphinxupquote{WP\_slow\_in}}
&
\sphinxAtStartPar
Carbon flux from vegetation to the slow decay\sphinxhyphen{}rate wood product pool
(kg m$^{\text{\sphinxhyphen{}2}}$ {[}360days{]}$^{\text{\sphinxhyphen{}1}}$).
&\\
\hline
\sphinxAtStartPar
\sphinxcode{\sphinxupquote{WP\_fast\_out}}
&
\sphinxAtStartPar
Carbon flux from the fast decay\sphinxhyphen{}rate wood product pool to atmosphere
(kg m$^{\text{\sphinxhyphen{}2}}$ {[}360days{]}$^{\text{\sphinxhyphen{}1}}$).
&\\
\hline
\sphinxAtStartPar
\sphinxcode{\sphinxupquote{WP\_med\_out}}
&
\sphinxAtStartPar
Carbon flux from the medium decay\sphinxhyphen{}rate wood product pool to atmosphere
(kg m$^{\text{\sphinxhyphen{}2}}$ {[}360days{]}$^{\text{\sphinxhyphen{}1}}$).
&\\
\hline
\sphinxAtStartPar
\sphinxcode{\sphinxupquote{WP\_slow\_out}}
&
\sphinxAtStartPar
Carbon flux from the slow decay\sphinxhyphen{}rate wood product pool to atmosphere
(kg m$^{\text{\sphinxhyphen{}2}}$ {[}360days{]}$^{\text{\sphinxhyphen{}1}}$).
&\\
\hline\sphinxstartmulticolumn{3}%
\begin{varwidth}[t]{\sphinxcolwidth{3}{3}}
\sphinxAtStartPar
Carbon conservation
\par
\vskip-\baselineskip\vbox{\hbox{\strut}}\end{varwidth}%
\sphinxstopmulticolumn
\\
\hline
\sphinxAtStartPar
\sphinxcode{\sphinxupquote{cnsrv\_carbon\_veg2}}
&
\sphinxAtStartPar
Error in land carbon conservation in veg2 routine (kg m\sphinxhyphen{}2)
&\\
\hline
\sphinxAtStartPar
\sphinxcode{\sphinxupquote{cnsrv\_carbon\_triffid}}
&
\sphinxAtStartPar
Error in land carbon conservation in triffid routine (kg m\sphinxhyphen{}2)
&\\
\hline
\sphinxAtStartPar
\sphinxcode{\sphinxupquote{cnsrv\_veg\_triffid}}
&
\sphinxAtStartPar
Error in vegetation carbon conservation in triffid routine (kg m\sphinxhyphen{}2)
&\\
\hline
\sphinxAtStartPar
\sphinxcode{\sphinxupquote{cnsrv\_soil\_triffid}}
&
\sphinxAtStartPar
Error in soil carbon conservation in triffid routine (kg m\sphinxhyphen{}2)
&\\
\hline
\sphinxAtStartPar
\sphinxcode{\sphinxupquote{cnsrv\_prod\_triffid}}
&
\sphinxAtStartPar
Error in wood product carbon conservation in triffid routine (kg m\sphinxhyphen{}2)
&\\
\hline\sphinxstartmulticolumn{3}%
\begin{varwidth}[t]{\sphinxcolwidth{3}{3}}
\sphinxAtStartPar
Thermal acclimation of photosynthesis
\par
\vskip-\baselineskip\vbox{\hbox{\strut}}\end{varwidth}%
\sphinxstopmulticolumn
\\
\hline
\sphinxAtStartPar
\sphinxcode{\sphinxupquote{t\_home\_gb}}
&
\sphinxAtStartPar
Long\sphinxhyphen{}term home temperature for C3 photosynthesis (K).
Only available if {\hyperref[\detokenize{namelists/jules_vegetation.nml:JULES_VEGETATION::photo_acclim_model}]{\sphinxcrossref{\sphinxcode{\sphinxupquote{photo\_acclim\_model}}}}} = 1 or 3.
&\\
\hline
\sphinxAtStartPar
\sphinxcode{\sphinxupquote{t\_growth\_gb}}
&
\sphinxAtStartPar
Short\sphinxhyphen{}term growth temperature for C3 photosynthesis (K).
Only available if {\hyperref[\detokenize{namelists/jules_vegetation.nml:JULES_VEGETATION::photo_acclim_model}]{\sphinxcrossref{\sphinxcode{\sphinxupquote{photo\_acclim\_model}}}}} = 2 or 3.
&\\
\hline
\end{longtable}\sphinxatlongtableend\end{savenotes}


\section{Vegetation nitrogen and related fluxes}
\label{\detokenize{output-variables:vegetation-nitrogen-and-related-fluxes}}

\begin{savenotes}\sphinxatlongtablestart\begin{longtable}[c]{|p{4.5cm}|p{8.8cm}|p{2.2cm}|}
\hline
\sphinxstyletheadfamily 
\sphinxAtStartPar
Name
&\sphinxstyletheadfamily 
\sphinxAtStartPar
Description
&\sphinxstyletheadfamily 
\sphinxAtStartPar
Dimensions
\\
\hline
\endfirsthead

\multicolumn{3}{c}%
{\makebox[0pt]{\sphinxtablecontinued{\tablename\ \thetable{} \textendash{} continued from previous page}}}\\
\hline
\sphinxstyletheadfamily 
\sphinxAtStartPar
Name
&\sphinxstyletheadfamily 
\sphinxAtStartPar
Description
&\sphinxstyletheadfamily 
\sphinxAtStartPar
Dimensions
\\
\hline
\endhead

\hline
\multicolumn{3}{r}{\makebox[0pt][r]{\sphinxtablecontinued{continues on next page}}}\\
\endfoot

\endlastfoot

\sphinxAtStartPar
\sphinxcode{\sphinxupquote{n\_veg}}
&
\sphinxAtStartPar
PFT plant nitrogen content N\_LEAF+N\_ROOT+N\_WOOD from carbon equivalents
(kg m$^{\text{\sphinxhyphen{}2}}$).
&
\sphinxAtStartPar
npft
\\
\hline
\sphinxAtStartPar
\sphinxcode{\sphinxupquote{n\_veg\_gb}}
&
\sphinxAtStartPar
Gridbox mean plant nitrogen content: n\_leaf+n\_root+n\_wood from carbon equivalents
(kg m$^{\text{\sphinxhyphen{}2}}$ ).
&\\
\hline
\sphinxAtStartPar
\sphinxcode{\sphinxupquote{n\_leaf}}
&
\sphinxAtStartPar
PFT leaf nitrogen scaled by LAI in sf\_stom (kg m$^{\text{\sphinxhyphen{}2}}$).
&
\sphinxAtStartPar
npft
\\
\hline
\sphinxAtStartPar
\sphinxcode{\sphinxupquote{n\_root}}
&
\sphinxAtStartPar
PFT root nitrogen scaled by LAI\_BAL in sf\_stom (kg m$^{\text{\sphinxhyphen{}2}}$).
&
\sphinxAtStartPar
npft
\\
\hline
\sphinxAtStartPar
\sphinxcode{\sphinxupquote{n\_stem}}
&
\sphinxAtStartPar
PFT stem nitrogen scaled by LAI in sf\_stom; scaled by LAI\_BAL if l\_stem\_resp\_fix=T
(kg m$^{\text{\sphinxhyphen{}2}}$).
&
\sphinxAtStartPar
npft
\\
\hline\sphinxstartmulticolumn{3}%
\begin{varwidth}[t]{\sphinxcolwidth{3}{3}}
\sphinxAtStartPar
Nitrogen fluxes
\par
\vskip-\baselineskip\vbox{\hbox{\strut}}\end{varwidth}%
\sphinxstopmulticolumn
\\
\hline
\sphinxAtStartPar
\sphinxcode{\sphinxupquote{deposition\_n}}
&
\sphinxAtStartPar
Nitrogen deposition (kg m$^{\text{\sphinxhyphen{}2}}$ s$^{\text{\sphinxhyphen{}1}}$).
&\\
\hline
\sphinxAtStartPar
\sphinxcode{\sphinxupquote{n\_demand}}
&
\sphinxAtStartPar
PFT total nitrogen demand
(kg m$^{\text{\sphinxhyphen{}2}}$ (360 days)$^{\text{\sphinxhyphen{}1}}$).
&
\sphinxAtStartPar
npft
\\
\hline
\sphinxAtStartPar
\sphinxcode{\sphinxupquote{n\_demand\_gb}}
&
\sphinxAtStartPar
Gridbox mean demand for nitrogen (kg m$^{\text{\sphinxhyphen{}2}}$ (360days)$^{\text{\sphinxhyphen{}1}}$).
&\\
\hline
\sphinxAtStartPar
\sphinxcode{\sphinxupquote{n\_fix}}
&
\sphinxAtStartPar
PFT fixed nitrogen (kg N m$^{\text{\sphinxhyphen{}2}}$ (360days)$^{\text{\sphinxhyphen{}1}}$).
&
\sphinxAtStartPar
npft
\\
\hline
\sphinxAtStartPar
\sphinxcode{\sphinxupquote{n\_fix\_gb}}
&
\sphinxAtStartPar
Gridbox mean nitrogen fixed by plants (kg m$^{\text{\sphinxhyphen{}2}}$ (360days)$^{\text{\sphinxhyphen{}1}}$).
&\\
\hline
\sphinxAtStartPar
\sphinxcode{\sphinxupquote{n\_uptake}}
&
\sphinxAtStartPar
PFT nitrogen taken up by plants (kg m$^{\text{\sphinxhyphen{}2}}$ (360 days)$^{\text{\sphinxhyphen{}1}}$).
&
\sphinxAtStartPar
npft
\\
\hline
\sphinxAtStartPar
\sphinxcode{\sphinxupquote{n\_uptake\_gb}}
&
\sphinxAtStartPar
Gridbox total nitrogen uptake by plants (kg N m$^{\text{\sphinxhyphen{}2}}$ (360days)$^{\text{\sphinxhyphen{}1}}$).
Only available if {\hyperref[\detokenize{namelists/jules_soil_biogeochem.nml:JULES_SOIL_BIOGEOCHEM::soil_bgc_model}]{\sphinxcrossref{\sphinxcode{\sphinxupquote{soil\_bgc\_model}}}}} = 2 or 3.
&\\
\hline
\sphinxAtStartPar
\sphinxcode{\sphinxupquote{n\_demand\_growth}}
&
\sphinxAtStartPar
PFT nitrogen demand for growth of existing plant biomass
(kg m$^{\text{\sphinxhyphen{}2}}$ (360 days)$^{\text{\sphinxhyphen{}1}}$).
&
\sphinxAtStartPar
npft
\\
\hline
\sphinxAtStartPar
\sphinxcode{\sphinxupquote{n\_uptake\_growth}}
&
\sphinxAtStartPar
PFT nitrogen taken up for growth of existing plant biomass
(kg m$^{\text{\sphinxhyphen{}2}}$ (360 days)$^{\text{\sphinxhyphen{}1}}$).
&
\sphinxAtStartPar
npft
\\
\hline
\sphinxAtStartPar
\sphinxcode{\sphinxupquote{n\_demand\_lit}}
&
\sphinxAtStartPar
PFT nitrogen demand of litter: nitrogen lost in leaf, wood and root biomass
(kg m$^{\text{\sphinxhyphen{}2}}$ (360 days)$^{\text{\sphinxhyphen{}1}}$).
&
\sphinxAtStartPar
npft
\\
\hline
\sphinxAtStartPar
\sphinxcode{\sphinxupquote{n\_demand\_spread}}
&
\sphinxAtStartPar
PFT nitrogen demand for spreading plants across gridbox
(kg m$^{\text{\sphinxhyphen{}2}}$ (360 days)$^{\text{\sphinxhyphen{}1}}$).
&
\sphinxAtStartPar
npft
\\
\hline
\sphinxAtStartPar
\sphinxcode{\sphinxupquote{n\_fertiliser}}
&
\sphinxAtStartPar
Nitrogen addition from fertiliser (kg N m$^{\text{\sphinxhyphen{}2}}$ (360days)$^{\text{\sphinxhyphen{}1}}$).
&
\sphinxAtStartPar
npft
\\
\hline\sphinxstartmulticolumn{3}%
\begin{varwidth}[t]{\sphinxcolwidth{3}{3}}
\sphinxAtStartPar
Nitrogen fluxes in litter, harvest and land use
\par
\vskip-\baselineskip\vbox{\hbox{\strut}}\end{varwidth}%
\sphinxstopmulticolumn
\\
\hline
\sphinxAtStartPar
\sphinxcode{\sphinxupquote{lit\_n\_t}}
&
\sphinxAtStartPar
Gridbox mean total nitrogen litter (kg m$^{\text{\sphinxhyphen{}2}}$ (360days)$^{\text{\sphinxhyphen{}1}}$).
(kg m$^{\text{\sphinxhyphen{}2}}$ s$^{\text{\sphinxhyphen{}1}}$).
&\\
\hline
\sphinxAtStartPar
\sphinxcode{\sphinxupquote{lit\_n}}
&
\sphinxAtStartPar
PFT nitrogen litter (kg N m$^{\text{\sphinxhyphen{}2}}$ (360days)$^{\text{\sphinxhyphen{}1}}$).
&
\sphinxAtStartPar
npft
\\
\hline
\sphinxAtStartPar
\sphinxcode{\sphinxupquote{leaf\_litN}}
&
\sphinxAtStartPar
PFT litter nitrogen due to leaf turnover (kg m$^{\text{\sphinxhyphen{}2}}$ )(360days)$^{\text{\sphinxhyphen{}1}}$).
&
\sphinxAtStartPar
npft
\\
\hline
\sphinxAtStartPar
\sphinxcode{\sphinxupquote{lit\_n\_ag}}
&
\sphinxAtStartPar
PFT nitrogen loss due to LU/agriculture (kg N m$^{\text{\sphinxhyphen{}2}}$ (360days)$^{\text{\sphinxhyphen{}1}}$).
&
\sphinxAtStartPar
npft
\\
\hline
\sphinxAtStartPar
\sphinxcode{\sphinxupquote{litterN}}
&
\sphinxAtStartPar
PFT local nitrogen litter production (kg N m$^{\text{\sphinxhyphen{}2}}$ (360days)$^{\text{\sphinxhyphen{}1}}$).
&
\sphinxAtStartPar
npft
\\
\hline
\sphinxAtStartPar
\sphinxcode{\sphinxupquote{lit\_n\_orig}}
&
\sphinxAtStartPar
PFT nitrogen litter including LU (kg N m$^{\text{\sphinxhyphen{}2}}$ (360days)$^{\text{\sphinxhyphen{}1}}$).
&
\sphinxAtStartPar
npft
\\
\hline
\sphinxAtStartPar
\sphinxcode{\sphinxupquote{root\_litN}}
&
\sphinxAtStartPar
PFT nitrogen lost as litter due to root turnover (kg m$^{\text{\sphinxhyphen{}2}}$ )(360days)$^{\text{\sphinxhyphen{}1}}$).
&
\sphinxAtStartPar
npft
\\
\hline
\sphinxAtStartPar
\sphinxcode{\sphinxupquote{wood\_litN}}
&
\sphinxAtStartPar
PFT litter nitrogen due to wood turnover (kg m$^{\text{\sphinxhyphen{}2}}$ )(360days)$^{\text{\sphinxhyphen{}1}}$).
&
\sphinxAtStartPar
npft
\\
\hline
\sphinxAtStartPar
\sphinxcode{\sphinxupquote{plant\_input\_n\_gb}}
&
\sphinxAtStartPar
Gridbox input of N to the soil by plant litterfall (kg m$^{\text{\sphinxhyphen{}2}}$ s$^{\text{\sphinxhyphen{}1}}$).
Only available with the ECOSSE soil model ({\hyperref[\detokenize{namelists/jules_soil_biogeochem.nml:JULES_SOIL_BIOGEOCHEM::soil_bgc_model}]{\sphinxcrossref{\sphinxcode{\sphinxupquote{soil\_bgc\_model}}}}} = 3).
&\\
\hline
\sphinxAtStartPar
\sphinxcode{\sphinxupquote{harvest\_n}}
&
\sphinxAtStartPar
flux of nitrogen to atmosphere due to harvest (kg N m$^{\text{\sphinxhyphen{}2}}$ (360days)$^{\text{\sphinxhyphen{}1}}$).
&
\sphinxAtStartPar
npft
\\
\hline
\sphinxAtStartPar
\sphinxcode{\sphinxupquote{harvest\_n\_gb}}
&
\sphinxAtStartPar
Gridbox flux of nitrogen to atmosphere due to harvest (kg N m$^{\text{\sphinxhyphen{}2}}$ (360days)$^{\text{\sphinxhyphen{}1}}$).
&\\
\hline
\sphinxAtStartPar
\sphinxcode{\sphinxupquote{harvest\_biocrop\_n}}
&
\sphinxAtStartPar
flux of nitrogen to product pools due to biocrop harvest (kg N m$^{\text{\sphinxhyphen{}2}}$
(360days)$^{\text{\sphinxhyphen{}1}}$).
&
\sphinxAtStartPar
npft
\\
\hline
\sphinxAtStartPar
\sphinxcode{\sphinxupquote{harvest\_biocrop\_n\_gb}}
&
\sphinxAtStartPar
Gridbox flux of nitrogen to product pools due to biocrop harvest (kg N m$^{\text{\sphinxhyphen{}2}}$
(360days)$^{\text{\sphinxhyphen{}1}}$).
&\\
\hline
\sphinxAtStartPar
\sphinxcode{\sphinxupquote{root\_abandon\_n}}
&
\sphinxAtStartPar
PFT nitrogen flux from roots abandoned during landuse change to soil
(kg N m$^{\text{\sphinxhyphen{}2}}$ (360days)$^{\text{\sphinxhyphen{}1}}$).
&
\sphinxAtStartPar
npft
\\
\hline
\sphinxAtStartPar
\sphinxcode{\sphinxupquote{root\_abandon\_n\_gb}}
&
\sphinxAtStartPar
Nitrogen from roots abandoned during landuse change to soil
kg N m$^{\text{\sphinxhyphen{}2}}$ (360days)$^{\text{\sphinxhyphen{}1}}$).
&\\
\hline\sphinxstartmulticolumn{3}%
\begin{varwidth}[t]{\sphinxcolwidth{3}{3}}
\sphinxAtStartPar
Nitrogen conservation
\par
\vskip-\baselineskip\vbox{\hbox{\strut}}\end{varwidth}%
\sphinxstopmulticolumn
\\
\hline
\sphinxAtStartPar
\sphinxcode{\sphinxupquote{cnsrv\_nitrogen\_triffd}}
&
\sphinxAtStartPar
Error in land nitrogen conservation in triffid routine (kg m\sphinxhyphen{}2)
&\\
\hline
\sphinxAtStartPar
\sphinxcode{\sphinxupquote{cnsrv\_vegN\_triffid}}
&
\sphinxAtStartPar
Error in vegetation nitrogen conservation in triffid routine (kg m\sphinxhyphen{}2)
&\\
\hline
\sphinxAtStartPar
\sphinxcode{\sphinxupquote{cnsrv\_soilN\_triffid}}
&
\sphinxAtStartPar
Error in soil nitrogen conservation in triffid routine (kg m\sphinxhyphen{}2)
&\\
\hline
\sphinxAtStartPar
\sphinxcode{\sphinxupquote{cnsrv\_n\_inorg\_triffid}}
&
\sphinxAtStartPar
Error in inorganic nitrogen conservation in triffid routine (kg m\sphinxhyphen{}2)
&\\
\hline
\end{longtable}\sphinxatlongtableend\end{savenotes}


\section{Soil carbon and related fluxes}
\label{\detokenize{output-variables:soil-carbon-and-related-fluxes}}

\begin{savenotes}\sphinxatlongtablestart\begin{longtable}[c]{|p{4.6cm}|p{8.7cm}|p{2.2cm}|}
\hline
\sphinxstyletheadfamily 
\sphinxAtStartPar
Name
&\sphinxstyletheadfamily 
\sphinxAtStartPar
Description
&\sphinxstyletheadfamily 
\sphinxAtStartPar
Dimensions
\\
\hline
\endfirsthead

\multicolumn{3}{c}%
{\makebox[0pt]{\sphinxtablecontinued{\tablename\ \thetable{} \textendash{} continued from previous page}}}\\
\hline
\sphinxstyletheadfamily 
\sphinxAtStartPar
Name
&\sphinxstyletheadfamily 
\sphinxAtStartPar
Description
&\sphinxstyletheadfamily 
\sphinxAtStartPar
Dimensions
\\
\hline
\endhead

\hline
\multicolumn{3}{r}{\makebox[0pt][r]{\sphinxtablecontinued{continues on next page}}}\\
\endfoot

\endlastfoot

\sphinxAtStartPar
\sphinxcode{\sphinxupquote{cs{[}\_soilt{]}}}
&
\sphinxAtStartPar
Carbon in each soil pool and each soil biogeochemistry layer
(kg m$^{\text{\sphinxhyphen{}2}}$).
&
\sphinxAtStartPar
cspool,cslayer
\\
\hline
\sphinxAtStartPar
\sphinxcode{\sphinxupquote{cs\_label}}
&
\sphinxAtStartPar
Labelled carbon in each soil pool and each soil biogeochemistry layer (kg m$^{\text{\sphinxhyphen{}2}}$).
Only available if {\hyperref[\detokenize{namelists/jules_soil_biogeochem.nml:JULES_SOIL_BIOGEOCHEM::l_label_frac_cs}]{\sphinxcrossref{\sphinxcode{\sphinxupquote{l\_label\_frac\_cs}}}}} = TRUE.
&
\sphinxAtStartPar
cspool,cslayer
\\
\hline
\sphinxAtStartPar
\sphinxcode{\sphinxupquote{cs\_gb}}
&
\sphinxAtStartPar
Gridbox total soil carbon (kg m$^{\text{\sphinxhyphen{}2}}$).
&\\
\hline
\sphinxAtStartPar
\sphinxcode{\sphinxupquote{cs\_label\_gb}}
&
\sphinxAtStartPar
Gridbox total labelled soil carbon (kg m$^{\text{\sphinxhyphen{}2}}$).
Only available if {\hyperref[\detokenize{namelists/jules_soil_biogeochem.nml:JULES_SOIL_BIOGEOCHEM::l_label_frac_cs}]{\sphinxcrossref{\sphinxcode{\sphinxupquote{l\_label\_frac\_cs}}}}} = TRUE.
&\\
\hline
\sphinxAtStartPar
\sphinxcode{\sphinxupquote{c\_bio\_gb}}
&
\sphinxAtStartPar
Gridbox soil carbon in biomass pool (kg m$^{\text{\sphinxhyphen{}2}}$).
Only available if {\hyperref[\detokenize{namelists/jules_soil_biogeochem.nml:JULES_SOIL_BIOGEOCHEM::soil_bgc_model}]{\sphinxcrossref{\sphinxcode{\sphinxupquote{soil\_bgc\_model}}}}} = 2 or 3.
&\\
\hline
\sphinxAtStartPar
\sphinxcode{\sphinxupquote{c\_dpm\_gb}}
&
\sphinxAtStartPar
Gridbox soil carbon in decomposable plant material pool (kg m$^{\text{\sphinxhyphen{}2}}$).
Only available if {\hyperref[\detokenize{namelists/jules_soil_biogeochem.nml:JULES_SOIL_BIOGEOCHEM::soil_bgc_model}]{\sphinxcrossref{\sphinxcode{\sphinxupquote{soil\_bgc\_model}}}}} = 2 or 3.
&\\
\hline
\sphinxAtStartPar
\sphinxcode{\sphinxupquote{c\_hum\_gb}}
&
\sphinxAtStartPar
Gridbox soil carbon in humus pool (kg m$^{\text{\sphinxhyphen{}2}}$).
Only available if {\hyperref[\detokenize{namelists/jules_soil_biogeochem.nml:JULES_SOIL_BIOGEOCHEM::soil_bgc_model}]{\sphinxcrossref{\sphinxcode{\sphinxupquote{soil\_bgc\_model}}}}} = 2 or 3.
&\\
\hline
\sphinxAtStartPar
\sphinxcode{\sphinxupquote{c\_rpm\_gb}}
&
\sphinxAtStartPar
Gridbox soil carbon in resistant plant material pool (kg m$^{\text{\sphinxhyphen{}2}}$).
Only available if {\hyperref[\detokenize{namelists/jules_soil_biogeochem.nml:JULES_SOIL_BIOGEOCHEM::soil_bgc_model}]{\sphinxcrossref{\sphinxcode{\sphinxupquote{soil\_bgc\_model}}}}} = 2 or 3.
&\\
\hline\sphinxstartmulticolumn{3}%
\begin{varwidth}[t]{\sphinxcolwidth{3}{3}}
\sphinxAtStartPar
Soil carbon fluxes
\par
\vskip-\baselineskip\vbox{\hbox{\strut}}\end{varwidth}%
\sphinxstopmulticolumn
\\
\hline
\sphinxAtStartPar
\sphinxcode{\sphinxupquote{co2\_soil\_gb}}
&
\sphinxAtStartPar
Gridbox C in CO$_{\text{2}}$ flux from soil to atmosphere (kg m$^{\text{\sphinxhyphen{}2}}$ s$^{\text{\sphinxhyphen{}1}}$).
Only available with the ECOSSE soil model ({\hyperref[\detokenize{namelists/jules_soil_biogeochem.nml:JULES_SOIL_BIOGEOCHEM::soil_bgc_model}]{\sphinxcrossref{\sphinxcode{\sphinxupquote{soil\_bgc\_model}}}}}
= 3).
&\\
\hline
\sphinxAtStartPar
\sphinxcode{\sphinxupquote{resp\_s}}
&
\sphinxAtStartPar
Respiration rate from each soil carbon pool each soil biogeochemistry layer
(kg m$^{\text{\sphinxhyphen{}2}}$ s$^{\text{\sphinxhyphen{}1}}$).
&
\sphinxAtStartPar
cspool,cslayer
\\
\hline
\sphinxAtStartPar
\sphinxcode{\sphinxupquote{resp\_label\_cs}}
&
\sphinxAtStartPar
Respiration rate from each labelled soil carbon pool each soil biogeochemistry layer
(kg m$^{\text{\sphinxhyphen{}2}}$ s$^{\text{\sphinxhyphen{}1}}$).
Only available if {\hyperref[\detokenize{namelists/jules_soil_biogeochem.nml:JULES_SOIL_BIOGEOCHEM::l_label_frac_cs}]{\sphinxcrossref{\sphinxcode{\sphinxupquote{l\_label\_frac\_cs}}}}} = TRUE.
&
\sphinxAtStartPar
cspool,cslayer
\\
\hline
\sphinxAtStartPar
\sphinxcode{\sphinxupquote{resp\_s\_gb}}
&
\sphinxAtStartPar
Gridbox total soil respiration carbon flux (kg m$^{\text{\sphinxhyphen{}2}}$ s$^{\text{\sphinxhyphen{}1}}$).
&\\
\hline
\sphinxAtStartPar
\sphinxcode{\sphinxupquote{resp\_label\_cs\_gb}}
&
\sphinxAtStartPar
Gridbox total labelled soil respiration carbon flux (kg m$^{\text{\sphinxhyphen{}2}}$ s$^{\text{\sphinxhyphen{}1}}$).
Only available if {\hyperref[\detokenize{namelists/jules_soil_biogeochem.nml:JULES_SOIL_BIOGEOCHEM::l_label_frac_cs}]{\sphinxcrossref{\sphinxcode{\sphinxupquote{l\_label\_frac\_cs}}}}} = TRUE.
&\\
\hline
\sphinxAtStartPar
\sphinxcode{\sphinxupquote{resp\_s\_to\_atmos\_gb}}
&
\sphinxAtStartPar
Respired carbon from soil carbon emitted to atmosphere (kg m$^{\text{\sphinxhyphen{}2}}$ s$^{\text{\sphinxhyphen{}1}}$).
&\\
\hline
\sphinxAtStartPar
\sphinxcode{\sphinxupquote{resp\_s\_dr\_out}}
&
\sphinxAtStartPar
Gridbox mean soil respiration carbon flux for driving TRIFFID
(kg m$^{\text{\sphinxhyphen{}2}}$ (360days)$^{\text{\sphinxhyphen{}1}}$)
This is the gross soil respiration; some of this carbon flux is from one soil carbon pool to
another.
&\\
\hline\sphinxstartmulticolumn{3}%
\begin{varwidth}[t]{\sphinxcolwidth{3}{3}}
\sphinxAtStartPar
Other soil carbon variables
\par
\vskip-\baselineskip\vbox{\hbox{\strut}}\end{varwidth}%
\sphinxstopmulticolumn
\\
\hline
\sphinxAtStartPar
\sphinxcode{\sphinxupquote{dpm\_ratio}}
&
\sphinxAtStartPar
Gridbox DPM:RPM ratio of overall litter input (:).
&\\
\hline
\sphinxAtStartPar
\sphinxcode{\sphinxupquote{fsth}}
&
\sphinxAtStartPar
Soil moisture modifier of soil respiration rate (\sphinxhyphen{}).
&
\sphinxAtStartPar
sm\_levels
\\
\hline
\sphinxAtStartPar
\sphinxcode{\sphinxupquote{ftemp}}
&
\sphinxAtStartPar
Temperature modifier of soil respiration rate (\sphinxhyphen{}).
&
\sphinxAtStartPar
sm\_levels
\\
\hline
\sphinxAtStartPar
\sphinxcode{\sphinxupquote{fprf}}
&
\sphinxAtStartPar
Modifier of soil respiration rate due to vegetation cover (\sphinxhyphen{}).
&\\
\hline
\sphinxAtStartPar
\sphinxcode{\sphinxupquote{soil\_CN}}
&
\sphinxAtStartPar
Soil C:N in each soil pool and each soil biogeochemistry layer
&
\sphinxAtStartPar
cspool,cslayer
\\
\hline
\sphinxAtStartPar
\sphinxcode{\sphinxupquote{soil\_cn\_gb}}
&
\sphinxAtStartPar
Gridbox total soil carbon : nitrogen ratio.
&\\
\hline\sphinxstartmulticolumn{3}%
\begin{varwidth}[t]{\sphinxcolwidth{3}{3}}
\sphinxAtStartPar
Soil methane variables
\par
\vskip-\baselineskip\vbox{\hbox{\strut}}\end{varwidth}%
\sphinxstopmulticolumn
\\
\hline
\sphinxAtStartPar
\sphinxcode{\sphinxupquote{fch4\_wetl}}
&
\sphinxAtStartPar
Gridbox scaled methane flux from wetland fraction using soil carbon as substrate if
ch4\_substrate=1, NPP as substrate if ch4\_substrate=2, or soil respiration as substrate if
ch4\_substrate=3 (10$^{\text{\sphinxhyphen{}9}}$ kg m$^{\text{\sphinxhyphen{}2}}$ s$^{\text{\sphinxhyphen{}1}}$).
&\\
\hline
\sphinxAtStartPar
\sphinxcode{\sphinxupquote{fch4\_wetl\_cs{[}\_soilt{]}}}
&
\sphinxAtStartPar
Gridbox methane flux from wetland fraction using soil carbon as substrate
(kg m$^{\text{\sphinxhyphen{}2}}$ s$^{\text{\sphinxhyphen{}1}}$).
&\\
\hline
\sphinxAtStartPar
\sphinxcode{\sphinxupquote{fch4\_wetl\_npp{[}\_soilt{]}}}
&
\sphinxAtStartPar
Gridbox methane flux from wetland fraction using NPP as substrate
(kg m$^{\text{\sphinxhyphen{}2}}$ s$^{\text{\sphinxhyphen{}1}}$).
&\\
\hline
\sphinxAtStartPar
\sphinxcode{\sphinxupquote{fch4\_wetl\_resps{[}\_soilt{]}}}
&
\sphinxAtStartPar
Gridbox methane flux from wetland fraction using soil respiration as substrate
(kg m$^{\text{\sphinxhyphen{}2}}$ s$^{\text{\sphinxhyphen{}1}}$).
&\\
\hline
\sphinxAtStartPar
\sphinxcode{\sphinxupquote{substr\_ch4}}
&
\sphinxAtStartPar
Carbon in substrate pool used by methanogens for each soil methane layer
(kg m$^{\text{\sphinxhyphen{}2}}$).
Only available if {\hyperref[\detokenize{namelists/jules_soil_biogeochem.nml:JULES_SOIL_BIOGEOCHEM::l_ch4_microbe}]{\sphinxcrossref{\sphinxcode{\sphinxupquote{l\_ch4\_microbe}}}}} = TRUE.
&
\sphinxAtStartPar
ch4layer
\\
\hline
\sphinxAtStartPar
\sphinxcode{\sphinxupquote{mic\_ch4}}
&
\sphinxAtStartPar
Carbon in methanogenic biomass for each soil methane layer (kg m$^{\text{\sphinxhyphen{}2}}$).
Only available if {\hyperref[\detokenize{namelists/jules_soil_biogeochem.nml:JULES_SOIL_BIOGEOCHEM::l_ch4_microbe}]{\sphinxcrossref{\sphinxcode{\sphinxupquote{l\_ch4\_microbe}}}}} = TRUE.
&
\sphinxAtStartPar
ch4layer
\\
\hline
\sphinxAtStartPar
\sphinxcode{\sphinxupquote{mic\_act\_ch4}}
&
\sphinxAtStartPar
Activity level of methanogenic biomass for each soil methane layer (\sphinxhyphen{}).
Only available if {\hyperref[\detokenize{namelists/jules_soil_biogeochem.nml:JULES_SOIL_BIOGEOCHEM::l_ch4_microbe}]{\sphinxcrossref{\sphinxcode{\sphinxupquote{l\_ch4\_microbe}}}}} = TRUE.
&
\sphinxAtStartPar
ch4layer
\\
\hline
\sphinxAtStartPar
\sphinxcode{\sphinxupquote{acclim\_ch4}}
&
\sphinxAtStartPar
Acclimation factor for methanogenic processes in each soil methane layer (\sphinxhyphen{}).
Only available if {\hyperref[\detokenize{namelists/jules_soil_biogeochem.nml:JULES_SOIL_BIOGEOCHEM::l_ch4_microbe}]{\sphinxcrossref{\sphinxcode{\sphinxupquote{l\_ch4\_microbe}}}}} = TRUE.
&
\sphinxAtStartPar
ch4layer
\\
\hline
\end{longtable}\sphinxatlongtableend\end{savenotes}


\section{Soil nitrogen and related fluxes}
\label{\detokenize{output-variables:soil-nitrogen-and-related-fluxes}}

\begin{savenotes}\sphinxatlongtablestart\begin{longtable}[c]{|p{4.0cm}|p{9.3cm}|p{2.2cm}|}
\hline
\sphinxstyletheadfamily 
\sphinxAtStartPar
Name
&\sphinxstyletheadfamily 
\sphinxAtStartPar
Description
&\sphinxstyletheadfamily 
\sphinxAtStartPar
Dimensions
\\
\hline
\endfirsthead

\multicolumn{3}{c}%
{\makebox[0pt]{\sphinxtablecontinued{\tablename\ \thetable{} \textendash{} continued from previous page}}}\\
\hline
\sphinxstyletheadfamily 
\sphinxAtStartPar
Name
&\sphinxstyletheadfamily 
\sphinxAtStartPar
Description
&\sphinxstyletheadfamily 
\sphinxAtStartPar
Dimensions
\\
\hline
\endhead

\hline
\multicolumn{3}{r}{\makebox[0pt][r]{\sphinxtablecontinued{continues on next page}}}\\
\endfoot

\endlastfoot

\sphinxAtStartPar
\sphinxcode{\sphinxupquote{n\_soil\_gb}}
&
\sphinxAtStartPar
Gridbox total soil nitrogen (organic and inorganic) (kg m$^{\text{\sphinxhyphen{}2}}$).
Only available if {\hyperref[\detokenize{namelists/jules_soil_biogeochem.nml:JULES_SOIL_BIOGEOCHEM::soil_bgc_model}]{\sphinxcrossref{\sphinxcode{\sphinxupquote{soil\_bgc\_model}}}}} = 2 or 3.
&\\
\hline
\sphinxAtStartPar
\sphinxcode{\sphinxupquote{ns}}
&
\sphinxAtStartPar
Gridbox organic nitrogen in each soil pool and each soil biogeochemistry
layer (kg m$^{\text{\sphinxhyphen{}2}}$).
&
\sphinxAtStartPar
cspool,cslayer
\\
\hline
\sphinxAtStartPar
\sphinxcode{\sphinxupquote{ns\_gb}}
&
\sphinxAtStartPar
Gridbox soil organic nitrogen (kg m$^{\text{\sphinxhyphen{}2}}$).
Only available if {\hyperref[\detokenize{namelists/jules_soil_biogeochem.nml:JULES_SOIL_BIOGEOCHEM::soil_bgc_model}]{\sphinxcrossref{\sphinxcode{\sphinxupquote{soil\_bgc\_model}}}}} = 2 or 3.
&\\
\hline
\sphinxAtStartPar
\sphinxcode{\sphinxupquote{n\_bio\_gb}}
&
\sphinxAtStartPar
Gridbox soil nitrogen in biomass pool (kg m$^{\text{\sphinxhyphen{}2}}$).
Only available if {\hyperref[\detokenize{namelists/jules_soil_biogeochem.nml:JULES_SOIL_BIOGEOCHEM::soil_bgc_model}]{\sphinxcrossref{\sphinxcode{\sphinxupquote{soil\_bgc\_model}}}}} = 2 or 3.
&\\
\hline
\sphinxAtStartPar
\sphinxcode{\sphinxupquote{n\_dpm\_gb}}
&
\sphinxAtStartPar
Gridbox soil nitrogen in decomposable plant material pool (kg m$^{\text{\sphinxhyphen{}2}}$).
Only available if {\hyperref[\detokenize{namelists/jules_soil_biogeochem.nml:JULES_SOIL_BIOGEOCHEM::soil_bgc_model}]{\sphinxcrossref{\sphinxcode{\sphinxupquote{soil\_bgc\_model}}}}} = 2 or 3.
&\\
\hline
\sphinxAtStartPar
\sphinxcode{\sphinxupquote{n\_rpm\_gb}}
&
\sphinxAtStartPar
Gridbox soil nitrogen in resistant plant material pool (kg m$^{\text{\sphinxhyphen{}2}}$).
Only available if {\hyperref[\detokenize{namelists/jules_soil_biogeochem.nml:JULES_SOIL_BIOGEOCHEM::soil_bgc_model}]{\sphinxcrossref{\sphinxcode{\sphinxupquote{soil\_bgc\_model}}}}} = 2 or 3.
&\\
\hline
\sphinxAtStartPar
\sphinxcode{\sphinxupquote{n\_hum\_gb}}
&
\sphinxAtStartPar
Gridbox soil nitrogen in humus pool (kg m$^{\text{\sphinxhyphen{}2}}$).
Only available if {\hyperref[\detokenize{namelists/jules_soil_biogeochem.nml:JULES_SOIL_BIOGEOCHEM::soil_bgc_model}]{\sphinxcrossref{\sphinxcode{\sphinxupquote{soil\_bgc\_model}}}}} = 2 or 3.
&\\
\hline
\sphinxAtStartPar
\sphinxcode{\sphinxupquote{n\_amm\_gb}}
&
\sphinxAtStartPar
Gridbox soil nitrogen in ammonium pool (kg m$^{\text{\sphinxhyphen{}2}}$).
Only available if {\hyperref[\detokenize{namelists/jules_soil_biogeochem.nml:JULES_SOIL_BIOGEOCHEM::soil_bgc_model}]{\sphinxcrossref{\sphinxcode{\sphinxupquote{soil\_bgc\_model}}}}} = 3.
&\\
\hline
\sphinxAtStartPar
\sphinxcode{\sphinxupquote{n\_nit\_gb}}
&
\sphinxAtStartPar
Gridbox soil ammonium in ammonium pool (kg m$^{\text{\sphinxhyphen{}2}}$).
Only available if {\hyperref[\detokenize{namelists/jules_soil_biogeochem.nml:JULES_SOIL_BIOGEOCHEM::soil_bgc_model}]{\sphinxcrossref{\sphinxcode{\sphinxupquote{soil\_bgc\_model}}}}} = 3.
&\\
\hline
\sphinxAtStartPar
\sphinxcode{\sphinxupquote{n\_inorg\_gb}}
&
\sphinxAtStartPar
Gridbox soil inorganic nitrogen (kg m$^{\text{\sphinxhyphen{}2}}$).
Only available if {\hyperref[\detokenize{namelists/jules_soil_biogeochem.nml:JULES_SOIL_BIOGEOCHEM::soil_bgc_model}]{\sphinxcrossref{\sphinxcode{\sphinxupquote{soil\_bgc\_model}}}}} = 2 or 3.
&\\
\hline
\sphinxAtStartPar
\sphinxcode{\sphinxupquote{n\_inorg\_avail\_pft}}
&
\sphinxAtStartPar
PFT inorganic nitrogen pool that is available for plant uptake
for each soil biogeochemistry layer (kg m$^{\text{\sphinxhyphen{}2}}$).
&
\sphinxAtStartPar
npft
\\
\hline\sphinxstartmulticolumn{3}%
\begin{varwidth}[t]{\sphinxcolwidth{3}{3}}
\sphinxAtStartPar
Soil nitrogen fluxes
\par
\vskip-\baselineskip\vbox{\hbox{\strut}}\end{varwidth}%
\sphinxstopmulticolumn
\\
\hline
\sphinxAtStartPar
\sphinxcode{\sphinxupquote{immob\_n}}
&
\sphinxAtStartPar
Soil nitrogen immobilisation in each soil pool
and each soil biogeochemistry layer (kg m$^{\text{\sphinxhyphen{}2}}$ (360days)$^{\text{\sphinxhyphen{}1}}$).
&
\sphinxAtStartPar
cspool,cslayer
\\
\hline
\sphinxAtStartPar
\sphinxcode{\sphinxupquote{immob\_n\_gb}}
&
\sphinxAtStartPar
Gridbox mean soil nitrogen immobilisation (kg m$^{\text{\sphinxhyphen{}2}}$ (360days)$^{\text{\sphinxhyphen{}1}}$).
&\\
\hline
\sphinxAtStartPar
\sphinxcode{\sphinxupquote{immob\_n\_pot}}
&
\sphinxAtStartPar
Soil potential nitrogen immobilisation in each soil pool
and each soil biogeochemistry layer (kg m$^{\text{\sphinxhyphen{}2}}$ (360days)$^{\text{\sphinxhyphen{}1}}$).
&
\sphinxAtStartPar
cspool,cslayer
\\
\hline
\sphinxAtStartPar
\sphinxcode{\sphinxupquote{immob\_n\_pot\_gb}}
&
\sphinxAtStartPar
Gridbox mean potential soil nitrogen immobilisation (kg m$^{\text{\sphinxhyphen{}2}}$ (360days)$^{\text{\sphinxhyphen{}1}}$).
&\\
\hline
\sphinxAtStartPar
\sphinxcode{\sphinxupquote{minl\_n\_pot}}
&
\sphinxAtStartPar
Soil potential nitrogen mineralisation in each pool soil pool
and each soil biogeochemistry layer (kg m$^{\text{\sphinxhyphen{}2}}$ (360days)$^{\text{\sphinxhyphen{}1}}$).
&
\sphinxAtStartPar
cspool,cslayer
\\
\hline
\sphinxAtStartPar
\sphinxcode{\sphinxupquote{minl\_n\_pot\_gb}}
&
\sphinxAtStartPar
Gridbox mean potential soil nitrogen mineralisation (kg m$^{\text{\sphinxhyphen{}2}}$ (360days)$^{\text{\sphinxhyphen{}1}}$).
&\\
\hline
\sphinxAtStartPar
\sphinxcode{\sphinxupquote{minl\_n}}
&
\sphinxAtStartPar
Soil nitrogen mineralisation in each pool soil pool
and each soil biogeochemistry layer (kg m$^{\text{\sphinxhyphen{}2}}$ (360days)$^{\text{\sphinxhyphen{}1}}$).
&
\sphinxAtStartPar
cspool,cslayer
\\
\hline
\sphinxAtStartPar
\sphinxcode{\sphinxupquote{minl\_gb}}
&
\sphinxAtStartPar
Gridbox mean soil nitrogen mineralisation (kg m$^{\text{\sphinxhyphen{}2}}$ (360days)$^{\text{\sphinxhyphen{}1}}$).
&\\
\hline
\sphinxAtStartPar
\sphinxcode{\sphinxupquote{n\_miner\_gb}}
&
\sphinxAtStartPar
Gribox rate of net mineralisation of soil N (kg m$^{\text{\sphinxhyphen{}2}}$ s$^{\text{\sphinxhyphen{}1}}$).
Only available with the ECOSSE soil model ({\hyperref[\detokenize{namelists/jules_soil_biogeochem.nml:JULES_SOIL_BIOGEOCHEM::soil_bgc_model}]{\sphinxcrossref{\sphinxcode{\sphinxupquote{soil\_bgc\_model}}}}}
= 3).
&\\
\hline
\sphinxAtStartPar
\sphinxcode{\sphinxupquote{n\_nitrif\_gb}}
&
\sphinxAtStartPar
Gridbox mean rate of nitrification, expressed as N (kg m$^{\text{\sphinxhyphen{}2}}$ s$^{\text{\sphinxhyphen{}1}}$).
&\\
\hline
\sphinxAtStartPar
\sphinxcode{\sphinxupquote{n\_denitrif\_gb}}
&
\sphinxAtStartPar
Gridbox mean rate of denitrification, expressed as N (kg m$^{\text{\sphinxhyphen{}2}}$ s$^{\text{\sphinxhyphen{}1}}$).
&\\
\hline
\sphinxAtStartPar
\sphinxcode{\sphinxupquote{n2o\_nitrif\_gb}}
&
\sphinxAtStartPar
Gridbox mean N in N$_{\text{2}}$O lost during nitrification, including partial nitrification
(kg m$^{\text{\sphinxhyphen{}2}}$ s$^{\text{\sphinxhyphen{}1}}$).
&\\
\hline
\sphinxAtStartPar
\sphinxcode{\sphinxupquote{n2o\_part\_nitrif\_gb}}
&
\sphinxAtStartPar
Gridbox mean N in N$_{\text{2}}$O lost by partial nitrification (kg m$^{\text{\sphinxhyphen{}2}}$ s$^{\text{\sphinxhyphen{}1}}$).
&\\
\hline
\sphinxAtStartPar
\sphinxcode{\sphinxupquote{n2\_denitrif\_gb}}
&
\sphinxAtStartPar
Gridbox mean N in N$_{\text{2}}$ lost from soil during denitrification
(kg m$^{\text{\sphinxhyphen{}2}}$ s$^{\text{\sphinxhyphen{}1}}$).
&\\
\hline
\sphinxAtStartPar
\sphinxcode{\sphinxupquote{n2o\_denitrif\_gb}}
&
\sphinxAtStartPar
Gridbox mean N in N$_{\text{2}}$O lost during denitrification (kg m$^{\text{\sphinxhyphen{}2}}$ s$^{\text{\sphinxhyphen{}1}}$).
&\\
\hline
\sphinxAtStartPar
\sphinxcode{\sphinxupquote{n2o\_soil\_gb}}
&
\sphinxAtStartPar
Gridbox mean N in N$_{\text{2}}$O flux from soil to atmosphere (kg m$^{\text{\sphinxhyphen{}2}}$ s$^{\text{\sphinxhyphen{}1}}$).
&\\
\hline
\sphinxAtStartPar
\sphinxcode{\sphinxupquote{no\_soil\_gb}}
&
\sphinxAtStartPar
Gridbox mean N in NO flux from soil to atmosphere (kg m$^{\text{\sphinxhyphen{}2}}$ s$^{\text{\sphinxhyphen{}1}}$).
&\\
\hline
\sphinxAtStartPar
\sphinxcode{\sphinxupquote{n\_fertiliser\_gb}}
&
\sphinxAtStartPar
Gridbox nitrogen addition from fertiliser (kg N m$^{\text{\sphinxhyphen{}2}}$ (360days)$^{\text{\sphinxhyphen{}1}}$).
&\\
\hline
\sphinxAtStartPar
\sphinxcode{\sphinxupquote{n\_gas\_gb}}
&
\sphinxAtStartPar
Gridbox mean mineralised nitrogen gas emissions (kg m$^{\text{\sphinxhyphen{}2}}$ (360days)$^{\text{\sphinxhyphen{}1}}$).
&\\
\hline
\sphinxAtStartPar
\sphinxcode{\sphinxupquote{n\_leach}}
&
\sphinxAtStartPar
Gridbox leached nitrogen loss term (kg N m$^{\text{\sphinxhyphen{}2}}$ s$^{\text{\sphinxhyphen{}1}}$).
&\\
\hline
\sphinxAtStartPar
\sphinxcode{\sphinxupquote{n\_loss}}
&
\sphinxAtStartPar
Gridbox nitrogen loss term (fixed fraction of n\_inorg) (kg N m$^{\text{\sphinxhyphen{}2}}$
(360days)$^{\text{\sphinxhyphen{}1}}$).
&\\
\hline
\end{longtable}\sphinxatlongtableend\end{savenotes}


\section{Fire}
\label{\detokenize{output-variables:fire}}

\begin{savenotes}\sphinxatlongtablestart\begin{longtable}[c]{|p{4.0cm}|p{9.3cm}|p{2.2cm}|}
\hline
\sphinxstyletheadfamily 
\sphinxAtStartPar
Name
&\sphinxstyletheadfamily 
\sphinxAtStartPar
Description
&\sphinxstyletheadfamily 
\sphinxAtStartPar
Dimensions
\\
\hline
\endfirsthead

\multicolumn{3}{c}%
{\makebox[0pt]{\sphinxtablecontinued{\tablename\ \thetable{} \textendash{} continued from previous page}}}\\
\hline
\sphinxstyletheadfamily 
\sphinxAtStartPar
Name
&\sphinxstyletheadfamily 
\sphinxAtStartPar
Description
&\sphinxstyletheadfamily 
\sphinxAtStartPar
Dimensions
\\
\hline
\endhead

\hline
\multicolumn{3}{r}{\makebox[0pt][r]{\sphinxtablecontinued{continues on next page}}}\\
\endfoot

\endlastfoot
\sphinxstartmulticolumn{3}%
\begin{varwidth}[t]{\sphinxcolwidth{3}{3}}
\sphinxAtStartPar
Fire indices
\par
\vskip-\baselineskip\vbox{\hbox{\strut}}\end{varwidth}%
\sphinxstopmulticolumn
\\
\hline
\sphinxAtStartPar
\sphinxcode{\sphinxupquote{fire\_mcarthur}}
&
\sphinxAtStartPar
McArthur Forest Fire Danger Index (No units)
&\\
\hline
\sphinxAtStartPar
\sphinxcode{\sphinxupquote{fire\_canadian}}
&
\sphinxAtStartPar
Canadian Fire Weather Index (No units).
&\\
\hline
\sphinxAtStartPar
\sphinxcode{\sphinxupquote{fire\_canadian\_ffmc}}
&
\sphinxAtStartPar
Canadian Fire Weather Index\sphinxhyphen{} Fine Fuel Moisture Code (No units).
&\\
\hline
\sphinxAtStartPar
\sphinxcode{\sphinxupquote{fire\_canadian\_dmc}}
&
\sphinxAtStartPar
Canadian Fire Weather Index\sphinxhyphen{} Duff Moisture Code (No units).
&\\
\hline
\sphinxAtStartPar
\sphinxcode{\sphinxupquote{fire\_canadian\_dc}}
&
\sphinxAtStartPar
Canadian Fire Weather Index\sphinxhyphen{} Drought Code (No units).
&\\
\hline
\sphinxAtStartPar
\sphinxcode{\sphinxupquote{fire\_canadian\_isi}}
&
\sphinxAtStartPar
Canadian Fire Weather Index\sphinxhyphen{} Initial Spread Index (No units).
&\\
\hline
\sphinxAtStartPar
\sphinxcode{\sphinxupquote{fire\_canadian\_bui}}
&
\sphinxAtStartPar
Canadian Fire Weather Index\sphinxhyphen{} Build\sphinxhyphen{}up Index (No units).
&\\
\hline
\sphinxAtStartPar
\sphinxcode{\sphinxupquote{fire\_nesterov}}
&
\sphinxAtStartPar
Nesterov Fire Index (No units).
&\\
\hline\sphinxstartmulticolumn{3}%
\begin{varwidth}[t]{\sphinxcolwidth{3}{3}}
\sphinxAtStartPar
Burnt area
\par
\vskip-\baselineskip\vbox{\hbox{\strut}}\end{varwidth}%
\sphinxstopmulticolumn
\\
\hline
\sphinxAtStartPar
\sphinxcode{\sphinxupquote{burnt\_area}}
&
\sphinxAtStartPar
PFT burnt area fraction (s$^{\text{\sphinxhyphen{}1}}$).
&
\sphinxAtStartPar
npft
\\
\hline
\sphinxAtStartPar
\sphinxcode{\sphinxupquote{burnt\_area\_gb}}
&
\sphinxAtStartPar
Gridbox mean burnt area fraction (s$^{\text{\sphinxhyphen{}1}}$).
&\\
\hline\sphinxstartmulticolumn{3}%
\begin{varwidth}[t]{\sphinxcolwidth{3}{3}}
\sphinxAtStartPar
Fire emissions
\par
\vskip-\baselineskip\vbox{\hbox{\strut}}\end{varwidth}%
\sphinxstopmulticolumn
\\
\hline
\sphinxAtStartPar
\sphinxcode{\sphinxupquote{emitted\_carbon}}
&
\sphinxAtStartPar
PFT emitted carbon flux (kg m$^{\text{\sphinxhyphen{}2}}$ s$^{\text{\sphinxhyphen{}1}}$).
&
\sphinxAtStartPar
npft
\\
\hline
\sphinxAtStartPar
\sphinxcode{\sphinxupquote{emitted\_carbon\_gb}}
&
\sphinxAtStartPar
Gridbox mean emitted carbon flux (kg m$^{\text{\sphinxhyphen{}2}}$ s$^{\text{\sphinxhyphen{}1}}$).
&\\
\hline
\sphinxAtStartPar
\sphinxcode{\sphinxupquote{emitted\_carbon\_DPM}}
&
\sphinxAtStartPar
Decomposable Plant Material emitted carbon flux (kg m$^{\text{\sphinxhyphen{}2}}$ s$^{\text{\sphinxhyphen{}1}}$).
&\\
\hline
\sphinxAtStartPar
\sphinxcode{\sphinxupquote{emitted\_carbon\_RPM}}
&
\sphinxAtStartPar
Resistant Plant Material emitted carbon flux (kg m$^{\text{\sphinxhyphen{}2}}$ s$^{\text{\sphinxhyphen{}1}}$).
&\\
\hline
\sphinxAtStartPar
\sphinxcode{\sphinxupquote{fire\_em\_CO2\_gb}}
&
\sphinxAtStartPar
Gridbox mean fire CO2 emission flux (kg m$^{\text{\sphinxhyphen{}2}}$ s$^{\text{\sphinxhyphen{}1}}$).
&\\
\hline
\sphinxAtStartPar
\sphinxcode{\sphinxupquote{fire\_em\_CO2\_DPM}}
&
\sphinxAtStartPar
Decomposable Plant Material fire CO2 emission flux (kg m$^{\text{\sphinxhyphen{}2}}$ s$^{\text{\sphinxhyphen{}1}}$).
&\\
\hline
\sphinxAtStartPar
\sphinxcode{\sphinxupquote{fire\_em\_CO2\_RPM}}
&
\sphinxAtStartPar
Resistant Plant Material fire CO2 emission flux (kg m$^{\text{\sphinxhyphen{}2}}$ s$^{\text{\sphinxhyphen{}1}}$).
&\\
\hline
\sphinxAtStartPar
\sphinxcode{\sphinxupquote{fire\_em\_CO\_gb}}
&
\sphinxAtStartPar
Gridbox mean fire CO emission flux (kg m$^{\text{\sphinxhyphen{}2}}$ s$^{\text{\sphinxhyphen{}1}}$).
&\\
\hline
\sphinxAtStartPar
\sphinxcode{\sphinxupquote{fire\_em\_CO\_DPM}}
&
\sphinxAtStartPar
Decomposable Plant Material fire CO emission flux (kg m$^{\text{\sphinxhyphen{}2}}$ s$^{\text{\sphinxhyphen{}1}}$).
&\\
\hline
\sphinxAtStartPar
\sphinxcode{\sphinxupquote{fire\_em\_CO\_RPM}}
&
\sphinxAtStartPar
Resistant Plant Material fire CO emission flux (kg m$^{\text{\sphinxhyphen{}2}}$ s$^{\text{\sphinxhyphen{}1}}$).
&\\
\hline
\sphinxAtStartPar
\sphinxcode{\sphinxupquote{fire\_em\_CH4\_gb}}
&
\sphinxAtStartPar
Gridbox mean fire CH4 emission flux (kg m$^{\text{\sphinxhyphen{}2}}$ s$^{\text{\sphinxhyphen{}1}}$).
&\\
\hline
\sphinxAtStartPar
\sphinxcode{\sphinxupquote{fire\_em\_CH4\_DPM}}
&
\sphinxAtStartPar
Decomposable Plant Material fire CH4 emission flux (kg m$^{\text{\sphinxhyphen{}2}}$ s$^{\text{\sphinxhyphen{}1}}$).
&\\
\hline
\sphinxAtStartPar
\sphinxcode{\sphinxupquote{fire\_em\_CH4\_RPM}}
&
\sphinxAtStartPar
Resistant Plant Material fire CH4 emission flux (kg m$^{\text{\sphinxhyphen{}2}}$ s$^{\text{\sphinxhyphen{}1}}$).
&\\
\hline
\sphinxAtStartPar
\sphinxcode{\sphinxupquote{fire\_em\_NOx\_gb}}
&
\sphinxAtStartPar
Gridbox mean fire NOx emission flux (kg m$^{\text{\sphinxhyphen{}2}}$ s$^{\text{\sphinxhyphen{}1}}$).
&\\
\hline
\sphinxAtStartPar
\sphinxcode{\sphinxupquote{fire\_em\_NOx\_DPM}}
&
\sphinxAtStartPar
Decomposable Plant Material fire NOx emission flux (kg m$^{\text{\sphinxhyphen{}2}}$ s$^{\text{\sphinxhyphen{}1}}$).
&\\
\hline
\sphinxAtStartPar
\sphinxcode{\sphinxupquote{fire\_em\_NOx\_RPM}}
&
\sphinxAtStartPar
Resistant Plant Material fire NOx emission flux (kg m$^{\text{\sphinxhyphen{}2}}$ s$^{\text{\sphinxhyphen{}1}}$).
&\\
\hline
\sphinxAtStartPar
\sphinxcode{\sphinxupquote{fire\_em\_SO2\_gb}}
&
\sphinxAtStartPar
Gridbox mean fire SO2 emission flux (kg m$^{\text{\sphinxhyphen{}2}}$ s$^{\text{\sphinxhyphen{}1}}$).
&\\
\hline
\sphinxAtStartPar
\sphinxcode{\sphinxupquote{fire\_em\_SO2\_DPM}}
&
\sphinxAtStartPar
Decomposable Plant Material fire SO2 emission flux (kg m$^{\text{\sphinxhyphen{}2}}$ s$^{\text{\sphinxhyphen{}1}}$).
&\\
\hline
\sphinxAtStartPar
\sphinxcode{\sphinxupquote{fire\_em\_SO2\_RPM}}
&
\sphinxAtStartPar
Resistant Plant Material fire SO2 emission flux (kg m$^{\text{\sphinxhyphen{}2}}$ s$^{\text{\sphinxhyphen{}1}}$).
&\\
\hline
\sphinxAtStartPar
\sphinxcode{\sphinxupquote{fire\_em\_OC\_gb}}
&
\sphinxAtStartPar
Gridbox mean fire OC emission flux (kg m$^{\text{\sphinxhyphen{}2}}$ s$^{\text{\sphinxhyphen{}1}}$).
&\\
\hline
\sphinxAtStartPar
\sphinxcode{\sphinxupquote{fire\_em\_OC\_DPM}}
&
\sphinxAtStartPar
Decomposable Plant Material fire OC emission flux (kg m$^{\text{\sphinxhyphen{}2}}$ s$^{\text{\sphinxhyphen{}1}}$).
&\\
\hline
\sphinxAtStartPar
\sphinxcode{\sphinxupquote{fire\_em\_OC\_RPM}}
&
\sphinxAtStartPar
Resistant Plant Material fire OC emission flux (kg m$^{\text{\sphinxhyphen{}2}}$ s$^{\text{\sphinxhyphen{}1}}$).
&\\
\hline
\sphinxAtStartPar
\sphinxcode{\sphinxupquote{fire\_em\_BC\_gb}}
&
\sphinxAtStartPar
Gridbox mean fire BC emission flux (kg m$^{\text{\sphinxhyphen{}2}}$ s$^{\text{\sphinxhyphen{}1}}$).
&\\
\hline
\sphinxAtStartPar
\sphinxcode{\sphinxupquote{fire\_em\_BC\_DPM}}
&
\sphinxAtStartPar
Decomposable Plant Material fire BC emission flux (kg m$^{\text{\sphinxhyphen{}2}}$ s$^{\text{\sphinxhyphen{}1}}$).
&\\
\hline
\sphinxAtStartPar
\sphinxcode{\sphinxupquote{fire\_em\_BC\_RPM}}
&
\sphinxAtStartPar
Resistant Plant Material fire BC emission flux (kg m$^{\text{\sphinxhyphen{}2}}$ s$^{\text{\sphinxhyphen{}1}}$).
&\\
\hline
\sphinxAtStartPar
\sphinxcode{\sphinxupquote{fire\_em\_CO2}}
&
\sphinxAtStartPar
PFT fire CO2 emission flux (kg m$^{\text{\sphinxhyphen{}2}}$ s$^{\text{\sphinxhyphen{}1}}$).
&
\sphinxAtStartPar
npft
\\
\hline
\sphinxAtStartPar
\sphinxcode{\sphinxupquote{fire\_em\_CO}}
&
\sphinxAtStartPar
PFT fire CO emission flux (kg m$^{\text{\sphinxhyphen{}2}}$ s$^{\text{\sphinxhyphen{}1}}$).
&
\sphinxAtStartPar
npft
\\
\hline
\sphinxAtStartPar
\sphinxcode{\sphinxupquote{fire\_em\_CH4}}
&
\sphinxAtStartPar
PFT fire CH4 emission flux (kg m$^{\text{\sphinxhyphen{}2}}$ s$^{\text{\sphinxhyphen{}1}}$).
&
\sphinxAtStartPar
npft
\\
\hline
\sphinxAtStartPar
\sphinxcode{\sphinxupquote{fire\_em\_NOx}}
&
\sphinxAtStartPar
PFT fire NOx emission flux (kg m$^{\text{\sphinxhyphen{}2}}$ s$^{\text{\sphinxhyphen{}1}}$).
&
\sphinxAtStartPar
npft
\\
\hline
\sphinxAtStartPar
\sphinxcode{\sphinxupquote{fire\_em\_SO2}}
&
\sphinxAtStartPar
PFT fire SO2 emission flux (kg m$^{\text{\sphinxhyphen{}2}}$ s$^{\text{\sphinxhyphen{}1}}$).
&
\sphinxAtStartPar
npft
\\
\hline
\sphinxAtStartPar
\sphinxcode{\sphinxupquote{fire\_em\_OC}}
&
\sphinxAtStartPar
PFT fire OC emission flux (kg m$^{\text{\sphinxhyphen{}2}}$ s$^{\text{\sphinxhyphen{}1}}$).
&
\sphinxAtStartPar
npft
\\
\hline
\sphinxAtStartPar
\sphinxcode{\sphinxupquote{fire\_em\_BC}}
&
\sphinxAtStartPar
PFT fire BC emission flux (kg m$^{\text{\sphinxhyphen{}2}}$ s$^{\text{\sphinxhyphen{}1}}$).
&
\sphinxAtStartPar
npft
\\
\hline
\end{longtable}\sphinxatlongtableend\end{savenotes}


\section{Crops}
\label{\detokenize{output-variables:crops}}
\sphinxAtStartPar
These variables are only available when {\hyperref[\detokenize{namelists/jules_surface_types.nml:JULES_SURFACE_TYPES::ncpft}]{\sphinxcrossref{\sphinxcode{\sphinxupquote{ncpft}}}}} \textgreater{} 0.


\begin{savenotes}\sphinxattablestart
\centering
\begin{tabulary}{\linewidth}[t]{|p{3.5cm}|p{9.8cm}|p{2.2cm}|}
\hline
\sphinxstyletheadfamily 
\sphinxAtStartPar
Name
&\sphinxstyletheadfamily 
\sphinxAtStartPar
Description
&\sphinxstyletheadfamily 
\sphinxAtStartPar
Dimensions
\\
\hline
\sphinxAtStartPar
\sphinxcode{\sphinxupquote{croplai}}
&
\sphinxAtStartPar
Crop PFT leaf area index (LAI).
&
\sphinxAtStartPar
ncpft
\\
\hline
\sphinxAtStartPar
\sphinxcode{\sphinxupquote{cropcanht}}
&
\sphinxAtStartPar
Crop PFT canopy height (m).
&
\sphinxAtStartPar
ncpft
\\
\hline
\sphinxAtStartPar
\sphinxcode{\sphinxupquote{cropleafc}}
&
\sphinxAtStartPar
Crop PFT carbon in leaf parts (kg m$^{\text{\sphinxhyphen{}2}}$).
&
\sphinxAtStartPar
ncpft
\\
\hline
\sphinxAtStartPar
\sphinxcode{\sphinxupquote{cropstemc}}
&
\sphinxAtStartPar
Crop PFT carbon in stem parts (kg m$^{\text{\sphinxhyphen{}2}}$).
&
\sphinxAtStartPar
ncpft
\\
\hline
\sphinxAtStartPar
\sphinxcode{\sphinxupquote{croprootc}}
&
\sphinxAtStartPar
Crop PFT root carbon pool (kg m$^{\text{\sphinxhyphen{}2}}$).
&
\sphinxAtStartPar
ncpft
\\
\hline
\sphinxAtStartPar
\sphinxcode{\sphinxupquote{cropreservec}}
&
\sphinxAtStartPar
Crop PFT carbon in stem reserve pool (kg m$^{\text{\sphinxhyphen{}2}}$).
&
\sphinxAtStartPar
ncpft
\\
\hline
\sphinxAtStartPar
\sphinxcode{\sphinxupquote{cropharvc}}
&
\sphinxAtStartPar
Crop PFT carbon in harvested parts (kg m$^{\text{\sphinxhyphen{}2}}$).
&
\sphinxAtStartPar
ncpft
\\
\hline
\sphinxAtStartPar
\sphinxcode{\sphinxupquote{cropyield}}
&
\sphinxAtStartPar
Crop PFT yield carbon (kg m$^{\text{\sphinxhyphen{}2}}$).

\sphinxAtStartPar
\sphinxcode{\sphinxupquote{cropyield}} is zero in every timestep where there is no harvest, so this
variable will mostly be used with {\hyperref[\detokenize{namelists/output.nml:JULES_OUTPUT_PROFILE::output_type}]{\sphinxcrossref{\sphinxcode{\sphinxupquote{output\_type}}}}}
set to ‘A’ or ‘X’.
&
\sphinxAtStartPar
ncpft
\\
\hline
\sphinxAtStartPar
\sphinxcode{\sphinxupquote{cropdvi}}
&
\sphinxAtStartPar
Crop PFT developmental index (DVI).
&
\sphinxAtStartPar
ncpft
\\
\hline
\sphinxAtStartPar
\sphinxcode{\sphinxupquote{cropsowdate}}
&
\sphinxAtStartPar
PFT sowing date (1.0 to 365.0).

\sphinxAtStartPar
If {\hyperref[\detokenize{namelists/jules_vegetation.nml:JULES_VEGETATION::l_prescsow}]{\sphinxcrossref{\sphinxcode{\sphinxupquote{l\_prescsow}}}}} = FALSE then this will always be 0.
&
\sphinxAtStartPar
ncpft
\\
\hline
\sphinxAtStartPar
\sphinxcode{\sphinxupquote{harvest\_counter}}
&
\sphinxAtStartPar
1 in a timestep where crop is harvested, 0 otherwise.

\sphinxAtStartPar
When used with {\hyperref[\detokenize{namelists/output.nml:JULES_OUTPUT_PROFILE::output_type}]{\sphinxcrossref{\sphinxcode{\sphinxupquote{output\_type}}}}} = ‘A’, will count
the number of harvests since the beginning of the run.
&
\sphinxAtStartPar
ncpft
\\
\hline
\sphinxAtStartPar
\sphinxcode{\sphinxupquote{harvest\_trigger}}
&
\sphinxAtStartPar
Indicates which condition triggered the harvest:
\begin{enumerate}
\sphinxsetlistlabels{\arabic}{enumi}{enumii}{}{.}%
\setcounter{enumi}{-1}
\item {} 
\sphinxAtStartPar
No harvest in timestep.

\item {} 
\sphinxAtStartPar
Crop reached maturity (DVI=2).

\item {} 
\sphinxAtStartPar
Crop leaf area index became too high (LAI\textgreater{}15).

\item {} 
\sphinxAtStartPar
Crop has flowered (DVI\textgreater{}1) and temperature in the second soil layer dropped
below {\hyperref[\detokenize{namelists/crop_params.nml:JULES_CROPPARM::t_mort_io}]{\sphinxcrossref{\sphinxcode{\sphinxupquote{t\_mort\_io}}}}}.

\item {} 
\sphinxAtStartPar
Crop has flowered (DVI\textgreater{}1), \sphinxcode{\sphinxupquote{cropharvc}} \textgreater{} 0 and
(\sphinxcode{\sphinxupquote{croprootc}} + \sphinxcode{\sphinxupquote{cropleafc}} + \sphinxcode{\sphinxupquote{cropstemc}} + \sphinxcode{\sphinxupquote{cropreservec}}) \textless{}
{\hyperref[\detokenize{namelists/crop_params.nml:JULES_CROPPARM::initial_carbon_io}]{\sphinxcrossref{\sphinxcode{\sphinxupquote{initial\_carbon\_io}}}}}.

\item {} 
\sphinxAtStartPar
{\hyperref[\detokenize{namelists/jules_vegetation.nml:JULES_VEGETATION::l_prescsow}]{\sphinxcrossref{\sphinxcode{\sphinxupquote{l\_prescsow}}}}} = T, crop has emerged (DVI\textgreater{}=0), and
the day of year is the day before the sowing date.

\end{enumerate}
&
\sphinxAtStartPar
ncpft
\\
\hline
\end{tabulary}
\par
\sphinxattableend\end{savenotes}


\section{Trace gas concentrations and fluxes}
\label{\detokenize{output-variables:trace-gas-concentrations-and-fluxes}}

\begin{savenotes}\sphinxattablestart
\centering
\begin{tabulary}{\linewidth}[t]{|p{3.0cm}|p{10.3cm}|p{2.2cm}|}
\hline
\sphinxstyletheadfamily 
\sphinxAtStartPar
Name
&\sphinxstyletheadfamily 
\sphinxAtStartPar
Description
&\sphinxstyletheadfamily 
\sphinxAtStartPar
Dimensions
\\
\hline
\sphinxAtStartPar
\sphinxcode{\sphinxupquote{co2\_mmr}}
&
\sphinxAtStartPar
Concentration of atmospheric CO2, expressed as a mass mixing ratio.
&\\
\hline
\sphinxAtStartPar
\sphinxcode{\sphinxupquote{flux\_o3\_stom}}
&
\sphinxAtStartPar
PFT flux of O3 to stomata (nmol m$^{\text{\sphinxhyphen{}2}}$ s$^{\text{\sphinxhyphen{}1}}$).
&
\sphinxAtStartPar
npft
\\
\hline
\sphinxAtStartPar
\sphinxcode{\sphinxupquote{o3\_exp\_fac}}
&
\sphinxAtStartPar
PFT ozone exposure factor.
&
\sphinxAtStartPar
npft
\\
\hline
\sphinxAtStartPar
\sphinxcode{\sphinxupquote{acetone}}
&
\sphinxAtStartPar
PFT monoterpene emission flux (kg m$^{\text{\sphinxhyphen{}2}}$ s$^{\text{\sphinxhyphen{}1}}$).
&
\sphinxAtStartPar
npft
\\
\hline
\sphinxAtStartPar
\sphinxcode{\sphinxupquote{acetone\_gb}}
&
\sphinxAtStartPar
Gridbox mean monoterpene emission flux (kg m$^{\text{\sphinxhyphen{}2}}$ s$^{\text{\sphinxhyphen{}1}}$).
&\\
\hline
\sphinxAtStartPar
\sphinxcode{\sphinxupquote{isoprene}}
&
\sphinxAtStartPar
PFT isoprene emission flux (kg m$^{\text{\sphinxhyphen{}2}}$ s$^{\text{\sphinxhyphen{}1}}$).
&
\sphinxAtStartPar
npft
\\
\hline
\sphinxAtStartPar
\sphinxcode{\sphinxupquote{isoprene\_gb}}
&
\sphinxAtStartPar
Gridbox mean isoprene emission flux (kg m$^{\text{\sphinxhyphen{}2}}$ s$^{\text{\sphinxhyphen{}1}}$).
&\\
\hline
\sphinxAtStartPar
\sphinxcode{\sphinxupquote{methanol}}
&
\sphinxAtStartPar
PFT methanol emission flux (kg m$^{\text{\sphinxhyphen{}2}}$ s$^{\text{\sphinxhyphen{}1}}$).
&
\sphinxAtStartPar
npft
\\
\hline
\sphinxAtStartPar
\sphinxcode{\sphinxupquote{methanol\_gb}}
&
\sphinxAtStartPar
Gridbox mean methanol emission flux (kg m$^{\text{\sphinxhyphen{}2}}$ s$^{\text{\sphinxhyphen{}1}}$).
&\\
\hline
\sphinxAtStartPar
\sphinxcode{\sphinxupquote{terpene}}
&
\sphinxAtStartPar
PFT monoterpene emission flux (kg m$^{\text{\sphinxhyphen{}2}}$ s$^{\text{\sphinxhyphen{}1}}$).
&
\sphinxAtStartPar
npft
\\
\hline
\sphinxAtStartPar
\sphinxcode{\sphinxupquote{terpene\_gb}}
&
\sphinxAtStartPar
Gridbox mean monoterpene emission flux (kg m$^{\text{\sphinxhyphen{}2}}$ s$^{\text{\sphinxhyphen{}1}}$).
&\\
\hline
\end{tabulary}
\par
\sphinxattableend\end{savenotes}


\section{Water resources}
\label{\detokenize{output-variables:water-resources}}

\begin{savenotes}\sphinxattablestart
\centering
\begin{tabulary}{\linewidth}[t]{|p{4.5cm}|p{8.8cm}|p{2.2cm}|}
\hline
\sphinxstyletheadfamily 
\sphinxAtStartPar
Name
&\sphinxstyletheadfamily 
\sphinxAtStartPar
Description
&\sphinxstyletheadfamily 
\sphinxAtStartPar
Dimensions
\\
\hline
\sphinxAtStartPar
\sphinxcode{\sphinxupquote{water\_demand}}
&
\sphinxAtStartPar
Demand for water across all water resource sectors (kg).
Only available if {\hyperref[\detokenize{namelists/jules_water_resources.nml:JULES_WATER_RESOURCES::l_water_resources}]{\sphinxcrossref{\sphinxcode{\sphinxupquote{l\_water\_resources}}}}} = TRUE.
&\\
\hline
\sphinxAtStartPar
\sphinxcode{\sphinxupquote{demand\_domestic}}
&
\sphinxAtStartPar
Demand for water for domestic use (kg).
Only available if {\hyperref[\detokenize{namelists/jules_water_resources.nml:JULES_WATER_RESOURCES::l_water_domestic}]{\sphinxcrossref{\sphinxcode{\sphinxupquote{l\_water\_domestic}}}}} = TRUE.
&\\
\hline
\sphinxAtStartPar
\sphinxcode{\sphinxupquote{demand\_environment}}
&
\sphinxAtStartPar
Demand for water for environmental use (kg).
Only available if {\hyperref[\detokenize{namelists/jules_water_resources.nml:JULES_WATER_RESOURCES::l_water_environment}]{\sphinxcrossref{\sphinxcode{\sphinxupquote{l\_water\_environment}}}}} = TRUE.
&\\
\hline
\sphinxAtStartPar
\sphinxcode{\sphinxupquote{demand\_industry}}
&
\sphinxAtStartPar
Demand for water for industrial use (kg).
Only available if {\hyperref[\detokenize{namelists/jules_water_resources.nml:JULES_WATER_RESOURCES::l_water_industry}]{\sphinxcrossref{\sphinxcode{\sphinxupquote{l\_water\_industry}}}}} = TRUE.
&\\
\hline
\sphinxAtStartPar
\sphinxcode{\sphinxupquote{demand\_irrigation}}
&
\sphinxAtStartPar
Demand for water for irrigation (kg).
Only available if {\hyperref[\detokenize{namelists/jules_water_resources.nml:JULES_WATER_RESOURCES::l_water_irrigation}]{\sphinxcrossref{\sphinxcode{\sphinxupquote{l\_water\_irrigation}}}}} = TRUE.
&\\
\hline
\sphinxAtStartPar
\sphinxcode{\sphinxupquote{demand\_livestock}}
&
\sphinxAtStartPar
Demand for water for livestock use (kg).
Only available if {\hyperref[\detokenize{namelists/jules_water_resources.nml:JULES_WATER_RESOURCES::l_water_livestock}]{\sphinxcrossref{\sphinxcode{\sphinxupquote{l\_water\_livestock}}}}} = TRUE.
&\\
\hline
\sphinxAtStartPar
\sphinxcode{\sphinxupquote{demand\_transfers}}
&
\sphinxAtStartPar
Demand for water for transfers (kg).
Only available if {\hyperref[\detokenize{namelists/jules_water_resources.nml:JULES_WATER_RESOURCES::l_water_transfers}]{\sphinxcrossref{\sphinxcode{\sphinxupquote{l\_water\_transfers}}}}} = TRUE.
&\\
\hline
\sphinxAtStartPar
\sphinxcode{\sphinxupquote{demand\_rate\_domestic}}
&
\sphinxAtStartPar
Demand rate for water for domestic use (kg s$^{\text{\sphinxhyphen{}1}}$).
Only available if {\hyperref[\detokenize{namelists/jules_water_resources.nml:JULES_WATER_RESOURCES::l_water_irrigation}]{\sphinxcrossref{\sphinxcode{\sphinxupquote{l\_water\_irrigation}}}}} = TRUE.
&\\
\hline
\sphinxAtStartPar
\sphinxcode{\sphinxupquote{demand\_rate\_industry}}
&
\sphinxAtStartPar
Demand rate for water for industrial use (kg s$^{\text{\sphinxhyphen{}1}}$).
Only available if {\hyperref[\detokenize{namelists/jules_water_resources.nml:JULES_WATER_RESOURCES::l_water_industry}]{\sphinxcrossref{\sphinxcode{\sphinxupquote{l\_water\_industry}}}}} = TRUE.
&\\
\hline
\sphinxAtStartPar
\sphinxcode{\sphinxupquote{demand\_rate\_livestock}}
&
\sphinxAtStartPar
Demand rate for water for livestock use (kg s$^{\text{\sphinxhyphen{}1}}$).
Only available if {\hyperref[\detokenize{namelists/jules_water_resources.nml:JULES_WATER_RESOURCES::l_water_livestock}]{\sphinxcrossref{\sphinxcode{\sphinxupquote{l\_water\_livestock}}}}} = TRUE.
&\\
\hline
\sphinxAtStartPar
\sphinxcode{\sphinxupquote{demand\_rate\_transfers}}
&
\sphinxAtStartPar
Demand rate for water for transfers (kg s$^{\text{\sphinxhyphen{}1}}$).
Only available if {\hyperref[\detokenize{namelists/jules_water_resources.nml:JULES_WATER_RESOURCES::l_water_transfers}]{\sphinxcrossref{\sphinxcode{\sphinxupquote{l\_water\_transfers}}}}} = TRUE.
&\\
\hline
\sphinxAtStartPar
\sphinxcode{\sphinxupquote{irrig\_water}}
&
\sphinxAtStartPar
Water applied as irrigation (kg m$^{\text{\sphinxhyphen{}2}}$ s$^{\text{\sphinxhyphen{}1}}$).
Only available if {\hyperref[\detokenize{namelists/jules_irrig.nml:JULES_IRRIG::l_irrig_dmd}]{\sphinxcrossref{\sphinxcode{\sphinxupquote{l\_irrig\_dmd}}}}} = TRUE.
&\\
\hline
\end{tabulary}
\par
\sphinxattableend\end{savenotes}


\section{Urban}
\label{\detokenize{output-variables:urban}}

\begin{savenotes}\sphinxattablestart
\centering
\begin{tabulary}{\linewidth}[t]{|p{3.5cm}|p{9.8cm}|p{2.2cm}|}
\hline
\sphinxstyletheadfamily 
\sphinxAtStartPar
Name
&\sphinxstyletheadfamily 
\sphinxAtStartPar
Description
&\sphinxstyletheadfamily 
\sphinxAtStartPar
Dimensions
\\
\hline
\sphinxAtStartPar
\sphinxcode{\sphinxupquote{wrr}}
&
\sphinxAtStartPar
Urban morphology: Repeating width ratio (or canyon fraction, W/R).
Calculated if {\hyperref[\detokenize{namelists/urban.nml:JULES_URBAN::l_urban_empirical}]{\sphinxcrossref{\sphinxcode{\sphinxupquote{l\_urban\_empirical}}}}} = TRUE,
otherwise it will be the same as the input value (see {\hyperref[\detokenize{namelists/ancillaries.nml:list-of-urban-properties}]{\sphinxcrossref{\DUrole{std,std-ref}{List of urban properties}}}}).
&\\
\hline
\sphinxAtStartPar
\sphinxcode{\sphinxupquote{hwr}}
&
\sphinxAtStartPar
Urban morphology: Height\sphinxhyphen{}to\sphinxhyphen{}width ratio (H/W). See for \sphinxcode{\sphinxupquote{wrr}} above.
Only used by MORUSES.
&\\
\hline
\sphinxAtStartPar
\sphinxcode{\sphinxupquote{hgt}}
&
\sphinxAtStartPar
Urban morphology: Building height (H). See for \sphinxcode{\sphinxupquote{wrr}} above. Only used by MORUSES.
&\\
\hline
\end{tabulary}
\par
\sphinxattableend\end{savenotes}


\section{IMOGEN}
\label{\detokenize{output-variables:imogen}}

\begin{savenotes}\sphinxattablestart
\centering
\begin{tabulary}{\linewidth}[t]{|p{3.5cm}|p{9.8cm}|p{2.2cm}|}
\hline
\sphinxstyletheadfamily 
\sphinxAtStartPar
Name
&\sphinxstyletheadfamily 
\sphinxAtStartPar
Description
&\sphinxstyletheadfamily 
\sphinxAtStartPar
Dimensions
\\
\hline
\sphinxAtStartPar
\sphinxcode{\sphinxupquote{c\_emiss\_out}}
&
\sphinxAtStartPar
Prescribed carbon emissions in IMOGEN (Gt / year).
This is a global value repeated over all land points and is only
available when IMOGEN is switched on.
&\\
\hline
\sphinxAtStartPar
\sphinxcode{\sphinxupquote{d\_land\_atmos\_co2}}
&
\sphinxAtStartPar
Change in atmospheric CO2 concentration from land\sphinxhyphen{}atmosphere.
CO2 feedbacks in IMOGEN (ppm / year). This is a global value
repeated over all land points and is only available when IMOGEN
is switched on.
&\\
\hline
\sphinxAtStartPar
\sphinxcode{\sphinxupquote{d\_ocean\_atmos}}
&
\sphinxAtStartPar
Change in atmospheric CO2 concentration from ocean\sphinxhyphen{}atmosphere
CO2 feedbacks in IMOGEN (ppm / year). This is a global value
repeated over all land points and is only available when IMOGEN is
switched on.
&\\
\hline
\end{tabulary}
\par
\sphinxattableend\end{savenotes}


\section{Grid and indexing variables}
\label{\detokenize{output-variables:grid-and-indexing-variables}}

\begin{savenotes}\sphinxattablestart
\centering
\begin{tabulary}{\linewidth}[t]{|p{2.5cm}|p{10.8cm}|p{2.2cm}|}
\hline
\sphinxstyletheadfamily 
\sphinxAtStartPar
Name
&\sphinxstyletheadfamily 
\sphinxAtStartPar
Description
&\sphinxstyletheadfamily 
\sphinxAtStartPar
Dimensions
\\
\hline
\sphinxAtStartPar
\sphinxcode{\sphinxupquote{grid\_area}}
&
\sphinxAtStartPar
Gridbox surface area (m$^{\text{2}}$).
Only available if {\hyperref[\detokenize{namelists/jules_water_resources.nml:JULES_WATER_RESOURCES::l_water_irrigation}]{\sphinxcrossref{\sphinxcode{\sphinxupquote{l\_water\_irrigation}}}}} = TRUE.
&
\sphinxAtStartPar
land+sea
\\
\hline
\sphinxAtStartPar
\sphinxcode{\sphinxupquote{grid\_area\_rp}}
&
\sphinxAtStartPar
River routing gridbox surface area (m$^{\text{2}}$).
Only available if {\hyperref[\detokenize{namelists/jules_rivers.nml:JULES_RIVERS::l_rivers}]{\sphinxcrossref{\sphinxcode{\sphinxupquote{l\_rivers}}}}} = TRUE.
Note that this is defined on the river routing model grid, not on the land point grid.
&
\sphinxAtStartPar
np\_rivers
\\
\hline
\sphinxAtStartPar
\sphinxcode{\sphinxupquote{land\_index}}
&
\sphinxAtStartPar
Index (gridbox number) of land points.
&\\
\hline
\sphinxAtStartPar
\sphinxcode{\sphinxupquote{lice\_index}}
&
\sphinxAtStartPar
Index (gridbox number) of land ice points.
&\\
\hline
\sphinxAtStartPar
\sphinxcode{\sphinxupquote{soil\_index}}
&
\sphinxAtStartPar
Index (gridbox number) of soil points.
&\\
\hline
\sphinxAtStartPar
\sphinxcode{\sphinxupquote{tile\_index}}
&
\sphinxAtStartPar
Index (gridbox number) of land points with each surface type.
&
\sphinxAtStartPar
ntype
\\
\hline
\end{tabulary}
\par
\sphinxattableend\end{savenotes}


\renewcommand{\indexname}{Fortran Namelist Index}
\begin{sphinxtheindex}
\let\bigletter\sphinxstyleindexlettergroup
\bigletter{c}
\item\relax\sphinxstyleindexentry{CABLE\_PFTPARM}\sphinxstyleindexpageref{namelists/cable_pftparm.nml:\detokenize{namelist-CABLE_PFTPARM}}
\item\relax\sphinxstyleindexentry{CABLE\_PROGS}\sphinxstyleindexpageref{namelists/cable_prognostics.nml:\detokenize{namelist-CABLE_PROGS}}
\item\relax\sphinxstyleindexentry{CABLE\_SOILPARM}\sphinxstyleindexpageref{namelists/cable_soilparm.nml:\detokenize{namelist-CABLE_SOILPARM}}
\item\relax\sphinxstyleindexentry{CABLE\_SURFACE\_TYPES}\sphinxstyleindexpageref{namelists/cable_surface_types.nml:\detokenize{namelist-CABLE_SURFACE_TYPES}}
\indexspace
\bigletter{f}
\item\relax\sphinxstyleindexentry{FIRE\_SWITCHES}\sphinxstyleindexpageref{namelists/fire.nml:\detokenize{namelist-FIRE_SWITCHES}}
\indexspace
\bigletter{i}
\item\relax\sphinxstyleindexentry{IMOGEN\_ANLG\_VALS\_LIST}\sphinxstyleindexpageref{namelists/imogen.nml:\detokenize{namelist-IMOGEN_ANLG_VALS_LIST}}
\item\relax\sphinxstyleindexentry{IMOGEN\_ONOFF\_SWITCH}\sphinxstyleindexpageref{namelists/imogen.nml:\detokenize{namelist-IMOGEN_ONOFF_SWITCH}}
\item\relax\sphinxstyleindexentry{IMOGEN\_RUN\_LIST}\sphinxstyleindexpageref{namelists/imogen.nml:\detokenize{namelist-IMOGEN_RUN_LIST}}
\indexspace
\bigletter{j}
\item\relax\sphinxstyleindexentry{JULES\_AGRIC}\sphinxstyleindexpageref{namelists/ancillaries.nml:\detokenize{namelist-JULES_AGRIC}}
\item\relax\sphinxstyleindexentry{JULES\_CO2}\sphinxstyleindexpageref{namelists/ancillaries.nml:\detokenize{namelist-JULES_CO2}}
\item\relax\sphinxstyleindexentry{JULES\_CROP\_PROPS}\sphinxstyleindexpageref{namelists/ancillaries.nml:\detokenize{namelist-JULES_CROP_PROPS}}
\item\relax\sphinxstyleindexentry{JULES\_CROPPARM}\sphinxstyleindexpageref{namelists/crop_params.nml:\detokenize{namelist-JULES_CROPPARM}}
\item\relax\sphinxstyleindexentry{JULES\_DEPOSITION}\sphinxstyleindexpageref{namelists/jules_deposition.nml:\detokenize{namelist-JULES_DEPOSITION}}
\item\relax\sphinxstyleindexentry{JULES\_DEPOSITION\_SPECIES}\sphinxstyleindexpageref{namelists/jules_deposition.nml:\detokenize{namelist-JULES_DEPOSITION_SPECIES}}
\item\relax\sphinxstyleindexentry{JULES\_DRIVE}\sphinxstyleindexpageref{namelists/drive.nml:\detokenize{namelist-JULES_DRIVE}}
\item\relax\sphinxstyleindexentry{JULES\_FLAKE}\sphinxstyleindexpageref{namelists/ancillaries.nml:\detokenize{namelist-JULES_FLAKE}}
\item\relax\sphinxstyleindexentry{JULES\_FRAC}\sphinxstyleindexpageref{namelists/ancillaries.nml:\detokenize{namelist-JULES_FRAC}}
\item\relax\sphinxstyleindexentry{JULES\_HYDROLOGY}\sphinxstyleindexpageref{namelists/jules_hydrology.nml:\detokenize{namelist-JULES_HYDROLOGY}}
\item\relax\sphinxstyleindexentry{JULES\_INITIAL}\sphinxstyleindexpageref{namelists/initial_conditions.nml:\detokenize{namelist-JULES_INITIAL}}
\item\relax\sphinxstyleindexentry{JULES\_INPUT\_GRID}\sphinxstyleindexpageref{namelists/model_grid.nml:\detokenize{namelist-JULES_INPUT_GRID}}
\item\relax\sphinxstyleindexentry{JULES\_IRRIG}\sphinxstyleindexpageref{namelists/jules_irrig.nml:\detokenize{namelist-JULES_IRRIG}}
\item\relax\sphinxstyleindexentry{JULES\_IRRIG\_PROPS}\sphinxstyleindexpageref{namelists/ancillaries.nml:\detokenize{namelist-JULES_IRRIG_PROPS}}
\item\relax\sphinxstyleindexentry{JULES\_LAND\_FRAC}\sphinxstyleindexpageref{namelists/model_grid.nml:\detokenize{namelist-JULES_LAND_FRAC}}
\item\relax\sphinxstyleindexentry{JULES\_LATLON}\sphinxstyleindexpageref{namelists/model_grid.nml:\detokenize{namelist-JULES_LATLON}}
\item\relax\sphinxstyleindexentry{JULES\_MODEL\_ENVIRONMENT}\sphinxstyleindexpageref{namelists/model_environment.nml:\detokenize{namelist-JULES_MODEL_ENVIRONMENT}}
\item\relax\sphinxstyleindexentry{JULES\_MODEL\_GRID}\sphinxstyleindexpageref{namelists/model_grid.nml:\detokenize{namelist-JULES_MODEL_GRID}}
\item\relax\sphinxstyleindexentry{JULES\_NLSIZES}\sphinxstyleindexpageref{namelists/model_grid.nml:\detokenize{namelist-JULES_NLSIZES}}
\item\relax\sphinxstyleindexentry{JULES\_NVEGPARM}\sphinxstyleindexpageref{namelists/nveg_params.nml:\detokenize{namelist-JULES_NVEGPARM}}
\item\relax\sphinxstyleindexentry{JULES\_OUTPUT}\sphinxstyleindexpageref{namelists/output.nml:\detokenize{namelist-JULES_OUTPUT}}
\item\relax\sphinxstyleindexentry{JULES\_OUTPUT\_PROFILE}\sphinxstyleindexpageref{namelists/output.nml:\detokenize{namelist-JULES_OUTPUT_PROFILE}}
\item\relax\sphinxstyleindexentry{JULES\_OVERBANK}\sphinxstyleindexpageref{namelists/jules_rivers.nml:\detokenize{namelist-JULES_OVERBANK}}
\item\relax\sphinxstyleindexentry{JULES\_OVERBANK\_PROPS}\sphinxstyleindexpageref{namelists/ancillaries.nml:\detokenize{namelist-JULES_OVERBANK_PROPS}}
\item\relax\sphinxstyleindexentry{JULES\_PDM}\sphinxstyleindexpageref{namelists/ancillaries.nml:\detokenize{namelist-JULES_PDM}}
\item\relax\sphinxstyleindexentry{JULES\_PFTPARM}\sphinxstyleindexpageref{namelists/pft_params.nml:\detokenize{namelist-JULES_PFTPARM}}
\item\relax\sphinxstyleindexentry{JULES\_PRESCRIBED}\sphinxstyleindexpageref{namelists/prescribed_data.nml:\detokenize{namelist-JULES_PRESCRIBED}}
\item\relax\sphinxstyleindexentry{JULES\_PRESCRIBED\_DATASET}\sphinxstyleindexpageref{namelists/prescribed_data.nml:\detokenize{namelist-JULES_PRESCRIBED_DATASET}}
\item\relax\sphinxstyleindexentry{JULES\_PRNT\_CONTROL}\sphinxstyleindexpageref{namelists/jules_prnt_control.nml:\detokenize{namelist-JULES_PRNT_CONTROL}}
\item\relax\sphinxstyleindexentry{JULES\_RADIATION}\sphinxstyleindexpageref{namelists/jules_radiation.nml:\detokenize{namelist-JULES_RADIATION}}
\item\relax\sphinxstyleindexentry{JULES\_RIVERS}\sphinxstyleindexpageref{namelists/jules_rivers.nml:\detokenize{namelist-JULES_RIVERS}}
\item\relax\sphinxstyleindexentry{JULES\_RIVERS\_PROPS}\sphinxstyleindexpageref{namelists/ancillaries.nml:\detokenize{namelist-JULES_RIVERS_PROPS}}
\item\relax\sphinxstyleindexentry{JULES\_SNOW}\sphinxstyleindexpageref{namelists/jules_snow.nml:\detokenize{namelist-JULES_SNOW}}
\item\relax\sphinxstyleindexentry{JULES\_SOIL}\sphinxstyleindexpageref{namelists/jules_soil.nml:\detokenize{namelist-JULES_SOIL}}
\item\relax\sphinxstyleindexentry{JULES\_SOIL\_BIOGEOCHEM}\sphinxstyleindexpageref{namelists/jules_soil_biogeochem.nml:\detokenize{namelist-JULES_SOIL_BIOGEOCHEM}}
\item\relax\sphinxstyleindexentry{JULES\_SOIL\_ECOSSE}\sphinxstyleindexpageref{namelists/jules_soil_ecosse.nml:\detokenize{namelist-JULES_SOIL_ECOSSE}}
\item\relax\sphinxstyleindexentry{JULES\_SOIL\_PROPS}\sphinxstyleindexpageref{namelists/ancillaries.nml:\detokenize{namelist-JULES_SOIL_PROPS}}
\item\relax\sphinxstyleindexentry{JULES\_SPINUP}\sphinxstyleindexpageref{namelists/timesteps.nml:\detokenize{namelist-JULES_SPINUP}}
\item\relax\sphinxstyleindexentry{JULES\_SURF\_HGT}\sphinxstyleindexpageref{namelists/model_grid.nml:\detokenize{namelist-JULES_SURF_HGT}}
\item\relax\sphinxstyleindexentry{JULES\_SURFACE}\sphinxstyleindexpageref{namelists/jules_surface.nml:\detokenize{namelist-JULES_SURFACE}}
\item\relax\sphinxstyleindexentry{JULES\_SURFACE\_TYPES}\sphinxstyleindexpageref{namelists/jules_surface_types.nml:\detokenize{namelist-JULES_SURFACE_TYPES}}
\item\relax\sphinxstyleindexentry{JULES\_TEMP\_FIXES}\sphinxstyleindexpageref{namelists/science_fixes.nml:\detokenize{namelist-JULES_TEMP_FIXES}}
\item\relax\sphinxstyleindexentry{JULES\_TIME}\sphinxstyleindexpageref{namelists/timesteps.nml:\detokenize{namelist-JULES_TIME}}
\item\relax\sphinxstyleindexentry{JULES\_TOP}\sphinxstyleindexpageref{namelists/ancillaries.nml:\detokenize{namelist-JULES_TOP}}
\item\relax\sphinxstyleindexentry{JULES\_TRIFFID}\sphinxstyleindexpageref{namelists/triffid_params.nml:\detokenize{namelist-JULES_TRIFFID}}
\item\relax\sphinxstyleindexentry{JULES\_URBAN}\sphinxstyleindexpageref{namelists/urban.nml:\detokenize{namelist-JULES_URBAN}}
\item\relax\sphinxstyleindexentry{JULES\_VEGETATION}\sphinxstyleindexpageref{namelists/jules_vegetation.nml:\detokenize{namelist-JULES_VEGETATION}}
\item\relax\sphinxstyleindexentry{JULES\_VEGETATION\_PROPS}\sphinxstyleindexpageref{namelists/ancillaries.nml:\detokenize{namelist-JULES_VEGETATION_PROPS}}
\item\relax\sphinxstyleindexentry{JULES\_WATER\_RESOURCES}\sphinxstyleindexpageref{namelists/jules_water_resources.nml:\detokenize{namelist-JULES_WATER_RESOURCES}}
\item\relax\sphinxstyleindexentry{JULES\_WATER\_RESOURCES\_PROPS}\sphinxstyleindexpageref{namelists/ancillaries.nml:\detokenize{namelist-JULES_WATER_RESOURCES_PROPS}}
\item\relax\sphinxstyleindexentry{JULES\_Z\_LAND}\sphinxstyleindexpageref{namelists/model_grid.nml:\detokenize{namelist-JULES_Z_LAND}}
\indexspace
\bigletter{o}
\item\relax\sphinxstyleindexentry{OASIS\_RIVERS}\sphinxstyleindexpageref{namelists/oasis_rivers.nml:\detokenize{namelist-OASIS_RIVERS}}
\indexspace
\bigletter{u}
\item\relax\sphinxstyleindexentry{URBAN\_PROPERTIES}\sphinxstyleindexpageref{namelists/ancillaries.nml:\detokenize{namelist-URBAN_PROPERTIES}}
\end{sphinxtheindex}

\renewcommand{\indexname}{Index}
\printindex
\end{document}